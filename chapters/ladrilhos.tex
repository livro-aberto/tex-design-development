\renewcommand\chapterillustration{./tessela}%Photo by <a href="https://unsplash.com/@teckhonc?utm_source=unsplash&amp;utm_medium=referral&amp;utm_content=creditCopyText">T.H. Chia</a> on <a href="https://unsplash.com/s/photos/tesselations?utm_source=unsplash&amp;utm_medium=referral&amp;utm_content=creditCopyText">Unsplash</a>
\def\chapterwhat{Ladrilhamentos regulares, semirregulares e irregulares. }

\def\chapterbecause{O ladrilhamento de uma superfície plana é uma cobertura do plano onde formas (por exemplo, polígonos) são repetidos sem sobreposição ou buracos. No caso de polígonos, estes podem ser regulares ou não. Os ladrilhos compostos por figuras geométricas podem ser vistos como uma arte, usados em outras aplicações, como papéis de parede, forros de madeira, estamparia de tecidos, bordados entre outros. Apesar do número de possibilidades de realizar um ladrilhamento parecer infinita, é possível comprovar que ao utilizar apenas polígonos regulares existe apenas 11 possibilidades. Contudo pode-se aprofundar o estudo e investigar a utilização de polígonos irregulares.} 
\chapter{Ladrilhamento}
\label{ladri-chap}

\mbox{}\thispagestyle{empty}\clearpage

\thispagestyle{empty}

\begin{center}
Projeto: LIVRO ABERTO DE MATEMÁTICA

\noindent \begin{tabular}{lcccr}
\includegraphics[scale=.15]{impa}& \quad\quad& \includegraphics[width=3cm]{logo} & \quad\quad& \includegraphics[scale=.24]{obmep} 
\end{tabular}
\end{center}

\vspace*{.3cm}

Cadastre-se como colaborador no site do projeto: \url{umlivroaberto.org}


% \begin{center}
%   \includegraphics[width=2cm]{canvas}
% \end{center}

\begin{tabular}{p{.15\textwidth}p{.7\textwidth}}
Título: & Ladrilhamento\\
\\
Ano/ Versão: & 2020 / versão 0.1 de 27 de maio de 2020\\
\\
Editora & Instituto Nacional de Matem\'atica Pura e Aplicada (IMPA-OS)\\
\\
Realização:& Olimp\'iada Brasileira de Matem\'atica das Escolas P\'ublicas (OBMEP)\\
\\
Produção:& Associação Livro Aberto\\
\\
Coordenação: & Fabio Simas, \\
			&  Augusto Teixeira (livroaberto@impa.br)\\
\\
  Autor: & Carmen Vieira Mathias (UFSM) \\
         & Lucas Schimith Zanon (SEDUC - RS) \\
\\
Revisor: & Ezequiel Sanchez Visgraf  \\
\\
Design: & Andreza Moreira (Tangentes Design) \\
\\
  Ilustrações: & --- \\ 
\\
Gráficos: & --- \\
\\
  Capa: & Foto de T.H. Chia, no Unsplash \\
  		& https://unsplash.com//@teckhonc \\

\end{tabular}


\begin{figure}[b]
\begin{minipage}[l]{5cm}
\centering

{\large Licença:}

  \includegraphics[width=3.5cm]{Figuras/cc-by-sa}
\end{minipage}\hfill
\begin{minipage}[c]{5cm}
\centering
{\large Desenvolvido por}



{\large Patrocínio:}
  \vspace{1em}
  \includegraphics[width=3.5cm]{itau}
\end{minipage}
\end{figure}

\clearpage


\explore{A arte de ladrilhar}
\label{ladri-exp-1}

Devido às suas características e sua estética decorativa, os ladrilhamentos foram utilizados tanto na arte quanto na arquitetura, fornecendo revestimentos para paredes, calçadas e tetos de muitas instalações. 

A origem dos ladrilhamentos ocorreu a cerca de  4.000 anos, quando os sumérios usavam azulejos de barro para compor elementos de decoração em suas casas e templos. A partir daí, o ladrilhamento encontrou seu lugar em elementos artísticos de muitas civilizações. 

No século XIX, intelectuais começaram a observar os ladrilhamentos presentes na natureza, a fim de explicar suas estruturas geométricas, o que resultou em numerosos estudos baseados em matemática. Um exemplo é  o artista Maurits Escher que usou o conceito de ladrilhamento para criar obras de arte. Esse artista holandês era fascinado por ladrilhamentos, também chamados de pavimentações ou mosaicos. 

Escher fez alguns ladrilhamentos enigmáticos, começando com uma forma básica e depois transformando-a usando isometrias.  Tais obras são muito complexas e em muitas delas aparecem formas de  animais e humanos, como por exemplo na obra  Lizard (lagartos) criada em 1942, ilustrada na figura \ref{lad-fig-1}.
. 

\begin{figure}[H]
\centering
\includegraphics[width=200bp]{lagartos}

\caption{Lizard, M.C. Escher. Fonte: \href{https://www.wikiart.org/en/m-c-escher/lizard-1}{Wikiart}}
\label{lad-fig-1}
\end{figure}


Mas, um ladrilhamento nem sempre cobre uma superfície plana. Por exemplo, essa  superfície pode ser a parte externa de uma bola ou de um abajur. Pode ser  a parte interna de um ovo decorado, a pele de uma cobra, um hexágono plano ou uma parede. Assim, neste capítulo, vamos reconhecer, descrever e criar ladrilhamentos, para explorar e percebê-los no ambiente que nos cerca.

\begin{reflection}
Escher criou o ladrilhamento ilustrado na figura xx transformando um hexágono em um lagarto. Como será que ele fez isso?
\end{reflection}



\begin{task}{Brincando de ser Escher} \label{at_brinc}
Será que os lagartos (figura \ref{lizard}) podem ser usados para ladrilhar o plano? 
Utilize o material disponibilizado por seu professor para construir o ladrilhamento.


\begin{figure}[H]
\centering
\includegraphics[width=300bp]{ladrilhamento3}
\label{lizard}
\caption{Lagartos.}
\end{figure}
\end{task}


\begin{task}{Reconhecendo um ladrilhamento} \label{rec_lad}
\begin{enumerate}
\item Considere  a figura \ref{ladr12}, composta de seis outras imagens. Decida se as imagens (numeradas de 1 a 6) podem ou não podem ser  ladrilhamento.

\begin{figure}[H]
\centering
\includegraphics[width=300bp]{bosta}
\label{ladr12}
\caption{Imagens que podem ilustrar ladrilhamentos.}
\end{figure}

\item Onde podemos encontrar ladrilhamentos no mundo ao nosso redor? Cite alguns exemplos?

\end{enumerate}

\end{task}



\arrange{A arte de ladrilhar}

No primeiro item da atividade \ref{rec_lad} nosso objetivo foi que você identificasse  quais das imagens eram ladrilhamentos .
Observamos que as figuras 1, 3, 4 e 6 não são ladrilhamentos, visto que os requisitos:  obedecer um padrão, não existir espaços não utilizados (lacunas) e não haver sobreposição de formas, não são cumpridos. A figura 2 ( penas do pássaro) parece obedecer essas condições, pois na metade inferior esquerda da foto, as penas se encaixam sem folgas ou sobreposições e claramente, esse padrão pode ser repetido para sempre para preencher um plano. No entanto, na metade superior direita da foto, muitas das penas estão sobrepostas. E como os "ladrilhos" não devem se sobrepor. Então, a figura não é um ladrilhamento.

Assim, a única figura que onde a forma básica se repete perfeitamente, sem espaços ou sobreposições, podendo ser reptida indefinidamente é a figura 5.
Assim,  quando revestimos uma superfície plana com figuras sem deixar falhas ou sobrepô-las, dizemos que houve um ladrilhamento dessa superfície. 

No segundo item parte da  atividade \ref{rec_lad} questionamos onde é possível encontrar encontrar ladrilhamentos no mundo ao nosso redor. 
Ao observarmos a natureza, um dos exemplos mais conhecidos de ladrilhamentos são os favos de mel, que são formas tridimensionais hexagonais. Pode-se observar que não há lacunas entre as formas. Sem lacunas, as abelhas usam seu espaço de armazenamento de forma muito eficiente. Outros exemplos são as peles de algumas cobras e a organização das escamas de alguns peixes (figura \ref{natureza}).


\begin{figure}[H]
\centering
\includegraphics[width=400bp]{nature1}
\label{natureza}
\caption{Ladrilhamentos na natureza. Fonte: Google Imagens}
\end{figure}

Também é possível perceber a presença de ladrilhamentos do plano no revestimento de pisos e paredes (figura \ref{natureza1}). 


\begin{figure}[H]
\centering
\includegraphics[width=400bp]{nature3}
\label{natureza1}
\caption{Revestimento de pisos e paredes. Fonte: Google Imagens}
\end{figure}

Um dos objetivos desse capítulo é que você construa alguns ladrilhamentos, e para começar, foi proposta a atividade \ref{at_brinc}. Foi possível construir um ladrilhamento usando apenas lagartos?  Mas o que os lagartos tem em comum com a matemática? 

Os ladrilhamentos têm sido amplamente utilizados na arte e na arquitetura desde os tempos antigos, mas o que existe associado a isso é  matemática. A teoria é extensa, mas explicaremos alguns princípios básicos para nos aproximar do que está por trás de belas obras de arte. 

Quando se trata de ladrilhamento em matemática, também conhecido como mosaico ou pavimentação, é necessário explicar vários termos técnicos com os quais a geometria opera. Por exemplo, uma forma fundamental (também chamado de ladrilho) é uma peça que é repetida para formar um ladrilhamento. Na atividade anterior, a peça que foi repetida foi o lagarto.


No caso da matemática, geralmente, trabalha-se com polígonos e nesse caso, para construir um ladrilhamento é necessário observar alguns aspectos e é sobre isso que trata a próxima seção.


\know{Tesselação} \label{tess}

Tesselação é outra palavra para ladrilhamento. Derivado da palavra latina, tessella, que eram os pequenos pedaços de pedra usados para fazer mosaicos romanos, uma tesselação ou ladrilho, é um padrão feito de uma ou mais formas que se encaixam sem lacunas ou sobreposições.
Destas palavras antigas derivam termos semelhantes que aparecem em várias aplicações práticas, da arte e arquitetura à ciência, tecnologia e produção.

Em design e arquitetura, a tesselação se refere à pavimentação de paredes, pisos ou outras superfícies com um padrão de pequenos ladrilhos feitos de cerâmica, vidro ou outros materiais. Essesa ladrilhoa normalmente são cortadas em formas geométricas que se encaixam perfeitamente em designs simples ou complexos formando um padrão aparentemente infinito. A aplicação de ladrilhamentos  na arquitetura pode é  amplamente encontrada em muitas civilizações do mundo, incluindo o antigo Egito, mouros, romanos, Pérsia, Arábia, Japão e China, existindo ao longo da história do desenvolvimento da arquitetura. Embora os ladrilhamentos geralmente consistam em formas abstratas, principalmente retângulos, hexágonos, octógonos e outros polígonos, também podem consistir em elementos figurativos, como no trabalho de artistas como M.C. Escher (1898-1972). Escher é famoso por suas tesselações compostas de cavalos, borboletas, pássaros e criaturas imaginárias.


\explore{Ladrilhando com polígonos do mesmo tipo}

Ladrilhamentos são frequentemente feitos de repetições de um padrão, chamado forma fundamental, como ilustra a figura \ref{porce}. No que segue, vamos aprender a fazer ladrilhamentos usando polígonos de um só tipo e investigar quais polígonos são capazes de ladrilhar o plano.

\begin{figure}[H]
\centering
\includegraphics[width=400bp]{porcelana}
\caption{Ladrilhamentos usando polígonos de um só tipo. Fonte: Google imagens}
\end{figure}


\begin{task}{Ladrilhamentos usando triângulos}\label{at_lad_tri}

\textbf{Parte 1:} Copie a forma de um dos triângulos do encarte em um papel de gramatura mais alta. Recorte o triângulo. Crie um desenho, traçando repetidamente o triângulo. Verifique se a folha de papel está coberta e se não há espaços entre os triângulos.

\textbf{Parte 2:} Vamos explorar  ladrilhamentos  usando triângulos. 
\begin{enumerate}

\item Sugerimos que em um primeiro momento utilize o  \href{https://www.geogebra.org/m/uuafzw8k}{aplicativo}  para construir um ladrilhamento do plano usando apenas triângulos equiláteros. Para isso, procure distribuir os triângulos ao redor de um vértice.

\item Utilize o  \href{https://www.geogebra.org/m/junvq3qd}{aplicativo} e escolha um dos três triângulos para construir um novo ladrilhamento usando apenas triângulos.
\item Qual a diferença do primeiro ladrilhamento construído para o segundo?
\item Utilize o  \href{https://www.geogebra.org/m/ejfw44rt}{aplicativo}. Depois responda as questões:
\begin{enumerate}
\item O ladrilhamento que você produziu no item a pode ser reproduzido nesse aplicativo?
\item 	Peças triangulares de formato irregular podem ser usadas para ladrilhar o plano? O que torna um triângulo irregular?
\item  Se os ladrilhos triangulares são congruentes, eles podem ser usados para formar um ladrilhamento? Como você pode saber se triângulos  são congruentes ou não? Explique seu raciocínio.
\item 	Ao construir os seus ladrilhamentos (item a e b) a interseção entre dois polígonos era um um lado ou um vértice do triângulo? Isso é importante? Justifique! 
\end{enumerate}
\item Escreva um breve parágrafo explicando quais informações geométricas você usou para criar seus ladrilhamentos? 
\end{enumerate}

\end{task}

\begin{task}{Ladrilhamentos usando quadriláteros} \label{lad_qua}

\textbf{Parte 1:} Copie a forma de um quadrilátero do encarte em um papel de gramatura mais alta. Recorte-o. Crie um novo ladrilhamento usando o mesmo processo usado na Parte 1 da atividade \ref{at_lad_tri}. 

\textbf{Parte 2:} Vamos explorar  ladrilhamentos  usando quadriláteros. 
\begin{enumerate}

\item Utilize o \href{https://www.geogebra.org/m/d6nvffqk}{aplicativo} e escolha um dos quadriláteros para construir um ladrilhamento. Apenas observe que a interseção entre dois polígonos seja um lado ou um vértice.
\item Utilize o \href{https://www.geogebra.org/m/mdybnpnq}{aplicativo} e escolha um dos quadriláteros para construir um novo ladrilhamento.
\item Qual a diferença nos ladrilhamentos construídos nos itens a) e b) ?
\item Utilize o \href{https://www.geogebra.org/m/ejfw44rt}{aplicativo}. Depois responda as questões:
\begin{enumerate}
\item O ladrilhamento que você produziu nos itens a) e b) podem ser reproduzidos nesse aplicativo?
\item Os quadriláteros não convexos podem ser usados para criar ladrilhamentos interessantes. O que caracteriza um quadrilátero não convexo?
\end{enumerate}

\end{enumerate}

\end{task}


\begin{task}{Ladrilhamentos usando pentágonos}\label{lad_pen}

\textbf{Parte 1:} Copie a forma de  um pentágono do encarte em um papel de gramatura mais alta. Recorte-o. Crie um novo ladrilhamento usando o mesmo processo usado na Parte 1 da atividade \ref{at_lad_tri}. 

\textbf{Parte 2:} Vamos explorar  ladrilhamentos  usando pentágonos. 
\begin{enumerate}

\item Utilize o \href{https://www.geogebra.org/m/exzjd4mh}{aplicativo} para construir um ladrilhamento usando pentágonos regulares.
\item Utilize o  \href{https://www.geogebra.org/m/sffzzsww}{aplicativo}  e escolha um dos pentágonos para construir um novo ladrilhamento.
\item Qual a diferença nos ladrilhamentos construídos nos itens a) e b) ?
\end{enumerate}
\end{task}

\begin{task}{Ladrilhamentos usando hexágonos} \label{lad_hex}

\textbf{Parte 1:} Copie a forma de  um hexágono do encarte em um papel de gramatura mais alta. Recorte-o. Crie um novo ladrilhamento usando o mesmo processo usado na Parte 1 da atividade \ref{at_lad_tri}. 

\textbf{Parte 2:} Vamos explorar  ladrilhamentos  usando  hexágonos. 
\begin{enumerate}

\item Utilize o  \href{https://www.geogebra.org/m/uqemfkhp#material/zfczbshq}{aplicativo} para construir um ladrilhamento usando hexágnos regulares.
\item Utilize o  \href{https://www.geogebra.org/m/uqemfkhp#material/pnhc6tep}{aplicativo} e ladrilhe o plano com o hexágono disponível.
\item Qual a diferença nos ladrilhamentos construídos nos itens a) e b) ?

\item Utilize o \href{https://www.geogebra.org/m/uqemfkhp#material/ub84tqyy}{aplicativo} movimente os pontos azuis e responda: Quais propriedades o hexágono precisa ter para ladrilhar o plano?

\end{enumerate}
\end{task}




\begin{task}{Que formas podem ser usadas para ladrilhar o plano?}\label{at_formas}
\begin{enumerate}

\item Considere os polígonos da primeira coluna da tabela.  Meça cada ângulo interior e registre suas medidas na tabela.

\item Com base no que foi feito nos itens anteriores quais polígonos ladrilham o plano? Registre suas respostas na tabela.

% Não está aparecendo o cabeçalho no pdf 

\begin{table}[H]
\centering
\begin{tabu} to \textwidth{|l|c|c|}
\hline
%\thead
\makecell{Polígono} & \makecell{Medida de \\ cada ângulo \\ interno} & \makecell{Previsão: \\ O polígono \\ ladrilha o plano?}  \\
\hline
Triângulo Equilátero & &  \\
\hline
Triângulo isóceles & &  \\
\hline
Quadrado & &  \\
\hline
Pentágono regular & &  \\
\hline
Hexágono regular & &  \\
\hline
Quadrilátero irregular & &  \\
\hline
Pentágono irregular & &  \\
\hline
Hexágono irregular & &  \\
\hline
\end{tabu}
\end{table}

\item Utilize o  \href{https://www.geogebra.org/m/uqemfkhp#material/eqwhddse}{aplicativo} para responder a seguinte questão:  Além dos polígonos citados na tabela, existem outros polígonos regulares  que ladrilham o plano?  Justifique!

\item Explique por que alguns polígonos ladrilham o plano, mas outros não. Existe um padrão?


\end{enumerate}

\end{task}

\arrange{Ladrilhando com polígonos do mesmo tipo}


Os ladrilhamentos com peças quadrangulares são comuns nos cômodos de nossas casas. As peças podem ser pintadas, esculpidas ou possuir saliências e combinadas formam as decorações dos ambientes, como ilustra a figura \ref{lad_qd}
 


\begin{figure}[H]
\centering
\includegraphics[width=180bp]{ladrilhamento14}
\label{lad_qd}
\caption{Peças quadrangulares. Fonte: Google Imagens }
\end{figure}

Também existem ladrilhamento hexagonais, apesar de menos usuais. Eles são encontrados em algumas pavimentações de ruas e áreas externas, nos favos, onde é armazenado o mel das abelhas. Também são comuns na confeção de colchas de retalhos  ou  de crochê (figura \ref{lad_hex}.

\begin{figure}[H]
\centering
\includegraphics[width=250bp]{ladrilhamento15}
\label{lad_hex}
\caption{Peças hexagonais. Fonte: Google Imagens }
\end{figure}

Nas atividades anteriores, foram propostas tarefas utilizando vários tipos de polígonos, regulares e irregulares. Em todas as atividades, solicitamos que fosse utilizado apenas um tipo de polígono. Assim, foram construídos ladrilhamentos monoédricos. Os ladrilhamentos desse tipo são os constituídos de polígonos congruentes entre si. No caso, onde são usados polígonos regulares do mesmo tipo, os ladrilhamentos monoédricos são chamados de ladrilhamentos regulares.

Em cada uma das atividades anteriores procuramos colocar os polígonos (regulares e irregulares) de um determinado tipo ao redor de um vértice de forma que ficassem “lado a lado” (observando para que a interseção entre dois polígonos seja um lado ou um vértice). Isso significa que não consideramos ladrilhamentos como o ilustrado na figura \ref{ladr_tri1}. 

\begin{figure}[H]
\centering
\includegraphics[width=150bp]{ladr_tri1}
\label{ladr_tri1}
\caption{Ladrilhamento com triângulos equiláteros.}
\end{figure}


Observamos que qo distribuir os polígonos ao redor do vértice,  temos duas possiblidades (figura \ref{lad_reg}): 
\begin{itemize}
\item	Completamos “a volta” e os polígonos se ajustam;
\item 	Não completamos a volta e se colocarmos mais um polígono, haverá uma sobreposição.
\end{itemize}


\begin{figure}[H]
\centering
\includegraphics[width=300bp]{ladmon}
\label{lad_reg}
\caption{Disposição dos polígonos ao redor de um ponto.}
\end{figure}

Em um primeiro momento trabalhamos com triângulos equiláteros, e percebemos que é possível colocar perfeitamente seis triângulos equiláteros ao redor de um vértice e ao tomarmos um novo vértice, é possível continuar o ladrilhamento (figura \ref{ladr_tri2}). Ou seja,  os ladrilhamentos que vamos construir serão sempre pensados de forma a preencher todo um plano, ilimitado em todas as direções. 

\begin{figure}[H]
\centering
\includegraphics[width=300bp]{lad_tri2}
\label{ladr_tri2}
\caption{Triângulos ao redor de um vértice.}
\end{figure}
 
 
Na atividade \ref{at_formas} solicitamos que os ângulos dos polígonos fossem medidos. Isso se justifica, pois, uma condição para que um polígono possa ladrilhar o plano é que a soma dos vários ângulos que se posicionam em torno de cada vértice resulte em um ângulo de $360 ^{\circ}$, como ilustra a figura \ref{angulos}

\begin{figure}[H]
\centering
\includegraphics[width=250bp]{ladrilhamento8}
\label{angulos}
\caption{Ângulos ao redor de um vértice.}
\end{figure}


Uma pergunta que precisamos responder é:  quais polígonos regulares, de mesmo tipo pavimentam o plano?

Um fato conhecido é que a medida dos ângulos internos de um polígono regular de $n$ lados é $\displaystyle \frac{180^{\circ}(n-2)}{n}$. Assim, para que se tenha um ladrilhamento formado exclusivamente por polígonos regulares de $n$ lados é preciso que a medida do ângulo interno ($a_n$) seja um divisor de $360^{\circ}$, ou seja,

\begin{equation*}
\frac{180^{\circ}(n-2)}{n}=180^{\circ} \left(1 - \frac{2}{n}\right)=\frac{360^{\circ}}{m},
\end{equation*}
para algum número natural $m\geq1$.

Daí vem,

\begin{equation*}
1-\frac{2}{n}=\frac{2}{m}.
\end{equation*}

Logo,
\begin{equation*}
\frac{1}{n}+\frac{1}{m}=2.
\end{equation*}

Como $n$ é o número de lados de um polígono regular, é um número natural maior ou igual a três. E como $m$ também é um número natural, as únicas soluções inteiras e positivas, possíveis para a equação anterior são $n=3$ (com $m=6$), $n=4$ (com $m=4$) e $n=6$ (com $m=3$). Essas soluções resultam em exatamente três ladrilhamentos e consistem em distribuir ao redor de cada vértice ou $6$ triângulos equiláteros, ou $4$ quadrados ou $3$ hexágonos regulares. Ou seja, existem apenas três tipos de ladrilhamentos regulares, ilustrados na figura \ref{ladr_reg}.



\begin{figure}[H]
\centering
\includegraphics[width=400bp]{ladr_reg}
\label{ladr_reg}
\caption{Ladrilhamentos regulares. Fonte \href{https://en.wikipedia.org/wiki/Euclidean_tilings_by_convex_regular_polygons}{Wikipedia}}
\end{figure}



E quais polígonos irregulares de mesmo tipo ladrilham o plano?
 
Na atividade \ref{lad_qua} foi possível construir ladrilhamentos utilizando alguns quadriláteros, em particular com paralelogramos, que são quadriláteros que possuem os lados opostos paralelos. Afirmamos que os paralelogramos ladrilham o plano, pois possuem ângulos dois a dois suplementares e lados opostos congruentes ( figura \ref{par}).

\begin{figure}[H]
\centering
\includegraphics[width=350bp]{par}
\label{par}
\caption{Ladrilhamento com paralelogramos}
\end{figure}


Como os quadrados, losangos e retângulos são paralelogramos, eles também ladrilham o plano.
Porém, se consideramos um quadrilátero qualquer como proposto no item b) da mesma atividade, usamos quadriláteros quaisquer para ladrilhar o plano e no aplicativo disponível no item c) foi possível perceber que qualquer quadrilátero pode ladrilhar o plano. Como a  soma dos ângulos internos de qualquer quadrilátero é $360^{\circ}$, para construir um ladrilhamento usando quadriláteros, basta arranja-los ao redor do vértice, utilizando transformações isométricas, de forma a coincidir os lados. 

Na atividade \ref{at_lad_tri} usamos triângulos para compor ladrilhamentos. Observe que ao considerar um triângulo qualquer, como o da figura \ref{triqqr} e traçar por dois vértices retas paralelas, respectivamente, aos outros lados, obtemos um paralelogramo. 

\begin{figure}[H]
\centering
\includegraphics[width=200bp]{triqqr}
\label{triqqr}
\caption{Triângulos.}
\end{figure}

Consequentemente, seja qual for o triângulo, ele ladrilhará o plano.

Também tentamos utilizar pentágonos para ladrilhar o plano ( na atividade \ref{lad_pen}) . Nesse caso, percebemos que os pentágonos regulares não ladrilham o plano. Porém, no item b)da mesma atividade, foi possível verificar  que alguns pentágonos o fazem. Mas que tipo de pentágonos irregulares ladrilham o plano?

Em 2015 matemáticos da  Universidade de Washington Bothell, descobriram um pentágono irregular que ladrilha o plano. Antes desta descoberta, havia apenas quatorze tipos conhecidos de pentágonos irregulares capazes de ladrilhar o  plano, cuja descoberta varia do início ao final do século XX. Em 2017, com o auxílio de algoritmos computacionais, foi provado que são exatamente 15 tipos de pentágonos irregulares que ladrilham o plano (figura \ref{pentagono}).
Figura xx: .

\begin{figure}[H]
\centering
\includegraphics[width=400bp]{pentagono}
\label{pentagono}
\caption{Ladrilhamentos com pentágonos irregulares. \href{encurtador.com.br/dmuE3}{Forbes}}
\end{figure}



Na atividade \ref{lad_hex} produzimos ladrilhamentos usando hexágonos irregulares (itens b e d). No início do século XX, K. Reinhardt um matemático alemão provou que existem três tipos de hexágonos irregulares que pavimentam o plano. No que segue mostraremos como são esses tipos de hexágonos. Assim, vamos considerar um hexágono $ABCDEF$ irregular com lados de medidas $a$, $b$, $c$, $d$, $e$ e $f$ e ângulos de medidas $\hat{A}... \hat{F}$ como indicados na figura \ref{hex1}.

\begin{figure}[H]
\centering
\includegraphics[width=200bp]{hex1}
\label{hex1}
\caption{Hexágono}
\end{figure}

O primeiro tipo de hexágono regular capaz de ladrilhar o plano, são aqueles que possuem um par de lados congruentes e paralelos. Consideremos, sem perda de generalidade, os lados $c$ e $f$ satisfazendo essas condições.  (figura \ref{hex2}.

\begin{figure}[H]
\centering
\includegraphics[width=200bp]{hex2}
\label{hex2}
\caption{Hexágono com um par de lados paralelos e congruentes.}
\end{figure}

 
Para realizar o ladrilhamento, vamos arranjar os polígonos no vértice $A$, por exemplo, de forma que os lados $c$ e $f$ coincidam e que o ângulo $\hat{B}$ coincida com $\hat{A}$ (figura \ref{hex3}).

\begin{figure}[H]
\centering
\includegraphics[width=400bp]{hex3}
\label{hex3}
\caption{Ladrilhamento com hexágonos do tipo 1.}
\end{figure}

Um fato importante de ser observado é que a condição um par de lados paralelos ($AF//DC$ ou $f//c$)  é equivalente a $\hat{D} + \hat{E} + \hat{F} = 360^{\circ}$.

De fato,  considerando que o segmento $AC$ divide o ângulo $\hat{A}$ em outros dois ângulos $\alpha$ e $\beta$ e o ângulo $\hat{C}$ nos ângulos $\delta$ e $\gamma$ (figura \ref{hex4}).


\begin{figure}[H]
\centering
\includegraphics[width=200bp]{hex4}
\label{hex4}
\caption{Ângulos no hexágono.}
\end{figure}

E, como os lados $AF$ e $DC$ são paralelos, segue que $\alpha +\gamma = 180^{\circ}$.

Mas, $$\hat{A} + \hat{B} + \hat{C} = \alpha + \beta + \hat{B} + \gamma + \delta.$$

Logo, $\hat{A} + \hat{B} + \hat{C} = 360^{\circ}$ . Pois  $\beta + \hat{B} + \gamma = 180^{\circ}$ ( ângulos internos do triângulo ABC).

Reciprocamente, supondo que $\hat{A} + \hat{B} + \hat{C} = 360^{\circ}$ resulta que  $\alpha +\gamma = 180^{\circ}$ ou que $AF//DC$.

Como a soma dos ângulos internos de um hexágono é $720^{\circ}$ resulta que a condição $AF//DC$ também equivale a $\hat{D} + \hat{E} + \hat{F} = 360^{\circ}$.



Os hexágonos classificados como  como do segundo tipo são o que satisfazem as seguintes condições: $c= f$, $b=d$ e  $\hat{A} + \hat{B} + \hat{D} = 360^{\circ}$ (figura\ref{hex5}).

\begin{figure}[H]
\centering
\includegraphics[width=200bp]{hex5}
\label{hex5}
\caption{Hexágonos do tipo 2.}
\end{figure}


Para realizar o ladrilhamento com esse tipo de polígono, os hexágonos devem ser arranjados em torno de um vértice ( $A$, por exemplo)  de modo que o lado de medida $c$ coincida com o lado de medida $f$ e o lado de medida $b$ coincida com o lado de medida $d$ ( figura \ref{hex6}).

\begin{figure}[H]
\centering
\includegraphics[width=400bp]{hex6}
\label{hex6}
\caption{Ladrilhamento com hexágonos do tipo 2.}
\end{figure}

No item d) da atividade \ref{lad_hex} apresentamos o terceiro tipo de hexágonos que ladrilham o plano. Nesse tipo, os lados adjacentes são congruentes e os ângulos entre esses lados são congruentes e medem $120^{\circ}$. Na figura \ref{hex7}, $a=f$, $b=c$, $d=e$ e $\hat{A}=\hat{C}=\hat{E}= 120^{\circ}$.



\begin{figure}[H]
\centering
\includegraphics[width=200bp]{hex7}
\label{hex7}
\caption{Hexágonos do tipo 3.}
\end{figure}

Para construir um ladrilhamento com esse tipo de polígono, basta arranjar os hexágono de forma que os ângulos de $120^{\circ}$ estejam posicionados em torno de um vértice (figura \ref{hex8}), observando para que os lados congruentes coincidam.

\begin{figure}[H]
\centering
\includegraphics[width=300bp]{hex8}
\label{hex8}
\caption{Ladrilhamento com hexágonos do tipo 3.}
\end{figure}


Quanto aos demais polígonos irregulares convexos ( com mais de 7 lados) esses não ladrilham o plano, pois não é possível realizar um arranjo desse tipo de polígono em torno de um vértice, sem que tenhamos espaços em branco ou sobreposições.

Observamos que nesse primeiro momento trabalhamos com ladrilhamentos monoédricos, ou seja, formados por ladrilhos de um só tipo. Mas se  ladrilhos forem polígonos regulares, de vários tipos. O que ocorre?


\practice{Vamos ladrilhar}

\begin{task}{Ladrilhamentos usando losangos}

Um losango (que também é um paralelogramo) fornece um padrão que cria a aparência de cubos em perspectiva, como ilustra a figura \ref{losango} .

	\begin{figure}[H]
	\centering
	\includegraphics[width=150bp]{ladrilhamento47}
	\label{losango}
    \caption{Ladrilhamento com losangos}
	\end{figure}
	
\begin{enumerate}
			\item O que esse losango possui de especial?
			\item Será que é possível construir outro tipo de ladrilhamento com esses losangos? (Use o GeoGebra ou os losangos do encarte)
\end{enumerate}
\end{task}

\begin{task}{Ladrilhamentos usando trapézios}
\begin{enumerate}
\item Será que é possível ladrilhar o plano usando trapézios? Justifique.

\item Um trapézio muito interessante, chamado 60-120 (figura\ref{trap1}, quando se trata de ladrilhamento, é o constituído por três triângulos equiláteros.

\begin{figure}[H]
\centering
\includegraphics[width=300bp]{ladrilhamento48}
\label{trap1}
\caption{Trapézio.}
\end{figure}

\begin{enumerate}
	\item A figura \ref{trap2} ilustra um ladrilhamento parcial feito com esse trapézio, em um hexágono regular (e como o hexágono ladrilha o plano, basta repetir o padrão hexagonal). Construa outro ladrilhamento parcial no hexágono usando apenas trapézios 60-120.

	\begin{figure}[H]
	\centering
	\includegraphics[width=250bp]{ladrilhamento49}
	\caption{Ladrilhamento parcial}
	\label{trap2}
	\end{figure}

	\item Com o trapézio 60-120 também é possível construir ladrilhamentos do tipo faixas decorativas, como ilustra a figura  \ref{trap3}. 

	\begin{figure}[H]
	\centering
	\includegraphics[width=300bp]{ladrilhamento50}
	\caption{Faixa decorativa}
     \label{trap3}
	\end{figure}
	
Faça um ladrilhamento diferente do apresentado usando trapézios 60-120.

\end{enumerate}
\end{enumerate}
\end{task}

\begin{task}{Ladrilhamento com pentágonos irregulares}
Algumas calçadas possuem uma pavimentação diferente, denominada \textit{Cairo Tiling} ou Pavimentação do Cairo (\hyperref[cairo]{figura\ref{cairo}}). Esse nome foi criado devido ao fato de várias ruas do Cairo (no Egito) serem pavimentadas com pedras desse tipo.

	\begin{figure}[H]
	\centering
	\includegraphics[width=250bp]{ladrilhamento17}
	\label{cairo}
	\caption{Cairo Tiling. Fonte \href{encurtador.com.br/eksyU}{Arab West Report}}
	\end{figure}

Observamos que os ladrilhos são pentágonos irregulares.

	\begin{enumerate}
		\item Desenhe um ladrilho pentagonal diferente do apresentado na figura, que possa pavimentar o plano. Explique como ele ladrilhará o plano.
		\item Esse é o único ladrilho pentagonal possível? Compare com seus colegas.
	\end{enumerate}

\end{task}



\begin{task}{Ladrilhamento dual}
O diagrama a seguir ilustra um ladrilhamento de quadrados. Adicionando um ponto ao centro de cada quadrado e unindo-os formando quadrados adjacentes, formamos um novo ladrilhamento, que se denomina dual do ladrilhamento original.

	\begin{figure}[H]
	\centering
	\includegraphics[width=350bp]{dual}
	\caption{Ladrilhamento Dual}
	\end{figure}

	\begin{enumerate}
		\item Que tipo de ladrilhamento é o dual do ladrilhamento de quadrados original?
		\item Desenhe um ladrilhamento de hexágonos regulares. Desenhe e descreva o seu dual;
		\item Desenhe um ladrilhamento de triângulos equiláteros. Desenhe e descreva o seu dual.
	\end{enumerate}

\end{task}



%%%%%%%%%%%%%%%%%%%%%%%%


\explore{Ladrilhando com polígonos regulares}



\begin{task}{Ladrilhando o plano com polígonos de mais de um tipo}\label{ladtipos}

\textbf{Parte 1 :} Escolha dois polígonos regulares diferentes que possam ser usados para criar um ladrilhamento. 
\begin{enumerate}
\item Quais polígonos foram escolhidos?
\item Desenhe o ladrilhamento usando papel milimetrado usando os dois polígonos. 
\item É possível fazer um ladrilhamento, arranjando  apenas dois polígonos ao redor de um vértice? Justifique! 
\end{enumerate}

\textbf{Parte 2:} Escolha três polígonos regulares diferentes que possam ser usados para criar um ladrilhamento. 
\begin{enumerate}
\item Quais polígonos foram escolhidos?
\item Desenhe o ladrilhamento usando papel milimetrado usando os dois polígonos. 
\item É possível fazer um ladrilhamento, arranjando  apenas três polígonos ao redor de um vértice? Justifique! 
\end{enumerate}


\textbf{Parte 3:} repita o procedimento realizado na \textbf{Parte 2 }usando 4,5, 6 e 7 polígonos.

\textbf{Parte 4:} socialize suas respostas com os colegas.

\end{task}


\arrange{Ladrilhando com polígonos regulares}

Na atividade \ref{ladtipos} investigamos  quais os tipos de ladrilhamentos lado-lado, cujos ladrilhos são polígonos regulares de mais de um tipo, podem ser construídos. Esses ladrilhamentos são denominados semirregulares. Alguns desses ladrilhamentos são encontrados em revestimentos de pisos. Alguns são formados por quadrados e octógonos regulares  e outros são formados por quadrados, triângulos equiláteros e hexágonos regulares (figura \ref{lad_tp1}).


\begin{figure}[H]
\centering
\includegraphics[width=300bp]{ladrilhamento18}
\label{lad_tp1}
\caption{Revestimentos. Fonte: Google Imagens}
\end{figure}

Um arranjo de polígonos ao redor de um determinado vértice é denominado de configuração. E existe uma notação para ela, que se deve ao número de lados dos polígonos que constituem o arranjo. Por exemplo, a configuração 4-8-8 significa que ao redor de um vértice há um quadrado, um octágono e um octágono, nesta ordem e em qualquer vértice do ladrilhamento. Já a configuração 3- 4- 6- 4 significa que ao redor de um vértice há um triângulo, um quadrado, um hexágono e um quadrado, nesta ordem e em qualquer vértice do ladrilhamento ((figura \ref{lad_tp2})).


\begin{figure}[H]
\centering
\includegraphics[width=350bp]{ladtp_2}
\label{lad_tp2}
\caption{Configuração.}
\end{figure}

A existência de ladrilhamentos lado-a-lado cujas peças são polígonos regulares já era conhecida pelos antigos pitagóricos da Matemática grega. Porém, a primeira pessoa a exibir os ladrilhamentos regulares e semirregulares ( formados por polígonos regulares de tipos diferentes) foi J. Kepler em um trabalho publicado no início do século XVII, onde consta o seguinte resultado:

\begin{observation}{Teorema de Kepler}
Existem exatamente 11 maneiras de se cobrir o plano utilizando-se exclusivamente polígonos regulares sujeitos as seguintes condições:
\begin{itemize}
\item	se dois polígonos regulares intersectam-se, então essa interseção é um lado ou um vértice comum.
\item	A distribuição dos polígonos regulares ao redor de cada vértice é sempre a mesma.
\end{itemize}

\end{observation}



\begin{knowledge}
Johannes Kepler (1571-1630), em sua obra \textit{Harmonia do Mundo}, de 1619,trouxe as primeiras investigações referentes à teoria da pavimentação do plano euclidiano utilizando polígonos regulares, apontando um tratamento matemático para o problema.
\end{knowledge}




Assim, considerando que em torno de um vértice podemos colocar um número $k$ de polígonos regulares, e que $60^{\circ}$ é o menor ângulo interno de um polígono regular, então o maior valor de $k$ é dado por $\displaystyle \frac{360^{\circ}}{60^{\circ}} = 6$, que corresponde a 6 triângulos equiláteros. Por outro lado, o menor número de polígonos necessários para realizar um ladrilhamento em torno de um vértice é 3, temos que  $3\leq k \leq 6$.

Observa-se então que será possível construir ladrilhamentos do plano com 3, 4, 5 ou 6 polígonos regulares em torno de cada vértice (figura \ref{lad_tp3})


\begin{figure}[H]
\centering
\includegraphics[width=300bp]{ladrilhamento21}
\label{lad_tp3}
\caption{Número de polígonos em torno de um vértice.}
\end{figure}

Examinaremos cada um dos casos separadamento.

Considerando $k=3$, ou seja, supondo que três polígonos regulares são arranjados em torno de um vértica de modo que não haja nem lacunas nem sobreposições, o primeiro com $n_1$ lados, o segundo com $n_2$ lados e o terceiro com $n_3$ lados (figura \ref{lad_tp4}).

\begin{figure}[H]
\centering
\includegraphics[width=200bp]{ladrilhamento22}
\label{lad_tp4}
\caption{Número de polígonos em torno de um vértice.}
\end{figure}

Como o ângulo interno de cada $n_i$-ágono, com $i=1,2,3$ é dado por

\begin{equation*}
180^{\circ}\left(1-\frac{2}{n_i}\right)
\end{equation*}

E como a soma dos ângulos em torno de um vértice é $360^{\circ}$, temos

\begin{equation*}
180^{\circ}\left(1-\frac{2}{n_1}\right)+180^{\circ}\left(1-\frac{2}{n_2}\right)+180^{\circ}\left(1-\frac{2}{n_3}\right)=360^{\circ}.
\end{equation*}

Daí,

\begin{equation*}
\frac{1}{n_1}+\frac{1}{n_2}+\frac{1}{n_3}=\frac{1}{2}
\end{equation*}

Para determinar as soluções inteiras e positivas da equação anterior, vamos supor, sem perda de generalidade, que $n_1\leq n_2\leq n_3$.

Logo,

\begin{equation*}
\frac{1}{n_2}\leq\frac{1}{n_1},\quad \frac{1}{n_3}\leq\frac{1}{n_1},
\end{equation*}

e portanto,

\begin{equation*}
\frac{1}{2}=\frac{1}{n_1}+\frac{1}{n_2}+\frac{1}{n_3}\leq\frac{1}{n_1}+\frac{1}{n_1}+\frac{1}{n_1}=\frac{3}{n_1},
\end{equation*}

ou seja, $n_1\leq6$.

Suponhamos que $n_1=3$, ou seja, que um dos polígonos dispostos ao redor do vértice seja um triângulo equilátero. Então,

\begin{equation*}
\frac{1}{3}+\frac{1}{n_2}+\frac{1}{n_3}=\frac{1}{2}.
\end{equation*}

Daí,

\begin{equation*}
\frac{1}{n_2}+\frac{1}{n_3}=\frac{1}{6}.
\end{equation*}

Ou ainda,

\begin{equation*}
\frac{1}{n_3}=\frac{n_2-6}{6n_2},
\end{equation*}

logo, $n_2\geq7$.

Por outro lado, como $n_2\geq n_3$, temos,

\begin{equation*}
\frac{n_2-6}{6n_2}=\frac{1}{n_3}\leq\frac{1}{n_2},
\end{equation*}

O que implica $n_2\leq12$.

Assim, substituindo os valores possíveis para $n_2$ e como $n_3$ é um número inteiro, obtemos as seguintes soluções:



\begin{table}[H]
\centering
\setlength\tabcolsep{5mm}
\begin{tabu} to \textwidth{|c|c|c|}
%\thead
\hline
$\bm{n_1}$ & $\bm{n_2}$ & $\bm{n_3}$ \\
\hline
$3$ & $7$ & $42$ \\
\hline
$3$ & $8$ & $24$ \\
\hline
$3$ & $9$ & $18$ \\
\hline
$3$ & $10$ & $15$ \\
\hline
$3$ & $12$ & $12$ \\
\hline
\end{tabu}
\end{table}

Suponhamos agora que $n_1=4$, ou seja, que um dos polígonos dispostos ao redor do vértice seja um quadrado. Então,

\begin{equation*}
\frac{1}{4}+\frac{1}{n_2}+\frac{1}{n_3}=\frac{1}{2}.
\end{equation*}

Daí,

\begin{equation*}
\frac{1}{n_2}+\frac{1}{n_3}=\frac{1}{4}.
\end{equation*}

Ou ainda,

\begin{equation*}
\frac{1}{n_3}=\frac{n_2-4}{4n_2},
\end{equation*}

logo, $n_2\geq5$.

Por outro lado, como $n_2\geq n_3$, temos,

\begin{equation*}
\frac{n_2-4}{4n_2}=\frac{1}{n_3}\leq\frac{1}{n_2},
\end{equation*}

o que implica $n_2\leq8$.

Assim, substituindo os valores possíveis para $n_2$ e como $n_3$ é um número inteiro, obtemos as seguintes soluções:


\begin{table}[H]
\centering
\setlength\tabcolsep{5mm}
\begin{tabu} to \textwidth{|c|c|c|}
%\thead
\hline
${n_1}$ & ${n_2}$ & ${n_3}$ \\
\hline
$4$ & $5$ & $20$ \\
\hline
$4$ & $6$ & $12$ \\
\hline
$4$ & $8$ & $8$ \\ 
\hline
\end{tabu}
\end{table}

Procedendo de forma análoga, considerando $n_1 = 5$, temos $5\leq n_2 \leq6$ e uma única solução da equação

\begin{equation*}
\frac{1}{5}+\frac{1}{n_2}+\frac{1}{n_3}=\frac{1}{2}
\end{equation*}

é $n_2=5$ e $n_3=10$. E, considerando $n_1=6$, a única solução é $n_2 = 6$ e $n_3=6$.

Portanto, as soluções inteiras positivas da equação

\begin{equation*}
\frac{1}{n_1}+\frac{1}{n_2}+\frac{1}{n_3}=\frac{1}{2},
\end{equation*} 

estão descritas na tabela a seguir:

\begin{table}[H]
\setlength\tabcolsep{5mm}
\centering
\begin{tabu} to \textwidth{|c|c|c|}
%\thead
\hline
$\bm{n_1}$ & $\bm{n_2}$ & $\bm{n_3}$ \\
\hline
$3$ & $7$ & $42$ \\
\hline
$3$ & $8$ & $24$ \\
\hline
$3$ & $9$ & $18$ \\
\hline
$3$ & $10$ & $15$ \\
\hline
$3$ & $12$ & $12$ \\
\hline
$4$ & $5$ & $20$ \\
\hline
$4$ & $6$ & $12$ \\
\hline
$4$ & $8$ & $8$ \\
\hline
$5$ & $5$ & $10$ \\
\hline
$6$ & $6$ & $6$ \\
\hline
\end{tabu}
\end{table}

Considerando $k=4$, ou seja, supondo que quatro polígonos regulares são arranjados em torno de um vértice de modo que não haja nem lacunas nem sobreposições, o primeiro com $n_1$ lados, o segundo com $n_2$ lados, o terceiro com $n_3$ lados e o quarto com $n_4$ lados (figura \ref{lad_tp5}).

\begin{figure}[H]
\centering
\includegraphics[width=200bp]{ladrilhamento23}
\label{lad_tp5}
\caption{Arranjo de 4 polígonos ao redor de um vértice.}
\end{figure}

As possíveis combinações de quatro polígonos regulares ao redor do vértice corresponde a determinar as soluções inteiras e positivas da equação:

\begin{equation*}
180^{\circ}\left(1-\frac{2}{n_1}\right)+180^{\circ}\left(1-\frac{2}{n_2}\right)+180^{\circ}\left(1-\frac{2}{n_3}\right)+180^{\circ}\left(1-\frac{2}{n_4}\right)=360^{\circ},
\end{equation*}

que equivale a

\begin{equation*}
\frac{1}{n_1}+\frac{1}{n_2}+\frac{1}{n_3}+\frac{1}{n_4}=1
\end{equation*}

Aplicando uma técnica de análise similar a anterior, é possível verificar que as únicas soluções inteiras e positivas dessa última equação, com $3\leq n_1\leq n_2 \leq n_3 \leq n_4$, são as seguintes

\begin{table}[H]
\centering
\setlength\tabcolsep{5mm}
\begin{tabu} to \textwidth{|c|c|c|c|}
%\thead
\hline
$\bm{n_1}$ & $\bm{n_2}$ & $\bm{n_3}$ & $\bm{n_4}$ \\
\hline
$3$ & $3$ & $4$ & $12$ \\
\hline
$3$ & $3$ & $6$ & $6$ \\
\hline
$3$ & $4$ & $4$ & $6$ \\
\hline
$4$ & $4$ & $4$ & $4$ \\
\hline
\end{tabu}
\end{table}

Usando procedimentos semelhantes, determinamos que a classificação das possíveis combinações de $k=5$ polígonos regulares em torno de um vértice de modo que não haja nem lacunas nem sobreposições corresponde a determinação das soluções inteiras e positivas da equação

\begin{equation*}
180^{\circ}\left(1-\frac{2}{n_1}\right)+180^{\circ}\left(1-\frac{2}{n_2}\right)+180^{\circ}\left(1-\frac{2}{n_3}\right)+180^{\circ}\left(1-\frac{2}{n_4}\right)+180^{\circ}\left(1-\frac{2}{n_5}\right)=360^{\circ},
\end{equation*}

que equivale a 

\begin{equation*}
\frac{1}{n_1}+\frac{1}{n_2}+\frac{1}{n_3}+\frac{1}{n_4}+\frac{1}{n_5}=\frac{3}{2}.
\end{equation*}

As únicas soluções inteiras e positivas dessa equação, com $3\leq n_2 \leq n_3 \leq n_4 \leq n_5$, estão descritas na tabela

\begin{table}[H]
\centering
\setlength\tabcolsep{5mm}
\begin{tabu} to \textwidth{|c|c|c|c|c|}
%\thead
\hline
$\bm{n_1}$ & $\bm{n_2}$ & $\bm{n_3}$ & $\bm{n_4}$ & $\bm{n_5}$ \\
\hline
$3$ & $3$ & $3$ & $3$ & $6$ \\
\hline
$3$ & $3$ & $3$ & $4$ & $4$ \\
\hline
\end{tabu}
\end{table}

Finalmente, considerando $k=6$ polígonos regulares ao redor de um vértice, temos que determinar as soluções inteiras e positivas da equação

\begin{equation*}
\frac{1}{n_1}+\frac{1}{n_2}+\frac{1}{n_3}+\frac{1}{n_4}+\frac{1}{n_5}+\frac{1}{n_6}=2,
\end{equation*}

cuja única solução é $n_1=n_2=n_3=n_4=n_5=n_6=3$

Resumindo, as possíveis combinações de polígonos regulares que podem ser arranjados em torno de um vértice a modo de cobrir o espaço ao redor dele sem lacunas nem estão dispostas na tabela:

\setlength\tabcolsep{5mm}
\begin{longtabu} to \textwidth{|c|c|c|c|c|c|c|}
\hline\endfirsthead
\cellcolor{\currentcolor!80}{\textcolor{white}{$\bm{n_1}$}} & \cellcolor{\currentcolor!80}{\textcolor{white}{$\bm{n_2}$}} & \cellcolor{\currentcolor!80}{\textcolor{white}{$\bm{n_3}$}} & \cellcolor{\currentcolor!80}{\textcolor{white}{$\bm{n_4}$}}& \cellcolor{\currentcolor!80}{\textcolor{white}{$\bm{n_5}$}} & \cellcolor{\currentcolor!80}{\textcolor{white}{$\bm{n_6}$}} & \cellcolor{\currentcolor!80}{\textcolor{white}{\textbf{Configuração}}} \\
\hline
$3$ & $7$ & $42$ & & & & $3-7-42$ \\
\hline
$3$ & $8$ & $24$ & & & & $3-8-24$ \\
\hline
$3$ & $9$ & $18$ & & & & $3-9-18$ \\
\hline
$3$ & $10$ & $15$ & & & & $ 3-10-15$ \\
\hline
$3$ & $12$ & $12$ & & & & $3-12-12$ \\
\hline
$4$ & $5$ & $20$ & & & & $4-5-20$ \\
\hline
$4$ & $6$ & $12$ & & & & $4-6-12$ \\
\hline
$4$ & $8$ & $8$ & & & & $4-8-8$ \\
\hline
$5$ & $5$ & $10$ & & & & $5-5-10$ \\
\hline
$6$ & $6$ & $6$ & & & & $6-6-6$ \\
\hline
$3$ & $3$ & $4$ & $12$ & & & $3-3-4-12$ \\
\hline
$3$ & $3$ & $6$ & $6$ & & & $3-3-6-6$ \\
\hline
$3$ & $4$ & $4$ & $6$ & & & $3-4-4-6$ \\
\hline
$4$ & $4$ & $4$ & $4$ & & & $4-4-4-4$ \\
\hline
$3$ & $3$ & $3$ & $3$ & $6$ & & $3-3-3-3-6$ \\
\hline
$3$ & $3$ & $3$ & $4$ & $4$ & & $3-3-3-4-4$ \\
\hline
$3$ & $3$ & $3$ & $3$ & $3$ & $3$ & $3-3-3-3-3-3$ \\
\hline
\end{longtabu}

Será que todas as  combinações /configurações que constam na tabela anterior podem ser construídas geometricamente de modo a ladrilhar o plano?


\explore{Ladrilhamentos semirregulares} \label{at_ladsemi}

\begin{task} {Investigando as combinações}

\textbf{Parte 1:} Considerando as  combinações obtidas para $k=3$.
 
\setlength\tabcolsep{5mm}
\begin{tabu} to \textwidth{|c|c|c|c|}
\hline
\cellcolor{\currentcolor!80}{\textcolor{white}{$\bm{n_1}$}} & \cellcolor{\currentcolor!80}{\textcolor{white}{$\bm{n_2}$}} & \cellcolor{\currentcolor!80}{\textcolor{white}{$\bm{n_3}$}} &  \cellcolor{\currentcolor!80}{\textcolor{white}{\textbf{Configuração}}} \\
\hline
$3$ & $7$ & $42$ & $3-7-42$ \\
\hline
$3$ & $8$ & $24$ &  $3-8-24$ \\
\hline
$3$ & $9$ & $18$ &  $3-9-18$ \\
\hline
$3$ & $10$ & $15$ &  $ 3-10-15$ \\
\hline
$3$ & $12$ & $12$ &  $3-12-12$ \\
\hline
$4$ & $5$ & $20$ &  $4-5-20$ \\
\hline
$4$ & $6$ & $12$ &  $4-6-12$ \\
\hline
$4$ & $8$ & $8$ &  $4-8-8$ \\
\hline
$5$ & $5$ & $10$ &  $5-5-10$ \\
\hline
$6$ & $6$ & $6$ &  $6-6-6$ \\
\hline

\end{tabu}

\begin{enumerate}

\item	Com quais delas é possível realizar um ladrilhamento no plano? Justifique?
\item	Justifique por que com algumas dessas combinações não é possível ladrilhar o plano.

\end{enumerate}


\textbf{Parte 2:} Considerando as  combinações obtidas para $k=4$. 
\setlength\tabcolsep{5mm}
\begin{tabu} to \textwidth{|c|c|c|c|c|}
\hline
\cellcolor{\currentcolor!80}{\textcolor{white}{$\bm{n_1}$}} & \cellcolor{\currentcolor!80}{\textcolor{white}{$\bm{n_2}$}} & \cellcolor{\currentcolor!80}{\textcolor{white}{$\bm{n_3}$}} & \cellcolor{\currentcolor!80}{\textcolor{white}{$\bm{n_4}$}}&  \cellcolor{\currentcolor!80}{\textcolor{white}{\textbf{Configuração}}} \\
\hline

$3$ & $3$ & $4$ & $12$ & $3-3-4-12$ \\
\hline
$3$ & $3$ & $6$ & $6$ &  $3-3-6-6$ \\
\hline
$3$ & $4$ & $4$ & $6$ & $3-4-4-6$ \\
\hline
$4$ & $4$ & $4$ & $4$ &  $4-4-4-4$ \\
\hline
\end{tabu}

Observe que ao dispor os polígonos ao redor de um vértice podemos obter configurações distintas ( figura \ref{ladreg1}

\begin{figure}[H]
\centering
\includegraphics[width=300bp]{ladreg1}
\label{ladreg1}
\caption{Arranjo de 4 polígonos ao redor de um vértice.}
\end{figure}

\begin{enumerate}
\item Justifique por que elas são diferentes? 
\item Com quais delas é possível ladrilhar o plano? Justifique!

\end{enumerate}

\textbf{Parte 3:} Considerando as  combinações obtidas para $k=5$ e $k=6$.

\setlength\tabcolsep{5mm}
\begin{longtabu} to \textwidth{|c|c|c|c|c|c|c|}
\hline\endfirsthead
\cellcolor{\currentcolor!80}{\textcolor{white}{$\bm{n_1}$}} & \cellcolor{\currentcolor!80}{\textcolor{white}{$\bm{n_2}$}} & \cellcolor{\currentcolor!80}{\textcolor{white}{$\bm{n_3}$}} & \cellcolor{\currentcolor!80}{\textcolor{white}{$\bm{n_4}$}}& \cellcolor{\currentcolor!80}{\textcolor{white}{$\bm{n_5}$}} & \cellcolor{\currentcolor!80}{\textcolor{white}{$\bm{n_6}$}} & \cellcolor{\currentcolor!80}{\textcolor{white}{\textbf{Configuração}}} \\
\hline
$3$ & $3$ & $3$ & $3$ & $6$ & & $3-3-3-3-6$ \\
\hline
$3$ & $3$ & $3$ & $4$ & $4$ & & $3-3-3-4-4$ \\
\hline
$3$ & $3$ & $3$ & $3$ & $3$ & $3$ & $3-3-3-3-3-3$ \\
\hline
\end{longtabu}

Com quais delas é possível ladrilhar o plano? Justifique!

\textbf{Parte 4:}
Observe a tabela e responda:

\setlength\tabcolsep{5mm}
\begin{longtabu} to \textwidth{|c|c|c|c|c|c|c|}
\hline\endfirsthead
\cellcolor{\currentcolor!80}{\textcolor{black}{$\bm{n_1}$}} & \cellcolor{\currentcolor!80}{\textcolor{black}{$\bm{n_2}$}} & \cellcolor{\currentcolor!80}{\textcolor{black}{$\bm{n_3}$}} & \cellcolor{\currentcolor!80}{\textcolor{black}{$\bm{n_4}$}}& \cellcolor{\currentcolor!80}{\textcolor{black}{$\bm{n_5}$}} & \cellcolor{\currentcolor!80}{\textcolor{black}{$\bm{n_6}$}} & \cellcolor{\currentcolor!80}{\textcolor{black}{\textbf{Configuração}}} \\
\hline
$3$ & $7$ & $42$ & & & & $3-7-42$ \\
\hline
$3$ & $8$ & $24$ & & & & $3-8-24$ \\
\hline
$3$ & $9$ & $18$ & & & & $3-9-18$ \\
\hline
$3$ & $10$ & $15$ & & & & $ 3-10-15$ \\
\hline
$3$ & $12$ & $12$ & & & & $3-12-12$ \\
\hline
$4$ & $5$ & $20$ & & & & $4-5-20$ \\
\hline
$4$ & $6$ & $12$ & & & & $4-6-12$ \\
\hline
$4$ & $8$ & $8$ & & & & $4-8-8$ \\
\hline
$5$ & $5$ & $10$ & & & & $5-5-10$ \\
\hline
$6$ & $6$ & $6$ & & & & $6-6-6$ \\
\hline
$3$ & $3$ & $4$ & $12$ & & & $3-3-4-12$ \\
\hline
$3$ & $3$ & $6$ & $6$ & & & $3-3-6-6$ \\
\hline
$3$ & $4$ & $4$ & $6$ & & & $3-4-4-6$ \\
\hline
$4$ & $4$ & $4$ & $4$ & & & $4-4-4-4$ \\
\hline
$3$ & $3$ & $3$ & $3$ & $6$ & & $3-3-3-3-6$ \\
\hline
$3$ & $3$ & $3$ & $4$ & $4$ & & $3-3-3-4-4$ \\
\hline
$3$ & $3$ & $3$ & $3$ & $3$ & $3$ & $3-3-3-3-3-3$ \\
\hline
\end{longtabu}

Quais as configurações que produzem ladrilhamentos do plano?


\end{task}

\arrange{Ladrilhamentos semirregulares}

Na primeira parte da atividade \ref{at_ladsemi} solicitamos investigar quais das dez possíveis combinações de 3 polígonos regulares, definiam um ladrilhamento semirregular. 
Vamos começar nossa investigação considerando uma combinação de três polígonos $3-n_2-n_3$, dos quais um é um triângulo regular($n_1=3$). 
Nessa configuração, $3$ se refere a um triângulo equilátero, $n_2$ e $n_3$ são polígonos regulares com lados $n_2$ e $n_3$, respectivamente. Os comprimentos das arestas de todos os polígonos são iguais. Para produzir um ladrilhamento, o triângulo deve ter a mesma configuração  de polígonos em todos os seus três vértices; ou seja, eles devem ser "rodeados" por polígonos da mesma maneira. A figura \ref{semi1} ilustra essa situação.  
 
\begin{figure}[H]
\centering
\includegraphics[width=150bp]{semi1}
\label{semi1}
\caption{Combinação de polígonos ao redor do vértice de um triângulo equilátero.}
\end{figure}

Na figura \ref{semi1}, os polígonos $n_2$ e $n_3$ estão dispostos para completar o vértice azul. No entanto, os vértices vermelhos serão completados apenas quando o polígono em "$??$" for $n_2$ ou $n_3$. Portando, a única maneira de fazer isso é definindo $n_2 = n_3$. Isso significa que o padrão se torna $3-n_2- n_2$. 
Isso elimina as configurações $3-7-42$, $3-8-24$, $3-9-18$ e $3-10-15$. Portanto, existe apenas uma combinação, $3-12-12$, com a qual se permite um ladrilhamento do plano ( figura \ref{semi2})

\begin{figure}[H]
\centering
\includegraphics[width=250bp]{semi2}
\label{semi2}
\caption{Ladrilhamento produzido a partir da configuração $3-12-12$.}
\end{figure}

O mesmo argumento vale para todos os polígonos de lados ímpares que devem ser combinados com outros dois outros polígonos regulares ( figura \ref{semi3}). Isso elimina as combinações $4-5-20$ (leia-o como $5-20-4$) e $5-5-10$.


\begin{figure}[H]
\centering
\includegraphics[width=150bp]{semi3}
\label{semi3}
\caption{Combinação de polígonos ao redor do vértice de um pentágono regular.}
\end{figure}

Pensando nas combinações de três polígonos ($3-n_2-n_3$), dos quais um é um quadrado (polígono regular de quatro lados). A  configuração seria $4-n_2-n_3$. A Figura \ref{semi4} ilustra que pode haver apenas duas maneiras nas quais a configuração $4-n_2n_3$ se estenderia, mantendo a mesma configuração de polígonos em todos os vértices. 

\begin{figure}[H]
\centering
\includegraphics[width=300bp]{semi4}
\label{semi4}
\caption{Combinação de polígonos ao redor do vértice de um quadrado.}
\end{figure}

Nesse caso ou $n_2= n_3$ ou $n_2$ e $n_3$ são colocados alternadamente ao redor do vértice. Isso qualifica as combinações $4-8-8$ e $4-6-12$ como ilustra a figura \ref{semi5}.

\begin{figure}[H]
\centering
\includegraphics[width=400bp]{semi5}
\label{semi5}
\caption{Ladrilhamentos nas configurações $4-8-8$ e $4-6-12$.}
\end{figure}

Obviamente a configuração $6-6-6$ também produz um ladrilhamento, que é o ladrilhamento regular feito com hexágonos, como já visto.

Resumindo, existem apenas quadro ladrilhamentos usando três polígonos regulares
ao redor de um vértice, e estes tem as sguintes configurações  $3-12-12$, $4-8-8$,  $4-6-12$  e $6-6-6$.

Na parte 2 da atividade \ref{at_ladsemi} consideramos as combinações obtidas para 4 polígonos regulares. 
Vamos investigar o que ocorre quando ao redor de um vértice tivermos  dois triângulos e outros dois polígonos, ou seja uma configuração do tipo $3-3-n_3-n_4$. Esses polígonos podem ser organizados como $3-3-n_3-n_4$ ou $3-n_3-3-n_4$. 
Começaremos nossa investigação colocando os polígonos $n_3$ e $n_4$ no vértice azul (Figura \ref{semi6}). 


\begin{figure}[H]
\centering
\includegraphics[width=150bp]{semi6}
\label{semi6}
\caption{Combinação de polígonos ao redor de um vértice de um triângulo equilátero.}
\end{figure}


Nesse caso, a configuração dos polígonos em torno do vértice azul seria $3-3-n_3-n_4$. Para ter a mesma configuração no vértice vermelho, os polígonos devem ser organizados no sentido contrário. No vértice branco, entretanto, haverá três triângulos, perfazendo a soma total dos ângulos $180^{\circ}$, não deixando nenhuma possibilidade de encaixar qualquer outro polígono. 

Portanto, configurações do tipo $3-3-n_3-n_4$ não produzem um ladrilhamento semirregular. Isso mostra a impossibilidade das configurações $3-3-4-12$ e $3-3-6-6$.

A outra possiblidade é um  arranjo do tipo $3-n_3-3-n_4$, ao redor do vértice azul, como ilustra a figura \ref{semi7}.



\begin{figure}[H]
\centering
\includegraphics[width=150bp]{semi7}
\label{semi7}
\caption{Combinação de polígonos ao redor de um vértice de um triângulo equilátero.}
\end{figure}

Observamos que para manter a mesma configuração no vértice branco, os polígonos $n_3$  e $n_4$ teriam que ser iguais. Assim, a configuração $3-3-4-12$ (dois triângulos, um quadrado e um dodecágono) não produzirá um ladrilhamento. Porém, a combinação $3-3-6-6$ quando reorganizada como $3-6-3-6$ produzirá um ladrilhamento semirregular ( figura \ref{semi8}).



\begin{figure}[H]
\centering
\includegraphics[width=250bp]{semi8}
\label{semi8}
\caption{Ladrilhamento na configuração $3-6-3-6$.}
\end{figure}

Fazendo um raciocínio semelhante, podemos concluir que a combinação $3-4-4-6$ não produz um ladrilhamento semirregular. No entanto, se os polígonos forem reorganizados como $3-4-6-4$ então, é possível produzir um ladrilhamento semirregular, como ilustra a figura \ref{semi9}.

\begin{figure}[H]
\centering
\includegraphics[width=250bp]{semi9}
\label{semi9}
\caption{Ladrilhamento na configuração $3-4-6-4$.}
\end{figure}


E obviamente, a configuração $4-4-4-4$ produz o ladrilhamento regular formado apenas por quadrados, conforme foi visto anteriormente.

A terceira parte da atividade \ref{at_ladsemi}, tinha como objetivo investigar as combinações com 5 e 6 polígonos ao redor de um vértice.

A combinação para 6 poligonos $3-3-3-3-3-3$ produz o ladrilhamento regular formado por triângulos equiláteros.

Ainda restam duas configurações de 5 polígonos. Vamos  considerá-las separadamente. A configuração $3-3-3-4-4$, também pode ser arranjada como  $3-3- 4-3-4$. A figura \ref{semi10} ilustra um arranjo em torno do vértice de um triângulo para ambas configurações.

\begin{figure}[H]
\centering
\includegraphics[width=250bp]{semi10}
\label{semi10}
\caption{Combinação de polígonos ao redor de um vértice de um triângulo equilátero.}
\end{figure}

Observamos que é possível criar ladrilhamentos semirregulares para as configurações  $3-3-3-4-4$ e $3-3-4-3-4$ (figura \ref{semi11}).

\begin{figure}[H]
\centering
\includegraphics[width=400bp]{semi11}
\label{semi11}
\caption{Ladrilhamentos semirregulares para as configurações  $3-3-3-4-4$ e $3-3-4-3-4$.}
\end{figure}

 

E, finalmente, a Figura \ref{semi12} ilustra o ladrilhamento produzido pela coonfiguração $3-3-3-3-6$.


\begin{figure}[H]
\centering
\includegraphics[width=250bp]{semi12}
\label{semi11}
\caption{Ladrilhamento produzido pela coonfiguração 3-3-3-3-6.}
\end{figure}


A tabela abaixo ilustra um resumo das configurações apresentadas.

%pensa em um pessoa que odeia tabelas do tex ( por isso coloquei uma figura)

 \begin{figure}[H]
\centering
\includegraphics[width=400bp]{semi13}

\end{figure}

A figura \ref{semit} ilustra todos os ladrilhamentos semirregulares.


\begin{figure}[H]
\centering
\includegraphics[width=400bp]{semit}
\label{semit}
\caption{Ladrilhamentos semirregulares.}
\end{figure}

Ou seja, acabamos de mostrar que existem exatamente 11 maneiras de se cobrir o plano utilizando-se exclusivamente polígonos regulares sujeitos as seguintes condições:
\begin{itemize}
\item se dois polígonos regulares intersectam-se, então essa interseção é um lado ou um vértice comum.
\item A distribuição dos polígonos regulares ao redor de cada vértice é sempre a mesma.
\end{itemize}

%%%%%%%%%%%%%%%




\explore{Outros ladrilhamentos}


\begin{task}{Ladrilhando com retângulos} \label{at_outros1}

Os pisos de algumas casa são cobertos por ladrilhos retangulares. O padrão ilustrado na figura \ref{peixe} é denominado espinha de peixe. 

	\begin{figure}[H]
	\centering
	\includegraphics[width=250bp]{ladrilhamento16}
\label{peixe}
	\caption{Ladrilhamento tipo espinha de peixe. Fonte: \href{http://www.tilehomeguide.com/tile-patterns-the-ultimate-quick-read-beginners-guide-including-secrets-of-tile-professionals-revealed/}{The Tile Home Guide}}
	\end{figure}

 Em papel quadriculado, crie dois ladrilhamentos  diferentes a partir de ladrilhos retangulares congruentes.

\end{task}


\begin{task}{Ladrilhamento do tipo Penrose}\label{at_outros2}

Parte da rua central de Helsinque na Finlândia (figura \ref{pen1} foi transformada em uma rua para pedestres e pavimentada com ladrilhos do tipo Penrose. Esses ladrilhos foram inventados pelo matemático inglês Roger Penrose na década de 1970.

\begin{figure}[H]
	\centering
	\includegraphics[width=300bp]{ladrilhamento51}
\label{pen1}
\caption{Pavimento da rua central de Helsinque. Fonte: Adaptado de  \href{https://stevethings.wordpress.com/2015/12/21/penrose-tiling/} {Stevethings} }
\end{figure}

Observe que os ladrilhos têm em duas formas diferentes, a pipa e a flecha. Esses quadriláteros podem ser construídos a partir de um pentágono regular, como ilustra a figura \ref{pen2}.

	\begin{figure}[H]
	\centering
	\includegraphics[width=300bp]{ladrilhamento52}
\label{pen2}
\caption{Pipa e Flecha}
	\end{figure}
	
	\begin{enumerate}
		\item Construa pipas e flechas usando um pentágono regular.
		\item Copie as formas em um papel de gramatura mais alta. Recorte-as e crie dois ladrilhamentos diferentes.
		\item Escreva um breve parágrafo explicando quais transformações geométricas você usou para criar o ladrilhamento no item anterior desta atividade. 
		\item O que os ladrilhamentos construídos nessa atividade possuem de diferente dos ladrilhamentos construídos anteriormente?
	\end{enumerate}
	
\end{task}


\begin{task}{Um ladrilhamento no estilo Escher}\label{at_outros3}
\begin{enumerate}
\item Vamos usar um triângulo equilátero para compor um ladrilhamento no estilo Escher, para isso:  
\begin{enumerate}

	\item Desenhe um triângulo equilátero em um papel. Recorte-o  cole-o em uma folha de papel de gramatura alta para construção. Recorte o triângulo novamente.

	\item Dentro do triângulo, desenhe uma curva que conecta dois vértices adjacentes. Corte ao longo da curva para remover uma parte do triângulo ( que vamos denominar peça), como ilustra a figura \ref{escher1}

	\begin{figure}[H]
	\centering
	\includegraphics[width=350bp]{ladrilhamento42}
\label{escher1}
\caption{Instruções para realizar o ladrilhamento.}
	\end{figure}

	\item Gire a peça que você removeu $60^{\circ}$ no sentido anti-horário sobre o vértice da extremidade superior da curva desenhada. Mova a peça para o outro lado do triângulo. Cole-a para completar seu ladrilho, como ilustra a figura \ref{escher2}.

	\begin{figure}[H]
	\centering
	\includegraphics[width=350bp]{ladrilhamento43}
	\label{escher2}
\caption{Instruções para realizar o ladrilhamento.}

	\end{figure}

	\item Para ladrilhar o plano, utilizando o lápis, faça um traço ao redor do ladrilho. Em seguida, gire e desenhe o ladrilho repetidamente até que se tenha o design pretendido.
	\item Adicione cores e desenhos ao ladrilho para que pareça uma obra de arte, como ilustra a figura \ref{escher3}.

	\begin{figure}[H]
	\centering
	\includegraphics[width=200bp]{ladrilhamento44}
	\label{escher3}
\caption{Ladrilhamento parcial.}
	\end{figure}
\end{enumerate}

	\item Repita as etapas de i) a v) usando um paralelogramo e translações, para criar outro ladrilhamento no estilo Escher.

\end{enumerate}
\end{task}




\arrange{Outros ladrilhamentos}


Na primeira seção desse capítulo, vimos que qualquer triângulo e qualquer quadrilátero ladrilham o plano, alguns pentágonos e hexágonos também o fazem. Além disso, exploramos ladrilhamentos semirregulares. 

Nas atividades anteriores \ref{at_outros1} e \ref{at_outros2}   exploramos um pouco mais sobre os ladrilhamentos usando polígonos, mas dessa vez, consideramos alguns polígonos bem especiais. A  atividade  \ref{at_outros1} tratou de ladrilhamentos usando ladrilhos retangulares.

Esse tipo de ladrilhamento é muito comum em  paredes e pisos revestidos com peças denominadas \textit{Subway Tiles }(azulejo de metrô). Construtores e arquitetos  há muito exploram diferentes maneiras de arranjar esses azulejos por razões estruturais e estéticas. Do ponto de vista  matemático, são interessantes como exemplos de ladrilhamentos que (em geral) não são  lado a lado, ou seja, existem arestas que não são comuns a dois polígonos, como ilustra a figura \ref{lad_ret1}.

\begin{figure}[H]
	\centering
	\includegraphics[width=400bp]{lad_ret_1}
	\label{lad_ret1}
\caption{Exemplos de ladrilhamentos.}
	\end{figure}


A figura \ref{lad_ret} ilustra exemplos de como os retângulos podem ser arranjados de forma realizar uma pavimentação.
 \begin{figure}[H]
	\centering
	\includegraphics[width=300bp]{lad_ret}
	\label{lad_ret}
\caption{Exemplos de ladrilhamentos.}
	\end{figure}

A atividade \ref{at_outros2} tratou de ladrilhos especiais que são construídos a partir de um pentágono regular. Essas peças são ditas especiais pois são constituídas por triângulos denominados áureos. De fato, considerando  um pentágono $ABCDE$ de lado unitário, sabemos que cada um dos ângulos internos desse polígono mede $108^{\circ}$ ( figura \ref{penr_9}). 

\begin{figure}[H]
	\centering
	\includegraphics[width=150bp]{penr9}
	\label{penr_9}
\caption{Medida do ângulo interno de um pentágono regular.}
	\end{figure}


Como as diagonais de um polígono regular dividem os ângulos internos em partes congruentes, no caso do pentágono regular, cada ângulo interno é dividido via duas diagonais. Logo o ângulo de  $108^{\circ}$ é dividido em três ângulos de $36^{\circ}$( figura \ref{penr10}). 


\begin{figure}[H]
	\centering
	\includegraphics[width=150bp]{penr10}
	\label{penr_9}
\caption{Medida do ângulo interno de um pentágono regular.}
	\end{figure}


Seja o ponto P como a intersecção das diagonais CE e BD, segue que  o triângulo $DEP$ é isósceles (pois $\hat{E}= 36^{\circ}$, $\hat{D}=\hat{P}$) e portanto, $EP=1$ (figura \ref{penr_11}). 


\begin{figure}[H]
	\centering
	\includegraphics[width=150bp]{penr11}
	\label{penr_11}
\caption{Triângulo $DEP$.}
	\end{figure}


Fazendo $PE =x$ e considerando os triângulos $PBC$ e $BCE$, temos que no triângulo $PBC$ ( figura\ref{penr_12}) , os ângulos internos $P\hat{C}B$ e $B\hat{P}C$ são congruentes e medem $72^{\circ}$ e o ângulo $C\hat{B}P$ mede $36^{\circ}$. 

\begin{figure}[H]
	\centering
	\includegraphics[width=150bp]{penr12}
	\label{penr_12}
\caption{Triângulo $PBC$.}
	\end{figure}


No triângulo $BCE$, os ângulos ângulos internos $E\hat{C}B$ e $C\hat{B}E$ são congruentes e medem $72^{\circ}$ e o ângulo $B\hat{E}C$ mede $36^{\circ}$ (figura \ref{penr_13}).


\begin{figure}[H]
	\centering
	\includegraphics[width=150bp]{penr13}
	\label{penr_11}
\caption{Triângulo $BCE$.}
	\end{figure}


Ou seja, os triângulos $PBC$ e $BCE$ são isósceles e semelhantes. Como estamos considerando $ABCDE$ unitário e  $PE = x$, segue que $EC= EB = 1+x$. E da semelhança de triângulos, resulta a seguinte equação:
$$\frac{x+1}{1}=\frac{1}{x}.$$

Cuja solução positiva é $x=\frac{sqrt{5}-1}{2}$.
Disso resulta que o ponto $P$ divide a diagonal $CE$ em média e extrema razão, ou seja, na razão áurea e por essa razão os triângulos possuem a denominação “triângulos áureos”. 

Em ambos os triângulos, o ângulos são múltiplos de $36^{\circ}$. A figura \ref{penr_14}  ilustra um  triângulo áureo obtusângulo e um  triângulo áureo acutângulo.

 \begin{figure}[H]
	\centering
	\includegraphics[width=150bp]{penr14}
	\label{penr_14}
\caption{Triângulos áureos.}
	\end{figure}



Observamos que os ladrilhos construídos na atividade \ref{at_outros2} ( pipa e flecha ou  \textit{Kite} e \textit{Dart}) são formados por esses triângulos. O ladrilho denominado pipa é formado por dois triângulos áureos acutângulos, unidos por seus lados maiores e o ladrilho flecha por dois triângulos áureos obtusângulos, unidos por seus lados menores, como ilustra a figura \ref{penr_15}. 

 \begin{figure}[H]
	\centering
	\includegraphics[width=150bp]{penr15}
	\label{penr_15}
\caption{Pipa e Flecha.}
	\end{figure}


Existem várias formas de arranjar as pipas e as flechas, mas apenas sete (figura \ref{penr_8}) combinações podem originar um ladrilhamento de  Penrose.

 \begin{figure}[H]
	\centering
	\includegraphics[width=400bp]{penr8}
	\label{penr_8}
\caption{Combinações de pipas e flechas.}
	\end{figure}



Essas formas  ladrilham o plano de forma não periódica , ou seja, não existe um padrão se repetindo indefinidamente. Porém, é possível recobrir o plano combinando esses ladrilhos (figura \ref{penr_16} ). 

 \begin{figure}[H]
	\centering
	\includegraphics[width=200bp]{penr16}
	\label{penr_16}
\caption{Exemplos de ladrilhamentos de Penrose.}
	\end{figure}

Se os ladrilhos estiverem dispostas de modo a produzir um padrão de repetição regular, o mosaico será denominado periódico. Os ladrilhamentos do tipo periódicos repetem o ladrilho em duas direções e formam padrões com simetria. Porém, se o arranjo de ladrilhos produzir um padrão irregular ou aleatório, ele será denominado aperiódico. Esses ladrilhamentos não têm simetria de translação e o padrão não pode ser repetido periodicamente apenas cobrindo uma parte do plano.

Mas um dos questionamentos realizados nesse capítulo ainda não foi respondido: “Mas o que os lagartos tem em comum com a matemática?" . Nesse sentido, a atividade \ref{at_outros3} nos auxilia a responder essa pergunta.

Observe que para produzir o ladrilhamento parcial apresentado na figura  \ref{escher3}, usamos além de rotações, uma técnica conhecida como técnica da dentada (ou mordida).

Essa técnica consiste em retirar uma parte da parte interna de um ladrilho, a partir de um de seus lados, e fixá-la na parte externa do mesmo ladrilho, a partir de outro lado, produzindo-se assim um novo ladrilho.

Por exemplo, para produzir um lagarto, são retiradas várias partes de um hexágono e essas partes são fixadas na parte externa do próprio hexágono, como ilustra a figura \ref{lagarto_hex}.


 \begin{figure}[H]
	\centering
	\includegraphics[width=400bp]{lagarto_hex}
	\label{lagarto_hex}
\caption{Produzindo um lagarto.}
	\end{figure}

A mesma técnica pode ser observada no vídeo \href{ https://youtu.be/T6L6bE_bTMo}{Anatomia de um lagarto} ou no  aplicativo \href{https://www.geogebra.org/m/zs2ud4w5} {Lizard} .


\practice{Outros ladrilhamentos }


\begin{task}{Como criar um ladrilhamento usando translações}

\begin{enumerate}
	\item Desenhe um hexágono regular ( do encarte ) em um pedaço de papel, recorte o hexágono e cole-o em uma folha de papelão (ou papel de gramatura  alta) com o qual deseja construir o ladrilhamento.
	\item Desenhe dois triângulos equiláteros ( do encarte) em um pedaço de papel. Recorte os triângulos e cole-os na mesma folha de papelão (ou papel) para que sejam anexados aos lados do hexágono, como ilustra a figura \ref{transf1}.

	\begin{figure}[H]
	\centering
	\includegraphics[width=150bp]{ladrilhamento27}
	\label{transf1}
\caption{Arranjo de um hexágono e dois triângulos}
	\end{figure}

	\item Recorte a forma combinada. Trace a forma em uma nova folha de papel. Desloque-a para que o hexágono se encaixe no espaço formado pelos dois triângulos. Trace ao redor da forma deslocada e repita mais duas vezes, como ilustra a figura \ref{transf2}

	\begin{figure}[H]
	\centering
	\includegraphics[width=300bp]{ladrilhamento28}
	\label{transf2}
\caption{Arranjo de um hexágono e dois triângulos}
	\end{figure}
	
\item Continue repetindo a figura formado pelos dois triângulos equiláteros e o hexágono regular, para produzir um ladrilhamento.	
	
	\begin{enumerate}
		\item De que outras maneiras você pode deslocar a figura formado pelos dois triângulos equiláteros e o hexágono regular?
		
	\item Descreva como usou uma tranformação para criar os ladrilhamentos.

\end{enumerate}
\end{enumerate}
\end{task}

\begin{task}{Identificando as transformações}

\begin{enumerate}
	\item Quais polígonos e quais deslocamentos são usados para criar o ladrilhamento ilustrado na figura \ref{transf3}?
	
	\begin{figure}[H]
	\centering
	\includegraphics[width=200bp]{ladrilhamento29}
	\label{transf3}
	\caption{Ladrilhamento}
	\end{figure}
	
	\item Existe a possibilidade de ladrilhar o plano usando os mesmos polígonos,  mas obtendo um padrão diferente?
	\item Esse é um ladrilhamento semirregular? Justifique!
\end{enumerate}
\end{task}


\begin{task}{Como criar um ladrilhamento usando rotação}

\begin{enumerate}
	\item Desenhe um triângulo equilátero em um pedaço de papel. Recorte o triângulo e cole-o em uma folha de papelão (ou papel de gramatura alta).
	\item Use uma lápis para contornar o seu ladrilho em um pedaço de papel.
	\item Gire o ladrilho $60^{\circ}$ em torno de um vértice até que a lateral do ladrilho caia ao longo da borda do traçado anterior, como ilustra a figura \ref{transf4}. Trace em torno do lado a lado novamente

	\begin{figure}[H]
	\centering
	\includegraphics[width=200bp]{ladrilhamento37}
\label{transf4}
\caption{Triângulo}
	\end{figure}

	\item Repita o terceiro passo até que uma volta completa tenha sido realizada. Depois responda as questões:
	\begin{enumerate}
		\item Que forma você criou?
		\item Quantas vezes você teve que girar o ladrilho para criar essa forma?
		\item Como você pode continuar usando rotações para aumentar o ladrilhamento?
		\item Descreva como usar a rotação de polígonos para criar um ladrilhamento.
	\item Quais tipos de polígonos podem ser usados para fazer um ladrilhamento com rotações?
		
	\end{enumerate}

	
\end{enumerate}

\end{task}

\begin{task}{Ladrilho de 12 lados}

A figura \ref{12lados} ilustra um encaixe formado por figuras irregulares de doze lados. 

	\begin{figure}[H]
	\centering
	\includegraphics[width=300bp]{ladrilhamento32}
\label{12lados}
\caption{Ladrilho de 12 lados}
	\end{figure}
	
	\begin{enumerate}
		\item Explique porque a figura de 12 lados ladrilha o plano perfeitamente.
		\item Usando lápis e papel tente criar uma figura irregular de 10 lados que também cobre um plano.
		\item Explique como foi possível ladrilhar o plano com a figura irregular de 10 lados.
		\item Usando lápis e papel tente criar uma figura irregular de 6 lados que também cobre um plano.
		\item Explique como foi possível ladrilhar o plano com a figura irregular de 6 lados.
	\end{enumerate}
	
\end{task}





%%%%%%%%%%%%%%%%%%%%%%

\exercise

\begin{enumerate}

	\item Anita e Vitória estão tentando descobrir como esse ladrilhamento foi feito.

	\begin{figure}[H]
	\centering
	\includegraphics[width=250bp]{ladrilhamento30}

	\end{figure}

	\begin{itemize}
	\item Anita: O ladrilhamento é baseado em refletir os triângulos rosas ao através do dodecágono azul.
	\item Vitória: O ladrilhamento é baseado na translação do dodecágono azul com dois triângulos rosas.
	\end{itemize}

	De quem é a resposta correta? Explique.


	\item Priscila está projetando um azulejo de cozinha que usa dois polígonos regulares diferentes. Ela então usa dois deslocamentos diferentes para criar um ladrilhamento. Use papel quadriculado para projetar um bloco que Priscila poderia usar. Mostre como ladrilhar o plano.

	\item Alice quer fazer revestir a arede do banheiro usando os dois polígonos mostrados.

	\begin{figure}[H]
	\centering
	\includegraphics[width=200bp]{ladrilhamento34}

	\end{figure}

	\begin{enumerate}
		\item Ela será capaz de criar um ladrilhamento usando essas formas? Explique
\item Será que Alice consegue fazer um ladrilhamento usando pentágonos regulares e triângulos, não necessariamente regulares?
	\end{enumerate}
	
	
	\item Uma forma de determinar padrões com pentágonos irregulares é considerar os duais de padrões com polígonos regulares de tipos diferentes com a mesma configuração em todo o vértice, especialmente os que possuem cinco polígonos ao redor do vértice.

Ao considerar a configuração 3-3-3-3-6, encontramos um ladrilhamento com pentágonos irregulares.

\begin{figure}[H]
	\centering
	\includegraphics[width=200bp]{ladrilhamento54}

\end{figure}

\begin{itemize}


\item Será possível encontrar esse tipo de ladrilhamento em outras pavimentações com os polígonos regulares de tipos diferentes que possuem 5 polígonos ao redor do vértice? 	Experimente a configuração 3-3-4-3-4.

	\begin{figure}[H]
	\centering
	\includegraphics[width=200bp]{ladrilhamento55}

	\end{figure}

\item 	E sem usar o ladrilhamento dual, será que é possível determinar um ladrilhamento com pentágonos partindo do ladrilhamento composto por quadrados?

	\begin{figure}[H]
	\centering
	\includegraphics[width=250bp]{ladrilhamento56}

	\end{figure}

\end{itemize}

\item  A parede de um banheiro foi coberta por ladrilhos hexagonais não regulares:

	\begin{figure}[H]
	\centering
	\includegraphics[width=350bp]{ladrilhamento57}

	\end{figure}
	\begin{enumerate}
		\item Qual a condição necessária para que um hexágono qualquer ladrilhe o plano?
		\item Construa um hexágono irregular com lados opostos paralelos congruentes.
		\item Esse hexágono ladrilha o plano? Justifique sua resposta.
		\item O hexágono que você construiu é convexo? E se não for convexo, a resposta dada ao item b) continua válida?
		\item Qual transformação foi utilizada para construir o ladrilhamento?
	\end{enumerate}
	
	
\item 
Sabrina está desenhando um papel de parede padronizado por um ladrilhamento. Ela decidiu usar o formato de letra \textbf{T} como base. Crie dois ladrilhamentos distindos usando as três letras ilustradas na figura \ref{ladt}.

	\begin{figure}[H]
	\centering
	\includegraphics[width=250bp]{ladrilhamento33}
	\label{ladt}
	\caption{Ladrilhos \textbf{T}}
	\end{figure}

\item  Um pentaminó é uma forma composta por cinco quadrados congruentes conectados pelo menos por um lado, como ilustra a figura \ref{pentamino}

	\begin{figure}[H]
	\centering
	\includegraphics[width=300bp]{ladrilhamento12}
	\label{pentamino}

	\caption{Dois jogos de pentaminó. Fonte: \href{https://www.ibilce.unesp.br/departamentos/matematica/eventos/3-cejta/regra-dos-jogos/9-ano---pentamino/}{UNESP.}}
	\end{figure}
	
	\begin{enumerate}
		\item Escolha dois dos pentaminós e tente fazer um ladrilhamento com cada um deles.
		\item Os pentaminós escolhidos ladrilharam o plano?
		\item Explique por que sim ou por que não.
	\end{enumerate}


	
	
\end{enumerate}





