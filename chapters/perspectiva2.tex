\chapterillustration{abertura-perspectiva2}{abertura-perspectiva2-professor}

\chapterwhat{Representações em Matemática (semiótica): exemplos, spectos cognitivos e culturais; projeções em perspectiva: conceitualização via definição 3D, propriedades com justificativas, aplicações (pinturas, ilusões de ótica); projeções paralelas: conceitualização, propriedades e aplicações (planta baixa, mapa de fuga, vistas e ilustrações em áreas diversas).}

\chapterbecause{As projeções em perspectiva fornecem um modelo matemático que auxilia na compreensão de como vemos, comunicamos e interagimos com o mundo. Já as projeções paralelas fornecem uma representação mais simples e fácil de se entender e, assim, elas têm sido utilizadas para a confecção de ilustrações em várias áreas: Arquitetura, Engenharia, Biologia, etc. Além disso, no dia a dia, é importante, por exemplo, saber interpretar diagramas 2D de objetos 3D que descrevem como montar uma cama, colocar um cartucho em uma impressora, abrir a porta de emergência do aviã o, descobrir a saída de emergência mais próxima em um hotel, etc.}

\chapter{Vistas ortogonais e representações em perspectiva}
\label{\detokenize{GE301::doc}}\label{\detokenize{GE301:vistas-ortogonais-e-representacoes-em-perspectiva}}

\def\graphics{Figuras/perspectiva2}

%%%% Página de créditos

% Autores
\autorum{Humberto Bortolossi (UFF)}
\autordois{Lhaylla Crisaff (UFF)}
\autortres{José Ezequiel Soto Sánchez}
\autorquatro{Letícia Rangel}

\revisorum{Wanderley Rezende}

\graficos{\makecell[tl]{Humberto Bortolossi (UFF) \\ Tarso Caldas}}

% Revisores

\autordacapa{Raphaël LR}{Unsplash}{https://unsplash.com/photos/5KO0SEN0WRs}
\versao{1.0}

\versaodigital{https://www.umlivroaberto.org/BookCloud/Volume_1/master/view/GE301.html}

\ccbysa

\creditos


\mainmatter

\begin{apresentacao}{Introdução}

\subsection{Habilidades e pré-requisitos}
Neste capítulo contempla-se a seguinte habilidade da segunda versão da \href{http://historiadabncc.mec.gov.br/documentos/bncc-2versao.revista.pdf}{Base Nacional Comum Curricular} (BNCC):

\begin{habilities}{EM13MT01}
Estabelecer relações entre vistas ortogonais e representações em perspectiva de figuras geométricas espaciais e de objetos do mundo físico e aplicar esse conhecimento em situações relacionadas ao mundo do trabalho.
\end{habilities}

Os pré-requisitos para este capítulo são:

\paragraph{Ensino Fundamental}
\begin{habilities}{EF02MA14}
Reconhecer, nomear e comparar figuras geométricas espaciais (cubo, bloco retangular, pirâmide, cone, cilindro e esfera), relacionando-as com objetos do mundo físico.

\tcbsubtitle{EF02MA15}
Reconhecer, comparar e nomear figuras planas (círculo, quadrado, retângulo e triângulo), por meio de características comuns, em desenhos apresentados em diferentes disposições ou em sólidos geométricos.

\tcbsubtitle{EF03MA13}
Associar figuras geométricas espaciais (cubo, bloco retangular, pirâmide, cone, cilindro e esfera) a objetos do mundo físico e nomear essas figuras.

\tcbsubtitle{EF06MA17}
Reconhecer, nomear e comparar polígonos, considerando lados, vértices e ângulos, e classificá-los em regulares e não regulares, tanto em suas representações no plano como em faces de poliedros.

\tcbsubtitle{EF06MA26}
Interpretar, descrever e desenhar plantas baixas simples de residências e vistas aéreas.

\tcbsubtitle{EF09MA12}
Reconhecer as condições necessárias e suficientes para que dois triângulos sejam semelhantes.

\tcbsubtitle{EF09MA16}
Reconhecer vistas ortogonais de figuras espaciais e aplicar esse conhecimento para desenhar objetos em perspectiva.
\end{habilities}


\paragraph{Ensino Médio}

\begin{habilities}{EM11MT06}
Compreender função como uma relação de dependência entre duas variáveis, as ideias de domínio, contradomínio e imagem, e suas representações algébricas e gráficas e utilizá-las para analisar, interpretar e resolver problemas em contextos diversos, inclusive fenômenos naturais, sociais e de outras áreas.

\tcbsubtitle{EM12MT01}
Compreender o teorema de Tales e aplicá-lo em demonstrações e na resolução de problemas, incluindo a divisão de segmentos em partes proporcionais

\tcbsubtitle{EM12MT03}
Utilizar a noção de semelhança para compreender as razões trigonométricas no triângulo retângulo, suas relações em triângulos quaisquer e aplicá-las em situações como o cálculo de medidas inacessíveis, entre outras.
\end{habilities}

\subsection{Tópicos que serão tratados no capítulo}
\begin{enumerate}
\item {} 
Representações 2D de objetos 3D: conceitualização, exemplos, usos na comunicação humana.

\item {} 
Projeções em perspectiva: conceitualização via definição no espaço tridimensional, propriedades e aplicações.

\item {} 
Projeções paralelas: conceitualização via definição no espaço tridimensional, propriedades e aplicações.
\end{enumerate}

\subsection{Objetivo geral do capítulo}
Levar o aluno a compreender como representações 2D de objetos 3D obtidas por projeções em perspectiva e paralelas fornecem modelos matemáticos que auxiliam na compreensão de como  vemos, comunicamos e interagimos com o mundo e, também, por um processo metacognitivo sobre essas projeções, levá-lo a refletir sobre o universo dos signos e dos processos significativos na natureza e na cultura (semiótica).

\subsection{Abordagem adotada no capítulo}

O capítulo inicia com atividades que procuram levar o aluno a ganhar uma consciência das diversas maneiras de se representar objetos 3D por um desenho 2D. Para isto, ele é convidado inicialmente a fazer na \hyperref[\detokenize{GE301-0:ativ-proj-atelier-geometrico}]{atividade Atelier Geométrico} um desenho de um conjunto de objetos geométricos (\hyperref[\detokenize{GE301-0:fig-proj-solidos-01}]{Figura \ref{\detokenize{GE301-0:fig-proj-solidos-01}}}) o mais fielmente que ele conseguir. Com esta atividade, você também poderá avaliar como seus alunos estão desenhando.

A \DUrole{xref,std,std-ref}{ativ-proj-lobo}, por sua vez, tem o intuito de provocar o aluno e levá-lo a perceber que uma determinada representação é ou não adequada dependendo do contexto e dos objetivos de uso. De fato, procura-se levá-lo a perceber que, mais do que usos e propósitos, as representações variam de acordo com a idade, a cultura e o gênero.

\citet{Cohn-2012} observa que, em geral, as pessoas reclamam quando alguém não escreve direito, mas essas mesmas pessoas são mais condescendentes quando se refere a desenhos. Por que considerar o desenho de forma diferente da linguagem? \citeauthor{Cohn-2012} argumenta que o desenvolvimento do desenho é análogo ao da linguagem. De fato, é importante ter em mente (a) que, como qualquer outra habilidade humana, com prática, é possível aprender a desenhar \citep{Edwards-2005}; (b) que habilidades visuais constituem um dos tipos reconhecidos de inteligência humana \citep{Gray-et-al-2004,Garner-2011}; (c) que o desenvolvimento das habilidades espaciais desenvolvem outros tipos de habilidades \citep{Van-Meter-et-al-2005,Fan-2015,Sinclair-et-al-2016,Khine-2017}.

Uma representação não se encerra em si. É preciso considerar também a componente de quem interpreta a representação: nós! A \DUrole{xref,std,std-ref}{ativ-proj-interpretando} procura destacar esse aspecto interpretativo. Mais precisamente, a atividade busca levar o aluno a perceber que a interpretação de uma imagem passa pelo cérebro e que, por este motivo, relações matemáticas presentes na imagem, como congruência, podem não ser compreendidas como tal gerando as assim chamadas “ilusões”.

Todas estas questões de representações e significados fazem parte da semiótica, disciplina que se ocupa do estudo dos signos e dos processos significativos na natureza e na cultura. Com o intuito de estimular uma reflexão metacognitiva sobre o tema, o termo “semiótica” é então apresentado explicitamente para o aluno em uma caixa “Para refletir”, junto com exemplos e indicações de leituras complementares.

Esta primeira parte se encerra com indicações sobre o porquê estudar projeções em perspectiva e projeções paralelas e, também, com um alerta sobre as especificidades das representações obtidas por essas projeções: reconhecê-las é fundamental para entender e se fazer entender em termos de comunicação visual.

Para o estudo das projeções em perspectiva e projeções paralelas que se segue, a seguinte estratégia pedagógica foi adotada (“concreteness fading” conforme \cite{Fyfe-et-al-2014}):

\begin{figure}[H]
\centering

\noindent\includegraphics[width=\linewidth]{{estrategia-pedagogica-01_1}.jpg}
\end{figure}

Mais precisamente, cada tipo de projeção é motivado com um experimento concreto, um modelo matemático abstrato que represente o experimento é então estabelecido e suas propriedades determinadas e, de posse deste conhecimento, conexões e previsões são feitas para o modelo concreto inicial. Os dois tipos de projeções (em perspectiva e paralelas) são desenvolvidos concomitantemente, pelos seguintes motivos: (1) ganha-se tempo na realização das atividades; (2) estimula-se aluno a comparar as propriedades dois tipos de projeção.

Para os experimentos concretos, escolhemos atividades com luzes e sombras com um celular (para as projeções em perspectiva) e com a luz solar (para as projeções paralelas). Os motivos para tal escolha são compartilhados por Leonardo da Vinci (1452-1519).
\begin{quote}

\begin{figure}[H]
\centering

\noindent\includegraphics[width=.8\linewidth]{{leonardo-da-vinci-01}.jpg}
\end{figure}
\end{quote}

Nestes experimentos, os alunos são levados a observar e descrever o comportamento das sombras de alguns objetos geométricos familiares, identificando o que varia e o que não varia de acordo com a posição do objeto, do anteparo de projeção e da fonte de luz.

Ao contrário do que normalmente se faz em livros didáticos quando se trata de projeções em perspectiva e projeções paralelas, a modelagem matemática dos experimentos é feita utilizando-se um modelo 3D e não a representação 2D que dela decorre.
\begin{quote}

\begin{figure}[H]
\centering

\noindent\includegraphics[width=\linewidth]{{2017-12-09_19-09-03}.jpg}
\end{figure}
\end{quote}

Esta abordagem tem várias vantagens: (1) ela é a mais natural e próxima dos experimentos que se deseja modelar; (2) todas as propriedades das representações 2D obtidas pelas projeções podem ser deduzidas e evita-se o uso de regras sem explicações; (3) aplicações recentes com recursos tecnológicos usam o modelo 3D e não especificações na representação plana; (4) ela se relaciona com episódios históricos que habitualmente não são apresentados; (5) ela oferece um excelente cenário de interesse onde se é possível exercitar geometria de posição e aplicar semelhança de triângulos.  As projeções em perspectiva, em particular, constituem uma excelente oportunidade do aluno apreciar uma característica importante da Matemática: o de identificar uma mesma estrutura (projeções em perspectiva) em fenômenos diferentes (sombras, pinturas, câmeras, modelos ópticos para o olho humano).

Aqui, talvez o maior desafio seja justamente o de explicar algo 3D tendo como principal instrumento uma mídia 2D (os desenhos estáticos nas páginas do livro didático). Neste sentido, usar materiais concretos e recursos tecnológicos ajuda bastante e, por este motivo, todos os diagramas de configurações tridimensionais são acompanhados de construções interativas feitas no GeoGebra. Cada cena 3D pode ser girada, ampliada ou reduzida e, em muitos casos, parâmetros da construção podem ser alterados. Esta possibilidade de movimentação constitui-se em um aspecto cognitivo importante \citep{Sinha-2009} que evita os equívocos de interpretação gerados pelas distorções das projeções. A propósito, o GeoGebra também foi usado para se criar os diagramas deste capítulo. Desta maneira, tem-se a garantia de que as ilustrações estão matematicamente consistentes. Além disso, essas construções estão disponíveis para que você faça modificações e derivações que ache necessárias. Estas construções do GeoGebra podem ser acessadas de computadores \textit{desktop}, \textit{tablets} e \textit{smartphones}.

As dificuldades advêm principalmente de dois fatores: por um lado, as projeções em perspectiva e paralelas são ambiguas (isto é, não injetivas) e não preservam comprimentos, ângulos, proporções, áreas, etc., de modo que as medidas na representação 2D podem não corresponder às medidas do objeto original 3D; por outro lado, não existe a cultura de se praticar a produção de desenhos, de modo que, em geral, quando os alunos são levados a fazer alguma representação 2D de objetos 3D, os desenhos produzidos são algumas vezes ingênuos.

Um exemplo de distrator típico é apresentado por \citet{Lellis-2009}: na figura a seguir, é comum um aluno desavisado pensar que entre \(A\), \(B\) e \(C\), é o ponto \(C\) que está mais próximo da reta \(r\) na configuração 3D (afinal, na projeção paralela, é o que acontece).

\begin{figure}[H]
\centering

\noindent\includegraphics[width=200bp]{{2017-12-10_18-14-52}.jpg}
\end{figure}

Dependendo do ponto de vista, retas que são reversas são projetadas em retas concorrentes, o que também costuma confudir os alunos.

\begin{figure}[H]
\centering

\includegraphics[width=.475\linewidth]{{2017-12-10_21-05-40}.jpg}\hspace{.5cm}
\includegraphics[width=.475\linewidth]{{2017-12-10_21-21-23}.jpg}
\end{figure}

Outros dois exemplos são dados por \citet{Volkert-2008}: na primeira figura, a poligonal ligando um vértice do cubo ao ponto médio da aresta pode ser interpretada de várias maneiras diferentes; na segunda figura, os ângulos retos da configuração 3D podem, ao mesmo tempo, na representação 2D, ser desenhados como um ângulo agudo e um ângulo obtuso.

\begin{figure}[H]
\centering

\noindent\includegraphics[width=200bp]{{ambiguidade-02}.jpg}
\end{figure}

Com relação à questão de ângulos, \citet{Fujita-et-al-2017} relatam o equívoco de alunos japoneses acharem que, na figura a seguir, o ângulo \(MFN\) ser reto, o que não é o caso.

\begin{figure}[H]
\centering

\noindent\includegraphics[width=120bp]{{perspectiva-angulo-01}.jpg}
\end{figure}

Os alunos serão confrontados com estes distratores na \DUrole{xref,std,std-ref}{ativ-proj-distratores}. Espera-se que, após a realização de todas as atividades anteriores, o aluno esteja mais bem preparado ao fazê-lo.

No estudo quantitativo dos comprimentos das projeções, optou-se por uma abordagem funcional (\hyperref[\detokenize{GE301-5:ativ-proj-comprimentos}]{atividade Comprimentos em projeções}). O uso do conceito de função nesta parte não é casual e vai além do propósito de uma mera conexão entre Geometria e Álgebra. A notação funcional permite, por exemplo, compactar informação e, com isto, articular melhor o pensamento.

As aplicações dos modelos matemáticos ao longo do capítulo se dão principalmente com relação a pinturas, mas as articulações e aplicações são muitas e variadas. No final do capítulo apresentamos uma lista de referências e sugestões de projetos que incluem temas como projeção mapeada, técnicas de cinema, jogos de computador e de tabuleiro, teatro, história das Artes, etc.

\paragraph{Sobre questões de adequação de tempo}

O assunto de projeções é muito rico. De fato, em nossa opinião, ele é um tópico chave, pois permite estabelecer conexões legítimas com várias áreas dentro e fora da Matemática e, também, permite desenvolver várias competências estabelecidas pela BNCC. Neste sentido, o capítulo foi redigido com a premissa de oferecer ao aluno e a você, professor, uma linha narrativa gradual, com vários desdobramentos, conexões e referências.

Contudo, caso haja alguma limitação no número de aulas necessárias para a condução de todas as atividades propostas neste capítulo, alguns cortes podem ser realizados em detrimento da consistência da narrativa e, também, em função da escolha do nível de profundidade que se deseja tratar o assunto. Ao invés de oferecer um material enxuto, preferimos manter todas as conexões e deixar para você, professor, a escolha consciente de um percurso no material do capítulo que equilibre o tempo disponível e o tratamento do assunto de representações e projeções.

\paragraph{Observações}
\begin{itemize}
\item {} 
Existem outras nomeclaturas para o que estamos denominando de “projeções em perspectiva” e “projeções paralelas”. Alguns textos mais antigos, usam, por exemplo, “perspectiva central”, “perspectiva ortogonal” de modo que, neste caso, perspectiva fica como sinônimo de projeção. Outros usam a palavra representação: “representação em perspectiva”, como ocorre no enunciado da habilidade na BNCC. Em textos mais recentes, principalmente os de computação gráfica, os termos mais usados são “projeção em perspectiva” e “projeção ortogonal” (aparece também o termo “projeção central”).

\item {} 
Em particular, é preciso ter atenção para o uso da palavra \index{vista}vista. Alguns livros, por exemplo, pedem para o aluno reconhecer a \textit{vista} do cubo a partir da direção dada pela seta azul em (A) na \hyperref[\detokenize{GE301-0:fig-proj-vistas-01}]{Figura \ref{\detokenize{GE301-0:fig-proj-vistas-01}}} e esperam como resposta a imagem (B), ou seja, uma projeção ortogonal do cubo vazado. Contudo, (B) \textit{não é o que se é visto} a partir da direção indicada. O que se vê é o resultado de uma projeção em perspectiva, a saber, a imagem (C).

\begin{figure}[H]
\centering
\capstart

\noindent\includegraphics[width=.9\linewidth]{{vistas-01_1}.jpg}
\caption{O que é uma \textit{vista}?}\label{\detokenize{GE301-0:fig-proj-vistas-01}}\label{\detokenize{GE301-0:id24}}
\end{figure}

\item {} 
Existem vários modelos matemáticos que tentam capturar como “vemos” \citep{Lindberg-1976,Howard-et-al-1995}. Para este capítulo, usaremos o modelo simples dado por projeções em perspectiva.

\end{itemize}

\paragraph{Leituras complementares}
\begin{itemize}
\item {} 
Caso o tópico de matrizes esteja na grade curricular de sua escola, recomendamos a leitura do caderno explicativo “Perspectiva e Anamorfismo” de Gladson Antunes, Leonardo Silvares e Michel Cambrainha, disponível \href{https://goo.gl/eQzPWH}{no link}, que oferece uma abordagem matricial das projeções em perspectiva.

\item {} 
Recomendamos a leitura do livro “Olhar, Saber, Representar: Sobre A Representação em Perspectiva” de Cláudia Flores (publicado em 2007 pela Musa Editora) que traz, por meio da análise de cerca de 70 obras do Renascimento, uma abordagem da história da arte em termos do nosso modo de olhar e da representação em perspectiva. Outra referência sobre este tema é o excelente livro “The Science of Art: Optical Themes in Western Art from Brunelleschi To Seurat” de Martin Kemp (publicado em 1992 pela Yale University Press).

\end{itemize}

\end{apresentacao}

\def\currentcolor{session1}

\begin{objectives}{Atelier geométrico}
{
\begin{itemize}
\item {} 
Para o aluno: criar desenhos próprios com os quais será possível, após a realização desta e da próxima atividade e sob a condução do professor, refletir sobre representações de objetos 3D no plano.

\item {} 
Para o professor: realizar um diagnóstico da turma no que se refere às habilidades de representação por meio de desenhos.

\end{itemize}
}{1}{1}
\end{objectives}
\begin{sugestions}{Atelier geométrico}
{
\begin{itemize}
\item {} 
Sugerimos que você use os seguintes sólidos geométricos: um cubo (por ser um objeto 3D matemático familiar ao contexto escolar desde as séries iniciais), um cilindro circular reto (por conta das bases circulares paralelas as quais, em projeções em perspectiva, não são simultaneamente visíveis) e uma esfera (ou um cone).

\begin{figure}[H]
\centering
\capstart

\noindent\includegraphics[width=200bp]{{fig-proj-solidos-01}.jpg}
\caption{Exemplo de um conjunto de sólidos.}\label{\detokenize{GE301-0:fig-proj-solidos-01a}}\label{\detokenize{GE301-0:id26}}\end{figure}


É importante que estes sólidos estejam dispostos de modo que pelo menos um fique parcialmente escondido atrás de outro, pois esta característica será verificada na produção dos alunos.

Não recomendamos o uso de modelos vazados (feitos de canudinhos, por exemplo) ou transparentes, pois estes tornam a cena mais complexa e difícil de se representar.

Certifique-se que todos os alunos consigam ver adequadamente os sólidos. Se estes foram muito pequenos e sua turma for numerosa, talvez seja adequado usar mais de um conjunto de sólidos em mais mesas, separando os alunos em torno delas.

\item {} 
Deixe seus alunos trabalharem livremente. Caso algum deles pergunte se seu desenho está ficando “bom” ou “correto”, comente que isto será discutido em grupo ao término da próxima atividade.
\end{itemize}
}{1}{1}
\end{sugestions}
\begin{sugestions}{Atelier geométrico}
{
\begin{itemize}
\item {} 
Durante a execução da atividade, circule entre os alunos e observe seus desenhos. É importante que, nesta etapa, você já diagnostique as habilidades de representação deles para a discussão que será feita em seguida.

Observe, por exemplo, se as posições relativas dos sólidos foram desenhadas corretamente, um atributo que, segundo \citep{Ebersbach-et-al-2011} e \citep{Willats-1977}, exige maturidade e flexibilidade cognitivas as quais normalmente se desenvolvem por volta dos 11 anos. Antes dessa idade, é comum os alunos desenharem os objetos dispostos separadamente, um ao lado do outro, mesmo quando, na visualização da cena, existe um objeto que está na frente de outro.

Com relação a desenhos de um cubo, \citep{Cox-et-al-1998} propuseram uma escala de aferição da “maturidade” da representação, a qual pode lhe ser útil.

Para o caso de um cilindro circular reto, \citep{Mitchelmore-1978} propõe a evolução em estágios descrita na \hyperref[\detokenize{GE301-0:fig-proj-escala-mitchelmore}]{Figura \ref{\detokenize{GE301-0:fig-proj-escala-mitchelmore}}}.

\begin{figure}[H]
\centering
\capstart

\noindent\includegraphics[width=250bp]{{escala_Page_2}.jpg}
\caption{Escala de aferição da “maturidade’ da representação do cubo.}\label{\detokenize{GE301-0:id27}}
\end{figure}

\begin{figure}[H]
\centering

\noindent\includegraphics[width=.8\linewidth]{{escala_Page_1}.jpg}
\label{\detokenize{GE301-0:fig-proj-escala-cox}}\end{figure}





\item {} 
Caso algum aluno já tenha terminado esta atividade, você pode sugerir que ele já trabalhe na próxima.
\end{itemize}
}{0}{9}
\end{sugestions}

\clearmargin
\begin{objectives}{É o Lobo!}
{
Refletir sobre representações de objetos 3D no plano, no caso, representações de um lobo.
}{1}{2}
\end{objectives}
\begin{sugestions}{É o Lobo!}
{
\begin{itemize}
\item {} 
Sugerimos que você inicie uma sistematização com a atividade das representações do lobo. Peça para que os alunos manifestem suas respostas e justificativas. Caso não apareçam naturalmente, apresente os argumentos e as ponderações do “Organizando as ideias” a seguir.

\item {} 
Passe então para a primeira atividade. Deixe os desenhos que foram feitos pelos alunos com os próprios alunos (você pode recolhê-los após a sistematização). Aqui, sugerimos fortemente que se apresente para os alunos o fato de que a representação muda com a idade. Desenhe no quadro algumas das imagens da \hyperref[\detokenize{GE301-0:fig-proj-escala-cox}]{Figura \ref{\detokenize{GE301-0:fig-proj-escala-cox}}} ou da \hyperref[\detokenize{GE301-0:fig-proj-escala-mitchelmore}]{Figura \ref{\detokenize{GE301-0:fig-proj-escala-mitchelmore}}}, comente sobre o “realismo intelectual” vs. “realismo visual” e a questão da “memória de trabalho”.

\item {} 
É importante que, no final da sistematização e do “Organizando as ideias” a seguir, o aluno perceba que existem representações diferentes com usos e qualidades próprias e específicas pois, afinal, duas destas representações (projeções em perspectivas e projeções paralelas, temas deste capítulo) serão abordadas nas seções seguintes.

\item {} 
Um outro estudo muito interessante que mostra como os aspectos sócio-culturais podem influenciar a maneira de como se desenha é o apresentado pela Revista Quartz (\url{https://goo.gl/ry3uqV}) para o caso de círculos e triângulos.

\begin{figure}[H]
\centering

\noindent\includegraphics[width=.5\linewidth]{{aspectos-culturais-02}.jpg}
\end{figure}

\end{itemize}
}{1}{2}
\end{sugestions}



\explore{Representando o Que Vemos}
\label{\detokenize{GE301-0::doc}}\label{\detokenize{GE301-0:explorando-representando-o-que-vemos}}
Desde a pré-história, o ser humano tem registrado em pinturas o que ele vê no mundo que o cerca. Na \hyperref[\detokenize{GE301-0:fig-proj-pintura-01}]{Figura \ref{\detokenize{GE301-0:fig-proj-pintura-01}}}, por exemplo, temos, em (a), um desenho de leões e bisões na Caverna de Chauvet na França (com cerca de 30000 anos de idade) e, em (b), uma pintura rupestre no Parque Nacional Serra da Capivara no Piauí (com cerca de 11000 anos de idade).

\begin{figure}[H]
\centering
\capstart

\noindent\includegraphics[width=\linewidth]{{fig-proj-pintura-01}.jpg}
\caption{Pinturas pré-históricas.}\label{\detokenize{GE301-0:fig-proj-pintura-01}}\label{\detokenize{GE301-0:id25}}\end{figure}

Ao longo da história, seja em paredes, páginas de livros, telas de pintura ou telas de computador, surgiram diversas formas de se representar os objetos tridimensionais que estão em nossa volta. Neste capítulo, estudaremos duas destas formas de representação, importantes por suas aplicações. Para que você possa entender melhor o contexto, iniciaremos com atividades cujo objetivo é levar você a ver como as pessoas representam o que veem e como nossos cérebros interpretam essas representações.


\phantomsection\label{\detokenize{GE301-0:ativ-proj-atelier-geometrico}}
\begin{task}{Atelier geométrico}

Seu professor irá dispor um conjunto de objetos geométricos sobre uma mesa e o objetivo desta tarefa é que você desenhe em uma folha de papel \textbf{o que você vê nesta cena} o mais fielmente que conseguir.

\begin{figure}[H]
\centering

\noindent\includegraphics[width=300bp]{{atelier}.jpg}
\label{\detokenize{GE301-0:fig-proj-solidos-01}}
\end{figure}

\end{task}

\phantomsection\label{\detokenize{GE301-0:ativ-proj-lobo}}
\begin{task}{É o Lobo!}

Na sua opinião, qual das seis imagens (A), (B), (C), (D), (E) e (F) a seguir melhor representa um lobo? Por quê?

\begin{figure}[H]
\centering
\capstart

\noindent\includegraphics[width=350bp]{{lobo_1}.jpg}
\caption{Seis representações de um lobo.}\label{\detokenize{GE301-0:fig-proj-lobo}}\label{\detokenize{GE301-0:id28}}\end{figure}

\end{task}

\arrange{Tudo É uma Questão de Comunicação!}
\label{\detokenize{GE301-0:organizando-as-ideias-tudo-e-uma-questao-de-comunicacao}}
Em um primeiro momento, você pode achar que a fotografia (A) na \hyperref[\detokenize{GE301-0:fig-proj-lobo}]{Figura \ref{\detokenize{GE301-0:fig-proj-lobo}}} é a “melhor” representação de um lobo. Mas, pense um pouco: “melhor” em que sentido? O “melhor” sempre pressupõe um critério e, por conseguinte, um contexto.

Por exemplo, caso você queira fazer menção a um lobo em uma mensagem de texto enviada por SMS, então certamente a representação (F) é a mais adequada. Agora, imagine que você está escrevendo um livro de Biologia e sua editora lhe disse que, por razões orçamentárias, apenas figuras em “preto e branco” serão aceitas. Neste caso, as representações (B) e (C) parecem ser a melhor opção. E se você estivesse ilustrando um livro infantil? Aí, as representações (D) e (E) poderiam dar um tom artístico mais pessoal ao livro.

A representação (E) pode parecer muito tosca e infantil, mas lembramos aqui uma frase célebre do pintor Pablo Picasso (1881-1973):  “Levei quatro anos para aprender a pintar como Rafael, mas levei a vida toda para aprender a desenhar como uma criança.”.

\begin{figure}[H]
\centering

\includegraphics[width=320bp]{{picasso}.jpg}
\caption{Os touros de Pablo Picasso.}
\label{\detokenize{GE301-0:fig-proj-picasso}}
\label{\detokenize{GE301-0:id29}}
\end{figure}


Do mesmo modo que um lobo pode ser representado de maneiras diferentes, existem diversas representações para os objetos geométricos tradicionais em Matemática (cubos, cilindros, esferas, pirâmides, etc.). Mais ainda, estudiosos descobriram que a forma de representar muda com a idade de uma pessoa.
O filósofo Georges Henri Luquet explica, por exemplo, que o desenho do cilindro do Estágio 2 na \hyperref[\detokenize{GE301-0:fig-proj-escala-mitchelmore}]{Figura \ref{\detokenize{GE301-0:fig-proj-escala-mitchelmore}}} deve-se a uma preponderância de um “realismo intelectual” em relação a um “realismo visual”: a pessoa sabe que um cilindro circular reto têm duas bases circulares e pensa, nesta etapa, que se não registrar estas estas duas bases circulares, o desenho estaria incompleto. Assim, esta pessoa está registrando o que pensa, não o que vê.

\begin{figure}[H]
\centering
\capstart

\noindent\includegraphics[width=400bp]{{cilindros}.jpg}
\caption{Representação de um cilindro em estágios etários diferentes.}\label{\detokenize{GE301-0:fig-proj-escala-mitchelmore}}\label{\detokenize{GE301-0:id30}}\end{figure}

O psicólogo Sergio Morra, por sua vez, argumenta que a complexidade das regras ou estratégias de organização espacial que uma pessoa consegue dominar está restrita pela quantidade de informação que ela pode assimilar e processar simultaneamente, ou seja, pela memória de trabalho. Assim, os desenhos podem ficar “mais realistas” a medida que a memória de trabalho da pessoa aumenta com a idade.

Outro aspecto interessante é que o meio cultural pode influenciar a maneira como uma pessoa representa objetos tridimensionais, como aponta o estudo de Gutierrez (1998). A \hyperref[\detokenize{GE301-0:fig-proj-aspectos-culturais-01}]{Figura \ref{\detokenize{GE301-0:fig-proj-aspectos-culturais-01}}}), por exemplo, mostra como filhos de tecelões, oleiros e fazendeiros de povoados isolados na Índia, entre 8 e 12 anos de idade, com pouca ou nenhuma escolaridade, desenheram cilindros e pirâmides que lhe foram apresentados.

\begin{figure}[H]
\centering
\capstart

\noindent\includegraphics[width=230bp]{{aspectos_culturais}.jpg}
\caption{Influência de fatores culturais na produção de desenhos em perspectiva (Gutierres, 1998)}\label{\detokenize{GE301-0:fig-proj-aspectos-culturais-01}}\label{\detokenize{GE301-0:id31}}\end{figure}

Muitos acham que a habilidade de desenhar é um dom que, quem não tem, nunca irá desenhar bem. Neurocientistas têm mostrado \textbf{que este não é o caso}! De fato, estudos científicos mostram (a) que, como qualquer outra habilidade humana, com prática e dedicação, é possível aprender a desenhar; (b) que habilidades visuais constituem um dos tipos reconhecidos de inteligência humana; (c) que o desenvolvimento das habilidades espaciais desenvolvem outros tipos de habilidades.

Ainda no contexto de objetos geométricos matemáticos, para você ter uma ideia da multiplicidade de representações, considere o problema de representar no plano o globo terrestre modelado como uma esfera. Essas representações nada mais são do que os \index{mapas cartográficos}mapas cartográficos da Geografia! Existem muitos deles, cada um com propriedades e usos específicos! A escolha do mapa depende do que se quer comunicar!

\begin{figure}[H]
\centering
\capstart

\noindent\includegraphics[width=350bp]{{mapas_1}.jpg}
\caption{Mapas cartográficos são representações no plano do globo terrestre modelado como uma superfície esférica.}\label{\detokenize{GE301-0:fig-proj-mapas-cartograficos}}\label{\detokenize{GE301-0:id32}}\end{figure}

Um ponto muito importante para o que se seguirá é ter em mente que, apesar de podermos representar o que vemos de formas diferentes com usos diferentes, certas representações são construídas de maneira bem específicas e, portanto, possuem propriedades que lhe são próprias. Reconhecer, compreender e empregar corretamente estas propriedades são habilidades fundamentais para você se comunicar adequadamente em termos visuais! Este será exatamente o caso das duas representações 2D de objetos 3D obtidas por projeções em perspectivas e projeções paralelas, temas deste capítulo!

A seguinte analogia entre desenho e escrita, inspirada no livro \emph{Desenho e Escrita como Sistemas de Representação} de Analice Dutra Pillar, pode lhe ajudar a perceber a importância de se dar atenção às características específicas de uma determinada representação. Você se comunica por escrito via WhatsApp e, também, ao fazer uma redação no ENEM. No WhatsApp, pela agilidade que é característica deste meio de comunicação, você usa abreviações: “tdb” (tudo bem), “pdc” (pode crer), “obg” (obrigado), etc. Mesmo com abreviações, as pessoas se entendem. Por outro lado, em uma redação do ENEM, exige-se que o texto seja escrito seguindo características específicas, a saber, “de acordo com a modalidade escrita formal da língua portuguesa”: você deve respeitar as regras ortográficas e gramaticais. Analogamente, existem várias maneiras de se desenhar um cubo. Contudo, os desenhos obtidos por projeções em perspectiva e projeções paralelas possuem propriedades específicas. São essas propriedades e suas aplicações que vamos estudar neste capítulo!

\begin{knowledge}

O matemático alemão Johann Carl Friedrich Gauss (1777-1855) demonstrou um teorema, o chamado \emph{egregium}, a partir do qual é possível deduzir o seguinte resultado: qualquer representação plana que se faça de um globo terrestre modelado como uma esfera \textbf{sempre} terá algum tipo de distorção, isto é, ela não preservará ângulos ou não preservará áreas ou não preservará distâncias. Na página web \textless{}\url{https://goo.gl/HbLnPW}\textgreater{}, você encontrará um aplicativo que permite visualizar essas distorções para diferentes mapas cartográficos: as curvas fechadas mais espessas (círculos no exemplo da figura a seguir) são, no mapa, as representações de círculos de mesmo raio desenhados sobre a superfície esférica do globo terrestre. A partir da comparação dos formatos relativos dessas curvas (a \index{indicatriz de Tissot}indicatriz de Tissot) é possível ter uma ideia das distorções presentes no mapa.


\begin{figure}[H]
\centering

\noindent\includegraphics[width=50bp]{{egregium-qrcode}.png}
\end{figure}

\begin{figure}[H]
\centering

\noindent\includegraphics[width=280bp]{{egregium_1}.jpg}
\end{figure}


Existem mapas que preservam um ou outro atributo geométrico. O mapa de Mercator, por exemplo, preserva ângulos (mas não preserva áreas) e possui uma característica adicional útil para a navegação: as curvas de rumo constante sobre a superfície terrestre são representadas por retas neste mapa.
\end{knowledge}

\cleardoublepage
\def\currentcolor{session1}
\begin{objectives}{Será que é?}
{
Perceber que a interpretação de uma imagem passa pelo cérebro e que, por este motivo, relações matemáticas presentes na imagem, como congruência, podem não ser compreendidas como tal.
}{1}{1}
\end{objectives}
\begin{sugestions}{Será que é?}
{
\begin{itemize}
\item {} 
Os dois exemplos a seguir mostram que a parte visual do nosso cérebro pode não reconhecer movimentos de translação e rotação como isometrias.

\item {} 
Para o segundo item do exercício, sugerimos o uso da construção GeoGebra disponível no endereço \textless{}\url{https://www.geogebra.org/m/BNCePM5C}\textgreater{}, com a qual é possível visualizar dinamicamente que os três carros são congruentes por meio de um carro extra e de um contorno que podem ser movidos na construção.

\begin{figure}[H]
\centering

\noindent\includegraphics[width=50bp]{{ponzo-illusion-06}.png}
\end{figure}

Para o segundo item do exercício, sugerimos o uso da construção GeoGebra disponível no endereço \textless{}\href{https://www.geogebra.org/m/mFSV2Mp6}{https://www.geogebra.org/m/mFSV2Mp6W}\textgreater{}, com a qual é possível visualizar dinamicamente que os paralelogramos que são as tampas das mesas são congruentes por meio de um paralelogramo congruente intermediário.

\begin{figure}[H]
\centering

\noindent\includegraphics[width=50bp]{{mesa-de-shepard-03}.png}
\end{figure}

\item {} 
Caso haja interesse da turma, o tópico de ilusões visuais pode ser aprofundadado, por exemplo, por meio de um projeto. Nesta linha, o livro {[}Shapiro-et-al-2017{]} traz um compêndio atual no contexto da Psicologia. Ilusões visuais são mais do que fatos curiosos, como mostra o livro “Fisiologia Aeroespacial: Conhecimentos Essenciais para Voar com Segurança” de Thais Russomano e João de Carvalho Castro (2012), para a área de aviação.
\end{itemize}
}{1}{1}
\end{sugestions}
\begin{answer}{Será que é?}
{
\begin{enumerate}
\item {} 
Tipicamente, as pessoas quando veem a \hyperref[\detokenize{GE301-1:fig-proj-ponzo}]{Figura \ref{\detokenize{GE301-1:fig-proj-ponzo}}} acham que os carros têm tamanhos diferentes, sendo o carro mais acima na rua considerado o maior. Contudo, na imagem, os três carros têm o mesmo tamanho! Este tipo de viés de interpretação foi primeiro demonstrado pelo psicólogo italiano Mario Ponzo (1882-1960). Ele sugeriu que a mente humana usa o que está em torno de um objeto para julgar o seu tamanho.

\end{enumerate}
\begin{enumerate}
\item {} 
Tipicamente, as pessoas quando veem a \hyperref[\detokenize{GE301-1:fig-proj-shepard}]{Figura \ref{\detokenize{GE301-1:fig-proj-shepard}}}, acham que a mesa à esquerda é a mais comprida. Contudo, na imagem, as duas mesas têm as mesmas medidas! De fato, os paralelogramos que são as tampas das mesas são congruentes! Este tipo de viés de interpretação foi primeiro publicado pelo cientista cognitivo Roger Newland Shepard (1929-) em seu livro \textit{Mind Sights} de 1990.

\end{enumerate}
}{0}
\end{answer}

\explore{Interpretando o Que Vemos}
\label{\detokenize{GE301-1::doc}}\label{\detokenize{GE301-1:explorando-interpretando-o-que-vemos}}\phantomsection\label{\detokenize{GE301-1:ativ-proj-interpretando}}
\begin{task}{Será que é?}

\begin{enumerate}
\item {} 
(Ponzo) Observe a \hyperref[\detokenize{GE301-1:fig-proj-ponzo}]{Figura \ref{\detokenize{GE301-1:fig-proj-ponzo}}}. Qual carro é maior na imagem?

\begin{figure}[H]
\centering
\capstart

\noindent\includegraphics[width=350bp]{{ponzo-illusion-04}.jpg}
\caption{Qual carro é maior na imagem?}\label{\detokenize{GE301-1:fig-proj-ponzo}}\label{\detokenize{GE301-1:id14}}\end{figure}

\item {} 
(Shepard) Observe a \hyperref[\detokenize{GE301-1:fig-proj-shepard}]{Figura \ref{\detokenize{GE301-1:fig-proj-shepard}}}. Qual mesa é mais comprida na imagem?

\begin{figure}[H]
\centering
\capstart

\noindent\includegraphics[width=350bp]{{mesa-de-shepard}.jpg}
\caption{Qual mesa é mais comprida na imagem?}\label{\detokenize{GE301-1:fig-proj-shepard}}\label{\detokenize{GE301-1:id15}}\end{figure}

\end{enumerate}
\end{task}


\arrange{Ver É Uma Atividade Complexa!}
\label{\detokenize{GE301-1:organizando-as-ideias-ver-e-uma-atividade-complexa}}
Os dois exemplos apresentados na atividade anterior mostram que o ato de ver e compreender uma imagem não se encerra na própria imagem, mas depende da maneira que nosso cérebro processa toda a informação e se ajusta ao estímulo visual.

Psicólogos têm mapeado outras situações onde nosso cérebro faz adequações visuais subjetivas ao contexto: forma, cor, iluminação, distância, localização e movimento. Mais ainda: não só o sistema visual é afetado por ilusões, os demais sentidos também o são. Um exemplo clássico é o Efeito McGurk que mostra \textbf{como o que você vê altera o modo como você ouve}! Experimente você mesmo por meio do \href{https://goo.gl/k241EQ}{vídeo} no YouTube.

\begin{figure}[H]
\centering

\noindent\includegraphics[width=50bp]{{efeito-mcgurk-01}.png}
\end{figure}

O fato de nosso cérebro estar sucetível a estes tipos de ilusões pode parecer um defeito a princípio mas, como mostra o cientista cognitivo Donald Hoffman nesta  \href{https://goo.gl/x5H5oa}{palestra TED}, isto é resultado de um processo evolutivo que garantiu a nossa sobrevivência.


\begin{figure}[H]
\centering

\noindent\includegraphics[width=50bp]{{ted-realidade-02}.png}
\end{figure}



\begin{figure}[H]
\centering
\capstart

\noindent\includegraphics[width=350bp]{{ted-realidade-01}.jpg}
\caption{\href{https://goo.gl/x5H5oa}{Link para o vídeo}}\label{\detokenize{GE301-1:id16}}\end{figure}

Outro aspecto da interpretação de representações 2D de objetos 3D se refere à questão de ambiguidade: um mesmo desenho plano pode ser a representação de objetos tridimensionais diferentes. Considere, por exemplo, a Imagem (A) na \hyperref[\detokenize{GE301-1:fig-proj-ambiguidade-01}]{Figura \ref{\detokenize{GE301-1:fig-proj-ambiguidade-01}}}. Ela pode ser a representação de um cubo visto de cima como na Imagem (B) ou de um cubo visto de baixo como na Imagem (C).

\begin{figure}[H]
\centering
\setlength{\columnsep}{0pt}

\begin{multicols}{3}
\begin{figure}[H]
\raggedleft
\begin{asy}
size(3cm);
currentprojection=orthographic(3,1,.5);

triple a = (0,0,0);
triple b = (1,0,0);
triple c = (1,1,0);
triple d = (0,1,0);

triple e = (0,0,1);
triple f = (1,0,1);
triple g = (1,1,1);
triple h = (0,1,1);

draw (a -- b -- c -- d -- cycle);
draw (e -- f -- g -- h -- cycle);
draw (a -- b -- f -- e -- cycle);
draw (c -- d -- h -- g -- cycle);
\end{asy}
\centering

(A)
\end{figure}

\begin{figure}[H]
\centering
\begin{asy}
size(3cm);
currentprojection=orthographic(3,1,.5);

triple a = (0,0,0);
triple b = (1,0,0);
triple c = (1,1,0);
triple d = (0,1,0);

triple e = (0,0,1);
triple f = (1,0,1);
triple g = (1,1,1);
triple h = (0,1,1);

draw(e -- f -- g -- h -- cycle);
draw(c -- d -- h -- g -- cycle);
draw(b -- c -- g -- f -- cycle);

draw(b -- a --d, dashed);
draw(a -- e, dashed);
\end{asy}
\centering

(B)
\end{figure}

\begin{figure}[H]
\raggedright
\begin{asy}
size(3cm);
currentprojection=orthographic(3,1,.5);

triple a = (0,0,0);
triple b = (1,0,0);
triple c = (1,1,0);
triple d = (0,1,0);

triple e = (0,0,1);
triple f = (1,0,1);
triple g = (1,1,1);
triple h = (0,1,1);

draw(a -- b -- c -- d -- cycle);
draw(a -- b -- f -- e -- cycle);
draw(a -- d -- h -- e -- cycle);

draw(h -- g -- c, dashed);
draw(g -- f, dashed);
\end{asy}
\centering


(C)
\end{figure}
\end{multicols}
\caption{Um cubo visto de cima ou de baixo?}\label{\detokenize{GE301-1:fig-proj-ambiguidade-01}}\label{\detokenize{GE301-1:id17}}\end{figure}

De fato, a Imagem (A) pode até mesmo nem ser a representação de um cubo, como mostra a animação da \hyperref[\detokenize{GE301-1:fig-proj-ambiguidade-02}]{Figura \ref{\detokenize{GE301-1:fig-proj-ambiguidade-02}}}. A Imagem (A) é conhecida como \index{Cubo de Necker}Cubo de Necker, em homenagem ao cristalógrafo Louis Albert Necker (1786-1861) que observou este tipo de ambiguidade em 1832.



\begin{figure}[H]
\centering
\capstart

\noindent\includegraphics[width=250bp]{{ambiguidade-02}.jpg}
\caption{\href{https://goo.gl/CXR6AG}{Versão interativa}}\label{\detokenize{GE301-1:fig-proj-ambiguidade-02}}\label{\detokenize{GE301-1:id18}}\end{figure}

Compreender como vemos e interpretamos representações 2D de objetos 3D obtidas por projeções centrais e paralelas é uma habilidade importante que afeta o modo de nos cuminicarmos e interagirmos com o mundo.

\begin{reflection}

Se nosso cérebro distorce os estímos que recebemos do mundo a nossa volta, como saber o que é real?
\end{reflection}

\begin{reflection}

Será que uma pessoa que nasceu cega mas que, posteriormente, recuperou sua visão, saberia ver de imediato? Ou seria necessário “ensiná-la a ver”? Como saber, por exemplo, onde a imagem de um objeto termina e a imagem de outro começa?  Esta \href{https://goo.gl/KLdhKg}{palestra TED} discute esses assuntos, mostra a importância do movimento no processo de se “aprender a ver” e conta como o trabalho do neurocientista indiano Pawan Sinha tem mudado a concepção sobre os mecanismos da visão e, também, as vidas de muitas crianças que nasceram cegas.

\begin{figure}[H]
\centering

\noindent\includegraphics[width=50bp]{{ted-aprendendo-a-ver-02}.png}
\end{figure}

\begin{figure}[H]
\centering
\capstart

\noindent\includegraphics[width=350bp]{{ted-aprendendo-a-ver-01}.jpg}
\caption{Palestra TED.}\label{\detokenize{GE301-1:id19}}\end{figure}
\end{reflection}


\begin{reflection}

Todas estas questões de representações e significados fazem parte da \index{semiótica}semiótica, disciplina que se ocupa do estudo dos signos e dos processos significativos na natureza e na cultura. Os signos, aqui, não estão restritos à desenhos em uma folha de papel. Eles podem ser qualquer veículo de significação ou representação de um objeto, de um conceito ou de uma ideia, como textos, sons e gestos. Um dos pontos destacados pela semiótica é a distinção entre a representação de algo e este próprio algo. Um exemplo clássico é dada pela pintura na \hyperref[\detokenize{GE301-1:fig-proj-semiotica-01}]{Figura \ref{\detokenize{GE301-1:fig-proj-semiotica-01}}}. O que é que está na pintura? Se você respondeu “cachimbo”, saiba que a legenda em Francês “Ceci n’est pas une pipe.” tem como tradução “Isto não é um cachimbo.”. Segundo o autor da pintura, o surrealista belga René Magritte (1898-1967), ele não poderia escrever o contrário, pois a pintura não é um cachimbo, mas uma representação de um cachimbo. O nome da pintura: “A Traição das Imagens”.

\begin{figure}[H]
\centering
\capstart

\noindent\includegraphics[width=300bp]{{semiotica-01}.jpg}
\caption{Pintura de René Magritte (1898-1967).}\label{\detokenize{GE301-1:fig-proj-semiotica-01}}\label{\detokenize{GE301-1:id20}}\end{figure}

Uma vez que a comunicação se dá por meio de signos, a semiótica é de interesse para muitas áreas: Propaganda, Cinema, Ciência, Literatura, Religião … Em Matemática, o aspecto semiótico é fundamental, como aponta \cite{Pinilla-2007}:

\begin{quote}
É importante ter em mente que os conceitos matemáticos não existem na realidade concreta. O ponto P, o número 3, adição, paralelismo entre retas não são objetos concretos os quais existem na realidade empírica. Eles são conceitos puros, ideais e abstratos e, desta maneira, eles não podem ser “exibidos empiricamente”, como em outras Ciências. Em Matemática, os conceitos só podem ser representados por um registro semiótico determinado. De fato, em Matemática, não trabalhamos diretamente com os objetos (isto é, com os conceitos), mas com suas representações semióticas.
\end{quote}

Caso você queira saber mais sobre semiótica, recomendamos começar com o livro “O que é semiótica?” da Coleção “Primeiros Passos” da Editora Brasiliense \citep{Santaella-1998}.
\end{reflection}
\phantomsection\label{\detokenize{GE301-1:obs-proj-por-que-estudar-o-assunto}}
\begin{observation}

No que se segue, iremos estudar duas formas de representação bem específicas: aquelas obtidas por projeções em perspectiva e projeções paralelas.

As projeções em perspectiva fornecem um modelo matemático para a visão humana e para dispositivos óticos (como câmeras) e o estudo deste modelo auxilia na compreensão de como vemos, comunicamos e interagimos com o mundo. As projeções paralelas, por sua vez, fornecem uma representação mais simples e mais fácil de se entender e, por este motivo, elas têm sido utilizadas para a confecção de ilustrações em várias áreas: Arquitetura, Engenharia, Biologia, Física, etc.

Cabe observar que projeções em perspectiva e paralelas fazem parte das habilidades espaciais as quais, por sua vez, constituem um dos tipos reconhecidos de inteligência humana \citep{Gray-et-al-2004,Garner-2011}.

As habilidades espaciais são particularmente críticas para profissões relacionadas com as áreas de Ciência, Tecnologia, Engenharia e Matemática (STEM), conforme apontam vários estudos recentes \citep{NRC-2006,Uttal-et-al-2012,Khine-2017,Newcombe-2017}.

Mesmo no dia a dia, é importante, por exemplo, saber interpretar os diagramas 2D de objetos 3D que descrevem como montar uma cama, colocar um cartucho em uma impressora, abrir a porta de emergência do avião, descobrir a saída de emergência mais próxima em um hotel ou em um estádio de futebol (mapa de fuga, saídas de emergência), etc.

\begin{figure}[H]
\centering

\noindent\includegraphics[width=400bp]{{planta-baixa-03}.jpg}
\caption{Mapa do circuito de visitação do terceiro andar do Aquário do Rio de Janeiro (fonte: Joselí Maria Silva dos Santos).}
\end{figure}



Como veremos, as representações obtidas por projeções em perspectiva e projeções paralelas possuem propriedades bem \textbf{específicas}. Reconhecer e usar essas propriedades adequadamente é importante para você entender e se fazer entender em termos de comunicação visual.
\end{observation}

\begin{reflection}

\begin{DUlineblock}{0em}
\item[] \emph{This life five windows of the soul}
\item[] \emph{Distorts the heaven from pole to pole}
\item[] \emph{And leads you to believe a lie}
\item[] \emph{When you see with, no thro’, the eye.}
\end{DUlineblock}

Poema do poeta, pintor, ilustrador e entalhador William Blake (1757-1827).
\end{reflection}

\cleardoublepage
\def\currentcolor{session1}
\begin{objectives}{Luzes e sombras}
{
\begin{itemize}
\item {} 
Observar o comportamento das sombras de alguns objetos geométricos familiares, identificando o que varia e o que não varia de acordo com a posição do objeto, do anteparo de projeção e da fonte de luz; descrever essas variações.

\item {} 
Comparar as propriedades dos dois tipos de sombras: aquelas produzidas por uma lanterna de celular (projeção em perspectiva) e aquelas produzidas pela luz solar (projeção paralela).

\end{itemize}
}{1}{1}
\end{objectives}
\begin{sugestions}{Luzes e sombras}
{
\begin{itemize}
\item {} 
Para a confecção do cubo vazado, o seguinte molde pode ser usado. Uma versão em PDF para impressão está disponível neste endereço: \url{https://goo.gl/ap7NV8}.

\begin{figure}[H]
\centering

\noindent\includegraphics[width=.6\linewidth]{{cubo-vazado_1}.jpg}
\end{figure}

Naturalmente, outros modelos podem ser usados, como aqueles feitos com canudinhos. O importante é que modelo não fique muito irregular e tenha as faces vazadas para que se possa ver todas as arestas.

\item {} 
Sugerimos que grupos diferentes usem triângulos com formatos diferentes. Se o triângulo for confeccionado com papel comum, sugerimos que ele seja reforçado (colando-se duas ou mais cópias) para evitar que ele se dobre durante a realização dos experimentos.

\item {} 
Dependendo do tempo disponível em seu planejamento para este capítulo, você pode considerar incluir outros objetos geométricos (esferas, cones, pirâmides).

\item {} 
Caso nenhum celular esteja disponível, pode-se usar uma lanterna pequena comum em substituição. O importante é que a fonte de luz seja bem concentrada para se evitar efeitos de penumbra.

\item {} 
Apesar das paredes e do chão serem anteparos naturais que podem ser usados, nos experimentos com a luz do Sol, recomendamos o emprego de uma folha de papel A4 ou cartolina para projetar as sombras. Isto, além de permitir regular o ângulo de incidência dos raios do Sol sobre o anteparo, permitirá também que a sombra fique mais nítida. Sugerimos o antemparo em uma cor clara, mas não totalmente branca, pois esta cor pode provocar desconforto nos experimentos com a luz solar por conta do reflexo.

\item {} 
No caso das atividades que envolvem medidas, não se espera que estas sejam feitas de forma muito precisa. O importante é que o aluno perceba, por exemplo, que com o ajuste certo duas medidas poderiam ser iguais ou que, decididamente, as medidas podem ser feitas diferentes.

\item {} 
Sugerimos que os alunos sejam divididos em grupos de quatro. Nesta configuração, o primeiro aluno segura o anteparo (folha de cartolina ou caderno), o segundo segura o objeto cuja sombra será estudada, o terceiro segura o celular e o quarto toma eventuais medidas das sombras e registra as respostas.

\item {} 
Para otimizar o uso do tempo em sala de aula, sugerimos fortemente que cada grupo fique responsável por um tipo diferente de objeto (lápis, triângulo e cubo vazado).
\end{itemize}
}{0}{9}
\end{sugestions}

\begin{answer}{Luzes e sombras}
{
\paragraph{Experimentos com o lápis}
\begin{enumerate}
\item {} \begin{itemize}
\item {} 
Resposta para o caso da luz da lanterna do celular:

\end{itemize}

Não, pois é possível posicionar o lápis paralelo ao anteparo de projeção com a lanterna apontada para o meio do lápis e, com isso, o comprimento da sombra aumentará a medida que aproximarmos a lanterna do lápis.  Caso afastemos a lanterna do lápis, a sombra diminuirá de comprimento.

\notas{
\adjustbox{valign=t}
{
\begin{minipage}{\linewidth}
\begin{figure}[H]
\centering
\capstart

\noindent\includegraphics[width=200bp]{{20180204_140208}.jpg}
\caption{Lápis e sua sombra com comprimentos diferentes.}\label{\detokenize{GE301-2:fig-experimentoslapislanterna1}}\label{\detokenize{GE301-2:id4}}
\end{figure}
\end{minipage}
}
}
\begin{itemize}
\item {} 
Resposta para o caso da luz do Sol:

\end{itemize}

Não, pois se posicionarmos o lápis paralelo e oblíquo ao anteparo de projeção com o anteparo perpendicular aos raios do Sol, as sombras terão comprimentos diferentes em cada um dos casos.

\item {} \begin{itemize}
\item {} 
Resposta para o caso da luz da lanterna do celular:

\end{itemize}

Sim. Se rotacionarmos o lápis em torno do ponto onde nossos dedos o tocam enquanto mantemos a lanterna apontada para este mesmo ponto, a sombra terá diversos comprimentos e um deles é o próprio comprimento do lápis.
\begin{itemize}

\item {} 
Resposta para o caso da luz do Sol:

\end{itemize}

Sim. Se posicionarmos o lápis e o anteparo de projeção perpediculares aos raios solares, o lápis e sua sombra terão o mesmo comprimento.

\notasfig{
\begin{figure}[H]
\centering
\capstart

\noindent\includegraphics[width=200bp]{{20180212_1453290}.jpg}
\caption{Lápis e sua sombra com comprimentos diferentes.}\label{\detokenize{GE301-2:fig-experimentoslapissol1}}\label{\detokenize{GE301-2:id5}}
\end{figure}
}

\end{enumerate}
}{1}
\end{answer}
\begin{answer}{Luzes e sombras}
{
\paragraph{Experimentos com o lápis}
\begin{enumerate}\setcounter{enumi}{2}
\item {} \begin{itemize}
\item {} 
Resposta para o caso da luz da lanterna do celular:

\end{itemize}

Não, ao posicionar o lápis de forma oblíqua ao anteparo de projeção com a lanterna apontando para o ponto onde nossos dedos o tocam, o comprimento da sombra não será dividido ao meio pela sombra da ponta dos dedos.

\notasfig{\begin{figure}[H]
\centering
\capstart

\noindent\includegraphics[width=200bp]{{20180204_140336}.jpg}
\caption{A sombra das pontas dos dedos não está no meio da sombra do lápis.}\label{\detokenize{GE301-2:fig-experimentoslapislanterna2}}\label{\detokenize{GE301-2:id6}}\end{figure}}
\begin{itemize}
\item {} 
Resposta para o caso da luz do Sol:

\end{itemize}

Sim, neste caso, a sombra das pontas dos dedos sempre estará no meio da sombra do lápis.

\notasfig{\begin{figure}[H]
\centering
\capstart

\noindent\includegraphics[width=200bp]{{20180212_1454080}.jpg}
\caption{A sombra das pontas dos dedos está no meio da sombra do lápis.}\label{\detokenize{GE301-2:fig-experimentoslapissol2}}\label{\detokenize{GE301-2:id7}}
\end{figure}}

\item {} \begin{itemize}
\item {} 
Resposta para o caso da luz da lanterna do celular:

\end{itemize}

Sim. Mais uma vez, se rotacionarmos o lápis em torno do ponto onde nossos dedos o tocam enquanto mantemos a lanterna apontada para este mesmo ponto, podemos encontrar o resultado desejado.
\begin{itemize}
\item {} 
Resposta para o caso da luz do Sol:

\end{itemize}

A configuração desejada não existe.

\item {} \begin{itemize}
\item {} 
Resposta para o caso da luz da lanterna do celular:

\end{itemize}

Quando o lápis é posicionado perpendicular ao anteparo de projeção e apontando para a lanterna, sua sombra é um pequeno círculo.

\notasfig{\begin{figure}[H]
\centering
\capstart

\noindent\includegraphics[width=200bp]{{20180204_140403}.jpg}
\caption{A sombra do lápis é um pequeno círculo.}\label{\detokenize{GE301-2:fig-experimentoslapislanterna3}}\label{\detokenize{GE301-2:id8}}\end{figure}}
\begin{itemize}
\item {} 
Resposta para o caso da luz do Sol:

\end{itemize}

Quando o lápis é posicionado paralelo aos raios solares e perpendicular ao anteparo de projeção, sua sombra também é um pequeno círculo.

\notasfig{\begin{figure}[H]
\centering
\capstart

\noindent\includegraphics[width=200bp]{{20180212_1453490}.jpg}
\caption{A sombra do lápis é, novamente, um pequeno círculo.}\label{\detokenize{GE301-2:fig-experimentoslapissol3}}\label{\detokenize{GE301-2:id9}}\end{figure}}

\item {} \begin{itemize}
\item {} 
Resposta para o caso da luz da lanterna do celular:

\end{itemize}

Não, pois ao movimentar o lápis, a sombra sempre se altera.
\begin{itemize}
\item {} 
Resposta para o caso da luz do Sol:

\end{itemize}

Sim, ao posicionar o lápis e o anteparo de projeção perpendiculares aos raios de Sol, aproximando ou afastando o lápis do anteparo seguindo a direção dos raios solares, a sombra permanece do mesmo comprimento.
\end{enumerate}
}{1}
\end{answer}
\clearmargin
\begin{answer}{Luzes e sombras}
{
\paragraph{Experimentos com um triângulo}

\begin{enumerate}
\item {} \begin{itemize}
\item {} 
Resposta para o caso da luz da lanterna do celular:

\end{itemize}

Sim, pois é possível movimentar o triângulo de modo que dois de seus lados tenham o mesmo tamanho de sombra.

\notasfig{\begin{figure}[H]
\centering
\capstart

\noindent\includegraphics[width=200bp]{{20180204_142355}.jpg}
\caption{A sombra do triângulo é um triângulo isósceles.}\label{\detokenize{GE301-2:fig-experimentostriangulolanterna1}}\label{\detokenize{GE301-2:id10}}\end{figure}}
\begin{itemize}
\item {} 
Resposta para o caso da luz do Sol:

\end{itemize}

Sim, pois é possível movimentar o triângulo de modo que dois de seus lados tenham o mesmo tamanho de sombra.

\item {} \begin{itemize}
\item {} 
Resposta para o caso da luz da lanterna do celular:

\end{itemize}

Sim, pois é possível movimentar o triângulo de modo que três de seus lados tenham o mesmo tamanho de sombra.

\notasfig{\begin{figure}[H]
\centering
\capstart

\noindent\includegraphics[width=200bp]{{20180204_142323}.jpg}
\caption{A sombra do triângulo é um triângulo equilátero.}\label{\detokenize{GE301-2:fig-experimentostriangulolanterna2}}\label{\detokenize{GE301-2:id11}}\end{figure}}
\begin{itemize}
\item {} 
Resposta para o caso da luz do Sol:

\end{itemize}

Sim, pois é possível movimentar o triângulo de modo que dois de seus lados tenham o mesmo tamanho de sombra.

\item {} \begin{itemize}
\item {} 
Resposta para o caso da luz da lanterna do celular:

\end{itemize}

A menor área possível é atingida quando o triângulo é posicionado de forma que sua sombra seja um segmento de reta.

\notasfig{\begin{figure}[H]
\centering
\capstart

\noindent\includegraphics[width=200bp]{{20180204_142612}.jpg}
\caption{A sombra do triângulo é um segmento de reta.}\label{\detokenize{GE301-2:fig-experimentostriangulolanterna3}}\label{\detokenize{GE301-2:id12}}\end{figure}}
\begin{itemize}
\item {} 
Resposta para o caso da luz do Sol:

\end{itemize}

Assim como no caso anterior, a menor área possível é atingida quando o triângulo é posicionado de forma que sua sombra seja um segmento de reta.

\item {} \begin{itemize}
\item {} 
Resposta para o caso da luz da lanterna do celular:

\end{itemize}

Não. Ao movimentar o triângulo, sua sombra sempre será alterada.
\begin{itemize}
\item {} 
Resposta para o caso da luz do Sol:

\end{itemize}

Sim. Para isso, basta movimentar o triângulo na direção dos raios do Sol.

\item {} \begin{itemize}
\item {} 
Resposta para o caso da luz da lanterna do celular:

\end{itemize}

Não. O ponto iluminado não é o baricentro do triângulo.
\begin{itemize}
\item {} 
Resposta para o caso da luz do Sol:

\end{itemize}

Sim. O ponto iluminado é o baricentro da sombra do triângulo.
\end{enumerate}
}{1}
\end{answer}
\clearmargin
\begin{answer}{Luzes e sombras}
{\fontsize{10.75}{13}\selectfont
\paragraph{Experimentos com o cubo vazado}

\begin{enumerate}
\item {} \begin{itemize}
\item {} 
Resposta para o caso da luz da lanterna do celular:

\end{itemize}

Não. Nem todas as sombras das arestas do cubo possuem o mesmo tamanho.

\notasfig{\begin{figure}[H]
\centering
\capstart

\noindent\includegraphics[width=200bp]{{20180204_135307}.jpg}
\caption{Sombras das areas do cubo possuem tamanhos diferentes.}\label{\detokenize{GE301-2:fig-experimentoscubolanterna1}}\label{\detokenize{GE301-2:id13}}\end{figure}}
\begin{itemize}
\item {} 
Resposta para o caso da luz do Sol:

\end{itemize}

Sim. As sombras de todas as arestas do cubo possuem o mesmo tamanho.

\notasfig{\begin{figure}[H]
\centering
\capstart

\noindent\includegraphics[width=200bp]{{20180212_145440}.jpg}
\caption{Sombras das areas do cubo possuem o mesmo tamanho.}\label{\detokenize{GE301-2:fig-experimentoscubosol1}}\label{\detokenize{GE301-2:id14}}\end{figure}}

\item {} \begin{itemize}
\item {} 
Resposta para o caso da luz da lanterna do celular:

\end{itemize}

Sim, é possível posicionar o cubo de forma que sua sombra seja semelhante à da \hyperref[\detokenize{GE301-2:fig-proj-quadrado-vazado-01}]{Figura \ref{\detokenize{GE301-2:fig-proj-quadrado-vazado-01}}}. Movimentando o cubo na direção perpendicular ao anteparo de projeção e mantendo a lanterna apontada para o meio da face do cubo mais distante do anteparo, a sombra continuará sendo um quadrado.

\notasfig{\begin{figure}[H]
\centering
\capstart

\noindent\includegraphics[width=200bp]{{20180204_135548}.jpg}
\caption{A sombra do cubo é um quadrado.}\label{\detokenize{GE301-2:fig-experimentoscubolanterna2}}\label{\detokenize{GE301-2:id15}}\end{figure}}
\begin{itemize}
\item {} 
Resposta para o caso da luz do Sol:

\end{itemize}

Sim, é possível encontrar a sombra do cubo semelhante à imagem da \hyperref[\detokenize{GE301-2:fig-proj-quadrado-vazado-01}]{Figura \ref{\detokenize{GE301-2:fig-proj-quadrado-vazado-01}}}. Movimentando o cubo na direção dos raios solares e mantendo o anteparo na mesma posição, a sombra se manterá como a da \hyperref[\detokenize{GE301-2:fig-proj-quadrado-vazado-01}]{Figura \ref{\detokenize{GE301-2:fig-proj-quadrado-vazado-01}}}.

\notasfig{\begin{figure}[H]
\centering
\capstart

\noindent\includegraphics[width=200bp]{{20180212_145456}.jpg}
\caption{A sombra do cubo é um quadrado.}\label{\detokenize{GE301-2:fig-experimentoscubosol2}}\label{\detokenize{GE301-2:id16}}\end{figure}}

\item {} \begin{itemize}
\item {} 
Resposta para o caso da luz da lanterna do celular:

\end{itemize}

Não. Nem sempre as arestas perpendiculares determinam sombras perpendiculares.

\notasfig{\begin{figure}[H]
\centering
\capstart

\noindent\includegraphics[width=200bp]{{20180204_135542}.jpg}
\caption{Arestas perpendiculares do cubo nem sempre determinam sombras perpendiculares.}\label{\detokenize{GE301-2:fig-experimentoscubolanterna3}}\label{\detokenize{GE301-2:id17}}\end{figure}}
\begin{itemize}
\item {} 
Resposta para o caso da luz do Sol:

\end{itemize}

Não. Nem sempre as arestas perpendiculares determinam sombras perpendiculares.

\notasfig{\begin{figure}[H]
\centering
\capstart

\noindent\includegraphics[width=200bp]{{20180212_145519}.jpg}
\caption{Arestas perpendiculares do cubo nem sempre determinam sombras perpendiculares.}\label{\detokenize{GE301-2:fig-experimentoscubosol3}}\label{\detokenize{GE301-2:id18}}\end{figure}}

\item {} \begin{itemize}
\item {} 
Resposta para o caso da luz da lanterna do celular:

\end{itemize}

Não. Nem sempre arestas paralelas possuem sombras paralelas.

\notasfig{\begin{figure}[H]
\centering
\capstart

\noindent\includegraphics[width=200bp]{{20180204_135451}.jpg}
\caption{Arestas paralelas do cubo nem sempre determinam sombras paralelas.}\label{\detokenize{GE301-2:fig-experimentoscubolanterna4}}\label{\detokenize{GE301-2:id19}}\end{figure}}
\begin{itemize}
\item {} 
Resposta para o caso da luz do Sol:

\end{itemize}

Não. Nem sempre arestas paralelas possuem sombras paralelas.

\notasfig{\begin{figure}[H]
\centering
\capstart

\noindent\includegraphics[width=200bp]{{20180212_1455210}.jpg}
\caption{Arestas paralelas do cubo nem sempre determinam sombras paralelas.}\label{\detokenize{GE301-2:fig-experimentoscubosol4}}\label{\detokenize{GE301-2:id20}}\end{figure}}

\end{enumerate}
}{1}
\end{answer}
\begin{answer}{Luzes e sombras}
{
\paragraph{Outros experimentos}

\begin{enumerate}
\item {} 
Observaria a direção da sombra determinada por algum objeto.

\item {} 
As sombras não se alteram.

\item {} 
\(DHGC\)

\item {} 
Se a lanterna se aproxima do cubo, sua sombra aumenta. Se a lanterna se afasta do cubo, sua sombra diminui.

\item {} 
\(AEHD\)

\end{enumerate}
}{0}
\end{answer}
\clearmargin
\begin{objectives}{Dois modelos de projeção}
{
Ponderar sobre concepções de modelos geométricos que permitam representar projeções de sombras considerando, para isto, algumas hipóteses simplificadoras.
}{1}{1}
\end{objectives}
\begin{sugestions}{Dois modelos de projeção}
{
Caso um objeto opaco seja iluminado por uma fonte não pontual de luz, o bloqueio da luz por este objeto produz uma sombra mais complexa, com regiões e intensidades diferentes, como mostra o exemplo da \hyperref[\detokenize{GE301-2:fig-proj-sombras-01}]{Figura \ref{\detokenize{GE301-2:fig-proj-sombras-01}}}.

\begin{figure}[H]
\centering
\capstart

\noindent\includegraphics[width=\linewidth]{{sombras-01}.jpg}
\caption{Bloqueio de uma fonte não pontual de luz.}\label{\detokenize{GE301-2:fig-proj-sombras-01}}\label{\detokenize{GE301-2:id21}}\end{figure}

}{1}{1}
\end{sugestions}
\begin{answer}{Dois modelos de projeção}
{
\begin{enumerate}
\item {} \begin{itemize}
\item {} 
A \hyperref[\detokenize{GE301-2:fig-experimentos-modelosdeproj1}]{Figura \ref{\detokenize{GE301-2:fig-experimentos-modelosdeproj1}}} mostra uma fonte de luz, alguns raios luminosos que dela emanam e o anteparo.

\end{itemize}

\begin{figure}[H]
\centering
\capstart

\noindent\includegraphics[width=280bp]{{Lanterna_Raios}.png}
\caption{Diagrama representando uma fonte de luz e raios luminosos que dela emanam.}\label{\detokenize{GE301-2:fig-experimentos-modelosdeproj1}}\label{\detokenize{GE301-2:id23}}\end{figure}
\begin{itemize}
\item {} 
A \hyperref[\detokenize{GE301-2:fig-experimentos-modelosdeproj2}]{Figura \ref{\detokenize{GE301-2:fig-experimentos-modelosdeproj2}}} mostra a mesma situação anterior, onde foi incluído o triângulo opaco desenhado na cor vermelha. Os raios de luz que chegam a este triângulo não alcançam o anteparo e portanto, foram pontilhados a partir do ponto onde tocam o triângulo. A sombra provocada pelo triângulo no anteparo foi desenhada na cor preta.

\end{itemize}

\begin{figure}[H]
\centering
\capstart

\noindent\includegraphics[width=280bp]{{Lanterna_Raios_Triangulo}.png}
\caption{Diagrama representando um fonte de luz e seus raios luminosos, assim como a sombra produzida por um triângulo opaco.}\label{\detokenize{GE301-2:fig-experimentos-modelosdeproj2}}\label{\detokenize{GE301-2:id24}}\end{figure}

\item {} \begin{itemize}
\item {} 
As crianças normalmente representam o Sol como em (A), mas o Sol emana raios solares em todas as direções e por isso, sua representação mais fiel é a mostrada em (C).

\item {} 
Costuma-se supor que os raios de luz solares que atingem a Terra são paralelos. O Sol emana luz de forma homogênea em todas as direções ao seu redor e, portanto, não é possível que os seus raios sejam todos paralelos entre si.

\end{itemize}

\begin{figure}[H]
\centering
\capstart

\noindent\includegraphics[width=200bp]{{SolTerraRaiosSolares}.png}
\caption{Ilustração contendo o Sol e a Terra que não considera a escala real.}\label{\detokenize{GE301-2:fig-experimentos-modelosdeproj5}}\label{\detokenize{GE301-2:id25}}\end{figure}

Apenas uma parte dos raios emitidos pelo Sol atingem a Terra. Na verdade, repare na \hyperref[\detokenize{GE301-2:fig-experimentos-modelosdeproj5}]{Figura \ref{\detokenize{GE301-2:fig-experimentos-modelosdeproj5}}}, que apenas a porção delimitada por raios que partem de pontos quase diametralmente opostos do Sol é que atingem a Terra, formando algo similar a um cone truncado.

A priori, você pode achar que, diante da situação mostrada na ilustração contida na \hyperref[\detokenize{GE301-2:fig-experimentos-modelosdeproj5}]{Figura \ref{\detokenize{GE301-2:fig-experimentos-modelosdeproj5}}}, é um absurdo supor que os raios solares que atingem a Terra sejam paralelos, já que é possível detectar nesta simples ilustração raios que não são paralelos (como os que desenhamos para mostrar a delimitação dos raios que atingem a Terra). Mas, note que esta ilustração foi feita sem considerar a escala existente entre esses dois astros, e então, apesar do posicionamento espacial estar correto e também a ideia de que apenas parte dos raios solares atingem a Terra, a ilustração não condiz fielmente com a realidade.

Para retratar fielmente a realidade, é preciso considerar que o Sol possui um diâmetro de aproximadamente \(1.390.000~\text{km}\) enquanto a Terra possui cerca de \(13.000~\text{km}\) de diâmetro, e que a distância entre eles é de cerca de \(150\) milhões de quilômetros (valores retirados do site da NASA - \textit{National Aeronautics and Space Administration}). Diante de valores tão altos, não é possível encontrar uma escala adequada para construir esta mesma ilustração e ainda fazer com que ela caiba na página deste livro! Para você ter uma melhor intuição de como valores tão grandes se comportam na prática, visite a página \textless{}\url{https://goo.gl/sVfkQJ}\textgreater{} e visualize os astros e sua distância em uma escala real.

\begin{figure}[H]
\centering
\capstart

\noindent\includegraphics[width=250bp]{{SolTerraRaiosSolares2}.png}
\caption{Ilustração contendo o Sol e a Terra, agora mais afastados, que não considera a escala real.}\label{\detokenize{GE301-2:fig-experimentos-modelosdeproj6}}\label{\detokenize{GE301-2:id26}}\end{figure}

Já que não podemos desenhar uma situação com uma escala real aqui nesta página, vamos tentar intuitivamente imaginar o que acontece na situação real. Na \hyperref[\detokenize{GE301-2:fig-experimentos-modelosdeproj6}]{Figura \ref{\detokenize{GE301-2:fig-experimentos-modelosdeproj6}}}, refizemos a mesma ilustração anterior, mas posicionamos a Terra e o Sol mais afastados. Comparando as duas ilustrações, podemos perceber que a inclinação dos raios solares é menor quando a Terra e o Sol estão mais afastados. Com a grande distância existente entre o Sol e a Terra (muito maior que a mostrada nas duas ilustrações), haverá então uma diminuição ainda maior na inclinação dos raios solares que atingem a Terra. Essa inclinação é tão pequena que os raios parecem ser paralelos! Por isso costuma-se supor que os raios são paralelos, apesar de eles não serem. Vamos calcular o ângulo de inclinação entre dois raios solares em uma situação específica para justificar nossa intuição de que a inclinação dos raios é muito pequena. Para isso calcularemos o ângulo entre dois raios usando Trigonometria.

\begin{figure}[H]
\centering
\capstart

\noindent\includegraphics[width=225bp]{{SolTerraAngulo1_1}.png}
\caption{Raios solares incidindo sobre o ponto \(P\) da Terra.}\label{\detokenize{GE301-2:fig-experimentos-modelosdeproj7}}\label{\detokenize{GE301-2:id27}}\end{figure}

Na ilustração mostrada na \hyperref[\detokenize{GE301-2:fig-experimentos-modelosdeproj7}]{Figura \ref{\detokenize{GE301-2:fig-experimentos-modelosdeproj7}}}, considere \(C_T\) e \(C_S\) como os centros da Terra e do Sol, respectivamente, e vamos calcular o ângulo \(Q\hat{P}R\) entre os raios solares \(PQ\) e \(PR\) que atingem o ponto \(P\) da Terra vindos do Sol. Note que há algo similar a um cone (não completo) de raios solares incidindo sobre \(P\), mas \(Q\hat{P}R\) é o maior ângulo possível entre dois raios solares que atingem \(P\). Quaisquer outros dois raios estarão contidos nesse cone e então, seu ângulo será menor que \(Q\hat{P}R\). Além disso, \(PQ\) e \(PR\) deverão ser perpendiculares ao raio do Sol para que, de fato, \(Q\hat{P}R\) seja o maior ângulo possível entre dois raios que atingem \(P\).

Traçando o segmento que une \(C_T\) e \(C_S\), o quadrilátero \(PQC_SR\) fica dividido em dois triângulos congruentes \(PQC_S\) e \(PRC_S\) (pelo caso lado, ângulo, lado). Assim, os ângulos \(Q\hat{P}C_S\) e \(R\hat{P}C_S\) são congruentes e serão chamados de \(\theta\) para facilitar a escrita. Como
\begin{equation*}
\begin{split}d(Q,C_S)= 695.000\end{split}
\end{equation*}
e
\begin{equation*}
\begin{split}\begin{array}{ll}
d(P,C_S) & = d(C_T, C_S)-d(C_T,P)\\
      &= 150.000.000-6.500\\
      &=149.993.500,
\end{array}\end{split}
\end{equation*}
então
\begin{equation*}
\begin{split}\sin\theta = \frac{d(Q,C_S)}{d(P,C_S)}=0{,}00463.\end{split}
\end{equation*}
Portanto, \(\theta=\arcsin(0,00463) = 0{,}26528\) e assim, como a medida do ângulo \(Q\hat{P}C_S\) é igual a \(2\theta\), temos que
\begin{equation*}
\begin{split}m(Q\hat{P}C_S)=0{,}53056^\circ.\end{split}
\end{equation*}
De acordo com os cálculos feitos acima, o maior ângulo  possível entre dois raios solares que atingem o ponto \(P\) é de \(0{,}53056^\circ\). Este valor é tão pequeno (podendo ser ainda menor), que torna-se praticamente imperceptível, e assim, os raios solares que atingem \(P\) parecem paralelos. Esse mesmo raciocínio pode ser utilizado para calcular o ângulo de inclinação de raios solares que atingem outros pontos. Sugerimos que você altere a posição do ponto \(P\) e faça os cálculos para outros casos da mesma forma que foi feito anteriormente.

Vamos agora comparar o tamanho da sombra de um lápis supondo que os raios solares que atingem a Terra são paralelos (projeção paralela) e não paralelos (projeção em perspectiva). Dessa forma, veremos o quão relevante é, de fato, a inclinação dos raios solares ao gerar sombras na Terra.

Na \hyperref[\detokenize{GE301-2:fig-experimentos-modelosdeproj8}]{Figura \ref{\detokenize{GE301-2:fig-experimentos-modelosdeproj8}}}, novamente sem utilizar uma escala real que considere os tamanhos do Sol e da Terra, temos uma ilustração da situação que pretendemos analisar. Suponhamos que o Sol, a Terra e o lápis estejam posicionados de tal forma que o segmento que une os centros da Terra e do Sol, que chamaremos de \(C_T\) e \(C_S\) respectivamente, divida o lápis ao meio. Suponhamos que o lápis possua \(8~\text{cm}\) de comprimento e que ele esteja a \(1~\text{m}\) da Terra.

\begin{figure}[H]
\centering
\capstart

\noindent\includegraphics[width=280bp]{{SolTerraLapis4}.png}
\caption{Ilustração contendo um lápis, o Sol e a Terra que não considera a escala real.}\label{\detokenize{GE301-2:fig-experimentos-modelosdeproj8}}\label{\detokenize{GE301-2:id28}}\end{figure}

Primeiramente, vamos considerar o caso em que os raios solares são paralelos. Para estudar a sombra do lápis neste caso, vamos supor que os raios solares estão incidindo perpendicularmente ao lápis. Como nosso objetivo é calcular apenas o tamanho da sombra de um lápis de \(8~\text{cm}\), na ilustração contida na \hyperref[\detokenize{GE301-2:fig-experimentos-modelosdeproj9}]{Figura \ref{\detokenize{GE301-2:fig-experimentos-modelosdeproj9}}} representaremos apenas a parte da Terra que é relevante para o nosso cálculo e, devido ao grande raio da Terra, locamente podemos considerá-la plana. Nesta ilustração, o segmento \(AC\) representa o lápis e então, a sombra do lápis está representada pelo segmento \(DF\) e terá a mesma medida de \(AC\). Portanto, a medida da sombra é \(16~\text{cm}\) (um estudo semelhante a este já havia sido foi feito anteriormente na Atividade: luzes e sombras).

\begin{figure}[H]
\centering
\capstart

\noindent\includegraphics[width=280bp]{{SolTerraLapis9}.png}
\caption{A sombra do lápis \(AC\) é dada pelo segmento \(DF\), casos os raios solares sejam paralelos e incidam perpendicularmos ao lápis.}\label{\detokenize{GE301-2:fig-experimentos-modelosdeproj9}}\label{\detokenize{GE301-2:id29}}\end{figure}

Para estudar agora o caso em que os raios solares não são paralelos, vamos usar a ilustração contida na \hyperref[\detokenize{GE301-2:fig-experimentos-modelosdeproj10}]{Figura \ref{\detokenize{GE301-2:fig-experimentos-modelosdeproj10}}}. O lápis está representado na figura pelo segmento \(AC\) e sua sombra pelo segmento \(DF\).

\begin{figure}[H]
\centering
\capstart

\noindent\includegraphics[width=280bp]{{SolTerraLapis10}.png}
\caption{A sombra do lápis \(AC\) é dada pelo segmento \(DF\) casos os raios solares não sejam paralelos.}\label{\detokenize{GE301-2:fig-experimentos-modelosdeproj10}}\label{\detokenize{GE301-2:id30}}\end{figure}

Como estamos supondo que o segmento \(C_TC_S\) divide o lápis em duas parte iguais, o ponto \(B\) é o ponto médio do segmento \(AC\). Assim, podemos concluir que os triângulos \(ABI\) e \(CBI\) são congruentes (pelo caso lado, ângulo, lado), o que implica que os ângulos \(A\hat{I}B\) e \(C\hat{I}B\) são congruentes. E, como  \(A\hat{I}B\) e \(C_S\hat{I}G\) são opostos pelo vértice, assim como os ângulos \(C\hat{I}B\) e \(C_S\hat{I}H\), podemos concluir que os ângulos \(A\hat{I}B, C_S\hat{I}G, C\hat{I}B\) e \(C_S\hat{I}H\) são congruentes. Portanto, os triângulos \(DEI\) e \(FEI\) são congruentes, assim como os triângulos \(C_SHI\) e \(C_SGI\). Logo, podemos trabalhar apenas com metade do lápis para facilitar nossos cálculos e posteriormente, basta multiplicar a medida da sombra de metade do lápis por \(2\).

Dessa forma, estamos interessados em encontrar o comprimento do segmento \(DE\) que representa a sombra de metade do lápis, representado pelo segmento \(AB\). Note que os triângulos \(ABI\), \(DEI\) e \(C_SGI\) são semelhantes (pois possuem dois ângulos congruentes), e  portanto satisfazem as seguintes razões de semelhança:
\begin{equation*}
\begin{split}\frac{AB}{BI}=\frac{DE}{EI}\end{split}
\end{equation*}
e
\begin{equation*}
\begin{split}\frac{AB}{AI}=\frac{C_SG}{C_SI}.\end{split}
\end{equation*}
Como a distância do lápis à Terra é de \(1~\text{m}\), então \(EB=1~\text{m}=0{,}001~\text{km}\). Além disso, \(AB=8~\text{cm}\) \(=0{,}00008~\text{km}\) e \(C_SG= 695.000~\text{km}\). Portanto,
\begin{equation*}
\begin{split}\begin{array}{ll}
 C_SI & = C_TC_S-C_TE-EB-BI\\
         & = 150.000.000-6.500-BI-0,001 \\
         & = 149.993.499,999-BI.
\end{array}\end{split}
\end{equation*}
Substituindo estes valores na equações acima, temos:
\begin{equation*}
\begin{split}\frac{0,00008}{BI}=\frac{DE}{0,001+BI}\end{split}
\end{equation*}
e
\begin{equation*}
\begin{split}\frac{0,00008}{\sqrt{BI^2+0,00008^2}} = \frac{695.000}{149.993.499,999-BI}.\end{split}
\end{equation*}
Resolvendo a segunda equação, encontramos \(BI=0{,}01726~\text{km}\) e substituindo este valor na primeira equação, vemos que a medida do segmento \(DE\) é \(8{,}4~\text{cm}\). Assim, a sombra do lápis, dada pelo segmento \(DF\), possui comprimento \(16{,}8~\text{cm}\).

Portanto, supondo que os raios solares que atingem a Terra são paralelos, a sombra de um lápis de \(16~\text{cm}\) seria de \(16~\text{cm}\). Supondo que estes raios não são paralelos, a sombra do mesmo lápis mede \(16{,}8~\text{cm}\). Note que a diferença entre os valores encontrados para as sombras é de \(5\%\) do comprimento do lápis.
\begin{itemize}
\item {} 
A \hyperref[\detokenize{GE301-2:fig-experimentos-modelosdeproj3}]{Figura \ref{\detokenize{GE301-2:fig-experimentos-modelosdeproj3}}} mostra a representação de alguns raios luminosos paralelos (desenhados da cor azul) que emanam do Sol sobre um anteparo de projeção.

\end{itemize}

\begin{figure}[H]
\centering
\capstart

\noindent\includegraphics[width=250bp]{{Sol_Raios3}.png}
\caption{Diagrama representando raios luminosos que emanam do Sol sobre um anteparo.}\label{\detokenize{GE301-2:fig-experimentos-modelosdeproj3}}\label{\detokenize{GE301-2:id31}}\end{figure}
\begin{itemize}
\item {} 
A \hyperref[\detokenize{GE301-2:fig-experimentos-modelosdeproj4}]{Figura \ref{\detokenize{GE301-2:fig-experimentos-modelosdeproj4}}} mostra a mesma situação anterior, mas agora foi incluído um triângulo opaco desenhado na cor vermelha. Alguns raios solares são impedidos de alcançar o anteparo ao encontrar o triângulo e portanto, foram pontilhados em nosso desenho a partir do ponto onde tocam o triângulo. A sombra provocada pelo triângulo no anteparo foi desenhada na cor preta.

\end{itemize}

\begin{figure}[H]
\centering
\capstart

\noindent\includegraphics[width=280bp]{{Sol_Raios_Triangulo3}.png}
\caption{Diagrama representando raios luminosos que emanam do Sol sobre um anteparo, assim como a sombra produzida por um triângulo opaco.}\label{\detokenize{GE301-2:fig-experimentos-modelosdeproj4}}\label{\detokenize{GE301-2:id32}}\end{figure}

\end{enumerate}
}{9}
\end{answer}

\explore{Projeções Paralelas e em Perspectiva}
\label{\detokenize{GE301-2:explorando-projecoes-em-perspectiva-e-projecoes-paralelas}}\label{\detokenize{GE301-2::doc}}\phantomsection\label{\detokenize{GE301-2:ativ-proj-luz-e-sombras}}
\begin{task}{Luzes e sombras}

Nesta atividade vamos explorar a geometria das sombras! Para isto, você receberá um \emph{kit} cujas sombras deverá analisar: um cubo vazado, um triângulo e um lápis (ou uma caneta ou ainda um canudinho plástico). Você também receberá uma folha de papel A4 ou uma cartolina que servirá como anteparo onde as sombras devem ser projetadas.

\textbf{No que se segue, o termo *configuração* significa uma escolha da posição do objeto, da fonte de luz e do anteparo, conforme o caso.} Assim, o termo \emph{para qualquer configuração} significa para qualquer escolha da posição do objeto, da fonte de luz e do anteparo.

\paragraph{Experimentos com um lápis}
\begin{enumerate}
\item {} 
O comprimento da sombra é sempre igual ao comprimento do lápis, independentemente da configuração?
\begin{itemize}
\item {} 
Resposta para o caso da luz da lanterna do celular:

\item {} 
Resposta para o caso da luz do Sol:

\end{itemize}

\item {} 
Existe alguma configuração para a qual o comprimento da sombra seja igual ao comprimento do lápis?
\begin{itemize}
\item {} 
Resposta para o caso da luz da lanterna do celular:

\item {} 
Resposta para o caso da luz do Sol:

\end{itemize}

\item {} 
Segure o seu lápis no meio com as pontas de seus dedos, isto é, considerando o lápis como se fosse um segmento de reta, segure-o pelo seu ponto médio. A sombra das pontas de seus dedos sempre está no meio da sombra do lápis independentemente da configuração?
\begin{itemize}
\item {} 
Resposta para o caso da luz da lanterna do celular:

\item {} 
Resposta para o caso da luz do Sol:

\end{itemize}

\item {} 
Segure o seu lápis, com as pontas de seus dedos, a aproximadamente 1/3 de uma das extremidades. Existe alguma configuração para a qual a sombra das pontas de seus dedos está no meio da sombra do lápis?
\begin{itemize}
\item {} 
Resposta para o caso da luz da lanterna do celular:

\item {} 
Resposta para o caso da luz do Sol:

\end{itemize}

\item {} 
Em qual configuração a sombra do lápis tem a menor área possível?
\begin{itemize}
\item {} 
Resposta para o caso da luz da lanterna do celular:

\item {} 
Resposta para o caso da luz do Sol:

\end{itemize}

\item {} 
Existe alguma configuração onde a sombra não se altere ao mover o lápis em alguma direção? Aqui, \emph{não se altere} signfica ser exatamente a mesma no mesmo lugar.
\begin{itemize}
\item {} 
Resposta para o caso da luz da lanterna do celular:

\item {} 
Resposta para o caso da luz do Sol:

\end{itemize}

\end{enumerate}

\paragraph{Experimentos com um triângulo}
\begin{enumerate}
\item {} 
Existe alguma configuração para a qual a sombra do triângulo é um triângulo isósceles?
\begin{itemize}
\item {} 
Resposta para o caso da luz da lanterna do celular:

\item {} 
Resposta para o caso da luz do Sol:

\end{itemize}

\item {} 
Existe alguma configuração para a qual a sombra do triângulo é um triângulo equilátero?
\begin{itemize}
\item {} 
Resposta para o caso da luz da lanterna do celular:

\item {} 
Resposta para o caso da luz do Sol:

\end{itemize}

\item {} 
Em qual configuração a sombra do triângulo tem a menor área possível?
\begin{itemize}
\item {} 
Resposta para o caso da luz da lanterna do celular:

\item {} 
Resposta para o caso da luz do Sol:

\end{itemize}

\item {} 
Existe alguma configuração onde a sombra do triângulo não se altere ao movê-lo em alguma direção? Qual?
\begin{itemize}
\item {} 
Resposta para o caso da luz da lanterna do celular:

\item {} 
Resposta para o caso da luz do Sol:

\end{itemize}

\item {} 
O baricentro de um triângulo é o  ponto de interseção das \index{medianas}medianas do triângulo, isto é, o ponto de interseção dos segmentos de reta que ligam um vértice ao ponto médio do lado oposto. Faça um furo no \index{baricentro}baricentro do seu triângulo, de forma que, ao expô-lo à luz, o ponto correspondente no anteparo ficará iluminado. Este ponto iluminado é baricentro da sombra do triângulo?
\begin{itemize}
\item {} 
Resposta para o caso da luz da lanterna do celular:

\item {} 
Resposta para o caso da luz do Sol:

\end{itemize}

\end{enumerate}

\paragraph{Experimentos com um cubo vazado}
\begin{enumerate}
\item {} 
As arestas do cubo vazado têm todas o mesmo tamanho. O mesmo acontece para as sombras destas arestas?
\begin{itemize}
\item {} 
Resposta para o caso da luz da lanterna do celular:

\item {} 
Resposta para o caso da luz do Sol:

\end{itemize}

\item {} 
Existe alguma configuração para a qual a sombra do cubo vazado seja semelhante à imagem da \hyperref[\detokenize{GE301-2:fig-proj-quadrado-vazado-01}]{Figura \ref{\detokenize{GE301-2:fig-proj-quadrado-vazado-01}}}? Em caso afirmativo, é possível manter esta sombra movendo o cubo vazado em alguma direção? Qual?

\begin{figure}[H]
\centering
\capstart

\noindent\includegraphics[width=150bp]{{quadrado-vazado-01_2}.jpg}
\caption{Quadrado vazado.}\label{\detokenize{GE301-2:fig-proj-quadrado-vazado-01}}\label{\detokenize{GE301-2:id1}}\end{figure}
\begin{itemize}
\item {} 
Resposta para o caso da luz da lanterna do celular:

\item {} 
Resposta para o caso da luz do Sol:

\end{itemize}

\item {} 
Arestas que são perpendiculares no cubo vazado têm sombras que são perpendiculares no anteparo de projeção?
\begin{itemize}
\item {} 
Resposta para o caso da luz da lanterna do celular:

\item {} 
Resposta para o caso da luz do Sol:

\end{itemize}

\item {} 
Arestas que são são paralelas no cubo vazado têm sombras que são paralelas no anteparo de projeção?
\begin{itemize}
\item {} 
Resposta para o caso da luz da lanterna do celular:

\item {} 
Resposta para o caso da luz do Sol:

\end{itemize}

\end{enumerate}

\paragraph{Outros experimentos}
\begin{enumerate}
\item {} 
Como você faria para determinar a direção de incidência dos raios solares no anteparo?

\item {} 
Posicione o anteparo perpendicularmente à direção de incidência dos raios solares. O que acontece com o formato da sombra do lápis, do triângulo ou do cubo se você movê-los \textbf{paralelamente} à direção de incidência dos raios solares?

\item {} 
Na \hyperref[\detokenize{GE301-2:fig-proj-sombra-vazada-01}]{Figura \ref{\detokenize{GE301-2:fig-proj-sombra-vazada-01}}}, PQRS é sombra de qual face do cubo vazado? Tente responder analisando apenas a figura e, depois, teste a sua resposta com um experimento!

\begin{figure}[H]
\centering
\capstart

\noindent\includegraphics[width=.9\linewidth]{{sombra-vazada-01_1}.jpg}
\caption{Sombra vazada.}\label{\detokenize{GE301-2:fig-proj-sombra-vazada-01}}\label{\detokenize{GE301-2:id2}}\end{figure}

\item {} 
Na configuração da \hyperref[\detokenize{GE301-2:fig-proj-sombra-vazada-01}]{Figura \ref{\detokenize{GE301-2:fig-proj-sombra-vazada-01}}}, o que acontece com a sombra do cubo vazada se a lanterna do celular se aproximar do cubo? E se a lanterna se afastar?

\item {} 
Na \hyperref[\detokenize{GE301-2:fig-proj-sombra-vazada-02}]{Figura \ref{\detokenize{GE301-2:fig-proj-sombra-vazada-02}}}, PQRS é sombra de qual face do cubo vazado? Tente responder analisando apenas a figura e, depois, teste a sua resposta com um experimento!

\begin{figure}[H]
\centering
\capstart

\noindent\includegraphics[width=.9\linewidth]{{sombra-vazada-02_1}.jpg}
\caption{Sombra vazada.}\label{\detokenize{GE301-2:fig-proj-sombra-vazada-02}}\label{\detokenize{GE301-2:id3}}\end{figure}

\end{enumerate}
\end{task}

\clearpage
\phantomsection\label{\detokenize{GE301-2:ativ-proj-modelos-de-projecao}}
\begin{task}{Dois modelos de projeção}


O objetivo desta atividade é levar você a ponderar sobre concepções de modelos geométricos que permitam representar projeções de sombras considerando, para isto, algumas hipóteses simplificadoras. Esses modelos serão úteis no que se segue ao longo do capítulo. De fato, com esse conhecimento, será possível explicar e quantificar os fenômenos que você observou na \DUrole{xref,std,std-ref}{ativ-proj-luz-e-sombras} e, também, compreender o seu uso em aplicações diversas.
\begin{enumerate}
\item {} 
Vamos supor que a lanterna do celular possa ser representada por um ponto que emite raios de luz.
\begin{itemize}
\item {} 
Desenhe, a lápis, um diagrama representando o ponto de luz, alguns raios luminosos que dele emanam e como estes atingem o anteparo.

\item {} 
No desenho que você fez no item anterior, inclua um triângulo opaco entre o ponto de luz e o anteparo. Que partes dos raios de luz deixarão de atingir o anteparo? Redesenhe estas partes usando uma linha tracejada. Como ficará desenhada a sombra do triângulo?

\end{itemize}

\item {} \begin{itemize}
\item {} 
Considere a \hyperref[\detokenize{GE301-2:fig-proj-raios-do-sol-03}]{Figura \ref{\detokenize{GE301-2:fig-proj-raios-do-sol-03}}}. Pergunta 1: qual representação do Sol é mais comum entre as crianças? (A), (B) ou (C)? Pergunta 2: qual representação do Sol é mais fiel ao comportamento dos raios de luz? (A), (B) ou (C)?

\end{itemize}

\begin{figure}[H]
\centering
\capstart

\noindent\includegraphics[width=350bp]{{raios-de-luz-03_1}.jpg}
\caption{Três representações dos raios do Sol.}\label{\detokenize{GE301-2:fig-proj-raios-do-sol-03}}\label{\detokenize{GE301-2:id22}}\end{figure}
\begin{itemize}
\item {} 
Uma simplificação frequentemente usada é a de admitir que os raios do Sol chegam à Terra paralelos entre si. Essa simplificação é razoável para você? Dê argumentos que justifiquem sua opinião!

\item {} 
Admitindo que os raios do Sol chegam à Terra paralelos entre si, desenhe, a lápis, um diagrama representando alguns raios solares atingindo o anteparo.

\item {} 
No desenho que você fez no item anterior, inclua um triângulo opaco. Que partes dos raios de luz deixarão de atingir o anteparo? Redesenhe estas partes usando uma linha tracejada. Como ficará desenhada a sombra do triângulo?

\end{itemize}

\end{enumerate}
\end{task}
\clearpage
\def\currentcolor{session4}
\begin{sugestions}{Projeções parelelas e em perspectiva}
{
\begin{itemize}
\item {} 
Compreender a definição de projeção em perspectiva pressupõe a compreensão da \hyperref[\detokenize{GE301-3:fig-proj-perspectiva-01}]{Figura \ref{\detokenize{GE301-3:fig-proj-perspectiva-01}}}, da \hyperref[\detokenize{GE301-3:fig-proj-perspectiva-02}]{Figura \ref{\detokenize{GE301-3:fig-proj-perspectiva-02}}} e da \hyperref[\detokenize{GE301-3:fig-proj-perspectiva-03}]{Figura \ref{\detokenize{GE301-3:fig-proj-perspectiva-03}}}. Elas são representações 2D de configurações 3D, justamente o assunto desenvolvido neste capítulo. Assim, se o único recurso didático disponível for tão somente este livro didático, o aluno se verá na árdua tarefa de tentar entender o que é uma imagem produzida por uma projeção em perspectiva usando aquilo que está querendo entender, a saber, imagens produzidas por projeções em perspectiva. Temos aí um dilema didático recursivo. Neste contexto, \textbf{sugerimos fortemente} que você use com seus alunos as construções do GeoGebra 3D disponíveis nos endereços \url{https://www.geogebra.org/m/TxYEqV5d} (projeção de um ponto), \url{https://www.geogebra.org/m/YpH3fH9E} (projeção de um segmento de reta), \url{https://www.geogebra.org/m/HFjMUzgz} (projeção de um triângulo) e \url{https://www.geogebra.org/m/GhC6qGVY} (projeção de um Cubo de Necker) para computadores desktop, tablets e smarphones.

\notasfig{\begin{figure}[H]
\centering

\noindent\includegraphics[width=\linewidth]{{perspectiva-celular-05_1}.jpg}
\end{figure}}

Por meio destas construções, é possível girar, ampliar e reduzir a cena 3D, bem como modificar as posições dos vários pontos que compõem a configuração. Com esse recurso dinâmico e de movimento, efeitos de ambiguidade e equívocos de interpretação podem ser minimizados.

\item {} 
Do mesmo modo e pelos mesmos motivos, \textbf{sugerimos fortemente} que, para a discussão sobre projeções paralelas, você use com seus alunos as construções do GeoGebra 3D disponíveis nos endereços \url{https://www.geogebra.org/m/ZAkSPrYN} (projeção de um ponto), \url{https://www.geogebra.org/m/zdw7rVHc} (projeção de um segmento de reta), \url{https://www.geogebra.org/m/Ug965Anr} (projeção de um triângulo) e \url{https://www.geogebra.org/m/EaSUX99g} (projeção de um Cubo de Necker).


\notasfig{\begin{figure}[H]
\centering

\noindent\includegraphics[width=\linewidth]{{paralela-celular-05}.jpg}
\end{figure}}

\item {} 
Recomendamos que você enfatize, em uma linguagem que julgue adequada a seus alunos, duas ideias fundamentais associadas às projeções em perspectiva e paralelas: (1) interdependência (o ponto projetado \(P'\) depende funcionalmente do ponto \(P\)); (2) variância (propriedades que são e não preservadas pelas projeções). Acreditamos que essa explicação ficará potencializada por meio das construções interativas do GeoGebra indicadas anteriormente.

\end{itemize}
}{1}{2}
\end{sugestions}

\arrange{Projeções Paralelas e em Perspectiva}
\label{\detokenize{GE301-3:organizando-as-ideias-projecoes-em-perspectiva-e-projecoes-paralelas}}\label{\detokenize{GE301-3::doc}}
\subsection{Projeções em perspectiva}

Na \DUrole{xref,std,std-ref}{ativ-proj-modelos-de-projecao}, vamos modelar a lanterna do celular como um ponto \(O\) e o anteparo como um plano \(\pi\) (o plano de projeção). Um objeto opaco, como o triângulo \(ABC\) na \hyperref[\detokenize{GE301-3:fig-proj-perspectiva-01}]{Figura \ref{\detokenize{GE301-3:fig-proj-perspectiva-01}}}, irá obstruir os raios de luz que emanam de \(O\), produzindo uma sombra sobre o plano \(\pi\). Como determinar exatamente quais pontos de \(\pi\) percentem à sombra? Para cada ponto \(P\) do triângulo \(ABC\), construa a reta \(OP\) que liga \(O\) a \(P\). Esta reta irá intersectar o plano \(\pi\) em um ponto \(P'\). Este ponto \(P'\) de interseção da reta \(OP\) com o plano \(\pi\) é, portanto, um ponto da sombra do triângulo \(ABC\). De fato, todo ponto \(P'\) da sombra é obtido por este processo, isto é, um ponto \(P'\) do plano pertence à sombra do triângulo \(ABC\) se, e somente se, existe um ponto \(P\) do triângulo \(ABC\) tal que a interseção da reta \(OP\) com o plano \(\pi\) é o ponto \(P'\). Além do ponto \(P'\), a \hyperref[\detokenize{GE301-3:fig-proj-perspectiva-01}]{Figura \ref{\detokenize{GE301-3:fig-proj-perspectiva-01}}} mostra também o processo para os pontos \(A'\), \(B'\) e \(C'\).

\begin{figure}[H]
\centering
\capstart

\noindent\includegraphics[width=350bp]{{projecao-perspectiva-01_2}.jpg}
\caption{Um modelo para o experimento com a lanterna do celular.}\label{\detokenize{GE301-3:fig-proj-perspectiva-01}}\label{\detokenize{GE301-3:id1}}\end{figure}

Vamos agora abstrair ainda mais o processo, ou seja, vamos considerar um contexto matemático que, apesar de inspirado por luzes e sombras, será puramente geométrico. Esta abstração será útil para modelar outras situações, como veremos mais adiante.

Desta maneira, considere no espaço tridimensional \({\mathbb R}^{3}\) um plano \(\pi\) e um ponto \(O\). Seja \(\psi\) o plano paralelo à \(\pi\) passando por \(O\). Se \(P\) é um ponto que não pertence a \(\psi\), então o ponto \(P'\) de intersecção entre a reta \(OP\) e o plano \(\pi\) é denominado \index{projeção em perspectiva}projeção em perspectiva do ponto \(P\) sobre o \index{plano de projeção}plano de projeção \(\pi\) com relação ao \index{centro}centro \(O\).

\begin{figure}[H]
\centering
\capstart

\noindent\includegraphics[width=300bp]{{projecao-perspectiva-05}.jpg}
\caption{\(P'\) é a projeção em perspectiva do ponto \(P\) sobre o plano de projeção \(\pi\) com relação ao centro \(O\).}\label{\detokenize{GE301-3:fig-proj-perspectiva-02}}\label{\detokenize{GE301-3:id2}}\end{figure}

Vamos agora considerar uma outra situação onde projeções em perspectiva aparecem. Suponha que você queira desenhar um quadro de uma cena. Mas você quer um quadro tão perfeito que, ao observá-lo frente à cena, ele se confunda como a própria cena. O pintor surrealista belga René Magritte (1898-1967) imaginou essa situação em alguns de seus quadros (\hyperref[\detokenize{GE301-3:fig-proj-janela-de-alberti-01}]{Figura \ref{\detokenize{GE301-3:fig-proj-janela-de-alberti-01}}}). Como produzir um tal quadro?

\begin{figure}[H]
\centering
\capstart

\noindent\includegraphics[width=400bp]{{rene-magritte-the-human-condition-03}.jpg}
\caption{Pinturas “A Condição Humana” do artista surrealista belga René Magritte (1898-1967).}\label{\detokenize{GE301-3:fig-proj-janela-de-alberti-01}}\label{\detokenize{GE301-3:id3}}\end{figure}

Suponha que a cena seja composta por um cubo, como no caso da \hyperref[\detokenize{GE301-3:fig-proj-janela-de-alberti-03}]{Figura \ref{\detokenize{GE301-3:fig-proj-janela-de-alberti-03}}}. Cada ponto do cubo está emitindo um raio luminoso para o olho do observador. Ao posicionar o quadro frente à cena, basta então desenharmos os pontos de interseção destes raios emitidos pelo cubo com o plano do quadro. Como cada ponto de interseção do quadro está alinhado com o respectivo ponto do cubo e o olho do observador, este não notará a diferença. É como se o quadro funcionasse como uma janela para a cena, analogia esta idealizada pelo pintor renascentista italiano Leon Battista Alberti (1404-1472).

\begin{figure}[H]
\centering
\capstart

\noindent\includegraphics[width=350bp]{{observacao}.jpg}
\caption{A métafora da janela.}\label{\detokenize{GE301-3:fig-proj-janela-de-alberti-03}}\label{\detokenize{GE301-3:id4}}\end{figure}

Note que este processo de produzir um quadro que funcione como uma janela nada mais é do que uma projeção em perspectiva: o centro \(O\) é a posição do olho do observador e o plano de projeção \(\pi\) é o plano do quadro.

\begin{figure}[H]
\centering
\capstart

\noindent\includegraphics[width=280bp]{{projecao-perspectiva-03_1}.jpg}
\caption{\(P'\) é a projeção em perspectiva do ponto \(P\) sobre o plano de projeção \(\pi\) com relação ao centro \(O\).}\label{\detokenize{GE301-3:fig-proj-perspectiva-03}}\label{\detokenize{GE301-3:id5}}\end{figure}

Enquanto nos experimentos com a luz da lanterna do celular o objeto ficava “entre” o centro \(O\) e o plano \(\pi\), no caso da métafora da Janela de Alberti, o plano \(\pi\) fica entre \(O\) e o objeto. Ainda assim, as duas situações são modeladas por projeções em perspectiva.

O objeto também pode ser posicionado de modo que o centro \(O\) fique entre este e o plano de projeção, como mostra a \hyperref[\detokenize{GE301-3:fig-proj-perspectiva-06}]{Figura \ref{\detokenize{GE301-3:fig-proj-perspectiva-06}}}. Este tipo de configuração modela um terceiro tipo de situação: as \index{câmeras obscuras}câmeras obscuras, modelos básicos de câmera fotográfica sem lentes (ver \hyperref[\detokenize{GE301-3:fig-proj-kircher-01}]{Figura \ref{\detokenize{GE301-3:fig-proj-kircher-01}}}).

\begin{figure}[H]
\centering
\capstart

\noindent\includegraphics[width=280bp]{{projecao-perspectiva-06}.jpg}
\caption{\(P'\) é a projeção em perspectiva do ponto \(P\) sobre o plano de projeção \(\pi\) com relação ao centro \(O\).}\label{\detokenize{GE301-3:fig-proj-perspectiva-06}}\label{\detokenize{GE301-3:id6}}\end{figure}

Supondo que a abertura da pupila seja pequena o suficiente e ignorando-se a presença de lentes e a curvatura da retina, o olho humano também pode ser considerado como uma câmera obscura e, assim, também modelado por projeções em perspectiva. É este modelo simplificado que consideraremos neste capítulo.


\begin{figure}[H]
\centering
\capstart

\noindent\includegraphics[width=180bp]{{Descartes_Diagram_of_ocular_refraction._Wellcome_L0012003}.jpg}
\caption{Esquema do olho proposto por René Descartes em sua obra \emph{A dióptrica} (fonte: \href{http://www.revistas.usp.br/ss/article/view/11212/12980}{Revista Scientiae Studia}).}\label{\detokenize{GE301-3:fig-proj-olho-humano-01}}\label{\detokenize{GE301-3:id7}}\end{figure}

Resumindo: projeções em perspectiva modelam pinturas (quando o plano de projeção está entre o observador e o objeto), sombras (quando o objeto está entre o observador e o plano de projeção) e câmeras e modelos simplificados do olho humano (quando o observador está entre o objeto e o plano de projeção).

% \begin{figure}[H]
% \centering

% \noindent\includegraphics[width=300bp]{{projecao-cubo}.png}
% \end{figure}

\begin{reflection}

Note que uma projeção em perspectiva pode ser interpretada como uma \textbf{função} \(f\) de domínio \({\mathbb R}^{3} - \psi\) e contradomínio \(\pi\) que, a cada ponto \(P \in {\mathbb R}^{3} - \psi\), faz associar o ponto \(P'\) de interseção entre a reta \(OP\) e o plano \(\pi\), onde \(\psi\) é o plano passando por \(O\) e paralelo a \(\pi\). Assim, no contexto da \hyperref[\detokenize{GE301-3:fig-proj-perspectiva-01}]{Figura \ref{\detokenize{GE301-3:fig-proj-perspectiva-01}}}, temos que \(f(P) = P'\), \(f(A) = A'\), \(f(B) = B'\) e \(f(C) = C'\).
\end{reflection}

\subsection{Projeções paralelas}

Na \DUrole{xref,std,std-ref}{ativ-proj-modelos-de-projecao}, vamos modelar o anteparo usado nos experimentos com raios solares como um plano \(\pi\). Um objeto opaco, como o triângulo \(ABC\) na \hyperref[\detokenize{GE301-3:fig-proj-paralela-01}]{Figura \ref{\detokenize{GE301-3:fig-proj-paralela-01}}}, irá obstruir os raios do Sol, os quais estamos supondo aqui serem todos paralelos, produzindo então uma sombra sobre o plano \(\pi\). Como determinar exatamente quais pontos de \(\pi\) percentem à sombra? Para cada ponto \(P\) do triângulo \(ABC\), construa a reta que é paralela à direção dos raios do Sol. Esta reta irá intersectar o plano \(\pi\) em ponto \(P'\). Este ponto \(P'\) é um ponto da sombra do triângulo \(ABC\). De fato, todo ponto \(P'\) da sombra é obtido por este processo, isto é, um ponto \(P'\) do plano pertence à sombra do triângulo \(ABC\) se, e somente se, existe um ponto \(P\) do triângulo \(ABC\) tal que a interseção da reta que passa por \(P\) e é paralela aos raios Sol com o plano \(\pi\) é o ponto \(P'\). Além do ponto \(P'\), a \hyperref[\detokenize{GE301-3:fig-proj-paralela-01}]{Figura \ref{\detokenize{GE301-3:fig-proj-paralela-01}}} mostra também o processo para os pontos \(A'\), \(B'\) e \(C'\).

\begin{figure}[H]
\centering
\capstart

\noindent\includegraphics[width=280bp]{{projecao-paralela-01_1}.jpg}
\caption{Um modelo para o experimento com a luz do Sol.}\label{\detokenize{GE301-3:fig-proj-paralela-01}}\label{\detokenize{GE301-3:id8}}\end{figure}

Vamos agora abstrair ainda mais o processo, ou seja, vamos considerar um contexto matemático que, apesar de inspirado por luzes e sombras, será puramente geométrico.

Desta maneira, considere no espaço tridimensional \({\mathbb R}^{3}\) um plano \(\pi\) e uma direção determinada por uma reta \(d\) que não é paralela ao plano \(\pi\). Se \(P\) é um ponto qualquer, então o ponto \(P'\) de intersecção entre a reta que passa por \(P\) e é paralela à reta \(d\) e o plano \(\pi\) é denominado \index{projeção paralela}projeção paralela do ponto \(P\) com relação a direção dada pela reta \(d\) sobre o plano de projeção \(\pi\).

\begin{figure}[H]
\centering
\capstart

\noindent\includegraphics[width=260bp]{{projecao-paralela-03_1}.jpg}
\caption{\(P'\) é a projeção paralela do ponto \(P\) com relação a direção dada pela reta \(d\) sobre o plano de projeção \(\pi\).}\label{\detokenize{GE301-3:fig-proj-paralela-03}}\label{\detokenize{GE301-3:id9}}\end{figure}

Se a reta \(d\) for perpendicular ao plano \(\pi\), então a projeção paralela é denominada \index{projeção ortogonal}projeção ortogonal. Uma projeção paralela que não é ortogonal é denominada \index{projeção oblíqua}projeção oblíqua.

\begin{figure}[H]
\centering
\capstart

\noindent\includegraphics[width=260bp]{{projecao-paralela-02_2}.jpg}
\caption{\(P'\) é a projeção ortogonal do ponto \(P\) com relação a direção dada pela reta \(d\) perpendicular ao plano de projeção \(\pi\) sobre este plano.}\label{\detokenize{GE301-3:fig-proj-paralela-02}}\label{\detokenize{GE301-3:id10}}\end{figure}

\begin{observation}
As projeções paralelas definidas aqui são generalizações para o espaço \({\mathbb R}^{3}\) das projeções paralelas no plano que você estudou na \DUrole{xref,std,std-ref}{ativ-projecao-paralela} do capítulo sobre Teorema de Tales. Aqui, a projeção é em um plano e, lá, em uma reta.
\end{observation}

\begin{reflection}

Note que uma projeção paralela pode ser interpretada como uma \textbf{função} \(f\) de domínio \({\mathbb R}^{3}\) e contradomínio \(\pi\) que, a cada ponto \(P \in {\mathbb R}^{3}\), faz associar o ponto \(P'\) de interseção entre a reta que passa por \(P\) e é paralela a reta \(d\) e o plano \(\pi\), supondo que \(d\) não é paralela ao plano \(\pi\). Assim, no contexto da \hyperref[\detokenize{GE301-3:fig-proj-paralela-01}]{Figura \ref{\detokenize{GE301-3:fig-proj-paralela-01}}}, temos que \(f(P) = P'\), \(f(A) = A'\), \(f(B) = B'\) e \(f(C) = C'\).
\end{reflection}

\begin{knowledge}

Com o Renascimento (século XIV-século XVII), os artistas começaram a fazer suas pinturas com a preocupação de retratar a realidade, isto é, retratar o que se vê. Para isso, eles consideraram o uso de princípios óticos geométricos e, em particular, das projeções em perspectiva. Vários aparatos foram idealizados com o próposito de produzir imagens realistas. Observe que o princípio básico de todos os dispositivos é o alinhamento do ponto do objeto a ser retratado, do ponto projetado no quadro e um centro fixo, tipicamente, o olho do observador.

\begin{figure}[H]
\centering
\capstart
\ifnum\aluno=1
\includegraphics[width=350bp]{{durer-01}.jpg}
\else
\includegraphics[width=330bp]{{durer-01}.jpg}
\fi

\caption{Dispositivo de Albrecht Dürer (1471-1528).}\label{\detokenize{GE301-3:id11}}\end{figure}

\begin{figure}[H]
\centering
\capstart

\noindent\includegraphics[width=350bp]{{cigoli-02}.jpg}
\caption{Dispositivo de Lodovico Cardi (Cigoli) (1559-1613).}\label{\detokenize{GE301-3:id12}}\end{figure}

\begin{figure}[H]
\centering
\capstart

\noindent\includegraphics[width=250bp]{{jamnitzer-01}.jpg}
\caption{Dispositivo de Wenzel Jamnitzer (1507/1508-1585).}\label{\detokenize{GE301-3:id13}}\end{figure}

\begin{figure}[H]
\centering
\capstart

\noindent\includegraphics[width=250bp]{{schmalcalder-02}.jpg}
\caption{Dispositivo de Charles Augustus Schmalcalder (1781-1843).}\label{\detokenize{GE301-3:id14}}\end{figure}

\begin{figure}[H]
\centering
\capstart

\noindent\includegraphics[width=250bp]{{kircher-01}.jpg}
\caption{Camera obscura de Athanasius Kircher (1601-1680).}\label{\detokenize{GE301-3:fig-proj-kircher-01}}\label{\detokenize{GE301-3:id15}}\end{figure}

Existiram dispositivos renascentistas que produziam desenhos em projeções paralelas? O único que se conhece até o momento é a máquina de Johannes Lencker (1523-1585) que produzir desenhos em projeções ortogonais.

\begin{figure}[H]
\centering
\capstart

\noindent\includegraphics[width=250bp]{{lencker-01}.jpg}
\caption{Dispositivo de Johannes Lencker (1523-1585).}\label{\detokenize{GE301-3:id16}}\end{figure}

Enquanto que os pintores renascentistas procuravam fazer seus quadros retratando as pessoas como as vemos, na Idade Média essa preocupação não aparecia. No lugar de princípios óticos geométricos, as regras medievais incluiam pintar as pessoas de acordo com o seu \emph{status} social: quanto maior o \emph{status}, maior o tamanho na pintura (\hyperref[\detokenize{GE301-3:fig-proj-medieval-social-03}]{Figura \ref{\detokenize{GE301-3:fig-proj-medieval-social-03}}}, \hyperref[\detokenize{GE301-3:fig-proj-medieval-social-05}]{Figura \ref{\detokenize{GE301-3:fig-proj-medieval-social-05}}}, \hyperref[\detokenize{GE301-3:fig-proj-medieval-social-01}]{Figura \ref{\detokenize{GE301-3:fig-proj-medieval-social-01}}}).

\begin{figure}[H]
\centering
\capstart

\noindent\includegraphics[width=200bp]{{medieval-social-03}.jpg}
\caption{São Lourenço entre Santos e Patrocinadores de Fra Filippo Lippi (1406-1469).}
\label{\detokenize{GE301-3:fig-proj-medieval-social-03}}

\end{figure}

\begin{figure}[H]
\centering
\capstart

\noindent\includegraphics[width=300bp]{{medieval-social-05}.jpg}
\caption{Henrique III acompanhando o Mestre de Obras (século XIV).}\label{\detokenize{GE301-3:fig-proj-medieval-social-05}}\label{\detokenize{GE301-3:id18}}\end{figure}

\begin{figure}[H]
\centering
\capstart

\noindent\includegraphics[width=200bp]{{medieval-social-01}.jpg}
\caption{Políptico da Misericórdia de Piero della Francesca (1415-1492).}\label{\detokenize{GE301-3:fig-proj-medieval-social-01}}\label{\detokenize{GE301-3:id19}}\end{figure}

\end{knowledge}

\clearpage
\def\currentcolor{session2}
\begin{objectives}{Feixe de retas}
{
Analisar a geometria do feixe de retas usados na construção de uma projeção em perspectiva e de uma projeção paralela e concluir que estes formam um cone no caso de uma projeção em perspectiva e um cilindro no caso de uma projeção paralela.
}{1}{2}
\end{objectives}
\mspace{-1em}
\begin{sugestions}{Feixe de retas}
{
\begin{itemize}
\item {} 
Sugerimos fortemente que, para a discussão sobre esta atividade, você use com seus alunos as construções do GeoGebra 3D disponíveis nos endereços \url{https://www.geogebra.org/m/kKqT9EhP} (projeções em perspectiva) e \url{https://www.geogebra.org/m/zdw7rVHc}y (projeções paralelas). Além do círculo e do quadrado, há outras curvas não prototípicas que podem ser usadas para evidenciar o conceito mais geral de cone e cilindro.


\begin{figure}[H]
\centering
\capstart

\noindent\includegraphics[width=.45\linewidth]{{praticando-feixe-de-retas-05}.jpg}
\caption{Projeção em perspectiva de uma curva não prototípica.}\label{\detokenize{GE301-4:fig-proj-praticando-05}}\label{\detokenize{GE301-4:id1}}\end{figure}
\item {} 
Provavelmente os alunos responderão que o conjunto de retas usados para construir a projeção em perspectiva do quadrado formam uma pirâmide de base quadrada, o que está correto se, no lugar de retas usássemos, segmentos de reta. Não obstante, sugerimos que você aproveite a oportunidade para comparar a pirâmide de base quadrada com o cone circular reto e mesmo o caso da curva não prototípica da \hyperref[\detokenize{GE301-4:fig-proj-praticando-05}]{Figura \ref{\detokenize{GE301-4:fig-proj-praticando-05}}} para mostrar o que há em comum em todos os casos: as retas que passam todas por um mesmo ponto (o centro \(O\)) e passam por pontos do objeto em questão. É esta característica que dá uma definição mais geral para cones/superfícies cônicas. Assim, a pirâmide de base quadrada pode ser considerada um cone. Cabe lembrar que por este motivo, projeções em perspectiva também são denominadas \index{projeções cônicas}projeções cônicas. Observação análoga cabe para projeções paralelas: o conjunto de retas usadas para construir uma projeção paralela formam um cilindro e, por este motivo, projeções paralelas também são conhecidas por \index{projeções cilíndricas}projeções cilíndricas.

\end{itemize}
}{1}{2}
\end{sugestions}
\begin{answer}{Feixe de retas}
{
\paragraph{Projeções em Perspectiva}
\begin{enumerate}
\item Se você respondeu que a superfície obtida ao desenhar todas as retas que passam por \(O\) e por pontos do círculo é um cone, sua intuição está te levando na direção correta, mas do ponto de vista matemático precisamos ser mais cuidadosos. Se considerarmos  a união de todos os segmentos de reta \(OP'\), onde \(P\) é um ponto do círculo e \(P'\) o ponto de inteserção da reta determinada por \(O\) e \(P\) com o plano \(\pi\), teremos um cone circular reto. Portanto, como a superfície que estamos estudando não é formada por segmentos de reta, e sim por retas, ela não pode ser um cone. Mas se todos os segmentos de reta que formam o cone forem prolongados infinitamente, obteremos a superfície procurada na questão.

Você pode estar se perguntando se esta superfície possui um nome e a resposta é sim! Dados um ponto \(A\) não contido em um plano \(\pi\) e um objeto geométrico \(\gamma\) contido em \(\pi\), a superfície formada por todas as retas que passam por \(A\) e por um ponto de \(\gamma\) é chamada de \textit{superfície cônica}. Repare que, todas as retas da superfície se intersectam em \(A\). De posse desta definição, podemos afirmar que a superfície que obtemos nesta questão é uma superfície cônica gerada a partir do ponto \(O\) (equivale ao ponto \(A\) da definição) e do círculo (equivale ao objeto \(\gamma\) da definição).

\item {} 
Ao desenhar todas as retas que passam por \(O\) e por pontos do quadrado é possível que você pense na forma de uma pirâmide, mas a superfície formada por essas retas não é exatamente uma pirâmide. Se considerarmos  a união de todos os segmentos de reta \(OP'\), onde \(P\) é um ponto do quadrado e \(P'\) o ponto de inteserção da reta determinada por \(O\) e \(P\) com o plano \(\pi\), teremos uma pirâmide reta de base quadrada. Assim, como a superfície que estamos estudando não é formada por segmentos de reta, e sim por retas, ela não pode ser uma pirâmide. Mas se todos os segmentos de reta que formam a pirâmide forem prolongados infinitamente, obteremos a superfície procurada na questão.

Pelo que vimos no item anterior, a superfície encontrada neste item é também uma superfície cônica. De fato, ela é um conjunto de retas que passam pelo ponto \(O\) e por pontos do quadrado, como determina a definição apresentada anteriormente.
\end{enumerate}
}{0}
\end{answer}
\begin{answer}{Feixe de retas}
{
\paragraph{Projeções paralelas}
\begin{enumerate}
\item Ao desenhar todas as retas que passam por pontos do círculo que são paralelas a \(d\), é possível que nos remetamos à forma de um cilindro, mas a superfície obtida por esta união de retas não é exatamente um cilindro. Se considerarmos apenas a união dos segmentos de reta \(PP'\), onde \(P\) é um ponto do círculo e \(P'\) o ponto de inteserção da reta paralela à \(d\) que passa por \(P\) com o plano \(\pi\), teremos um cilindro circular reto. Na situação apresentada na atividade, como a superfície que estamos estudando não é formada por segmentos de reta, e sim por retas, ela não pode ser um cilindro. Caso todos os segmentos de reta que formam o cilindro sejam prolongados infinitamente, obteremos a superfície procurada na questão.

A superfície deste caso não pode ser chamada de superfície cônica, pois ela não foi gerada por um ponto e por um objeto como nos casos anteriores. Vamos, então, precisar de uma nova definição. Dados um objeto geométrico \(\gamma\) contido em um plano \(\pi\) e uma reta \(r\) não contida em \(\pi\), a superfície formada por todas as retas que passam por pontos de \(\gamma\) e são paralelas à \(r\) é chamada de \textit{superfície cilíndrica}. Sendo assim, a superfície que obtemos nesta questão é uma superfície cilíndrica gerada a partir do círculo (equivale ao objeto \(\gamma\) da definição) e \(d\) (equivale reta \(r\) da definição).

\item {} 
Neste caso, se desenharmos todas as retas que passam por pontos do quadrado que são paralelas a \(d\), é possível que nos lembremos da forma de um prisma, mas a superfície obtida por esta união de retas não é um prisma. Se considerarmos todos os segmentos de reta \(PP'\), onde \(P\) é um ponto do quadrado e \(P'\) o ponto de inteserção da reta paralela à \(d\) que passa por \(P\) com o plano \(\pi\), teremos um prisma reto de base quadrangular. Mas a superfície que estamos estudando não é formada por segmentos de reta, e sim por retas, logo ela não pode ser um prisma. Caso todos os segmentos de reta que formam o prisma sejam prolongados infinitamente, obteremos a superfície procurada na questão.

Pelo que vimos no item anterior, a superfície encontrada neste item é também uma superfície cilíndrica. De fato, ela é um conjunto de retas que passam por pontos do círculo e são paralelas à \(d\), como determina a definição apresentada anteriormente.
\end{enumerate}
}{9}
\end{answer}
\clearmargin
\begin{objectives}{Projetando curvas que estão sobre um cone e um cilindro}
{
Concluir que pontos diferentes em um mesmo feixe de retas associados a uma mesma projeção em perspectiva ou uma mesma projeção paralela têm a mesma projeção.
}{1}{2}
\end{objectives}
\begin{sugestions}{Projetando curvas que estão sobre um cone e um cilindro}
{
\begin{itemize}
\item {} 
A Curva 1 é um círculo, a Curva 2 é uma elipse e a Curva 3 é uma hélice.

\item {} 
Sugerimos fortemente que, para a discussão sobre esta atividade, você use com seus alunos as construções do GeoGebra 3D disponíveis nos endereços \textless{}\url{https://www.geogebra.org/m/NNjgC2Aj}\textgreater{} (cone) e \textless{}\url{https://www.geogebra.org/m/NrqMykdJ}\textgreater{} (cilindro). Além do círculo e do quadrado, há outras curvas não prototípicas que podem ser usadas para evidenciar o conceito mais geral de cone e cilindro.

\item {} 
No caso das projeções das hélices, observe para seus alunos que pontos diferentes da curva são projetados no mesmo ponto do plano \(\pi\). De fato, observe que todos os pontos de uma reta usada para obter a projeção de um ponto têm a mesma projeção do ponto. Sugerimos que você destaque esta propriedade para seus alunos. Assim, usando a terminologia de funções, segue-se que as projeções em perspectiva e projeções paralelas \textbf{não são} funções injetivas.

\end{itemize}
}{1}{2}
\end{sugestions}
\begin{answer}{Projetando curvas que estão sobre um cone e um cilindro}
{

\paragraph{Cone}
\begin{enumerate}
\item {} 
Para encontrar a projeção em perspectiva de uma curva com relação ao ponto \(O\) sobre o plano \(\pi\) é preciso traçar as retas que passam pelo ponto \(O\) e pela curva, e encontrar suas interseções com o plano \(\pi\). De fato, como as três curvas em questão estão sobre o mesmo cone, as retas que passam por \(O\) e por pontos da curva serão prologamentos dos segmentos de reta que formam o cone (aqueles que possuem extremidade em \(O\) e em seu círculo de base). Logo, a projeção em perspectiva das curvas 1, 2 e 3 com relação ao centro de projeção \(O\) sobre o plano \(\pi\) é um círculo contido no plano \(\pi\).

\item {} 
As pinturas conteriam a projeção em perspectiva das três curvas em relação à posição do olho do observador, que como vimos no item anterior é um círculo.

\item {} 
Neste caso, o objeto a ser projetado é uma reta localizada sobre o cone. Repare que as retas que passam pelo ponto \(O\) e por pontos da reta a ser projetada coincidem com ela mesma, e por isso sua interseção com o plano \(\pi\) é um ponto. Portanto, a projeção em perspectiva da reta em relação ao ponto \(O\) sobre o plano \(\pi\) é seu ponto de interseção com o plano \(\pi\).

\item {} 
As retas que passam pelo ponto \(O\) e por pontos do cone são prolongamentos dos segmentos de reta que possuem extremidades em \(O\) e em pontos da base do cone. Assim, como a interseção dessas retas com o plano \(\pi\) é um círculo, então a projeção em perspectiva do cone é um círculo.
\end{enumerate}
}{1}
\end{answer}
\clearmargin
\begin{answer}{Projetando curvas que estão sobre um cone e um cilindro}
{
\paragraph{Cilindro}
\begin{enumerate}
\item Para encontrar a projeção paralela de uma curva com relação à direção dada no plano \(\pi\) é preciso traçar as retas que passam pelos pontos da curva e são paralelas à direção dada, e então, encontrar suas interseções com \(\pi\). De fato, como as três curvas em questão estão sobre o mesmo cilindro cujo eixo é a direção de projeção escolhida, as retas que passam por seus pontos e são paralelas ao eixo do cilindro serão prologamentos dos segmentos de reta que formam o cilindro (ou seja, aqueles segmentos que possuem extremidades sobre seus círculos de base). Logo, a projeção paralela das curvas 1, 2 e 3 com relação à direção dada pelo eixo do cilindro sobre o plano \(\pi\) é o círculo contido no plano \(\pi\).

\item {} 
A projeção paralela de uma reta sobre o cilindro com relação à direção dada pelo eixo do cilindro sobre o plano \(\pi\) é um ponto. De fato, essa reta já é paralela ao eixo do cilindro, e portanto, para encontrar sua projeção basta encontrar sua interseção com o plano \(\pi\), que é um ponto.

\item {} 
A projeção paralela do cilindro com relação à direção dada pelo eixo do cilindro sobre o plano \(\pi\) é dada por um círculo, pois as retas que passam por pontos do cilindro e são paralelas ao seu eixo são prolongamentos dos segmentos de reta que o formam. Assim, a interseção dessas retas com o plano \(\pi\) é a própria base do cilindro.
\end{enumerate}
}{1}
\end{answer}
\begin{objectives}{Construindo objetos geométrico peculiares}
{
Construir objetos geométricos que satisfazem certas propriedades pré-estabelecidas de interesse prático ou artístico usando, para isto, propriedades das projeções em perspectica e das projeções ortogonais.
}{1}{2}
\end{objectives}
\begin{sugestions}{Construindo objetos geométrico peculiares}
{
\begin{itemize}
\item {} 
Sugerimos fortemente que, para a discussão sobre esta atividade, você use com seus alunos as construções do GeoGebra 3D disponíveis nos endereços \textless{}\url{https://www.geogebra.org/m/X2rA45gS}\textgreater{} (projeção da sinalização de solo no para-brisa),  \textless{}\url{https://www.geogebra.org/m/xjMqSPX2}\textgreater{} (projeção do para-brisa no solo), \textless{}\url{https://www.geogebra.org/m/Uxtn6hxy}\textgreater{} (peça “Squaring The Circle”), \textless{}\url{https://www.geogebra.org/m/Q7eXY36j}\textgreater{} (problema dos buracos).

\item {} 
Estes vídeos \url{https://youtu.be/pNjh1Ji\_rg8}\textgreater{} e \url{https://youtu.be/2xtA-IABcP4}\textgreater{} explicam, respectivamente, com o uso do GeoGebra para celulares, a construção da peça “Squaring The Circle” e da solução do problema da mesa com  três buracos.

\notasfig{\begin{figure}[H]
\centering

\noindent\includegraphics[width=300bp]{{problema-dos-tres-buracos}.png}
\end{figure}}

\item {} 
Observe para seus alunos como o conectivo lógico “e” se associa com interseções: no “Squaring The Circle”, a imagem de um ponto de vista deve ser um círculo “e” de outro ponto de vista deve ser um quadrado. A solução é então obtida pela interseção de dois cones (um de base circular e o outro de base quadrada).

\item {} 
Para que a palavra “ESCOLA” apareça sem distorções no para-brisa, seu desenho no chão, além de ter uma altura mais esticada, é também mais larga da parte de cima, como mostra a figura a seguir. Com isto, se a parte de baixo tem uma largura próxima a largura da rua, a parte de cima teria que ser pintada na calçada, o que não é viável. Por este motivo, em geral, essas sinalizações são pintadas com largura constante, mas altura bem esticada. A imagem vista no para-brisa então ainda mostrará distorções, mas será mais legível. O documento “Sinalização Horizontal” do Departamento Nacional de Infraestrutura de Transportes (DNIT), disponível no endereço \textless{}\url{https://goo.gl/CTTyaE}\textgreater{}, apresenta no Apêndice D o formato exatado de como as letras e números devem ser desenhados.


\notasfig{\begin{figure}[H]
\centering

\noindent\includegraphics[width=.7\linewidth]{{aviso-na-rua-03}.jpg}

\noindent\includegraphics[width=.7\linewidth]{{aviso-na-rua-04}.jpg}
\end{figure}}

\end{itemize}
}{1}{1}
\end{sugestions}
\begin{answer}{Construindo objetos geométrico peculiares}
{
\begin{enumerate}
\item {} 
\begin{enumerate}
\item {} 
Para que a palavra “ESCOLA” seja vista por um motorista pelo para-brisa de seu carro sem distorções, é preciso que ela seja feita utilizando projeção em perspectiva. Para isso, o centro de projeção deve coincidir com a posição dos olhos do motorista, o objeto a ser projetado deve ser a própria palavra “ESCOLA” (escrita em sua forma regular sem distorções) que deve ser posicionada a uma curta distância do motorista e o plano de projeção será uma parte do solo à frente do carro. Para encontrar a projeção, basta traçar retas que passam pela posição dos olhos do motorista e por pontos das letras da palavra “ESCOLA”, e determinar suas interseções com o plano do solo, como mostrado na \hyperref[\detokenize{GE301-4:fig-proj-obj-peculiares-solucao-escola-01}]{Figura \ref{\detokenize{GE301-4:fig-proj-obj-peculiares-solucao-escola-01}}}.


\begin{figure}[H]
\centering
\capstart

\noindent\includegraphics[width=320bp]{{ProjecaoEscola}.png}
\caption{Construção da projeção da palavra “ESCOLA”.}\label{\detokenize{GE301-4:fig-proj-obj-peculiares-solucao-escola-01}}\label{\detokenize{GE301-4:id6}}\end{figure}

Observe que, é possível escolher pontos mais representativos das letras da palavra “ESCOLA” para serem projetados (os extremos das letras, os pontos de interseção entre os segmentos de reta que formam a letra e etc) e, então, ligar suas projeções já no solo para encontrar a perspectiva desejada. Certamente a parte superior da palavra “ESCOLA” será mais larga que a inferior, e ela deverá estar mais esticada do que a palavra em sua forma regular. Veja o resultado final na figura abaixo.

\begin{figure}[H]
\centering
\capstart

\noindent\includegraphics[width=.7\linewidth]{{aviso-na-rua-03}.jpg}
\caption{Resultado final da projeção da palavra “ESCOLA”.}\label{\detokenize{GE301-4:fig-proj-obj-peculiares-solucao-escola-02}}\label{\detokenize{GE301-4:id7}}\end{figure}

A construção mostrada na \hyperref[\detokenize{GE301-4:fig-proj-obj-peculiares-solucao-escola-01}]{Figura \ref{\detokenize{GE301-4:fig-proj-obj-peculiares-solucao-escola-01}}} foi feita no Geogebra 3D e está disponível em \textless{}\url{https://www.geogebra.org/m/X2rA45gS}\textgreater{} e \textless{}\url{https://www.geogebra.org/m/xjMqSPX2}\textgreater{}.
\item {} 
Não. O desenho da palavra “ESCOLA” foi feito utilizando projeção em perspectiva em relação à posição dos olhos do motorista. Neste caso, a pintura parecerá distorcida sempre que o carro não estiver posicionado exatamente nesta posição fixada.

\end{enumerate}

\item {} 
Para construir esta peça é preciso que suas projeções em perspectiva ao variar o centro e plano de projeção sejam um círculo e um quadrado.


\begin{figure}[H]
\centering
\capstart

\noindent\includegraphics[width=.7\linewidth]{{ConstrucaoDaPeca}.png}
\caption{Construção da obra “Squaring The Circle”. A primeira figura é o cone com vértice \(C\), a segunda é a pirâmide de base quadrada com vértice \(P\), a terceira é a união do cone com a pirâmide, e a quarta é a interseção entre o cone e a pirâmide (\(C\) e \(P\) foram mantidos para auxiliar na compreensão, mas não fazem parte da interseção).}\label{\detokenize{GE301-4:fig-proj-obj-peculiares-solucao-peca-01}}\label{\detokenize{GE301-4:id8}}\end{figure}

A última figura mostrada na \hyperref[\detokenize{GE301-4:fig-proj-obj-peculiares-solucao-peca-01}]{Figura \ref{\detokenize{GE301-4:fig-proj-obj-peculiares-solucao-peca-01}}} é a interseção do cilindro com a pirâmide. Na \hyperref[\detokenize{GE301-4:fig-proj-obj-peculiares-solucao-peca-02}]{Figura \ref{\detokenize{GE301-4:fig-proj-obj-peculiares-solucao-peca-02}}}, a mesma peça pode ser vista de outro ângulo.

\begin{figure}[H]
\centering
\capstart

\noindent\includegraphics[width=90bp]{{Peca_1}.png}
\caption{Peça da obra “Squaring The Circle”.}\label{\detokenize{GE301-4:fig-proj-obj-peculiares-solucao-peca-02}}\label{\detokenize{GE301-4:id9}}\end{figure}

O vídeo \textless{}\url{https://youtu.be/pNjh1Ji\_rg8}\textgreater{} apresenta a construção da peça “Squaring The Circle”. E na construção feita no Geogebra 3D disponível em \textless{}\url{https://www.geogebra.org/m/Uxtn6hxy}\textgreater{}, é possível interagir com a peça criada anteriormente.
\end{enumerate}
}{9}
\end{answer}
\clearmargin
\begin{answer}{Construindo objetos geométrico peculiares}
{
\begin{enumerate}\setcounter{enumi}{2}
\item {} 
Sim, é possível construir uma rolha que possa ser usada para tapar os três buracos da \hyperref[\detokenize{GE301-4:fig-proj-cork-plug-01}]{Figura \ref{\detokenize{GE301-4:fig-proj-cork-plug-01}}}, mas para isso é necessário que, dependendo de seu posicionamento em relação a um plano de projeção, sua projeção paralela possa ser um quadrado, um círculo e um triângulo como os dos buracos.


Observe que se construíssemos um prisma reto utilizando o quadrado do primeiro buraco como base, sua projeção paralela poderia ser o quadrado. Um cilindro reto com o círculo do segundo buraco como base teria como projeção paralela um círculo. E, um prisma triangular com base igual ao triângulo do terceiro buraco forneceria o triângulo desejado como projeção paralela. Veja a \hyperref[\detokenize{GE301-4:fig-proj-obj-peculiares-solucao-rolha-01}]{Figura \ref{\detokenize{GE301-4:fig-proj-obj-peculiares-solucao-rolha-01}}}.

\begin{figure}[H]
\centering
\capstart

\noindent\includegraphics[width=.7\linewidth]{{Solidos_1}.png}
\caption{Imagem dos três sólidos que devem ser criados para a construção da rolha.}\label{\detokenize{GE301-4:fig-proj-obj-peculiares-solucao-rolha-01}}\label{\detokenize{GE301-4:id10}}\end{figure}

Agora, é preciso “juntar” estes três sólidos para formar uma única rolha. Para isso, basta intersectá-los de forma que seus eixos sejam todos perpendiculares entre sim. A interseção dos três sólidos construídos terá as três projeções paralelas desejadas e assim, se encaixará nos três buracos.

\notasfig{\begin{figure}[H]
\centering
\capstart

\noindent\includegraphics[width=.7\linewidth]{{IntersecoesPrismasCilindro_1}.png}
\caption{Interseção dos três sólidos e a rolha.}\label{\detokenize{GE301-4:fig-proj-obj-peculiares-solucao-rolha-02}}\label{\detokenize{GE301-4:id11}}\end{figure}}

Para entender melhor o processo descrito acima, use a construção do Geogebra 3D disponível no endereço \url{https://www.geogebra.org/m/Q7eXY36j}\textgreater{} e o vídeo disponível em \url{https://youtu.be/2xtA-IABcP4}.

Observe que se construíssemos um prisma reto utilizando o quadrado do primeiro buraco como base, sua projeção paralela poderia ser o quadrado. Um cilindro reto com o círculo do segundo buraco como base teria como projeção paralela um círculo. E, um prisma triangular com base igual ao triângulo do terceiro buraco forneceria o triângulo desejado como projeção paralela. Agora, é preciso “juntar” estes três sólidos para formar uma única rolha. Para isso, basta intersectá-los de forma que seus eixos sejam todos perpendiculares entre sim. A interseção dos três sólidos construídos terá as três projeções paralelas desejadas e assim, se encaixará nos três buracos.
\end{enumerate}
}{1}
\end{answer}


\practice{}
\label{\detokenize{GE301-4::doc}}\label{\detokenize{GE301-4:praticando-1}}\phantomsection\label{\detokenize{GE301-4:ativ-proj-feixe-de-retas}}
\begin{task}{Feixe de retas}

\paragraph{Projeções em Perspectiva}
\begin{enumerate}


\item {} 
Na figura a seguir, (1) a reta que passa pelo ponto \(O\) e o centro do círculo é perpendicular ao plano \(\pi\) e (2) o círculo é paralelo a \(\pi\). Como vimos, para determinar a projeção em perspectiva do círculo com relação ao centro \(O\) sobre o plano de projeção \(\pi\), é necessário construir retas que passam por \(O\) e por pontos do círculo. Se desenharmos todas estas retas, que tipo de superfície será obtida?


\begin{figure}[H]
\centering

\noindent\includegraphics[width=300bp]{{praticando-feixe-de-retas-01}.jpg}
\end{figure}
\item {} 
Na figura a seguir, (1) a reta que passa pelo ponto \(O\) e o centro do quadrado é perpendicular ao plano \(\pi\) e (2) o quadrado é paralelo a \(\pi\). Como vimos, para determinar a projeção em perspectiva do quadrado com relação ao centro \(O\) sobre o plano de projeção \(\pi\), é necessário construir retas que passam por \(O\) e por pontos do quadrado. Se desenharmos todas estas retas, que tipo de superfície será obtida?


\begin{figure}[H]
\centering

\noindent\includegraphics[width=300bp]{{praticando-feixe-de-retas-02}.jpg}
\end{figure}
\end{enumerate}


\paragraph{Projeções paralelas}
\begin{enumerate}



\item {} 
Na figura a seguir, (1) a reta \(d\) que passa pelo centro do círculo é perpendicular ao plano \(\pi\) e (2) o círculo é paralelo a \(\pi\). Como vimos, para determinar a projeção paralela do círculo com relação a direção dada por \(d\) sobre o plano de projeção \(\pi\), é necessário construir retas que passam por pontos do círculo e que são paralelas a \(d\). Se desenharmos todas estas retas, que tipo de superfície será obtida?

\begin{figure}[H]
\centering

\noindent\includegraphics[width=300bp]{{praticando-feixe-de-retas-03}.jpg}
\end{figure}

\item {} 
Na figura a seguir, (1) a reta \(d\) que passa pelo centro do quadrado é perpendicular ao plano \(\pi\) e (2) o quadrado é paralelo a \(\pi\). Como vimos, para determinar a projeção paralela do quadrado com relação a direção dada por \(d\) sobre o plano de projeção \(\pi\), é necessário construir retas que passam por pontos do quadrado e que são paralelas a \(d\). Se desenharmos todas estas retas, que tipo de superfície será obtida?

\begin{figure}[H]
\centering

\noindent\includegraphics[width=300bp]{{praticando-feixe-de-retas-04}.jpg}
\end{figure}


\end{enumerate}
\end{task}

\phantomsection\label{\detokenize{GE301-4:ativ-proj-cone-cilindro}}
\begin{task}{Projetando curvas que estão sobre um cone e um cilindro}

\paragraph{Cone}

As três imagens a seguir exibem três curvas diferentes, mas que possuem uma característica em comum: elas estão sobre um mesmo cone circular reto cuja base é paralela ao plano \(\pi\). Para sua comodidade, em cada imagem, a curva é desenhada sem e com o cone. Caso tenha acesso a Internet (inclusive de um celular), você pode interagir com essas curvas e visualizá-las de pontos de vista diferentes por meio do aplicativo GeoGebra disponível em: \textless{}\url{https://www.geogebra.org/m/NNjgC2Aj}\textgreater{}.


\begin{figure}[H]
\centering

\noindent\includegraphics[width=250bp]{{perspectiva-varios-01}.jpg}
\end{figure}

\begin{figure}[H]
\centering

\noindent\includegraphics[width=250bp]{{perspectiva-varios-02}.jpg}
\end{figure}


\begin{figure}[H]
\centering

\noindent\includegraphics[width=250bp]{{perspectiva-varios-03}.jpg}
\end{figure}

\begin{enumerate}

\item {} 
Qual é a projeção em perspectiva destas três curvas sobre o plano \(\pi\) com relação ao centro \(O\)? Justifique sua resposta!

\item {} 
Usando a analogia de pintura que funciona como uma janela (conforme o que vimos com relação à \hyperref[\detokenize{GE301-3:fig-proj-janela-de-alberti-01}]{Figura \ref{\detokenize{GE301-3:fig-proj-janela-de-alberti-01}}} e à \hyperref[\detokenize{GE301-3:fig-proj-janela-de-alberti-03}]{Figura \ref{\detokenize{GE301-3:fig-proj-janela-de-alberti-03}}}), se você pintasse um quadro para cada uma das três curvas, tendo o ponto \(O\) como a posição do olho do observador, o que seria pintado nos três quadros?

\item {} 
Qual é a projeção em perspectiva de uma reta que passa por \(O\) sobre o cone com relação ao centro \(O\) sobre o plano \(\pi\)? Justifique sua resposta!

\item {} 
Qual é a projeção em perspectiva do próprio cone com relação ao centro \(O\) sobre o plano \(\pi\)? Justifique sua resposta!
\end{enumerate}

\paragraph{Cilindro}

As três imagens a seguir exibem três curvas diferentes, mas que possuem uma característica em comum: elas estão sobre um mesmo cilindro circular reto cuja base é paralela ao plano \(\pi\). Para sua comodidade, em cada imagem, a curva é desenhada sem e com o cilindro. Caso tenha acesso a Internet (inclusive de um celular), você pode interagir com essas curvas e visualizá-las de pontos de vista diferentes por meio do aplicativo GeoGebra disponível em: \url{https://www.geogebra.org/m/NrqMykdJ}.

\begin{figure}[H]
\centering

\noindent\includegraphics[width=250bp]{{paralela-varios-01}.jpg}
\end{figure}

\begin{figure}[H]
\centering

\noindent\includegraphics[width=250bp]{{paralela-varios-02}.jpg}
\end{figure}
\begin{figure}[H]
\centering

\noindent\includegraphics[width=250bp]{{paralela-varios-03}.jpg}
\end{figure}
\begin{enumerate}[itemsep=0pt]
\item {} 
Qual é a projeção paralela destas três curvas com relação à direção dada pelo eixo do cilindro sobre o plano \(\pi\)? Justifique sua resposta!

\item {} 
Qual é a projeção paralela de uma reta sobre o cilindro com relação à direção dada pelo eixo do cilindro sobre o plano \(\pi\)? Justifique sua resposta!

\item {} 
Qual é a projeção paralela do próprio cilindro com relação à direção dada pelo eixo do cilindro sobre o plano \(\pi\)? Justifique sua resposta!
\end{enumerate}
\end{task}

\phantomsection\label{\detokenize{GE301-4:ativ-proj-construindo}}
\begin{task}{Construindo objetos geométricos peculiares}

\begin{enumerate}
\item {} \begin{enumerate}[itemsep=0pt]
\item {} 
Deseja-se pintar a palavra “ESCOLA” em uma rua para advertir os motoristas da proximidade de uma escola. Contudo, se a palavra for pintada normalmente, como na \hyperref[\detokenize{GE301-4:fig-proj-aviso-na-rua-01}]{Figura \ref{\detokenize{GE301-4:fig-proj-aviso-na-rua-01}}} (B), o motorista verá pelo para-brisa uma imagem distorcida pela perspectiva, como na \hyperref[\detokenize{GE301-4:fig-proj-aviso-na-rua-01}]{Figura \ref{\detokenize{GE301-4:fig-proj-aviso-na-rua-01}}} (C).


\begin{figure}[H]
\centering
\capstart
\ifnum\aluno=1
\noindent\includegraphics[width=330bp]{{aviso-na-rua-01_4}.jpg}
\else
\noindent\includegraphics[width=330bp]{{aviso-na-rua-01_4}.jpg}
\fi
\caption{Estudo de sinalização de solo em uma rua.}\label{\detokenize{GE301-4:fig-proj-aviso-na-rua-01}}\label{\detokenize{GE301-4:id2}}\end{figure}

Como deveria ser pintada a palavra na rua para que, vista pelo para-brisa de um carro, ela fosse visualizada sem distorções, como na \hyperref[\detokenize{GE301-4:fig-proj-aviso-na-rua-02}]{Figura \ref{\detokenize{GE301-4:fig-proj-aviso-na-rua-02}}}. Aqui, é suficiente que você descreva um procedimento de como obter o desenho da palavra na rua: você não precisa efetivamente fazer o desenho da palavra.

\begin{figure}[H]
\centering
\capstart

\noindent\includegraphics[width=350bp]{{aviso-na-rua-02}.jpg}
\caption{Imagem no para-brisa sem distorções.}\label{\detokenize{GE301-4:fig-proj-aviso-na-rua-02}}\label{\detokenize{GE301-4:id3}}\end{figure}
\item {} 
O desenho da palavra que você propôs para ser pintada na rua no item anterior seria vista \textbf{sempre} sem distorções a medida que o carro se aproxima da palavra pintada?

\end{enumerate}
\item {} 
O grupo Troika tem como missão “desenvolver obras artísticas com um interesse particular na percepção e experiência espacial, desafiando prescrições de conhecimento, controle, e o que significa ser humano na era da tecnologia”. A obra “Squaring The Circle” (Quadratura do Círculo) é uma peça feita de ferro que, quando observada de um ponto de vista particular, o que se vê é um círculo e, a mesma peça, quando observada de outro ponto de vista, se mostra como um quadrado.


\begin{figure}[H]
\centering
\capstart

\noindent\includegraphics[width=\linewidth]{{gif_1}.jpg}
\caption{Squaring The Circle (Quadratura do Círculo) do grupo Troika (fonte: \url{http://troika.uk.com}.}\label{\detokenize{GE301-4:id4}}\end{figure}


\clearpage
Como construir uma tal peça? Aqui, é suficiente que você descreva um procedimento matemático de como obtê-la: você não precisa explicitar equações para o formato geométrico da peça.

\item {} 
Este é um desafio antigo e que apareceu na edição de agosto de 1958 da revista Scientific American. A \hyperref[\detokenize{GE301-4:fig-proj-cork-plug-01}]{Figura \ref{\detokenize{GE301-4:fig-proj-cork-plug-01}}} exibe uma mesa com três buracos: um na forma de um quadrado, o outro na forma de um círculo e o terceiro na forma de um triângulo isósceles. O diâmetro do círculo, o lado do quadrado, a base do triângulo isósceles e sua respectiva altura têm a mesma medida.

\begin{figure}[H]
\centering
\capstart

\noindent\includegraphics[width=300bp]{{cork-plug-table}.png}
\caption{Uma mesa com três buracos.}\label{\detokenize{GE301-4:fig-proj-cork-plug-01}}\label{\detokenize{GE301-4:id5}}\end{figure}

Pergunta: é possível construir uma rolha que possa ser usada para tapar qualquer um dos três buracos, um por vez? Em caso afirmativo, descreva um procedimento matemático de como obtê-la.

\end{enumerate}
\end{task}

\begin{reflection}

\begin{figure}[H]
\centering

\noindent\includegraphics[width=400bp]{{tirinha1}.jpg}
\end{figure}


\begin{figure}[H]
\centering

\noindent\includegraphics[width=350bp, trim={0 8cm 0 0}, clip]{{tirinha2}.jpg}
\end{figure}

No dicionário, a palavra "prisma" tem o sentido figurado de "modo de ver ou considerar algo, ponto de vista". 

Contudo, o modo que vemos está mais relacionado com projeções em perspectiva as quais, por sua vez, estão relacionadas em cones e pirâmides (pirâmides podem ser consideradas como um tipo especial de cone).

Assim, pelo menos do ponto de vista da teoria das projeções, falar "ver deste cone" no lugar de ver "ver deste prisma" faz mais sentido!
\end{reflection}
\clearpage
\clearmargin
\clearmargin
\clearmargin
\begin{objectives}{Retas paralelas, obras de arte e o infinito}
{
\begin{itemize}
\item {} 
Compreender a propriedade de que prolongamentos de projeções em perspectiva de retas paralelas que não são paralelas ao plano de projeção se encontram em um ponto.

\item {} 
Compreender a propriedade de que a projeção paralela de um feixe de retas paralelas que não são paralelas à direção da projeção são retas paralelas.

\item {} 
Identificar a existência de elementos de projeções em perspectiva e projeções paralelas em fotografias e pinturas históricas.

\end{itemize}
}{1}{1}
\end{objectives}
\begin{sugestions}{Retas paralelas, obras de arte e o infinito}
{
\begin{itemize}
\item {} 
O desenvolvimento e os questionamentos das PARTES 1 e 2 constituem um excelente exercício de geometria de posição.

\item {} 
Esta atividade é uma oportunidade do aluno ter contato com a noção de infinito. De fato, projeções em perspectiva permitem “materializar” o infinito em uma direção: no plano de projeção, ele pode ser identificado com o ponto de fuga associado à direção.

\item {} 
Na PARTE 1, supõe-se que \(O \not\in \pi\) a fim de se evitar uma projeção em perspectiva degenerada pois, se \(O \in \pi\), então a imagem da projeção em perspectiva é \(\{O\}\).

\item {} 
Sugerimos fortemente que, para a discussão da PARTE 1, você use com seus alunos a construção do GeoGebra 3D disponível no endereço \url{https://www.geogebra.org/m/pcx56y49}. Ela implementa os cinco passos da \hyperref[\detokenize{GE301-5:fig-proj-infinito-02}]{Figura \ref{\detokenize{GE301-5:fig-proj-infinito-02}}} à \hyperref[\detokenize{GE301-5:fig-proj-infinito-06}]{Figura \ref{\detokenize{GE301-5:fig-proj-infinito-06}}}, com a vantagem de se poder girar a cena e se poder escolher direções diferentes para a reta \(r\).


\end{itemize}
}{1}{1}
\end{sugestions}
\begin{sugestions}{Retas paralelas, obras de arte e o infinito}
{
\begin{itemize}
\item {} 
Na discussão das PARTES 1 e 2 é interessante relacionar o que foi estabelecido sobre projeções de retas paralelas com a projeção dos lados paralelos do cubo vazado na \DUrole{xref,std,std-ref}{ativ-proj-luz-e-sombras}.

\item {} 
Para a PARTE 3, sugerimos que os alunos sejam divididos em grupos de cinco e que sejam distribuídas pelo menos uma imagem desenhada em perspectiva (como a pintura \hyperref[\detokenize{GE301-5:fig-proj-ponto-de-fuga-01}]{Figura \ref{\detokenize{GE301-5:fig-proj-ponto-de-fuga-01}}} ou a foto \hyperref[\detokenize{GE301-5:fig-proj-ponto-de-fuga-02}]{Figura \ref{\detokenize{GE301-5:fig-proj-ponto-de-fuga-02}}}), outra desenhada em projeção paralela (como a \hyperref[\detokenize{GE301-5:fig-proj-ponto-de-fuga-03}]{Figura \ref{\detokenize{GE301-5:fig-proj-ponto-de-fuga-03}}}), outra desenhada em “perspectiva inversa” (como a \hyperref[\detokenize{GE301-5:fig-proj-ponto-de-fuga-04}]{Figura \ref{\detokenize{GE301-5:fig-proj-ponto-de-fuga-04}}}) e, finalmente, uma que parece estar desenhada em perspectiva, mas não está  (como a \hyperref[\detokenize{GE301-5:fig-proj-ponto-de-fuga-05}]{Figura \ref{\detokenize{GE301-5:fig-proj-ponto-de-fuga-05}}}).

\item {} 
Muitos desenhos japoneses antigos, a exemplo da \hyperref[\detokenize{GE301-5:fig-proj-ponto-de-fuga-03}]{Figura \ref{\detokenize{GE301-5:fig-proj-ponto-de-fuga-03}}}, foram feitos com projeções paralelas, uma característica da cultura da época. Você pode encontrar outros desenhos japoneses nos seguintes endereços: \url{https://ukiyo-e.org/} e \url{http://www.jaodb.com/}. Ainda no contexto cultural, iconografias bizantinas e russas antigas compartilham a “perspectiva inversa” da \hyperref[\detokenize{GE301-5:fig-proj-ponto-de-fuga-04}]{Figura \ref{\detokenize{GE301-5:fig-proj-ponto-de-fuga-04}}}.

\item {} 
Versões interativas feitas no GeoGebra da \hyperref[\detokenize{GE301-5:fig-proj-ponto-de-fuga-01}]{Figura \ref{\detokenize{GE301-5:fig-proj-ponto-de-fuga-01}}} à \hyperref[\detokenize{GE301-5:fig-proj-ponto-de-fuga-05}]{Figura \ref{\detokenize{GE301-5:fig-proj-ponto-de-fuga-05}}} estão disponíveis neste endereço: \url{https://www.geogebra.org/m/kFpnVERB}. O botão “S” exibe a solução e o botão “I” posiciona as retas para a configuração inicial. Com este recurso computacional, economiza-se papel que seria necessário para imprimir as imagens necessárias para a PARTE 3.
\end{itemize}
}{1}{2}
\end{sugestions}
\clearmargin
\clearmargin
\begin{answer}{Retas paralelas, obras de arte e o infinito}
{
\paragraph{Parte 1}
\textbf{Pergunta 1.} Suponhamos que \(F\) seja um ponto da projeção da reta \(r\) em \(\pi\). Isto significa que existe um ponto \(P\in r\) tal que a reta \(t\) que passa por \(O\) e \(P\) intersecta \(\pi\) em \(F\). Logo, \(t\) e \(s\) contêm os pontos \(O\) e \(F\), o que implica que \(t=s\). Observe, então, que \(s\) e \(r\) contêm o ponto \(O\), o que mostra que \(s\) e \(r\) são concorrentes. Como, por hipótese, as retas \(s\) e \(r\) são paralelas, temos um absurdo. Portanto, concluímos que o ponto \(F\) não pertence à projeção da reta \(r\).

\textbf{Pergunta 2.} Podemos perceber que o que torna \(F\) um ponto “especial” ao se calcular a projeção da reta \(r\) é o fato dele ser o ponto de interseção da reta \(s\) com o plano \(\pi\), sendo \(s\) a reta paralela à \(r\) passando por \(O\). Se, então, tomarmos uma reta qualquer paralela à reta \(r\), ela continuará sendo paralela à reta \(s\) que intersecta \(\pi\) em \(F\). E daí, da mesma forma que anteriormente, \(F\) não poderá fazer parte da projeção da nova reta.
}{1}
\end{answer}
\clearmargin
\begin{answer}{Retas paralelas, obras de arte e o infinito}
{
\textbf{Pergunta 3.} Se \(O\) fosse um ponto da reta \(r\), a projeção de \(r\) seria apenas o ponto de interseção da reta \(r\) com o plano \(\pi\).

\textbf{Pergunta 4.} A projeção em perspectiva de um feixe de retas paralelas que são paralelas ao plano de projeção é também um feixe de retas paralelas. Veja a figura abaixo.

\begin{figure}[H]
\centering
\capstart

\noindent\includegraphics[width=.5\linewidth]{{AtRetasParalObrasArteInfinito_2_1}.png}
\caption{Retas \(r\) e \(s\) paralelas entre si e ao plano de projeção \(\pi\), e suas projeções em perspectiva \(r'\) e \(s'\), respectivamente.}\label{\detokenize{GE301-5:fig-atretasparalobrasarteinfinitopar1per4}}\label{\detokenize{GE301-5:id14}}
\end{figure}

Para conjecturar, vamos trabalhar apenas com duas retas \(r\) e \(s\) paralelas entre si e ao plano de projeção para facilitar nosso raciocínio que pode ser generalizado para um número qualquer de retas. Primeiramente, note que a projeção em perspectiva de uma reta paralela ao plano de projeção será também uma reta. Portanto, as projeções das retas \(r\) e \(s\) serão duas retas que chamaremos de \(r'\) e \(s'\), respectivamente, e que suporemos serem distintas. Caso \(r'\) e \(s'\) não fossem retas paralelas, elas seriam concorrentes em um ponto \(P\) (como as duas retas estão dentro do mesmo plano, que é o plano de projeção, elas não poderiam ser reversas). Neste caso, existiriam dois pontos diferentes \(A\) e \(B\) pertencentes às retas paralelas \(r\) e \(s\), respectivamente, que foram projetados em \(P\). Isso significaria que as retas que passam pelo centro de projeção e por \(A\) e \(B\) se encontram em \(P\), o que não é possível pois \(A\) e \(B\) pertencem à retas paralelas. Logo, a projeção de duas retas paralelas ao plano de projeção deve ser duas retas paralelas.

Podemos pensar de outra forma também que talvez você considere de mais fácil entendimento. Se \(r'\) é a projeção de \(r\) no plano de projeção, então \(r'\) é paralela a \(r\). Do mesmo modo, \(s'\) é paralela a \(s\). Mas, então \(r'\) é paralela a \(s'\), porque \(r'\) é paralela a \(r\), \(r\) é paralela a \(s\) e \(s\) é paralela a \(s'\).
}{1}
\end{answer}
\clearmargin
\begin{answer}{Retas paralelas, obras de arte e o infinito}
{\paragraph{Parte 2}


Como percebemos no caso das projeções em perspectiva que retas paralelas ao plano de projeção e não paralelas possuem comportamentos diferentes ao serem projetadas, vamos estudar os dois casos em separado aqui também.  Além disso, vamos também considerar que a direção de projeção pode ou não ser perpendicular ao plano de projeção.

\textit{Retas paralelas entre si e não paralelas ao plano de projeção}

Primeiramente, vamos tomar duas retas \(r\) e \(s\) paralelas entre si, mas não paralelas ao plano de projeção. Além disso, considere a direção de projeção \(d\) oblíqua em relação ao plano de projeção \(\pi\), como exibido na \hyperref[\detokenize{GE301-5:fig-atretasparalobrasarteinfinitopar2-1}]{Figura \ref{\detokenize{GE301-5:fig-atretasparalobrasarteinfinitopar2-1}}}.

\begin{figure}[H]
\centering
\capstart

\noindent\includegraphics[width=300bp]{{AtRetasParalObrasArteInfinito_3_2}.png}
\caption{Retas \(r\) e \(s\) paralelas entre si e \(d\) a direção de projeção oblíqua em relação ao plano de projeção \(\pi\).}\label{\detokenize{GE301-5:fig-atretasparalobrasarteinfinitopar2-1}}\label{\detokenize{GE301-5:id15}}\end{figure}

Para determinar a projeção paralela da reta \(r\) em relação à \(d\) sobre \(\pi\), devemos, de acordo com a definição, para cada ponto de \(r\), determinar a interseção da reta que passa pelo ponto e é paralela à \(d\) com o plano \(\pi\). Tome os pontos \(A\) e \(B\) pertencentes à reta \(r\), como na \hyperref[\detokenize{GE301-5:fig-atretasparalobrasarteinfinitopar2-2}]{Figura \ref{\detokenize{GE301-5:fig-atretasparalobrasarteinfinitopar2-2}}}. As projeções dos pontos \(A\) e \(B\) serão chamadas \(A'\) e \(B'\), respectivamente.

\begin{figure}[H]
\centering
\capstart

\noindent\includegraphics[width=300bp]{{AtRetasParalObrasArteInfinito_4_3}.png}
\caption{Projeções paralelas \(A'\) e \(B'\) dos pontos \(A\) e \(B\) da reta \(r\) em relação à direção \(d\) sobre o plano \(\pi\).}\label{\detokenize{GE301-5:fig-atretasparalobrasarteinfinitopar2-2}}\label{\detokenize{GE301-5:id16}}\end{figure}

Observe que o ponto \(B\) é a interseção da reta \(r\) com o plano \(\pi\) e por isso, sua projeção coincide com ele mesmo. Projetando todos os pontos de \(r\), encontramos a reta \(r'\) que passa por \(A'\) e  \(B'\), conforme \hyperref[\detokenize{GE301-5:fig-atretasparalobrasarteinfinitopar2-3}]{Figura \ref{\detokenize{GE301-5:fig-atretasparalobrasarteinfinitopar2-3}}}. Assim, \(r'\) é a projeção paralela da reta \(r\) em relação à \(d\) sobre o plano \(\pi\).

\begin{figure}[H]
\centering
\capstart

\noindent\includegraphics[width=300bp]{{AtRetasParalObrasArteInfinito_6_1}.png}
\caption{A reta \(r'\) é a projeção paralela da reta \(r\) em relação à direção \(d\) sobre o plano \(\pi\).}\label{\detokenize{GE301-5:fig-atretasparalobrasarteinfinitopar2-3}}\label{\detokenize{GE301-5:id17}}\end{figure}

Afim de aplicar novamente a definição de projeção paralela à reta \(s\), vamos tomar os pontos \(C\) e \(D\) de \(s\). As retas que passam por esses pontos e são paralelas à \(d\) intersectam \(\pi\) em \(C'\) e \(D'\), respectivamente. Se replicássemos esse raciocínio para todos os pontos de \(s\), teríamos que a reta \(s'\) que passa por \(C'\) e \(D'\) é a projeção paralela de \(s\) em relação à \(d\) sobre o plano \(\pi\).

A \hyperref[\detokenize{GE301-5:fig-atretasparalobrasarteinfinitopar2-4}]{Figura \ref{\detokenize{GE301-5:fig-atretasparalobrasarteinfinitopar2-4}}} exibe as retas \(r'\) e \(s'\) que são as projeções de \(r\) e \(s\) em relação à \(d\) sobre o plano \(\pi\).

\begin{figure}[H]
\centering
\capstart

\noindent\includegraphics[width=300bp]{{AtRetasParalObrasArteInfinito_5_2}.png}
\caption{\(r'\) e \(s'\) são as projeções paralelas das retas \(r\) e \(s\) em relação à direção \(d\) sobre o plano \(\pi\).}\label{\detokenize{GE301-5:fig-atretasparalobrasarteinfinitopar2-4}}\label{\detokenize{GE301-5:id18}}\end{figure}

Repare que a reta que passa pelo ponto \(B\) e é paralela à \(d\) e a reta \(r\) determinam um plano, que chamaremos de \(\pi_r\). Portanto, \(\pi_r\) é paralelo à reta \(d\) e, além disso, a forma como \(\pi_r\) foi determinado faz com que ele contenha todas as retas paralelas à \(d\) que passam por pontos de \(r\). Assim, \(r'\) é a reta de interseção de \(\pi_r\) com \(\pi\). Se pensarmos de maneira análoga para a reta \(s\), poderíamos determinar o plano \(\pi_s\) a partir da reta que passa pelo ponto \(C\) e é paralela à \(d\) e a reta \(s\). \(\pi_s\) é também paralelo à reta \(d\) e contém todas as retas paralelas à \(d\) que passam por pontos de \(s\). Dessa forma, \(s'\) é a reta de interseção de \(\pi_s\) com \(\pi\). Se \(\pi_r\) e \(\pi_s\) são dois planos paralelos à \(d\) que não possuem interseção, então eles são paralelos entre si, e isto implica que suas interseções com \(\pi\) serão paralelas. Logo, \(r'\) e \(s'\) são retas paralelas.

No caso anterior, as retas \(r\) e \(s\) são oblíquas em relação ao plano de projeção, assim como a direção de projeção \(d\). Caso \(d\) fosse perpendicular à \(\pi\), não haveria nenhuma alteração na projeção das retas, ou seja, as projeções paralelas de retas paralelas entre si e oblíquas à \(\pi\) também seriam retas paralelas entre si. Veja a figura \hyperref[\detokenize{GE301-5:fig-atretasparalobrasarteinfinitopar2-5}]{Figura \ref{\detokenize{GE301-5:fig-atretasparalobrasarteinfinitopar2-5}}}.

\begin{figure}[H]
\centering
\capstart

\noindent\includegraphics[width=300bp]{{AtRetasParalObrasArteInfinito_13_3}.png}
\caption{A direção de projeção \(d\) é perpendicular ao plano de projeção \(\pi\), e \(r'\) e \(s'\) são as projeções paralelas das retas \(r\) e \(s\) em relação à direção \(d\) sobre o plano \(\pi\).}\label{\detokenize{GE301-5:fig-atretasparalobrasarteinfinitopar2-5}}\label{\detokenize{GE301-5:id19}}\end{figure}

Se tomarmos retas \(r\) e \(s\) paralelas entre si e perpendiculares ao plano de projeção teremos duas situações diferentes. Na primeira, caso \(d\) seja uma reta oblíqua em relação à \(\pi\), então teremos uma situação semelhante aos dois casos anteriores. Veja a figura \hyperref[\detokenize{GE301-5:fig-atretasparalobrasarteinfinitopar2-6}]{Figura \ref{\detokenize{GE301-5:fig-atretasparalobrasarteinfinitopar2-6}}}.

\begin{figure}[H]
\centering
\capstart

\noindent\includegraphics[width=300bp]{{AtRetasParalObrasArteInfinito_12_1}.png}
\caption{As retas \(r\) e \(s\) são perpendiculares ao plano de projeção \(\pi\) e \(d\) é oblíqua em relação à \(\pi\). As \(r'\) e \(s'\) são as projeções paralelas das retas \(r\) e \(s\) em relação à direção \(d\) sobre o plano \(\pi\).}\label{\detokenize{GE301-5:fig-atretasparalobrasarteinfinitopar2-6}}\label{\detokenize{GE301-5:id20}}\end{figure}

Porém, se tivermos \(r, s\) e \(d\) perpendiculares ao plano de projeção \(\pi\), a situação será um pouco diferente. Sendo a direção \(d\) perpendicular à \(\pi\), a reta que passa por pontos de \(r\) e é paralela à \(d\) coincide com \(r\). O mesmo acontece com a reta \(s\). Como a interseção de \(r\) e \(s\) com o plano \(\pi\) são os pontos \(A\) e \(B\), respectivamente, então as projeções paralelas de \(r\) e \(s\) serão os pontos \(A\) e \(B\). Veja a \hyperref[\detokenize{GE301-5:fig-atretasparalobrasarteinfinitopar2-7}]{Figura \ref{\detokenize{GE301-5:fig-atretasparalobrasarteinfinitopar2-7}}}.

\begin{figure}[H]
\centering
\capstart

\noindent\includegraphics[width=300bp]{{AtRetasParalObrasArteInfinito_11_2}.png}
\caption{A direção de projeção \(d\) é perpendicular ao plano de projeção \(\pi\), assim como \(r\) e \(s\). A projeção paralela das retas \(r\) e \(s\) serão os pontos \(A\) e \(B\), onde \(r\) e \(s\) intersectam o plano \(\pi\).}\label{\detokenize{GE301-5:fig-atretasparalobrasarteinfinitopar2-7}}\label{\detokenize{GE301-5:id21}}\end{figure}

\textit{Retas paralelas entre si e paralelas ao plano de projeção}

Vamos agora tomar duas retas \(r\) e \(s\) paralelas entre si e ao plano de projeção \(\pi\), além da reta \(d\) que será a direção de projeção e \(\pi\) o plano de projeção. Neste caso, \(d\) pode ser ou não perpendicular ao plano de projeção. Veja a \hyperref[\detokenize{GE301-5:fig-atretasparalobrasarteinfinitopar2-8}]{Figura \ref{\detokenize{GE301-5:fig-atretasparalobrasarteinfinitopar2-8}}}.

\begin{figure}[H]
\centering
\capstart

\noindent\includegraphics[width=350bp]{{AtRetasParalObrasArteInfinito_10}.png}
\caption{Retas \(r\) e \(s\) paralelas entre si e ao plano de projeção \(\pi\).}\label{\detokenize{GE301-5:fig-atretasparalobrasarteinfinitopar2-8}}\label{\detokenize{GE301-5:id22}}\end{figure}

Seguindo o mesmo raciocínio dos casos anteriores, tomemos os pontos \(A\) e \(B\) pertencentes à reta \(r\) e vamos projetá-los sobre \(\pi\) conforme mostra a \hyperref[\detokenize{GE301-5:fig-atretasparalobrasarteinfinitopar2-9}]{Figura \ref{\detokenize{GE301-5:fig-atretasparalobrasarteinfinitopar2-9}}}. Para isso, basta tomar a reta passando por cada ponto e paralela à \(d\), e então calcular sua interseção com \(\pi\). Chamaremos de \(A'\) e \(B'\) as projeções dos pontos \(A\) e \(B\), respectivamente.

\begin{figure}[H]
\centering
\capstart

\noindent\includegraphics[width=300bp]{{AtRetasParalObrasArteInfinito_9_1}.png}
\caption{Projeções paralelas \(A'\) e \(B'\) dos pontos \(A\) e \(B\) da reta \(r\) em relação à direção \(d\) sobre o plano \(\pi\).}\label{\detokenize{GE301-5:fig-atretasparalobrasarteinfinitopar2-9}}\label{\detokenize{GE301-5:id23}}\end{figure}

Projetando todos os pontos de \(r\), encontramos a reta \(r'\) que passa por \(A'\) e \(B'\), que é a projeção paralela à reta \(r\) em relação à direção \(d\) sobre o plano \(\pi\). Observe que \(r'\) é paralela à \(r\). Veja a \hyperref[\detokenize{GE301-5:fig-atretasparalobrasarteinfinitopar2-10}]{Figura \ref{\detokenize{GE301-5:fig-atretasparalobrasarteinfinitopar2-10}}}.

\begin{figure}[H]
\centering
\capstart

\noindent\includegraphics[width=300bp]{{AtRetasParalObrasArteInfinito_8}.png}
\caption{A reta \(r'\) é a projeção paralela da reta \(r\) em relação à direção \(d\) sobre o plano \(\pi\).}\label{\detokenize{GE301-5:fig-atretasparalobrasarteinfinitopar2-10}}\label{\detokenize{GE301-5:id24}}\end{figure}

Para encontrar a projeção paralela da reta \(s\), tomemos os pontos \(C\) e \(D\) de \(s\). As retas que passam por esses pontos e são paralelas à \(d\) intersectam \(\pi\) em \(C'\) e \(D'\), respectivamente. Se replicássemos esse raciocínio para todos os pontos de \(s\), teríamos que a reta \(s'\) que passa por \(C'\) e \(D'\) é a projeção paralela de \(s\) em relação à direção \(d\) sobre o plano \(\pi\). Note que \(s'\) é paralela à \(s\).

A \hyperref[\detokenize{GE301-5:fig-atretasparalobrasarteinfinitopar2-11}]{Figura \ref{\detokenize{GE301-5:fig-atretasparalobrasarteinfinitopar2-11}}} exibe as retas \(r'\) e \(s'\) que são as projeções paralelas das retas \(r\) e \(s\).

\begin{figure}[H]
\centering
\capstart

\noindent\includegraphics[width=300bp]{{AtRetasParalObrasArteInfinito_7}.png}
\caption{\(r\) e \(s\) são retas paralelas ao plano de projeção, e \(r'\) e \(s'\) são as projeções paralelas das retas \(r\) e \(s\) em relação à direção \(d\) no plano \(\pi\).}\label{\detokenize{GE301-5:fig-atretasparalobrasarteinfinitopar2-11}}\label{\detokenize{GE301-5:id25}}\end{figure}

Como \(r\) é paralela à \(s\), \(r\) é paralela à \(r'\) e \(s\) é paralela à \(s'\), concluímos que \(r'\) é paralela à \(s'\). Ou seja, a projeção paralela de retas paralelas entre si e paralelas ao plano de projeção será também retas paralelas.

Por tudo que vimos, podemos concluir que as projeções paralelas de retas paralelas são retas também paralelas, com exceção para as retas perpendiculares ao plano de projeção que são projetadas perpendicularmente no plano de projeção. Neste caso, as projeções serão pontos.

}{9}
\end{answer}
\begin{answer}{Retas paralelas, obras de arte e o infinito}
{
\paragraph{Parte 3}

Versões interativas feitas no GeoGebra da \hyperref[\detokenize{GE301-5:fig-proj-ponto-de-fuga-01}]{Figura \ref{\detokenize{GE301-5:fig-proj-ponto-de-fuga-01}}} à \hyperref[\detokenize{GE301-5:fig-proj-ponto-de-fuga-05}]{Figura \ref{\detokenize{GE301-5:fig-proj-ponto-de-fuga-05}}} estão disponíveis no endereço \url{https://www.geogebra.org/m/kFpnVERB}. O botão “S” exibe a solução e o botão “I” posiciona as retas para a configuração inicial. A seguir, apresentaremos a resposta final de cada uma das figuras.

\begin{figure}[H]
\centering
\capstart

\noindent\includegraphics[width=.7\linewidth]{{material-fZWyKPYy-final}.png}
\caption{A Última Ceia de Leonardo da Vinci (1452-1519) com segmentos desenhados sobre a imagem (fonte: Wikimedia Commons).}\label{\detokenize{GE301-5:fig-proj-ponto-de-fuga-01-solucao}}\label{\detokenize{GE301-5:id26}}\end{figure}

\begin{figure}[H]
\centering
\capstart

\noindent\includegraphics[width=.4\linewidth]{{material-nMXfrneZ-final}.png}
\caption{Linha de trem na cidade de Orléans na França com segmentos desenhados sobre a imagem (fonte: Wikimedia Commons).}\label{\detokenize{GE301-5:fig-proj-ponto-de-fuga-02-solucao}}\label{\detokenize{GE301-5:id27}}\end{figure}

\begin{figure}[H]
\centering
\capstart

\noindent\includegraphics[width=.4\linewidth]{{material-q9entV6e-final}.png}
\caption{Pintura japonesa em papel do século XIII com segmentos desenhados sobre a imagem (fonte: University of Maryland).}\label{\detokenize{GE301-5:fig-proj-ponto-de-fuga-03-solucao}}\label{\detokenize{GE301-5:id28}}\end{figure}

}{1}
\end{answer}
\begin{answer}{Retas paralelas, obras de arte e o infinito}
{
\begin{figure}[H]
\centering
\capstart

\noindent\includegraphics[width=125bp]{{material-brCxFwkH-final}.png}
\caption{Ícone bizantino na Igreja de São Clemente em Ohrid, República da Macedônia, com segmentos desenhados sobre a imagem(fonte: Wikimedia Commons).}\label{\detokenize{GE301-5:fig-proj-ponto-de-fuga-04-solucao}}\label{\detokenize{GE301-5:id29}}\end{figure}

\begin{figure}[H]
\centering
\capstart

\noindent\includegraphics[width=125bp]{{material-wAKGaw7e-final}.png}
\caption{O Casal Arnolfini do pintor flamengo Jan van Eyck (1390-1441) com segmentos desenhados sobre a imagem(fonte: Wikimedia Commons).}\label{\detokenize{GE301-5:fig-proj-ponto-de-fuga-05-solucao}}\label{\detokenize{GE301-5:id30}}\end{figure}

\begin{figure}[H]
\centering
\capstart

\noindent\includegraphics[width=.7\linewidth]{{material-W3A6HwxC-final}.png}
\caption{Palácio da Assembléia em Chandigarh na Índia com segmentos desenhados sobre a imagem (fonte: Wikimedia Commons).}\label{\detokenize{GE301-5:fig-proj-ponto-de-fuga-06-solucao}}\label{\detokenize{GE301-5:id31}}\end{figure}
}{1}
\end{answer}

\begin{knowledge}

Uma situação semelhante a da sinalização de trânsito descrita na \DUrole{xref,std,std-ref}{ativ-proj-construindo} é a confecção de paineis de propaganda em gramados de campos de futebol. Se eles forem desenhados sem distorções, suas imagens transmitidas pelas emissoras de TV ficarão distorcidas. Assim, para que a imagem fique correta quando observada pela câmera de TV, sua projeção em perspectiva é que deve ser desenhada no gramado.

\begin{figure}[H]
\centering
\capstart

\noindent\includegraphics[width=300bp]{{futebol1}.jpg}
\caption{Campo de Futebol}\label{\detokenize{GE301-4:id12}}\end{figure}

\begin{figure}[H]
\centering
\capstart

\noindent\includegraphics[width=300bp]{{futebol2}.jpg}
\caption{Perspectiva do campo}\label{\detokenize{GE301-4:id13}}\end{figure}

Note como a projeção depende da posição do observador: enquanto a câmera de TV transmite uma imagem sem distorções do painel de propaganda, uma pessoa sentada junto ao painel o verá bem distorcido.

Em artes plásticas, esta imagem distorcida que é vista corretamente de um certo ponto de vista é denominada \index{anamorfose}anamorfose. A palavra vem do Grego: \emph{ana} (de volta, de novo) e \emph{morphe} (forma). Além de distorções provocadas por projeções em perspectiva, a anamorfose inclui também distorções via espelhos cilíndricos, cônicos e piramidais.

Um exemplo clássico de anamorfose é dado pelo quadro “Os Embaixadores” (1533) do artista alemão Hans Holbein, O Jovem (1497/1498-1543). Você consegue identificar a parte do quadro em anamorfose?

\begin{figure}[H]
\centering
\capstart

\noindent\includegraphics[width=250bp]{{hans}.jpg}
\caption{“Os Embaixadores” de Hans Holbein, O Jovem (fonte: \href{https://en.wikipedia.org/wiki/File:Hans\_Holbein\_the\_Younger\_-\_The\_Ambassadors\_-\_Google\_Art\_Project.jpg}{Wikimedia Commons}).}\label{\detokenize{GE301-4:id14}}

\end{figure}

Anamorfose também já foi usada para esconder imagens sensíveis, como as imagens produzidas pelo artista alemão Erhard Schön (c. 1491\textendash{}1542). Você consegue identificar o que está representado em anamorfose?

\begin{figure}[H]
\centering
\capstart

\noindent\includegraphics[width=\linewidth]{{schon-02}.jpg}
\caption{“Aus, du alter Tor!” de Erhard Schön (fonte: \href{http://www.exploramuseum.de/images/pressefotos/anamorphoseAUSDUALTERTOR1\_m.jpg}{Explora Museum}).}
\label{\detokenize{GE301-4:id15}}\end{figure}

\begin{figure}[H]
\centering
\capstart

\noindent\includegraphics[width=\linewidth]{{schon-03}.jpg}
\caption{“Was siehst du?” de Erhard Schön (fonte: \href{http://www.britishmuseum.org/research/collection\_online/collection\_object\_details/collection\_image\_gallery.aspx?partid=1\&assetid=30265001\&objectid=1355159}{The British Museum}).}
\label{\detokenize{GE301-4:id16}}\end{figure}

Um belo exemplo de uso artístico da anamorfose no Brasil é o projeto “Luz nas Vielas” do grupo espanhol Boa Mistura que pintou, junto com os moradores da Vila Brasilândia em São Paulo, palavras como “firmeza”, “amor”, “doçura” nas paredes das vielas do bairro. Para conhecer mais sobre o projeto, acesse o vídeo \href{https://www.youtube.com/watch?v=Zi8ekDi7uLQ}{Poesia e Magia} no YouTube ou a \href{http://www.boamistura.com/\#/project/luz-nas-vielas-2}{página oficial do grupo}.

\begin{figure}[H]
\centering
\capstart

\noindent\includegraphics[width=\linewidth]{{boa-mistura-01}.jpg}
\caption{Anamorfose do projeto “Luz nas Vielas” do grupo Boa Mistura (fonte: \href{https://www.youtube.com/watch?v=gKRNLXghU94}{TEDx Talks})}\label{\detokenize{GE301-4:id17}}\end{figure}

Quer gerar suas próprias anamorfoses? Aqui estão dois softwares gratuitos que fazem isso a partir de uma imagem digital (arquivo jpg) de sua escolha: o \href{http://kejebodo.blogspot.com.br/2013/06/simple-anamorphic-converter.html}{Simple Anamorphic Converter} (distorções via projeções em perspectiva) e o \href{https://www.anamorphosis.com/software.html}{Anamorph Me!} (distorções via projeções paralelas, cilíndricas e cônicas).

\begin{figure}[H]
\centering
\capstart

\noindent\includegraphics[width=400bp]{{anamorfose-02}.jpg}
\caption{Brincando com anamorfose.}\label{\detokenize{GE301-4:id18}}\end{figure}
\end{knowledge}

\begin{knowledge}

Existem muitas produções artísticas que produzem o efeito de múltiplas projeções com múltiplos signifcados como visto na \DUrole{xref,std,std-ref}{ativ-proj-construindo}. Indicamos aqui duas referências: as esculturas do artista \href{https://www.jvmuntean.com/\#intro}{John V. Muntean} e do matemático \href{http://home.mims.meiji.ac.jp/~sugihara/Welcomee.html}{Kokichi Sugihara}.

Caso você queira construir uma versão simples de uma destas peças, um molde para ser impresso e recortado está disponível \href{https://goo.gl/ddFnuf}{neste endereço}. Um vídeo exibindo as etapas de montagem pode ser acessado no \href{https://youtu.be/QTNg0ofgB78}{YouTube}. Caso você tenha acesso a uma impressora 3D, o arquivo STL para impressão podem ser obtido gratuitamente no \href{https://www.thingiverse.com/thing:1657791}{Thingverse}.

% \begin{figure}[H]
% \centering
% \capstart

% \noindent\includegraphics{{sugihara-richeson}.png}
% \caption{Uma peça peculiar feita de papel (fonte: \href{https://youtu.be/QTNg0ofgB78}{David Recheson}).}\label{\detokenize{GE301-4:id19}}\end{figure}
\end{knowledge}



\practice{}
\label{\detokenize{GE301-5::doc}}\label{\detokenize{GE301-5:praticando-2}}\phantomsection\label{\detokenize{GE301-5:ativ-proj-infinito}}
\begin{task}{Retas paralelas, obras de arte e o infinito}

Você já percebeu em uma estrada ou em um corredor comprido (\hyperref[\detokenize{GE301-5:fig-proj-infinito-01}]{Figura \ref{\detokenize{GE301-5:fig-proj-infinito-01}}}) que elementos da cena que são paralelos como as linhas do acostamento ou as linhas das paredes não são vistos como paralelos e parecem convergir para um ponto? Nesta atividade, veremos como este fenômeno é explicado pelas projeções em perspectiva.

\begin{figure}[H]
\centering
\capstart

\noindent\includegraphics[width=400bp]{{infinito-01}.jpg}
\caption{Corredores paralelos (fonte: PEXELS e Wikimedia Commons).}\label{\detokenize{GE301-5:fig-proj-infinito-01}}\label{\detokenize{GE301-5:id1}}\end{figure}

Este tipo de situação é traduzido pela frase popular “Retas paralelas se encontram no infinito!”. Note, contudo, que as retas paralelas na cena tridimensional \textbf{nunca} se encontram. A concorrência ocorre para os prolongamentos das projeções em perspectiva de retas paralelas que não são paralelas ao plano de projeção.

\paragraph{Parte 1}

Vamos primeiro compreender como a projeção em perspectiva de uma reta não paralela ao plano de projeção pode ser obtida por meio da interseção de dois planos. Caso queira acompanhar os passos descritos a seguir com um modelo interativo que pode ser girado e ampliado, acesse (inclusive do seu celular) o endereço: \textless{}\url{https://www.geogebra.org/m/pcx56y49}\textgreater{}.

Como na \hyperref[\detokenize{GE301-5:fig-proj-infinito-02}]{Figura \ref{\detokenize{GE301-5:fig-proj-infinito-02}}}, considere uma projeção em perspectiva determinada por um centro \(O\) e um plano de projeção \(\pi\). Suponha que o ponto \(O\) \emph{não pertença} ao plano \(\pi\). Considere também uma reta \(r\) não paralela ao plano \(\pi\) e que não passa pelo ponto \(O\).

\begin{figure}[H]
\centering
\capstart

\noindent\includegraphics[width=300bp]{{infinito-02_2}.jpg}
\caption{Projeção em perspectiva de uma reta: passo 1.}\label{\detokenize{GE301-5:fig-proj-infinito-02}}\label{\detokenize{GE301-5:id2}}\end{figure}

Para determinar a projeção em perspectiva de uma reta \(r\), devemos, de acordo com a definição, para cada ponto de \(r\), determinar a interseção da reta que passa pelo ponto e o centro \(O\) com o plano \(\pi\). A \hyperref[\detokenize{GE301-5:fig-proj-infinito-03}]{Figura \ref{\detokenize{GE301-5:fig-proj-infinito-03}}}     exibe as projeções \(A'\), \(B'\) e \(C'\) dos pontos \(A\), \(B\) e \(C\) da reta \(r\).

\begin{figure}[H]
\centering
\capstart

\noindent\includegraphics[width=300bp]{{infinito-03_3}.jpg}
\caption{Projeção em perspectiva de uma reta: passo 2.}\label{\detokenize{GE301-5:fig-proj-infinito-03}}\label{\detokenize{GE301-5:id3}}\end{figure}

Observe que os pontos da reta \(r\) projetados pertencem ao plano \(\pi\) e, também, ao plano \(\phi\) que passa por \(O\) e contém a reta \(r\), conforme a \hyperref[\detokenize{GE301-5:fig-proj-infinito-04}]{Figura \ref{\detokenize{GE301-5:fig-proj-infinito-04}}}.

\begin{figure}[H]
\centering
\capstart

\noindent\includegraphics[width=300bp]{{infinito-04}.jpg}
\caption{Projeção em perspectiva de uma reta: passo 3.}\label{\detokenize{GE301-5:fig-proj-infinito-04}}\label{\detokenize{GE301-5:id4}}\end{figure}

Em particular, os pontos da reta \(r\) projetados sobre o plano \(\pi\) pertencem à interseção \(r'\) dos dois planos \(\pi\) e \(\phi\).

\begin{figure}[H]
\centering
\capstart

\noindent\includegraphics[width=300bp]{{infinito-05_3}.jpg}
\caption{Projeção em perspectiva de uma reta: passo 4.}\label{\detokenize{GE301-5:fig-proj-infinito-05}}\label{\detokenize{GE301-5:id5}}\end{figure}

Existe uma reta \(s\) que passa por \(O\) e é paralela a reta \(r\). Essa reta pertence ao plano \(\phi\) e intersectará o plano \(\pi\) e, portanto, a reta \(r'\) em um ponto \(F\), conforme a \hyperref[\detokenize{GE301-5:fig-proj-infinito-06}]{Figura \ref{\detokenize{GE301-5:fig-proj-infinito-06}}}. Neste contexto, o ponto \(F\) é denominado \index{ponto de fuga}ponto de fuga associado à direção dada pela reta \(r\).

\begin{figure}[H]
\centering
\capstart

\noindent\includegraphics[width=300bp]{{infinito-06_1}.jpg}
\caption{Projeção em perspectiva de uma reta: passo 5.}\label{\detokenize{GE301-5:fig-proj-infinito-06}}\label{\detokenize{GE301-5:id6}}\end{figure}

\textbf{Pergunta 1.} Não existe ponto algum da reta \(r\) cuja projeção sobre o plano \(\pi\) seja o ponto \(F\). Por quê?

Note que, em particular, a projeção da reta \(r\) \textbf{não é} uma reta mas, sim, uma reta menos um ponto: \(r' - \{F\}\). Se prolongarmos esta projeção incluindo o ponto \(F\), obteremos uma reta: \(r'\).

\textbf{Pergunta 2.} Suponha no que foi feito até agora, a reta \(r\) seja trocada por uma outra reta diferente, mas paralela a reta \(r\). O ponto \(F\) para esta nova reta será o mesmo, não mudará. Por quê?

\textbf{Pergunta 3.} Supomos inicialmente que a reta \(r\) não passa pelo ponto \(O\). O que mudaria no que foi feito se \(O\) fosse um ponto de \(r\)?

Considere agora o caso de duas ou mais retas paralelas que não são paralelas ao plano de projeção \(\pi\). Em decorrência do que foi estabelecido nas Perguntas 1 e 2, sabemos que suas projeções sobre o plano \(\pi\) são retas menos um mesmo ponto \(F\) e que, se prolongássemos essas projeções, obteríamos retas concorrentes no ponto \(F\). A \hyperref[infinito-08]{Figura \ref{infinito-08}} e a \hyperref[infinito-09]{Figura \ref{infinito-09}} ilustram essa propriedade.
Isto também explica a “convergência”
das linhas do acostamento e das linhas das paredes na \hyperref[\detokenize{GE301-5:fig-proj-infinito-01}]{Figura \ref{\detokenize{GE301-5:fig-proj-infinito-01}}}.

\begin{figure}[H]
\centering

\noindent\includegraphics[width=275bp]{{infinito-08}.jpg}
\caption{Projeção em perspectiva de um feixe com 3 retas paralelas (\textless{}\url{https://www.geogebra.org/m/EAuVTyTG}\textgreater{}).}
\label{infinito-08}

\end{figure}

\begin{figure}[H]
\centering
\capstart

\noindent\includegraphics[width=275bp]{{infinito-09_1}.jpg}
\caption{Projeção em perspectiva de um feixe com muitas retas paralelas (\textless{}\url{https://www.geogebra.org/m/ycxHtZEP}\textgreater{}).}\label{infinito-09}\end{figure}

\textbf{Pergunta 4.} Qual é a projeção em perspectiva de um feixe de retas paralelas que são paralelas ao plano de projeção? Faça uma conjectura e justifique-a!

\paragraph{Parte 2}

\textbf{Pergunta 1.} O desenvolvimento feito na PARTE 1 trata de projeções em perspectiva de retas paralelas. O que pode ser dito sobre \emph{projeções paralelas de retas paralelas}? Faça uma conjectura e justifique-a!

\paragraph{Parte 3}

Caso uma pintura queira retratar a realidade segundo a metáfora da janela de Alberti (\hyperref[\detokenize{GE301-3:fig-proj-janela-de-alberti-03}]{Figura \ref{\detokenize{GE301-3:fig-proj-janela-de-alberti-03}}}), os elementos desenhados devem respeitar as propriedades das projeções em perspectiva. Em particular, segmentos de retas que são paralelos na cena tridimensional devem ser desenhados como segmentos de reta cujos prolongamentos se encontram em um ponto de fuga.

Você receberá reproduções de pinturas, desenhos e fotos do seu professor (as imagens a seguir dão alguns exemplos).


\begin{figure}[H]
\centering
\ifnum\aluno=1
\includegraphics[width=300bp]{{ponto-de-fuga-01_2}.jpg}
\else
\includegraphics[width=290bp]{{ponto-de-fuga-01_2}.jpg}
\fi

\caption{A Última Ceia de Leonardo da Vinci (1452-1519) (fonte: Wikimedia Commons).}\label{\detokenize{GE301-5:fig-proj-ponto-de-fuga-01}}\label{\detokenize{GE301-5:id8}}\end{figure}

\begin{multicols}{2}
\begin{figure}[H]
\centering
\includegraphics[width=\linewidth]{{ponto-de-fuga-02}.jpg}
\caption{Linha de trem na cidade de Orléans na França (fonte: Wikimedia Commons).}
\end{figure}

\begin{figure}[H]
\centering
\includegraphics[width=\linewidth]{{ponto-de-fuga-03}.jpg}
\caption{Pintura japonesa em papel do século XIII (fonte: \href{http://faculty.philosophy.umd.edu/jhbrown/digitaltech/index.html}{University of Maryland}).}\label{\detokenize{GE301-5:fig-proj-ponto-de-fuga-03}}\label{\detokenize{GE301-5:id10}}
\end{figure}
\end{multicols}

\begin{multicols}{2}
\begin{figure}[H]
\centering
\includegraphics[height=200bp]{{ponto-de-fuga-04}.jpg}
\caption{Ícone bizantino na Igreja de São Clemente em Ohrid, República da Macedônia (fonte: Wikimedia Commons).}\label{\detokenize{GE301-5:fig-proj-ponto-de-fuga-04}}\label{\detokenize{GE301-5:id11}}
\end{figure}

\begin{figure}[H]
\centering


\includegraphics[height=200bp]{{ponto-de-fuga-07}.jpg}
\caption{O Casal Arnolfini do pintor flamengo Jan van Eyck (1390-1441) (fonte: Wikimedia Commons).}\label{\detokenize{GE301-5:fig-proj-ponto-de-fuga-05}}\label{\detokenize{GE301-5:id12}}
\end{figure}
\end{multicols}

\begin{figure}[H]
\centering


\includegraphics[width=300bp]{{ponto-de-fuga-09}.jpg}
\caption{Palácio da Assembleia em Chandigarh na Índia (fonte: Wikimedia Commons).}\label{\detokenize{GE301-5:fig-proj-ponto-de-fuga-09}}\label{\detokenize{GE301-5:id13}}
\end{figure}


Em cada uma delas, você deve identificar quais são os elementos que são supostamente paralelos na cena tridimensional sendo registrada (contornos de paredes, ladrilhos, etc.) e, então, desenhar segmentos de reta nesses elementos da imagem. Aqui está um exemplo.


\begin{figure}[H]
\centering

\noindent\includegraphics[width=425bp]{{ponto-de-fuga-08_1}.jpg}
\end{figure}

\end{task}

\begin{reflection}

\begin{figure}[H]
\centering

\noindent\includegraphics[width=\linewidth]{{tirinha3}.jpg}
\end{figure}


\begin{figure}[H]
\centering
\capstart

\noindent\includegraphics[width=\linewidth]{{calvin-haroldo-perspectiva}.jpg}
\caption{Calvin, Haroldo e Perspectiva!}\label{\detokenize{GE301-5:id32}}\end{figure}
\end{reflection}

\clearpage

\begin{objectives}{Comprimentos em projeções}
{
Compreender quantitativamente como se relacionam as medidas de comprimento de um objeto e de sua imagem por uma projeção em perspectiva.
}{1}{1}
\end{objectives}
\begin{sugestions}{Comprimentos em projeções}
{
\begin{itemize}
\item {} 
O uso do conceito de função nesta atividade não é casual e vai além do propósito de uma mera conexão entre Geometria e Álgebra. Observe, por exemplo, como a notação funcional permite compactar informação: no lugar de falar “Qual deve ser o valor de \(x\) para que o valor de \(h\)‘ correspondente seja igual ao dobro do valor de \(h'\) que você obteve no primeiro item?”, basta falar “Qual é o valor de \(x\) para o qual \(f(x) = \frac{1}{2} \, f(6)\)?”. No espírito da \DUrole{xref,std,std-ref}{ativ-funcoes-enchendo-o-cone}, é importante articular as representações em linguagem natural e a linguagem funcional e explicitar para o alunos as vantagens e desvantagens de cada representação.

\item {} 
As construções interativas do GeoGebra para as figuras das Etapas 1 e 3 estão disponíveis nos endereços: \url{https://www.geogebra.org/m/HKuqwxXn}, \url{https://www.geogebra.org/m/u4mkzbmP}, \url{https://www.geogebra.org/m/UGFWgAQ5}, \url{https://www.geogebra.org/m/SubrgSmG}, \url{https://www.geogebra.org/m/Du4285XX}, \url{https://www.geogebra.org/m/YjAhaCNu} e \url{https://www.geogebra.org/m/jGxrcxvw}. Caso os alunos tenham dificuldades em interpretar as versões estáticas, recomendamos o uso destas versões dinâmicas. Note, por exemplo, que na \hyperref[\detokenize{GE301-5:fig-proj-comprimento-04}]{Figura \ref{\detokenize{GE301-5:fig-proj-comprimento-04}}}, que é por si a imagem de uma projeção em perspectiva, o segmento \(R'S'\) aparenta não ter o mesmo comprimento do segmento \(A'B'\) mas, de fato, eles possuem o mesmo comprimento no espaço tridimensional. Ao usar a construção interativa \url{https://www.geogebra.org/m/Du4285XX}, você e o aluno poderão girar a cena e, assim, desfazer essa percepção provocada pela figura estática. Outro exemplo: na \hyperref[\detokenize{GE301-5:fig-proj-ladrilhamentos-01}]{Figura \ref{\detokenize{GE301-5:fig-proj-ladrilhamentos-01}}}, o ângulo \(OPR\) não parece ser reto na imagem mas, na cena tridimensional, é reto. Novamente, a explicação está no fato de que a \hyperref[\detokenize{GE301-5:fig-proj-ladrilhamentos-01}]{Figura \ref{\detokenize{GE301-5:fig-proj-ladrilhamentos-01}}} foi construída com uma projeção em perspectiva e, portanto, do ponto de vista escolhido, a amplitude do ângulo \(OPR\) foi projetada de forma distorcida.

\end{itemize}
}{1}{1}
\end{sugestions}
\clearmargin
\begin{answer}{Comprimentos em projeções - Parte 1}
{
\paragraph{Etapa 1}
\begin{enumerate}
\item Como os triângulos \(OB'A'\) e \(OBA\) são semelhantes (pois possuem dois ângulos correspondentes congruentes), temos que
\begin{equation*}
\begin{split}\frac{h'}{3}=\frac26 \Leftrightarrow h'=1.\end{split}
\end{equation*}
Em outras palavras, sendo a distância do centro de projeção ao plano de projeção igual a \(3\) e a distância do centro de projeção ao segmento \(AB\) (a ser projetado) igual \(6\), a projeção do segmento de reta \(AB\) de comprimento \(2\) terá comprimento \(1\).

\item {} 
Novamente usando a razão de semelhança entre os triângulos \(OB'A'\) e \(OBA\), mas agora sem substituir o valor de \(x\) temos:
\begin{equation*}
\begin{split}\frac{h'}{3}=\frac{2}{x} \Longleftrightarrow h'=\frac6x.\end{split}
\end{equation*}
Assim, a função procurada é \(f(x)=\frac{6}{x}\). Note que, como esta função modela a situação apresentada na figura, os valores de \(x\) devem ser apenas positivos, pois \(x\) é a distância de \(O\) até \(A\). Além disso, \(O\) não pode coincidir com \(A'\), o que implica que x também não pode ser \(0\). Portanto, o domínio da função \(f\) é o intervalo \(]0,\infty[\).

\begin{figure}[H]
\centering
\capstart

\noindent\includegraphics[width=.5\linewidth]{{funcao6x_2}.png}
\caption{Gráfico da função \(f(x)=\frac{6}{x}\) com \(x\in ]0,\infty[\).}\label{\detokenize{GE301-5:fig-proj-comprimentos-solucao-01}}\label{\detokenize{GE301-5:id46}}\end{figure}

\item {} 
Se \(f(x) = \frac12 \, f(6)\), então
\begin{equation*}
\begin{split}\frac6x = \frac12 \cdot 1 \Longleftrightarrow x = 12.\end{split}
\end{equation*}
Portanto, para que a projeção do segmento de reta \(AB\) tenha a metade do comprimento que tinha quando \(x = OA = 6\) é preciso que \(x = 12\).


\end{enumerate}
}{1}
\end{answer}
\begin{answer}{Comprimentos em projeções - Parte 1}
{
\begin{enumerate}\setcounter{enumi}{3}
\item {} 
Se \(f(x) = 2 \, f(6)\), então
\begin{equation*}
\begin{split}\frac6x = 2 \cdot 1 \Longleftrightarrow x = 3.\end{split}
\end{equation*}
Sendo assim, para que a projeção do segmento de reta \(AB\) tenha o dobro do comprimento que tinha quando \(x = OA = 6\) é preciso que \(x = 3\).

\item {} 
Se \(f(x)=h\), então \(\frac6x=2\). Logo, \(x=3\). Sendo assim, para que a projeção do segmento de reta \(AB\) e \(AB\) tenham o mesmo comprimento, é preciso que a distância de \(AB\) ao plano de projeção seja \(3\). Já no contexto das imagens dos dedos, podemos dizer que os dois dedos estão sobre plano de projeção.

Se \(f(x)>h\), então \(\frac6x>2\). Logo, \(x<3\). Portanto, para que o comprimento da projeção do segmento de reta \(AB\) seja maior que seu próprio comprimento, é preciso que a distância de \(AB\) até \(O\) seja menor que \(3\). No contexto das imagens dos dedos, isto significa que os dedos estão posicionados antes do plano de projeção, ou seja, entre o observador e o plano de projeção. Neste caso, o tamanho da projeção dos dedos no plano de projeção seria maior que o tamanho dos próprios dedos.

Se \(f(x)<h\), então \(\frac6x<2\). Logo, \(x>3\). Ao contrário do que vimos no caso anterior, para que o comprimento da projeção do segmento de reta \(AB\) seja menor que seu comprimento, é preciso que a distância de \(AB\) até \(O\) seja maior que \(3\). No contexto das imagens dos dedos, isto quer dizer que o objeto está posicionado após o plano de projeção. Assim, sua projeção terá tamanho menor que o tamanho dos dedos.

\item {} 
Se existissem \(x_1\) e \(x_2\) tais que \(f(x_1)=f(x_2)\), então teríamos
\begin{equation*}
\begin{split}\begin{array}{ll}
& \frac{6}{x_1} = \frac{6}{x_2} \\
\Longleftrightarrow & 6x_1 = 6x_2 \\
\Longleftrightarrow & 6(x_1 - x_2)=0 \\
\Longleftrightarrow & x_1 = x_2.
\end{array}\end{split}
\end{equation*}
Portanto, se \(f(x_1)=f(x_2)\), então \(x_1 = x_2\). Isto quer dizer que, se as projeções de dois segmentos possuem o mesmo comprimento, então eles estão posicionados a uma mesma distância de \(O\). No contexto das imagens dos dedos, isto significa que se a projeção dos dois dedos possui o mesmo tamanho, então eles estão a uma mesma distância do observador, e consequentemente eles estão a uma mesma distância do plano de projeção.

\item {} 
Se os valores de \(x\) vão ficando arbritariamente grandes, então os valores correspondentes de \(h'\) ficarão arbritariamente pequenos. Analise mais uma vez o gráfico mostrado na \hyperref[\detokenize{GE301-5:fig-proj-comprimentos-solucao-01}]{Figura \ref{\detokenize{GE301-5:fig-proj-comprimentos-solucao-01}}}. Isto significa, no contexto das imagens dos dedos, que a medida que os dedos se afastam do observador, sua projeção diminui de tamanho.

\item {} 
Se os valores de \(x\) vão ficando arbritariamente próximos de \(0\) mas com valores maiores do que \(0\), então os valores correspondentes de \(h'\) ficarão arbritariamente grandes. Novamente, analise o gráfico mostrado na \hyperref[\detokenize{GE301-5:fig-proj-comprimentos-solucao-01}]{Figura \ref{\detokenize{GE301-5:fig-proj-comprimentos-solucao-01}}}. Na situação dos dedos, isto significa que quanto mais próximo do observador estiverem os dedos, maiores serão suas projeções.

\item {} 
Para que a projeção de \(CD\) seja o segmento \(A'B'\), os triângulos \(OB'A'\) e \(ODC\) devem ser semelhantes. Então, \(\overline{A'B'}/d = \overline{CD}/15\). Como \(d = 3\), segue-se que \(\overline{CD} = 5 \, \overline{A'B'}\).
\end{enumerate}

\paragraph{Etapa 2}

\begin{enumerate}
\item {} 
Esta pergunta pode ser respondida usando o mesmo argumento do item a) da Pergunta 1, mas sem utilizar valores dados na etapa anterior para \(h\) e \(d\). Como os triângulos \(OB'A'\) e \(OBA\) são semelhantes, temos que
\begin{equation*}
\begin{split}\frac{h'}{d}=\frac{h}{x} \Leftrightarrow h'=\frac{hd}{x}.\end{split}
\end{equation*}
\item {} 
Utilizando a mesma semelhança de triângulos do item anterior, se a distância de \(AB\) ao ponto \(O\) é igual a \(2x\), teremos:
\begin{equation*}
\begin{split}\frac{h'}{d}=\frac{h}{2x} \Leftrightarrow h'=\frac{hd}{2x}=\frac{\frac{hd}{x}}2.\end{split}
\end{equation*}
Comparando o valor de \(h'\) deste e do item anterior, podemos concluir que a afirmação é verdadeira.

\end{enumerate}
}{9}
\end{answer}
\begin{answer}{Comprimentos em projeções - Parte 1}
{
\paragraph{Etapa 3}
\begin{enumerate}
\item 

Observe na \hyperref[\detokenize{GE301-5:fig-proj-comprimento-02}]{Figura \ref{\detokenize{GE301-5:fig-proj-comprimento-02}}} que os triângulos \(OS'R'\) e \(OSR\) são semelhantes (o ângulo \(O\) é comum aos dois triângulos e \(O\hat{R'}S'\) é congruente à \(O\hat{R}S\)), então
\begin{equation*}
\begin{split}\frac{S'R'}{R'O} = \frac{SR}{RO}  \Longleftrightarrow \frac{R'S'}{R'O} = \frac{h}{RO} \Longleftrightarrow R'S'= \frac{h \cdot R'O}{RO}.\end{split}
\end{equation*}
Analogamente, os triângulos \(OA'R'\) e \(OAR\) também são semelhantes (o ângulo \(O\) é comum aos dois triângulos e \(O\hat{A'}R'\) é congruente à \(O\hat{A}R\)). Logo,
\begin{equation*}
\begin{split}\frac{OA'}{R'O} = \frac{OA}{RO}  \Longleftrightarrow \frac{d}{R'O} = \frac{x}{RO} \Longleftrightarrow RO = \frac{x \cdot R'O}{d}.\end{split}
\end{equation*}
Substituindo o valor de \(RO\) em \(S'R'\) temos:
\begin{equation*}
\begin{split}R'S'= \frac{h \cdot R'O}{\frac{x \cdot R'O}{d}} = \frac{hd}{x}.\end{split}
\end{equation*}
Pela atividade a da etapa 2, sabemos que \(h'=\frac{hd}{x}\) e com isso, podemos concluir que \(R'S'\) possui o mesmo comprimento do segmento \(A'B'\). Portanto, a projeção da translação de um segmento paralelamente ao plano \(\pi\) terá o mesmo comprimento do segmento.

Neste caso, temos três opções:
\begin{itemize}
\item {} 
Se \(x>d\), ou seja, \(RS\) está posicionado após o plano de projeção, então \(R'S'\) terá comprimento menor que \(AB\) e \(RS\).

\item {} 
Se \(x=d\), ou seja, \(RS\) está posicionado sobre o plano de projeção, então \(R'S'\) terá comprimento o mesmo comprimento que \(AB\) e \(RS\).

\item {} 
Se \(x<d\), ou seja, \(RS\) está posicionado antes do plano de projeção, então \(R'S'\) terá comprimento maior que \(AB\) e \(RS\).

\end{itemize}
No contexto da visualização das imagens de seus dois dedos, temos que se afastarmos os dedos indicadores e mantivermos os dois em um plano paralelo ao plano de projeção \(\pi\) (“translação” de um dos dedos paralelamente ao plano de projeção), então suas projeções terão sempre o mesmo comprimento.

\item {} 
Se, ao invés de um retângulo, o quadrilátero \(ARSB\) for um paralelogramo, teremos a mesma situação da atividade anterior e exatamente as mesmas possiblidades.

Já no contexto da visualização das imagens de seus dois dedos, temos que se afastarmos os dedos indicadores e elevarmos um em relação ao outro (mais uma vez, uma “translação” de um dedo paralelamente ao plano de projeção), além de  mantê-los dentro de um plano paralelo ao plano de projeção \(\pi\), então suas projeções terão sempre o mesmo comprimento.
\end{enumerate}
}{1}
\end{answer}
\begin{answer}{Comprimentos em projeções - Parte 1}
{
\paragraph{Etapa 3}
\begin{enumerate}\setcounter{enumi}{2}
\item {} 
Observe na \hyperref[\detokenize{GE301-5:fig-proj-comprimento-04}]{Figura \ref{\detokenize{GE301-5:fig-proj-comprimento-04}}} que os triângulos \(OAB\) e \(OAS\) são congruentes (o lado \(OA\) é comum aos dois triângulos, os ângulos \(O\hat{A}B\) e \(O\hat{A}S\) são congruentes, e os lados \(AB\) e \(AS\) são congruentes), os triângulos \(OA'S'\) e \(OAS\) são semelhantes (o ângulo \(O\) é comum aos dois triângulos e \(O\hat{R'}S'\) é congruente à \(O\hat{R}S\)) e os triângulos \(OA'B'\) e \(OAB\) são semelhantes (visto na etapa anterior). Assim, pela transitividade da semelhança de triângulos, podemos concluir que os triângulos \(OA'B'\) e \(OA'S'\) são semelhantes. Como \(OA'\) é lado comum aos triângulos \(OA'B'\) e \(OA'S'\), então, na verdade, eles são congruentes. Dessa forma, temos que \(A'S'\) possui a mesma medida \(h'\) do segmento \(A'B'\). Sendo assim, temos aqui as mesmas mesmas possibilidades das duas questões anteriores.

No contexto da visualização das imagens de seus dedos, temos os dois dedos mantendo a mesma base e um deles sendo rotacionado em relação ao outro dentro do plano \(\omega\) paralelo ao plano de projeção \(\pi\). Nesta situação, como os dedos possuem o mesmo comprimento, suas projeções terão também o mesmo comprimento.

\item {} 
O segmento \(RS\) pode ser obtido fazendo uma translação e depois uma rotação de uma cópia do segumento \(AB\). Portanto, de acordo com o que vimos anteriormente, a projeção \(R'S'\) do segmento \(RS\) terá o mesmo comprimento da projeção \(A'B'\) do segmento \(AB\), já que \(RS\) e \(AB\) possuem o mesmo comprimento \(h\). Sendo assim, as possibilidade analisadas no item (a) também serão válidas na situação mais geral.

No contexto da visualização das imagens de seus dedos, teremos que unificar as situações anteriores. Isso quer dizer que enquanto um dos dedos permanece fixado, o outro pode ser disposto em qualquer posição desde que dentro do plano \(omega\) (ou seja, o segundo dedo está sofrendo translações e rotações). Pelo que estudamos nos itens anteriores, como os dedos possuem o mesmo comprimento, suas projeções terão também o mesmo comprimento. 
\end{enumerate}
}{1}
\end{answer}
\begin{answer}{Comprimentos em projeções - Parte 1}
{
\paragraph{Etapa 3}
\begin{enumerate}\setcounter{enumi}{3}
\item {} 
A afirmação é verdadeira. Como vimos nos casos anteriores, o comprimento da projeção de um segmento que mede \(h\) será \(\frac{hd}{x}\), ou seja, o comprimento da projeção depende apenas de \(d\) e \(x\).
\end{enumerate}

\paragraph{Etapa 4}

Vamos considerar um plano de projeção \(\pi\) e uma direção \(d\) de projeção (que pode ser perpendicular ou oblíqua à \(\pi\)). Considere um segmento de reta \(AB\) paralelo ao plano de projeção \(\pi\). Os pontos \(A'\) e \(B'\) são, respectivamente, as projeções paralelas de \(A\) e \(B\) sobre o plano de projeção \(\pi\) na direção de \(d\). Note que \(ABB'A'\) é um paralelogramo, pois possui lados opostos paralelos. Isto implica que \(A'B'\) possui o mesmo comprimento de \(AB\), ou seja, o comprimento da projeção paralela de um segmento paralelo ao plano de projeção coincide com o comprimento do segmento.

\begin{figure}[H]
\centering
\capstart

\noindent\includegraphics[width=.6\linewidth]{{ProjecaoParalelaSegmentosParalelos}.png}
\caption{Projeção paralela do segmento \(AB\) sobre o plano \(\pi\) na direção \(d\).}\label{\detokenize{GE301-5:fig-proj-comprimentos-solucao-03}}\label{\detokenize{GE301-5:id47}}\end{figure}

Seguindo esse mesmo raciocínio, para quaisquer dois segmentos de mesmo comprimento paralelos à \(\pi\), podemos concluir que suas projeções paralelas sobre o plano \(\pi\) na direção \(d\) também terão os mesmos comprimentos.
}{1}
\end{answer}
\clearmargin
\clearmargin
\begin{answer}{Comprimentos em projeções - Parte 2}
{
\paragraph{Etapa 1}
\begin{enumerate}
\item {} 
Como os triângulos \(OUP'\) e \(P'VP\) são semelhantes (pois o ângulo em \(P\) é comum aos dois triângulos e os ângulos \(O\hat{U}P'\) e \(P'\hat{V}P\) são congruentes), temos que
\begin{equation*}
\begin{split}\frac{VP}{VP'}=\frac{UP}{UO} \Longleftrightarrow \frac{1}{y}=\frac{6}{3} \Longleftrightarrow y=\frac12.\end{split}
\end{equation*}
\item {} 
Utilizando a mesma semelhança de triângulos do item anterior, assim como sua razão de semelhança, encontramos
\begin{equation*}
\begin{split}\frac{VP}{VP'}=\frac{UP}{UO} \Longleftrightarrow \frac{x-5}{y}=\frac{x}{3} \Longleftrightarrow y=\frac{3x-15}{x}.\end{split}
\end{equation*}

\notasfig{\begin{figure}[H]
\centering
\capstart

\noindent\includegraphics[width=.8\linewidth]{{funcao3x-15x}.png}
\caption{Gráfico da função \(g(x)=\frac{3x-15}{x}\) com \(x\geq 5\).}\label{\detokenize{GE301-5:fig-proj-comprimentos-solucao-02}}\label{\detokenize{GE301-5:id48}}\end{figure}}

\item {} 
Se \(g(x) = \frac12 \, g(6)\), então
\begin{equation*}
\begin{split}\frac{3x-15}{x} = \frac12 \cdot \frac12 \Longleftrightarrow x = \frac{60}{11}.\end{split}
\end{equation*}
\item {} 
Se \(g(x) = 2 \, g(6)\), então
\begin{equation*}
\begin{split}\frac{3x-15}{x} = 2 \cdot \frac12 \Longleftrightarrow x = \frac{15}{2}.\end{split}
\end{equation*}
\item {} 
Analisando a \hyperref[\detokenize{GE301-5:fig-proj-comprimentos-solucao-02}]{Figura \ref{\detokenize{GE301-5:fig-proj-comprimentos-solucao-02}}}, é possível dizer que a afirmação é verdadeira. Além disso, note que se \(x>5\), então \(-\frac{15}{x}<0\). Assim, \(3-\frac{15}{x} < 3 \Longleftrightarrow g(x)<3\).

\item {} 
Se existissem \(x_1>d\) e \(x_2>d\) tais que \(g(x_1)=g(x_2)\), então teríamos
\begin{equation*}
\begin{split}\begin{array}{ll}
& \dfrac{3x_1-15}{x_1} = \dfrac{3x_2-15}{x_2} \\
\Longleftrightarrow & 3x_1x_2-15x_2 = 3x_1x_2-15x_1 \\
\Longleftrightarrow & 15(x_1 - x_2)=0 \\
\Longleftrightarrow & x_1 = x_2.
\end{array}\end{split}
\end{equation*}
Portanto, se \(g(x_1)=g(x_2)\), então \(x_1 = x_2\).

\item {} 
A função \(g(x)\) pode ser escrita da forma \(g(x)=3-\frac{15}{x}\). Note que, quanto maior os valores de \(x\), menores são os valores de \(\frac{15}{x}\) e assim, \(3-\frac{15}{x}\) ficará cada vez mais próximo de \(3\). Logo, se os valores \(x\) vão ficando arbitrariamente grandes, os valores de \(g(x)=y\) vão ficando mais próximos de \(3\).

Isto pode ainda ser observado no gráfico da função \(g\) na \hyperref[\detokenize{GE301-5:fig-proj-comprimentos-solucao-02}]{Figura \ref{\detokenize{GE301-5:fig-proj-comprimentos-solucao-02}}}.
\end{enumerate}
}{1}
\end{answer}
\begin{answer}{Comprimentos em projeções - Parte 2}
{
\paragraph{Etapa 2}
\begin{enumerate}
\item {} 
Utilizando a mesma semelhança de triângulos do item (a) da Etapa 1, assim como sua razão de semelhança, temos
\begin{equation*}
\begin{split}\frac{VP}{VP'}=\frac{UP}{UO} \Longleftrightarrow \frac{x-d}{y}=\frac{x}{a} \Longleftrightarrow y=\frac{a(x-d)}{x}.\end{split}
\end{equation*}
\item {} 
Utilizando a mesma semelhança de triângulos do item anterior, se a distância de \(RS\) ao ponto \(U\) é igual a \(2x\), teremos:
\begin{equation*}
\begin{split}\frac{2x-d}{y}=\frac{2x}{a} \Longleftrightarrow y=\frac{2xa-da}{2x}=2\left(\frac{a(x-\frac{d}{2})}{x}\right).\end{split}
\end{equation*}
Comparando o valor de \(y\) deste e do item anterior, podemos concluir que a afirmação é falsa, já que o valor encontrado neste item não corresponde ao dobro do encontrado no item anterior.

\end{enumerate}
}{0}
\end{answer}
\clearmargin
\begin{answer}{Comprimentos em projeções - Parte 2}
{
\paragraph{Etapa 3}
A \hyperref[\detokenize{GE301-5:fig-proj-comprimentos-solucao-04}]{Figura \ref{\detokenize{GE301-5:fig-proj-comprimentos-solucao-04}}} mostra a projeção em perspectiva do quadriculado \(RABS\) feita no Geogebra no endereço \url{https://www.geogebra.org/m/jGxrcxvw}.

\begin{figure}[H]
\centering
\capstart

\noindent\includegraphics[width=.5\linewidth]{{Ladrilhamento_Parte2Etapa3}.png}
\caption{Projeção em perspectiva do quadriculado \(RABS\) da \hyperref[\detokenize{GE301-5:fig-proj-ladrilhamentos-02}]{Figura \ref{\detokenize{GE301-5:fig-proj-ladrilhamentos-02}}}.}\label{\detokenize{GE301-5:fig-proj-comprimentos-solucao-04}}\label{\detokenize{GE301-5:id49}}\end{figure}

\paragraph{Etapa 4}

Neste caso, precisamos estudar duas possibilidades. Na primeira, considere que a direção de projeção \(d\) seja paralela ao lado \(RA\) (e a todos os que são paralelos a ele) do quadriculado, ou seja, perpendicular ao plano de projeção \(\pi\). Como \(RABS\) está contido em um plano \(\gamma\) perpendicular ao plano de projeção \(\pi\), a projeção do quadriculado \(RABS\) será o segmento de reta \(R'S'\), onde \(R'\) e \(S'\) são as projeções paralelas dos pontos \(R\) e \(S\), respectivamente. Isso acontece pois, segmentos perpendiculares ao plano de projeção são projetados em pontos e segmentos paralelos ao plano de projeção são projetados em segmentos de mesmo comprimento, como já foi estudado antes.

Caso a direção de projeção \(d\) seja oblíqua em relação ao plano de projeção, a projeção do quadriculado \(RABS\) será um novo quadriculado \(R'A'B'S'\), onde \(R', A', B'\) e \(S'\) são as projeções paralelas dos pontos \(R, A, B\) e \(S\), respectivamente. Repare que, como \(RS\) é um segmento de reta paralelo ao plano de projeção, então \(R'S'\) será paralelo à \(RS\) e terá o mesmo comprimento de \(RS\), assim como \(A'B'\) será paralelo à \(AB\) e terá o mesmo comprimento de \(AB\). Assim, podemos concluir que o quadrilátero \(R'A'B'S'\) é um paralelogramo (podendo ser mais alongado ou mais achatado dependendo da direção de projeção) e sua parte interna será quadriculada de acordo com a direção de projeção \(d\).
}{1}
\end{answer}

\phantomsection
\begin{task}{Comprimentos em projeções}
\label{\detokenize{GE301-5:ativ-proj-comprimentos}}

\subsection{Parte 1}

Coloque os seus dois dedos indicadores um do lado do outro. Eles têm o mesmo tamanho, não é? Agora afaste um deles. Seus dedos continuam com o mesmo tamanho? Sim, mas você \textbf{vê} o dedo que está mais longe menor, não é? Esta propriedade pode ser explicada via projeções em perspectiva e a exploraremos nesta atividade.

\begin{figure}[H]
\centering
\capstart

\noindent\includegraphics[width=400bp]{{dedos-03}.jpg}
\caption{Visualização de imagens dos dois dedos indicadores.}\label{\detokenize{GE301-5:fig-proj-dedos-03}}\label{\detokenize{GE301-5:id33}}\end{figure}

Vamos começar com uma configuração bem simples. Considere a \hyperref[\detokenize{GE301-5:fig-proj-comprimento-01}]{Figura \ref{\detokenize{GE301-5:fig-proj-comprimento-01}}}. Nela, o segmento \(AB\) é paralelo ao plano de projeção \(\pi\) e o segmento \(OA\), por sua vez, é perpendicular a \(\pi\). Os pontos \(A'\) e \(B'\) são, respectivamente, as projeções de \(A\) e \(B\) sobre o plano \(\pi\) com relação ao centro \(O\). Considere as medidas de comprimento \(h = AB\), \(x = OA\), \(h' = A'B'\) e \(d = OA'\). Nosso objetivo é estudar como o comprimento \(h'\) da projeção sobre o plano \(\pi\) se relaciona com o comprimento \(h\) do segmento \(AB\) (você pode imaginar que \(h\) é o comprimento real do seu dedo e \(h'\) é o comprimento da imagem que você vê de seu dedo quando ele está a uma distância \(x\)).

\begin{figure}[H]
\centering
\capstart

\ifnum\aluno=1
\noindent\includegraphics[width=300bp]{{perspectiva-comprimento-01_1}.jpg}
\else
\noindent\includegraphics[width=290bp]{{perspectiva-comprimento-01_1}.jpg}
\fi

\caption{Configuração geométrica simples (versão interativa: \textless{}\url{https://www.geogebra.org/m/HKuqwxXn}\textgreater{}).}\label{\detokenize{GE301-5:fig-proj-comprimento-01}}\label{\detokenize{GE301-5:id34}}\end{figure}

\needspace{.15\textheight}

\paragraph{Etapa 1}

Considere que \(h = 2\) e \(d = 3\).
\begin{enumerate}
\item {} 
Determine o valor de \(h'\) para \(x = 6\).

\item {} 
Mais geralmente, determine \(h'\) como uma função \(f\) de \(x\). Qual é o domínio desta função? Note que, usando o conceito de função, o item anterior está lhe pedindo para calcular \(f(6)\).

\item {} 
Qual deve ser o valor de \(x\) para que o valor de \(h\)‘ correspondente seja igual à metade do valor de \(h'\) que você obteve no primeiro item? Em outras palavras, qual é o valor de \(x\) para o qual \(f(x) = \frac{1}{2} \, f(6)\)?

\item {} 
Qual deve ser o valor de \(x\) para que o valor de \(h\)‘ correspondente seja igual ao dobro do valor de \(h'\) que você obteve no primeiro item? Em outras palavras, qual é o valor de \(x\) para o qual \(f(x) = 2 \, f(6)\)?

\item {} 
Para que valores de \(x\) tem-se \(f(x) = h\)? E \(f(x) > h\)? E \(f(x) < h\)? Interprete no contexto de visualização das imagens de seus dois dedos indicadores em analogia à \hyperref[\detokenize{GE301-5:fig-proj-dedos-03}]{Figura \ref{\detokenize{GE301-5:fig-proj-dedos-03}}}.

\item {} 
Existem valores diferentes de \(x_{1}\) e \(x_{2}\) para os quais \(f(x_{1}) = f(x_{2})\)? Interprete no contexto de visualização das imagens de seus dois dedos indicadores em analogia à \hyperref[\detokenize{GE301-5:fig-proj-dedos-03}]{Figura \ref{\detokenize{GE301-5:fig-proj-dedos-03}}}.

\item {} 
Se os valores de \(x\) vão ficando arbitrariamente grandes, o que se pode dizer a respeito dos valores de \(h'\) correspondentes? Interprete no contexto de visualização das imagens de seus dois dedos indicadores em analogia à \hyperref[\detokenize{GE301-5:fig-proj-dedos-03}]{Figura \ref{\detokenize{GE301-5:fig-proj-dedos-03}}}.

\item {} 
Se os valores de \(x\) vão ficando arbitrariamente próximos de \(0\) com valores maiores do que \(0\), o que se pode dizer a respeito dos valores de \(h'\) correspondentes? Interprete no contexto de visualização das imagens de seus dois dedos indicadores em analogia à \hyperref[\detokenize{GE301-5:fig-proj-dedos-03}]{Figura \ref{\detokenize{GE301-5:fig-proj-dedos-03}}}.

\item {} 
Deseja-se construir um segmento \(CD\) cuja projeção em perspectiva sobre o plano \(\pi\) com relação ao centro \(O\) também seja o segmento \(A'B'\), mas cuja distância até \(O\) seja igual a 15. Qual deve ser o comprimento do segmento \(CD\)?

\end{enumerate}

Justifique todas as respostas!

\needspace{.15\textheight}

\paragraph{Etapa 2}
\begin{enumerate}
\item {} 
Generalize o Item b) da Etapa 1: determine \(h'\) como função de \(x\) em termos de \(h\) e \(d\) (isto é, sem especificar valores numéricos particulares para \(h\) e \(d\)).

\item {} 
Verdadeiro ou falso? No contexto da \hyperref[\detokenize{GE301-5:fig-proj-comprimento-01}]{Figura \ref{\detokenize{GE301-5:fig-proj-comprimento-01}}}, sem atribuir valores numéricos específicos para \(h\) e \(d\), verdadeiro ou falso? Se dobrarmos a distância \(x\) do segmento \(AB\) até o ponto \(O\), então o comprimento \(h'\) de sua projeção ficará reduzido à metade.

\end{enumerate}

Justifique todas as respostas!

\needspace{.10\textheight}

\paragraph{Etapa 3}
\begin{enumerate}
\item {} 
A \hyperref[\detokenize{GE301-5:fig-proj-comprimento-02}]{Figura \ref{\detokenize{GE301-5:fig-proj-comprimento-02}}} foi construída a partir da \hyperref[\detokenize{GE301-5:fig-proj-comprimento-01}]{Figura \ref{\detokenize{GE301-5:fig-proj-comprimento-01}}} acrescentando-se um segmento \(RS\) que é uma “cópia” do segmento \(AB\) obtida transladando-se o segmento \(AB\) paralelamente ao plano \(\pi\). Mais precisamente, \(RS\) é tal que \(ARSB\) é um retângulo que é paralelo ao plano \(\pi\). O segmento \(R'S'\) é a projeção em perspectiva do segmento \(RS\) sobre o plano \(\pi\) com relação ao centro \(O\). Pergunta: o comprimento do segmento \(R'S'\) é maior, menor ou igual ao comprimento \(h'\) do segmento \(A'B'\) que é projeção do segmento \(AB\)? Interprete no contexto de visualização das imagens de seus dois dedos indicadores em analogia à \hyperref[\detokenize{GE301-5:fig-proj-dedos-03}]{Figura \ref{\detokenize{GE301-5:fig-proj-dedos-03}}}.

\begin{figure}[H]
\centering
\capstart

\noindent\includegraphics[width=225bp]{{perspectiva-comprimento-02}.jpg}
\caption{Uma variação da \hyperref[\detokenize{GE301-5:fig-proj-comprimento-01}]{Figura \ref{\detokenize{GE301-5:fig-proj-comprimento-01}}} (versão interativa: \textless{}\url{https://www.geogebra.org/m/u4mkzbmP}\textgreater{}).}\label{\detokenize{GE301-5:fig-proj-comprimento-02}}\label{\detokenize{GE301-5:id35}}\end{figure}

\item {} 
E se, agora, ao invés de um retângulo, o quadrilátero \(ARSB\) fosse um paralelogramo qualquer paralelo ao plano \(\pi\)? Pergunta: o comprimento do segmento \(R'S'\) é maior, menor ou igual ao comprimento \(h'\) do segmento \(A'B'\) que é projeção do segmento \(AB\)?  Interprete no contexto de visualização das imagens de seus dois dedos indicadores em analogia à \hyperref[\detokenize{GE301-5:fig-proj-dedos-03}]{Figura \ref{\detokenize{GE301-5:fig-proj-dedos-03}}}.

\begin{figure}[H]
\centering
\capstart

\noindent\includegraphics[width=225bp]{{perspectiva-comprimento-03}.jpg}
\caption{Outra variação da \hyperref[\detokenize{GE301-5:fig-proj-comprimento-01}]{Figura \ref{\detokenize{GE301-5:fig-proj-comprimento-01}}} (versão interativa: \textless{}\url{https://www.geogebra.org/m/UGFWgAQ5}\textgreater{}).}\label{\detokenize{GE301-5:fig-proj-comprimento-03}}\label{\detokenize{GE301-5:id36}}\end{figure}

\item {} 
Nos dois itens anteriores, o segmento \(RS\) foi considerado como paralelo ao segmento \(AB\). Vamos relaxar esta hipótese, considerando que \(RS\) não precisa ser paralelo a \(AB\), mas que (1) \(RS\) tem o mesmo comprimento \(h\) de \(AB\), (2) \(R = A\) e (3) \(RS\) está contido no plano \(\omega\) que é paralelo a \(\pi\) e que passa por \(A\). Neste caso,  o comprimento do segmento \(R'S'\) é maior, menor ou igual ao comprimento \(h'\) do segmento \(A'B'\) que é projeção do segmento \(AB\)? Interprete no contexto de visualização das imagens de seus dois dedos indicadores em analogia à \hyperref[\detokenize{GE301-5:fig-proj-dedos-03}]{Figura \ref{\detokenize{GE301-5:fig-proj-dedos-03}}}.

\begin{figure}[H]
\centering
\capstart

\noindent\includegraphics[width=250bp]{{perspectiva-comprimento-05_1}.jpg}
\caption{Outra variação da \hyperref[\detokenize{GE301-5:fig-proj-comprimento-01}]{Figura \ref{\detokenize{GE301-5:fig-proj-comprimento-01}}} (versão interativa: \textless{}\url{https://www.geogebra.org/m/SubrgSmG}\textgreater{}).}\label{\detokenize{GE301-5:fig-proj-comprimento-05}}\label{\detokenize{GE301-5:id37}}\end{figure}

\item {} 
Vamos generalizar um pouco mais: agora, \(RS\) é um segmento qualquer que satisfaz duas condições: (1) seu comprimento é igual ao comprimento \(h\) do segmento \(AB\) e (2) \(RS\) está contido no plano \(\omega\) que é paralelo a \(\pi\) e que passa por \(A\). Neste caso, o comprimento do segmento \(R'S'\) é maior, menor ou igual ao comprimento \(h'\) do segmento \(A'B'\) que é projeção do segmento \(AB\)?  Interprete no contexto de visualização das imagens de seus dois dedos indicadores em analogia à \hyperref[\detokenize{GE301-5:fig-proj-dedos-03}]{Figura \ref{\detokenize{GE301-5:fig-proj-dedos-03}}}.

\begin{figure}[H]
\centering
\capstart

\noindent\includegraphics[width=250bp]{{perspectiva-comprimento-04_2}.jpg}
\caption{Ainda outra variação da \hyperref[\detokenize{GE301-5:fig-proj-comprimento-01}]{Figura \ref{\detokenize{GE301-5:fig-proj-comprimento-01}}} (versão interativa: \textless{}\url{https://www.geogebra.org/m/Du4285XX}\textgreater{}).}\label{\detokenize{GE301-5:fig-proj-comprimento-04}}\label{\detokenize{GE301-5:id38}}\end{figure}

\item {} 
Verdadeiro ou falso? Se \(RS\) é um segmento que é paralelo ao plano de projeção \(\pi\), então sua projeção sobre \(\pi\) com relação a um centro \(O\) depende apenas de dois números: a distância \(d\) de \(O\) ao plano \(\pi\) e da distância \(x\) de \(O\) ao plano \(\omega\) que é paralelo a \(\pi\) e que passa por \(R\).

\end{enumerate}

Justifique todas as respostas!

\paragraph{Etapa 4}

As Etapas 1, 2 e 3 trataram da relação entre os comprimentos de segmentos de retas paralelos ao plano de projeção e os comprimentos de suas \emph{projeções em perspectiva} nesse plano. O que dizer de projeções paralelas? Isto é, qual é a relação entre os comprimentos de segmentos de retas paralelos ao plano de projeção e os comprimentos de suas \emph{projeções paralelas} nesse plano? Faça uma conjectura e justifique-a!

\subsection{Parte 2}


Tendo em mente a metáfora da janela de Alberti (\hyperref[\detokenize{GE301-3:fig-proj-janela-de-alberti-03}]{Figura \ref{\detokenize{GE301-3:fig-proj-janela-de-alberti-03}}}), um problema que desafiou artistas, especialmente os renascentistas, foi o de desenhar ladrilhamentos e tabuleiros de xadrez. A \hyperref[\detokenize{GE301-5:fig-proj-ladrilhos-37}]{Figura \ref{\detokenize{GE301-5:fig-proj-ladrilhos-37}}}, a \hyperref[\detokenize{GE301-5:fig-proj-xadrez-10}]{Figura \ref{\detokenize{GE301-5:fig-proj-xadrez-10}}}, a \hyperref[\detokenize{GE301-5:fig-proj-ladrilhos-18}]{Figura \ref{\detokenize{GE301-5:fig-proj-ladrilhos-18}}} e a \hyperref[\detokenize{GE301-5:fig-proj-xadrez-09}]{Figura \ref{\detokenize{GE301-5:fig-proj-xadrez-09}}} ilustram algumas tentativas. Perceba que para produzir um desenho realístico, que se pareça com uma fotografia, os comprimentos dos vários elementos do ladrilhamento e do tabuleiro devem satisfazer as propriedades das projeções em perspectiva. Estudaremos algumas destas propriedades nesta PARTE 2.

\begin{multicols}{2}
\begin{figure}[H]
\centering
\capstart

\noindent\includegraphics[width=\linewidth]{{ladrilhos-37-Rogier_van_der_Weyden_-_Presentation_Miniature_Chroniques_de_Hainaut_KBR_9242}.jpg}
\caption{Miniatura de Rogier van der Weyden (1399/1400 -1464) (fonte: Wikimedia Commons).}\label{\detokenize{GE301-5:fig-proj-ladrilhos-37}}\label{\detokenize{GE301-5:id39}}\end{figure}

\begin{figure}[H]
\centering
\capstart

\noindent\includegraphics[width=\linewidth]{{xadrez-10-Alfonso-LJ-27V}.jpg}
\caption{Miniatura do Livro dos Jogos (1283) (fonte: Wikimedia Commons).}\label{\detokenize{GE301-5:fig-proj-xadrez-10}}\label{\detokenize{GE301-5:id40}}\end{figure}
\end{multicols}


\begin{multicols}{2}
\begin{figure}[H]
\centering
\capstart

\noindent\includegraphics[width=175bp]{{ladrilhos-18-Portrait_of_an_Artist_in_His_Studio_by_Michiel_van_Musscher}.jpg}
\caption{Quadro “Retrato de Um Artista em Seu Estúdio” do pintor holandês Michiel van Musscher (1645-1705) (fonte: Wikimedia Commons).}\label{\detokenize{GE301-5:fig-proj-ladrilhos-18}}\label{\detokenize{GE301-5:id41}}\end{figure}
\columnbreak
\null\vfill
\begin{figure}[H]
\centering
\capstart

\noindent\includegraphics[width=\linewidth]{{xadrez-09-Lucas_van_Leyden_-_The_Game_of_Chess_-_WGA12919}.jpg}
\caption{Quadro “O Jogo de Xadrez” do pintor holandês Lucas van Leyden (1494-1533) (fonte: Wikimedia Commons).}\label{\detokenize{GE301-5:fig-proj-xadrez-09}}\label{\detokenize{GE301-5:id42}}
\end{figure}
\vfill\null
\end{multicols}

\paragraph{Etapa 1}

Considere a \hyperref[\detokenize{GE301-5:fig-proj-ladrilhamentos-01}]{Figura \ref{\detokenize{GE301-5:fig-proj-ladrilhamentos-01}}}. Nela, há dois planos perpendiculares: o plano de projeção \(\pi\) e o plano \(\gamma\) que representa o chão. Um segmento de reta \(RS\) de comprimento \(h\) está contido no plano \(\gamma\) e ele é paralelo ao plano \(\pi\). Na figura, \(P\) é o ponto médio de \(RS\). Como de costume, o ponto \(O\) representa a posição do observador. O ponto \(U\) é a projeção ortogonal de \(O\) sobre \(\gamma\) e, portanto, \(a = OU\) é a altura do observador com relação ao plano do chão \(\gamma\). Agora, uma condição importante que irá simplificar nosso estudo: vamos supor que o ponto \(O\) é tal que o segmento \(OP\) é perpendicular ao segmento \(RS\), ou seja, o ângulo \(OPR\) é reto (na \hyperref[\detokenize{GE301-5:fig-proj-ladrilhamentos-01}]{Figura \ref{\detokenize{GE301-5:fig-proj-ladrilhamentos-01}}} ele não aparenta ser reto por conta da distorção da projeção em perspectiva usada para produzir a figura). Os pontos \(R'\), \(P'\) e \(S'\) são as projeções em perspectiva sobre o plano \(\pi\) com relação ao centro \(O\) dos pontos \(R\), \(P\) e \(S\), respectivamente. O comprimento do segmento projetado \(R'S'\) é \(h'\).
O ponto \(V\) é a interseção do plano \(\pi\) com a reta \(UP\) e, portanto, \(y = VP'\) é a altura do ponto \(P'\) com relação ao plano do chão \(\gamma\).

Como o comprimento \(h'\) do segmento projetado \(R'S'\) varia de acordo com os valores de \(d\), \(h\) e \(x\), você estudou na PARTE 1. O objetivo agora é determinar como a altura \(y\) deste segmento com relação ao plano \(\gamma\) varia de acordo com os valores de \(a\), \(d\) e \(x\). Com essas duas informações será possível criar um método para fazer desenhos em perspectiva de ladrilhamentos e tabuleiros de xadrez com precisão na configuração descrita na \hyperref[\detokenize{GE301-5:fig-proj-ladrilhamentos-01}]{Figura \ref{\detokenize{GE301-5:fig-proj-ladrilhamentos-01}}}.

\begin{figure}[H]
\centering
\capstart

\noindent\includegraphics[width=215bp]{{perspectiva-ladrilhamentos-01_2}.jpg}
\caption{Situação preliminar (versão interativa:\url{https://www.geogebra.org/m/YjAhaCNu}).}\label{\detokenize{GE301-5:fig-proj-ladrilhamentos-01}}\label{\detokenize{GE301-5:id43}}\end{figure}

Suponha que \(a = 3\), \(d = 5\) e \(h = 4\).
\begin{enumerate}
\item {} 
Determine o valor de \(y\) para \(x = 6\).

\item {} 
Mais geralmente, determine \(y\) como função \(g\) de \(x\) para \(x \geq d\). A restrição de que \(x\) seja sempre maior do que ou igual a \(d\) é porque estamos interessados apenas no caso em que o plano \(\pi\) está entre o observador \(O\) e o segmento de reta \(RS\) (o caso de uma pintura).

\item {} 
Qual deve ser o valor de \(x\) para que o valor de \(y\) correspondente seja igual à metade do valor de \(y\) que você obteve no primeiro item? Em outras palavras, qual é o valor de \(x\) para o qual \(g(x) = \frac{1}{2} \, g(6)\)?

\item {} 
Qual deve ser o valor de \(x\) para que o valor de \(y\) correspondente seja igual ao dobro do valor de \(y\) que você obteve no primeiro item? Em outras palavras, qual é o valor de \(x\) para o qual \(g(x) = 2 \, g(6)\)?

\item {} 
Verdadeiro ou falso? Para todo \(x > d\), tem-se \(g(x) < a\).

\item {} 
Existem valores diferentes de \(x_{1} > d\) e \(x_{2} > d\) para os quais \(g(x_{1}) = g(x_{2})\)?

\item {} 
Se os valores de \(x\) vão ficando arbitrariamente grandes, o que se pode dizer a respeito dos valores de \(y\) correspondentes?

\end{enumerate}

Justifique todas as respostas!

\paragraph{Etapa 2}
\begin{enumerate}
\item {} 
Generalize o Item b) da Etapa 1: determine \(y\) como função de \(x\) em termos de \(a\), \(d\) e \(h\) (isto é, sem especificar valores numéricos particulares para \(a\), \(d\) e \(h\).

\item {} 
VNo contexto da \hyperref[\detokenize{GE301-5:fig-proj-ladrilhamentos-01}]{Figura \ref{\detokenize{GE301-5:fig-proj-ladrilhamentos-01}}}, sem atribuir valores numéricos específicos para \(a\), \(d\) e \(h\), verdadeiro ou falso? Se dobrarmos a distância \(x\) do segmento \(RS\) até o ponto \(U\), então a altura \(y\) com relação ao plano \(\gamma\) de sua projeção dobrará também.

\end{enumerate}

Justifique todas as respostas!

\paragraph{Etapa 3}

Desafio final: usando o que você aprendeu até agora nesta atividade, desenhe a projeção em perspectiva do quadriculado \(RABS\) na \hyperref[\detokenize{GE301-5:fig-proj-ladrilhamentos-02}]{Figura \ref{\detokenize{GE301-5:fig-proj-ladrilhamentos-02}}}. Considere \(a = 3\), \(d = 5\), \(x = 6\) e \(h = 4\). O quadrado \(RABS\) está dividido em \(4 \times 4 = 16\) quadrados menores congruentes.

\begin{figure}[H]
\centering
\capstart

\noindent\includegraphics[width=225bp]{{perspectiva-ladrilhamentos-02}.jpg}
\caption{Projeção em perspectiva de um quadriculado \(4 \times 4\) (versão interativa: \textless{}\url{https://www.geogebra.org/m/jGxrcxvw}\textgreater{}).}\label{\detokenize{GE301-5:fig-proj-ladrilhamentos-02}}\label{\detokenize{GE301-5:id44}}\end{figure}

Registre sua resposta na \hyperref[\detokenize{GE301-5:fig-proj-ladrilhamentos-03}]{Figura \ref{\detokenize{GE301-5:fig-proj-ladrilhamentos-03}}} onde, para sua comodidade, já se encontra desenhada a projeção \(R'S'\) do segmento \(RS\) com relação ao centro \(O\).

\begin{figure}[H]
\centering
\capstart

\noindent\includegraphics[width=225bp]{{perspectiva-ladrilhamentos-03}.jpg}
\caption{Plano \(\pi\) com a projeção \(R'S'\) do segmento \(RS\) com relação ao centro \(O\).}\label{\detokenize{GE301-5:fig-proj-ladrilhamentos-03}}\label{\detokenize{GE301-5:id45}}\end{figure}

\paragraph{Etapa 4}

As Etapas 1, 2 e 3 conduziram você a investigar \emph{projeções em perspectiva} de um quadriculado. O que pode ser disto sobre \emph{projeções paralelas} de um quadriculado como o da \hyperref[\detokenize{GE301-5:fig-proj-ladrilhamentos-02}]{Figura \ref{\detokenize{GE301-5:fig-proj-ladrilhamentos-02}}}? Faça uma conjectura e justifique-a!
\end{task}

\clearpage
\begin{objectives}{Perspectiva em perspectiva}
{
Analisar e identificar as implicações de desenhos relacionados com projeções em perspectiva.
}{1}{1}
\end{objectives}
\begin{sugestions}{Perspectiva em perspectiva}
{
O diagrama 3D na \hyperref[\detokenize{GE301-6:fig-proj-diagrama-de-setores-01}]{Figura \ref{\detokenize{GE301-6:fig-proj-diagrama-de-setores-01}}} está disponível como uma construção interativa no GeoGebra: \url{https://www.geogebra.org/m/yFJXfXq2}. É importante observar para os alunos que, por conta das distorções das projeções em perspectiva, diagramas 3D estáticos podem ser interpretados de forma equivocada! Neste caso, um diagrama 2D é preferível.
}{1}{1}
\end{sugestions}
\begin{answer}{Perspectiva em perspectiva}
{
\begin{enumerate}
\item A seguir, listaremos alguns dos erros visíveis da gravura, sendo que outros ainda podem ser detectados:
\begin{itemize}
\item {} 
O homem de pé no canto direito da tela está segurando uma vara de pescar cujo anzol passa pelo outro homem que está pescando na margem do rio.

\item {} 
A flâmula está presa alinhadamente em dois prédios distintos de uma maneira que não condiz com a realidade.

\item {} 
A flâmula está sobreposta por duas árvores que estão plantadas no morro que fica atrás dos prédios.

\item {} 
O homem que está bem no alto do morro, tem acesso a uma pessoa que está na janela do prédio.

\item {} 
É possível visualizar duas portas da igreja que estão posicionadas em direções perpendiculares.

\item {} 
Os tamanhos dos animais crescem a medida que eles se afastam no horizonte.

\item {} 
O telhado do prédio que contém uma pessoa na janela pode ser visto, embora a cena esteja numa perspectiva de baixo para cima.

\end{itemize}
\end{enumerate}
}{1}
\end{answer} 
\clearmargin
\begin{answer}{Perspectiva em perspectiva}
{
\begin{enumerate}\setcounter{enumi}{1}
\item 
\begin{itemize}
\item {} 
Com as informações obtidas no diagrama de setores 3D não é possível identificar qual é mais frequente entre A ou C, e entre B ou D.

\item {} 
Dadas as porcentagens de cada item, fica claro que o diagrama apresentado na figura acima realmente não representa a real situação de frequência dos itens A, B, C, e D. Veja a versão 2D do diagrama na \hyperref[\detokenize{GE301-6:fig-proj-graficosetores-sol}]{Figura \ref{\detokenize{GE301-6:fig-proj-graficosetores-sol}}}. Esta imagem foi gerada na construção interativa feita no Geogebra e disponível em \url{https://www.geogebra.org/m/yFJXfXq2}.

\end{itemize}

\begin{figure}[H]
\centering
\capstart

\noindent\includegraphics[width=200bp]{{GraficoSetor2D}.png}
\caption{Diagrama de setores 2D.}\label{\detokenize{GE301-6:fig-proj-graficosetores-sol}}\label{\detokenize{GE301-6:id7}}
\end{figure}

Pela versão 2D do diagrama, conseguimos determinar que o item A é o mais frequente entre A e C, e B é o item mais frequente entre B e D.

\item {} 
As projeções das bases dos dois paralelepípedos não são paralelas e congruentes, logo a imagem não representa a projeção perspectiva dos dois blocos.
\end{enumerate}
}{1}
\end{answer}
\clearmargin
\begin{answer}{Perspectiva em perspectiva}
{
\begin{enumerate}\setcounter{enumi}{2}
\item Observe na figura a seguir que as retas desenhadas em vermelho sobre o icosaedro regular são paralelas. Se a imagem tivesse sido feita em projeção paralela, todas as três retas vermelhas deveriam ser paralelas. Como não são, a imagem não é uma projeção paralela do icosaedro.  Se a imagem tivesse sido feita em projeção perspectiva, todas as três retas vermelhas deveriam se encontrar em um único ponto, o centro de projeção. Como isto também não acontece, a imagem não é uma projeção perspectiva do icosaedro. Sendo assim, o emblema da Mathematical Association of America não foi feito em projeção paralela nem perspectiva.

\begin{figure}[H]
\centering
\capstart

\noindent\includegraphics[width=150bp]{{emblemaMMA_retasparalelas}.png}
\caption{Retas paralelas sobre um icosaedro regular.}\label{\detokenize{GE301-6:fig-proj-emblemamaa-sol}}\label{\detokenize{GE301-6:id8}}
\end{figure}
\end{enumerate}
}{1}
\end{answer}
\begin{objectives}{}
{
Julgar situações com foco nos distratores (retas que não são concorrentes mas cujas projeções o são,  pontos que não são colineares mas cujas projeções o são, distorções de comprimento e ângulo).
}{1}{1}
\end{objectives}
\begin{sugestions}{}
{
As construções interativas feitas com o GeoGebra estão disponíveis nos endereços
\url{https://www.geogebra.org/m/aGaRUudY},
\url{https://www.geogebra.org/m/jNFQMfhH},
\url{https://www.geogebra.org/m/C4GQJnVk} e
}{1}{1}
\end{sugestions}
\clearmargin
\begin{answer}{}
{
\begin{enumerate}
\item Não. As retas \(BD\) e \(EG\) estão contidas em faces opostas do cubo, portanto elas não se encontram. Veja a figura abaixo.

\begin{figure}[H]
\centering

\noindent\includegraphics[width=160bp]{{Distrator_a}.png}
\end{figure}

Utilize a construção interativa disponível em \url{https://www.geogebra.org/m/aGaRUudY} para visualizar melhor a situação.

\item {} 
Primeiramente, repare que os pontos \(P, C\) e \(Q\) estão contidos no plano da face \(CDHG\) que é perpendicular à reta \(r\). Além disso, \(r\) intersecta tal plano em \(G\), o que implica que o ponto (entre \(P, C\) e \(Q\)) mais próximo de \(r\) será também o ponto mais próximo de \(G\). Observe a figura abaixo.

\begin{figure}[H]
\centering

\noindent\includegraphics[width=200bp]{{Distrator_b}.png}
\end{figure}

Observe que \(CG\) é perpendicular à reta \(r\), logo a distância de \(C\) à reta \(r\) é dada pelo comprimento do segmento de reta \(CG\). \(PCG\) é um triângulo retângulo onde \(PG\) é sua hipotenusa e \(CG\) é o seu cateto, então o comprimento de \(PG\) é maior que o comprimento de \(CG\). \(QCG\) é também um triângulo retângulo onde \(QG\) é sua hipotenusa e \(CG\) é o seu cateto, então o comprimento de \(QG\) é maior que o comprimento de \(CG\). Assim, \(CG\) é menor que \(PG\) e \(QG\) ao mesmo tempo. Portanto, \(C\) é o ponto mais próximo de \(G\), e consequentemente, o ponto mais próximo de \(r\).

Para um melhor entendimento do que foi discutido acima, acesse a construção interativa disponível em \url{https://www.geogebra.org/m/C4GQJnVk}.
\end{enumerate}
}{1}
\end{answer}
\clearmargin
\begin{answer}{}
{
\begin{enumerate}
\item Não. A reta \(DM\) está contida no plano que contém a face \(ABCD\) do cubo e portanto, não pode cortar sua face oposta \(EFGH\) no ponto \(N\) como sugerido por Simplício. Veja isso na figura abaixo.

\begin{figure}[H]
\centering

\noindent\includegraphics[width=130bp]{{Distrator_c}.png}
\end{figure}

A construção interativa disponível em \url{https://www.geogebra.org/m/jNFQMfhH} pode te ajudar a entender melhor a situação.

\item {} \begin{enumerate}
\item {} 
Não. De fato, \(DM = MF = FN = ND\), pois esses segmentos são hipotenusas de triângulos retângulos formados por catetos que medem o mesmo que a aresta do cubo e metade dela. Mas isto não implica que \(DMFN\) é um quadrado, e portanto seus ângulos internos são retos. Com a congruência de todos os lados de \(DMFN\) podemos afirmar que o quadrilátero é um losango, mas não podemos afirmar que é um quadrado.

\item {} 
Não. Observe que as diagonais do quadrilátero \(DMFN\) são os segmentos \(FD\) e \(MN\), onde \(FD\) é também a diagonal do cubo e \(MN\) possui o mesmo comprimento da diagonal de uma face do cubo. Como a diagonal do cubo é maior que a diagonal da face, então \(FD\) e \(MN\) não são congruentes, o que implica que o quadrilátero \(DMFN\) não é um quadrado. Assim, o ângulo \(MFN\) não é reto.

\end{enumerate}
\end{enumerate}
}{1}
\end{answer}
\clearmargin
\begin{objectives}{Vistas ortogonais}
{
Compreender as projeções ortogonais simultâneas de objetos 3D sobre três planos, dois a dois perpendiculares, as quais constituem as assim denominadas \textit{vistas ortogonais} principais.
}{1}{1}
\end{objectives}
\begin{sugestions}{Vistas ortogonais}
{
Como preparação para essa atividade, recomendamos
o aplicativo “Projeções Ortogonais” que está
disponível no endereço: \url{http://www.cdme.im-uff.mat.br/html5/pro/}. Com ele, é possível visualizar
diversos objetos tridimensionais e suas projeções ortogonais nos três planos coordenados.
}{1}{1}
\end{sugestions}
\clearmargin
\clearmargin

\begin{answer}{Vistas ortogonais}
{\paragraph{Etapa 1}

\begin{enumerate}
\begin{multicols}{2}
\item {} 
\adjustbox{valign=t}
{
\includegraphics[width=.9\linewidth]{{vistasortogonais_cubo}.png}
}

\item {} 
\adjustbox{valign=t}
{
\includegraphics[width=.9\linewidth]{{vistasortogonais_cubogirado}.png}
}
\end{multicols}

\begin{multicols}{2}
\item {} 
\adjustbox{valign=t}
{
\includegraphics[width=.9\linewidth]{{vistasortogonais_esfera}.png}
}

\item {} 
\adjustbox{valign=t}
{
\includegraphics[width=.9\linewidth]{{vistasortogonais_cilindro}.png}
}
\end{multicols}

\item {} 
\adjustbox{valign=t}
{
\includegraphics[width=.45\linewidth]{{vistasortogonais_cone2}.png}
}
\end{enumerate}
}{1}
\end{answer}
\clearmargin
\begin{answer}{Vistas ortogonais}
{
\paragraph{Etapa 2}

A seguir serão apresentados os objetos cujas projeções ortogonais sobre os planos frontal, horizontal e lateral foram dadas anterioramente.
\begin{enumerate}
\begin{multicols}{2}
\item {} 
\adjustbox{valign=t}
{
\includegraphics[width=.9\linewidth]{{Etapa2_a_1}.png}
}

\item {} 
\adjustbox{valign=t}
{
\includegraphics[width=.9\linewidth]{{Etapa2_b_1}.png}
}
\end{multicols}

\begin{multicols}{2}
\item {} 
\adjustbox{valign=t}
{
\includegraphics[width=.9\linewidth]{{Etapa2_c_1}.png}
}

\item {} 
\adjustbox{valign=t}
{
\includegraphics[width=.9\linewidth]{{Etapa2_d_1}.png}
}
\end{multicols}

\begin{multicols}{2}
\item {} 
\adjustbox{valign=t}
{
\includegraphics[width=.9\linewidth]{{Etapa2_e_1}.png}
}


\item {} 
\adjustbox{valign=t}
{
\includegraphics[width=.9\linewidth]{{Etapa2_f_1}.png}
}
\end{multicols}

\item {} 
\adjustbox{valign=t}
{
\includegraphics[width=.45\linewidth]{{Etapa2_g_1}.png}
}
\end{enumerate}
}{1}
\end{answer}
\clearmargin
\begin{answer}{Vistas ortogonais}
{
\paragraph{Etapa 3}
A seguir serão apresentados dois objetos cujas projeções ortogonais sobre os planos frontal, horizontal e lateral são quadrados de mesmo lado. Um dos objetos é um cubo com faces paralelas aos planos de projeção e o outro é uma linha poligonal fechada formada por segmentos de reta coincidentes com 6 arestas do cubo feito anteriormente. Veja as figuras abaixo.

\begin{multicols}{2}
\begin{figure}[H]
\centering

\noindent\includegraphics[width=.9\linewidth]{{Etapa3_1}.png}
\end{figure}

\begin{figure}[H]
\centering

\noindent\includegraphics[width=.9\linewidth]{{Etapa3_2}.png}
\end{figure}
\end{multicols}

Na verdade, esta não é a única linha poligonal fechada que possui quadrados como projeções ortogonais sobre os planos frontal, horizontal e lateral. Veja na figura abaixo para um outro exemplo.

\begin{figure}[H]
\centering

\noindent\includegraphics[width=.45\linewidth]{{Etapa3_3}.png}
\end{figure}


\paragraph{Etapa 4}

A estrutura 3D procurada pode ser vista na figura abaixo.

\notasfig{\begin{figure}[H]
\centering

\noindent\includegraphics[width=.45\linewidth]{{objeto_etapa4}.png}
\end{figure}}

Neste caso, as vistas do objeto no formato mostrado na \hyperref[\detokenize{GE301-6:fig-proj-vistas-ortogonais-04}]{Figura \ref{\detokenize{GE301-6:fig-proj-vistas-ortogonais-04}}} (C), incluindo a lateral, podem ser encontradas na figura a seguir.

\notasfig{\begin{figure}[H]
\centering

\noindent\includegraphics[width=.45\linewidth]{{vistasortogonais_objeto_etapa4}.png}
\end{figure}}

Na verdade, este é apenas um exemplo de um objeto que possui as vistas frontal e superior dadas. É possível encontrar outros objetos que possuam essas mesmas vistas.
}{1}
\end{answer} 

\practice{}
\label{\detokenize{GE301-6::doc}}\label{\detokenize{GE301-6:praticando-3}}\phantomsection\label{\detokenize{GE301-6:ativ-proj-perspectiva-em-perspectiva}}
\begin{task}{Perspectiva em perspectiva}

\begin{enumerate}
\item {} 
Com o objetivo de alertar as pessoas sobre as inconsistências que podem ocorrer caso um desenho seja feito sem o conhecimento das propriedades das projeções em perspectiva, o pintor (e também cartunista, crítico social e satirista) inglês William Hogarth FRSA (1697-1764) produziu uma gravura intitulada “Sátira sobre a Falsa Perspectiva” a qual apresenta alguns exemplos intencionais de efeitos confusos e equívocos de perspectiva. Tente identificar esses elementos estranhos na gravura!

\begin{figure}[H]
\centering

\noindent\includegraphics[width=.85\linewidth]{{perspectiva-em-perspectiva-01}.jpg}
\end{figure}

\item {} 
Muitas pessoas gostam de fazer diagramas estatísticos em 3D, como na figura a seguir.

\begin{figure}[H]
\centering
\capstart

\noindent\includegraphics[width=250bp]{{perspectiva-diagrama-de-setores-01}.jpg}
\caption{Diagrama de setores 3D.}\label{\detokenize{GE301-6:fig-proj-diagrama-de-setores-01}}\label{\detokenize{GE301-6:id5}}\end{figure}
\begin{itemize}
\item {} 
Segundo este diagrama, qual item é mais frequente? A ou C? B ou D?

\item {} 
O diagrama foi construído com as seguintes frequências relativas: Item A com \(11\%\), Item B com \(42\%\), Item C com \(5\%\) e Item D com \(42\%\). Construa, com estes dados, uma versão 2D do diagrama de setores da \hyperref[\detokenize{GE301-6:fig-proj-diagrama-de-setores-01}]{Figura \ref{\detokenize{GE301-6:fig-proj-diagrama-de-setores-01}}}. Com este diagrama 2D, qual item é mais frequente? A ou C? B ou D?

\end{itemize}

\item {} 
Por que a imagem abaixo não pode ser uma projeção em perspectiva de dois blocos, isto é, dois paralelepípedos retos retângulos?

\begin{figure}[H]
\centering

\noindent\includegraphics[width=350bp]{{perspectiva-em-perspectiva-05}.jpg}
\end{figure}

\item {} 
(Grunbaum, 1985) A sociedade profissional americana \emph{Mathematical Association of America} (MAA) tem como emblema um icosaedro regular. A imagem a seguir exibe o emblema na capa de uma edição de 1984 da revista \emph{Mathematics Magazine} publicada pela MAA. O matemático Branko Grunbaum percebeu que o icosaedro da imagem não pode ser nem uma projeção em perspectiva e nem uma projeção paralela de um icosaedro regular. Por quê? Dica: tente analisar os elementos que são paralelos em um icosaedro regular tridimensional e como estes elementos estão projetados na imagem.

\begin{figure}[H]
\centering
\capstart

\noindent\includegraphics[width=150bp]{{perspectiva-em-perspectiva-04}.jpg}
\caption{Emblema  capa de uma edição de 1984 da revista \emph{Mathematics Magazine} publicada pela Mathematics Association of America.}\label{\detokenize{GE301-6:id6}}\end{figure}

\end{enumerate}
\end{task}

\phantomsection\label{\detokenize{GE301-6:ativ-proj-distratores}}

\begin{task}{}
\url{https://www.geogebra.org/m/bd5f8KTg}.
\begin{enumerate}
\item {} 
Simplício está estudando Geometria Espacial em um livro e se depara com a figura a seguir.
\begin{figure}[H]
\centering

\begin{asy}
size(6cm);

currentprojection=orthographic(3,1.5,3.5);

triple a = (0,0,0);
triple b = (1,0,0);
triple c = (1,1,0);
triple d = (0,1,0);

triple e = (0,0,1);
triple f = (1,0,1);
triple g = (1,1,1);
triple h = (0,1,1);

triple j = (.5,1,0);

draw (a -- b -- c -- d -- cycle);
draw (e -- f -- g -- h -- cycle);
draw (a -- b -- f -- e -- cycle);
draw (c -- d -- h -- g -- cycle);


label ("H", (a), align=NW);
label ("E", (b), align=SW);
label ("F", (c), align=S);
label ("G", (d), align=SE);

label ("D", (e), align=NW);
label ("A", (f), align=NW);
label ("B", (g), align=SW);
label ("C", (h), align=NE);

real f(real x){return -x+1;}
path s1 = graph(f,-.3,1.3);

path3 d1 = path3(s1);

draw(d1);


real f(real x){return x;}
path s2 = graph(f,-.2,2);

path3 d2 = shift(0,0,1)*path3(s2);

draw(d2);

triple [] array={a,b,c,d,e,f,g,h};
for(triple i:array) {
dot(i, linewidth(1.5));
}
\end{asy}
\label{\detokenize{GE301-6:fig-proj-distratores-1}}\end{figure}

O livro diz que \(ABCDEFGH\) é uma projeção em perspectiva de um cubo e pergunta quantos pontos de interseção existem entre as retas \(BD\) e \(EG\). Simplício responde: “Pergunta fácil! Existe um único ponto de interseção entre \(GD\) e \(EG\). Este ponto \(P\) aqui, como podemos ver claramente!”.

\begin{figure}[H]
\centering

\begin{asy}
size(6cm);

currentprojection=orthographic(3,1.5,3.5);

triple a = (0,0,0);
triple b = (1,0,0);
triple c = (1,1,0);
triple d = (0,1,0);

triple e = (0,0,1);
triple f = (1,0,1);
triple g = (1,1,1);
triple h = (0,1,1);

triple j = (.5,1,0);

draw (a -- b -- c -- d -- cycle);
draw (e -- f -- g -- h -- cycle);
draw (a -- b -- f -- e -- cycle);
draw (c -- d -- h -- g -- cycle);


label ("H", (a), align=NW);
label ("E", (b), align=SW);
label ("F", (c), align=S);
label ("G", (d), align=SE);

label ("D", (e), align=NW);
label ("A", (f), align=NW);
label ("B", (g), align=SW);
label ("C", (h), align=NE);

real f(real x){return -x+1;}
path s1 = graph(f,-.3,1.3);

path3 d1 = path3(s1);

draw(d1);


real f(real x){return x;}
path s2 = graph(f,-.2,2);

path3 d2 = shift(0,0,1)*path3(s2);

draw(d2);

triple [] array={a,b,c,d,e,f,g,h};
for(triple i:array) {
dot(i, linewidth(1.5));
}

dot((.86,1,.67), linewidth(2.5));

draw ((.7,1,.9) -- (.8,.98,.67), arrow=Arrow3(TeXHead2));
\end{asy}
\label{\detokenize{GE301-6:fig-proj-distratores-2}}\end{figure}

Você concorda com a resposta de Simplício? Por que sim? Por que não?

\item {} 
(Adaptado de \citealp{Lellis-2009}) Na figura a seguir \(ABCDEFGH\) é uma projeção paralela de um cubo. Qual ponto está mais próximo da reta \(r = FG\)? O ponto \(P\), o ponto \(C\) ou o ponto \(Q\)?

\begin{figure}[H]
\centering

\begin{asy}
size(6.5cm);

currentprojection=orthographic(3,1,1);

triple a = (0,0,0);
triple b = (1,0,0);
triple c = (1,1,0);
triple d = (0,1,0);

triple e = (0,0,1);
triple f = (1,0,1);
triple g = (1,1,1);
triple h = (0,1,1);

triple j = (.5,1,0);

triple k = (0,.3,1);
triple l = (0,1.5,1);

draw (a -- b -- c -- d -- cycle);
draw (e -- f -- g -- h -- cycle);
draw (a -- b -- f -- e -- cycle);
draw (c -- d -- h -- g -- cycle);


label ("H", (a), align=NW);
label ("E", (b), align=SW);
label ("F", (c), align=S);
label ("G", (d), align=SE);

label ("D", (e), align=NW);
label ("A", (f), align=NW);
label ("B", (g), align=SW);
label ("C", (h), align=NE);

label ("P", (k), align=N);
label ("Q", (l), align=N);

real f(real x){return 1;}
path s1 = graph(f,-3,2);

path3 d1 = path3(s1);

draw(d1);


draw((0,0,1) -- (0,1.5,1));

triple [] array={a,b,c,d,e,f,g,h,k,l};
for(triple i:array) {
dot(i, linewidth(1.5));
}

label("$r$", (-2,1,0),SE);
\end{asy}
\label{\detokenize{GE301-6:fig-proj-distratores-3}}\end{figure}

\item {} 
(Adaptado de \citealp{Volkert-2008}) Eis outra pergunta do livro de Geometria Espacial que Simplício está estudando: “Existem três pontos distintos, cada um em arestas distintas de um cubo e que sejam colineares?”.

\begin{figure}[H]
\centering

\begin{asy}
size(5cm);

currentprojection=orthographic(3,1,.5);

triple a = (0,0,0);
triple b = (1,0,0);
triple c = (1,1,0);
triple d = (0,1,0);

triple e = (0,0,1);
triple f = (1,0,1);
triple g = (1,1,1);
triple h = (0,1,1);

triple j = (.5,1,0);

draw (a -- b -- c -- d -- cycle);
draw (e -- f -- g -- h -- cycle);
draw (a -- b -- f -- e -- cycle);
draw (c -- d -- h -- g -- cycle);


label ("H", (a), align=NW);
label ("E", (b), align=SW);
label ("F", (c), align=S);
label ("G", (d), align=SE);

label ("D", (e), align=NW);
label ("A", (f), align=NW);
label ("B", (g), align=SW);
label ("C", (h), align=NE);
\end{asy}

\end{figure}

Simplício pensa: “Existem sim! Eu construo o ponto médio \(M\) da aresta \(AB\) e o ponto médio \(N\) da aresta \(FG\). Trançando o segmento \(DN\), vejo que ele passa por \(M\). Pronto: os pontos \(D\), \(M\) e \(N\) são distintos, cada um está em uma aresta diferente e eles são colineares!”.

\begin{figure}[H]
\centering

\begin{asy}
size(5cm);

currentprojection=orthographic(2,1/2,.75);

triple a = (0,0,0);
triple b = (1,0,0);
triple c = (1,1,0);
triple d = (0,1,0);

triple e = (0,0,1);
triple f = (1,0,1);
triple g = (1,1,1);
triple h = (0,1,1);

triple j = (.5,1,0);

draw (a -- b -- c -- d -- cycle);
draw (e -- f -- g -- h -- cycle);
draw (a -- b -- f -- e -- cycle);
draw (c -- d -- h -- g -- cycle);

draw(e -- j);

label ("H", (a), align=NW);
label ("E", (b), align=SW);
label ("F", (c), align=S);
label ("G", (d), align=SE);

label ("D", (e), align=NW);
label ("A", (f), align=NW);
label ("B", (g), align=SW);
label ("C", (h), align=NE);

label ("N", (j), align=SE);

label ("M", (f+(0,.525,0)), align=NE);

dot(f+(0,.525,0), linewidth(1.5));

dot(a, linewidth(1.5));
dot(b, linewidth(1.5));
dot(c, linewidth(1.5));
dot(d, linewidth(1.5));
dot(e, linewidth(1.5));
dot(f, linewidth(1.5));
dot(g, linewidth(1.5));
dot(h, linewidth(1.5));
dot(j, linewidth(1.5));

\end{asy}
\end{figure}

Você concorda com a resposta de Simplício? Por que sim? Por que não?

\item {} 
(Adaptado de \citealp{Fujita-et-al-2017}) No seu livro de Geometria Espacial, Simplício lê o enunciado de uma questão: “Na figura a seguir, \(ABCDEFGH\) é um cubo, \(M\) é ponto médio da aresta \(AE\) e \(N\) é ponto médio da aresta \(CG\). O ângulo \(MFN\) é reto? Justifique sua resposta!”.

\begin{figure}[H]
\centering
\begin{asy}
size(5cm);

currentprojection=orthographic(3,1,.5);

triple a = (0,0,0);
triple b = (1,0,0);
triple c = (1,1,0);
triple d = (0,1,0);

triple e = (0,0,1);
triple f = (1,0,1);
triple g = (1,1,1);
triple h = (0,1,1);

triple i = (1,0,.5);
triple j = (0,1,.5);

draw (i -- e -- j, dashed);
draw (i -- c -- j);


draw (a -- b -- c -- d -- cycle);
draw (e -- f -- g -- h -- cycle);
draw (a -- b -- f -- e -- cycle);
draw (c -- d -- h -- g -- cycle);

label ("H", (a), align=NW);
label ("E", (b), align=SW);
label ("F", (c), align=S);
label ("G", (d), align=SE);

label ("D", (e), align=NW);
label ("A", (f), align=NW);
label ("B", (g), align=NW);
label ("C", (h), align=NE);

label ("M", (i), align=W);
label ("N", (j), align=E);

triple [] array={a,b,c,d,e,f,g,h,i,j};
for(triple j:array){
	dot(j, linewidth(1.5));
}
\end{asy}
\end{figure}

Simplício dá como resposta “Sim, o ângulo \(MFN\) é reto!” e dá como justificativa “O quadrilátero \(DMFN\) é um quadrado, pois \(DM = MF = FN = ND\) e, sendo um quadrado, seus ângulos internos são todos retos!”.
\begin{enumerate}
\item {} 
A justificativa de Simplício está correta? Justifique sua resposta!

\item {} 
A resposta de Simplício está correta? Justifique sua resposta.

\end{enumerate}

\end{enumerate}
\end{task}

\phantomsection\label{\detokenize{GE301-6:ativ-proj-vistas-ortogonais}}
\begin{task}{Vistas ortogonais}

Em desenho técnico, uma prática comum para se representar objetos 3D (como o objeto em (A) na \hyperref[\detokenize{GE301-6:fig-proj-vistas-ortogonais-03}]{Figura \ref{\detokenize{GE301-6:fig-proj-vistas-ortogonais-03}}}) é o de de projetá-lo ortogonalmente sobre três planos que são dois a dois perpendiculares (como os planos em (B) na \hyperref[\detokenize{GE301-6:fig-proj-vistas-ortogonais-03}]{Figura \ref{\detokenize{GE301-6:fig-proj-vistas-ortogonais-03}}}).

Tipicamente, como em (C) na \hyperref[\detokenize{GE301-6:fig-proj-vistas-ortogonais-03}]{Figura \ref{\detokenize{GE301-6:fig-proj-vistas-ortogonais-03}}}, os planos são posicionados de forma a ficarem, na medida do possível, parelelos às faces do objeto 3D (isto quando, naturalmente, o objeto tem faces planas). Este tipo de escolha tem uma vantagem: as projeções das faces paralelas sobre um dos planos de projeção serão congruentes às faces originais. Em termos de desenho técnico, as projeções estarão em \index{verdadeira grandeza}verdadeira grandeza. No contexto de construção de peças e equipamentos, esta congruência é um dos motivos para o uso de projeções ortogonais, em oposição às projeções em perspectiva, para representações 2D de objetos 3D. Como faces não paralelas a um plano de projeção vão aparecer distorcidas, três planos são considerados, cada plano representando em verdadeira grandeza os elementos do objeto que lhe são paralelos.

Os três planos são denominados de \index{plano frontal}plano frontal, \index{plano horizontal}plano horizontal e \index{plano lateral}plano lateral. A atribuição de um destes nomes a um determinado plano é uma escolha arbitrária, em princípio pois, afinal, ao se girar o objeto 3D, podemos converter uma escolha de nomes em outra. As imagens (D), (E) e (F) exibem uma atribuição de nomes. É claro, se o objeto tem naturalmente uma base horizontal (por exemplo, o fundo de uma caixa), então é razoável denominar o plano paralelo a essa base de plano horizontal. Do mesmo modo, se objeto tem naturalmente uma frente, então é conveniente denominar o plano paralelo a esta frente de plano frontal. Uma explicação análoga pode ser dada para a escolha do plano lateral.


\begin{minipage}{\linewidth}
\begin{figure}[H]
\centering
\begin{multicols}{3}

\begin{figure}[H]
\centering

\begin{asy}
size(3.7cm);

currentprojection=orthographic(1.5,1.5,1/2);

triple A = (.4,.4,.35);
triple B = (.4,.7,.35);
triple C = (.7,.7,.35);
triple D = (.7,.4,.35);

triple E = (A+(0,0,.1));
triple F = (B+(0,0,.1));
triple G = (E+(.1,0,0));
triple H = (F+(.1,0,0));

triple I = (C+(0,0,.3));
triple J = (H+(0,0,.2));

triple K = (D+(0,0,.4));
triple L = (G+(0,0,.3));

triple M = (K+(0,.1,0));
triple N = (L+(0,.1,0));

draw(B -- C -- I -- J -- H -- F -- cycle);

draw(C -- D -- K -- M -- I -- cycle);
draw(D -- K -- L -- G -- E -- A -- cycle);
draw(K -- L -- N -- M -- cycle);
draw(M -- N -- J -- I -- cycle);

draw (A -- B -- C -- D -- cycle);
draw (E -- F -- H -- G -- cycle);
draw (A -- B -- F -- E -- cycle);

draw(surface(B -- C -- I -- J -- H -- F -- cycle), verde*80);

draw(surface(C -- D -- K -- M -- I -- cycle), verde*80);
draw(surface(D -- K -- L -- G -- E -- A -- cycle), verde*80);
draw(surface(K -- L -- N -- M -- cycle), verde*80);
draw(surface(M -- N -- J -- I -- cycle), verde*80);

draw (surface(A -- B -- C -- D -- cycle), verde*80);
draw (surface(E -- F -- H -- G -- cycle), verde*80);
draw (surface(A -- B -- F -- E -- cycle), verde*80);

draw (surface(H -- J -- N --L -- G -- cycle), verde*80);

dot(A, linewidth(2));
dot(B, linewidth(2));
dot(C, linewidth(2));
dot(D, linewidth(2));
dot(E, linewidth(2));
dot(F, linewidth(2));
dot(G, linewidth(2));
dot(H, linewidth(2));
dot(I, linewidth(2));
dot(J, linewidth(2));
dot(K, linewidth(2));
dot(L, linewidth(2));
dot(M, linewidth(2));
dot(N, linewidth(2));

triple a = (0,0,0);
triple b = (1,0,0);
triple c = (1,1,0);
triple d = (0,1,0);

triple e = (0,0,1);
triple f = (1,0,1);
triple g = (1,1,1);
triple h = (0,1,1);

draw ((a -- b -- f -- e -- cycle), white);
draw ((a -- d -- h -- e -- cycle), white);
draw ((a -- b -- c -- d -- cycle), white);
\end{asy}

(A)
\end{figure}

\begin{figure}[H]
\centering

\begin{asy}
size(3.7cm);

currentprojection=orthographic(1.5,1.5,1/2);

triple A = (.4,.4,.35);
triple B = (.4,.7,.35);
triple C = (.7,.7,.35);
triple D = (.7,.4,.35);

triple E = (A+(0,0,.1));
triple F = (B+(0,0,.1));
triple G = (E+(.1,0,0));
triple H = (F+(.1,0,0));

triple I = (C+(0,0,.3));
triple J = (H+(0,0,.2));

triple K = (D+(0,0,.4));
triple L = (G+(0,0,.3));

triple M = (K+(0,.1,0));
triple N = (L+(0,.1,0));

//draw(B -- C -- I -- J -- H -- F -- cycle);

//draw(C -- D -- K -- M -- I -- cycle);
//draw(D -- K -- L -- G -- E -- A -- cycle);
//draw(K -- L -- N -- M -- cycle);
//draw(M -- N -- J -- I -- cycle);

//draw (A -- B -- C -- D -- cycle);
//draw (E -- F -- H -- G -- cycle);
//draw (A -- B -- F -- E -- cycle);

//draw(surface(B -- C -- I -- J -- H -- F -- cycle), verde*80);

//draw(surface(C -- D -- K -- M -- I -- cycle), verde*80);
//draw(surface(D -- K -- L -- G -- E -- A -- cycle), verde*80);
//draw(surface(K -- L -- N -- M -- cycle), verde*80);
//draw(surface(M -- N -- J -- I -- cycle), verde*80);

//draw (surface(A -- B -- C -- D -- cycle), verde*80);
//draw (surface(E -- F -- H -- G -- cycle), verde*80);
//draw (surface(A -- B -- F -- E -- cycle), verde*80);

triple a = (0,0,0);
triple b = (1,0,0);
triple c = (1,1,0);
triple d = (0,1,0);

triple e = (0,0,1);
triple f = (1,0,1);
triple g = (1,1,1);
triple h = (0,1,1);

draw ((a -- b -- f -- e -- cycle));
draw ((a -- d -- h -- e -- cycle));
draw ((a -- b -- c -- d -- cycle));
\end{asy}

(B)
\end{figure}

\begin{figure}[H]
\centering

\begin{asy}
size(3.7cm);

currentprojection=orthographic(1.5,1.5,1/2);

triple A = (.4,.4,.35);
triple B = (.4,.7,.35);
triple C = (.7,.7,.35);
triple D = (.7,.4,.35);

triple E = (A+(0,0,.1));
triple F = (B+(0,0,.1));
triple G = (E+(.1,0,0));
triple H = (F+(.1,0,0));

triple I = (C+(0,0,.3));
triple J = (H+(0,0,.2));

triple K = (D+(0,0,.4));
triple L = (G+(0,0,.3));

triple M = (K+(0,.1,0));
triple N = (L+(0,.1,0));

draw(B -- C -- I -- J -- H -- F -- cycle);

draw(C -- D -- K -- M -- I -- cycle);
draw(D -- K -- L -- G -- E -- A -- cycle);
draw(K -- L -- N -- M -- cycle);
draw(M -- N -- J -- I -- cycle);

draw (A -- B -- C -- D -- cycle);
draw (E -- F -- H -- G -- cycle);
draw (A -- B -- F -- E -- cycle);

draw(surface(B -- C -- I -- J -- H -- F -- cycle), verde*80);

draw(surface(C -- D -- K -- M -- I -- cycle), verde*80);
draw(surface(D -- K -- L -- G -- E -- A -- cycle), verde*80);
draw(surface(K -- L -- N -- M -- cycle), verde*80);
draw(surface(M -- N -- J -- I -- cycle), verde*80);

draw (surface(A -- B -- C -- D -- cycle), verde*80);
draw (surface(E -- F -- H -- G -- cycle), verde*80);
draw (surface(A -- B -- F -- E -- cycle), verde*80);

draw (surface(H -- J -- N --L -- G -- cycle), verde*80);


triple a = (0,0,0);
triple b = (1,0,0);
triple c = (1,1,0);
triple d = (0,1,0);

triple e = (0,0,1);
triple f = (1,0,1);
triple g = (1,1,1);
triple h = (0,1,1);

draw ((a -- b -- f -- e -- cycle));
draw ((a -- d -- h -- e -- cycle));
draw ((a -- b -- c -- d -- cycle));

dot(A, linewidth(2));
dot(B, linewidth(2));
dot(C, linewidth(2));
dot(D, linewidth(2));
dot(E, linewidth(2));
dot(F, linewidth(2));
dot(G, linewidth(2));
dot(H, linewidth(2));
dot(I, linewidth(2));
dot(J, linewidth(2));
dot(K, linewidth(2));
dot(L, linewidth(2));
dot(M, linewidth(2));
dot(N, linewidth(2));
\end{asy}

(C)
\end{figure}

\end{multicols}

\begin{multicols}{3}
\begin{figure}[H]
\centering

\begin{asy}
size(3.7cm);

currentprojection=orthographic(1.5,1.5,1/2);

triple A = (.4,.4,.35);
triple B = (.4,.7,.35);
triple C = (.7,.7,.35);
triple D = (.7,.4,.35);

triple E = (A+(0,0,.1));
triple F = (B+(0,0,.1));
triple G = (E+(.1,0,0));
triple H = (F+(.1,0,0));

triple I = (C+(0,0,.3));
triple J = (H+(0,0,.2));

triple K = (D+(0,0,.4));
triple L = (G+(0,0,.3));

triple M = (K+(0,.1,0));
triple N = (L+(0,.1,0));

draw(B -- C -- I -- J -- H -- F -- cycle);

draw(C -- D -- K -- M -- I -- cycle);
draw(D -- K -- L -- G -- E -- A -- cycle, "carlinhos");
draw(K -- L -- N -- M -- cycle);
draw(M -- N -- J -- I -- cycle);

draw (A -- B -- C -- D -- cycle);
draw (E -- F -- H -- G -- cycle);
draw (A -- B -- F -- E -- cycle);

draw(surface(B -- C -- I -- J -- H -- F -- cycle), verde*80);

draw(surface(C -- D -- K -- M -- I -- cycle), verde*80);
draw(surface(D -- K -- L -- G -- E -- A -- cycle), verde*80);
draw(surface(K -- L -- N -- M -- cycle), verde*80);
draw(surface(M -- N -- J -- I -- cycle), verde*80);

draw (surface(A -- B -- C -- D -- cycle), verde*80);
draw (surface(E -- F -- H -- G -- cycle), verde*80);
draw (surface(A -- B -- F -- E -- cycle), verde*80);

draw (surface(H -- J -- N --L -- G -- cycle), verde*80);


triple x = (A + (0,-.4,0));
triple w = (D + (0,-.4,0));
triple p = (E + (0,-.4,0));
triple q = (G + (0,-.4,0));
triple y = (K + (0,-.4,0));
triple z = (L + (0,-.4,0));
triple u = (I + (0,-.7,0));
triple v = (J + (0,-.7,0));

draw (x -- w -- y -- z --q -- p -- cycle);
draw (u -- v);


triple a = (0,0,0);
triple b = (1,0,0);
triple c = (1,1,0);
triple d = (0,1,0);

triple e = (0,0,1);
triple f = (1,0,1);
triple g = (1,1,1);
triple h = (0,1,1);



draw ((a -- b -- f -- e -- cycle));
draw ((a -- d -- h -- e -- cycle));
draw ((a -- b -- c -- d -- cycle));

dot(A, linewidth(2));
dot(B, linewidth(2));
dot(C, linewidth(2));
dot(D, linewidth(2));
dot(E, linewidth(2));
dot(F, linewidth(2));
dot(G, linewidth(2));
dot(H, linewidth(2));
dot(I, linewidth(2));
dot(J, linewidth(2));
dot(K, linewidth(2));
dot(L, linewidth(2));
dot(M, linewidth(2));
dot(N, linewidth(2));

//draw (a -- b -- c -- d -- cycle);
//draw (e -- f -- g -- h -- cycle);
//draw (a -- b -- f -- e -- cycle);
//draw (c -- d -- h -- g -- cycle);



draw((0,.9,.5) -- (0,.6,.5), arrow=Arrow3(TeXHead2));

path3 f1 =  ((.8,0,1.05) -- (.2,0,1.05));

string txt = "Plano Frontal";
draw(labelpath(txt, subpath(f1,0,1),angle=180));
\end{asy}

(D)
\end{figure}

\begin{figure}[H]
\centering

\begin{asy}
size(3.7cm);

currentprojection=orthographic(1.5,1.5,1/2);

triple A = (.4,.4,.35);
triple B = (.4,.7,.35);
triple C = (.7,.7,.35);
triple D = (.7,.4,.35);

triple E = (A+(0,0,.1));
triple F = (B+(0,0,.1));
triple G = (E+(.1,0,0));
triple H = (F+(.1,0,0));

triple I = (C+(0,0,.3));
triple J = (H+(0,0,.2));

triple K = (D+(0,0,.4));
triple L = (G+(0,0,.3));

triple M = (K+(0,.1,0));
triple N = (L+(0,.1,0));

draw(B -- C -- I -- J -- H -- F -- cycle);

draw(C -- D -- K -- M -- I -- cycle);
draw(D -- K -- L -- G -- E -- A -- cycle);
draw(K -- L -- N -- M -- cycle);
draw(M -- N -- J -- I -- cycle);

draw (A -- B -- C -- D -- cycle);
draw (E -- F -- H -- G -- cycle);
draw (A -- B -- F -- E -- cycle);

draw(surface(B -- C -- I -- J -- H -- F -- cycle), verde*80);

draw(surface(C -- D -- K -- M -- I -- cycle), verde*80);
draw(surface(D -- K -- L -- G -- E -- A -- cycle), verde*80);
draw(surface(K -- L -- N -- M -- cycle), verde*80);
draw(surface(M -- N -- J -- I -- cycle), verde*80);

draw (surface(A -- B -- C -- D -- cycle), verde*80);
draw (surface(E -- F -- H -- G -- cycle), verde*80);
draw (surface(A -- B -- F -- E -- cycle), verde*80);

draw (surface(H -- J -- N --L -- G -- cycle), verde*80);


triple x = (A + (0,0,-.35));
triple w = (B + (0,0,-.35));
triple p = (C + (0,0,-.35));
triple q = (D + (0,0,-.35));
triple y = (G + (0,0,-.45));
triple z = (H + (0,0,-.45));
triple u = (M + (0,0,-.75));
triple v = (N + (0,0,-.75));

draw (x -- w -- p -- q -- cycle);
draw (y -- z);
draw (u -- v);

triple a = (0,0,0);
triple b = (1,0,0);
triple c = (1,1,0);
triple d = (0,1,0);

triple e = (0,0,1);
triple f = (1,0,1);
triple g = (1,1,1);
triple h = (0,1,1);

draw ((a -- b -- f -- e -- cycle));
draw ((a -- d -- h -- e -- cycle));
draw ((a -- b -- c -- d -- cycle));

dot(A, linewidth(2));
dot(B, linewidth(2));
dot(C, linewidth(2));
dot(D, linewidth(2));
dot(E, linewidth(2));
dot(F, linewidth(2));
dot(G, linewidth(2));
dot(H, linewidth(2));
dot(I, linewidth(2));
dot(J, linewidth(2));
dot(K, linewidth(2));
dot(L, linewidth(2));
dot(M, linewidth(2));
dot(N, linewidth(2));

//draw (a -- b -- c -- d -- cycle);
//draw (e -- f -- g -- h -- cycle);
//draw (a -- b -- f -- e -- cycle);
//draw (c -- d -- h -- g -- cycle);



draw((0,.5,.9) -- (0,.5,.6), arrow=Arrow3(TeXHead2));

path3 f1 =  ((0,0.2,1.05) -- (0,0.8,1.05));

string txt = "Plano Lateral";
draw(labelpath(txt, subpath(f1,0,1),angle=270));
\end{asy}
\\
(E)
\end{figure}

\begin{figure}[H]
\centering

\begin{asy}
size(3.7cm);

currentprojection=orthographic(1.5,1.5,1/2);

triple A = (.4,.4,.35);
triple B = (.4,.7,.35);
triple C = (.7,.7,.35);
triple D = (.7,.4,.35);

triple E = (A+(0,0,.1));
triple F = (B+(0,0,.1));
triple G = (E+(.1,0,0));
triple H = (F+(.1,0,0));

triple I = (C+(0,0,.3));
triple J = (H+(0,0,.2));

triple K = (D+(0,0,.4));
triple L = (G+(0,0,.3));

triple M = (K+(0,.1,0));
triple N = (L+(0,.1,0));

draw(B -- C -- I -- J -- H -- F -- cycle);

draw(C -- D -- K -- M -- I -- cycle);
draw(D -- K -- L -- G -- E -- A -- cycle);
draw(K -- L -- N -- M -- cycle);
draw(M -- N -- J -- I -- cycle);

draw (A -- B -- C -- D -- cycle);
draw (E -- F -- H -- G -- cycle);
draw (A -- B -- F -- E -- cycle);

draw(surface(B -- C -- I -- J -- H -- F -- cycle), verde*80);

draw(surface(C -- D -- K -- M -- I -- cycle), verde*80);
draw(surface(D -- K -- L -- G -- E -- A -- cycle), verde*80);
draw(surface(K -- L -- N -- M -- cycle), verde*80);
draw(surface(M -- N -- J -- I -- cycle), verde*80);

draw (surface(A -- B -- C -- D -- cycle), verde*80);
draw (surface(E -- F -- H -- G -- cycle), verde*80);
draw (surface(A -- B -- F -- E -- cycle), verde*80);

draw (surface(H -- J -- N --L -- G -- cycle), verde*80);


triple x = (C - (.7,0,0));
triple w = (D - (.7,0,0));
triple p = (K - (.7,0,0));
triple q = (M - (.7,0,0));
triple y = (I - (.7,0,0));
triple z = (E - (.4,0,0));
triple u = (F - (.4,0,0));

draw (x -- w -- p --q --y -- cycle);
draw (z -- u, dashed);

triple a = (0,0,0);
triple b = (1,0,0);
triple c = (1,1,0);
triple d = (0,1,0);

triple e = (0,0,1);
triple f = (1,0,1);
triple g = (1,1,1);
triple h = (0,1,1);

draw ((a -- b -- f -- e -- cycle));
draw ((a -- d -- h -- e -- cycle));
draw ((a -- b -- c -- d -- cycle));

dot(A, linewidth(2));
dot(B, linewidth(2));
dot(C, linewidth(2));
dot(D, linewidth(2));
dot(E, linewidth(2));
dot(F, linewidth(2));
dot(G, linewidth(2));
dot(H, linewidth(2));
dot(I, linewidth(2));
dot(J, linewidth(2));
dot(K, linewidth(2));
dot(L, linewidth(2));
dot(M, linewidth(2));
dot(N, linewidth(2));

//draw (a -- b -- c -- d -- cycle);
//draw (e -- f -- g -- h -- cycle);
//draw (a -- b -- f -- e -- cycle);
//draw (c -- d -- h -- g -- cycle);



draw((.9,0,.5) -- (.6,0,.5), arrow=Arrow3(TeXHead2));

path3 f1 =  ((0.9,1.15,0) -- (0.1,1.15,0));

string txt = "Plano Horizontal";
draw(labelpath(txt, subpath(f1,0,1),angle=100));
\end{asy}
\\
(F)
\end{figure}

\end{multicols}
\caption{Representação das vistas ortogonais}
\label{\detokenize{GE301-6:fig-proj-vistas-ortogonais-03}}
\end{figure}
\end{minipage}



Nas imagens (D), (E) e (F) da \hyperref[\detokenize{GE301-6:fig-proj-vistas-ortogonais-03}]{Figura \ref{\detokenize{GE301-6:fig-proj-vistas-ortogonais-03}}}, as setas indicam quais pontos serão projetados nos respectivos planos, da seguinte maneira: supondo-se que o objeto é opaco e que “raios de luz” chegam no sentido da seta, estes raios atingirão pontos do objeto e serão então bloqueados, não atingindo outros pontos. Apenas os pontos que recebem “raios de luz” serão projetados no plano. Estes pontos serão desenhados com uma linha sólida. Partes do objeto que ficam “escondidos” são desenhados com uma linha pontilhada, como acontece na imagem (F).

Na \hyperref[\detokenize{GE301-6:fig-proj-vistas-ortogonais-04}]{Figura \ref{\detokenize{GE301-6:fig-proj-vistas-ortogonais-04}}} as três projeções nos três planos estão desenhadas simultaneamente com o sólido 3D(imagem (A)) e sem ele (imagem (B)). As três projeções podem então ser dispostas lado a lado em um mesmo plano, gerando então a representação clássica das \index{vistas principais}vistas principais (imagem (C)): a \index{vista frontal}vista frontal (projeção ortogonal no plano frontal), a \index{vista lateral}vista lateral (projeção ortogonal no plano lateral) e a \index{vista superior}vista superior (projeção ortogonal no plano horizontal). Apesar do objeto 3D original ter as faces pintadas de vermelho, nas projeções apenas os contornos estão desenhados.


\begin{figure}[H]
\centering
\begin{multicols}{2}


\begin{asy}
size(6cm);

currentprojection=orthographic(1.5,1.5,1/2);

triple A = (.4,.4,.35);
triple B = (.4,.7,.35);
triple C = (.7,.7,.35);
triple D = (.7,.4,.35);

triple E = (A+(0,0,.1));
triple F = (B+(0,0,.1));
triple G = (E+(.1,0,0));
triple H = (F+(.1,0,0));

triple I = (C+(0,0,.3));
triple J = (H+(0,0,.2));

triple K = (D+(0,0,.4));
triple L = (G+(0,0,.3));

triple M = (K+(0,.1,0));
triple N = (L+(0,.1,0));

draw(B -- C -- I -- J -- H -- F -- cycle);

draw(C -- D -- K -- M -- I -- cycle);
draw(D -- K -- L -- G -- E -- A -- cycle);
draw(K -- L -- N -- M -- cycle);
draw(M -- N -- J -- I -- cycle);

draw (A -- B -- C -- D -- cycle);
draw (E -- F -- H -- G -- cycle);
draw (A -- B -- F -- E -- cycle);

draw(surface(B -- C -- I -- J -- H -- F -- cycle), verde*80);

draw(surface(C -- D -- K -- M -- I -- cycle), verde*80);
draw(surface(D -- K -- L -- G -- E -- A -- cycle), verde*80);
draw(surface(K -- L -- N -- M -- cycle), verde*80);
draw(surface(M -- N -- J -- I -- cycle), verde*80);

draw (surface(A -- B -- C -- D -- cycle), verde*80);
draw (surface(E -- F -- H -- G -- cycle), verde*80);
draw (surface(A -- B -- F -- E -- cycle), verde*80);

draw (surface(H -- J -- N --L -- G -- cycle), verde*80);


triple x = (C - (.7,0,0));
triple w = (D - (.7,0,0));
triple p = (K - (.7,0,0));
triple q = (M - (.7,0,0));
triple y = (I - (.7,0,0));
triple z = (E - (.4,0,0));
triple u = (F - (.4,0,0));

draw (x -- w -- p --q --y -- cycle);
draw (z -- u, dashed);

triple x1 = (A + (0,-.4,0));
triple w1 = (D + (0,-.4,0));
triple p1 = (E + (0,-.4,0));
triple q1 = (G + (0,-.4,0));
triple y1 = (K + (0,-.4,0));
triple z1 = (L + (0,-.4,0));
triple u1 = (I + (0,-.7,0));
triple v1 = (J + (0,-.7,0));

draw (x1 -- w1 -- y1 -- z1 --q1 -- p1 -- cycle);
draw (u1 -- v1);

triple x2 = (A + (0,0,-.35));
triple w2 = (B + (0,0,-.35));
triple p2 = (C + (0,0,-.35));
triple q2 = (D + (0,0,-.35));
triple y2 = (G + (0,0,-.45));
triple z2 = (H + (0,0,-.45));
triple u2 = (M + (0,0,-.75));
triple v2 = (N + (0,0,-.75));

draw (x2 -- w2 -- p2 -- q2 -- cycle);
draw (y2 -- z2);
draw (u2 -- v2);

triple a = (0,0,0);
triple b = (1,0,0);
triple c = (1,1,0);
triple d = (0,1,0);

triple e = (0,0,1);
triple f = (1,0,1);
triple g = (1,1,1);
triple h = (0,1,1);

draw ((a -- b -- f -- e -- cycle));
draw ((a -- d -- h -- e -- cycle));
draw ((a -- b -- c -- d -- cycle));

dot(A, linewidth(2));
dot(B, linewidth(2));
dot(C, linewidth(2));
dot(D, linewidth(2));
dot(E, linewidth(2));
dot(F, linewidth(2));
dot(G, linewidth(2));
dot(H, linewidth(2));
dot(I, linewidth(2));
dot(J, linewidth(2));
dot(K, linewidth(2));
dot(L, linewidth(2));
dot(M, linewidth(2));
dot(N, linewidth(2));

//draw (a -- b -- c -- d -- cycle);
//draw (e -- f -- g -- h -- cycle);
//draw (a -- b -- f -- e -- cycle);
//draw (c -- d -- h -- g -- cycle);


draw((0,.9,.2) -- (0,.6,.2), arrow=Arrow3(TeXHead2));
draw((.9,0,.2) -- (.6,0,.2), arrow=Arrow3(TeXHead2));
draw((0,.2,.9) -- (0,.2,.6), arrow=Arrow3(TeXHead2));

path3 f1 =  ((.8,0,1.05) -- (.2,0,1.05));

string txt = "Plano Frontal";
draw(labelpath(txt, subpath(f1,0,1),angle=180));

path3 f2 =  ((0,0.2,1.05) -- (0,0.8,1.05));

string txt = "Plano Lateral";
draw(labelpath(txt, subpath(f2,0,1),angle=270));

path3 f3 =  ((0.9,1.15,0) -- (0.1,1.15,0));

string txt = "Plano Horizontal";
draw(labelpath(txt, subpath(f3,0,1),angle=100));
\end{asy}
\\
(A)

\begin{asy}
size(6cm);

currentprojection=orthographic(1.5,1.5,1/2);

triple A = (.4,.4,.35);
triple B = (.4,.7,.35);
triple C = (.7,.7,.35);
triple D = (.7,.4,.35);

triple E = (A+(0,0,.1));
triple F = (B+(0,0,.1));
triple G = (E+(.1,0,0));
triple H = (F+(.1,0,0));

triple I = (C+(0,0,.3));
triple J = (H+(0,0,.2));

triple K = (D+(0,0,.4));
triple L = (G+(0,0,.3));

triple M = (K+(0,.1,0));
triple N = (L+(0,.1,0));

//draw(B -- C -- I -- J -- H -- F -- cycle);

//draw(C -- D -- K -- M -- I -- cycle);
//draw(D -- K -- L -- G -- E -- A -- cycle);
//draw(K -- L -- N -- M -- cycle);
//draw(M -- N -- J -- I -- cycle);

//draw (A -- B -- C -- D -- cycle);
//draw (E -- F -- H -- G -- cycle);
//draw (A -- B -- F -- E -- cycle);

//draw(surface(B -- C -- I -- J -- H -- F -- cycle), verde*80);

//draw(surface(C -- D -- K -- M -- I -- cycle), verde*80);
//draw(surface(D -- K -- L -- G -- E -- A -- cycle), verde*80);
//draw(surface(K -- L -- N -- M -- cycle), verde*80);
//draw(surface(M -- N -- J -- I -- cycle), verde*80);

//draw (surface(A -- B -- C -- D -- cycle), verde*80);
//draw (surface(E -- F -- H -- G -- cycle), verde*80);
//draw (surface(A -- B -- F -- E -- cycle), verde*80);

triple x = (C - (.7,0,0));
triple w = (D - (.7,0,0));
triple p = (K - (.7,0,0));
triple q = (M - (.7,0,0));
triple y = (I - (.7,0,0));
triple z = (E - (.4,0,0));
triple u = (F - (.4,0,0));

draw (x -- w -- p --q --y -- cycle);
draw (z -- u, dashed);

triple x1 = (A + (0,-.4,0));
triple w1 = (D + (0,-.4,0));
triple p1 = (E + (0,-.4,0));
triple q1 = (G + (0,-.4,0));
triple y1 = (K + (0,-.4,0));
triple z1 = (L + (0,-.4,0));
triple u1 = (I + (0,-.7,0));
triple v1 = (J + (0,-.7,0));

draw (x1 -- w1 -- y1 -- z1 --q1 -- p1 -- cycle);
draw (u1 -- v1);

triple x2 = (A + (0,0,-.35));
triple w2 = (B + (0,0,-.35));
triple p2 = (C + (0,0,-.35));
triple q2 = (D + (0,0,-.35));
triple y2 = (G + (0,0,-.45));
triple z2 = (H + (0,0,-.45));
triple u2 = (M + (0,0,-.75));
triple v2 = (N + (0,0,-.75));

draw (x2 -- w2 -- p2 -- q2 -- cycle);
draw (y2 -- z2);
draw (u2 -- v2);

triple a = (0,0,0);
triple b = (1,0,0);
triple c = (1,1,0);
triple d = (0,1,0);

triple e = (0,0,1);
triple f = (1,0,1);
triple g = (1,1,1);
triple h = (0,1,1);

draw ((a -- b -- f -- e -- cycle));
draw ((a -- d -- h -- e -- cycle));
draw ((a -- b -- c -- d -- cycle));



path3 f1 =  ((.8,0,1.05) -- (.2,0,1.05));

string txt = "Plano Frontal";
draw(labelpath(txt, subpath(f1,0,1),angle=180));

path3 f2 =  ((0,0.2,1.05) -- (0,0.8,1.05));

string txt = "Plano Lateral";
draw(labelpath(txt, subpath(f2,0,1),angle=270));

path3 f3 =  ((0.9,1.15,0) -- (0.1,1.15,0));

string txt = "Plano Horizontal";
draw(labelpath(txt, subpath(f3,0,1),angle=100));
\end{asy}
\\
(B)

\end{multicols}

\begin{asy}
size(8cm);

currentprojection=orthographic(1.5,1.5,1/2);

triple A = (.4,.4,.35);
triple B = (.4,.7,.35);
triple C = (.7,.7,.35);
triple D = (.7,.4,.35);

triple E = (A+(0,0,.1));
triple F = (B+(0,0,.1));
triple G = (E+(.1,0,0));
triple H = (F+(.1,0,0));

triple I = (C+(0,0,.3));
triple J = (H+(0,0,.2));

triple K = (D+(0,0,.4));
triple L = (G+(0,0,.3));

triple M = (K+(0,.1,0));
triple N = (L+(0,.1,0));

//draw(B -- C -- I -- J -- H -- F -- cycle);

//draw(C -- D -- K -- M -- I -- cycle);
//draw(D -- K -- L -- G -- E -- A -- cycle);
//draw(K -- L -- N -- M -- cycle);
//draw(M -- N -- J -- I -- cycle);

//draw (A -- B -- C -- D -- cycle);
//draw (E -- F -- H -- G -- cycle);
//draw (A -- B -- F -- E -- cycle);

//draw(surface(B -- C -- I -- J -- H -- F -- cycle), verde*80);

//draw(surface(C -- D -- K -- M -- I -- cycle), verde*80);
//draw(surface(D -- K -- L -- G -- E -- A -- cycle), verde*80);
//draw(surface(K -- L -- N -- M -- cycle), verde*80);
//draw(surface(M -- N -- J -- I -- cycle), verde*80);

//draw (surface(A -- B -- C -- D -- cycle), verde*80);
//draw (surface(E -- F -- H -- G -- cycle), verde*80);
//draw (surface(A -- B -- F -- E -- cycle), verde*80);

//triple x = (C - (.7,0,0));
//triple w = (D - (.7,0,0));
//triple p = (K - (.7,0,0));
//triple q = (M - (.7,0,0));
//triple y = (I - (.7,0,0));
//triple z = (E - (.4,0,0));
//triple u = (F - (.4,0,0));

//draw (x -- w -- p --q --y -- cycle);
//draw (z -- u, dashed);

//triple x1 = (A + (0,-.4,0));
//triple w1 = (D + (0,-.4,0));
//triple p1 = (E + (0,-.4,0));
//triple q1 = (G + (0,-.4,0));
//triple y1 = (K + (0,-.4,0));
//triple z1 = (L + (0,-.4,0));
//triple u1 = (I + (0,-.7,0));
//triple v1 = (J + (0,-.7,0));

//draw (x1 -- w1 -- y1 -- z1 --q1 -- p1 -- cycle);
//draw (u1 -- v1);

//pair A = (.4,.4);
//pair B = (.4,.5);
//pair C = (.3,.7);
//pair D = (.7,.4);

//draw(A--B--C--D);

//triple E = (A+(0,0,.1));
//triple F = (B+(0,0,.1));
//triple G = (E+(.1,0,0));
//triple H = (F+(.1,0,0));

//triple I = (C+(0,0,.3));
//triple J = (H+(0,0,.2));

//triple K = (D+(0,0,.4));
//triple L = (G+(0,0,.3));

//triple M = (K+(0,.1,0));
//triple N = (L+(0,.1,0));

//triple x2 = (A + (0,0,-.35));
//triple w2 = (B + (0,0,-.35));
//triple p2 = (C + (0,0,-.35));
//triple q2 = (D + (0,0,-.35));
//triple y2 = (G + (0,0,-.45));
//triple z2 = (H + (0,0,-.45));
//triple u2 = (M + (0,0,-.75));
//triple v2 = (N + (0,0,-.75));

//draw (x2 -- w2 -- p2 -- q2 -- cycle);
//draw (y2 -- z2);
//draw (u2 -- v2);

pair a = (0,0);
pair b = (1,0);
pair c = (1,1);
pair d = (0,1);

pair A = (.7,.25);
pair B = (.7,.25);
pair C = (.7,.35);
pair D = (.6,.35);

pair E = (.6,.75);
pair F = (.4,.75);
pair G = (.4,.25);

pair H = (E - (0,.1));
pair I = (F - (0,.1));

draw (shift(-.05)*(H -- I));
draw (shift(-.05)*(A -- B -- C -- D -- E -- F -- G -- cycle));

//pair e = (0,0,1);
//pair f = (1,0,1);
//pair g = (1,1,1);
//pair h = (0,1,1);

//draw ((a -- b -- f -- e -- cycle));
//draw ((a -- d -- h -- e -- cycle));
draw ((a -- b -- c -- d -- cycle));
draw (shift(1,0)*(a -- b -- c -- d -- cycle));

pair A1 = (.4,.25);
pair B1 = (.7,.25);
pair C1 = (.7,.6);
pair D1 = (.5,.75);
pair E1 = (.4,.75);
pair F1 = (A1+(0,.1));
pair G1 = (B1+(0,.1));

draw (shift(.95,0)*(A1 -- B1 -- C1 -- D1 -- E1) -- cycle);
draw (shift(.95,0)*(F1 -- G1), dashed);


draw (shift(0,-1)*(a -- b -- c -- d -- cycle));

pair A2 = (.3,.3);
pair B2 = (.7,.3);
pair C2 = (.7,.7);
pair D2 = (.3,.7);
pair E2 = (.6,.7);
pair F2 = (.6,.3);
pair G2 = (.3,.6);
pair H2 = (.6,.6);

draw (shift(0,-1)*(A2 -- B2 -- C2 -- D2) -- cycle);
draw (shift(0,-1)*(E2 -- F2));
draw (shift(0,-1)*(G2 -- H2));


path f1 =  ((0,1) -- (1,1));

label(scale(.8)*"Vista Frontal", (.5,1), align=S);
label(scale(.8)*"Vista Lateral", shift(1,0)*(.5,1), align=S);
label(scale(.8)*"Vista Superior", shift(0,-1)*(.5,1), align=S);

label(rotate(90)*scale(.8)*"Plano Frontal",(0,.5), align=W);
label(rotate(90)*scale(.8)*"Plano Horizontal",shift(0,-1)*(0,.5), align=W);
label(rotate(90)*scale(.8)*"Plano Lateral",(2.1,.5));
\end{asy}
\\
(C)

\caption{Projeções ortogonais nos planos}
\label{\detokenize{GE301-6:fig-proj-vistas-ortogonais-04}}
\end{figure}
% \begin{figure}[H]
% \centering

% \noindent\includegraphics[width=300bp]{{vistas-ortogonais-04}.jpg}
% \caption{Projeções ortogonais nos planos}
% \label{\detokenize{GE301-6:fig-proj-vistas-ortogonais-04}}\end{figure}
\vspace{1em}

\paragraph{Etapa 1}

Desenhe as vistas principais ortogonais de cada um dos objetos geométricos apresentados a seguir. Dê sua resposta como em (C) em \hyperref[\detokenize{GE301-6:fig-proj-vistas-ortogonais-04}]{Figura \ref{\detokenize{GE301-6:fig-proj-vistas-ortogonais-04}}}.

\begin{enumerate}
\begin{multicols}{2}
\item
\adjustbox{valign=t}{
\begin{minipage}{\linewidth}
\begin{asy}
size(5cm);

currentprojection=perspective(2.53,1.60,1.27);

triple a = (0,0,0);
triple b = (1,0,0);
triple c = (1,1,0);
triple d = (0,1,0);

triple e = (0,0,1);
triple f = (1,0,1);
triple g = (1,1,1);
triple h = (0,1,1);

draw ((a -- b -- f -- e -- cycle));
draw ((a -- d -- h -- e -- cycle));
draw ((a -- b -- c -- d -- cycle));


triple A = (1/4,1/4,1/4);
triple B = (3/4,1/4,1/4);
triple C = (3/4,3/4,1/4);
triple D = (1/4,3/4,1/4);

triple E = (1/4,1/4,3/4);
triple F = (3/4,1/4,3/4);
triple G = (3/4,3/4,3/4);
triple H = (1/4,3/4,3/4);

draw (A -- B -- C -- D -- cycle);
draw (E -- F -- G -- H -- cycle);

draw (A -- E);
draw (B -- F);
draw (C -- G);
draw (D -- H);

draw (surface(A -- B -- C -- D -- cycle), verde*80);
draw (surface(E -- F -- G -- H -- cycle), verde*80);

draw (surface(A -- E -- F -- B -- cycle), verde*80);
draw (surface(B -- F -- G -- C -- cycle), verde*80);
draw (surface(C -- G -- H -- D -- cycle), verde*80);
draw (surface(D -- H -- E -- A -- cycle), verde*80);
\end{asy}
\end{minipage}}

\item
\adjustbox{valign=t}{
\begin{minipage}{\linewidth}
\begin{asy}
size(5cm);

currentprojection=perspective(2.53,1.60,1.27);



triple a = (0,0,0);
triple b = (1,0,0);
triple c = (1,1,0);
triple d = (0,1,0);

triple e = (0,0,1);
triple f = (1,0,1);
triple g = (1,1,1);
triple h = (0,1,1);

draw ((a -- b -- f -- e -- cycle));
draw ((a -- d -- h -- e -- cycle));
draw ((a -- b -- c -- d -- cycle));


triple A = rotate(45,(0,.5,.5), (1,.5,.5))*(1/4,1/4,1/4);
triple B = rotate(45,(0,.5,.5), (1,.5,.5))*(3/4,1/4,1/4);
triple C = rotate(45,(0,.5,.5), (1,.5,.5))*(3/4,3/4,1/4);
triple D = rotate(45,(0,.5,.5), (1,.5,.5))*(1/4,3/4,1/4);

triple E = rotate(45,(0,.5,.5), (1,.5,.5))*(1/4,1/4,3/4);
triple F = rotate(45,(0,.5,.5), (1,.5,.5))*(3/4,1/4,3/4);
triple G = rotate(45,(0,.5,.5), (1,.5,.5))*(3/4,3/4,3/4);
triple H = rotate(45,(0,.5,.5), (1,.5,.5))*(1/4,3/4,3/4);

draw (A -- B -- C -- D -- cycle);
draw (E -- F -- G -- H -- cycle);

draw (A -- E);
draw (B -- F);
draw (C -- G);
draw (D -- H);

draw (surface(A -- B -- C -- D -- cycle), verde*80);
draw (surface(E -- F -- G -- H -- cycle), verde*80);

draw (surface(A -- E -- F -- B -- cycle), verde*80);
draw (surface(B -- F -- G -- C -- cycle), verde*80);
draw (surface(C -- G -- H -- D -- cycle), verde*80);
draw (surface(D -- H -- E -- A -- cycle), verde*80);
\end{asy}
\end{minipage}}
\end{multicols}

\begin{multicols}{2}

\item
\adjustbox{valign=t}{
\begin{minipage}{\linewidth}
\begin{asy}
size(5cm);

currentprojection=perspective(2.53,1.60,1.27);



triple a = (0,0,0);
triple b = (1,0,0);
triple c = (1,1,0);
triple d = (0,1,0);

triple e = (0,0,1);
triple f = (1,0,1);
triple g = (1,1,1);
triple h = (0,1,1);

draw ((a -- b -- f -- e -- cycle));
draw ((a -- d -- h -- e -- cycle));
draw ((a -- b -- c -- d -- cycle));


draw(surface(sphere(c=(.5,.5,.5), r=1/3)), verde*80);
\end{asy}
\end{minipage}}

\item
\adjustbox{valign=t}{
\begin{minipage}{\linewidth}
\begin{asy}
size(5cm);

currentprojection=perspective(2.53,1.60,1.27);



triple a = (0,0,0);
triple b = (1,0,0);
triple c = (1,1,0);
triple d = (0,1,0);

triple e = (0,0,1);
triple f = (1,0,1);
triple g = (1,1,1);
triple h = (0,1,1);

draw ((a -- b -- f -- e -- cycle));
draw ((a -- d -- h -- e -- cycle));
draw ((a -- b -- c -- d -- cycle));

draw(cylinder((1/2,1/2,1/5), 1/3,3/5,(0,0,1)));
draw(surface(cylinder((1/2,1/2,1/5), 1/3,3/5,(0,0,1))), verde*80);
draw(surface(circle((1/2,1/2,4/5), r=1/3,Z)), verde*80);
//draw((circle((1/2,1/2,4/5), r=1/3,Z)));
draw(cylinder((1/2,1/2,1/5), 1/3,3/5,(0,0,1)), black+linewidth(.5));
\end{asy}
\end{minipage}}

\end{multicols}
\begin{multicols}{2}
\item
\adjustbox{valign=t}{
\begin{minipage}{\linewidth}
\begin{asy}
size(5cm);

currentprojection=perspective(2.53,1.60,1.27);



triple a = (0,0,0);
triple b = (1,0,0);
triple c = (1,1,0);
triple d = (0,1,0);

triple e = (0,0,1);
triple f = (1,0,1);
triple g = (1,1,1);
triple h = (0,1,1);

draw ((a -- b -- f -- e -- cycle));
draw ((a -- d -- h -- e -- cycle));
draw ((a -- b -- c -- d -- cycle));

draw(surface(cone((1/2,1/2,1/5), 1/3,3/5,(0,0,1))), verde*80);
\end{asy}
\end{minipage}}
\end{multicols}
\end{enumerate}

\newpage
\paragraph{Etapa 2}

Em cada uma das figuras a seguir, desenhe um objeto cujas projeções ortogonais sobre os planos frontal, horizontal e lateral são aquelas apresentadas.

\begin{enumerate}
\begin{multicols}{2}

\item
\adjustbox{valign=t}{
\begin{minipage}{\linewidth}
\begin{asy}
size(4.75cm);

currentprojection=perspective(2.53,1.60,1.27);

triple a = (0,0,0);
triple b = (1,0,0);
triple c = (1,1,0);
triple d = (0,1,0);

triple e = (0,0,1);
triple f = (1,0,1);
triple g = (1,1,1);
triple h = (0,1,1);

draw ((a -- b -- f -- e -- cycle));
draw ((a -- d -- h -- e -- cycle));
draw ((a -- b -- c -- d -- cycle));

draw((1/2,1/4,0)--(1/2,3/4,0));
draw((0,1/4,1/2)--(0,3/4,1/2));

//draw (a -- b -- c -- d -- cycle);
//draw (e -- f -- g -- h -- cycle);
//draw (a -- b -- f -- e -- cycle);
//draw (c -- d -- h -- g -- cycle);
\end{asy}
\end{minipage}}


\item
\adjustbox{valign=t}{
\begin{minipage}{\linewidth}
\begin{asy}
size(4.75cm);

currentprojection=perspective(2.53,1.60,1.27);

triple a = (0,0,0);
triple b = (1,0,0);
triple c = (1,1,0);
triple d = (0,1,0);

triple e = (0,0,1);
triple f = (1,0,1);
triple g = (1,1,1);
triple h = (0,1,1);

draw ((a -- b -- f -- e -- cycle));
draw ((a -- d -- h -- e -- cycle));
draw ((a -- b -- c -- d -- cycle));

draw((1/2,1/4,0)--(1/2,3/4,0));
draw((1/2,0,1/4)--(1/2,0,3/4));
draw((0,1/4,1/4)--(0,3/4,3/4));

//draw (a -- b -- c -- d -- cycle);
//draw (e -- f -- g -- h -- cycle);
//draw (a -- b -- f -- e -- cycle);
//draw (c -- d -- h -- g -- cycle);
\end{asy}
\end{minipage}}
\end{multicols}

\begin{multicols}{2}

\item
\adjustbox{valign=t}{
\begin{minipage}{\linewidth}
\begin{asy}
size(4.75cm);

currentprojection=perspective(2.53,1.60,1.27);

triple a = (0,0,0);
triple b = (1,0,0);
triple c = (1,1,0);
triple d = (0,1,0);

triple e = (0,0,1);
triple f = (1,0,1);
triple g = (1,1,1);
triple h = (0,1,1);

draw ((a -- b -- f -- e -- cycle));
draw ((a -- d -- h -- e -- cycle));
draw ((a -- b -- c -- d -- cycle));

draw((3/4,0,1/4)--(1/4,0,3/4));
draw((1/2+sqrt(2)/8,1/4,0)--(1/2-sqrt(2)/8,3/4,0));
draw((0,1/2-sqrt(2)/8,1/4)--(0,1/2+sqrt(2)/8,3/4));


//draw (a -- b -- c -- d -- cycle);
//draw (e -- f -- g -- h -- cycle);
//draw (a -- b -- f -- e -- cycle);
//draw (c -- d -- h -- g -- cycle);

\end{asy}
\end{minipage}}


\item
\adjustbox{valign=t}{
\begin{minipage}{\linewidth}
\begin{asy}
size(4.75cm);

currentprojection=perspective(2.53,1.60,1.27);

triple a = (0,0,0);
triple b = (1,0,0);
triple c = (1,1,0);
triple d = (0,1,0);

triple e = (0,0,1);
triple f = (1,0,1);
triple g = (1,1,1);
triple h = (0,1,1);

draw ((a -- b -- f -- e -- cycle));
draw ((a -- d -- h -- e -- cycle));
draw ((a -- b -- c -- d -- cycle));

draw((1/4,0,1/2)--(3/4,0,1/2));
draw((0,1/4,1/2)--(0,3/4,1/2));

draw((1/4,1/4,0)--(3/4,1/4,0)--(3/4,3/4,0) --(1/4,3/4,0) -- cycle);

\end{asy}
\end{minipage}}
\end{multicols}
\begin{multicols}{2}

\item
\adjustbox{valign=t}{
\begin{minipage}{\linewidth}
\begin{asy}
size(4.75cm);
currentprojection=perspective(2.53,1.60,1.27);

triple a = (0,0,0);
triple b = (1,0,0);
triple c = (1,1,0);
triple d = (0,1,0);

triple e = (0,0,1);
triple f = (1,0,1);
triple g = (1,1,1);
triple h = (0,1,1);

draw ((a -- b -- f -- e -- cycle));
draw ((a -- d -- h -- e -- cycle));
draw ((a -- b -- c -- d -- cycle));


draw((0,1/4,1/2-sqrt(2)/8)--(0,3/4,1/2+sqrt(2)/8));

draw((1/4,0,1/2-sqrt(2)/8)--(3/4,0,1/2-sqrt(2)/8)--(3/4,0,1/2+sqrt(2)/8)--(1/4,0,1/2+sqrt(2)/8)--cycle);

draw((1/4,1/2-sqrt(2)/8,0)--(3/4,1/2-sqrt(2)/8,0)--(3/4,1/2+sqrt(2)/8,0)--(1/4,1/2+sqrt(2)/8,0)--cycle);


\end{asy}
\end{minipage}}


\item
\adjustbox{valign=t}{
\begin{minipage}{\linewidth}
\begin{asy}
size(4.75cm);

currentprojection=perspective(2.53,1.60,1.27);


triple a = (0,0,0);
triple b = (1,0,0);
triple c = (1,1,0);
triple d = (0,1,0);

triple e = (0,0,1);
triple f = (1,0,1);
triple g = (1,1,1);
triple h = (0,1,1);

draw ((a -- b -- f -- e -- cycle));
draw ((a -- d -- h -- e -- cycle));
draw ((a -- b -- c -- d -- cycle));

real f(real x){return -5*x^2+5*(x)-.5;}
path s = graph(f,0.15,.85);

path3 d = shift(0,.8,.2)*rotate(135,(0,0,0),(1,0,0))*path3(s);
//draw(d);

draw(planeproject(XY*unitsquare3)*d);
draw(planeproject(YZ*unitsquare3)*d);
draw(planeproject(ZX*unitsquare3)*d);
\end{asy}
\end{minipage}}
\end{multicols}
\begin{multicols}{2}

\item
\adjustbox{valign=t}{
\begin{minipage}{\linewidth}
\begin{asy}
settings.render=8;
size(4.75cm);

currentprojection=perspective(2.53,1.60,1.27);

real x(real t) {return (1+cos(2pi*t)+2)/6;}
real y(real t) {return (1+sin(2pi*t)+2)/6;}
real z(real t) {return (t+.5)/6;}

path3 p=graph(x,y,z,0,5,operator ..);

//draw(p,Arrow3);
draw(planeproject(XY*unitsquare3)*p);
draw(planeproject(YZ*unitsquare3)*p);
draw(planeproject(ZX*unitsquare3)*p);

triple a = (0,0,0);
triple b = (1,0,0);
triple c = (1,1,0);
triple d = (0,1,0);

triple e = (0,0,1);
triple f = (1,0,1);
triple g = (1,1,1);
triple h = (0,1,1);

draw ((a -- b -- f -- e -- cycle));
draw ((a -- d -- h -- e -- cycle));
draw ((a -- b -- c -- d -- cycle));
\end{asy}
\end{minipage}}
\end{multicols}
\end{enumerate}

\vspace{3em}
\paragraph{Etapa 3}

Desenhe \textbf{dois objetos diferentes} cujas projeções ortogonais sobre os planos frontal, horizontal e lateral são aquelas na figura a seguir. Nota: as projeções são congruentes e são formadas pelos quatro lados de um mesmo quadrado.

\begin{figure}[H]
\centering

\begin{asy}
size(6cm);
currentprojection=perspective(2.53,1.60,1.27);

triple a = (0,0,0);
triple b = (1,0,0);
triple c = (1,1,0);
triple d = (0,1,0);

triple e = (0,0,1);
triple f = (1,0,1);
triple g = (1,1,1);
triple h = (0,1,1);

draw ((a -- b -- f -- e -- cycle));
draw ((a -- d -- h -- e -- cycle));
draw ((a -- b -- c -- d -- cycle));

draw((1/4,1/4,0)--(3/4,1/4,0)--(3/4,3/4,0) --(1/4,3/4,0) -- cycle);
draw((0,1/4,1/4)--(0,3/4,1/4)--(0,3/4,3/4) --(0,1/4,3/4) -- cycle);
draw((1/4,0,1/4)--(3/4,0,1/4)--(3/4,0,3/4) --(1/4,0,3/4) -- cycle);


\end{asy}
\end{figure}

\paragraph{Etapa 4}


\textbf{(Adaptado de um problema proposto por Martin Gardner)} A figura a seguir exibe as vistas frontal e superior de uma estrutura 3D de madeira. Como seria este objeto 3D e sua vista lateral?


\begin{figure}[H]
\centering

\begin{asy}
size(9cm);

currentprojection=orthographic(1.5,1.5,1/2);

pair a = (0,0);
pair b = (1,0);
pair c = (1,1);
pair d = (0,1);

draw ((a -- b -- c -- d -- cycle));

pair A = (.2,.2);
pair B = (.8,.2);
pair C = (.8,.8);
pair D = (.2,.8);

pair A1 = (.35,.35);
pair B1 = (.65,.35);
pair C1 = (.65,.65);
pair D1 = (.35,.65);

filldraw (A -- B -- C -- D -- cycle, fillpen=verde);
filldraw (A1 -- B1 -- C1 -- D1 -- cycle, fillpen=verde+white);

draw (shift(1,0)*(a -- b -- c -- d -- cycle));
draw ((2,0) -- (2,1));


draw (shift(0,-1)*(a -- b -- c -- d -- cycle));
filldraw (shift(0,-1)*(A -- B -- C -- D -- cycle), fillpen=verde);
filldraw (shift(0,-1)*(A1 -- B1 -- C1 -- D1 -- cycle),fillpen=verde+white);

label(scale(.8)*"Vista Frontal", (.5,1), align=S);
label(scale(.8)*"Vista Lateral", shift(1,0)*(.5,1), align=S);
label(scale(.8)*"Vista Superior", shift(0,-1)*(.5,1), align=S);
label(scale(.8)*"Objeto 3D", shift(1,-1)*(.5,1), align=S);


label(rotate(90)*scale(.8)*"Plano Frontal",(0,.5), align=W);
label(rotate(90)*scale(.8)*"Plano Horizontal",shift(0,-1)*(0,.5), align=W);
label(rotate(90)*scale(.8)*"Plano Lateral",(2.1,.5));

label(scale(3)*"?", (1.5,.5));
label(scale(3)*"?", (1.5,-.5));
\end{asy}
\end{figure}

Importante:
\begin{itemize}
\item {} 
A estrutura não tem segmentos ou linhas pintadas sobre ela.

\item {} 
Todos os contornos escondidos da estrutura devem ser desenhas com linhas tracejadas. Assim, em particular, as vistas frontal e superior não possuem contornos escondidos.

\item {} 
Uma vez que a estrutura é feita de madeira, isto significa que nenhuma de suas partes pode ter espessura zero.ave zero thickness.

\end{itemize}
\end{task}

\begin{observation}

É preciso ter atenção para o uso da palavra \index{vista}vista. Autores diferentes dão significados diferentes à palavra, significados estes que podem, inclusive, ser diferentes de sua interpretação comum (“aquilo que se apresenta ao olhar, que se vê”). Por exemplo, é comum encontrar em livros de arquitetura e engenharia exercícios que pedem para determinar a \emph{vista} de um objeto a partir de uma direção dada, como na \sphinxcode{fig proj-vistas-observacao}. A resposta esperada por estes livros é a imagem (B), ou seja, uma projeção ortogonal do cubo vazado. Contudo, (B) \emph{não é o que se é visto} a partir da direção indicada. O que se vê é melhor descrito por uma projeção em perspectiva, a saber, a imagem (C). Pegue o cubo vazado que você usou na \DUrole{xref,std,std-ref}{ativ-proj-luz-e-sombras} e veja por você mesmo.


\begin{figure}[H]
\centering
\begin{multicols}{3}
\begin{figure}[H]
\centering
\begin{asy}
size(5.2cm);
currentlight.background=box2;
currentprojection=perspective(.5,1.5,1.2);

triple a = (0,0,0);
triple b = (1,0,0);
triple c = (1,1,0);
triple d = (0,1,0);

triple e = (0,0,1);
triple f = (1,0,1);
triple g = (1,1,1);
triple h = (0,1,1);

draw (a -- b -- c -- d -- cycle);
draw (e -- f -- g -- h -- cycle);
draw (a -- b -- f -- e -- cycle);
draw (c -- d -- h -- g -- cycle);

triple a1 = (0.1,0.1,0.1);
triple b1 = (.9,0.1,0.1);
triple c1 = (.9,.9,0.1);
triple d1 = (0.1,.9,0.1);

triple e1 = (0.1,0.1,.9);
triple f1 = (.9,0.1,.9);
triple g1 = (.9,.9,.9);
triple h1 = (0.1,.9,.9);

//draw (a1 -- b1 -- c1 -- d1 -- cycle);
//draw (e1 -- f1 -- g1 -- h1 -- cycle);
//draw (a1 -- b1 -- f1 -- e1 -- cycle);
//draw (c1 -- d1 -- h1 -- g1 -- cycle);

triple a2 = (0.1,0.1,0);
triple b2 = (.9,0.1,0);
triple c2 = (.9,.9,0);
triple d2 = (0.1,.9,0);

draw (a2 -- b2 -- c2 -- d2 -- cycle);

triple e2 = (0.1,0.1,1);
triple f2 = (.9,0.1,1);
triple g2 = (.9,.9,1);
triple h2 = (0.1,.9,1);

draw (e2 -- f2 -- g2 -- h2 -- cycle);

triple i2 = (0.1,0,.9);
triple j2 = (.9,0,.9);
triple k2 = (.9,0,.1);
triple l2 = (0.1,0,.1);

draw (i2 -- j2 --k2 -- l2 -- cycle);

triple m2 = (0.1,1,.9);
triple n2 = (.9,1,.9);
triple o2 = (.9,1,.1);
triple p2 = (0.1,1,.1);

draw (m2 -- n2 -- o2 -- p2 -- cycle);

triple q2 = (0,0.1,0.1);
triple r2 = (0,0.9,0.1);
triple s2 = (0,.9,.9);
triple t2 = (0,.1,.9);

draw (q2 -- r2 -- s2 --t2 -- cycle);

triple u2 = (1,0.1,0.1);
triple v2 = (1,0.9,0.1);
triple w2 = (1,.9,.9);
triple x2 = (1,.1,.9);

draw (u2 -- v2 -- w2 --x2 -- cycle);

draw (a -- a2 -- b2 -- b -- c -- c2 -- d2 -- d);
draw (e -- e2 -- f2 -- f -- g -- g2 -- h2 -- h);
draw (b -- u2 -- v2 -- c -- g -- w2 -- x2 -- f);
draw (a -- q2 -- r2 -- d -- h -- s2 -- t2 -- e);
draw (a -- l2 -- k2 -- b -- f -- j2 -- i2 -- e);
draw (d -- p2 -- o2 -- c -- g -- n2 -- m2 -- h);

draw (a2 -- l2 -- k2 -- b2);
draw (a2 -- l2 -- p2 -- d2);
draw (c2 -- o2 -- p2 -- d2);
draw (l2 -- q2 -- r2 -- p2);
draw (b2 -- k2 -- o2 -- c2);

draw (e2 -- i2 -- m2 -- h2 -- cycle);
draw (e2 -- i2 -- j2 -- f2 -- cycle);
draw (i2 -- t2 -- s2 -- m2 -- cycle);
draw (f2 -- g2 -- n2 -- j2 -- cycle);
draw (j2 -- x2 -- u2 -- k2 -- cycle);
draw (v2 -- o2 -- n2 -- w2 -- cycle);

draw (surface(a2 -- l2 -- k2 -- b2 -- cycle), verde*80);
draw (surface(a2 -- l2 -- p2 -- d2 -- cycle), verde*80);
draw (surface(c2 -- o2 -- p2 -- d2 -- cycle), verde*80);
draw (surface(l2 -- q2 -- r2 -- p2 -- cycle), verde*80);
draw (surface(b2 -- k2 -- o2 -- c2 -- cycle), verde*80);



draw (surface(a -- l2 -- k2 -- b -- cycle), verde*80);
draw (surface(b -- u2 -- v2 -- c -- cycle), verde*80);
draw (surface(c -- o2 -- p2 -- d -- cycle), verde*80);
draw (surface(a -- q2 -- r2 -- d -- cycle), verde*80);

draw (surface(b -- u2 -- v2 -- c -- cycle), verde*80);
draw (surface(b -- f -- x2 -- u2 -- cycle), verde*80);
draw (surface(f -- g -- w2 -- x2 -- cycle), verde*80);
draw (surface(g -- c -- v2 -- w2 -- cycle), verde*80);

draw (surface(c -- g -- n2 -- o2 -- cycle), verde*80);
draw (surface(g -- h -- m2 -- n2 -- cycle), verde*80);
draw (surface(h -- d -- p2 -- m2 -- cycle), verde*80);

draw (surface(e -- f -- f2 -- e2 -- cycle), verde*80);
draw (surface(f -- f2 -- g2 -- g -- cycle), verde*80);
draw (surface(g -- g2 -- h2 -- h -- cycle), verde*80);
draw (surface(h -- h2 -- e2 -- e -- cycle), verde*80);

draw (surface(a -- a2 -- b2 -- b -- cycle), verde*80);
draw (surface(b -- b2 -- c2 -- c -- cycle), verde*80);
draw (surface(c -- c2 -- d2 -- d -- cycle), verde*80);
draw (surface(d -- d2 -- a2 -- a -- cycle), verde*80);

draw (surface(a -- e -- i2 -- l2 -- cycle), verde*80);
draw (surface(e -- f -- j2 -- i2 -- cycle), verde*80);
draw (surface(b -- f -- j2 -- k2 -- cycle), verde*80);

draw (surface(e -- h -- s2 -- t2 -- cycle), verde*80);
draw (surface(h -- d -- r2 -- s2 -- cycle), verde*80);
draw (surface(e -- a -- q2 -- t2 -- cycle), verde*80);

draw (u2 -- k2 -- j2 -- x2);

draw (surface(u2 -- k2 -- j2 -- x2 -- cycle), verde*80);
draw (surface(k2 -- u2 -- v2 -- o2 -- cycle), verde*80);
draw (surface(l2 -- q2 -- t2 -- i2 -- cycle), verde*80);
draw (surface(e2 -- i2 -- m2 -- h2 -- cycle), verde*80);
draw (surface(e2 -- i2 -- j2 -- f2 -- cycle), verde*80);
draw (surface(i2 -- t2 -- s2 -- m2 -- cycle), verde*80);
draw (surface(s2 -- m2 -- p2 -- r2 -- cycle), verde*80);
draw (surface(h2 -- g2 -- n2 -- m2 -- cycle), verde*80);
draw (surface(f2 -- g2 -- n2 -- j2 -- cycle), verde*80);
draw (surface(j2 -- n2 -- w2 -- x2 -- cycle), verde*80);
draw (surface(j2 -- x2 -- u2 -- k2 -- cycle), verde*80);
draw (surface(v2 -- o2 -- n2 -- w2 -- cycle), verde*80);

draw((.5,.5,1.5) -- (.5,.5,1.2), arrow=Arrow3,blue);
\end{asy}
\\
(A)

\end{figure}
\begin{figure}[H]
\centering
Projeção
\\
Orogonal
\vspace{.4cm}
\begin{asy}
size(4cm);
currentlight.background=box2;
currentprojection=orthographic(0,0,1);

triple a = (0,0,0);
triple b = (1,0,0);
triple c = (1,1,0);
triple d = (0,1,0);

triple e = (0,0,1);
triple f = (1,0,1);
triple g = (1,1,1);
triple h = (0,1,1);

draw (a -- b -- c -- d -- cycle);
draw (e -- f -- g -- h -- cycle);
draw (a -- b -- f -- e -- cycle);
draw (c -- d -- h -- g -- cycle);

triple a1 = (0.1,0.1,0.1);
triple b1 = (.9,0.1,0.1);
triple c1 = (.9,.9,0.1);
triple d1 = (0.1,.9,0.1);

triple e1 = (0.1,0.1,.9);
triple f1 = (.9,0.1,.9);
triple g1 = (.9,.9,.9);
triple h1 = (0.1,.9,.9);

//draw (a1 -- b1 -- c1 -- d1 -- cycle);
//draw (e1 -- f1 -- g1 -- h1 -- cycle);
//draw (a1 -- b1 -- f1 -- e1 -- cycle);
//draw (c1 -- d1 -- h1 -- g1 -- cycle);

triple a2 = (0.1,0.1,0);
triple b2 = (.9,0.1,0);
triple c2 = (.9,.9,0);
triple d2 = (0.1,.9,0);

draw (a2 -- b2 -- c2 -- d2 -- cycle);

triple e2 = (0.1,0.1,1);
triple f2 = (.9,0.1,1);
triple g2 = (.9,.9,1);
triple h2 = (0.1,.9,1);

draw (e2 -- f2 -- g2 -- h2 -- cycle);

triple i2 = (0.1,0,.9);
triple j2 = (.9,0,.9);
triple k2 = (.9,0,.1);
triple l2 = (0.1,0,.1);

draw (i2 -- j2 --k2 -- l2 -- cycle);

triple m2 = (0.1,1,.9);
triple n2 = (.9,1,.9);
triple o2 = (.9,1,.1);
triple p2 = (0.1,1,.1);

draw (m2 -- n2 -- o2 -- p2 -- cycle);

triple q2 = (0,0.1,0.1);
triple r2 = (0,0.9,0.1);
triple s2 = (0,.9,.9);
triple t2 = (0,.1,.9);

draw (q2 -- r2 -- s2 --t2 -- cycle);

triple u2 = (1,0.1,0.1);
triple v2 = (1,0.9,0.1);
triple w2 = (1,.9,.9);
triple x2 = (1,.1,.9);

draw (u2 -- v2 -- w2 --x2 -- cycle);

draw (a -- a2 -- b2 -- b -- c -- c2 -- d2 -- d);
draw (e -- e2 -- f2 -- f -- g -- g2 -- h2 -- h);
draw (b -- u2 -- v2 -- c -- g -- w2 -- x2 -- f);
draw (a -- q2 -- r2 -- d -- h -- s2 -- t2 -- e);
draw (a -- l2 -- k2 -- b -- f -- j2 -- i2 -- e);
draw (d -- p2 -- o2 -- c -- g -- n2 -- m2 -- h);

draw (a2 -- l2 -- k2 -- b2);
draw (a2 -- l2 -- p2 -- d2);
draw (c2 -- o2 -- p2 -- d2);
draw (l2 -- q2 -- r2 -- p2);
draw (b2 -- k2 -- o2 -- c2);

draw (e2 -- i2 -- m2 -- h2 -- cycle);
draw (e2 -- i2 -- j2 -- f2 -- cycle);
draw (i2 -- t2 -- s2 -- m2 -- cycle);
draw (f2 -- g2 -- n2 -- j2 -- cycle);
draw (j2 -- x2 -- u2 -- k2 -- cycle);
draw (v2 -- o2 -- n2 -- w2 -- cycle);

draw (surface(a2 -- l2 -- k2 -- b2 -- cycle), verde*80);
draw (surface(a2 -- l2 -- p2 -- d2 -- cycle), verde*80);
draw (surface(c2 -- o2 -- p2 -- d2 -- cycle), verde*80);
draw (surface(l2 -- q2 -- r2 -- p2 -- cycle), verde*80);
draw (surface(b2 -- k2 -- o2 -- c2 -- cycle), verde*80);



draw (surface(a -- l2 -- k2 -- b -- cycle), verde*80);
draw (surface(b -- u2 -- v2 -- c -- cycle), verde*80);
draw (surface(c -- o2 -- p2 -- d -- cycle), verde*80);
draw (surface(a -- q2 -- r2 -- d -- cycle), verde*80);

draw (surface(b -- u2 -- v2 -- c -- cycle), verde*80);
draw (surface(b -- f -- x2 -- u2 -- cycle), verde*80);
draw (surface(f -- g -- w2 -- x2 -- cycle), verde*80);
draw (surface(g -- c -- v2 -- w2 -- cycle), verde*80);

draw (surface(c -- g -- n2 -- o2 -- cycle), verde*80);
draw (surface(g -- h -- m2 -- n2 -- cycle), verde*80);
draw (surface(h -- d -- p2 -- m2 -- cycle), verde*80);

draw (surface(e -- f -- f2 -- e2 -- cycle), verde*80);
draw (surface(f -- f2 -- g2 -- g -- cycle), verde*80);
draw (surface(g -- g2 -- h2 -- h -- cycle), verde*80);
draw (surface(h -- h2 -- e2 -- e -- cycle), verde*80);

draw (surface(a -- a2 -- b2 -- b -- cycle), verde*80);
draw (surface(b -- b2 -- c2 -- c -- cycle), verde*80);
draw (surface(c -- c2 -- d2 -- d -- cycle), verde*80);
draw (surface(d -- d2 -- a2 -- a -- cycle), verde*80);

draw (surface(a -- e -- i2 -- l2 -- cycle), verde*80);
draw (surface(e -- f -- j2 -- i2 -- cycle), verde*80);
draw (surface(b -- f -- j2 -- k2 -- cycle), verde*80);

draw (surface(e -- h -- s2 -- t2 -- cycle), verde*80);
draw (surface(h -- d -- r2 -- s2 -- cycle), verde*80);
draw (surface(e -- a -- q2 -- t2 -- cycle), verde*80);

draw (u2 -- k2 -- j2 -- x2);

draw (surface(u2 -- k2 -- j2 -- x2 -- cycle), verde*80);
draw (surface(k2 -- u2 -- v2 -- o2 -- cycle), verde*80);
draw (surface(l2 -- q2 -- t2 -- i2 -- cycle), verde*80);
draw (surface(e2 -- i2 -- m2 -- h2 -- cycle), verde*80);
draw (surface(e2 -- i2 -- j2 -- f2 -- cycle), verde*80);
draw (surface(i2 -- t2 -- s2 -- m2 -- cycle), verde*80);
draw (surface(s2 -- m2 -- p2 -- r2 -- cycle), verde*80);
draw (surface(h2 -- g2 -- n2 -- m2 -- cycle), verde*80);
draw (surface(f2 -- g2 -- n2 -- j2 -- cycle), verde*80);
draw (surface(j2 -- n2 -- w2 -- x2 -- cycle), verde*80);
draw (surface(j2 -- x2 -- u2 -- k2 -- cycle), verde*80);
draw (surface(v2 -- o2 -- n2 -- w2 -- cycle), verde*80);

draw((.5,.5,1.5) -- (.5,.5,1.2), arrow=Arrow3,blue);
\end{asy}
\\
(B)

\end{figure}
\begin{figure}[H]
\centering

Projeção
\\
em Perspectiva
\vspace{.4cm}

\begin{asy}
size(4cm);
currentlight.background=box2;
currentprojection=perspective(.5,.5,1);

triple a = (0,0,0);
triple b = (1,0,0);
triple c = (1,1,0);
triple d = (0,1,0);

triple e = (0,0,1);
triple f = (1,0,1);
triple g = (1,1,1);
triple h = (0,1,1);

draw (a -- b -- c -- d -- cycle);
draw (e -- f -- g -- h -- cycle);
draw (a -- b -- f -- e -- cycle);
draw (c -- d -- h -- g -- cycle);

triple a1 = (0.1,0.1,0.1);
triple b1 = (.9,0.1,0.1);
triple c1 = (.9,.9,0.1);
triple d1 = (0.1,.9,0.1);

triple e1 = (0.1,0.1,.9);
triple f1 = (.9,0.1,.9);
triple g1 = (.9,.9,.9);
triple h1 = (0.1,.9,.9);

//draw (a1 -- b1 -- c1 -- d1 -- cycle);
//draw (e1 -- f1 -- g1 -- h1 -- cycle);
//draw (a1 -- b1 -- f1 -- e1 -- cycle);
//draw (c1 -- d1 -- h1 -- g1 -- cycle);

triple a2 = (0.1,0.1,0);
triple b2 = (.9,0.1,0);
triple c2 = (.9,.9,0);
triple d2 = (0.1,.9,0);

draw (a2 -- b2 -- c2 -- d2 -- cycle);

triple e2 = (0.1,0.1,1);
triple f2 = (.9,0.1,1);
triple g2 = (.9,.9,1);
triple h2 = (0.1,.9,1);

draw (e2 -- f2 -- g2 -- h2 -- cycle);

triple i2 = (0.1,0,.9);
triple j2 = (.9,0,.9);
triple k2 = (.9,0,.1);
triple l2 = (0.1,0,.1);

draw (i2 -- j2 --k2 -- l2 -- cycle);

triple m2 = (0.1,1,.9);
triple n2 = (.9,1,.9);
triple o2 = (.9,1,.1);
triple p2 = (0.1,1,.1);

draw (m2 -- n2 -- o2 -- p2 -- cycle);

triple q2 = (0,0.1,0.1);
triple r2 = (0,0.9,0.1);
triple s2 = (0,.9,.9);
triple t2 = (0,.1,.9);

draw (q2 -- r2 -- s2 --t2 -- cycle);

triple u2 = (1,0.1,0.1);
triple v2 = (1,0.9,0.1);
triple w2 = (1,.9,.9);
triple x2 = (1,.1,.9);

draw (u2 -- v2 -- w2 --x2 -- cycle);

draw (a -- a2 -- b2 -- b -- c -- c2 -- d2 -- d);
draw (e -- e2 -- f2 -- f -- g -- g2 -- h2 -- h);
draw (b -- u2 -- v2 -- c -- g -- w2 -- x2 -- f);
draw (a -- q2 -- r2 -- d -- h -- s2 -- t2 -- e);
draw (a -- l2 -- k2 -- b -- f -- j2 -- i2 -- e);
draw (d -- p2 -- o2 -- c -- g -- n2 -- m2 -- h);

draw (a2 -- l2 -- k2 -- b2);
draw (a2 -- l2 -- p2 -- d2);
draw (c2 -- o2 -- p2 -- d2);
draw (l2 -- q2 -- r2 -- p2);
draw (b2 -- k2 -- o2 -- c2);

draw (e2 -- i2 -- m2 -- h2 -- cycle);
draw (e2 -- i2 -- j2 -- f2 -- cycle);
draw (i2 -- t2 -- s2 -- m2 -- cycle);
draw (f2 -- g2 -- n2 -- j2 -- cycle);
draw (j2 -- x2 -- u2 -- k2 -- cycle);
draw (v2 -- o2 -- n2 -- w2 -- cycle);

draw (surface(a2 -- l2 -- k2 -- b2 -- cycle), verde*80);
draw (surface(a2 -- l2 -- p2 -- d2 -- cycle), verde*80);
draw (surface(c2 -- o2 -- p2 -- d2 -- cycle), verde*80);
draw (surface(l2 -- q2 -- r2 -- p2 -- cycle), verde*80);
draw (surface(b2 -- k2 -- o2 -- c2 -- cycle), verde*80);



draw (surface(a -- l2 -- k2 -- b -- cycle), verde*80);
draw (surface(b -- u2 -- v2 -- c -- cycle), verde*80);
draw (surface(c -- o2 -- p2 -- d -- cycle), verde*80);
draw (surface(a -- q2 -- r2 -- d -- cycle), verde*80);

draw (surface(b -- u2 -- v2 -- c -- cycle), verde*80);
draw (surface(b -- f -- x2 -- u2 -- cycle), verde*80);
draw (surface(f -- g -- w2 -- x2 -- cycle), verde*80);
draw (surface(g -- c -- v2 -- w2 -- cycle), verde*80);

draw (surface(c -- g -- n2 -- o2 -- cycle), verde*80);
draw (surface(g -- h -- m2 -- n2 -- cycle), verde*80);
draw (surface(h -- d -- p2 -- m2 -- cycle), verde*80);

draw (surface(e -- f -- f2 -- e2 -- cycle), verde*80);
draw (surface(f -- f2 -- g2 -- g -- cycle), verde*80);
draw (surface(g -- g2 -- h2 -- h -- cycle), verde*80);
draw (surface(h -- h2 -- e2 -- e -- cycle), verde*80);

draw (surface(a -- a2 -- b2 -- b -- cycle), verde*80);
draw (surface(b -- b2 -- c2 -- c -- cycle), verde*80);
draw (surface(c -- c2 -- d2 -- d -- cycle), verde*80);
draw (surface(d -- d2 -- a2 -- a -- cycle), verde*80);

draw (surface(a -- e -- i2 -- l2 -- cycle), verde*80);
draw (surface(e -- f -- j2 -- i2 -- cycle), verde*80);
draw (surface(b -- f -- j2 -- k2 -- cycle), verde*80);

draw (surface(e -- h -- s2 -- t2 -- cycle), verde*80);
draw (surface(h -- d -- r2 -- s2 -- cycle), verde*80);
draw (surface(e -- a -- q2 -- t2 -- cycle), verde*80);

draw (u2 -- k2 -- j2 -- x2);

draw (surface(u2 -- k2 -- j2 -- x2 -- cycle), verde*80);
draw (surface(k2 -- u2 -- v2 -- o2 -- cycle), verde*80);
draw (surface(l2 -- q2 -- t2 -- i2 -- cycle), verde*80);
draw (surface(e2 -- i2 -- m2 -- h2 -- cycle), verde*80);
draw (surface(e2 -- i2 -- j2 -- f2 -- cycle), verde*80);
draw (surface(i2 -- t2 -- s2 -- m2 -- cycle), verde*80);
draw (surface(s2 -- m2 -- p2 -- r2 -- cycle), verde*80);
draw (surface(h2 -- g2 -- n2 -- m2 -- cycle), verde*80);
draw (surface(f2 -- g2 -- n2 -- j2 -- cycle), verde*80);
draw (surface(j2 -- n2 -- w2 -- x2 -- cycle), verde*80);
draw (surface(j2 -- x2 -- u2 -- k2 -- cycle), verde*80);
draw (surface(v2 -- o2 -- n2 -- w2 -- cycle), verde*80);

draw((.5,.5,1.5) -- (.5,.5,1.2), arrow=Arrow3,blue);

\end{asy}
\\
(C)

\end{figure}
\end{multicols}

\caption{O que é uma \emph{vista}?}\label{\detokenize{GE301-6:fig-proj-vistas-observacao}}\label{\detokenize{GE301-6:id9}}
\end{figure}
% \begin{figure}[H]
% \centering
% \capstart

% \noindent\includegraphics[width=300bp]{{vistas-01_1}.jpg}
% \caption{O que é uma \emph{vista}?}\label{\detokenize{GE301-6:fig-proj-vistas-observacao}}\label{\detokenize{GE301-6:id9}}\end{figure}
\end{observation}

\begin{observation}

A Associação Brasileira de Normas Técnicas (ABNT) em seu documento NBR 10067 estabelece princípios gerais de representação em desenho técnico. Nele são consideradas seis vistas ortogonais simultâneas, três a mais ao que foi feito na \hyperref[\detokenize{GE301-6:ativ-proj-vistas-ortogonais}]{Atividade Vistas Ortogonais}. A \hyperref[\detokenize{GE301-6:fig-proj-vistas-ortogonais-06}]{Figura \ref{\detokenize{GE301-6:fig-proj-vistas-ortogonais-06}}} traz exatamente o exemplo apresentado neste documento. Uma versão interativa que pode ser acessada por meio de um navegador (inclusive o de seu celular) está disponível aqui: \url{https://www.geogebra.org/m/SR7HrtkN}.

\begin{figure}[H]
\centering
\begin{multicols}{2}
\begin{figure}[H]
\centering
\begin{asy}
size(5cm);

currentprojection=orthographic(.75,1,.5);
currentlight.background=box2;


triple A = (.4,.4,.35);
triple B = (.4,.7,.35);
triple C = (.7,.7,.35);
triple D = (.7,.4,.35);

triple E = (A+(0,0,.1));
triple F = (B+(0,0,.1));
triple G = (E+(.1,0,0));
triple H = (F+(.1,0,0));

triple I = (C+(0,0,.3));
triple J = (H+(0,0,.2));

triple K = (D+(0,0,.4));
triple L = (G+(0,0,.3));

triple M = (K+(0,.1,0));
triple N = (L+(0,.1,0));

draw(B -- C -- I -- J -- H -- F -- cycle);

draw(C -- D -- K -- M -- I -- cycle);
draw(D -- K -- L -- G -- E -- A -- cycle);
draw(K -- L -- N -- M -- cycle);
draw(M -- N -- J -- I -- cycle);

draw (A -- B -- C -- D -- cycle);
draw (E -- F -- H -- G -- cycle);
draw (A -- B -- F -- E -- cycle);

draw(surface(B -- C -- I -- J -- H -- F -- cycle), verde*80);

draw(surface(C -- D -- K -- M -- I -- cycle), verde*80);
draw(surface(D -- K -- L -- G -- E -- A -- cycle), verde*80);
draw(surface(K -- L -- N -- M -- cycle), verde*80);
draw(surface(M -- N -- J -- I -- cycle), verde*80);

draw (surface(A -- B -- C -- D -- cycle), verde*80);
draw (surface(E -- F -- H -- G -- cycle), verde*80);
draw (surface(A -- B -- F -- E -- cycle), verde*80);

draw (surface(H -- J -- N --L -- G -- cycle), verde*80);


triple x = (C - (.7,0,0));
triple w = (D - (.7,0,0));
triple p = (K - (.7,0,0));
triple q = (M - (.7,0,0));
triple y = (I - (.7,0,0));
triple z = (E - (.4,0,0));
triple u = (F - (.4,0,0));

draw (x -- w -- p --q --y -- cycle);
draw (z -- u, dashed);

draw (shift(1,0,0)*(x -- w -- p --q --y -- cycle));
draw (shift(1,0,0)*(z -- u));

triple x1 = (A + (0,-.4,0));
triple w1 = (D + (0,-.4,0));
triple p1 = (E + (0,-.4,0));
triple q1 = (G + (0,-.4,0));
triple y1 = (K + (0,-.4,0));
triple z1 = (L + (0,-.4,0));
triple u1 = (I + (0,-.7,0));
triple v1 = (J + (0,-.7,0));

draw (x1 -- w1 -- y1 -- z1 --q1 -- p1 -- cycle);
draw (u1 -- v1);

draw (shift(0,1,0)*(x1 -- w1 -- y1 -- z1 --q1 -- p1 -- cycle));
draw (shift(0,1,0)*(u1 -- v1), dashed);

triple x2 = (A + (0,0,-.35));
triple w2 = (B + (0,0,-.35));
triple p2 = (C + (0,0,-.35));
triple q2 = (D + (0,0,-.35));
triple y2 = (G + (0,0,-.45));
triple z2 = (H + (0,0,-.45));
triple u2 = (M + (0,0,-.75));
triple v2 = (N + (0,0,-.75));

draw (x2 -- w2 -- p2 -- q2 -- cycle);
draw (y2 -- z2);
draw (u2 -- v2);

draw (shift(0,0,1)*(x2 -- w2 -- p2 -- q2 -- cycle));
draw (shift(0,0,1)*(y2 -- z2), dashed);
draw (shift(0,0,1)*(u2 -- v2), dashed);

triple a = (0,0,0);
triple b = (1,0,0);
triple c = (1,1,0);
triple d = (0,1,0);

triple e = (0,0,1);
triple f = (1,0,1);
triple g = (1,1,1);
triple h = (0,1,1);

draw ((a -- b -- f -- e -- cycle));
draw ((a -- d -- h -- e -- cycle));
draw ((a -- b -- c -- d -- cycle));
draw ((b -- c -- g -- f -- cycle));
draw ((c -- d -- h -- g -- cycle));
\end{asy}

\end{figure}

\begin{figure}[H]
\centering

\begin{asy}
size(7cm);

currentprojection=orthographic(1.5,1.5,1/2);

triple A = (.4,.4,.35);
triple B = (.4,.7,.35);
triple C = (.7,.7,.35);
triple D = (.7,.4,.35);

triple E = (A+(0,0,.1));
triple F = (B+(0,0,.1));
triple G = (E+(.1,0,0));
triple H = (F+(.1,0,0));

triple I = (C+(0,0,.3));
triple J = (H+(0,0,.2));

triple K = (D+(0,0,.4));
triple L = (G+(0,0,.3));

triple M = (K+(0,.1,0));
triple N = (L+(0,.1,0));



pair a = (0,0);
pair b = (1,0);
pair c = (1,1);
pair d = (0,1);

pair A = (.7,.25);
pair B = (.7,.25);
pair C = (.7,.35);
pair D = (.6,.35);

pair E = (.6,.75);
pair F = (.4,.75);
pair G = (.4,.25);

pair H = (E - (0,.1));
pair I = (F - (0,.1));

draw ((a -- b -- c -- d -- cycle));
draw (shift(-.05)*(H -- I));
draw (shift(-.05)*(A -- B -- C -- D -- E -- F -- G -- cycle));

draw (reflect((2.5,0),(2.5,1))*shift(2,0)*(a -- b -- c -- d -- cycle));
draw (reflect((2.5,0),(2.5,1))*shift(2,0)*shift(-.05)*(H -- I), dashed);
draw (reflect((2.5,0),(2.5,1))*shift(2,0)*shift(-.05)*(A -- B -- C -- D -- E -- F -- G -- cycle));

pair A1 = (.4,.25);
pair B1 = (.7,.25);
pair C1 = (.7,.6);
pair D1 = (.5,.75);
pair E1 = (.4,.75);
pair F1 = (A1+(0,.1));
pair G1 = (B1+(0,.1));

draw (shift(1,0)*(a -- b -- c -- d -- cycle));

draw (shift(.95,0)*(A1 -- B1 -- C1 -- D1 -- E1) -- cycle);
draw (shift(.95,0)*(F1 -- G1), dashed);

draw (reflect((-.5,0),(-.5,1))*shift(-1,0)*(a -- b -- c -- d -- cycle));

draw (reflect((-.5,0),(-.5,1))*shift(-1.05,0)*(A1 -- B1 -- C1 -- D1 -- E1) -- cycle);
draw (reflect((-.5,0),(-.5,1))*shift(-1.05,0)*(F1 -- G1));

pair A2 = (.3,.3);
pair B2 = (.7,.3);
pair C2 = (.7,.7);
pair D2 = (.3,.7);
pair E2 = (.6,.7);
pair F2 = (.6,.3);
pair G2 = (.3,.6);
pair H2 = (.6,.6);

draw (shift(0,-1)*(a -- b -- c -- d -- cycle));

draw (shift(0,-1)*(A2 -- B2 -- C2 -- D2) -- cycle);
draw (shift(0,-1)*(E2 -- F2));
draw (shift(0,-1)*(G2 -- H2));

draw (shift(0,1)*(a -- b -- c -- d -- cycle));

draw (reflect((0,1.5),(1,1.5))*shift(0,1)*(A2 -- B2 -- C2 -- D2) -- cycle);
draw (reflect((0,1.5),(1,1.5))*shift(0,1)*(E2 -- F2), dashed);
draw (reflect((0,1.5),(1,1.5))*shift(0,1)*(G2 -- H2), dashed);


\end{asy}

\end{figure}
\end{multicols}
\begin{figure}[H]
\centering

\begin{asy}
size(8cm);

currentprojection=orthographic(.5,1,.5);
currentlight.background=box2;

triple a = (0,0,0);
triple b = (1,0,0);
triple c = (1,1,0);
triple d = (0,1,0);

triple e = (0,0,1);
triple f = (1,0,1);
triple g = (1,1,1);
triple h = (0,1,1);

triple ref1 = (rotate(45,a,e)*(d));
triple ref2 = (rotate(45,a,e)*(h));

triple A = (.4,.4,.35);
triple B = (.4,.7,.35);
triple C = (.7,.7,.35);
triple D = (.7,.4,.35);

triple E = (A+(0,0,.1));
triple F = (B+(0,0,.1));
triple G = (E+(.1,0,0));
triple H = (F+(.1,0,0));

triple I = (C+(0,0,.3));
triple J = (H+(0,0,.2));

triple K = (D+(0,0,.4));
triple L = (G+(0,0,.3));

triple M = (K+(0,.1,0));
triple N = (L+(0,.1,0));

draw(B -- C -- I -- J -- H -- F -- cycle);

draw(C -- D -- K -- M -- I -- cycle);
draw(D -- K -- L -- G -- E -- A -- cycle);
draw(K -- L -- N -- M -- cycle);
draw(M -- N -- J -- I -- cycle);

draw (A -- B -- C -- D -- cycle);
draw (E -- F -- H -- G -- cycle);
draw (A -- B -- F -- E -- cycle);

draw(surface(B -- C -- I -- J -- H -- F -- cycle), verde*80);

draw(surface(C -- D -- K -- M -- I -- cycle), verde*80);
draw(surface(D -- K -- L -- G -- E -- A -- cycle), verde*80);
draw(surface(K -- L -- N -- M -- cycle), verde*80);
draw(surface(M -- N -- J -- I -- cycle), verde*80);

draw (surface(A -- B -- C -- D -- cycle), verde*80);
draw (surface(E -- F -- H -- G -- cycle), verde*80);
draw (surface(A -- B -- F -- E -- cycle), verde*80);

draw (surface(H -- J -- N --L -- G -- cycle), verde*80);


triple x = (C - (.7,0,0));
triple w = (D - (.7,0,0));
triple p = (K - (.7,0,0));
triple q = (M - (.7,0,0));
triple y = (I - (.7,0,0));
triple z = (E - (.4,0,0));
triple u = (F - (.4,0,0));

draw (rotate(45,a,e)*(x -- w -- p --q --y -- cycle));
draw (rotate(45,a,e)*(z -- u), dashed);

draw (rotate(-45,b,f)*(shift(1,0,0)*(x -- w -- p --q --y -- cycle)));
draw (rotate(-45,b,f)*(shift(1,0,0)*(z -- u)));

triple x1 = (A + (0,-.4,0));
triple w1 = (D + (0,-.4,0));
triple p1 = (E + (0,-.4,0));
triple q1 = (G + (0,-.4,0));
triple y1 = (K + (0,-.4,0));
triple z1 = (L + (0,-.4,0));
triple u1 = (I + (0,-.7,0));
triple v1 = (J + (0,-.7,0));

draw (x1 -- w1 -- y1 -- z1 --q1 -- p1 -- cycle);
draw (u1 -- v1);

draw (rotate(100,ref1,ref2)*(shift(-sqrt(2)/2,-1+sqrt(2)/2,0)*shift(0,1,0)*(x1 -- w1 -- y1 -- z1 --q1 -- p1 -- cycle)));
draw (rotate(100,ref1,ref2)*(shift(-sqrt(2)/2,-1+sqrt(2)/2,0)*shift(0,1,0)*(u1 -- v1)), dashed);

triple x2 = (A + (0,0,-.35));
triple w2 = (B + (0,0,-.35));
triple p2 = (C + (0,0,-.35));
triple q2 = (D + (0,0,-.35));
triple y2 = (G + (0,0,-.45));
triple z2 = (H + (0,0,-.45));
triple u2 = (M + (0,0,-.75));
triple v2 = (N + (0,0,-.75));

draw (rotate(-45,a,b)*(x2 -- w2 -- p2 -- q2 -- cycle));
draw (rotate(-45,a,b)*(y2 -- z2));
draw (rotate(-45,a,b)*(u2 -- v2));

draw (rotate(45,e,f)*(shift(0,0,1)*(x2 -- w2 -- p2 -- q2 -- cycle)));
draw (rotate(45,e,f)*(shift(0,0,1)*(y2 -- z2)), dashed);
draw (rotate(45,e,f)*(shift(0,0,1)*(u2 -- v2)), dashed);



path3 Z1 = ((a -- b -- c -- d -- cycle));
path3 Z2 = ((e -- f -- g -- h -- cycle));
path3 X1 = ((a -- d -- h -- e -- cycle));
path3 X2 = ((b -- c -- g -- f -- cycle));
path3 Y1 = ((a -- b -- f -- e -- cycle));
path3 Y2 = ((c -- d -- h -- g -- cycle));



draw(rotate(-45, a, b)*Z1);
draw(rotate(45, e, f)*Z2);
draw(Y1);
draw(rotate(100,ref1,ref2)*(shift(-sqrt(2)/2,-1+sqrt(2)/2,0)*Y2));
draw(rotate(45,a, e)*X1);
draw(rotate(-45,b, f)*X2);

//draw (a -- b -- c -- d -- cycle);
//draw (e -- f -- g -- h -- cycle);
//draw (a -- b -- f -- e -- cycle);
//draw (c -- d -- h -- g -- cycle);
\end{asy}

\end{figure}

\caption{As seis vistas ortogonais do documento NBR 10067 da ABNT.}\label{\detokenize{GE301-6:fig-proj-vistas-ortogonais-06}}\label{\detokenize{GE301-6:id10}}
\end{figure}

% \begin{figure}[H]
% \centering
% \capstart

% \noindent\includegraphics[width=300bp]{{vistas-ortogonais-06}.jpg}
% \caption{As seis vistas ortogonais do documento NBR 10067 da ABNT.}\label{\detokenize{GE301-6:fig-proj-vistas-ortogonais-06}}\label{\detokenize{GE301-6:id10}}\end{figure}

Observamos que as definições das vistas ortogonais como definidas no documento da ABNT não são universais. Nos Estados Unidos, por exemplo, as projeções em cada par de planos paralelos no paralelepípedo da {\hyperref[\detokenize{GE301-6:fig-proj-vistas-ortogonais-06}]{\sphinxcrossref{\DUrole{std,std-ref}{As seis vistas ortogonais do documento NBR 10067 da ABNT.}}}} são permutados.
\end{observation}

\begin{knowledge}

Projeções ortogonais já foram um segredo militar!

A \index{Geometria Descritiva}Geometria Descritiva é o ramo da geometria que estuda a representação de objetos tridimensionais em duas dimensões através de um certo conjunto específico de procedimentos. As técnicas resultantes são importantes para a engenharia, a arquitetura, o design gráfico e as artes (\hyperref[\detokenize{GE301-6:fig-proj-geometria-descritiva-01}]{Figura \ref{\detokenize{GE301-6:fig-proj-geometria-descritiva-01}}}). A base teórica para a geometria descritiva é fornecida pelas projeções ortogonais.

\begin{figure}[H]
\centering
\capstart

\noindent\includegraphics[width=300bp]{{geometria-descritiva-01}.jpg}
\caption{Projeções ortogonais de um carro e de uma cabeça humana.}\label{\detokenize{GE301-6:fig-proj-geometria-descritiva-01}}\label{\detokenize{GE301-6:id11}}\end{figure}

O matemático francês Gaspard Monge (1746-1818) é considerado fundador da geometria descritiva. Ele a usou em engenharia militar (construção de fortificações) durante a época de Napoleão Bonaparte. De fato, geometria descritiva já foi considerada um segredo militar.

\begin{figure}[H]
\centering
\capstart

\noindent\includegraphics[width=200bp]{{gaspard-monge-01}.jpg}
\caption{Gaspard Monge (1746-1818).}\label{\detokenize{GE301-6:fig-proj-gaspar-monge-01}}\label{\detokenize{GE301-6:id12}}\end{figure}

Dennis Lieu e Sheryl Sorby, no excelente livro Visualization, Modeling, and Graphics for Engineering Design, apresentam o contexto histórico:
\begin{quote}

A pólvora começou a ser usada no mundo ocidental durante o Renascimento, assim como o canhão. Os canhões tornaram obsoletas a maioria das fortalezas construídas durante a era medieval. As muralhas não conseguiam suportar o impacto dos projéteis de canhão. Assim, as fortalezas precisavam ser remodeladas para suportar os tiros de canhão. Na França, um novo estilo de fortificação mais resistente foi então desenvolvido. A fortificação era construída com muros inclinados que ajudavam a defletir o tiro de canhão e não desmoronavam da mesma maneira que as muralhas planas verticais, quando atingidas diretamente. As novas fortalezas eram geometricamente mais complicadas de se construir do que suas predecessoras com muralhas verticais. Mais ainda, o perímetro da fortaleza evoluiu de um formato simples retangular para um formato pentagonal com uma extensão proeminente em cada ápice. Este formato de perímetro e o uso de muros inclinados resultaram em paredes que se justapunham em ângulos não usuais, os quais não podiam ser medidos facilmente ou diretamente. {[}…{]}

Felizmente, os franceses tinham Gaspard Monge, que desenvolveu uma técnica de análise gráfica chamada geometria descritiva. {[}…{]} As técnicas de geometria descritiva permitiram que os engenheiros da época criassem qualquer ponto de vista de um objeto geométrico a partir de dois pontos de vista existentes. Ao criar o ponto de vista apropriado, os engenheiros podiam observar e medir os atributos de um objeto. {[}…{]} A geometria complexa, os ângulos de interseção incomuns, e a altura das muralhas tinham a intenção de maximizar o fogo cruzado sobre um inimigo em aproximação sem revelar o interior da fortaleza. {[}…{]}

A astúcia dos franceses na construção de fortificações manteve a França como o principal poder europeu até o século XVIII. Na época, a geometria descritiva era considerada um segredo do estado francês, cuja divulgação era crime punível com a morte. Como resultado da aliança entre a França e o recém-constituído Estados Unidos, muitas fortificações dos EUA utilizaram projetos franceses. Como exemplo, temos o Forte McHenry que foi construído em 1806 e é primorosamente preservado em Baltimore, Maryland. O Forte McHenry sobreviveu ao bombardeamento inglês durante a Guerra de 1812 e tem importância porque ele inspirou Scott Key a escrever The Star Spangled Banner, o hino nacional dos EUA.
\end{quote}

\begin{figure}[H]
\centering
\capstart

\ifnum\aluno=1
\noindent\includegraphics[width=400bp]{{fig-fort-mchenry}.jpg}
\else
\noindent\includegraphics[width=390bp]{{fig-fort-mchenry}.jpg}
\fi

\caption{Forte McHenry em Baltimore, Maryland, EUA (fonte: IAN Image and Video Library).}\label{\detokenize{GE301-6:fig-proj-forte-01}}\label{\detokenize{GE301-6:id13}}\end{figure}
\end{knowledge}

\begin{knowledge}

O cientista cognitivo americano Douglas Richard Hofstadter (1945-) concebeu, para a capa de seu livro “Gödel Escher Bach: Um Entrelaçamento de Gênios Brilhantes”, um objeto bem peculiar: suas projeções ortogonais em três planos produzem as letras “G” (de Gödel), “E” (de Escher) e “B” (de Bach).

\begin{figure}[H]
\centering
\capstart

\noindent\includegraphics[width=225bp]{{geb-01}.png}
\caption{GEB (fonte: \href{https://www.flickr.com/photos/maxbraun/3205365815}{Max Brown}).}\label{\detokenize{GE301-6:id14}}\end{figure}

Inspirado por esta ideia, o Projeto CDME da Universidade Federal Fluminense concebeu um jogo para praticar visualização espacial e vocabulário: para cada objeto, você deve identificar as letras formadas por projeções e dispô-las em uma ordem a fim de formar uma palavra sem acentos do dicionário. Além do Português, existem fases em Inglês, Espanhol e Francês! O jogo pode ser acessado de qualquer navegador, incluindo o do smartphone.

\begin{figure}[H]
\centering

\noindent\includegraphics[width=50bp]{{triplets-qr}.png}
\end{figure}

% \begin{figure}[H]
% \centering
% \capstart

% \noindent\includegraphics[width=300bp]{{triplets-exemplo}.png}
% \caption{Jogo \href{http://www.cdme.im-uff.mat.br/html5/triplets/triplets-html/triplets-br.html}{Trip-Lets} do Projeto CDME da UFF.}\label{\detokenize{GE301-6:id15}}\end{figure}
\end{knowledge}

\begin{knowledge}

Em softwares especializados de computação gráfica (Blender, Autocad, Autodesk 3DS Max, etc.), um recurso comum é o assim denominado \emph{quad view} que faz com que o programa exiba quatro janelas de visualização simultâneas: três janelas com as três vistas ortogonais principais mais uma quarta janela com a projeção em perspectiva. As janelas com as vistas ortogonais são usadas para uma interação mais precisa com o objeto 3D, enquanto que a janela com a projeção em perspectiva permite visualizar como o objeto será visto de uma posição arbitrária.

\begin{figure}[H]
\centering
\capstart

\noindent\includegraphics[width=400bp]{{blender-04}.jpg}
\caption{Sistema \emph{quadview} no software gratuito de computação gráfica Blender.}\label{\detokenize{GE301-6:id16}}\end{figure}
\end{knowledge}

\begin{knowledge}

Existem perguntas sobre projeções ortogonais que a humanidade ainda não conhece as respostas. Vamos agora descrever uma dessas perguntas. Considere, a título de exemplo, um cubo que será projetado em um plano. Se você girar o cubo de forma que nenhuma de suas faces fique paralela à direção perpendicular a este plano de projeção, sua projeção será um polígono com 6 lados. Por este motivo, o cubo é denominado um \index{poliedro equiprojetivo}poliedro equiprojetivo de índice 6. Nesta construção interativa feita no GeoGebra \textless{}\url{https://www.geogebra.org/m/bF3y2m7K}\textgreater{} você pode constatar este fato.

\begin{figure}[H]
\centering

\noindent\includegraphics[width=250bp]{{poliedros-equiprojetivos-01}.png}
\end{figure}

Mais geralmente, dizemos que um poliedro é equiprojetivo de índice \(k\) se ao girá-lo de forma que nenhuma de suas faces fique paralela à direção perpendicular ao plano de projeção, sua projeção sobre este plano sempre será um polígono com \(k\) lados.

Observe que existem poliedros que não são equiprojetivos. O tetraedro regular, por exemplo. A imagem a seguir mostra duas posições do tetraedro regular nas quais nenhuma de suas faces é paralela à reta perpendicular ao plano horizontal. Em uma dessas posições, sua projeção é um polígono de 3 lados. Na outra, a projeção é um polígono de 4 lados.

\begin{multicols}{2}
\begin{figure}[H]
\centering
\begin{asy}
size(5cm);

currentprojection=perspective(2.53,1.60,1.27);

triple a = (0,0,0);
triple b = (1,0,0);
triple c = (1,1,0);
triple d = (0,1,0);

triple e = (0,0,1);
triple f = (1,0,1);
triple g = (1,1,1);
triple h = (0,1,1);

draw ((a -- b -- f -- e -- cycle));
draw ((a -- d -- h -- e -- cycle));
draw ((a -- b -- c -- d -- cycle));


triple A = ((1-sqrt(3)/3)/2,1/6,0);
triple B = ((1-sqrt(3)/3)/2,5/6,0);
triple C = (sqrt(3)/3+(1-sqrt(3)/3)/2,1/2,0);

triple D = (sqrt(3)/3+(1-sqrt(3)/3)/2-sqrt(3)*2/9,1/2,0);

//draw(A -- D -- B -- D -- C);
//draw (A -- B -- C -- cycle);

triple Q = (0,0,(1-sqrt(24)/9)/2);


triple A1 = (A+Q);
triple B1 = (B+Q);
triple C1 = (C+Q);

draw(surface(A1 -- B1 -- C1 -- cycle), verde*80);

triple D1 = (sqrt(3)/3+(1-sqrt(3)/3)/2-sqrt(3)*2/9,1/2,(1-sqrt(24)/9)/2+sqrt(24)/9);

path3 p1 = (A1 -- B1 -- C1 -- cycle);
path3 p2 = (A1 -- D1 -- C1 -- cycle);
path3 p3 = (A1 -- D1 -- B1 -- cycle);
path3 p4 = (B1 -- D1 -- C1 -- cycle);

draw(p1);			
draw(p2);
draw(p3);
draw(p4);

draw(planeproject(XY*unitsquare3)*p1);
draw(planeproject(XY*unitsquare3)*p2);
draw(planeproject(XY*unitsquare3)*p3);

draw(surface(A1 -- D1 -- C1 -- cycle), verde*80+opacity(.9));
draw(surface(A1 -- D1 -- B1 -- cycle), verde*80+opacity(.9));
draw(surface(B1 -- D1 -- C1 -- cycle), verde*80+opacity(.9));

draw((A1 -- B1 -- C1 -- cycle));
draw((A1 -- D1 -- C1 -- cycle));
draw((A1 -- D1 -- B1 -- cycle));
draw((B1 -- D1 -- C1 -- cycle));
\end{asy}
\caption{A projeção do tetraedo regular no plano horizontal formando um polígono de 3 lados}
\end{figure}

\begin{figure}[H]
\centering
\begin{asy}
size(5cm);

currentprojection=perspective(2.53,1.60,1.27);

triple a = (0,0,0);
triple b = (1,0,0);
triple c = (1,1,0);
triple d = (0,1,0);

triple e = (0,0,1);
triple f = (1,0,1);
triple g = (1,1,1);
triple h = (0,1,1);

draw ((a -- b -- f -- e -- cycle));
draw ((a -- d -- h -- e -- cycle));
draw ((a -- b -- c -- d -- cycle));


triple A = ((1-sqrt(3)/3)/2,1/6,0);
triple B = ((1-sqrt(3)/3)/2,5/6,0);
triple C = (sqrt(3)/3+(1-sqrt(3)/3)/2,1/2,0);

triple D = (sqrt(3)/3+(1-sqrt(3)/3)/2-sqrt(3)*2/9,1/2,0);

//draw(A -- D -- B -- D -- C);
//draw (A -- B -- C -- cycle);

triple Q = (0,0,(1-sqrt(24)/9)/2);

triple r1 = (.5,0,.5);
triple r2 = (.5,1,.5);


triple A1 = rotate(-54.5,r1,r2)*(A+Q);
triple B1 = rotate(-54.5,r1,r2)*(B+Q);
triple C1 = rotate(-54.5,r1,r2)*(C+Q);

triple D1 = rotate(-54.5,r1,r2)*(sqrt(3)/3+(1-sqrt(3)/3)/2-sqrt(3)*2/9,1/2,(1-sqrt(24)/9)/2+sqrt(24)/9);

path3 p1 = (A1 -- B1 -- C1 -- cycle);
path3 p2 = (A1 -- D1 -- C1 -- cycle);
path3 p3 = (A1 -- D1 -- B1 -- cycle);
path3 p4 = (B1 -- D1 -- C1 -- cycle);

draw(shift(0,0,.1)*p1);			
draw(shift(0,0,.1)*p2);
draw(shift(0,0,.1)*p3);
draw(shift(0,0,.1)*p4);

draw(planeproject(XY*unitsquare3)*p1);
draw(planeproject(XY*unitsquare3)*p2);
draw(planeproject(XY*unitsquare3)*p3);
draw(planeproject(XY*unitsquare3)*p4);

draw(shift(0,0,.1)*surface(A1 -- B1 -- C1 -- cycle), verde*80+opacity(.5));

draw(shift(0,0,.1)*surface(A1 -- D1 -- C1 -- cycle), verde*80+opacity(.9));
draw(shift(0,0,.1)*surface(A1 -- D1 -- B1 -- cycle), verde*80+opacity(.9));
draw(shift(0,0,.1)*surface(B1 -- D1 -- C1 -- cycle), verde*80+opacity(.9));

//draw((A1 -- B1 -- C1 -- cycle));
//draw((A1 -- D1 -- C1 -- cycle));
//draw((A1 -- D1 -- B1 -- cycle));
//draw((B1 -- D1 -- C1 -- cycle));
\end{asy}
\caption{A projeção do tetraedo no plano horizontal formando um polígono de 4 lados}
\end{figure}

\end{multicols}

Dado um valor \(\geq 3\) para \(k\), quais são todos os poliedros equiprojetivos de índice \(k\)? Esta é uma pergunta que ninguém conseguiu responder até o presente momento. Ela é um entre muitos problemas em aberto em Matemática.
\end{knowledge}


\know{Projeção mapeada}
\label{\detokenize{GE301-A::doc}}\label{\detokenize{GE301-A:sec-proj-saber-mais-e-projetos}}\label{\detokenize{GE301-A:para-saber-mais-e-sugestoes-de-projetos}}

\label{\detokenize{GE301-A:sub-projecaomapeada}}\label{\detokenize{GE301-A:projecao-mapeada}}
Projeção mapeada (\emph{mapping projection}, em inglês) é um termo empregado para designar um conjunto de técnicas usadas para projetar vídeos sobre superfícies diversas tais como fachadas de prédios, monumentos, corpo humano e etc. Com esta técnica é possível cobrir um objeto real com imagens que se adaptam perfeitamente às irregularidades da superfície de projeção, podendo assim criar ilusões de ótica, dar vida a objetos estáticos, obter texturas e efeitos tridimensionais, e etc. Para isso, são necessários projetores e softwares especializados que reconstroem o espaço real através da adição de um espaço virtual cuidadosamente projetado.

A projeção em perspectiva que você aprendeu neste capítulo é a utilizada nesta técnica. Neste caso, os vídeos são criados respeitando a perspectiva da superfície de projeção em relação ao projetor e, também, suas características como tamanho, forma e etc.

Durante a cerimônia de abertura dos jogos olímpicos de 2016 na cidade do Rio de Janeiro, a projeção mapeada foi utilizada em diversos momentos. 106 projetores foram usados para cobrir com imagens o campo do estádio do Maracanã (onde ocorreu o evento) e outros objetos cenográficos criando um cenário virtual rico em detalhes, cores e emoção.  As imagens abaixo, mostram dois momentos diferentes do belíssimo espetáculo proporcionado pelas projeções mapeadas.

\begin{figure}[H]
\centering
\capstart

\noindent\includegraphics[width=\linewidth]{{OpeningCerimonyRio2016_montagem}.jpg}
\caption{Imagens da cerimônia de abertura dos jogos olímpicos de 2016 na cidade do Rio de Janeiro (Fonte: \href{http://radiografico.com.br/projects/93}{site radiografico}).}\label{\detokenize{GE301-A:id2}}\end{figure}

Esta também é a técnica usada pela Walt Disney World em diversos shows e atrações de seus parques de diversão desde 1969. Um exemplo, é o show \emph{Happily ever after}, onde o Castelo da Cinderela (localizado no parque \emph{Magic Kingdon} em Orlando-EUA) se transforma em uma enorme tela de projeção que reproduz a história de diferentes personagens da Disney. Para saber sobre o processo de criação das projeções no castelo, recomendamos o vídeo \emph{Advanced Projection Mapping Tech Coming to ‘Happily Ever After’} disponibilizado pela Disney Parks em seu \href{https://www.youtube.com/watch?v=WGrKuosEkWw}{canal do YouTube}. O vídeo tem duração de 2 minutos e as legendas em inglês podem ser ativadas.

\begin{figure}[H]
\centering
\capstart

\noindent\includegraphics[width=350bp]{{CasteloDisney3}.jpg}
\caption{Vídeo \emph{Advanced Projection Mapping Tech Coming to ‘Happily Ever After’} disponível em \textless{}\url{https://www.youtube.com/watch?v=WGrKuosEkWw}\textgreater{}.}\label{\detokenize{GE301-A:id3}}\end{figure}

O Cristo Redentor, famoso ponto turístico da cidade do Rio de Janeiro, fechou os braços em um abraço simbólico à cidade através de projeção mapeada. Esse abraço foi parte da campanha de combate à violência e exploração sexual de crianças “Carinho de Verdade”, idealizada pelo cineasta brasileiro Fernando Salis em 2010. Embalados pelo som das Bachianas Brasileiras número 7 de Villa Lobos, oito projetores cobriram a estátua do Cristo com imagens do Rio de Janeiro e criaram uma ilusão de movimento dos braços da estátua. Um casamento perfeito entre Arte, Matemática e Tecnologia!

Para assistir a todas as projeções feitas no Cristo Redentor nesta campanha, sugerimos o vídeo \emph{Projeção do abraço do Cristo no Rio, de Fernando Salis 19/10/2010} disponibilizado pelo próprio cineasta em \href{https://www.youtube.com/watch?v=PNzi5JS46U8}{seu canal do YouTube}. Não se esqueça de ativar o som para assistir esse vídeo que vai te impressionar!

\begin{figure}[H]
\centering
\capstart

\noindent\includegraphics[width=280bp]{{AbracoCristo}.jpg}
\caption{Vídeo \emph{Projeção do abraço do Cristo no Rio, de Fernando Salis 19/10/2010} disponível em \textless{}\url{https://www.youtube.com/watch?v=PNzi5JS46U8}\textgreater{}.}\label{\detokenize{GE301-A:id4}}\end{figure}

Um outro exemplo de utilização desta técnica pode ser visto no vídeo \emph{Omote real times face tracking \& projection mapping}, também \href{https://www.youtube.com/watch?v=eVNDYgMrvUU}{disponível no YouTube}. Neste caso, a superfície de projeção escolhida é a face de uma pessoa, que se transforma através de diferentes texturas. O vídeo tem duração de 2:18 minutos e não precisa de legendas.

\begin{figure}[H]
\centering
\capstart

\ifnum\aluno=1
\noindent\includegraphics[width=300bp]{{FaceProjection}.jpg}
\else
\noindent\includegraphics[width=290bp]{{FaceProjection}.jpg}
\fi

\caption{Vídeo \emph{Omote real times face tracking \& projection mapping} disponível em \textless{}\url{https://www.youtube.com/watch?v=eVNDYgMrvUU}\textgreater{}.}\label{\detokenize{GE301-A:id5}}\end{figure}

A seguir, listaremos uma série de vídeos da plataforma YouTube que ilustram o uso desta técnica impressionante:
\begin{enumerate}
\item {} 
\url{https://www.youtube.com/watch?v=lX6JcybgDFo}

\item {} 
\url{https://www.youtube.com/watch?v=D6EPGutC9Z0}

\item {} 
\url{https://youtu.be/PKMCB5v8pt0}

\item {} 
\url{https://www.youtube.com/watch?v=P1az8bbuOLg}

\end{enumerate}

Se você ficou com vontade de tentar utilizar a projeção mapeada, sugerimos acessar o site \url{http://projection-mapping.org/} e escolher um dos softwares disponíveis de acordo com seu sistema operacional. Divirta-se!


\subsection{Projeções no cinema}
\label{\detokenize{GE301-A:projecoes-no-cinema}}\label{\detokenize{GE301-A:sub-cinema}}
Para criar efeitos visuais nos filmes, é muito comum usar técnicas que envolvem projeção em perspectiva. Uma técnica bastante conhecida é chamada perspectiva forçada. A perspectiva forçada é uma ilusão de ótica que faz com que objetos pareçam maiores/menores do que são ou mais próximos/distantes uns dos outros.

Esta técnica foi amplamente utilizada nos filmes \emph{O Senhor dos Anéis}, uma trilogia baseada nos livros do britânico J. R. R. Tolkien. Em cenas onde aparecem os \emph{Hobbits} (criaturas pequenas se comparadas com humanos mas que foram encenadas por humanos), eles eram posicionados mais distantes da câmera enquanto os outros personagens da mesma cena eram posicionados mais próximos da câmera. Na perspectiva da câmera, quem está mais perto dela é maior do que quem está mais longe. E assim, os \emph{Hobbits} ficaram de tamanho reduzido no filme! Na cena mostrada na figura abaixo, duas mesas são utilizadas para haver afastamento dos personagens em relação à câmera, mas no filme essas duas mesas parecem uma só. Se você nunca assistiu essa trilogia, sugerimos que o faça e fique atento para os truques de perspectiva usados no filme. A Matemática realmente está presente onde menos esperamos!

\begin{figure}[H]
\centering
\capstart

\noindent\includegraphics[width=325bp]{{LordOfRings}.jpg}
\caption{Vídeo \emph{How Lord of the Rings used forced perspective shots with a moving camera} disponível em \textless{}\url{https://www.youtube.com/watch?v=QWMFpxkGO\_s}\textgreater{}.}\label{\detokenize{GE301-A:id6}}\end{figure}

Se você quiser entender todos os detalhes da perspectiva forçada, assista o vídeo \emph{The Math and Science of Forced Perspective} disponível em \textless{}\url{https://www.youtube.com/watch?v=pl4ah\_HvWkg\&t=187s}\textgreater{}. O vídeo tem cerca de 15 minutos e possui legendas em inglês.

Outro efeito utilizado no cinema que faz uso de projeções é o Efeito Vertigo (\emph{Dolly Zoom}, em inglês). Esse efeito é produzido quando aumentamos ou diminuímos o alcance da lente através do zoom enquanto alteramos a posição da câmera na direção oposta ao zoom. Assim, o personagem da cena permanece no foco, enquanto a perspectiva visual muda com a aproximação ou afastamento  do cenário.

% \begin{figure}[H]
% \centering
% \capstart

% \noindent\includegraphics[width=300bp]{{Vertigo}.png}
% \caption{Cena da torre do sino do filme Vertigo que utilizou a técnica (Fonte: \url{https://www.youtube.com/watch?v=sKJeTaIEldM}).}\label{\detokenize{GE301-A:id7}}\end{figure}

Utilizado no famoso filme \emph{Vertigo} dirigido por Alfred Hitchcock, que deu nome ao efeito aqui no Brasil, as cenas da torre do sino da capela e da perseguição policial que acabou com a morte de um policial são excelentes exemplos de uso dessa técnica. O movimento simultâneo da câmera e o uso do zoom causam uma distorção visual, gerando até mesmo náuseas em quem assiste a cena. O filme recebeu muitas críticas na ocasião de sua estréia, mas hoje é conhecido como uma obra de arte de Hitchcock.

\begin{figure}[H]
\centering
\capstart

\noindent\includegraphics[width=140bp]{{AlfredHitchcock}.jpg}
\caption{Alfred Hitchcock, o célebre diretor de cinema, o primeiro a utilizar o efeito Vertigo no cinema (Fonte: Wikimedia Commons).}\label{\detokenize{GE301-A:id8}}\end{figure}

Para saber mais, vamos listar alguns vídeos da plataforma YouTube que podem ajudá-lo a entender melhor o efeito criado com esta técnica:
\begin{enumerate}
\item {} 
\url{https://www.youtube.com/watch?v=neaOds5\_3js} disponível com áudio em português.

\item {} 
\url{https://www.youtube.com/watch?v=sKJeTaIEldM}

\item {} 
\url{https://www.youtube.com/watch?v=WIpMtL68G8w}

\end{enumerate}

Você poder tentar simular este efeito usando uma câmera. Se reúna com um colega, organize uma cena interessante e tente usar o \emph{zoom in} ao mesmo tempo que se afasta da cena, ou então, o contrário. Cuidado com o efeito vertigo!


\subsection{Jogos que utilizam projeções}
\label{\detokenize{GE301-A:jogos-que-utilizam-projecoes}}\label{\detokenize{GE301-A:sub-jogos}}
Vários jogos concretos ou digitais utilizam projeções, sejam elas apenas para criar a cena do jogo ou como uma componente da dinâmica do jogo. E, em alguns, as ambiguidades existentes nas projeções em perspectiva são o tema principal.  É claro que as projeções, neste caso, aparecem de uma forma mais informal do que a estudada neste capítulo, mas por trás da confecção do jogo, toda a Matemática aqui discutida certamente foi utilizada.

Os jogos para \emph{Playstation} chamados \emph{Echochrome} e \emph{Echochrome II} foram os pioneiros em utilizar efeitos de ilusão de ótica em sua dinâmica. Estes dois jogos inspiraram a criação de muitos jogos que conhecemos hoje. Em \emph{Echochrome}, o objetivo é levar o personagem através de um caminho do começo ao fim. Para isso, são necessárias mudanças de perspectiva que transformam caminhos impossíveis em factíveis.

% \begin{figure}[H]
% \centering
% \capstart

% \noindent\includegraphics[width=300bp]{{Echochrome3}.png}
% \caption{Jogo \emph{Echochrome} (Fonte: \url{https://www.youtube.com/watch?v=GybxIwfU4rI})}\label{\detokenize{GE301-A:id9}}\end{figure}

Já em \emph{Echochrome II}, para conduzir o personagem do início ao fim do caminho, são utilizadas luzes e sombras, que criam as ilusões de ótica por onde o personagem deve caminhar. Cada jogador enxerga de uma forma diferente os enigmas presentes no jogo, e portanto, existem diferentes possibilidades de condução do personagem.

% \begin{figure}[H]
% \centering
% \capstart

% \noindent\includegraphics[width=300bp]{{EchochromeII3}.png}
% \caption{Jogo \emph{Echochrome II} (Fonte: \url{https://www.youtube.com/watch?v=bWMSpmqVUOY})}\label{\detokenize{GE301-A:id10}}\end{figure}

A seguir, vamos listar alguns jogos que envolvem projeção. Escolha um jogo, utilize tudo que você aprendeu até aqui e se divirta!

\paragraph{Jogos para vídeo-games}
\begin{itemize}
\item {} 
Fez: este é um jogo do tipo quebra-cabeça onde o personagem principal, chamado Gomez, imagina viver em um mundo 2D, mas ao receber um chapéu \emph{Fez} percebe que o mundo é 3D. O jogador tem que ajudar Gomez a viver nesse mundo novo considerando as projeções do mundo 3D que possui. O objetivo é realinhar plataformas e resolver charadas para restaurar a ordem do universo.

\end{itemize}

Para entender melhor o funcionamento do jogo, assista o vídeo disponível em \textless{}\url{https://www.youtube.com/watch?v=HFNIH3m6i2s}\textgreater{}. Este vídeo possui áudio em português e cerca de 11 minutos.

Este jogo está disponível para \emph{Playstation} e \emph{Xbox}.

\begin{figure}[H]
\centering
\capstart

\noindent\includegraphics[width=\linewidth]{{Fez}.png}
\caption{Jogo Fez (Imagens de divulgação)}\label{\detokenize{GE301-A:id11}}\end{figure}
\begin{itemize}
\item {} 
The Bridge: neste jogo, o objetivo é levar o personagem principal Escher para a porta de saída. O caminho a ser percorrido é cheio de  enigmas que são inspirados nas obras do artista M. C. Escher. O jogador tem que desafiar a gravidade, girar a cena, abrir portas, subir escadas e etc para assim conduzir o personagem pelo caminho da saída.

\end{itemize}

% \begin{figure}[H]
% \centering
% \capstart

% \noindent\includegraphics[width=300bp]{{TheBridge2}.png}
% \caption{Jogo \emph{The Bridge} (Fonte: \url{https://www.youtube.com/watch?v=h8hOGbdoJdw})}\label{\detokenize{GE301-A:id12}}\end{figure}

Este jogo está disponível para \emph{Playstation} e \emph{Xbox}, assim como para computadores e celulares.

\paragraph{Jogos para celular}
\begin{itemize}
\item {} 
Monument Valley: neste jogo, o jogador deve resolver quebra-cabeças inspirados nas criações do artista M. C. Escher. É claro que os cenários são cheios de ilusões de ótica e arquiteturas que desafiam a lógica. O jogo está disponível para IOS e Android.

\end{itemize}

\begin{figure}[H]
\centering
\capstart

\noindent\includegraphics[width=210bp]{{MonumentValley}.png}
\caption{Jogo Monument Valley (Fonte: \url{https://www.monumentvalleygame.com/})}\label{\detokenize{GE301-A:id13}}\end{figure}
\begin{itemize}
\item {} 
Shadowmatic: o objetivo deste jogo é trabalhar com as sombras de objetos projetadas sobre uma parede provocadas por uma fonte de luz fora da tela. Os objetos que são projetados normalmente não se assemelham a nenhum objeto real e o jogador deve movimentá-lo até a sombra formar uma silhueta reconhecível. O jogo está disponível para IOS e Android.

\begin{figure}[H]
\centering
\capstart

\noindent\includegraphics[width=240bp]{{Shadowmatic}.png}
\caption{Jogo Shadowmatic (Fonte: \url{https://www.shadowmatic.com/})}\label{\detokenize{GE301-A:id14}}\end{figure}

\end{itemize}



\paragraph{Jogo para computador}
\begin{itemize}
\item {} 
Perspective: neste jogo, o jogador deve mover o avatar em um cenário 3D que muda de acordo com a perspectiva. Nele nada é impossível, e sim questão de perspectiva! O jogo é gratuito e pode ser acessado no site \textless{}\url{http://games.digipen.edu/games/perspective\#.WnW81ZOpmCQ.}\textgreater{}

\end{itemize}

% \begin{figure}[H]
% \centering
% \capstart

% \noindent\includegraphics[width=320bp]{{Perspective2}.png}
% \caption{Jogo Perspective (Fonte: \url{https://www.youtube.com/watch?v=SS4r9Fq3beU\&t=25s})}\label{\detokenize{GE301-A:id15}}\end{figure}

\paragraph{Jogos concretos}
\begin{itemize}
\item {} 
La Boca: neste jogo, o jogador escolhe um outro jogador dentre os demais competidores que o ajudará a construir um sólido geométrico com um conjunto de blocos menores disponibilizados pelo jogo. Cada um dos dois jogadores terá acesso à informações de apenas uma face do sólido (uma vista do sólido), sendo uma oposta à outra, e assim, cooperativamente eles devem fazer a construção.

\end{itemize}

\begin{figure}[H]
\centering
\capstart

\noindent\includegraphics[width=210bp]{{LaBocaJogo}.jpg}
\caption{Jogo La Boca produzido pela empresa alemã Kosmos (Imagem de divulgação)}\label{\detokenize{GE301-A:id16}}\end{figure}

O vídeo em português disponível em \textless{}\url{https://www.youtube.com/watch?v=n7yiM\_zak0Y}\textgreater{} vai te ajudar a entender melhor o andamento do jogo. São apenas 3 minutos de vídeo. Vale a pena assistir!

OBS.: Se você ficou curioso com o nome do jogo, saiba que ele é inspirado na rua \emph{Caminito} do bairro \emph{La Boca}, que fica localizado em Buenos Aires-Argentina. Esta é uma rua com muitas casas com fachadas coloridas e ponto turístico certo de quem visita a cidade.
\begin{itemize}
\item {} 
Papertown: neste jogo, dois jogadores que são adversários, devem construir uma cidade (chamada \emph{Paper Town}, que traduzindo para português seria \emph{Cidade de Papel}) com peças em papel que apresentam partes da cidade desenhadas em perspectiva. A perspectiva deve ser respeitada durante todo o decorrer da partida. Esse é um jogo que envolve criatividade, imaginação e muita geometria.

\end{itemize}

\begin{figure}[H]
\centering
\capstart

\noindent\includegraphics[width=400bp]{{Papertown}.png}
\caption{Jogo Papertown produzido pela editora brasileira RedBox (Fonte: \url{http://rodrigorego.com.br/papertown.html})}\label{\detokenize{GE301-A:id17}}\end{figure}

Assista a uma partida do jogo disponível em \textless{}\url{https://www.youtube.com/watch?v=b\_uhElq1sWM}\textgreater{}. O vídeo possui áudio em português e cerca de 33 minutos.

\paragraph{Jogos em desenvolvimento}
\begin{itemize}
\item {} 
Graybles: neste jogo são mostradas várias perspectivas de uma mesma cena e o jogador deve percorrer um caminho correto usando informações de todas as perspectivas. Veja o vídeo disponível em \textless{}\url{https://www.youtube.com/watch?v=ub3UM30-vcI}\textgreater{} para uma demonstração do jogo. O vídeo é bem rápido e possui legendas em inglês.

\item {} 
Pillow Castle: neste jogo é utilizada a projeção forçada (já discutida nesta seção) para criar ilusões de ótica e assim montar diferentes cenários. Este vídeo com legendas em inglês com duranção de cerca de 7 minutos vai te surpreender: \textless{}\url{https://www.youtube.com/watch?v=HOfll06X16c}\textgreater{}.

\end{itemize}


\subsection{Teatro de sombras}
\label{\detokenize{GE301-A:teatro-de-sombras}}\label{\detokenize{GE301-A:sub-teatrodesombras}}
Não é apenas em plataformas digitais que as projeções podem ser úteis. Vamos entender um pouco como as projeções podem ser utilizadas para criar espetáculos de teatro, chamados teatro de sombras, que são vistos como os precursores do cinema.

O \emph{teatro de sombras} é uma forma bem antiga de contar histórias com o auxílio de sombras, criadas por bonecos, que dão vida aos personagens da história. Uma fonte de luz incide sobre uma tela translúcida que oculta os bonecos, deixando visíveis apenas suas sombras. Os bonecos são controlados por pessoas que também ficam ocultas durante o espetáculo e que, normalmente, também confeccionam os bonecos. A história é contada utilizando as sombras geradas tanto pela movimentação dos bonecos quanto da fonte de luz.

No vídeo \emph{Traditional Chinese Shadow Puppet Performance, Bazhong, China} disponível neste \href{https://vimeo.com/41524173}{link} é possível assistir a um teatro de sombras tradicional da China. Neste caso, a fonte de luz é o próprio sol e os personagens são as sombras de bonecos confeccionados com tecido, papel e pequenos gravetos que são usados para sua manipulação. O vídeo possui o som do ambiente, produzido ao vivo por artistas da região.

\begin{figure}[H]
\centering
\capstart

\noindent\includegraphics[width=300bp]{{TradicionalChineseShadowPuppetPerformance}.png}
\caption{Vídeo \emph{Traditional Chinese Shadow Puppet Performance, Bazhong, China} \textless{}\url{https://vimeo.com/41524173}\textgreater{}.}\label{\detokenize{GE301-A:id18}}\end{figure}

Esta forma de arte remonta da pré-história quando o homem se encantava com suas sombras projetadas nas paredes de cavernas, mas não há um consenso sobre a origem exata do que chamamos de teatro de sombras. Há uma lenda que se passa no ano 121 na China que pode ser a origem dessa arte milenar. Segundo a lenda, o imperador Wu Ti, da dinastia Han, se desesperou com a morte de sua bailarina favorita e, então, ordenou ao mago da corte que a trouxesse de volta do “Reino das Sombras”, caso contrário ele seria decapitado. O mago então confeccionou a silhueta da bailarina com pele de peixe e, usando um lençol que deixava transparecer a luz do sol, ao som de uma flauta, recriou os movimentos leves e graciosos da bailarina no jardim do castelo do imperador. Neste momento, é possível que tenha surgido uma das mais antigas formas de projeção, que nos remete ao que estudamos neste capítulo.

% \begin{figure}[H]
% \centering
% \capstart

% \noindent\includegraphics[width=230bp]{{TeatroSombra_Lenda}.png}
% \caption{Vídeo \emph{Teatro de Sombras} \textless{}\url{https://www.youtube.com/watch?v=QXMlVgNquNs}\textgreater{} que mostra uma encenação da lenda.}\label{\detokenize{GE301-A:id19}}\end{figure}

Em 2011, o teatro de sombras da China foi denominado Patrimônio Cultural da Humanidade pela \emph{Unesco}. Este tipo de arte ainda é muito comum no país, onde a habilidade de manipular simultaneamente vários bonecos passa de pai para filho. Segundo a Unesco, este tipo de arte difunde conhecimento, promove valores culturais e entretém a comunidade. Sugerimos que você assista o \href{https://www.youtube.com/watch?v=8-mzqxZNp2g}{vídeo disponível} no canal da Unesco do YouTube. Você vai se impressionar com a habilidade dos manipuladores!

% \begin{figure}[H]
% \centering
% \capstart

% \noindent\includegraphics[width=230bp]{{Unesco}.png}
% \caption{Vídeo \emph{Chinese shadow puppetry} \textless{}\url{https://www.youtube.com/watch?v=8-mzqxZNp2g}\textgreater{}.}\label{\detokenize{GE301-A:id20}}\end{figure}

Em sua trajetória histórica, o teatro de sombras adquiriu características de acordo com as diversas culturas das regiões que o produziram e se popularizou especialmente na Ásia, em países como China, Indonésia, Malásia, Tailândia, Camboja, Índia e Nepal. Se você quiser saber um pouco mais sobre a história dessa arte e seus desdobramentos nos diversos países, sugerimos o livro \emph{Shadow Puppets and Shadow Play} de David Currel.

Nos dias atuais, algumas companhias de dança e teatro espalhadas pelo mundo continuam a criar espetáculos de sombra utilizando mais que bonecos e a luz do sol. É possível encontrar espaços destinados apenas a este tipo de arte, como o teatro russo \emph{Shadow Fireflies Theater} que possui um grupo de artistas fixos que encenam seus espetáculos. As sombras utilizadas nestes espetáculos são dos corpos dos próprios artistas que criam silhuetas humanas ou objetos diversos para contar suas histórias. Recomendamos, em especial, o vídeo \emph{Shadow Theatre “Fireflies” - New Year’s Dream} disponível do canal do teatro no \href{https://www.youtube.com/watch?v=AzS0VwXOlWs}{YouTube}.

% \begin{figure}[H]
% \centering
% \capstart

% \noindent\includegraphics[width=300bp]{{Russia}.png}
% \caption{Vídeo \emph{Shadow Theatre “Fireflies” - New Year’s Dream} \textless{}\url{https://www.youtube.com/watch?v=AzS0VwXOlWs}\textgreater{}.}\label{\detokenize{GE301-A:id21}}\end{figure}

No programa de TV inglês \emph{Britain’s Got Talent 2013}, o grupo de teatro de sombras húngaro chamado \emph{Attraction} emocionou os jurados com uma apresentação ao vivo de tirar o fôlego. Você pode assistir a esta apresentação na íntegra no canal do programa no \href{https://www.youtube.com/watch?v=JOZS\_Vq6eKw}{YouTube}. Vale a pena conferir!

% \begin{figure}[H]
% \centering
% \capstart

% \noindent\includegraphics[width=300bp]{{BGT}.png}
% \caption{Vídeo \emph{Attraction’s semi-final shadow theatre performance\textbar{} Semi-Final 5\textbar{}Britain’s Got Talent 2013} \textless{}\url{https://www.youtube.com/watch?v=JOZS\_Vq6eKw}\textgreater{}.}\label{\detokenize{GE301-A:id22}}\end{figure}

E se você ficou com vontade de encenar um espetáculo com seus amigos ou na sua escola, sugerimos que você assista os vídeos abaixo. Para garantir o efeito das sombras, você deve usar uma lanterna e uma parede como tela de projeção semelhante ao que foi feito na atividade da lanterna na seção X. Você também pode construir um mini teatro com materiais reciclados como mostra um dos vídeos a seguir e tentar usar até mesmo uma vela como fonte de luz.
\begin{enumerate}
\item {} 
\url{https://www.youtube.com/watch?v=BUqpQVbEX9M}

\item {} 
\url{https://www.youtube.com/watch?v=Gx7nw5QC0zQ}

\item {} 
\url{https://www.youtube.com/watch?v=Uv-MdaBfk8U}

\item {} 
\url{https://www.youtube.com/watch?v=-hL28SkHf1g}

\item {} 
\url{https://www.youtube.com/watch?v=gzAUIXu7-pY}

\end{enumerate}

% \begin{figure}[H]
% \centering
% \capstart

% \noindent\includegraphics[width=300bp]{{Maos}.png}
% \caption{Vídeo \emph{Como fazer SOMBRAS DE ANIMAIS com as mãos para crianças} \textless{}\url{https://www.youtube.com/watch?v=Gx7nw5QC0zQ}\textgreater{}.}\label{\detokenize{GE301-A:id23}}\end{figure}

No endereço \url{http://fabianaeaarte.blogspot.com/2012/06/teatro-de-sombras.html} você também pode encontrar um projeto pronto para construir seu próprio espetáculo com sombras. Convide um colega e mãos a obra!

\clearpage
\def\currentcolor{cor1}
% Página 1
\begin{answer}{Exercícios}
{\exerciselist
\begin{enumerate}
\item Alternativa \textit{e)}

Em outras palavras, o exercício está pedindo que encontremos a projeção ortogonal sobre o plano da mesa dos objetos posicionados sobre ela.

Observando a cena de cima, o cubo menor e os dois cubos empilhados estão em cantos opostos e em diferentes linhas. Portanto, eliminamos as possibilidades \textit{a)} e \textit{d)}. Além disso, no sentido horário, estão posicionados: a pirâmide, o cubo, o cilindro e os dois cubos empilhados. A única alternativa, entre as opções \textit{b)}, \textit{c)} e \textit{e)}, em que podemos encontrar esta ordenação no sentido horário é a alternativa \textit{e)}.
\end{enumerate}
}{1}
\end{answer}
\clearmargin

% Página 2
\begin{answer}{Exercícios}
{\exerciselist
\begin{enumerate}\setcounter{enumi}{1}
\item Alternativa \textit{d)}

Para responder este exercício, basta observar que \(d=\frac23 d'\) e usar semelhança de triângulos. Veja a figura abaixo:
\phantomsection\label{\detokenize{GE301-E:fig-projecoes-enem2009-solucao}}
\begin{figure}[H]
\centering

\noindent\includegraphics[width=.7\linewidth]{{ENEM2009_Q69}.png}
\label{\detokenize{GE301-E:fig-projecoes-enem2009-solucao}}\end{figure}

Como o triângulo \(ABC\) é semelhante ao triângulo \(ADE\) (\(\hat{A}\) é ângulo comum aos dois triângulos e \(m(\hat{B})=m(\hat{D})=90^0\)), temos:
\begin{equation*}
\begin{split}\frac{a}{b}=\frac{c}{2/3c'} \Longleftrightarrow \frac{b}{a}=\frac{2d'}{3c}.\end{split}
\end{equation*}
Portanto, a opção correta é a apresentada no item \textit{d)}.

\item Alternativa \textit{e)}.

Os pontos \(A\) e \(B\) estão localizados sobre um mesmo paralelo do globo terrestre, isto implica que o arco \(AB\) está contido em um plano paralelo ao plano \(\alpha\). Portanto sua projeção ortogonal em \(\alpha\) será também um arco congruente à \(AB\). Já os pontos \(B\) e \(C\) estão localizados sobre um mesmo meridiano e não são simétricos em relação à linha do Equador. Se tormarmos um plano contendo \(B\), \(C\) e o ponto de interseção do arco \(BC\) com a linha do Equador, ele conterá o arco \(BC\) e será perpendicular à \(\alpha\). Logo, a projeção ortogonal do arco \(BC\) sobre o plano \(\alpha\) é um segmento de reta. Sendo assim, a projeção ortogonal do caminho traçado sobre o globo terrestre é a mostrada no item \textit{e)}.
\end{enumerate}
}{1}
\end{answer}
\clearmargin

% Página 3
\begin{answer}{Exercícios}
{\exerciselist
\begin{enumerate}\setcounter{enumi}{3}
\item Segundo o gabarito oficial divulgado pelo INEP, a resposta correta desta questão é a alternativa E que contém um octaedro facilmente construído com ripas rígidas de madeira que tenham o mesmo tamanho. Porém, a luz do que estudamos neste capítulo, percebemos que a questão está equivocada, a começar pelo enunciado ao afirmar que a obra Belvedere de Escher não pode ser reproduzida em um modelo tridimensional e também pela resposta. Tudo depende da perspectiva que se enxerga o mundo e seus objetos!

Primeiramente, veja a animação a seguir contendo uma reprodução tridimensional da obra Belvedere de Escher feito em uma impressora 3D . Dependendo da posição que se vê a obra, é possível visualizar ou não especificamente o desenho de Escher.

\notasfig{\begin{figure}[H]
\centering
\capstart

\noindent\href{https://www.umlivroaberto.org/BookCloud/Volume_1/master/view/_images/Belvedere.gif}{\includegraphics[width=250bp]{{Belvedere}.png}}
\caption{Vídeo \textit{Escher for Real The Belvedere, Waterfall, Necker Cube, Penrose Triangle 3D Printing from Technion} disponível em \url{https://www.youtube.com/watch?v=PNzi5JS46U8}.}\label{\detokenize{GE301-E:fig-projecoes-enem2007-solucao-0}}\label{\detokenize{GE301-E:id2}}\end{figure}}

Já nas opções de resposta, acreditamos que o desenho (B), conhecido como \textit{triângulo de Penrose}, pode ser reproduzido em um modelo tridimensional com ripas rígidas de madeira que tenham o mesmo tamanho. Veja a animação a seguir:

\notasfig{\begin{figure}[H]
\centering
\capstart

\noindent\href{https://www.umlivroaberto.org/BookCloud/Volume_1/master/view/_images/PenroseTriangle2.gif}{\includegraphics[width=250bp]{{PenroseTriangle2}.png}}
\caption{Fonte: \url{http://www.uff.br/cdme/}. Clique na imagem para a Versão animada}\label{\detokenize{GE301-E:fig-projecoes-enem2007-solucao-1}}\label{\detokenize{GE301-E:id3}}\end{figure}}

Um modelo concreto do triângulo de Penrose pode ser encontrado na cidade de Perth-Austrália. A escultura, produzida por Brian McKay e Ahmad Abas, possui \(13{,}5~\text{m}\) de altura e foi criada em 1997 em um concurso de revitalização de parte da cidade. Dependendo da perspectiva que se visualiza o objeto, uma nova projeção é avistada. E, existe uma posição que faz com que a projeção do objeto seja a mostrada na figura (B).
\phantomsection\label{\detokenize{GE301-E:fig-projecoes-enem2007-solucao-2}}

\notasfig{\begin{figure}[H]
\centering

\noindent\includegraphics[width=400bp]{{PenroseTriangle}.jpg}
\label{\detokenize{GE301-E:fig-projecoes-enem2007-solucao-2}}\end{figure}}

De fato, o cubo do item (A), conhecido como \textit{cubo impossível}, não pode ser criado com ripas rígidas de madeira que tenham o mesmo tamanho. Porém, apesar de ser chamado de “impossível”, ele pode sim ser reproduzido em um modelo tridimensional. Na \hyperref[\detokenize{GE301-E:fig-projecoes-enem2007-solucao-3}]{Figura \ref{\detokenize{GE301-E:fig-projecoes-enem2007-solucao-3}}}, veja que se uma das arestas do cubo tiver um buraco e dependendo da perspetiva que ele é observado, teremos a imagem do item (A) sendo sua projeção.

\notasfig{\begin{figure}[H]
\centering
\capstart

\noindent\href{https://www.umlivroaberto.org/BookCloud/Volume_1/master/view/_images/Q6_Exercicio.gif}{\includegraphics[width=250bp]{{Q6_Exercicio}.png}}
\caption{Fonte: \url{http://www.uff.br/cdme/}. Clique na imagem para a Versão animada}\label{\detokenize{GE301-E:fig-projecoes-enem2007-solucao-3}}\label{\detokenize{GE301-E:id4}}\end{figure}}

Existe ainda uma outra forma de se gerar o cubo impossível e sem que ele contenha buracos. Veja na animação abaixo.

\notasfig{\begin{figure}[H]
\centering
\capstart

\noindent\href{https://www.umlivroaberto.org/BookCloud/Volume_1/master/view/_images/Q6_Exercicio2.gif}{\includegraphics[width=200bp]{{Q6_Exercicio2}.png}}
\caption{Fonte: \url{http://www.uff.br/cdme/}. Clique na imagem para a Versão animada}\label{\detokenize{GE301-E:fig-projecoes-enem2007-solucao-4}}\label{\detokenize{GE301-E:id5}}\end{figure}}

Sugerimos que você tente desvendar os mistérios dos outros desenhos desta questão e tente descobrir se relaxando a hipótese da criação dos modelos com ripas rígidas de madeira que tenham o mesmo tamanho é possível reproduzí-los em modelos tridimensionais.
\end{enumerate}
}{1}
\end{answer}
\clearmargin

% Página 4
\begin{answer}{Exercícios}
{\exerciselist
\begin{enumerate}\setcounter{enumi}{4}
\item Alternativa \textit{c)}

Os pontos \(P, A\) e \(E\) estão sobre uma mesma reta que é perpendicular ao piso da casa, logo a projeção ortogonal de \(P, A\) e \(E\) será coincidente. Como os pontos \(A, B, C, D\) e \(E\) equidistam do corrimão, a projeção ortogonal do corrimão será um ponto equidistante das projeções ortogonais dos pontos \(A, B, C, D\) e \(E\) e portanto, a projeção ortogonal do corrimão será o centro de um círculo \(\mathcal C\) que contém as projeções ortogonais dos pontos \(A, B, C, D\) e \(E\). Além disso, os pontos \(A, B, C, D\) e \(E\) estão igualmente espaçados sobre o corrimão, portanto, como estamos interessados apenas no caminho percorrido de \(A\) até \(D\), a projeção ortogonal procurada terá \(3/4\) do círculo \(\mathcal C\). Logo, o item \textit{c)} contém a figura que melhor representa a projeção ortogonal pedida.
\end{enumerate}
}{1}
\end{answer}
\clearmargin

% Página 5
\begin{answer}{Exercícios}
{\exerciselist
\begin{enumerate}\setcounter{enumi}{5}
\item Alternativa \textit{b)}

Observe que, ao movimentar a gangorra, a trajetória dos pontos \(A\) e \(B\) estão contidas em um círculo centrado no pivô. Este círculo é perpendicular ao plano do chão, e portanto, devido a forma que o movimento é realizado, a projeção ortogonal da trajetória são dois segmentos de reta na horizontal, como os mostrados no item \textit{b)}.


\item Alternativa \textit{a)}

Primeiramente, vamos escolher um dos cubinhos na figura inicial para estudar o que acontece com ele e ao seu redor. O cubinho escolhido está marcado com uma estrelinha vermelha na figura abaixo. Note que, esse cubinho deve aparecer bem na frente caso olhássemos o bloco por trás.
\phantomsection\label{\detokenize{GE301-E:fig-projecoes-obm2014-solucao}}
\begin{figure}[H]
\centering

\noindent\includegraphics[width=150bp]{{OBM2014-fase1N1-Q20-cantomarcado}.png}
\label{\detokenize{GE301-E:fig-projecoes-obm2014-solucao}}\end{figure}

Portanto, já podemos excluir a alternativa apresentada em \textit{c)}, pois este cubinho foi pintado de preto na figura. Além disso, observe na figura acima que o cubinho marcado possui dois cubinhos adjacentes de cores diferentes. Isso não acontece na opção \textit{e)}, que também deve ser excluída.

Pelo enunciado, o bloco retangular possui seis cubinhos pretos e seis cubinhos brancos, mas pela figura conseguimos ver 6 cubinhos brancos e 4 cubinhos pretos. Isto quer dizer que os dois cubinhos que não estão visíveis na figura inicial, abaixo do cubo marcado com a estrelinha e embaixo do preto ao seu lado, são da cor preta. Assim, as possibilidade \textit{b)} e \textit{d)} também não são compatíveis com a figura inicial. Portanto, a opção apresentada no item \textit{a)} é a correta.
\end{enumerate}
}{1}
\end{answer}
\clearmargin

% Página 6
\begin{answer}{Exercícios}
{\exerciselist
\begin{enumerate}\setcounter{enumi}{7}
\item Alternativa \textit{e)}

Observe que apenas o item E mostra o anel mais da direita à frente do anel da esquerda e por trás do anel superior. Portanto, a esboço que melhor representa os anéis de Borromeo é o do item \textit{e)}.

\item Alternativa \textit{a)}

Vamos fazer uma contagem de baixo para cima de quantos blocos estão faltando em cada nível. O primeiro nível está totalmente completo. O segundo nível precisa de um bloco para ser completado, assim como o terceiro que possui exatamente a mesma configuração. Para completar o quarto nível, será preciso utilizar três blocos, assim como no quinto nível. Finalmente, no sexto nível são necessários quatro blocos para que ele fique completo. Portanto, no total utilizamos 12 blocos para contruir o paralelogramo pedido. Logo, a opção correta é a \textit{a)}.

É importante observar que a construção do bloco poderia ser pensada de outras formas que não a sugerida acima. Como o volume do cubo inicial e da parte que falta ser completada são fixados, o número de blocos utilizados na solução será único independente da configuração.
\end{enumerate}
}{1}
\end{answer}
\clearmargin

% Página 7
\begin{answer}{Exercícios}
{\exerciselist
\begin{enumerate}\setcounter{enumi}{9}
\item Para resolver esta questão precisamos observar as fatias de tamanho \(3 \times 3 \times 1\), pois seu total de cubinhos será o mesmo independente de como a contagem for feita. Por exemplo, se considerarmos a fatia de tamanho \(3 \times 3 \times 1\) marcada com \(1, 3, a, 2, 2\) e \(1\)), é possível observar que:
\begin{equation*}
\begin{split}1+3+a=2+2+1 \Longleftrightarrow a = 1.\end{split}
\end{equation*}
Observando diferentes fatias obtemos as seguintes equações diretamente:
\begin{align*}
1+2+1=m+2+2 &\Longleftrightarrow m=0,\\
2+x+2=3+3+1 &\Longleftrightarrow x=3,\\
a+1+d=1+3+m &\Longleftrightarrow d=2.\\
\end{align*}
Repare que na fatia que possui as marcações \(c, b, 1, 2, x, 2\) temos:
\begin{equation*}
\begin{split}c+b+1=2+x+2 \Longleftrightarrow c+b=6.\end{split}
\end{equation*}
Como os valores de \(b\) e \(c\) podem ser no máximo 6, então concluímos que \(b=c=3\). Seguindo então com o mesmo raciocínio, encontramos:
\begin{align*}
1+c+f=2+1+2 &\Longleftrightarrow f = 1.\\
f+e+d=2+2+1 &\Longleftrightarrow e=2.
\end{align*}
Concluímos então que: \(a=1\), \(b=c=3\), \(d=e=2\), \(f=1\), \(x=3\) e \(m=0\).

\item Alternativa \textit{c)}

Para responder esta questão, é preciso analisar quantos cubinhos foram empilhados em cada posição. Observe o quadradinho central da vista de cima da alternativa \textit{c)}, que aponta que há cubos empilhados nesta posição do bloco. Este quadradinho está mais fácil de analisar, pois não possui quadradinho acima, abaixo, à direita e à esquerda. Note que há apenas um quadradinho no meio da vista de frente, então, segundo a alternatica \textit{c)}, na posição central do bloco haverá um cubo. Agora, há dois quadradinhos sendo vistos no meio da vista da esquerda, então, segundo a alternatica \textit{c)}, na posição central do bloco haverá dois cubos. Isto é impossível e a alternativa correta é a \textit{c)}.

Na figura abaixo, os números dentro de cada quadradinho indicam possíveis quantidades de cubos empilhados para construir o bloco nas alternativas \textit{a)}, \textit{b)}, \textit{d)} e \textit{e)}. Essas alternativas trazem opções que estão de acordo com as vistas da esquerda e da frente.

\notasfig{\begin{figure}[H]
\centering

\noindent\includegraphics[width=.9\linewidth]{{OBM2009-fase1N1-Q20-solucao_1}.png}
\label{\detokenize{GE301-E:fig-projecoes-obm2009q20-solucao}}\end{figure}}
\end{enumerate}
}{1}
\end{answer}
\clearmargin

% Página 8
\begin{answer}{Exercícios}
{\exerciselist
\begin{enumerate}\setcounter{enumi}{11}
\item Alternativa \textit{b)}

Na figura abaixo, os quadradinhos que estiveram em contato com o bloco na sua posição inicial foram marcados com um círculo vermelho e os números contidos no interior dos quadradinhos representam em qual giro tal quadradinho foi tocado pelo bloco. Por esta figura, concluímos que 19 quadradinhos foram tocados pelo bloco. Logo, a respota correta é a alternatica \textit{b)}.
\phantomsection\label{\detokenize{GE301-E:fig-projecoes-obm2005-solucao}}
\begin{figure}[H]
\centering

\noindent\includegraphics[width=200bp]{{OBM2005_Q25_solucao}.png}
\label{\detokenize{GE301-E:fig-projecoes-obm2005-solucao}}\end{figure}

\item Alternativa \textit{c)}

Note que as figuras I e III representam o mesmo objeto em posições diferentes, mas não representam o objeto dado (este é o espelhamento do objeto dado em relação a coluna de dois cubinhos). A figura II representa o objeto dado visto por trás. Já a figura IV também representa o objeto dado, após duas rotações consecutivas. Portanto, a alternativa correta é a \textit{c)}.
\end{enumerate}
}{1}
\end{answer}
\clearmargin

% Página 9
\begin{answer}{Exercícios}
{\exerciselist
\begin{enumerate}\setcounter{enumi}{13}
\item Alternativa \textit{b)}

É claro que uma pessoa posicionada do lado de fora da loja conseguirá ver todos os cubos que estão encostados no vidro da loja, ou seja, a pessoa já consegue ver \(8\) cubos. Na segunda linha de cubinhos (contadas do vidro para dentro da loja), a pessoa consegue visualizar mais 4 cubos  que estão marcados com bolinhas azuis na figura abaixo. Com as marcações feitas em vermelho nesta figura, percebemos que a aresta \(PQ\) impede a vista dos cubos marcados com X, além de possibilitar que afirmemos que a pessoa consegue visualizar os cubos marcados com uma bolinha vermelha. Repare que não é pssível que a pessoa visualize esses cubos marcados com X por sua parte de cima, devido ao seu tamanho.
\phantomsection\label{\detokenize{GE301-E:fig-projecoes-obm2004-solucao}}
\begin{figure}[H]
\centering

\noindent\includegraphics[width=200bp]{{OBM2004_N1F3_Q19_solucao}.png}
\label{\detokenize{GE301-E:fig-projecoes-obm2004-solucao}}\end{figure}

\item Alternativa \textit{b)}

Na opção \textit{a)}, o círculo preto está acima do círculo branco, então a estrelinha deveria estar à direita do círculo branco, mas não está. Logo \textit{a)} não é a resposta correta, assim como \textit{e)}, que contém o mesmo cubo de \textit{a)}. Em \textit{c)}, o círculo branco foi colocado acima da estrelinha, logo o círculo preto deveria estar à direita da estrelinha, mas não está. Portanto, \textit{c)} também não pode ser a resposta, assim como \textit{d)} que contém o mesmo cubo de \textit{c)}. Na alternativa \textit{b)}, o círculo preto está abaixo do círculo branco e à direita da estrelinha, exatamente como foi desenhado na cartolina. Portanto, a resposta correta é a \textit{b)}.
\end{enumerate}
}{1}
\end{answer}
\clearmargin


% Página 10
\begin{answer}{Exercícios}
{\exerciselist
\begin{enumerate}\setcounter{enumi}{15}
\item Alternativa \textit{a)}

Observe que a face entre o cubo de face azul (já contém O e P) e o cubo de face vermelha (já contém O, Q e S) deve conter a letra T. Logo, a face oposta à letra T contém a letra S.

\item Alternativa A

Sobrepomos a figura 1 com a figura 2 (agora pintada de vermelho) e as quatro possibilidades estão mostradas na figura abaixo.
\phantomsection\label{\detokenize{GE301-E:fig-projecoes-obmep2015-solucao}}
\begin{figure}[H]
\centering

\noindent\includegraphics[width=350bp]{{OBMEP2015_F1N2_Q3_solucao}.png}
\label{\detokenize{GE301-E:fig-projecoes-obmep2015-solucao}}\end{figure}

Dentre as possibilidades, a única que possui uma parte cinza (parte que ficou descoberta depois de sobrepormos a figura 1 com a figura 2) disponível entre as alternativas dada é a primeira. Neste caso, a solução correta é a alternativa \textit{a)}.
\end{enumerate}
}{1}
\end{answer}
\clearmargin

% Página 11
\begin{answer}{Exercícios}
{\exerciselist
\begin{enumerate}\setcounter{enumi}{17}
\item Alternativa E

A alternativa correta é a E, pois uma circunferência contida em um plano perpendicular ao plano de projeção é um segmento de reta.

\item \begin{enumerate}
\item {} 
O plano \(\psi\) é excluído do domínio da função \(f\), pois as retas que passam por \(O\) e por pontos de \(\psi\) estão contidas em \(\psi\) e portanto, são paralelas ao plano \(\pi\). Como tais retas não intersectam o plano \(\pi\), os pontos do plano \(\psi\) não possuem projeções em perspectiva com relação ao centro \(O\) sobre o plano de projeção \(\pi\). Desta forma, os pontos de \(\psi\) não podem estar no domínio de \(f\).

\item {} 
Existem infinitos pontos \(X\) tais que \(f(X)=Y\) para um dado \(Y\). Na verdade, todos os pontos sobre a reta \(OY\) são projetados em \(Y\). Se desenharmos todos os pontos com a propriedade pedida, teríamos uma reta, como dissemos anteriormente.

\item {} 
Verdadeiro. Dado um ponto \(P\) pertencente à \({\mathbb R}^{3} - \psi\), temos que \(f(P)\) é o ponto de interseção da reta que passa por \(O\) e \(P\). Como \(f(P)\) está no domínio da função \(f\), ele pode ser projetado novamente. Para isso, note que a reta que passa por \(O\) e por \(f(P)\) intersecta o plano \(\pi\) em \(f(P)\). Logo, \(f(f(P)) = f(P)\) para todo ponto \(P\) do domínio de \(f\).

\item {} 
Sim. A função que modela a projeção paralela, a função identidade: \(f(X)=X\) para todo \(X\) pertencente ao domínio de \(f\),  e a função constante: \(f(X)=N\) para todo \(X\) pertencente ao domínio de \(f\) e \(N\) um número real qualquer fixado.

\item {} 
Falso. Como já dissemos no item (b), todos os pontos de uma reta que passa por \(O\) e por um ponto de \({\mathbb R}^{3} - \psi\) são projetados no mesmo ponto. Dessa forma, se \(f(P)=f(Q)\) isto não significa que \(P=Q\).

\item {} 
Se o centro \(O\) de projeção está contido no plano de projeção \(\pi\), então as retas que passam por pontos \(P\) de \({\mathbb R}^{3} - \psi\) e por \(O\) intersectam \(\pi\) em \(O\). Logo, \(f(P)=O\) para todo \(P\in{\mathbb R}^{3} - \psi\).

\end{enumerate}
\end{enumerate}
}{1}
\end{answer}
\begin{answer}{Exercícios}
{\exerciselist
\begin{enumerate}\setcounter{enumi}{19}
\item \begin{enumerate}
\item {} 
Repare que se a direção de projeção \(d\) é paralela ao plano de projeção \(\pi\), então a reta paralela à \(d\) que passa por um ponto \(P\) do domínio de \(f\) também será paralela ao plano \(\pi\) e com isso não intersectará o plano \(\pi\). Sendo assim, não existiria projeção paralela do ponto \(P\), ou seja, teríamos um ponto no domínio da função que não teria um ponto no contradomínio, o que não é possível. Portanto, a direção de projeção não pode ser paralela ao plano de projeção.

\item {} 
Dado um ponto \(Y\in\pi\), podemos observar que \(Y\) será a projeção paralela de todos os pontos \(X\) pertencentes à reta que passa por \(Y\) e é paralela a \(d\). Se desenharmos todos os pontos com a propriedade pedida, teríamos uma reta, como dissemos anteriormente.

\item {} 
Verdadeiro. Dado um ponto \(P\) pertencente à \({\mathbb R}^{3}\), temos que \(f(P)\) é o ponto de interseção da reta que passa por \(P\) e é paralela à \(d\). O ponto \(f(P)\) é um ponto de \({\mathbb R}^{3}\) e então, está no domínio da função \(f\). Além disso, o ponto \(f(P)\) é também o ponto de interseção da reta que passa por \(f(P)\) e é paralela à \(d\), logo \(f(f(P)) = f(P)\) para todo ponto \(P\) do domínio de \(f\).

\item {} 
Sim. A função que modela a projeção perspectiva vista no exercício anterior, a função identidade: \(f(X)=X\) para todo \(X\) pertencente ao domínio de \(f\),  e a função constante: \(f(X)=N\) para todo \(X\) pertencente ao domínio de \(f\) e \(N\) um número real qualquer fixado.

\item {} 
Falso. Como já dissemos no item (b), todos os pontos de uma reta paralela à direção de projeção \(d\) são projetados no mesmo ponto. Dessa forma, se \(f(P)=f(Q)\) isto não significa que \(P=Q\).

\end{enumerate}
\end{enumerate}
}{0}
\end{answer}
\clearmargin

% Página 12
\begin{objectives}{Exercício 21}
{
Este exercício tem por objetivo relacionar as projeções em perspectiva com funções quadráticas, no sentido que a diferença \(w_{n} = g(n + 1) - g(n)\) entre as alturas de dois segmentos projetados consecutivos é inversamente proporcional a uma função quadrática em \(n\).
}{1}{2}
\end{objectives}
\begin{answer}{Exercícios}
{\exerciselist
\begin{enumerate}\setcounter{enumi}{20}
\item \begin{enumerate}
\item {} 
Vamos usar a mesma ideia usada na Etapa 1 da PARTE 2 da \hyperref[\detokenize{GE301-5:ativ-proj-comprimentos}]{atividade Comprimentos em projeções}.

\begin{figure}[H]
\centering
\capstart

\noindent\includegraphics[width=300bp]{{ladrilhamento_projperspectiva_1}.png}
\caption{Triângulos semelhantes \(OUP_n\) e \(P_n'VP_n\).}\label{\detokenize{GE301-E:fig-proj-perspectiva-ladrilhamentoinfinito-ex-sol}}\label{\detokenize{GE301-E:id8}}\end{figure}

Na \hyperref[\detokenize{GE301-E:fig-proj-perspectiva-ladrilhamentoinfinito-ex-sol}]{Figura \ref{\detokenize{GE301-E:fig-proj-perspectiva-ladrilhamentoinfinito-ex-sol}}} vemos um corte da \hyperref[\detokenize{GE301-E:fig-proj-perspectiva-ladrilhamentos-04-ex}]{Figura \ref{\detokenize{GE301-E:fig-proj-perspectiva-ladrilhamentos-04-ex}}}, onde \(P_{n}\) é o ponto médio do segmento \(R_{n}S_{n}\) e \(P_{n}'\) sua projeção sobre o plano \(\pi\). Assim, os triângulos \(OUP_n\) e \(P_n'VP_n\) são semelhantes e portanto:
\begin{equation*}
\begin{split}\frac{OU}{UP_n} = \frac{P_n'V}{VP_n} \Longleftrightarrow \frac{3}{5+n} = \frac{y_n}{n} \Longleftrightarrow y_n=g(n)=\frac{3n}{5+n}.\end{split}
\end{equation*}
Assim, se o valor de \(n\) fica arbritariamente grande, os valores de \(3n\) e \(5+n\) também ficam arbritariamente grandes, mas \(3n\) cresce mais rápido que \(5+n\). Logo, se o valor de \(n\) aumentar muito, o mesmo acontecerá com o valor de \(y_n\).

\item {} 
Se \(w_n=y_{n+1}-y_n\), então

\end{enumerate}
\begin{equation*}
\begin{split}w_n = \frac{3(n+1)}{5+(n+1)} - \frac{3n}{5+n} = \frac{15}{(n+5)(n+6)}.\end{split}
\end{equation*}
Logo, quando \(n\) fica arbritariamente grande, o denominador \((5+n)(6+n)\) também fica grande. Daí, \(\frac{15}{(5+n)(6+n)}\) tende a ficar arbritariamente pequeno. Este fato comprova o que já havíamos percebido na \hyperref[\detokenize{GE301-E:fig-proj-perspectiva-ladrilhamentos-04-ex}]{Figura \ref{\detokenize{GE301-E:fig-proj-perspectiva-ladrilhamentos-04-ex}}}.
\end{enumerate}

}{1}
\end{answer}
\exercise

\begin{enumerate}
\item {[}Texto e figuras retirados do site da OBM{]}(OBM2008 - Questão 6 da 1a fase-nível 1) Sobre uma mesa retangular de uma sala foram colocados quatro sólidos, mostrados no desenho. Uma câmera no teto da sala, bem acima da mesa, fotografou o conjunto.

\begin{figure}[H]
\centering

\noindent\includegraphics[width=224bp]{{OBM2008-fase1N1-Q6}.png}
\label{\detokenize{GE301-E:fig-projecoes-obm2008q6}}\end{figure}

Qual dos esboços a seguir representa melhor essa fotografia?
\begin{figure}[H]
\centering

\noindent\includegraphics[width=400bp]{{OBM2008-fase1N1-Q6-respostas-editadas}.png}
\label{\detokenize{GE301-E:fig-projecoes-obm2008q6-respostas}}\end{figure}

\item {[}Texto e figuras retirados da internet{]}(ENEM2009 - questão 69) A fotografia mostra uma turista aparentemente beijando a esfinge de Gizé, no Egito. A figura a seguir mostra como, na verdade, foram posicionadas a câmera fotográfica, a turista e a esfinge.

\begin{figure}[H]
\centering
\capstart

\noindent\includegraphics[width=275bp]{{QAnulada_ENEM2009}.png}
\caption{Questão 69 do ENEM 2009.}\label{\detokenize{GE301-E:fig-projecoes-enem2009-q69}}\label{\detokenize{GE301-E:id1}}\end{figure}

Medindo-se com uma régua diretamente na fotografia, verifica-se que a medida do queixo até o alto da cabeça da turista é igual a \(\frac23\) da medida do queixo da esfinge até o alto da sua cabeça. Considere que essas medidas na realidade são representadas por \(d\) e \(d’\), respectivamente, que a distância da esfinge à lente da câmera fotográfica, localizada no plano horizontal do queixo da turista e da esfinge, é representada por \(b\), e que a distância da turista à mesma lente, por \(a\). A razão entre \(b\) e \(a\) será dada por:

\begin{tasks}(2)
\task \(\dfrac{b}{a}=\dfrac{d'}{c}\)

\task \(\dfrac{b}{a}=\dfrac{2d}{3c}\)

\task \(\dfrac{b}{a}=\dfrac{3d'}{2c}\)

\task \(\dfrac{b}{a}=\dfrac{2d'}{3c}\)

\task \(\dfrac{b}{a}=\dfrac{2d’}{c}\)
\end{tasks}


\item {[}Texto e figuras retirados do site do INEP{]}(ENEM2016 - Questão 178 caderno azul) A figura representa o globo terrestre e nela estão marcados os pontos \(A, B\) e \(C\). Os pontos \(A\) e \(B\) estão localizados sobre um mesmo paralelo, e os pontos \(B\) e \(C\), sobre um mesmo meridiano. É traçado um caminho do ponto \(A\) até \(C\), pela superfície do globo, passando por \(B\), de forma que o trecho de \(A\) até \(B\) se dê sobre o paralelo que passa por \(A\) e \(B\) e, o trecho de \(B\) até \(C\) se dê sobre o meridiano que passa por \(B\) e \(C\). Considere que o plano \(\alpha\) é paralelo à linha do equador na figura.

\begin{figure}[H]
\centering

\noindent\includegraphics[width=200bp]{{Q178_ENEM2016_1}.png}
\label{\detokenize{GE301-E:fig-projecoes-enem2016q178}}\end{figure}

A projeção ortogonal, no plano \(\alpha\), do caminho traçado no globo terrestre pode ser representada por:

\begin{tasks}(3)
\task \adjustbox{valign=t}
{
\parbox[125pt]{125pt}{\vfill\centering\includegraphics[width=.65\linewidth]{{ENEM-2016-Q178-respoaseditadas-versao2a}.png}\vfill}
}

\task \adjustbox{valign=t}
{
\parbox[125pt]{125pt}{\vfill\centering\includegraphics[width=.8\linewidth]{{ENEM-2016-Q178-respoaseditadas-versao2b}.png}\vfill}
}

\task \adjustbox{valign=t}
{
\parbox[125pt]{125pt}{\vfill\centering\includegraphics[width=.8\linewidth]{{ENEM-2016-Q178-respoaseditadas-versao2c}.png}\vfill}
}

\task \adjustbox{valign=t}
{
\parbox[125pt]{125pt}{\vfill\centering\includegraphics[width=.8\linewidth]{{ENEM-2016-Q178-respoaseditadas-versao2d}.png}\vfill}
}

\task \adjustbox{valign=t}
{
\parbox[125pt]{125pt}{\vfill\centering\includegraphics[width=.8\linewidth]{{ENEM-2016-Q178-respoaseditadas-versao2e}.png}\vfill}
}
\end{tasks}

% \begin{figure}[H]
% \centering

% \noindent\includegraphics[width=135bp]{{ENEM-2016-Q178-respoaseditadas-versao2}.png}
% \end{figure}
\clearpage
\item {[}Texto e figuras retiradas do site do INEP{]}(ENEM2007 - Questão 5) Representar objetos tridimensionais em uma folha de papel nem sempre é tarefa fácil. O artista holandês Escher (1898-1972) explorou essa dificuldade criando várias figuras planas impossíveis de serem construídas como objetos tridimensionais, a exemplo da litografia Belvedere, reproduzida ao lado. Considere que um marceneiro tenha encontrado algumas figuras supostamente desenhadas por Escher e deseje construir uma delas com ripas rígidas de madeira que tenham o mesmo tamanho.

\begin{figure}[H]
\centering

\noindent\includegraphics[width=150bp]{{Q5_ENEM2007}.png}
\label{\detokenize{GE301-E:fig-projecoes-enem2007q5}}\end{figure}

Qual dos desenhos a seguir ele poderia reproduzir em um modelo tridimensional real?

\begin{figure}[H]
\centering

\noindent\includegraphics[width=250bp]{{ENEM_2007_Q5_versao2}.png}
\label{\detokenize{GE301-E:fig-projecoes-enem2007q5-respostas}}\end{figure}

\item {[}Texto e figuras retirados do site do INEP{]}(ENEM2014 - Questão 160 caderno azul) O acesso entre os dois andares de uma casa é feito através de uma escada circular (escada caracol), representada na figura. Os cinco pontos \(A, B, C, D, E\) sobre o corrimão estão igualmente espaçados, e os pontos \(P, A\) e \(E\) estão em uma mesma reta. Nessa escada, uma pessoa caminha deslizando a mão sobre o corrimão do ponto \(A\) até o ponto \(D\).

\begin{figure}[H]
\centering

\noindent\includegraphics[width=175bp]{{Q160_ENEM2014}.png}
\label{\detokenize{GE301-E:fig-projecoes-enem2014q160}}\end{figure}

A figura que melhor representa a projeção ortogonal sobre o piso da casa (plano), do caminho percorrida pela mão dessa pessoa é:

\begin{tasks}(3)
\task \adjustbox{valign=t}
{
\includegraphics[width=.65\linewidth]{{ENEM-2014-Q160-respostaseditadas-versao2a}.png}
}

\task \adjustbox{valign=t}
{
\includegraphics[width=.65\linewidth]{{ENEM-2014-Q160-respostaseditadas-versao2b}}
}

\task \adjustbox{valign=t}
{
\includegraphics[width=.65\linewidth]{{ENEM-2014-Q160-respostaseditadas-versao2c}}
}

\task \adjustbox{valign=t}
{
\includegraphics[width=.65\linewidth]{{ENEM-2014-Q160-respostaseditadas-versao2d}}
}

\task \adjustbox{valign=t}
{
\includegraphics[width=.65\linewidth]{{ENEM-2014-Q160-respostaseditadas-versao2e}}
}
\end{tasks}

% \begin{figure}[H]
% \centering

% \noindent\includegraphics[width=110bp]{{ENEM-2014-Q160-respostaseditadas-versao2}.png}
% \label{\detokenize{GE301-E:fig-projecoes-enem2014q160-respostas}}\end{figure}

\item {[}Texto e figuras retirados do site do INEP{]}(ENEM2013 - Questão 180 caderno azul) Gangorra é um brinquedo que consiste de uma tábua longa e estreita equilibrada e fixada no seu ponto central (pivô). Nesse brinquedo, duas pessoas sentam-se nas extremidade e, alternadamente, impulsionam-se para cima, fazendo descer a extremidade oposta, realizado, assim, o movimento da gangorra. Considere a gangorra representada na figura, em que os pontos \(A\) e \(B\) são equidistantes do pivô:

\begin{figure}[H]
\centering

\noindent\includegraphics[width=200bp]{{Q180_ENEM2013}.png}
\label{\detokenize{GE301-E:fig-projecoes-enem2013q180}}\end{figure}

A projeção ortogonal da trajetória dos pontos \(A\) e \(B\), sobre o plano do chão da gangorra, quando esta se encontra em movimento, é:

\begin{tasks}(3)
\task \adjustbox{valign=t}
{
\includegraphics[width=.8\linewidth]{{ENEM-2013-Q180-respostaseditadas-versao2a}.png}
}

\task \adjustbox{valign=t}
{
\includegraphics[width=.8\linewidth]{{ENEM-2013-Q180-respostaseditadas-versao2b}.png}
}

\task \adjustbox{valign=t}
{
\includegraphics[width=.8\linewidth]{{ENEM-2013-Q180-respostaseditadas-versao2c}.png}
}

\task \adjustbox{valign=t}
{
\includegraphics[width=.8\linewidth]{{ENEM-2013-Q180-respostaseditadas-versao2d}.png}
}

\task \adjustbox{valign=t}
{
\includegraphics[width=.8\linewidth]{{ENEM-2013-Q180-respostaseditadas-versao2e}.png}
}
\end{tasks}

% \begin{figure}[H]
% \centering

% \noindent\includegraphics[width=140bp]{{ENEM-2013-Q180-respostaseditadas-versao2}.png}
% \label{\detokenize{GE301-E:fig-projecoes-enem2013q180-respostas}}\end{figure}

\item {[}Texto e figuras retirados do site da OBM{]}(OBM2014 - Questão 20 da 1a fase-nível 1) A figura abaixo mostra um bloco retangular montado com seis cubinhos pretos e seis cubinhos brancos, todos de mesmo tamanho.

\begin{figure}[H]
\centering

\noindent\includegraphics[width=150bp]{{OBM2014-fase1N1-Q20}.png}
\label{\detokenize{GE301-E:fig-projecoes-obm2014q20}}\end{figure}

Qual das figuras abaixo mostra o mesmo bloco visto por trás?

\begin{figure}[H]
\centering

\noindent\includegraphics[width=\linewidth]{{OBM2014-fase1N1-Q20-respostas-editadas}.png}
\label{\detokenize{GE301-E:fig-projecoes-obm2014q20-respostas}}\end{figure}

OBS.: Esta questão também está presente no nível 2.

\item {[}Texto e figuras retirados do site do INEP{]}(ENEM2009 - Questão 149 caderno azul) Em Florença, Itália, na Igreja de Santa Croce, é possível encontrar um portão em que aparecem os anéis de Borromeo. Alguns historiadores acreditavam que os círculos representavam as três artes: escultura, pintura e arquitetura, pois elas eram tão próximas quanto inseparáveis.

\begin{figure}[H]
\centering

\noindent\includegraphics[width=150bp]{{Q149_ENEM2009}.png}
\label{\detokenize{GE301-E:fig-projecoes-enem2009q149}}\end{figure}

Qual dos esboços a seguir melhor representa os anéis de Borromeo?

\begin{figure}[H]
\centering

\noindent\includegraphics[width=280bp]{{ENEM_2009_Q149_editadas}.png}
\label{\detokenize{GE301-E:fig-projecoes-enem2009q149-respostas}}\end{figure}

\item {[}Texto e figuras retirados do site da OBM{]}(OBM2013 - Questão 4 da 1a fase-nível 1) Esmeralda está construindo um paralelepípedo usando blocos menores iguais.

\begin{figure}[H]
\centering

\noindent\includegraphics[width=150bp]{{OBM2013-fase1N1}.png}
\label{\detokenize{GE301-E:fig-projecoes-obm2013q4}}\end{figure}

Para terminar sua tarefa, quantos blocos Esmeralda ainda deve colocar?

\begin{enumerate}
\begin{multicols}{5}
\item 12

\item 14

\item 16

\item 18

\item 20
\end{multicols}
\end{enumerate}

OBS.: Esta questão também está presente no nível 2.

\item {[}Texto e figuras retirados do site da OBM{]}(OBM2010 - Questão3 da 3a fase-nível 1) Dado um sólido formado por cubos de 1 cm de aresta, como mostra a figura abaixo da esquerda, podemos indicar a quantidade de cubos em cada direção, como mostra a figura abaixo da direita.

\begin{figure}[H]
\centering

\noindent\includegraphics[width=275bp]{{OBM2010-fase3N1-Q3}.png}
\label{\detokenize{GE301-E:fig-projecoes-obm2010q3}}\end{figure}

Esmeraldino montou um sólido com cubos de 1 cm de aresta e fez uma figura similar acima. Encontre os valores de \(a, b, c, d, e, f, x\) e \(m\).

\begin{figure}[H]
\centering

\noindent\includegraphics[width=130bp]{{OBM2010-fase3N1-Q3_2}.png}
\label{\detokenize{GE301-E:fig-projecoes-obm2010q3-2}}\end{figure}

\item {[}Texto e figuras retirados do site da OBM{]}(OBM2009 - Questão 20 da 1a fase-nível 1) Alguns cubos foram empilhados formando um bloco. As figuras abaixo representam a vista da esquerda e da frente desse bloco.

\begin{figure}[H]
\centering

\noindent\includegraphics[width=225bp]{{OBM2009-fase1N1-Q20}.png}
\label{\detokenize{GE301-E:fig-projecoes-obm2009q20}}\end{figure}

Olhando o bloco de cima, qual das figuras a seguir não pode ser vista?

\begin{figure}[H]
\centering

\noindent\includegraphics[width=450bp]{{OBM2009-fase1N1-Q20-respostas-editadas}.png}
\label{\detokenize{GE301-E:fig-projecoes-obm2009q20-respostas}}\end{figure}

\item {[}Texto e figuras retirados do site da OBM{]}(OBM2005 - Questão 25 da 1a fase-nível 2) Um bloco de dimensões \(1\times 2 \times 3\) é colocado sobre um tabuleiro \(8\times 8\), como mostra a figura, com a face X, de dimensões \(1\times 2\), virada para baixo. Giramos o bloco em torno de uma de suas arestas de modo que a face Y fique virada para baixo. Em seguida, giramos novamente o bloco, mas desta vez de modo que a face Z fique virada para baixo. Giramos o bloco mais três vezes, fazendo com que as faces X, Y e Z fiquem viradas para baixo, nessa ordem. Quantos quadradinhos diferentes do tabuleiro estiveram em contato com o bloco?

\begin{figure}[H]
\centering

\noindent\includegraphics[width=220bp]{{OBM2005-fase1N2-Q25}.png}
\label{\detokenize{GE301-E:fig-projecoes-obm2005q25}}\end{figure}

\begin{enumerate}
\begin{multicols}{5}
\item 18

\item 19

\item 20

\item 21

\item 22
\end{multicols}
\end{enumerate}

\item {[}Texto e figuras retirados do site da OBM{]}(OBM2004 - Questão 24 da 1a fase-nível 1) Observe a figura:

\begin{figure}[H]
\centering

\noindent\includegraphics[width=150bp]{{OBM2004-fase1N1-Q24}.png}
\label{\detokenize{GE301-E:fig-projecoes-obm2004q24}}\end{figure}

Duas das figuras abaixo representam o objeto acima colocado em outras posições.

\begin{figure}[H]
\centering

\noindent\includegraphics[width=400bp]{{OBM2004-fase1N1-Q24-respostas}.png}
\label{\detokenize{GE301-E:fig-projecoes-obm2004q24-respostas}}\end{figure}

Elas são:

\begin{enumerate}
\begin{multicols}{5}
\item I e II

\item I e IV

\item II e IV

\item I e III

\item II e III
\end{multicols}
\end{enumerate}

\item {[}Texto e figuras retirados do site da OBM{]}(OBM2004 - Questão 19 da 1a fase-nível 3) Dono de uma loja empilhou vários blocos medindo \(0{,}8~\text{m} \times 0{,}8~\text{m} \times 0{,}8~\text{m}\) no canto da loja e encostados numa parede de vidro que dá para a rua, conforme mostra a figura abaixo.

\begin{figure}[H]
\centering

\noindent\includegraphics[width=175bp]{{OBM2004-fase1N3-Q19}.png}
\label{\detokenize{GE301-E:fig-projecoes-obm2004q219}}\end{figure}

Quantos blocos no máximo, uma pessoa de \(1{,}80~\text{m}\) de altura que está do lado de fora da loja pode enxergar?

Obs. Consideramos que uma pessoa pode enxergar uma caixa se consegue ver uma pequena região de área positiva de sua superfície.

\begin{enumerate}
\begin{multicols}{5}
\item 13

\item 14

\item 15

\item 16

\item 17
\end{multicols}
\end{enumerate}

\item {[}Texto e figuras retirados do site da OBM{]}(OBM2000 - Questão 20 da 1a fase-nível 1) A figura abaixo foi desenhada em cartolina e dobrada de modo a formar um cubo.

\begin{figure}[H]
\centering

\noindent\includegraphics[width=165bp]{{OBM2000-fase1N1-Q20}.png}
\label{\detokenize{GE301-E:fig-projecoes-obm2000q20}}\end{figure}

Qual das alternativas mostra o cubo assim formado?

\begin{figure}[H]
\centering

\noindent\includegraphics[width=450bp]{{OBM2000-fase1N1-Q20-respostaseditadas}.png}
\label{\detokenize{GE301-E:fig-projecoes-obm2008-0q20-respostas}}\end{figure}

\clearpage
\item {[}Texto retirados do site da OBMEP{]}(OBMEP2017 - Questão 4 da 1a fase-nível 3)  Zequinha tem três dados iguais, com letras O, P, Q, R, S e T em suas faces. Ele juntou esses dados como na figura, de modo que as faces em contato tivessem a mesma letra. Qual é a letra na face oposta à que tem a letra T?

\begin{figure}[H]
\centering

\noindent\includegraphics[width=150bp]{{OBMEP2017-fase1N3-Q4}.png}
\label{\detokenize{GE301-E:fig-projecoes-obmep2017q4}}\end{figure}

\begin{enumerate}
\begin{multicols}{5}
\item S

\item R

\item Q

\item P

\item O
\end{multicols}
\end{enumerate}

\item {[}Texto retirado do site da OBMEP{]}(OBMEP2015 - Questão 3 da 1a fase-nível 2) A peça da Figura 1 foi montada juntando-se duas peças, sem sobreposição.

\begin{figure}[H]
\centering
\capstart

\noindent\includegraphics[height=90bp]{{OBMEP2015_F1N2_Q3_1_1}.png}
\caption{Figura 1.}\label{\detokenize{GE301-E:fig-projecoes-obmep2015q3}}\label{\detokenize{GE301-E:id6}}\end{figure}

Uma das peças utilizadas foi a da Figura 2.

\begin{figure}[H]
\centering
\capstart

\noindent\includegraphics[height=90bp]{{OBMEP2015_F1N2_Q3_2_1}.png}
\caption{Figura 2.}\label{\detokenize{GE301-E:fig-projecoes-obmep2015q3-2}}\label{\detokenize{GE301-E:id7}}\end{figure}

Qual foi a outra peça utilizada?

\begin{figure}[H]
\centering

\noindent\includegraphics[height=100bp]{{OBMEP2015_F1N2_Q3_respostas_1}.png}
\label{\detokenize{GE301-E:fig-projecoes-obmep2015q3-respostas}}\end{figure}

\item {[}UFSCar-2001{]} Considere um plano \(\alpha\) e um ponto \(P\) qualquer do espaço. Se por \(P\) traçarmos a reta perpendicular a \(\alpha\), a intersecção dessa reta com \(\alpha\) é um ponto chamado projeção ortogonal do ponto \(P\) sobre \(\alpha\). No caso de uma figura \(F\) do espaço, a projeção ortogonal de \(F\) sobre \(\alpha\) é definida pelo conjunto das projeções ortogonais de seus pontos.

Com relação a um plano \(\alpha\) qualquer fixado, pode-se dizer que:

\begin{enumerate}
\item a projeção ortogonal de um segmento de reta pode resultar numa semirreta.

\item a projeção ortogonal de uma reta sempre resulta numa reta.

\item a projeção ortogonal de uma parábola pode resultar num segmento de reta.

\item a projeção ortogonal de um triângulo pode resultar num quadrilátero.

\item a projeção ortogonal de uma circunferência pode resultar num segmento de reta.
\end{enumerate}

\item Vimos que uma projeção em perspectiva com relação a um centro \(O\) em um plano de projeção \(\pi\) pode ser considerada como uma função \(f\) de domínio \({\mathbb R}^{3} - \psi\) e contradomínio \(\pi\), com \(\psi\) o plano paralelo a \(\pi\) que passa por \(O\).
\begin{enumerate}
\item {} 
Por que ao se modelar uma projeção em perspectiva por meio de uma função, os pontos do plano \(\psi\) são excluídos do seu domínio?

\item {} 
Dado um ponto \(Y \in \pi\) quantos pontos \(X\) existem tais que \(f(X) = Y\)? Se você desenhasse todos os pontos \(X\) que satisfazem essa propriedade, o que apareceria desenhado?

\item {} 
Verdadeiro ou falso? Para cada ponto \(P \in {\mathbb R}^{3} - \psi\), vale que \(f(f(P)) = f(P)\). Justifique sua resposta.

\item {} 
Você conhece outras funções \(f\) tais que \(f(f(P)) = f(P)\) para todo ponto \(P\) do domínio de \(f\)? Quais?

\item {} 
Verdadeiro ou falso? Se \(f(P) = f(Q)\), então \(P = Q\). Justifique sua resposta.

\item {} 
Se o centro \(O\) é um ponto do plano de projeção \(\pi\), determine \(f(P)\) para todo \(P \in {\mathbb R}^{3} - \psi\).

\end{enumerate}

\item Vimos que uma projeção paralela com relação a uma direção \(d\) sobre um plano de projeção \(\pi\), com a direção \(d\) não paralela a \(\pi\), pode ser considerada como uma função \(f\) de domínio \({\mathbb R}^{3}\) e contradomínio \(\pi\).
\begin{enumerate}
\item {} 
Por que ao se modelar uma projeção paralela por meio de uma função, supõe-se que a direção \(d\) não seja paralela ao plano \(\pi\)?

\item {} 
Dado um ponto \(Y \in \pi\) quantos pontos \(X\) existem tais que \(f(X) = Y\)? Se você desenhasse todos os pontos \(X\) que satisfazem essa propriedade, o que apareceria desenhado?

\item {} 
Verdadeiro ou falso? Para cada ponto \(P \in {\mathbb R}^{3}\), vale que \(f(f(P)) = f(P)\). Justifique sua resposta.

\item {} 
Você conhece outras funções \(f\) tais que \(f(f(P)) = f(P)\) para todo ponto \(P\) do domínio de \(f\)? Quais?

\item {} 
Verdadeiro ou falso? Se \(f(P) = f(Q)\), então \(P = Q\). Justifique sua resposta.

\end{enumerate}

\item Considere a configuração geométrica da \hyperref[\detokenize{GE301-5:fig-proj-ladrilhamentos-01}]{Figura \ref{\detokenize{GE301-5:fig-proj-ladrilhamentos-01}}} da Etapa 1 da PARTE 2 da \hyperref[\detokenize{GE301-5:ativ-proj-comprimentos}]{atividade Comprimentos em projeções}, ainda com os dados \(a = 3\), \(d = 5\) e \(h = 4\) e \(x = 6\). Suponha agora que vários segmentos congruentes a \(RS\) sejam desenhados no plano \(\gamma\): \(R_{7}S_{7}\), \(R_{8}S_{8}\), \(\ldots\), \(R_{n}S_{n}\), \(\ldots\) de tal modo que (1) para cada \(n \geq 7\), o quadrilátero \(RR_{n}S_{n}S\) é um retângulo e (2) para cada \(n \geq 7\), a distância de \(V\) até o segmento \(R_{n}S_{n}\) é \(n\). Assim, os segmentos \(R_{n}S_{n}\) estão uniformemente espaçados, com a distância entre dois segmentos consecutivos sendo igual a \(1\). Observe as alturas \(y_{n} = g(n)\) das projeções \(R_{n}'S_{n}'\) sobre o plano \(\pi\) com relação ao plano “do chão” \(\gamma\): elas aumentam a medida que \(n\) aumentam, mas a diferença entre duas consecutivas parece diminuir.

\begin{figure}[H]
\centering

\noindent\includegraphics[width=300bp]{{perspectiva-ladrilhamentos-04}.jpg}
\caption{}
\label{\detokenize{GE301-E:fig-proj-perspectiva-ladrilhamentos-04-ex}}\end{figure}
\end{enumerate}




\ifnum\aluno=1
\clearpage
\else
\notasfinais
\fi

\bibliographystyle{apalike-pt}
\bibliography{../Bibliografia/perspectiva2_bibliografia.bib}

\nocite{*}

