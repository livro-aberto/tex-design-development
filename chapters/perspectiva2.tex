\renewcommand\chapterillustration{abertura-perspectiva2.jpeg}

\renewcommand\chapterwhat{Representações em Matemática (semió-
tica): exemplos, aspectos cognitivos e cultu-
rais; projeções em perspectiva: conceitua-
lização via definição 3D, propriedades com

justificativas, aplicações (pinturas, ilusões

de ótica); projeções paralelas: conceitua-
lização, propriedades e aplicações (planta

baixa, mapa de fuga, vistas e ilustrações
em áreas diversas).}

\renewcommand\chapterbecause{As projeções em perspectiva fornecem um modelo matemático que auxilia na compre-
ensão de como vemos, comunicamos e inte-
ragimos com o mundo. Já as projeções pa-
ralelas fornecem uma representação mais

simples e fácil de se entender e, assim, elas

têm sido utilizadas para a confecção de ilus-
trações em várias áreas: Arquitetura, En-
genharia, Biologia, etc. Além disso, no dia

a dia, é importante, por exemplo, saber in-
terpretar diagramas 2D de objetos 3D que

descrevem como montar uma cama, colocar
um cartucho em uma impressora, abrir a
porta de emergência do avião, descobrir a
saída de emergência mais próxima em um
hotel, etc.}

\chapter{Vistas ortogonais e representações em perspectiva}
\label{\detokenize{GE301::doc}}\label{\detokenize{GE301:vistas-ortogonais-e-representacoes-em-perspectiva}}

\mbox{}\thispagestyle{empty}\clearpage

\thispagestyle{empty}

\begin{center}
Projeto: LIVRO ABERTO DE MATEMÁTICA

\noindent \begin{tabular}{lcccr}
\includegraphics[scale=.15]{impa}& \quad\quad& \includegraphics[width=3cm]{logo} & \quad\quad& \includegraphics[scale=.24]{obmep} 
\end{tabular}
\end{center}

\vspace*{.3cm}

Cadastre-se como colaborador no site do projeto: \url{umlivroaberto.org}

Versão digital do capítulo:

\url{https://www.umlivroaberto.org/BookCloud/Volume_1/master/view/GE301.html}


\begin{tabular}{p{.15\textwidth}p{.7\textwidth}}
Título: & Vistas Ortogonais e Representações em Perspectiva\\
\\
Ano/ Versão: & 2020 / versão 1.0 de \today\\
\\
Editora & Instituto Nacional de Matem\'atica Pura e Aplicada (IMPA-OS)\\
\\
Realização:& Olimp\'iada Brasileira de Matem\'atica das Escolas P\'ublicas (OBMEP)\\
\\
Produção:& Associação Livro Aberto\\
\\
Coordenação:& Fabio Simas, \\
            & Augusto Teixeira (livroaberto@impa.br)\\
\\
  Autores: & Humberto Bortolossi (UFF),\\
        & Lhaylla Crisaff (UFF),\\
             \\
Revisão: & Wanderley Rezende  \\
                
\\
Design: & Andreza Moreira (Tangentes Design) \\
\\
  Ilustrações: & --- \\ 
\\
Gráficos: & Humberto Bortolossi (UFF), \\
		      & Tarso Caldas (Licenciando da UNIRIO)\\

\\
  Capa: & Foto de Raphaël LR, no Unplash \\
  		& https://unsplash.com/photos/5KO0SEN0WRs \\

\end{tabular}

\begin{figure}[b]
\begin{minipage}[l]{5cm}
\centering

{\large Licença:}

  \includegraphics[width=3.5cm]{cc-by-sa1}
\end{minipage}\hfill
\begin{minipage}[c]{5cm}
\centering
{\large Desenvolvido por}

\includegraphics[width=2.5cm]{logo-associacao.jpg}
\end{minipage}
\begin{minipage}[r]{5cm}
\centering

{\large Patrocínio:}
  \vspace{1em}
  \includegraphics[width=3.5cm]{itau}
\end{minipage}
\end{figure}

\mainmatter

\explore{representando o que vemos}
\label{\detokenize{GE301-0::doc}}\label{\detokenize{GE301-0:explorando-representando-o-que-vemos}}
Desde a pré-história, o ser humano tem registrado em pinturas o que ele vê no mundo que o cerca. Na \hyperref[\detokenize{GE301-0:fig-proj-pintura-01}]{Figura \ref{\detokenize{GE301-0:fig-proj-pintura-01}}}, por exemplo, temos, em (a), um desenho de leões e bisões na Caverna de Chauvet na França (com cerca de 30000 anos de idade) e, em (b), uma pintura rupestre no Parque Nacional Serra da Capivara no Piauí (com cerca de 11000 anos de idade).

\begin{figure}[H]
\centering
\capstart

\noindent\includegraphics[width=400bp]{{fig-proj-pintura-01}.jpg}
\caption{Pinturas pré-históricas.}\label{\detokenize{GE301-0:fig-proj-pintura-01}}\label{\detokenize{GE301-0:id25}}\end{figure}

Ao longo da história, seja em paredes, páginas de livros, telas de pintura ou telas de computador, surgiram diversas formas de se representar os objetos tridimensionais que estão em nossa volta. Neste capítulo, estudaremos duas destas formas de representação, importantes por suas aplicações. Para que você possa entender melhor o contexto, iniciaremos com atividades cujo objetivo é levar você a ver como as pessoas representam o que veem e como nossos cérebros interpretam essas representações.
\phantomsection\label{\detokenize{GE301-0:ativ-proj-atelier-geometrico}}
\begin{task}{Atelier geométrico}

Seu professor irá dispor um conjunto de objetos geométricos sobre uma mesa e o objetivo desta tarefa é que você desenhe em uma folha de papel \textbf{o que você vê nesta cena} o mais fielmente que conseguir.

\begin{figure}[H]
\centering

\noindent\includegraphics[width=300bp]{{atelier}.jpg}
\end{figure}

\end{task}
\phantomsection\label{\detokenize{GE301-0:ativ-proj-lobo}}
\begin{task}{É O Lobo!}



Na sua opinião, qual das seis imagens (A), (B), (C), (D), (E) e (F) a seguir melhor representa um lobo? Por quê?

\begin{figure}[H]
\centering
\capstart

\noindent\includegraphics[width=350bp]{{lobo_1}.jpg}
\caption{Seis representações de um lobo.}\label{\detokenize{GE301-0:fig-proj-lobo}}\label{\detokenize{GE301-0:id28}}\end{figure}

\end{task}


\arrange{tudo é uma questão de comunicação!}
\label{\detokenize{GE301-0:organizando-as-ideias-tudo-e-uma-questao-de-comunicacao}}
Em um primeiro momento, você pode achar que a fotografia (A) na \hyperref[\detokenize{GE301-0:fig-proj-lobo}]{Figura \ref{\detokenize{GE301-0:fig-proj-lobo}}} é a “melhor” representação de um lobo. Mas, pense um pouco: “melhor” em que sentido? O “melhor” sempre pressupõe um critério e, por conseguinte, um contexto.

Por exemplo, caso você queira fazer menção a um lobo em uma mensagem de texto enviada por SMS, então certamente a representação (F) é a mais adequada. Agora, imagine que você está escrevendo um livro de Biologia e sua editora lhe disse que, por razões orçamentárias, apenas figuras em “preto e branco” serão aceitas. Neste caso, as representações (B) e (C) parecem ser a melhor opção. E se você estivesse ilustrando um livro infantil? Aí, as representações (D) e (E) poderiam dar um tom artístico mais pessoal ao livro.

A representação (E) pode parecer muito tosca e infantil, mas lembramos aqui uma frase célebre do pintor Pablo Picasso (1881-1973):  “Levei quatro anos para aprender a pintar como Rafael, mas levei a vida toda para aprender a desenhar como uma criança.”.

\begin{figure}[H]
\centering

\includegraphics[width=320bp]{{picasso}.jpg}
\caption{Os touros de Pablo Picasso.}
\label{\detokenize{GE301-0:fig-proj-picasso}}
\label{\detokenize{GE301-0:id29}}
\end{figure}


Do mesmo modo que um lobo pode ser representado de maneiras diferentes, existem diversas representações para os objetos geométricos tradicionais em Matemática (cubos, cilindros, esferas, pirâmides, etc.). Mais ainda, estudiosos descobriram que a forma de representar muda com a idade de uma pessoa.
O filósofo Georges Henri Luquet explica, por exemplo, que o desenho do cilindro do Estágio 2 na \hyperref[\detokenize{GE301-0:fig-proj-escala-mitchelmore}]{Figura \ref{\detokenize{GE301-0:fig-proj-escala-mitchelmore}}} deve-se a uma preponderância de um “realismo intelectual” em relação a um “realismo visual”: a pessoa sabe que um cilindro circular reto têm duas bases circulares e pensa, nesta etapa, que se não registrar estas estas duas bases circulares, o desenho estaria incompleto. Assim, esta pessoa está registrando o que pensa, não o que vê.

\begin{figure}[H]
\centering
\capstart

\noindent\includegraphics[width=400bp]{{cilindros}.jpg}
\caption{Representação de um cilindro em estágios etários diferentes.}\label{\detokenize{GE301-0:fig-proj-escala-mitchelmore}}\label{\detokenize{GE301-0:id30}}\end{figure}

O psicólogo Sergio Morra, por sua vez, argumenta que a complexidade das regras ou estratégias de organização espacial que uma pessoa consegue dominar está restrita pela quantidade de informação que ela pode assimilar e processar simultaneamente, ou seja, pela memória de trabalho. Assim, os desenhos podem ficar “mais realistas” a medida que a memória de trabalho da pessoa aumenta com a idade.

Outro aspecto interessante é que o meio cultural pode influenciar a maneira como uma pessoa representa objetos tridimensionais, como aponta o estudo de Gutierrez (1998). A \hyperref[\detokenize{GE301-0:fig-proj-aspectos-culturais-01}]{Figura \ref{\detokenize{GE301-0:fig-proj-aspectos-culturais-01}}}), por exemplo, mostra como filhos de tecelões, oleiros e fazendeiros de povoados isolados na Índia, entre 8 e 12 anos de idade, com pouca ou nenhuma escolaridade, desenheram cilindros e pirâmides que lhe foram apresentados.

\begin{figure}[H]
\centering
\capstart

\noindent\includegraphics[width=230bp]{{aspectos_culturais}.jpg}
\caption{Influência de fatores culturais na produção de desenhos em perspectiva (Gutierres, 1998)}\label{\detokenize{GE301-0:fig-proj-aspectos-culturais-01}}\label{\detokenize{GE301-0:id31}}\end{figure}

Muitos acham que a habilidade de desenhar é um dom que, quem não tem, nunca irá desenhar bem. Neurocientistas têm mostrado \textbf{que este não é o caso}! De fato, estudos científicos mostram (a) que, como qualquer outra habilidade humana, com prática e dedicação, é possível aprender a desenhar; (b) que habilidades visuais constituem um dos tipos reconhecidos de inteligência humana; (c) que o desenvolvimento das habilidades espaciais desenvolvem outros tipos de habilidades.

Ainda no contexto de objetos geométricos matemáticos, para você ter uma ideia da multiplicidade de representações, considere o problema de representar no plano o globo terrestre modelado como uma esfera. Essas representações nada mais são do que os \index{mapas cartográficos}mapas cartográficos da Geografia! Existem muitos deles, cada um com propriedades e usos específicos! A escolha do mapa depende do que se quer comunicar!

\begin{figure}[H]
\centering
\capstart

\noindent\includegraphics[width=270bp]{{mapas_1}.jpg}
\caption{Mapas cartográficos são representações no plano do globo terrestre modelado como uma superfície esférica.}\label{\detokenize{GE301-0:fig-proj-mapas-cartograficos}}\label{\detokenize{GE301-0:id32}}\end{figure}

Um ponto muito importante para o que se seguirá é ter em mente que, apesar de podermos representar o que vemos de formas diferentes com usos diferentes, certas representações são construídas de maneira bem específicas e, portanto, possuem propriedades que lhe são próprias. Reconhecer, compreender e empregar corretamente estas propriedades são habilidades fundamentais para você se comunicar adequadamente em termos visuais! Este será exatamente o caso das duas representações 2D de objetos 3D obtidas por projeções em perspectivas e projeções paralelas, temas deste capítulo!

A seguinte analogia entre desenho e escrita, inspirada no livro \emph{Desenho e Escrita como Sistemas de Representação} de Analice Dutra Pillar, pode lhe ajudar a perceber a importância de se dar atenção às características específicas de uma determinada representação. Você se comunica por escrito via WhatsApp e, também, ao fazer uma redação no ENEM. No WhatsApp, pela agilidade que é característica deste meio de comunicação, você usa abreviações: “tdb” (tudo bem), “pdc” (pode crer), “obg” (obrigado), etc. Mesmo com abreviações, as pessoas se entendem. Por outro lado, em uma redação do ENEM, exige-se que o texto seja escrito seguindo características específicas, a saber, “de acordo com a modalidade escrita formal da língua portuguesa”: você deve respeitar as regras ortográficas e gramaticais. Analogamente, existem várias maneiras de se desenhar um cubo. Contudo, os desenhos obtidos por projeções em perspectiva e projeções paralelas possuem propriedades específicas. São essas propriedades e suas aplicações que vamos estudar neste capítulo!

\begin{knowledge}{}

O matemático alemão Johann Carl Friedrich Gauss (1777-1855) demonstrou um teorema, o chamado \emph{egregium}, a partir do qual é possível deduzir o seguinte resultado: qualquer representação plana que se faça de um globo terrestre modelado como uma esfera \textbf{sempre} terá algum tipo de distorção, isto é, ela não preservará ângulos ou não preservará áreas ou não preservará distâncias. Na página web \textless{}\url{https://goo.gl/HbLnPW}\textgreater{}, você encontrará um aplicativo que permite visualizar essas distorções para diferentes mapas cartográficos: as curvas fechadas mais espessas (círculos no exemplo da figura a seguir) são, no mapa, as representações de círculos de mesmo raio desenhados sobre a superfície esférica do globo terrestre. A partir da comparação dos formatos relativos dessas curvas (a \index{indicatriz de Tissot}indicatriz de Tissot) é possível ter uma ideia das distorções presentes no mapa.


\begin{figure}[H]
\centering

\noindent\includegraphics[width=50bp]{egregium-qrcode.png}
\end{figure}

\begin{figure}[H]
\centering

\noindent\includegraphics[width=280bp]{{egregium_1}.jpg}
\end{figure}


Existem mapas que preservam um ou outro atributo geométrico. O mapa de Mercator, por exemplo, preserva ângulos (mas não preserva áreas) e possui uma característica adicional útil para a navegação: as curvas de rumo constante sobre a superfície terrestre são representadas por retas neste mapa.
\end{knowledge}


\explore{interpretando o que vemos}
\label{\detokenize{GE301-1::doc}}\label{\detokenize{GE301-1:explorando-interpretando-o-que-vemos}}\phantomsection\label{\detokenize{GE301-1:ativ-proj-interpretando}}
\begin{task}{Será que é?}

\begin{enumerate}
\item {} 
(Ponzo) Observe a \hyperref[\detokenize{GE301-1:fig-proj-ponzo}]{Figura \ref{\detokenize{GE301-1:fig-proj-ponzo}}}. Qual carro é maior na imagem?

\begin{figure}[H]
\centering
\capstart

\noindent\includegraphics[width=300bp]{{ponzo-illusion-04}.jpg}
\caption{Qual carro é maior na imagem?}\label{\detokenize{GE301-1:fig-proj-ponzo}}\label{\detokenize{GE301-1:id14}}\end{figure}

\item {} 
(Shepard) Observe a \hyperref[\detokenize{GE301-1:fig-proj-shepard}]{Figura \ref{\detokenize{GE301-1:fig-proj-shepard}}}. Qual mesa é mais comprida na imagem?

\begin{figure}[H]
\centering
\capstart

\noindent\includegraphics[width=300bp]{{mesa-de-shepard}.jpg}
\caption{Qual mesa é mais comprida na imagem?}\label{\detokenize{GE301-1:fig-proj-shepard}}\label{\detokenize{GE301-1:id15}}\end{figure}

\end{enumerate}
\end{task}


\arrange{ver é uma atividade complexa!}
\label{\detokenize{GE301-1:organizando-as-ideias-ver-e-uma-atividade-complexa}}
Os dois exemplos apresentados na atividade anterior mostram que o ato de ver e compreender uma imagem não se encerra na própria imagem, mas depende da maneira que nosso cérebro processa toda a informação e se ajusta ao estímulo visual.

Psicólogos têm mapeado outras situações onde nosso cérebro faz adequações visuais subjetivas ao contexto: forma, cor, iluminação, distância, localização e movimento. Mais ainda: não só o sistema visual é afetado por ilusões, os demais sentidos também o são. Um exemplo clássico é o Efeito McGurk que mostra \textbf{como o que você vê altera o modo como você ouve}! Experimente você mesmo por meio do \href{https://goo.gl/k241EQ}{vídeo} no YouTube.

\begin{figure}[H]
\centering

\noindent\includegraphics[width=50bp]{{efeito-mcgurk-01}.png}
\end{figure}

O fato de nosso cérebro estar sucetível a estes tipos de ilusões pode parecer um defeito a princípio mas, como mostra o cientista cognitivo Donald Hoffman nesta  \href{https://goo.gl/x5H5oa}{palestra TED}, isto é resultado de um processo evolutivo que garantiu a nossa sobrevivência.


\begin{figure}[H]
\centering

\noindent\includegraphics[width=50bp]{{ted-realidade-02}.png}
\end{figure}



\begin{figure}[H]
\centering
\capstart

\noindent\includegraphics[width=350bp]{{ted-realidade-01}.jpg}
\caption{\href{https://goo.gl/x5H5oa}{Link para o vídeo}}\label{\detokenize{GE301-1:id16}}\end{figure}

Outro aspecto da interpretação de representações 2D de objetos 3D se refere à questão de ambiguidade: um mesmo desenho plano pode ser a representação de objetos tridimensionais diferentes. Considere, por exemplo, a Imagem (A) na \hyperref[\detokenize{GE301-1:fig-proj-ambiguidade-01}]{Figura \ref{\detokenize{GE301-1:fig-proj-ambiguidade-01}}}. Ela pode ser a representação de um cubo visto de cima como na Imagem (B) ou de um cubo visto de baixo como na Imagem (C).

\begin{figure}[H]
\centering
\setlength{\columnsep}{0pt}

\begin{multicols}{3}
\begin{figure}[H]
\raggedleft
\begin{asy}
size(3cm);
currentprojection=orthographic(3,1,.5);

triple a = (0,0,0);
triple b = (1,0,0);
triple c = (1,1,0);
triple d = (0,1,0);

triple e = (0,0,1);
triple f = (1,0,1);
triple g = (1,1,1);
triple h = (0,1,1);

draw (a -- b -- c -- d -- cycle);
draw (e -- f -- g -- h -- cycle);
draw (a -- b -- f -- e -- cycle);
draw (c -- d -- h -- g -- cycle);
\end{asy}
\centering
\\
(A)
\end{figure}

\begin{figure}[H]
\centering
\begin{asy}
size(3cm);
currentprojection=orthographic(3,1,.5);

triple a = (0,0,0);
triple b = (1,0,0);
triple c = (1,1,0);
triple d = (0,1,0);

triple e = (0,0,1);
triple f = (1,0,1);
triple g = (1,1,1);
triple h = (0,1,1);

draw(e -- f -- g -- h -- cycle);
draw(c -- d -- h -- g -- cycle);
draw(b -- c -- g -- f -- cycle);

draw(b -- a --d, dashed);
draw(a -- e, dashed);
\end{asy}
\centering
\\
(B)
\end{figure}

\begin{figure}[H]
\raggedright
\begin{asy}
size(3cm);
currentprojection=orthographic(3,1,.5);

triple a = (0,0,0);
triple b = (1,0,0);
triple c = (1,1,0);
triple d = (0,1,0);

triple e = (0,0,1);
triple f = (1,0,1);
triple g = (1,1,1);
triple h = (0,1,1);

draw(a -- b -- c -- d -- cycle);
draw(a -- b -- f -- e -- cycle);
draw(a -- d -- h -- e -- cycle);

draw(h -- g -- c, dashed);
draw(g -- f, dashed);
\end{asy}
\centering
\\

(C)
\end{figure}
\end{multicols}
\caption{Um cubo visto de cima ou de baixo?}\label{\detokenize{GE301-1:fig-proj-ambiguidade-01}}\label{\detokenize{GE301-1:id17}}\end{figure}

De fato, a Imagem (A) pode até mesmo nem ser a representação de um cubo, como mostra a animação da \hyperref[\detokenize{GE301-1:fig-proj-ambiguidade-02}]{Figura \ref{\detokenize{GE301-1:fig-proj-ambiguidade-02}}}. A Imagem (A) é conhecida como \index{Cubo de Necker}Cubo de Necker, em homenagem ao cristalógrafo Louis Albert Necker (1786-1861) que observou este tipo de ambiguidade em 1832.



\begin{figure}[H]
\centering
\capstart

\noindent\includegraphics[width=200bp]{{ambiguidade-02}.jpg}
\caption{\href{https://goo.gl/CXR6AG}{Versão interativa}}\label{\detokenize{GE301-1:fig-proj-ambiguidade-02}}\label{\detokenize{GE301-1:id18}}\end{figure}

Compreender como vemos e interpretamos representações 2D de objetos 3D obtidas por projeções centrais e paralelas é uma habilidade importante que afeta o modo de nos cuminicarmos e interagirmos com o mundo.

\begin{reflection}{}

Se nosso cérebro distorce os estímos que recebemos do mundo a nossa volta, como saber o que é real?
\end{reflection}

\begin{reflection}{}

Será que uma pessoa que nasceu cega mas que, posteriormente, recuperou sua visão, saberia ver de imediato? Ou seria necessário “ensiná-la a ver”? Como saber, por exemplo, onde a imagem de um objeto termina e a imagem de outro começa?  Esta \href{https://goo.gl/KLdhKg}{palestra TED} discute esses assuntos, mostra a importância do movimento no processo de se “aprender a ver” e conta como o trabalho do neurocientista indiano Pawan Sinha tem mudado a concepção sobre os mecanismos da visão e, também, as vidas de muitas crianças que nasceram cegas.

\begin{figure}[H]
\centering

\noindent\includegraphics[width=50bp]{{ted-aprendendo-a-ver-02}.png}
\end{figure}

\begin{figure}[H]
\centering
\capstart

\noindent\includegraphics[width=350bp]{{ted-aprendendo-a-ver-01}.jpg}
\caption{Palestra TED.}\label{\detokenize{GE301-1:id19}}\end{figure}
\end{reflection}


\begin{reflection}{}

Todas estas questões de representações e significados fazem parte da \index{semiótica}semiótica, disciplina que se ocupa do estudo dos signos e dos processos significativos na natureza e na cultura. Os signos, aqui, não estão restritos à desenhos em uma folha de papel. Eles podem ser qualquer veículo de significação ou representação de um objeto, de um conceito ou de uma ideia, como textos, sons e gestos. Um dos pontos destacados pela semiótica é a distinção entre a representação de algo e este próprio algo. Um exemplo clássico é dada pela pintura na \hyperref[\detokenize{GE301-1:fig-proj-semiotica-01}]{Figura \ref{\detokenize{GE301-1:fig-proj-semiotica-01}}}. O que é que está na pintura? Se você respondeu “cachimbo”, saiba que a legenda em Francês “Ceci n’est pas une pipe.” tem como tradução “Isto não é um cachimbo.”. Segundo o autor da pintura, o surrealista belga René Magritte (1898-1967), ele não poderia escrever o contrário, pois a pintura não é um cachimbo, mas uma representação de um cachimbo. O nome da pintura: “A Traição das Imagens”.

\begin{figure}[H]
\centering
\capstart

\noindent\includegraphics[width=300bp]{{semiotica-01}.jpg}
\caption{Pintura de René Magritte (1898-1967).}\label{\detokenize{GE301-1:fig-proj-semiotica-01}}\label{\detokenize{GE301-1:id20}}\end{figure}

Uma vez que a comunicação se dá por meio de signos, a semiótica é de interesse para muitas áreas: Propaganda, Cinema, Ciência, Literatura, Religião … Em Matemática, o aspecto semiótico é fundamental, como aponta Pinilla (2007):

\begin{quote}
É importante ter em mente que os conceitos matemáticos não existem na realidade concreta. O ponto P, o número 3, adição, paralelismo entre retas não são objetos concretos os quais existem na realidade empírica. Eles são conceitos puros, ideais e abstratos e, desta maneira, eles não podem ser “exibidos empiricamente”, como em outras Ciências. Em Matemática, os conceitos só podem ser representados por um registro semiótico determinado. De fato, em Matemática, não trabalhamos diretamente com os objetos (isto é, com os conceitos), mas com suas representações semióticas.
\end{quote}

Caso você queira saber mais sobre semiótica, recomendamos começar com o livro “O que é semiótica?” da Coleção “Primeiros Passos” da Editora Brasiliense (Santaella, 1998).
\end{reflection}
\phantomsection\label{\detokenize{GE301-1:obs-proj-por-que-estudar-o-assunto}}
\begin{observation}{}

No que se segue, iremos estudar duas formas de representação bem específicas: aquelas obtidas por projeções em perspectiva e projeções paralelas.

As projeções em perspectiva fornecem um modelo matemático para a visão humana e para dispositivos óticos (como câmeras) e o estudo deste modelo auxilia na compreensão de como vemos, comunicamos e interagimos com o mundo. As projeções paralelas, por sua vez, fornecem uma representação mais simples e mais fácil de se entender e, por este motivo, elas têm sido utilizadas para a confecção de ilustrações em várias áreas: Arquitetura, Engenharia, Biologia, Física, etc.

Cabe observar que projeções em perspectiva e paralelas fazem parte das habilidades espaciais as quais, por sua vez, constituem um dos tipos reconhecidos de inteligência humana (Gray et al, 2004), (Gardner, 2011).

As habilidades espaciais são particularmente críticas para profissões relacionadas com as áreas de Ciência, Tecnologia, Engenharia e Matemática (STEM), conforme apontam vários estudos recentes (NRC, 2006), (Utta et al, 2012), (Khine, 2017), (Newcombe, 2017).

Mesmo no dia a dia, é importante, por exemplo, saber interpretar os diagramas 2D de objetos 3D que descrevem como montar uma cama, colocar um cartucho em uma impressora, abrir a porta de emergência do avião, descobrir a saída de emergência mais próxima em um hotel ou em um estádio de futebol (mapa de fuga, saídas de emergência), etc.
\begin{quote}

\begin{figure}[H]
\centering

\noindent\includegraphics[width=400bp]{{planta-baixa-03}.jpg}
\end{figure}

Mapa do circuito de visitação do terceiro andar do Aquário do Rio de Janeiro (fonte: Joselí Maria Silva dos Santos).
\end{quote}

Como veremos, as representações obtidas por projeções em perspectiva e projeções paralelas possuem propriedades bem \textbf{específicas}. Reconhecer e usar essas propriedades adequadamente é importante para você entender e se fazer entender em termos de comunicação visual.
\end{observation}

\begin{reflection}{}

\begin{DUlineblock}{0em}
\item[] \emph{This life five windows of the soul}
\item[] \emph{Distorts the heaven from pole to pole}
\item[] \emph{And leads you to believe a lie}
\item[] \emph{When you see with, no thro’, the eye.}
\end{DUlineblock}

Poema do poeta, pintor, ilustrador e entalhador William Blake (1757-1827).
\end{reflection}


\explore{projeções paralelas e em perspectiva}
\label{\detokenize{GE301-2:explorando-projecoes-em-perspectiva-e-projecoes-paralelas}}\label{\detokenize{GE301-2::doc}}\phantomsection\label{\detokenize{GE301-2:ativ-proj-luz-e-sombras}}
\begin{task}{Luzes e sombras}

Nesta atividade vamos explorar a geometria das sombras! Para isto, você receberá um \emph{kit} cujas sombras deverá analisar: um cubo vazado, um triângulo e um lápis (ou uma caneta ou ainda um canudinho plástico). Você também receberá uma folha de papel A4 ou uma cartolina que servirá como anteparo onde as sombras devem ser projetadas.

\textbf{No que se segue, o termo *configuração* significa uma escolha da posição do objeto, da fonte de luz e do anteparo, conforme o caso.} Assim, o termo \emph{para qualquer configuração} significa para qualquer escolha da posição do objeto, da fonte de luz e do anteparo.

\textbf{Experimentos com um lápis}
\begin{enumerate}
\item {} 
O comprimento da sombra é sempre igual ao comprimento do lápis, independentemente da configuração?
\begin{itemize}
\item {} 
Resposta para o caso da luz da lanterna do celular:

\item {} 
Resposta para o caso da luz do Sol:

\end{itemize}

\item {} 
Existe alguma configuração para a qual o comprimento da sombra seja igual ao comprimento do lápis?
\begin{itemize}
\item {} 
Resposta para o caso da luz da lanterna do celular:

\item {} 
Resposta para o caso da luz do Sol:

\end{itemize}

\item {} 
Segure o seu lápis no meio com as pontas de seus dedos, isto é, considerando o lápis como se fosse um segmento de reta, segure-o pelo seu ponto médio. A sombra das pontas de seus dedos sempre está no meio da sombra do lápis independentemente da configuração?
\begin{itemize}
\item {} 
Resposta para o caso da luz da lanterna do celular:

\item {} 
Resposta para o caso da luz do Sol:

\end{itemize}

\item {} 
Segure o seu lápis, com as pontas de seus dedos, a aproximadamente 1/3 de uma das extremidades. Existe alguma configuração para a qual a sombra das pontas de seus dedos está no meio da sombra do lápis?
\begin{itemize}
\item {} 
Resposta para o caso da luz da lanterna do celular:

\item {} 
Resposta para o caso da luz do Sol:

\end{itemize}

\item {} 
Em qual configuração a sombra do lápis tem a menor área possível?
\begin{itemize}
\item {} 
Resposta para o caso da luz da lanterna do celular:

\item {} 
Resposta para o caso da luz do Sol:

\end{itemize}

\item {} 
Existe alguma configuração onde a sombra não se altere ao mover o lápis em alguma direção? Aqui, \emph{não se altere} signfica ser exatamente a mesma no mesmo lugar.
\begin{itemize}
\item {} 
Resposta para o caso da luz da lanterna do celular:

\item {} 
Resposta para o caso da luz do Sol:

\end{itemize}

\end{enumerate}

\textbf{Experimentos com um triângulo}
\begin{enumerate}
\item {} 
Existe alguma configuração para a qual a sombra do triângulo é um triângulo isósceles?
\begin{itemize}
\item {} 
Resposta para o caso da luz da lanterna do celular:

\item {} 
Resposta para o caso da luz do Sol:

\end{itemize}

\item {} 
Existe alguma configuração para a qual a sombra do triângulo é um triângulo equilátero?
\begin{itemize}
\item {} 
Resposta para o caso da luz da lanterna do celular:

\item {} 
Resposta para o caso da luz do Sol:

\end{itemize}

\item {} 
Em qual configuração a sombra do triângulo tem a menor área possível?
\begin{itemize}
\item {} 
Resposta para o caso da luz da lanterna do celular:

\item {} 
Resposta para o caso da luz do Sol:

\end{itemize}

\item {} 
Existe alguma configuração onde a sombra do triângulo não se altere ao movê-lo em alguma direção? Qual?
\begin{itemize}
\item {} 
Resposta para o caso da luz da lanterna do celular:

\item {} 
Resposta para o caso da luz do Sol:

\end{itemize}

\item {} 
O baricentro de um triângulo é o  ponto de interseção das \index{medianas}medianas do triângulo, isto é, o ponto de interseção dos segmentos de reta que ligam um vértice ao ponto médio do lado oposto. Faça um furo no \index{baricentro}baricentro do seu triângulo, de forma que, ao expô-lo à luz, o ponto correspondente no anteparo ficará iluminado. Este ponto iluminado é baricentro da sombra do triângulo?
\begin{itemize}
\item {} 
Resposta para o caso da luz da lanterna do celular:

\item {} 
Resposta para o caso da luz do Sol:

\end{itemize}

\end{enumerate}

\textbf{Experimentos com um cubo vazado}
\begin{enumerate}
\item {} 
As arestas do cubo vazado têm todas o mesmo tamanho. O mesmo acontece para as sombras destas arestas?
\begin{itemize}
\item {} 
Resposta para o caso da luz da lanterna do celular:

\item {} 
Resposta para o caso da luz do Sol:

\end{itemize}

\item {} 
Existe alguma configuração para a qual a sombra do cubo vazado seja semelhante à imagem da \hyperref[\detokenize{GE301-2:fig-proj-quadrado-vazado-01}]{Figura \ref{\detokenize{GE301-2:fig-proj-quadrado-vazado-01}}}? Em caso afirmativo, é possível manter esta sombra movendo o cubo vazado em alguma direção? Qual?

\begin{figure}[H]
\centering
\capstart

\noindent\includegraphics[width=100bp]{{quadrado-vazado-01_2}.jpg}
\caption{Quadrado vazado.}\label{\detokenize{GE301-2:fig-proj-quadrado-vazado-01}}\label{\detokenize{GE301-2:id1}}\end{figure}
\begin{itemize}
\item {} 
Resposta para o caso da luz da lanterna do celular:

\item {} 
Resposta para o caso da luz do Sol:

\end{itemize}

\item {} 
Arestas que são perpendiculares no cubo vazado têm sombras que são perpendiculares no anteparo de projeção?
\begin{itemize}
\item {} 
Resposta para o caso da luz da lanterna do celular:

\item {} 
Resposta para o caso da luz do Sol:

\end{itemize}

\item {} 
Arestas que são são paralelas no cubo vazado têm sombras que são paralelas no anteparo de projeção?
\begin{itemize}
\item {} 
Resposta para o caso da luz da lanterna do celular:

\item {} 
Resposta para o caso da luz do Sol:

\end{itemize}

\end{enumerate}

\textbf{Outros experimentos}
\begin{enumerate}
\item {} 
Como você faria para determinar a direção de incidência dos raios solares no anteparo?

\item {} 
Posicione o anteparo perpendicularmente à direção de incidência dos raios solares. O que acontece com o formato da sombra do lápis, do triângulo ou do cubo se você movê-los \textbf{paralelamente} à direção de incidência dos raios solares?

\item {} 
Na \hyperref[\detokenize{GE301-2:fig-proj-sombra-vazada-01}]{Figura \ref{\detokenize{GE301-2:fig-proj-sombra-vazada-01}}}, PQRS é sombra de qual face do cubo vazado? Tente responder analisando apenas a figura e, depois, teste a sua resposta com um experimento!

\begin{figure}[H]
\centering
\capstart

\noindent\includegraphics[width=300bp]{{sombra-vazada-01_1}.jpg}
\caption{Sombra vazada.}\label{\detokenize{GE301-2:fig-proj-sombra-vazada-01}}\label{\detokenize{GE301-2:id2}}\end{figure}

\item {} 
Na configuração da \hyperref[\detokenize{GE301-2:fig-proj-sombra-vazada-01}]{Figura \ref{\detokenize{GE301-2:fig-proj-sombra-vazada-01}}}, o que acontece com a sombra do cubo vazada se a lanterna do celular se aproximar do cubo? E se a lanterna se afastar?

\item {} 
Na \hyperref[\detokenize{GE301-2:fig-proj-sombra-vazada-02}]{Figura \ref{\detokenize{GE301-2:fig-proj-sombra-vazada-02}}}, PQRS é sombra de qual face do cubo vazado? Tente responder analisando apenas a figura e, depois, teste a sua resposta com um experimento!

\begin{figure}[H]
\centering
\capstart

\noindent\includegraphics[width=300bp]{{sombra-vazada-02_1}.jpg}
\caption{Sombra vazada.}\label{\detokenize{GE301-2:fig-proj-sombra-vazada-02}}\label{\detokenize{GE301-2:id3}}\end{figure}

\end{enumerate}
\end{task}

\phantomsection\label{\detokenize{GE301-2:ativ-proj-modelos-de-projecao}}
\begin{task}{Dois modelos de projeção}


O objetivo desta atividade é levar você a ponderar sobre concepções de modelos geométricos que permitam representar projeções de sombras considerando, para isto, algumas hipóteses simplificadoras. Esses modelos serão úteis no que se segue ao longo do capítulo. De fato, com esse conhecimento, será possível explicar e quantificar os fenômenos que você observou na \DUrole{xref,std,std-ref}{ativ-proj-luz-e-sombras} e, também, compreender o seu uso em aplicações diversas.
\begin{enumerate}
\item {} 
Vamos supor que a lanterna do celular possa ser representada por um ponto que emite raios de luz.
\begin{itemize}
\item {} 
Desenhe, a lápis, um diagrama representando o ponto de luz, alguns raios luminosos que dele emanam e como estes atingem o anteparo.

\item {} 
No desenho que você fez no item anterior, inclua um triângulo opaco entre o ponto de luz e o anteparo. Que partes dos raios de luz deixarão de atingir o anteparo? Redesenhe estas partes usando uma linha tracejada. Como ficará desenhada a sombra do triângulo?

\end{itemize}

\item {} \begin{itemize}
\item {} 
Considere a \hyperref[\detokenize{GE301-2:fig-proj-raios-do-sol-03}]{Figura \ref{\detokenize{GE301-2:fig-proj-raios-do-sol-03}}}. Pergunta 1: qual representação do Sol é mais comum entre as crianças? (A), (B) ou (C)? Pergunta 2: qual representação do Sol é mais fiel ao comportamento dos raios de luz? (A), (B) ou (C)?

\end{itemize}

\begin{figure}[H]
\centering
\capstart

\noindent\includegraphics[width=350bp]{{raios-de-luz-03_1}.jpg}
\caption{Três representações dos raios do Sol.}\label{\detokenize{GE301-2:fig-proj-raios-do-sol-03}}\label{\detokenize{GE301-2:id22}}\end{figure}
\begin{itemize}
\item {} 
Uma simplificação frequentemente usada é a de admitir que os raios do Sol chegam à Terra paralelos entre si. Essa simplificação é razoável para você? Dê argumentos que justifiquem sua opinião!

\item {} 
Admitindo que os raios do Sol chegam à Terra paralelos entre si, desenhe, a lápis, um diagrama representando alguns raios solares atingindo o anteparo.

\item {} 
No desenho que você fez no item anterior, inclua um triângulo opaco. Que partes dos raios de luz deixarão de atingir o anteparo? Redesenhe estas partes usando uma linha tracejada. Como ficará desenhada a sombra do triângulo?

\end{itemize}

\end{enumerate}
\end{task}

\arrange{projeções paralelas e em perspectiva}
\label{\detokenize{GE301-3:organizando-as-ideias-projecoes-em-perspectiva-e-projecoes-paralelas}}\label{\detokenize{GE301-3::doc}}
\textbf{Projeções em perspectiva}

Na \DUrole{xref,std,std-ref}{ativ-proj-modelos-de-projecao}, vamos modelar a lanterna do celular como um ponto \(O\) e o anteparo como um plano \(\pi\) (o plano de projeção). Um objeto opaco, como o triângulo \(ABC\) na \hyperref[\detokenize{GE301-3:fig-proj-perspectiva-01}]{Figura \ref{\detokenize{GE301-3:fig-proj-perspectiva-01}}}, irá obstruir os raios de luz que emanam de \(O\), produzindo uma sombra sobre o plano \(\pi\). Como determinar exatamente quais pontos de \(\pi\) percentem à sombra? Para cada ponto \(P\) do triângulo \(ABC\), construa a reta \(OP\) que liga \(O\) a \(P\). Esta reta irá intersectar o plano \(\pi\) em um ponto \(P'\). Este ponto \(P'\) de interseção da reta \(OP\) com o plano \(\pi\) é, portanto, um ponto da sombra do triângulo \(ABC\). De fato, todo ponto \(P'\) da sombra é obtido por este processo, isto é, um ponto \(P'\) do plano pertence à sombra do triângulo \(ABC\) se, e somente se, existe um ponto \(P\) do triângulo \(ABC\) tal que a interseção da reta \(OP\) com o plano \(\pi\) é o ponto \(P'\). Além do ponto \(P'\), a \hyperref[\detokenize{GE301-3:fig-proj-perspectiva-01}]{Figura \ref{\detokenize{GE301-3:fig-proj-perspectiva-01}}} mostra também o processo para os pontos \(A'\), \(B'\) e \(C'\).

\begin{figure}[H]
\centering
\capstart

\noindent\includegraphics[width=350bp]{{projecao-perspectiva-01_2}.jpg}
\caption{Um modelo para o experimento com a lanterna do celular.}\label{\detokenize{GE301-3:fig-proj-perspectiva-01}}\label{\detokenize{GE301-3:id1}}\end{figure}

Vamos agora abstrair ainda mais o processo, ou seja, vamos considerar um contexto matemático que, apesar de inspirado por luzes e sombras, será puramente geométrico. Esta abstração será útil para modelar outras situações, como veremos mais adiante.

Desta maneira, considere no espaço tridimensional \({\mathbb R}^{3}\) um plano \(\pi\) e um ponto \(O\). Seja \(\psi\) o plano paralelo à \(\pi\) passando por \(O\). Se \(P\) é um ponto que não pertence a \(\psi\), então o ponto \(P'\) de intersecção entre a reta \(OP\) e o plano \(\pi\) é denominado \index{projeção em perspectiva}projeção em perspectiva do ponto \(P\) sobre o \index{plano de projeção}plano de projeção \(\pi\) com relação ao \index{centro}centro \(O\).

\begin{figure}[H]
\centering
\capstart

\noindent\includegraphics[width=300bp]{{projecao-perspectiva-05}.jpg}
\caption{\(P'\) é a projeção em perspectiva do ponto \(P\) sobre o plano de projeção \(\pi\) com relação ao centro \(O\).}\label{\detokenize{GE301-3:fig-proj-perspectiva-02}}\label{\detokenize{GE301-3:id2}}\end{figure}

Vamos agora considerar uma outra situação onde projeções em perspectiva aparecem. Suponha que você queira desenhar um quadro de uma cena. Mas você quer um quadro tão perfeito que, ao observá-lo frente à cena, ele se confunda como a própria cena. O pintor surrealista belga René Magritte (1898-1967) imaginou essa situação em alguns de seus quadros (\hyperref[\detokenize{GE301-3:fig-proj-janela-de-alberti-01}]{Figura \ref{\detokenize{GE301-3:fig-proj-janela-de-alberti-01}}}). Como produzir um tal quadro?

\begin{figure}[H]
\centering
\capstart

\noindent\includegraphics[width=400bp]{{rene-magritte-the-human-condition-03}.jpg}
\caption{Pinturas “A Condição Humana” do artista surrealista belga René Magritte (1898-1967).}\label{\detokenize{GE301-3:fig-proj-janela-de-alberti-01}}\label{\detokenize{GE301-3:id3}}\end{figure}

Suponha que a cena seja composta por um cubo, como no caso da \hyperref[\detokenize{GE301-3:fig-proj-janela-de-alberti-03}]{Figura \ref{\detokenize{GE301-3:fig-proj-janela-de-alberti-03}}}. Cada ponto do cubo está emitindo um raio luminoso para o olho do observador. Ao posicionar o quadro frente à cena, basta então desenharmos os pontos de interseção destes raios emitidos pelo cubo com o plano do quadro. Como cada ponto de interseção do quadro está alinhado com o respectivo ponto do cubo e o olho do observador, este não notará a diferença. É como se o quadro funcionasse como uma janela para a cena, analogia esta idealizada pelo pintor renascentista italiano Leon Battista Alberti (1404-1472).

\begin{figure}[H]
\centering
\capstart

\noindent\includegraphics[width=350bp]{{observacao}.jpg}
\caption{A métafora da janela.}\label{\detokenize{GE301-3:fig-proj-janela-de-alberti-03}}\label{\detokenize{GE301-3:id4}}\end{figure}

Note que este processo de produzir um quadro que funcione como uma janela nada mais é do que uma projeção em perspectiva: o centro \(O\) é a posição do olho do observador e o plano de projeção \(\pi\) é o plano do quadro.

\begin{figure}[H]
\centering
\capstart

\noindent\includegraphics[width=280bp]{{projecao-perspectiva-03_1}.jpg}
\caption{\(P'\) é a projeção em perspectiva do ponto \(P\) sobre o plano de projeção \(\pi\) com relação ao centro \(O\).}\label{\detokenize{GE301-3:fig-proj-perspectiva-03}}\label{\detokenize{GE301-3:id5}}\end{figure}

Enquanto nos experimentos com a luz da lanterna do celular o objeto ficava “entre” o centro \(O\) e o plano \(\pi\), no caso da métafora da Janela de Alberti, o plano \(\pi\) fica entre \(O\) e o objeto. Ainda assim, as duas situações são modeladas por projeções em perspectiva.

O objeto também pode ser posicionado de modo que o centro \(O\) fique entre este e o plano de projeção, como mostra a \hyperref[\detokenize{GE301-3:fig-proj-perspectiva-06}]{Figura \ref{\detokenize{GE301-3:fig-proj-perspectiva-06}}}. Este tipo de configuração modela um terceiro tipo de situação: as \index{câmeras obscuras}câmeras obscuras, modelos básicos de câmera fotográfica sem lentes (ver \hyperref[\detokenize{GE301-3:fig-proj-kircher-01}]{Figura \ref{\detokenize{GE301-3:fig-proj-kircher-01}}}).

\begin{figure}[H]
\centering
\capstart

\noindent\includegraphics[width=280bp]{{projecao-perspectiva-06}.jpg}
\caption{\(P'\) é a projeção em perspectiva do ponto \(P\) sobre o plano de projeção \(\pi\) com relação ao centro \(O\).}\label{\detokenize{GE301-3:fig-proj-perspectiva-06}}\label{\detokenize{GE301-3:id6}}\end{figure}

Supondo que a abertura da pupila seja pequena o suficiente e ignorando-se a presença de lentes e a curvatura da retina, o olho humano também pode ser considerado como uma câmera obscura e, assim, também modelado por projeções em perspectiva. É este modelo simplificado que consideraremos neste capítulo.


\begin{figure}[H]
\centering
\capstart

\noindent\includegraphics[width=180bp]{{Descartes_Diagram_of_ocular_refraction._Wellcome_L0012003}.jpg}
\caption{Esquema do olho proposto por René Descartes em sua obra \emph{A dióptrica} (fonte: \href{http://www.revistas.usp.br/ss/article/view/11212/12980}{Revista Scientiae Studia}).}\label{\detokenize{GE301-3:fig-proj-olho-humano-01}}\label{\detokenize{GE301-3:id7}}\end{figure}

Resumindo: projeções em perspectiva modelam pinturas (quando o plano de projeção está entre o observador e o objeto), sombras (quando o objeto está entre o observador e o plano de projeção) e câmeras e modelos simplificados do olho humano (quando o observador está entre o objeto e o plano de projeção).

% \begin{figure}[H]
% \centering

% \noindent\includegraphics[width=300bp]{{projecao-cubo}.png}
% \end{figure}

\begin{reflection}{}

Note que uma projeção em perspectiva pode ser interpretada como uma \textbf{função} \(f\) de domínio \({\mathbb R}^{3} - \psi\) e contradomínio \(\pi\) que, a cada ponto \(P \in {\mathbb R}^{3} - \psi\), faz associar o ponto \(P'\) de interseção entre a reta \(OP\) e o plano \(\pi\), onde \(\psi\) é o plano passando por \(O\) e paralelo a \(\pi\). Assim, no contexto da \hyperref[\detokenize{GE301-3:fig-proj-perspectiva-01}]{Figura \ref{\detokenize{GE301-3:fig-proj-perspectiva-01}}}, temos que \(f(P) = P'\), \(f(A) = A'\), \(f(B) = B'\) e \(f(C) = C'\).
\end{reflection}

\textbf{Projeções paralelas}

Na \DUrole{xref,std,std-ref}{ativ-proj-modelos-de-projecao}, vamos modelar o anteparo usado nos experimentos com raios solares como um plano \(\pi\). Um objeto opaco, como o triângulo \(ABC\) na \hyperref[\detokenize{GE301-3:fig-proj-paralela-01}]{Figura \ref{\detokenize{GE301-3:fig-proj-paralela-01}}}, irá obstruir os raios do Sol, os quais estamos supondo aqui serem todos paralelos, produzindo então uma sombra sobre o plano \(\pi\). Como determinar exatamente quais pontos de \(\pi\) percentem à sombra? Para cada ponto \(P\) do triângulo \(ABC\), construa a reta que é paralela à direção dos raios do Sol. Esta reta irá intersectar o plano \(\pi\) em ponto \(P'\). Este ponto \(P'\) é um ponto da sombra do triângulo \(ABC\). De fato, todo ponto \(P'\) da sombra é obtido por este processo, isto é, um ponto \(P'\) do plano pertence à sombra do triângulo \(ABC\) se, e somente se, existe um ponto \(P\) do triângulo \(ABC\) tal que a interseção da reta que passa por \(P\) e é paralela aos raios Sol com o plano \(\pi\) é o ponto \(P'\). Além do ponto \(P'\), a \hyperref[\detokenize{GE301-3:fig-proj-paralela-01}]{Figura \ref{\detokenize{GE301-3:fig-proj-paralela-01}}} mostra também o processo para os pontos \(A'\), \(B'\) e \(C'\).

\begin{figure}[H]
\centering
\capstart

\noindent\includegraphics[width=280bp]{{projecao-paralela-01_1}.jpg}
\caption{Um modelo para o experimento com a luz do Sol.}\label{\detokenize{GE301-3:fig-proj-paralela-01}}\label{\detokenize{GE301-3:id8}}\end{figure}

Vamos agora abstrair ainda mais o processo, ou seja, vamos considerar um contexto matemático que, apesar de inspirado por luzes e sombras, será puramente geométrico.

Desta maneira, considere no espaço tridimensional \({\mathbb R}^{3}\) um plano \(\pi\) e uma direção determinada por uma reta \(d\) que não é paralela ao plano \(\pi\). Se \(P\) é um ponto qualquer, então o ponto \(P'\) de intersecção entre a reta que passa por \(P\) e é paralela à reta \(d\) e o plano \(\pi\) é denominado \index{projeção paralela}projeção paralela do ponto \(P\) com relação a direção dada pela reta \(d\) sobre o plano de projeção \(\pi\).

\begin{figure}[H]
\centering
\capstart

\noindent\includegraphics[width=280bp]{{projecao-paralela-03_1}.jpg}
\caption{\(P'\) é a projeção paralela do ponto \(P\) com relação a direção dada pela reta \(d\) sobre o plano de projeção \(\pi\).}\label{\detokenize{GE301-3:fig-proj-paralela-03}}\label{\detokenize{GE301-3:id9}}\end{figure}

Se a reta \(d\) for perpendicular ao plano \(\pi\), então a projeção paralela é denominada \index{projeção ortogonal}projeção ortogonal. Uma projeção paralela que não é ortogonal é denominada \index{projeção oblíqua}projeção oblíqua.

\begin{figure}[H]
\centering
\capstart

\noindent\includegraphics[width=280bp]{{projecao-paralela-02_2}.jpg}
\caption{\(P'\) é a projeção ortogonal do ponto \(P\) com relação a direção dada pela reta \(d\) perpendicular ao plano de projeção \(\pi\) sobre este plano.}\label{\detokenize{GE301-3:fig-proj-paralela-02}}\label{\detokenize{GE301-3:id10}}\end{figure}

Observação. As projeções paralelas definidas aqui são generalizações para o espaço \({\mathbb R}^{3}\) das projeções paralelas no plano que você estudou na \DUrole{xref,std,std-ref}{ativ-projecao-paralela} do capítulo sobre Teorema de Tales. Aqui, a projeção é em um plano e, lá, em uma reta.

\begin{reflection}{}

Note que uma projeção paralela pode ser interpretada como uma \textbf{função} \(f\) de domínio \({\mathbb R}^{3}\) e contradomínio \(\pi\) que, a cada ponto \(P \in {\mathbb R}^{3}\), faz associar o ponto \(P'\) de interseção entre a reta que passa por \(P\) e é paralela a reta \(d\) e o plano \(\pi\), supondo que \(d\) não é paralela ao plano \(\pi\). Assim, no contexto da \hyperref[\detokenize{GE301-3:fig-proj-paralela-01}]{Figura \ref{\detokenize{GE301-3:fig-proj-paralela-01}}}, temos que \(f(P) = P'\), \(f(A) = A'\), \(f(B) = B'\) e \(f(C) = C'\).
\end{reflection}

\begin{knowledge}{}

Com o Renascimento (século XIV-século XVII), os artistas começaram a fazer suas pinturas com a preocupação de retratar a realidade, isto é, retratar o que se vê. Para isso, eles consideraram o uso de princípios óticos geométricos e, em particular, das projeções em perspectiva. Vários aparatos foram idealizados com o próposito de produzir imagens realistas. Observe que o princípio básico de todos os dispositivos é o alinhamento do ponto do objeto a ser retratado, do ponto projetado no quadro e um centro fixo, tipicamente, o olho do observador.

\begin{figure}[H]
\centering
\capstart

\noindent\includegraphics[width=300bp]{{durer-01}.jpg}
\caption{Dispositivo de Albrecht Dürer (1471-1528).}\label{\detokenize{GE301-3:id11}}\end{figure}

\begin{figure}[H]
\centering
\capstart

\noindent\includegraphics[width=300bp]{{cigoli-02}.jpg}
\caption{Dispositivo de Lodovico Cardi (Cigoli) (1559-1613).}\label{\detokenize{GE301-3:id12}}\end{figure}

\begin{figure}[H]
\centering
\capstart

\noindent\includegraphics[width=250bp]{{jamnitzer-01}.jpg}
\caption{Dispositivo de Wenzel Jamnitzer (1507/1508-1585).}\label{\detokenize{GE301-3:id13}}\end{figure}

\begin{figure}[H]
\centering
\capstart

\noindent\includegraphics[width=250bp]{{schmalcalder-02}.jpg}
\caption{Dispositivo de Charles Augustus Schmalcalder (1781-1843).}\label{\detokenize{GE301-3:id14}}\end{figure}

\begin{figure}[H]
\centering
\capstart

\noindent\includegraphics[width=250bp]{{kircher-01}.jpg}
\caption{Camera obscura de Athanasius Kircher (1601-1680).}\label{\detokenize{GE301-3:fig-proj-kircher-01}}\label{\detokenize{GE301-3:id15}}\end{figure}

Existiram dispositivos renascentistas que produziam desenhos em projeções paralelas? O único que se conhece até o momento é a máquina de Johannes Lencker (1523-1585) que produzir desenhos em projeções ortogonais.

\begin{figure}[H]
\centering
\capstart

\noindent\includegraphics[width=250bp]{{lencker-01}.jpg}
\caption{Dispositivo de Johannes Lencker (1523-1585).}\label{\detokenize{GE301-3:id16}}\end{figure}

Enquanto que os pintores renascentistas procuravam fazer seus quadros retratando as pessoas como as vemos, na Idade Média essa preocupação não aparecia. No lugar de princípios óticos geométricos, as regras medievais incluiam pintar as pessoas de acordo com o seu \emph{status} social: quanto maior o \emph{status}, maior o tamanho na pintura (\hyperref[\detokenize{GE301-3:fig-proj-medieval-social-03}]{Figura \ref{\detokenize{GE301-3:fig-proj-medieval-social-03}}}, \hyperref[\detokenize{GE301-3:fig-proj-medieval-social-05}]{Figura \ref{\detokenize{GE301-3:fig-proj-medieval-social-05}}}, \hyperref[\detokenize{GE301-3:fig-proj-medieval-social-01}]{Figura \ref{\detokenize{GE301-3:fig-proj-medieval-social-01}}}).

\begin{figure}[H]
\centering
\capstart

\noindent\includegraphics[width=200bp]{{medieval-social-03}.jpg}
\caption{São Lourenço entre Santos e Patrocinadores de Fra Filippo Lippi (1406-1469).}
\label{\detokenize{GE301-3:fig-proj-medieval-social-03}}

\end{figure}

\begin{figure}[H]
\centering
\capstart

\noindent\includegraphics[width=300bp]{{medieval-social-05}.jpg}
\caption{Henrique III acompanhando o Mestre de Obras (século XIV).}\label{\detokenize{GE301-3:fig-proj-medieval-social-05}}\label{\detokenize{GE301-3:id18}}\end{figure}

\begin{figure}[H]
\centering
\capstart

\noindent\includegraphics[width=200bp]{{medieval-social-01}.jpg}
\caption{Políptico da Misericórdia de Piero della Francesca (1415-1492).}\label{\detokenize{GE301-3:fig-proj-medieval-social-01}}\label{\detokenize{GE301-3:id19}}\end{figure}

\end{knowledge}


\practice{}
\label{\detokenize{GE301-4::doc}}\label{\detokenize{GE301-4:praticando-1}}\phantomsection\label{\detokenize{GE301-4:ativ-proj-feixe-de-retas}}
\begin{task}{Feixe de retas}

\begin{enumerate}
\item {} 
(Projeções em Perspectiva)
\begin{itemize}
\item {} 
Na figura a seguir, (1) a reta que passa pelo ponto \(O\) e o centro do círculo é perpendicular ao plano \(\pi\) e (2) o círculo é paralelo a \(\pi\). Como vimos, para determinar a projeção em perspectiva do círculo com relação ao centro \(O\) sobre o plano de projeção \(\pi\), é necessário construir retas que passam por \(O\) e por pontos do círculo. Se desenharmos todas estas retas, que tipo de superfície será obtida?

\end{itemize}

\begin{figure}[H]
\centering

\noindent\includegraphics[width=300bp]{{praticando-feixe-de-retas-01}.jpg}
\end{figure}
\begin{itemize}
\item {} 
Na figura a seguir, (1) a reta que passa pelo ponto \(O\) e o centro do quadrado é perpendicular ao plano \(\pi\) e (2) o quadrado é paralelo a \(\pi\). Como vimos, para determinar a projeção em perspectiva do quadrado com relação ao centro \(O\) sobre o plano de projeção \(\pi\), é necessário construir retas que passam por \(O\) e por pontos do quadrado. Se desenharmos todas estas retas, que tipo de superfície será obtida?

\end{itemize}

\begin{figure}[H]
\centering

\noindent\includegraphics[width=300bp]{{praticando-feixe-de-retas-02}.jpg}
\end{figure}

\item {} 
(Projeções Paralelas)
\begin{itemize}
\item {} 
Na figura a seguir, (1) a reta \(d\) que passa pelo centro do círculo é perpendicular ao plano \(\pi\) e (2) o círculo é paralelo a \(\pi\). Como vimos, para determinar a projeção paralela do círculo com relação a direção dada por \(d\) sobre o plano de projeção \(\pi\), é necessário construir retas que passam por pontos do círculo e que são paralelas a \(d\). Se desenharmos todas estas retas, que tipo de superfície será obtida?

\begin{figure}[H]
\centering

\noindent\includegraphics[width=300bp]{{praticando-feixe-de-retas-03}.jpg}
\end{figure}

\item {} 
Na figura a seguir, (1) a reta \(d\) que passa pelo centro do quadrado é perpendicular ao plano \(\pi\) e (2) o quadrado é paralelo a \(\pi\). Como vimos, para determinar a projeção paralela do quadrado com relação a direção dada por \(d\) sobre o plano de projeção \(\pi\), é necessário construir retas que passam por pontos do quadrado e que são paralelas a \(d\). Se desenharmos todas estas retas, que tipo de superfície será obtida?

\begin{figure}[H]
\centering

\noindent\includegraphics[width=300bp]{{praticando-feixe-de-retas-04}.jpg}
\end{figure}

\end{itemize}

\end{enumerate}
\end{task}

\phantomsection\label{\detokenize{GE301-4:ativ-proj-cone-cilindro}}
\begin{task}{Projetando curvas que estão sobre um cone e um cilindro}

\begin{enumerate}
\item {} 
(Cone) As três imagens a seguir exibem três curvas diferentes, mas que possuem uma característica em comum: elas estão sobre um mesmo cone circular reto cuja base é paralela ao plano \(\pi\). Para sua comodidade, em cada imagem, a curva é desenhada sem e com o cone. Caso tenha acesso a Internet (inclusive de um celular), você pode interagir com essas curvas e visualizá-las de pontos de vista diferentes por meio do aplicativo GeoGebra disponível em: \textless{}\url{https://www.geogebra.org/m/NNjgC2Aj}\textgreater{}.


\centering
\begin{figure}[H]
\centering

\noindent\includegraphics[width=250bp]{{perspectiva-varios-01}.jpg}
\end{figure}

\begin{figure}[H]
\centering

\noindent\includegraphics[width=250bp]{{perspectiva-varios-02}.jpg}
\end{figure}


\begin{figure}[H]
\centering

\noindent\includegraphics[width=250bp]{{perspectiva-varios-03}.jpg}
\end{figure}


\begin{itemize}
\item {} 
Qual é a projeção em perspectiva destas três curvas sobre o plano \(\pi\) com relação ao centro \(O\)? Justifique sua resposta!

\item {} 
Usando a analogia de pintura que funciona como uma janela (conforme o que vimos com relação à \hyperref[\detokenize{GE301-3:fig-proj-janela-de-alberti-01}]{Figura \ref{\detokenize{GE301-3:fig-proj-janela-de-alberti-01}}} e à \hyperref[\detokenize{GE301-3:fig-proj-janela-de-alberti-03}]{Figura \ref{\detokenize{GE301-3:fig-proj-janela-de-alberti-03}}}), se você pintasse um quadro para cada uma das três curvas, tendo o ponto \(O\) como a posição do olho do observador, o que seria pintado nos três quadros?

\item {} 
Qual é a projeção em perspectiva de uma reta que passa por \(O\) sobre o cone com relação ao centro \(O\) sobre o plano \(\pi\)? Justifique sua resposta!

\item {} 
Qual é a projeção em perspectiva do próprio cone com relação ao centro \(O\) sobre o plano \(\pi\)? Justifique sua resposta!

\end{itemize}

\item {} 
(Cilindro) As três imagens a seguir exibem três curvas diferentes, mas que possuem uma característica em comum: elas estão sobre um mesmo cilindro circular reto cuja base é paralela ao plano \(\pi\). Para sua comodidade, em cada imagem, a curva é desenhada sem e com o cilindro. Caso tenha acesso a Internet (inclusive de um celular), você pode interagir com essas curvas e visualizá-las de pontos de vista diferentes por meio do aplicativo GeoGebra disponível em: \textless{}\url{https://www.geogebra.org/m/NrqMykdJ}\textgreater{}.

\begin{figure}[H]
\centering

\noindent\includegraphics[width=250bp]{{paralela-varios-01}.jpg}
\end{figure}

\begin{figure}[H]
\centering

\noindent\includegraphics[width=250bp]{{paralela-varios-02}.jpg}
\end{figure}
\begin{figure}[H]
\centering

\noindent\includegraphics[width=250bp]{{paralela-varios-03}.jpg}
\end{figure}
\begin{itemize}
\item {} 
Qual é a projeção paralela destas três curvas com relação à direção dada pelo eixo do cilindro sobre o plano \(\pi\)? Justifique sua resposta!

\item {} 
Qual é a projeção paralela de uma reta sobre o cilindro com relação à direção dada pelo eixo do cilindro sobre o plano \(\pi\)? Justifique sua resposta!

\item {} 
Qual é a projeção paralela do próprio cilindro com relação à direção dada pelo eixo do cilindro sobre o plano \(\pi\)? Justifique sua resposta!

\end{itemize}

\end{enumerate}
\end{task}

\phantomsection\label{\detokenize{GE301-4:ativ-proj-construindo}}
\begin{task}{Construindo objetos geométricos peculiares}

\begin{enumerate}
\item {} \begin{itemize}
\item {} 
Deseja-se pintar a palavra “ESCOLA” em uma rua para advertir os motoristas da proximidade de uma escola. Contudo, se a palavra for pintada normalmente, como na \hyperref[\detokenize{GE301-4:fig-proj-aviso-na-rua-01}]{Figura \ref{\detokenize{GE301-4:fig-proj-aviso-na-rua-01}}} (B), o motorista verá pelo para-brisa uma imagem distorcida pela perspectiva, como na \hyperref[\detokenize{GE301-4:fig-proj-aviso-na-rua-01}]{Figura \ref{\detokenize{GE301-4:fig-proj-aviso-na-rua-01}}} (C).

\end{itemize}

\begin{figure}[H]
\centering
\capstart

\noindent\includegraphics[width=350bp]{{aviso-na-rua-01_4}.jpg}
\caption{Estudo de sinalização de solo em uma rua.}\label{\detokenize{GE301-4:fig-proj-aviso-na-rua-01}}\label{\detokenize{GE301-4:id2}}\end{figure}

Como deveria ser pintada a palavra na rua para que, vista pelo para-brisa de um carro, ela fosse visualizada sem distorções, como na \hyperref[\detokenize{GE301-4:fig-proj-aviso-na-rua-02}]{Figura \ref{\detokenize{GE301-4:fig-proj-aviso-na-rua-02}}}. Aqui, é suficiente que você descreva um procedimento de como obter o desenho da palavra na rua: você não precisa efetivamente fazer o desenho da palavra.

\begin{figure}[H]
\centering
\capstart

\noindent\includegraphics[width=300bp]{{aviso-na-rua-02}.jpg}
\caption{Imagem no para-brisa sem distorções.}\label{\detokenize{GE301-4:fig-proj-aviso-na-rua-02}}\label{\detokenize{GE301-4:id3}}\end{figure}
\begin{itemize}
\item {} 
O desenho da palavra que você propôs para ser pintada na rua no item anterior seria vista \textbf{sempre} sem distorções a medida que o carro se aproxima da palavra pintada?

\end{itemize}

\item {} 
O grupo Troika tem como missão “desenvolver obras artísticas com um interesse particular na percepção e experiência espacial, desafiando prescrições de conhecimento, controle, e o que significa ser humano na era da tecnologia”. A obra “Squaring The Circle” (Quadratura do Círculo) é uma peça feita de ferro que, quando observada de um ponto de vista particular, o que se vê é um círculo e, a mesma peça, quando observada de outro ponto de vista, se mostra como um quadrado.
\phantomsection\label{\detokenize{GE301-4:fig-proj-squaring-the-circle-01}}
% \begin{figure}[H]
% \centering

% \noindent\includegraphics[width=100bp]{{ezgif-4-7c3b461d5e}.png}
% \label{\detokenize{GE301-4:fig-proj-squaring-the-circle-01}}\end{figure}

\begin{figure}[H]
\centering
\capstart

\noindent\includegraphics[width=400bp]{{gif_1}.jpg}
\caption{Squaring The Circle (Quadratura do Círculo) do grupo Troika (fonte: \textless{}\url{http://troika.uk.com}\textgreater{}.}\label{\detokenize{GE301-4:id4}}\end{figure}

Como construir uma tal peça? Aqui, é suficiente que você descreva um procedimento matemático de como obtê-la: você não precisa explicitar equações para o formato geométrico da peça.

\item {} 
Este é um desafio antigo e que apareceu na edição de agosto de 1958 da revista Scientific American. A \hyperref[\detokenize{GE301-4:fig-proj-cork-plug-01}]{Figura \ref{\detokenize{GE301-4:fig-proj-cork-plug-01}}} exibe uma mesa com três buracos: um na forma de um quadrado, o outro na forma de um círculo e o terceiro na forma de um triângulo isósceles. O diâmetro do círculo, o lado do quadrado, a base do triângulo isósceles e sua respectiva altura têm a mesma medida.

\begin{figure}[H]
\centering
\capstart

\noindent\includegraphics[width=250bp]{{cork-plug-table}.png}
\caption{Uma mesa com três buracos.}\label{\detokenize{GE301-4:fig-proj-cork-plug-01}}\label{\detokenize{GE301-4:id5}}\end{figure}

Pergunta: é possível construir uma rolha que possa ser usada para tapar qualquer um dos três buracos, um por vez? Em caso afirmativo, descreva um procedimento matemático de como obtê-la.

\end{enumerate}
\end{task}

\begin{reflection}{}

\begin{figure}[H]
\centering

\noindent\includegraphics[width=400bp]{{tirinha1}.jpg}
\end{figure}


\begin{figure}[H]
\centering

\noindent\includegraphics[width=350bp, trim={0 8cm 0 0}, clip]{{tirinha2}.jpg}
\end{figure}

No dicionário, a palavra "prisma" tem o sentido figurado de "modo de ver ou considerar algo, ponto de vista". 

Contudo, o modo que vemos está mais relacionado com projeções em perspectiva as quais, por sua vez, estão relacionadas em cones e pirâmides (pirâmides podem ser consideradas como um tipo especial de cone).

Assim, pelo menos do ponto de vista da teoria das projeções, falar "ver deste cone" no lugar de ver "ver deste prisma" faz mais sentido!
\end{reflection}

\begin{knowledge}{}

Uma situação semelhante a da sinalização de trânsito descrita na \DUrole{xref,std,std-ref}{ativ-proj-construindo} é a confecção de paineis de propaganda em gramados de campos de futebol. Se eles forem desenhados sem distorções, suas imagens transmitidas pelas emissoras de TV ficarão distorcidas. Assim, para que a imagem fique correta quando observada pela câmera de TV, sua projeção em perspectiva é que deve ser desenhada no gramado.

\begin{figure}[H]
\centering
\capstart

\noindent\includegraphics[width=300bp]{{futebol1}.jpg}
\caption{Campo de Futebol}\label{\detokenize{GE301-4:id12}}\end{figure}

\begin{figure}[H]
\centering
\capstart

\noindent\includegraphics[width=300bp]{{futebol2}.jpg}
\caption{Perspectiva do campo}\label{\detokenize{GE301-4:id13}}\end{figure}

Note como a projeção depende da posição do observador: enquanto a câmera de TV transmite uma imagem sem distorções do painel de propaganda, uma pessoa sentada junto ao painel o verá bem distorcido.

Em artes plásticas, esta imagem distorcida que é vista corretamente de um certo ponto de vista é denominada \index{anamorfose}anamorfose. A palavra vem do Grego: \emph{ana} (de volta, de novo) e \emph{morphe} (forma). Além de distorções provocadas por projeções em perspectiva, a anamorfose inclui também distorções via espelhos cilíndricos, cônicos e piramidais.

Um exemplo clássico de anamorfose é dado pelo quadro “Os Embaixadores” (1533) do artista alemão Hans Holbein, O Jovem (1497/1498-1543). Você consegue identificar a parte do quadro em anamorfose?

\begin{figure}[H]
\centering
\capstart

\noindent\includegraphics[width=300bp]{hans.jpg}
\caption{“Os Embaixadores” de Hans Holbein, O Jovem (fonte: \href{https://en.wikipedia.org/wiki/File:Hans\_Holbein\_the\_Younger\_-\_The\_Ambassadors\_-\_Google\_Art\_Project.jpg}{Wikimedia Commons}).}\label{\detokenize{GE301-4:id14}}

\end{figure}

Anamorfose também já foi usada para esconder imagens sensíveis, como as imagens produzidas pelo artista alemão Erhard Schön (c. 1491\textendash{}1542). Você consegue identificar o que está representado em anamorfose?

\begin{figure}[H]
\centering
\capstart

\noindent\includegraphics[width=400bp]{{schon-02}.jpg}
\caption{“Aus, du alter Tor!” de Erhard Schön (fonte: \href{http://www.exploramuseum.de/images/pressefotos/anamorphoseAUSDUALTERTOR1\_m.jpg}{Explora Museum}).}
\label{\detokenize{GE301-4:id15}}\end{figure}

\begin{figure}[H]
\centering
\capstart

\noindent\includegraphics[width=400bp]{{schon-03}.jpg}
\caption{“Was siehst du?” de Erhard Schön (fonte: \href{http://www.britishmuseum.org/research/collection\_online/collection\_object\_details/collection\_image\_gallery.aspx?partid=1\&assetid=30265001\&objectid=1355159}{The British Museum}).}
\label{\detokenize{GE301-4:id16}}\end{figure}

Um belo exemplo de uso artístico da anamorfose no Brasil é o projeto “Luz nas Vielas” do grupo espanhol Boa Mistura que pintou, junto com os moradores da Vila Brasilândia em São Paulo, palavras como “firmeza”, “amor”, “doçura” nas paredes das vielas do bairro. Para conhecer mais sobre o projeto, acesse o vídeo \href{https://www.youtube.com/watch?v=Zi8ekDi7uLQ}{Poesia e Magia} no YouTube ou a \href{http://www.boamistura.com/\#/project/luz-nas-vielas-2}{página oficial do grupo}.

\begin{figure}[H]
\centering
\capstart

\noindent\includegraphics[width=400bp]{{boa-mistura-01}.jpg}
\caption{Anamorfose do projeto “Luz nas Vielas” do grupo Boa Mistura (fonte: \href{https://www.youtube.com/watch?v=gKRNLXghU94}{TEDx Talks})}\label{\detokenize{GE301-4:id17}}\end{figure}

Quer gerar suas próprias anamorfoses? Aqui estão dois softwares gratuitos que fazem isso a partir de uma imagem digital (arquivo jpg) de sua escolha: o \href{http://kejebodo.blogspot.com.br/2013/06/simple-anamorphic-converter.html}{Simple Anamorphic Converter} (distorções via projeções em perspectiva) e o \href{https://www.anamorphosis.com/software.html}{Anamorph Me!} (distorções via projeções paralelas, cilíndricas e cônicas).

\begin{figure}[H]
\centering
\capstart

\noindent\includegraphics[width=400bp]{{anamorfose-02}.jpg}
\caption{Brincando com anamorfose.}\label{\detokenize{GE301-4:id18}}\end{figure}
\end{knowledge}

\begin{knowledge}{}

Existem muitas produções artísticas que produzem o efeito de múltiplas projeções com múltiplos signifcados como visto na \DUrole{xref,std,std-ref}{ativ-proj-construindo}. Indicamos aqui duas referências: as esculturas do artista \href{https://www.jvmuntean.com/\#intro}{John V. Muntean} e do matemático \href{http://home.mims.meiji.ac.jp/~sugihara/Welcomee.html}{Kokichi Sugihara}.

Caso você queira construir uma versão simples de uma destas peças, um molde para ser impresso e recortado está disponível \href{https://goo.gl/ddFnuf}{neste endereço}. Um vídeo exibindo as etapas de montagem pode ser acessado no \href{https://youtu.be/QTNg0ofgB78}{YouTube}. Caso você tenha acesso a uma impressora 3D, o arquivo STL para impressão podem ser obtido gratuitamente no \href{https://www.thingiverse.com/thing:1657791}{Thingverse}.

% \begin{figure}[H]
% \centering
% \capstart

% \noindent\includegraphics{{sugihara-richeson}.png}
% \caption{Uma peça peculiar feita de papel (fonte: \href{https://youtu.be/QTNg0ofgB78}{David Recheson}).}\label{\detokenize{GE301-4:id19}}\end{figure}
\end{knowledge}


\practice{}
\label{\detokenize{GE301-5::doc}}\label{\detokenize{GE301-5:praticando-2}}\phantomsection\label{\detokenize{GE301-5:ativ-proj-infinito}}
\begin{task}{Retas paralelas, obras de arte e o infinito}

Você já percebeu em uma estrada ou em um corredor comprido (\hyperref[\detokenize{GE301-5:fig-proj-infinito-01}]{Figura \ref{\detokenize{GE301-5:fig-proj-infinito-01}}}) que elementos da cena que são paralelos como as linhas do acostamento ou as linhas das paredes não são vistos como paralelos e parecem convergir para um ponto? Nesta atividade, veremos como este fenômeno é explicado pelas projeções em perspectiva.

\begin{figure}[H]
\centering
\capstart

\noindent\includegraphics[width=400bp]{{infinito-01}.jpg}
\caption{Corredores paralelos (fonte: PEXELS e Wikimedia Commons).}\label{\detokenize{GE301-5:fig-proj-infinito-01}}\label{\detokenize{GE301-5:id1}}\end{figure}

Este tipo de situação é traduzido pela frase popular “Retas paralelas se encontram no infinito!”. Note, contudo, que as retas paralelas na cena tridimensional \textbf{nunca} se encontram. A concorrência ocorre para os prolongamentos das projeções em perspectiva de retas paralelas que não são paralelas ao plano de projeção.

\textbf{PARTE 1.}

Vamos primeiro compreender como a projeção em perspectiva de uma reta não paralela ao plano de projeção pode ser obtida por meio da interseção de dois planos. Caso queira acompanhar os passos descritos a seguir com um modelo interativo que pode ser girado e ampliado, acesse (inclusive do seu celular) o endereço: \textless{}\url{https://www.geogebra.org/m/pcx56y49}\textgreater{}.

Como na \hyperref[\detokenize{GE301-5:fig-proj-infinito-02}]{Figura \ref{\detokenize{GE301-5:fig-proj-infinito-02}}}, considere uma projeção em perspectiva determinada por um centro \(O\) e um plano de projeção \(\pi\). Suponha que o ponto \(O\) \emph{não pertença} ao plano \(\pi\). Considere também uma reta \(r\) não paralela ao plano \(\pi\) e que não passa pelo ponto \(O\).

\begin{figure}[H]
\centering
\capstart

\noindent\includegraphics[width=300bp]{{infinito-02_2}.jpg}
\caption{Projeção em perspectiva de uma reta: passo 1.}\label{\detokenize{GE301-5:fig-proj-infinito-02}}\label{\detokenize{GE301-5:id2}}\end{figure}

Para determinar a projeção em perspectiva de uma reta \(r\), devemos, de acordo com a definição, para cada ponto de \(r\), determinar a interseção da reta que passa pelo ponto e o centro \(O\) com o plano \(\pi\). A \hyperref[\detokenize{GE301-5:fig-proj-infinito-03}]{Figura \ref{\detokenize{GE301-5:fig-proj-infinito-03}}}     exibe as projeções \(A'\), \(B'\) e \(C'\) dos pontos \(A\), \(B\) e \(C\) da reta \(r\).

\begin{figure}[H]
\centering
\capstart

\noindent\includegraphics[width=300bp]{{infinito-03_3}.jpg}
\caption{Projeção em perspectiva de uma reta: passo 2.}\label{\detokenize{GE301-5:fig-proj-infinito-03}}\label{\detokenize{GE301-5:id3}}\end{figure}

Observe que os pontos da reta \(r\) projetados pertencem ao plano \(\pi\) e, também, ao plano \(\phi\) que passa por \(O\) e contém a reta \(r\), conforme a \hyperref[\detokenize{GE301-5:fig-proj-infinito-04}]{Figura \ref{\detokenize{GE301-5:fig-proj-infinito-04}}}.

\begin{figure}[H]
\centering
\capstart

\noindent\includegraphics[width=300bp]{{infinito-04}.jpg}
\caption{Projeção em perspectiva de uma reta: passo 3.}\label{\detokenize{GE301-5:fig-proj-infinito-04}}\label{\detokenize{GE301-5:id4}}\end{figure}

Em particular, os pontos da reta \(r\) projetados sobre o plano \(\pi\) pertencem à interseção \(r'\) dos dois planos \(\pi\) e \(\phi\).

\begin{figure}[H]
\centering
\capstart

\noindent\includegraphics[width=300bp]{{infinito-05_3}.jpg}
\caption{Projeção em perspectiva de uma reta: passo 4.}\label{\detokenize{GE301-5:fig-proj-infinito-05}}\label{\detokenize{GE301-5:id5}}\end{figure}

Existe uma reta \(s\) que passa por \(O\) e é paralela a reta \(r\). Essa reta pertence ao plano \(\phi\) e intersectará o plano \(\pi\) e, portanto, a reta \(r'\) em um ponto \(F\), conforme a \hyperref[\detokenize{GE301-5:fig-proj-infinito-06}]{Figura \ref{\detokenize{GE301-5:fig-proj-infinito-06}}}. Neste contexto, o ponto \(F\) é denominado \index{ponto de fuga}ponto de fuga associado à direção dada pela reta \(r\).

\begin{figure}[H]
\centering
\capstart

\noindent\includegraphics[width=300bp]{{infinito-06_1}.jpg}
\caption{Projeção em perspectiva de uma reta: passo 5.}\label{\detokenize{GE301-5:fig-proj-infinito-06}}\label{\detokenize{GE301-5:id6}}\end{figure}

\textbf{Pergunta 1.} Não existe ponto algum da reta \(r\) cuja projeção sobre o plano \(\pi\) seja o ponto \(F\). Por quê?

Note que, em particular, a projeção da reta \(r\) \textbf{não é} uma reta mas, sim, uma reta menos um ponto: \(r' - \{F\}\). Se prolongarmos esta projeção incluindo o ponto \(F\), obteremos uma reta: \(r'\).

\textbf{Pergunta 2.} Suponha no que foi feito até agora, a reta \(r\) seja trocada por uma outra reta diferente, mas paralela a reta \(r\). O ponto \(F\) para esta nova reta será o mesmo, não mudará. Por quê?

\textbf{Pergunta 3.} Supomos inicialmente que a reta \(r\) não passa pelo ponto \(O\). O que mudaria no que foi feito se \(O\) fosse um ponto de \(r\)?

Considere agora o caso de duas ou mais retas paralelas que não são paralelas ao plano de projeção \(\pi\). Em decorrência do que foi estabelecido nas Perguntas 1 e 2, sabemos que suas projeções sobre o plano \(\pi\) são retas menos um mesmo ponto \(F\) e que, se prolongássemos essas projeções, obteríamos retas concorrentes no ponto \(F\). A \hyperref[infinito-08]{Figura \ref{infinito-08}} e a \hyperref[infinito-09]{Figura \ref{infinito-09}} ilustram essa propriedade.
Isto também explica a “convergência”
das linhas do acostamento e das linhas das paredes na \hyperref[\detokenize{GE301-5:fig-proj-infinito-01}]{Figura \ref{\detokenize{GE301-5:fig-proj-infinito-01}}}.
\phantomsection\label{\detokenize{GE301-5:fig-proj-infinito-08}}
\begin{figure}[H]
\centering

\noindent\includegraphics[width=300bp]{{infinito-08}.jpg}
\caption{Projeção em perspectiva de um feixe com 3 retas paralelas (\textless{}\url{https://www.geogebra.org/m/EAuVTyTG}\textgreater{}).}
\label{infinito-08}

\end{figure}

\begin{figure}[H]
\centering
\capstart

\noindent\includegraphics[width=300bp]{{infinito-09_1}.jpg}
\caption{Projeção em perspectiva de um feixe com muitas retas paralelas (\textless{}\url{https://www.geogebra.org/m/ycxHtZEP}\textgreater{}).}\label{infinito-09}\end{figure}

\textbf{Pergunta 4.} Qual é a projeção em perspectiva de um feixe de retas paralelas que são paralelas ao plano de projeção? Faça uma conjectura e justifique-a!

\textbf{PARTE 2}

\textbf{Pergunta 1.} O desenvolvimento feito na PARTE 1 trata de projeções em perspectiva de retas paralelas. O que pode ser dito sobre \emph{projeções paralelas de retas paralelas}? Faça uma conjectura e justifique-a!

\textbf{PARTE 3.}

Caso uma pintura queira retratar a realidade segundo a metáfora da janela de Alberti (\hyperref[\detokenize{GE301-3:fig-proj-janela-de-alberti-03}]{Figura \ref{\detokenize{GE301-3:fig-proj-janela-de-alberti-03}}}), os elementos desenhados devem respeitar as propriedades das projeções em perspectiva. Em particular, segmentos de retas que são paralelos na cena tridimensional devem ser desenhados como segmentos de reta cujos prolongamentos se encontram em um ponto de fuga.

Você receberá reproduções de pinturas, desenhos e fotos do seu professor (as imagens a seguir dão alguns exemplos).

\begin{figure}[H]
\centering
\includegraphics[width=300bp]{{ponto-de-fuga-01_2}.jpg}
\caption{A Última Ceia de Leonardo da Vinci (1452-1519) (fonte: Wikimedia Commons).}\label{\detokenize{GE301-5:fig-proj-ponto-de-fuga-01}}\label{\detokenize{GE301-5:id8}}\end{figure}

\begin{figure}[H]
\centering
\includegraphics[width=300bp]{{ponto-de-fuga-02}.jpg}
\caption{Linha de trem na cidade de Orléans na França (fonte: Wikimedia Commons).}
\end{figure}

\begin{figure}[H]
\centering
\includegraphics[width=300bp]{{ponto-de-fuga-03}.jpg}
\caption{Pintura japonesa em papel do século XIII (fonte: \href{http://faculty.philosophy.umd.edu/jhbrown/digitaltech/index.html}{University of Maryland}).}\label{\detokenize{GE301-5:fig-proj-ponto-de-fuga-03}}\label{\detokenize{GE301-5:id10}}
\end{figure}

\begin{figure}[H]
\centering
\includegraphics[width=200bp]{{ponto-de-fuga-04}.jpg}
\caption{Ícone bizantino na Igreja de São Clemente em Ohrid, República da Macedônia (fonte: Wikimedia Commons).}\label{\detokenize{GE301-5:fig-proj-ponto-de-fuga-04}}\label{\detokenize{GE301-5:id11}}
\end{figure}

\begin{figure}[H]
\centering


\includegraphics[width=150bp]{{ponto-de-fuga-07}.jpg}
\caption{O Casal Arnolfini do pintor flamengo Jan van Eyck (1390-1441) (fonte: Wikimedia Commons).}\label{\detokenize{GE301-5:fig-proj-ponto-de-fuga-05}}\label{\detokenize{GE301-5:id12}}
\end{figure}

\begin{figure}[H]
\centering


\includegraphics[width=300bp]{{ponto-de-fuga-09}.jpg}
\caption{Palácio da Assembleia em Chandigarh na Índia (fonte: Wikimedia Commons).}\label{\detokenize{GE301-5:fig-proj-ponto-de-fuga-09}}\label{\detokenize{GE301-5:id13}}\end{figure}


Em cada uma delas, você deve identificar quais são os elementos que são supostamente paralelos na cena tridimensional sendo registrada (contornos de paredes, ladrilhos, etc.) e, então, desenhar segmentos de reta nesses elementos da imagem. Aqui está um exemplo.


\begin{figure}[H]
\centering

\noindent\includegraphics[width=400bp]{{ponto-de-fuga-08_1}.jpg}
\end{figure}

\end{task}

\begin{reflection}{}

\begin{figure}[H]
\centering

\noindent\includegraphics[width=400bp]{{tirinha3}.jpg}
\end{figure}


\begin{figure}[H]
\centering
\capstart

\noindent\includegraphics[width=400bp]{{calvin-haroldo-perspectiva}.jpg}
\caption{Calvin, Haroldo e Perspectiva!}\label{\detokenize{GE301-5:id32}}\end{figure}
\end{reflection}
\phantomsection\label{\detokenize{GE301-5:ativ-proj-comprimentos}}
\begin{task}{Comprimentos em projeções}

\textbf{PARTE 1}

Coloque os seus dois dedos indicadores um do lado do outro. Eles têm o mesmo tamanho, não é? Agora afaste um deles. Seus dedos continuam com o mesmo tamanho? Sim, mas você \textbf{vê} o dedo que está mais longe menor, não é? Esta propriedade pode ser explicada via projeções em perspectiva e a exploraremos nesta atividade.

\begin{figure}[H]
\centering
\capstart

\noindent\includegraphics[width=400bp]{{dedos-03}.jpg}
\caption{Visualização de imagens dos dois dedos indicadores.}\label{\detokenize{GE301-5:fig-proj-dedos-03}}\label{\detokenize{GE301-5:id33}}\end{figure}

Vamos começar com uma configuração bem simples. Considere a \hyperref[\detokenize{GE301-5:fig-proj-comprimento-01}]{Figura \ref{\detokenize{GE301-5:fig-proj-comprimento-01}}}. Nela, o segmento \(AB\) é paralelo ao plano de projeção \(\pi\) e o segmento \(OA\), por sua vez, é perpendicular a \(\pi\). Os pontos \(A'\) e \(B'\) são, respectivamente, as projeções de \(A\) e \(B\) sobre o plano \(\pi\) com relação ao centro \(O\). Considere as medidas de comprimento \(h = AB\), \(x = OA\), \(h' = A'B'\) e \(d = OA'\). Nosso objetivo é estudar como o comprimento \(h'\) da projeção sobre o plano \(\pi\) se relaciona com o comprimento \(h\) do segmento \(AB\) (você pode imaginar que \(h\) é o comprimento real do seu dedo e \(h'\) é o comprimento da imagem que você vê de seu dedo quando ele está a uma distância \(x\)).
\begin{quote}

\begin{figure}[H]
\centering
\capstart

\noindent\includegraphics[width=300bp]{{perspectiva-comprimento-01_1}.jpg}
\caption{Configuração geométrica simples (versão interativa: \textless{}\url{https://www.geogebra.org/m/HKuqwxXn}\textgreater{}).}\label{\detokenize{GE301-5:fig-proj-comprimento-01}}\label{\detokenize{GE301-5:id34}}\end{figure}
\end{quote}

\needspace{.15\textheight}

\textbf{Etapa 1.}

Considere que \(h = 2\) e \(d = 3\).
\begin{enumerate}
\item {} 
Determine o valor de \(h'\) para \(x = 6\).

\item {} 
Mais geralmente, determine \(h'\) como uma função \(f\) de \(x\). Qual é o domínio desta função? Note que, usando o conceito de função, o item anterior está lhe pedindo para calcular \(f(6)\).

\item {} 
Qual deve ser o valor de \(x\) para que o valor de \(h\)‘ correspondente seja igual à metade do valor de \(h'\) que você obteve no primeiro item? Em outras palavras, qual é o valor de \(x\) para o qual \(f(x) = \frac{1}{2} \, f(6)\)?

\item {} 
Qual deve ser o valor de \(x\) para que o valor de \(h\)‘ correspondente seja igual ao dobro do valor de \(h'\) que você obteve no primeiro item? Em outras palavras, qual é o valor de \(x\) para o qual \(f(x) = 2 \, f(6)\)?

\item {} 
Para que valores de \(x\) tem-se \(f(x) = h\)? E \(f(x) > h\)? E \(f(x) < h\)? Interprete no contexto de visualização das imagens de seus dois dedos indicadores em analogia à \hyperref[\detokenize{GE301-5:fig-proj-dedos-03}]{Figura \ref{\detokenize{GE301-5:fig-proj-dedos-03}}}.

\item {} 
Existem valores diferentes de \(x_{1}\) e \(x_{2}\) para os quais \(f(x_{1}) = f(x_{2})\)? Interprete no contexto de visualização das imagens de seus dois dedos indicadores em analogia à \hyperref[\detokenize{GE301-5:fig-proj-dedos-03}]{Figura \ref{\detokenize{GE301-5:fig-proj-dedos-03}}}.

\item {} 
Se os valores de \(x\) vão ficando arbitrariamente grandes, o que se pode dizer a respeito dos valores de \(h'\) correspondentes? Interprete no contexto de visualização das imagens de seus dois dedos indicadores em analogia à \hyperref[\detokenize{GE301-5:fig-proj-dedos-03}]{Figura \ref{\detokenize{GE301-5:fig-proj-dedos-03}}}.

\item {} 
Se os valores de \(x\) vão ficando arbitrariamente próximos de \(0\) com valores maiores do que \(0\), o que se pode dizer a respeito dos valores de \(h'\) correspondentes? Interprete no contexto de visualização das imagens de seus dois dedos indicadores em analogia à \hyperref[\detokenize{GE301-5:fig-proj-dedos-03}]{Figura \ref{\detokenize{GE301-5:fig-proj-dedos-03}}}.

\item {} 
Deseja-se construir um segmento \(CD\) cuja projeção em perspectiva sobre o plano \(\pi\) com relação ao centro \(O\) também seja o segmento \(A'B'\), mas cuja distância até \(O\) seja igual a 15. Qual deve ser o comprimento do segmento \(CD\)?

\end{enumerate}

Justifique todas as respostas!

\needspace{.15\textheight}

\textbf{Etapa 2.}
\begin{enumerate}
\item {} 
Generalize o Item b) da Etapa 1: determine \(h'\) como função de \(x\) em termos de \(h\) e \(d\) (isto é, sem especificar valores numéricos particulares para \(h\) e \(d\)).

\item {} 
Verdadeiro ou falso? No contexto da \hyperref[\detokenize{GE301-5:fig-proj-comprimento-01}]{Figura \ref{\detokenize{GE301-5:fig-proj-comprimento-01}}}, sem atribuir valores numéricos específicos para \(h\) e \(d\), verdadeiro ou falso? Se dobrarmos a distância \(x\) do segmento \(AB\) até o ponto \(O\), então o comprimento \(h'\) de sua projeção ficará reduzido à metade.

\end{enumerate}

Justifique todas as respostas!

\needspace{.10\textheight}

\textbf{Etapa 3.}
\begin{enumerate}
\item {} 
A \hyperref[\detokenize{GE301-5:fig-proj-comprimento-02}]{Figura \ref{\detokenize{GE301-5:fig-proj-comprimento-02}}} foi construída a partir da \hyperref[\detokenize{GE301-5:fig-proj-comprimento-01}]{Figura \ref{\detokenize{GE301-5:fig-proj-comprimento-01}}} acrescentando-se um segmento \(RS\) que é uma “cópia” do segmento \(AB\) obtida transladando-se o segmento \(AB\) paralelamente ao plano \(\pi\). Mais precisamente, \(RS\) é tal que \(ARSB\) é um retângulo que é paralelo ao plano \(\pi\). O segmento \(R'S'\) é a projeção em perspectiva do segmento \(RS\) sobre o plano \(\pi\) com relação ao centro \(O\). Pergunta: o comprimento do segmento \(R'S'\) é maior, menor ou igual ao comprimento \(h'\) do segmento \(A'B'\) que é projeção do segmento \(AB\)? Interprete no contexto de visualização das imagens de seus dois dedos indicadores em analogia à \hyperref[\detokenize{GE301-5:fig-proj-dedos-03}]{Figura \ref{\detokenize{GE301-5:fig-proj-dedos-03}}}.

\begin{figure}[H]
\centering
\capstart

\noindent\includegraphics[width=280bp]{{perspectiva-comprimento-02}.jpg}
\caption{Uma variação da \hyperref[\detokenize{GE301-5:fig-proj-comprimento-01}]{Figura \ref{\detokenize{GE301-5:fig-proj-comprimento-01}}} (versão interativa: \textless{}\url{https://www.geogebra.org/m/u4mkzbmP}\textgreater{}).}\label{\detokenize{GE301-5:fig-proj-comprimento-02}}\label{\detokenize{GE301-5:id35}}\end{figure}

\item {} 
E se, agora, ao invés de um retângulo, o quadrilátero \(ARSB\) fosse um paralelogramo qualquer paralelo ao plano \(\pi\)? Pergunta: o comprimento do segmento \(R'S'\) é maior, menor ou igual ao comprimento \(h'\) do segmento \(A'B'\) que é projeção do segmento \(AB\)?  Interprete no contexto de visualização das imagens de seus dois dedos indicadores em analogia à \hyperref[\detokenize{GE301-5:fig-proj-dedos-03}]{Figura \ref{\detokenize{GE301-5:fig-proj-dedos-03}}}.

\begin{figure}[H]
\centering
\capstart

\noindent\includegraphics[width=280bp]{{perspectiva-comprimento-03}.jpg}
\caption{Outra variação da \hyperref[\detokenize{GE301-5:fig-proj-comprimento-01}]{Figura \ref{\detokenize{GE301-5:fig-proj-comprimento-01}}} (versão interativa: \textless{}\url{https://www.geogebra.org/m/UGFWgAQ5}\textgreater{}).}\label{\detokenize{GE301-5:fig-proj-comprimento-03}}\label{\detokenize{GE301-5:id36}}\end{figure}

\item {} 
Nos dois itens anteriores, o segmento \(RS\) foi considerado como paralelo ao segmento \(AB\). Vamos relaxar esta hipótese, considerando que \(RS\) não precisa ser paralelo a \(AB\), mas que (1) \(RS\) tem o mesmo comprimento \(h\) de \(AB\), (2) \(R = A\) e (3) \(RS\) está contido no plano \(\omega\) que é paralelo a \(\pi\) e que passa por \(A\). Neste caso,  o comprimento do segmento \(R'S'\) é maior, menor ou igual ao comprimento \(h'\) do segmento \(A'B'\) que é projeção do segmento \(AB\)? Interprete no contexto de visualização das imagens de seus dois dedos indicadores em analogia à \hyperref[\detokenize{GE301-5:fig-proj-dedos-03}]{Figura \ref{\detokenize{GE301-5:fig-proj-dedos-03}}}.
\begin{quote}

\begin{figure}[H]
\centering
\capstart

\noindent\includegraphics[width=300bp]{{perspectiva-comprimento-05_1}.jpg}
\caption{Outra variação da \hyperref[\detokenize{GE301-5:fig-proj-comprimento-01}]{Figura \ref{\detokenize{GE301-5:fig-proj-comprimento-01}}} (versão interativa: \textless{}\url{https://www.geogebra.org/m/SubrgSmG}\textgreater{}).}\label{\detokenize{GE301-5:fig-proj-comprimento-05}}\label{\detokenize{GE301-5:id37}}\end{figure}
\end{quote}

\item {} 
Vamos generalizar um pouco mais: agora, \(RS\) é um segmento qualquer que satisfaz duas condições: (1) seu comprimento é igual ao comprimento \(h\) do segmento \(AB\) e (2) \(RS\) está contido no plano \(\omega\) que é paralelo a \(\pi\) e que passa por \(A\). Neste caso, o comprimento do segmento \(R'S'\) é maior, menor ou igual ao comprimento \(h'\) do segmento \(A'B'\) que é projeção do segmento \(AB\)?  Interprete no contexto de visualização das imagens de seus dois dedos indicadores em analogia à \hyperref[\detokenize{GE301-5:fig-proj-dedos-03}]{Figura \ref{\detokenize{GE301-5:fig-proj-dedos-03}}}.

\begin{figure}[H]
\centering
\capstart

\noindent\includegraphics[width=300bp]{{perspectiva-comprimento-04_2}.jpg}
\caption{Ainda outra variação da \hyperref[\detokenize{GE301-5:fig-proj-comprimento-01}]{Figura \ref{\detokenize{GE301-5:fig-proj-comprimento-01}}} (versão interativa: \textless{}\url{https://www.geogebra.org/m/Du4285XX}\textgreater{}).}\label{\detokenize{GE301-5:fig-proj-comprimento-04}}\label{\detokenize{GE301-5:id38}}\end{figure}

\item {} 
Verdadeiro ou falso? Se \(RS\) é um segmento que é paralelo ao plano de projeção \(\pi\), então sua projeção sobre \(\pi\) com relação a um centro \(O\) depende apenas de dois números: a distância \(d\) de \(O\) ao plano \(\pi\) e da distância \(x\) de \(O\) ao plano \(\omega\) que é paralelo a \(\pi\) e que passa por \(R\).

\end{enumerate}

Justifique todas as respostas!

\textbf{Etapa 4.}

As Etapas 1, 2 e 3 trataram da relação entre os comprimentos de segmentos de retas paralelos ao plano de projeção e os comprimentos de suas \emph{projeções em perspectiva} nesse plano. O que dizer de projeções paralelas? Isto é, qual é a relação entre os comprimentos de segmentos de retas paralelos ao plano de projeção e os comprimentos de suas \emph{projeções paralelas} nesse plano? Faça uma conjectura e justifique-a!

\textbf{PARTE 2}
\begin{quote}
\end{quote}

Tendo em mente a metáfora da janela de Alberti (\hyperref[\detokenize{GE301-3:fig-proj-janela-de-alberti-03}]{Figura \ref{\detokenize{GE301-3:fig-proj-janela-de-alberti-03}}}), um problema que desafiou artistas, especialmente os renascentistas, foi o de desenhar ladrilhamentos e tabuleiros de xadrez. A \hyperref[\detokenize{GE301-5:fig-proj-ladrilhos-37}]{Figura \ref{\detokenize{GE301-5:fig-proj-ladrilhos-37}}}, a \hyperref[\detokenize{GE301-5:fig-proj-xadrez-10}]{Figura \ref{\detokenize{GE301-5:fig-proj-xadrez-10}}}, a \hyperref[\detokenize{GE301-5:fig-proj-ladrilhos-18}]{Figura \ref{\detokenize{GE301-5:fig-proj-ladrilhos-18}}} e a \hyperref[\detokenize{GE301-5:fig-proj-xadrez-09}]{Figura \ref{\detokenize{GE301-5:fig-proj-xadrez-09}}} ilustram algumas tentativas. Perceba que para produzir um desenho realístico, que se pareça com uma fotografia, os comprimentos dos vários elementos do ladrilhamento e do tabuleiro devem satisfazer as propriedades das projeções em perspectiva. Estudaremos algumas destas propriedades nesta PARTE 2.
\begin{quote}

\begin{figure}[H]
\centering
\capstart

\noindent\includegraphics[width=300bp]{{ladrilhos-37-Rogier_van_der_Weyden_-_Presentation_Miniature_Chroniques_de_Hainaut_KBR_9242}.jpg}
\caption{Miniatura de Rogier van der Weyden (1399/1400 -1464) (fonte: Wikimedia Commons).}\label{\detokenize{GE301-5:fig-proj-ladrilhos-37}}\label{\detokenize{GE301-5:id39}}\end{figure}

\begin{figure}[H]
\centering
\capstart

\noindent\includegraphics[width=300bp]{{xadrez-10-Alfonso-LJ-27V}.jpg}
\caption{Miniatura do Livro dos Jogos (1283) (fonte: Wikimedia Commons).}\label{\detokenize{GE301-5:fig-proj-xadrez-10}}\label{\detokenize{GE301-5:id40}}\end{figure}

\begin{figure}[H]
\centering
\capstart

\noindent\includegraphics[width=250bp]{{ladrilhos-18-Portrait_of_an_Artist_in_His_Studio_by_Michiel_van_Musscher}.jpg}
\caption{Quadro “Retrato de Um Artista em Seu Estúdio” do pintor holandês Michiel van Musscher (1645-1705) (fonte: Wikimedia Commons).}\label{\detokenize{GE301-5:fig-proj-ladrilhos-18}}\label{\detokenize{GE301-5:id41}}\end{figure}

\begin{figure}[H]
\centering
\capstart

\noindent\includegraphics[width=300bp]{{xadrez-09-Lucas_van_Leyden_-_The_Game_of_Chess_-_WGA12919}.jpg}
\caption{Quadro “O Jogo de Xadrez” do pintor holandês Lucas van Leyden (1494-1533) (fonte: Wikimedia Commons).}\label{\detokenize{GE301-5:fig-proj-xadrez-09}}\label{\detokenize{GE301-5:id42}}\end{figure}
\end{quote}

\textbf{Etapa 1.}

Considere a \hyperref[\detokenize{GE301-5:fig-proj-ladrilhamentos-01}]{Figura \ref{\detokenize{GE301-5:fig-proj-ladrilhamentos-01}}}. Nela, há dois planos perpendiculares: o plano de projeção \(\pi\) e o plano \(\gamma\) que representa o chão. Um segmento de reta \(RS\) de comprimento \(h\) está contido no plano \(\gamma\) e ele é paralelo ao plano \(\pi\). Na figura, \(P\) é o ponto médio de \(RS\). Como de costume, o ponto \(O\) representa a posição do observador. O ponto \(U\) é a projeção ortogonal de \(O\) sobre \(\gamma\) e, portanto, \(a = OU\) é a altura do observador com relação ao plano do chão \(\gamma\). Agora, uma condição importante que irá simplificar nosso estudo: vamos supor que o ponto \(O\) é tal que o segmento \(OP\) é perpendicular ao segmento \(RS\), ou seja, o ângulo \(OPR\) é reto (na \hyperref[\detokenize{GE301-5:fig-proj-ladrilhamentos-01}]{Figura \ref{\detokenize{GE301-5:fig-proj-ladrilhamentos-01}}} ele não aparenta ser reto por conta da distorção da projeção em perspectiva usada para produzir a figura). Os pontos \(R'\), \(P'\) e \(S'\) são as projeções em perspectiva sobre o plano \(\pi\) com relação ao centro \(O\) dos pontos \(R\), \(P\) e \(S\), respectivamente. O comprimento do segmento projetado \(R'S'\) é \(h'\).
O ponto \(V\) é a interseção do plano \(\pi\) com a reta \(UP\) e, portanto, \(y = VP'\) é a altura do ponto \(P'\) com relação ao plano do chão \(\gamma\).

Como o comprimento \(h'\) do segmento projetado \(R'S'\) varia de acordo com os valores de \(d\), \(h\) e \(x\), você estudou na PARTE 1. O objetivo agora é determinar como a altura \(y\) deste segmento com relação ao plano \(\gamma\) varia de acordo com os valores de \(a\), \(d\) e \(x\). Com essas duas informações será possível criar um método para fazer desenhos em perspectiva de ladrilhamentos e tabuleiros de xadrez com precisão na configuração descrita na \hyperref[\detokenize{GE301-5:fig-proj-ladrilhamentos-01}]{Figura \ref{\detokenize{GE301-5:fig-proj-ladrilhamentos-01}}}.
\begin{quote}

\begin{figure}[H]
\centering
\capstart

\noindent\includegraphics[width=300bp]{{perspectiva-ladrilhamentos-01_2}.jpg}
\caption{Situação preliminar (versão interativa: \textless{}\url{https://www.geogebra.org/m/YjAhaCNu}\textgreater{}).}\label{\detokenize{GE301-5:fig-proj-ladrilhamentos-01}}\label{\detokenize{GE301-5:id43}}\end{figure}
\end{quote}

Suponha que \(a = 3\), \(d = 5\) e \(h = 4\).
\begin{enumerate}
\item {} 
Determine o valor de \(y\) para \(x = 6\).

\item {} 
Mais geralmente, determine \(y\) como função \(g\) de \(x\) para \(x \geq d\). A restrição de que \(x\) seja sempre maior do que ou igual a \(d\) é porque estamos interessados apenas no caso em que o plano \(\pi\) está entre o observador \(O\) e o segmento de reta \(RS\) (o caso de uma pintura).

\item {} 
Qual deve ser o valor de \(x\) para que o valor de \(y\) correspondente seja igual à metade do valor de \(y\) que você obteve no primeiro item? Em outras palavras, qual é o valor de \(x\) para o qual \(g(x) = \frac{1}{2} \, g(6)\)?

\item {} 
Qual deve ser o valor de \(x\) para que o valor de \(y\) correspondente seja igual ao dobro do valor de \(y\) que você obteve no primeiro item? Em outras palavras, qual é o valor de \(x\) para o qual \(g(x) = 2 \, g(6)\)?

\item {} 
Verdadeiro ou falso? Para todo \(x > d\), tem-se \(g(x) < a\).

\item {} 
Existem valores diferentes de \(x_{1} > d\) e \(x_{2} > d\) para os quais \(g(x_{1}) = g(x_{2})\)?

\item {} 
Se os valores de \(x\) vão ficando arbitrariamente grandes, o que se pode dizer a respeito dos valores de \(y\) correspondentes?

\end{enumerate}

Justifique todas as respostas!

\textbf{Etapa 2.}
\begin{enumerate}
\item {} 
Generalize o Item b) da Etapa 1: determine \(y\) como função de \(x\) em termos de \(a\), \(d\) e \(h\) (isto é, sem especificar valores numéricos particulares para \(a\), \(d\) e \(h\).

\item {} 
VNo contexto da \hyperref[\detokenize{GE301-5:fig-proj-ladrilhamentos-01}]{Figura \ref{\detokenize{GE301-5:fig-proj-ladrilhamentos-01}}}, sem atribuir valores numéricos específicos para \(a\), \(d\) e \(h\), verdadeiro ou falso? Se dobrarmos a distância \(x\) do segmento \(RS\) até o ponto \(U\), então a altura \(y\) com relação ao plano \(\gamma\) de sua projeção dobrará também.

\end{enumerate}

Justifique todas as respostas!

\textbf{Etapa 3.}

Desafio final: usando o que você aprendeu até agora nesta atividade, desenhe a projeção em perspectiva do quadriculado \(RABS\) na \hyperref[\detokenize{GE301-5:fig-proj-ladrilhamentos-02}]{Figura \ref{\detokenize{GE301-5:fig-proj-ladrilhamentos-02}}}. Considere \(a = 3\), \(d = 5\), \(x = 6\) e \(h = 4\). O quadrado \(RABS\) está dividido em \(4 \times 4 = 16\) quadrados menores congruentes.
\begin{quote}

\begin{figure}[H]
\centering
\capstart

\noindent\includegraphics[width=300bp]{{perspectiva-ladrilhamentos-02}.jpg}
\caption{Projeção em perspectiva de um quadriculado \(4 \times 4\) (versão interativa: \textless{}\url{https://www.geogebra.org/m/jGxrcxvw}\textgreater{}).}\label{\detokenize{GE301-5:fig-proj-ladrilhamentos-02}}\label{\detokenize{GE301-5:id44}}\end{figure}
\end{quote}

Registre sua resposta na \hyperref[\detokenize{GE301-5:fig-proj-ladrilhamentos-03}]{Figura \ref{\detokenize{GE301-5:fig-proj-ladrilhamentos-03}}} onde, para sua comodidade, já se encontra desenhada a projeção \(R'S'\) do segmento \(RS\) com relação ao centro \(O\).
\begin{quote}

\begin{figure}[H]
\centering
\capstart

\noindent\includegraphics[width=300bp]{{perspectiva-ladrilhamentos-03}.jpg}
\caption{Plano \(\pi\) com a projeção \(R'S'\) do segmento \(RS\) com relação ao centro \(O\).}\label{\detokenize{GE301-5:fig-proj-ladrilhamentos-03}}\label{\detokenize{GE301-5:id45}}\end{figure}
\end{quote}

\textbf{Etapa 4.}

As Etapas 1, 2 e 3 conduziram você a investigar \emph{projeções em perspectiva} de um quadriculado. O que pode ser disto sobre \emph{projeções paralelas} de um quadriculado como o da \hyperref[\detokenize{GE301-5:fig-proj-ladrilhamentos-02}]{Figura \ref{\detokenize{GE301-5:fig-proj-ladrilhamentos-02}}}? Faça uma conjectura e justifique-a!
\end{task}

\practice{}
\label{\detokenize{GE301-6::doc}}\label{\detokenize{GE301-6:praticando-3}}\phantomsection\label{\detokenize{GE301-6:ativ-proj-perspectiva-em-perspectiva}}
\begin{task}{Perspectiva em perspectiva}

\begin{enumerate}
\item {} 
Com o objetivo de alertar as pessoas sobre as inconsistências que podem ocorrer caso um desenho seja feito sem o conhecimento das propriedades das projeções em perspectiva, o pintor (e também cartunista, crítico social e satirista) inglês William Hogarth FRSA (1697-1764) produziu uma gravura intitulada “Sátira sobre a Falsa Perspectiva” a qual apresenta alguns exemplos intencionais de efeitos confusos e equívocos de perspectiva. Tente identificar esses elementos estranhos na gravura!

\begin{figure}[H]
\centering

\noindent\includegraphics[width=300bp]{{perspectiva-em-perspectiva-01}.jpg}
\end{figure}

\item {} 
Muitas pessoas gostam de fazer diagramas estatísticos em 3D, como na figura a seguir.

\begin{figure}[H]
\centering
\capstart

\noindent\includegraphics[width=250bp]{{perspectiva-diagrama-de-setores-01}.jpg}
\caption{Diagrama de setores 3D.}\label{\detokenize{GE301-6:fig-proj-diagrama-de-setores-01}}\label{\detokenize{GE301-6:id5}}\end{figure}
\begin{itemize}
\item {} 
Segundo este diagrama, qual item é mais frequente? A ou C? B ou D?

\item {} 
O diagrama foi construído com as seguintes frequências relativas: Item A com \(11\%\), Item B com \(42\%\), Item C com \(5\%\) e Item D com \(42\%\). Construa, com estes dados, uma versão 2D do diagrama de setores da \hyperref[\detokenize{GE301-6:fig-proj-diagrama-de-setores-01}]{Figura \ref{\detokenize{GE301-6:fig-proj-diagrama-de-setores-01}}}. Com este diagrama 2D, qual item é mais frequente? A ou C? B ou D?

\end{itemize}

\item {} 
Por que a imagem abaixo não pode ser uma projeção em perspectiva de dois blocos, isto é, dois paralelepípedos retos retângulos?

\begin{figure}[H]
\centering

\noindent\includegraphics[width=350bp]{{perspectiva-em-perspectiva-05}.jpg}
\end{figure}

\item {} 
(Grunbaum, 1985) A sociedade profissional americana \emph{Mathematical Association of America} (MAA) tem como emblema um icosaedro regular. A imagem a seguir exibe o emblema na capa de uma edição de 1984 da revista \emph{Mathematics Magazine} publicada pela MAA. O matemático Branko Grunbaum percebeu que o icosaedro da imagem não pode ser nem uma projeção em perspectiva e nem uma projeção paralela de um icosaedro regular. Por quê? Dica: tente analisar os elementos que são paralelos em um icosaedro regular tridimensional e como estes elementos estão projetados na imagem.

\begin{figure}[H]
\centering
\capstart

\noindent\includegraphics[width=200bp]{{perspectiva-em-perspectiva-04}.jpg}
\caption{Emblema  capa de uma edição de 1984 da revista \emph{Mathematics Magazine} publicada pela Mathematics Association of America.}\label{\detokenize{GE301-6:id6}}\end{figure}

\end{enumerate}
\end{task}

\phantomsection\label{\detokenize{GE301-6:ativ-proj-distratores}}

\bigskip\hrule\bigskip


\textless{}\url{https://www.geogebra.org/m/bd5f8KTg}\textgreater{}.
\begin{enumerate}
\item {} 
Simplício está estudando Geometria Espacial em um livro e se depara com a figura a seguir.
\phantomsection\label{\detokenize{GE301-6:fig-proj-distratores-1}}
\begin{figure}[H]
\centering

\begin{asy}
size(6cm);

currentprojection=orthographic(3,1.5,3.5);

triple a = (0,0,0);
triple b = (1,0,0);
triple c = (1,1,0);
triple d = (0,1,0);

triple e = (0,0,1);
triple f = (1,0,1);
triple g = (1,1,1);
triple h = (0,1,1);

triple j = (.5,1,0);

draw (a -- b -- c -- d -- cycle);
draw (e -- f -- g -- h -- cycle);
draw (a -- b -- f -- e -- cycle);
draw (c -- d -- h -- g -- cycle);


label ("H", (a), align=NW);
label ("E", (b), align=SW);
label ("F", (c), align=S);
label ("G", (d), align=SE);

label ("D", (e), align=NW);
label ("A", (f), align=NW);
label ("B", (g), align=SW);
label ("C", (h), align=NE);

real f(real x){return -x+1;}
path s1 = graph(f,-.3,1.3);

path3 d1 = path3(s1);

draw(d1);


real f(real x){return x;}
path s2 = graph(f,-.2,2);

path3 d2 = shift(0,0,1)*path3(s2);

draw(d2);

triple [] array={a,b,c,d,e,f,g,h};
for(triple i:array) {
dot(i, linewidth(1.5));
}
\end{asy}
\label{\detokenize{GE301-6:fig-proj-distratores-1}}\end{figure}

O livro diz que \(ABCDEFGH\) é uma projeção em perspectiva de um cubo e pergunta quantos pontos de interseção existem entre as retas \(BD\) e \(EG\). Simplício responde: “Pergunta fácil! Existe um único ponto de interseção entre \(GD\) e \(EG\). Este ponto \(P\) aqui, como podemos ver claramente!”.
\phantomsection\label{\detokenize{GE301-6:fig-proj-distratores-2}}
\begin{figure}[H]
\centering

\begin{asy}
size(6cm);

currentprojection=orthographic(3,1.5,3.5);

triple a = (0,0,0);
triple b = (1,0,0);
triple c = (1,1,0);
triple d = (0,1,0);

triple e = (0,0,1);
triple f = (1,0,1);
triple g = (1,1,1);
triple h = (0,1,1);

triple j = (.5,1,0);

draw (a -- b -- c -- d -- cycle);
draw (e -- f -- g -- h -- cycle);
draw (a -- b -- f -- e -- cycle);
draw (c -- d -- h -- g -- cycle);


label ("H", (a), align=NW);
label ("E", (b), align=SW);
label ("F", (c), align=S);
label ("G", (d), align=SE);

label ("D", (e), align=NW);
label ("A", (f), align=NW);
label ("B", (g), align=SW);
label ("C", (h), align=NE);

real f(real x){return -x+1;}
path s1 = graph(f,-.3,1.3);

path3 d1 = path3(s1);

draw(d1);


real f(real x){return x;}
path s2 = graph(f,-.2,2);

path3 d2 = shift(0,0,1)*path3(s2);

draw(d2);

triple [] array={a,b,c,d,e,f,g,h};
for(triple i:array) {
dot(i, linewidth(1.5));
}

dot((.86,1,.67), linewidth(2.5));

draw ((.7,1,.9) -- (.8,.98,.67), arrow=Arrow3(TeXHead2));
\end{asy}
\label{\detokenize{GE301-6:fig-proj-distratores-2}}\end{figure}

Você concorda com a resposta de Simplício? Por que sim? Por que não?

\item {} 
(Adaptado de Lellis, 2009) Na figura a seguir \(ABCDEFGH\) é uma projeção paralela de um cubo. Qual ponto está mais próximo da reta \(r = FG\)? O ponto \(P\), o ponto \(C\) ou o ponto \(Q\)?
\phantomsection\label{\detokenize{GE301-6:fig-proj-distratores-3}}
\begin{figure}[H]
\centering

\begin{asy}
size(6.5cm);

currentprojection=orthographic(3,1,1);

triple a = (0,0,0);
triple b = (1,0,0);
triple c = (1,1,0);
triple d = (0,1,0);

triple e = (0,0,1);
triple f = (1,0,1);
triple g = (1,1,1);
triple h = (0,1,1);

triple j = (.5,1,0);

triple k = (0,.3,1);
triple l = (0,1.5,1);

draw (a -- b -- c -- d -- cycle);
draw (e -- f -- g -- h -- cycle);
draw (a -- b -- f -- e -- cycle);
draw (c -- d -- h -- g -- cycle);


label ("H", (a), align=NW);
label ("E", (b), align=SW);
label ("F", (c), align=S);
label ("G", (d), align=SE);

label ("D", (e), align=NW);
label ("A", (f), align=NW);
label ("B", (g), align=SW);
label ("C", (h), align=NE);

label ("P", (k), align=N);
label ("Q", (l), align=N);

real f(real x){return 1;}
path s1 = graph(f,-3,2);

path3 d1 = path3(s1);

draw(d1);


draw((0,0,1) -- (0,1.5,1));

triple [] array={a,b,c,d,e,f,g,h,k,l};
for(triple i:array) {
dot(i, linewidth(1.5));
}

label("$r$", (-2,1,0),SE);
\end{asy}
\label{\detokenize{GE301-6:fig-proj-distratores-3}}\end{figure}

\item {} 
(Adaptado de Volkert, 2008) Eis outra pergunta do livro de Geometria Espacial que Simplício está estudando: “Existem três pontos distintos, cada um em arestas distintas de um cubo e que sejam colineares?”.

\begin{figure}[H]
\centering

\begin{asy}
size(5cm);

currentprojection=orthographic(3,1,.5);

triple a = (0,0,0);
triple b = (1,0,0);
triple c = (1,1,0);
triple d = (0,1,0);

triple e = (0,0,1);
triple f = (1,0,1);
triple g = (1,1,1);
triple h = (0,1,1);

triple j = (.5,1,0);

draw (a -- b -- c -- d -- cycle);
draw (e -- f -- g -- h -- cycle);
draw (a -- b -- f -- e -- cycle);
draw (c -- d -- h -- g -- cycle);


label ("H", (a), align=NW);
label ("E", (b), align=SW);
label ("F", (c), align=S);
label ("G", (d), align=SE);

label ("D", (e), align=NW);
label ("A", (f), align=NW);
label ("B", (g), align=SW);
label ("C", (h), align=NE);
\end{asy}

\end{figure}

Simplício pensa: “Existem sim! Eu construo o ponto médio \(M\) da aresta \(AB\) e o ponto médio \(N\) da aresta \(FG\). Trançando o segmento \(DN\), vejo que ele passa por \(M\). Pronto: os pontos \(D\), \(M\) e \(N\) são distintos, cada um está em uma aresta diferente e eles são colineares!”.

\begin{figure}[H]
\centering

\begin{asy}
size(5cm);

currentprojection=orthographic(2,1/2,.75);

triple a = (0,0,0);
triple b = (1,0,0);
triple c = (1,1,0);
triple d = (0,1,0);

triple e = (0,0,1);
triple f = (1,0,1);
triple g = (1,1,1);
triple h = (0,1,1);

triple j = (.5,1,0);

draw (a -- b -- c -- d -- cycle);
draw (e -- f -- g -- h -- cycle);
draw (a -- b -- f -- e -- cycle);
draw (c -- d -- h -- g -- cycle);

draw(e -- j);

label ("H", (a), align=NW);
label ("E", (b), align=SW);
label ("F", (c), align=S);
label ("G", (d), align=SE);

label ("D", (e), align=NW);
label ("A", (f), align=NW);
label ("B", (g), align=SW);
label ("C", (h), align=NE);

label ("N", (j), align=SE);

label ("M", (f+(0,.525,0)), align=NE);

dot(f+(0,.525,0), linewidth(1.5));

dot(a, linewidth(1.5));
dot(b, linewidth(1.5));
dot(c, linewidth(1.5));
dot(d, linewidth(1.5));
dot(e, linewidth(1.5));
dot(f, linewidth(1.5));
dot(g, linewidth(1.5));
dot(h, linewidth(1.5));
dot(j, linewidth(1.5));

\end{asy}
\end{figure}

Você concorda com a resposta de Simplício? Por que sim? Por que não?

\item {} 
(Adaptado de Fujita et al, 2017) No seu livro de Geometria Espacial, Simplício lê o enunciado de uma questão: “Na figura a seguir, \(ABCDEFGH\) é um cubo, \(M\) é ponto médio da aresta \(AE\) e \(N\) é ponto médio da aresta \(CG\). O ângulo \(MFN\) é reto? Justifique sua resposta!”.

\begin{figure}[H]
\centering
\begin{asy}
size(5cm);

currentprojection=orthographic(3,1,.5);

triple a = (0,0,0);
triple b = (1,0,0);
triple c = (1,1,0);
triple d = (0,1,0);

triple e = (0,0,1);
triple f = (1,0,1);
triple g = (1,1,1);
triple h = (0,1,1);

triple i = (1,0,.5);
triple j = (0,1,.5);

draw (i -- e -- j, dashed);
draw (i -- c -- j);


draw (a -- b -- c -- d -- cycle);
draw (e -- f -- g -- h -- cycle);
draw (a -- b -- f -- e -- cycle);
draw (c -- d -- h -- g -- cycle);

label ("H", (a), align=NW);
label ("E", (b), align=SW);
label ("F", (c), align=S);
label ("G", (d), align=SE);

label ("D", (e), align=NW);
label ("A", (f), align=NW);
label ("B", (g), align=NW);
label ("C", (h), align=NE);

label ("M", (i), align=W);
label ("N", (j), align=E);

triple [] array={a,b,c,d,e,f,g,h,i,j};
for(triple j:array){
	dot(j, linewidth(1.5));
}
\end{asy}
\end{figure}

Simplício dá como resposta “Sim, o ângulo \(MFN\) é reto!” e dá como justificativa “O quadrilátero \(DMFN\) é um quadrado, pois \(DM = MF = FN = ND\) e, sendo um quadrado, seus ângulos internos são todos retos!”.
\begin{itemize}
\item {} 
A justificativa de Simplício está correta? Justifique sua resposta!

\item {} 
A resposta de Simplício está correta? Justifique sua resposta.

\end{itemize}

\end{enumerate}

\phantomsection\label{\detokenize{GE301-6:ativ-proj-vistas-ortogonais}}
\begin{task}{Vistas ortogonais}

Em desenho técnico, uma prática comum para se representar objetos 3D (como o objeto em (A) na \hyperref[\detokenize{GE301-6:fig-proj-vistas-ortogonais-03}]{Figura \ref{\detokenize{GE301-6:fig-proj-vistas-ortogonais-03}}}) é o de de projetá-lo ortogonalmente sobre três planos que são dois a dois perpendiculares (como os planos em (B) na \hyperref[\detokenize{GE301-6:fig-proj-vistas-ortogonais-03}]{Figura \ref{\detokenize{GE301-6:fig-proj-vistas-ortogonais-03}}}).

Tipicamente, como em (C) na \hyperref[\detokenize{GE301-6:fig-proj-vistas-ortogonais-03}]{Figura \ref{\detokenize{GE301-6:fig-proj-vistas-ortogonais-03}}}, os planos são posicionados de forma a ficarem, na medida do possível, parelelos às faces do objeto 3D (isto quando, naturalmente, o objeto tem faces planas). Este tipo de escolha tem uma vantagem: as projeções das faces paralelas sobre um dos planos de projeção serão congruentes às faces originais. Em termos de desenho técnico, as projeções estarão em \index{verdadeira grandeza}verdadeira grandeza. No contexto de construção de peças e equipamentos, esta congruência é um dos motivos para o uso de projeções ortogonais, em oposição às projeções em perspectiva, para representações 2D de objetos 3D. Como faces não paralelas a um plano de projeção vão aparecer distorcidas, três planos são considerados, cada plano representando em verdadeira grandeza os elementos do objeto que lhe são paralelos.

Os três planos são denominados de \index{plano frontal}plano frontal, \index{plano horizontal}plano horizontal e \index{plano lateral}plano lateral. A atribuição de um destes nomes a um determinado plano é uma escolha arbitrária, em princípio pois, afinal, ao se girar o objeto 3D, podemos converter uma escolha de nomes em outra. As imagens (D), (E) e (F) exibem uma atribuição de nomes. É claro, se o objeto tem naturalmente uma base horizontal (por exemplo, o fundo de uma caixa), então é razoável denominar o plano paralelo a essa base de plano horizontal. Do mesmo modo, se objeto tem naturalmente uma frente, então é conveniente denominar o plano paralelo a esta frente de plano frontal. Uma explicação análoga pode ser dada para a escolha do plano lateral.
\phantomsection\label{\detokenize{GE301-6:fig-proj-vistas-ortogonais-03}}


\begin{figure}[H]
\centering

\begin{multicols}{3}

\begin{figure}[H]
\centering

\begin{asy}
size(4cm);

currentprojection=orthographic(1.5,1.5,1/2);

triple A = (.4,.4,.35);
triple B = (.4,.7,.35);
triple C = (.7,.7,.35);
triple D = (.7,.4,.35);

triple E = (A+(0,0,.1));
triple F = (B+(0,0,.1));
triple G = (E+(.1,0,0));
triple H = (F+(.1,0,0));

triple I = (C+(0,0,.3));
triple J = (H+(0,0,.2));

triple K = (D+(0,0,.4));
triple L = (G+(0,0,.3));

triple M = (K+(0,.1,0));
triple N = (L+(0,.1,0));

draw(B -- C -- I -- J -- H -- F -- cycle);

draw(C -- D -- K -- M -- I -- cycle);
draw(D -- K -- L -- G -- E -- A -- cycle);
draw(K -- L -- N -- M -- cycle);
draw(M -- N -- J -- I -- cycle);

draw (A -- B -- C -- D -- cycle);
draw (E -- F -- H -- G -- cycle);
draw (A -- B -- F -- E -- cycle);

draw(surface(B -- C -- I -- J -- H -- F -- cycle), verde*80);

draw(surface(C -- D -- K -- M -- I -- cycle), verde*80);
draw(surface(D -- K -- L -- G -- E -- A -- cycle), verde*80);
draw(surface(K -- L -- N -- M -- cycle), verde*80);
draw(surface(M -- N -- J -- I -- cycle), verde*80);

draw (surface(A -- B -- C -- D -- cycle), verde*80);
draw (surface(E -- F -- H -- G -- cycle), verde*80);
draw (surface(A -- B -- F -- E -- cycle), verde*80);

draw (surface(H -- J -- N --L -- G -- cycle), verde*80);

dot(A, linewidth(2));
dot(B, linewidth(2));
dot(C, linewidth(2));
dot(D, linewidth(2));
dot(E, linewidth(2));
dot(F, linewidth(2));
dot(G, linewidth(2));
dot(H, linewidth(2));
dot(I, linewidth(2));
dot(J, linewidth(2));
dot(K, linewidth(2));
dot(L, linewidth(2));
dot(M, linewidth(2));
dot(N, linewidth(2));

triple a = (0,0,0);
triple b = (1,0,0);
triple c = (1,1,0);
triple d = (0,1,0);

triple e = (0,0,1);
triple f = (1,0,1);
triple g = (1,1,1);
triple h = (0,1,1);

draw ((a -- b -- f -- e -- cycle), white);
draw ((a -- d -- h -- e -- cycle), white);
draw ((a -- b -- c -- d -- cycle), white);
\end{asy}
\\
(A)
\end{figure}

\begin{figure}[H]
\centering

\begin{asy}
size(4cm);

currentprojection=orthographic(1.5,1.5,1/2);

triple A = (.4,.4,.35);
triple B = (.4,.7,.35);
triple C = (.7,.7,.35);
triple D = (.7,.4,.35);

triple E = (A+(0,0,.1));
triple F = (B+(0,0,.1));
triple G = (E+(.1,0,0));
triple H = (F+(.1,0,0));

triple I = (C+(0,0,.3));
triple J = (H+(0,0,.2));

triple K = (D+(0,0,.4));
triple L = (G+(0,0,.3));

triple M = (K+(0,.1,0));
triple N = (L+(0,.1,0));

//draw(B -- C -- I -- J -- H -- F -- cycle);

//draw(C -- D -- K -- M -- I -- cycle);
//draw(D -- K -- L -- G -- E -- A -- cycle);
//draw(K -- L -- N -- M -- cycle);
//draw(M -- N -- J -- I -- cycle);

//draw (A -- B -- C -- D -- cycle);
//draw (E -- F -- H -- G -- cycle);
//draw (A -- B -- F -- E -- cycle);

//draw(surface(B -- C -- I -- J -- H -- F -- cycle), verde*80);

//draw(surface(C -- D -- K -- M -- I -- cycle), verde*80);
//draw(surface(D -- K -- L -- G -- E -- A -- cycle), verde*80);
//draw(surface(K -- L -- N -- M -- cycle), verde*80);
//draw(surface(M -- N -- J -- I -- cycle), verde*80);

//draw (surface(A -- B -- C -- D -- cycle), verde*80);
//draw (surface(E -- F -- H -- G -- cycle), verde*80);
//draw (surface(A -- B -- F -- E -- cycle), verde*80);

triple a = (0,0,0);
triple b = (1,0,0);
triple c = (1,1,0);
triple d = (0,1,0);

triple e = (0,0,1);
triple f = (1,0,1);
triple g = (1,1,1);
triple h = (0,1,1);

draw ((a -- b -- f -- e -- cycle));
draw ((a -- d -- h -- e -- cycle));
draw ((a -- b -- c -- d -- cycle));
\end{asy}
\\
(B)
\end{figure}

\begin{figure}[H]
\centering

\begin{asy}
size(4cm);

currentprojection=orthographic(1.5,1.5,1/2);

triple A = (.4,.4,.35);
triple B = (.4,.7,.35);
triple C = (.7,.7,.35);
triple D = (.7,.4,.35);

triple E = (A+(0,0,.1));
triple F = (B+(0,0,.1));
triple G = (E+(.1,0,0));
triple H = (F+(.1,0,0));

triple I = (C+(0,0,.3));
triple J = (H+(0,0,.2));

triple K = (D+(0,0,.4));
triple L = (G+(0,0,.3));

triple M = (K+(0,.1,0));
triple N = (L+(0,.1,0));

draw(B -- C -- I -- J -- H -- F -- cycle);

draw(C -- D -- K -- M -- I -- cycle);
draw(D -- K -- L -- G -- E -- A -- cycle);
draw(K -- L -- N -- M -- cycle);
draw(M -- N -- J -- I -- cycle);

draw (A -- B -- C -- D -- cycle);
draw (E -- F -- H -- G -- cycle);
draw (A -- B -- F -- E -- cycle);

draw(surface(B -- C -- I -- J -- H -- F -- cycle), verde*80);

draw(surface(C -- D -- K -- M -- I -- cycle), verde*80);
draw(surface(D -- K -- L -- G -- E -- A -- cycle), verde*80);
draw(surface(K -- L -- N -- M -- cycle), verde*80);
draw(surface(M -- N -- J -- I -- cycle), verde*80);

draw (surface(A -- B -- C -- D -- cycle), verde*80);
draw (surface(E -- F -- H -- G -- cycle), verde*80);
draw (surface(A -- B -- F -- E -- cycle), verde*80);

draw (surface(H -- J -- N --L -- G -- cycle), verde*80);


triple a = (0,0,0);
triple b = (1,0,0);
triple c = (1,1,0);
triple d = (0,1,0);

triple e = (0,0,1);
triple f = (1,0,1);
triple g = (1,1,1);
triple h = (0,1,1);

draw ((a -- b -- f -- e -- cycle));
draw ((a -- d -- h -- e -- cycle));
draw ((a -- b -- c -- d -- cycle));

dot(A, linewidth(2));
dot(B, linewidth(2));
dot(C, linewidth(2));
dot(D, linewidth(2));
dot(E, linewidth(2));
dot(F, linewidth(2));
dot(G, linewidth(2));
dot(H, linewidth(2));
dot(I, linewidth(2));
dot(J, linewidth(2));
dot(K, linewidth(2));
dot(L, linewidth(2));
dot(M, linewidth(2));
dot(N, linewidth(2));
\end{asy}
\\
(C)
\end{figure}

\end{multicols}

\begin{multicols}{3}
\begin{figure}[H]
\centering

\begin{asy}
size(4cm);

currentprojection=orthographic(1.5,1.5,1/2);

triple A = (.4,.4,.35);
triple B = (.4,.7,.35);
triple C = (.7,.7,.35);
triple D = (.7,.4,.35);

triple E = (A+(0,0,.1));
triple F = (B+(0,0,.1));
triple G = (E+(.1,0,0));
triple H = (F+(.1,0,0));

triple I = (C+(0,0,.3));
triple J = (H+(0,0,.2));

triple K = (D+(0,0,.4));
triple L = (G+(0,0,.3));

triple M = (K+(0,.1,0));
triple N = (L+(0,.1,0));

draw(B -- C -- I -- J -- H -- F -- cycle);

draw(C -- D -- K -- M -- I -- cycle);
draw(D -- K -- L -- G -- E -- A -- cycle, "carlinhos");
draw(K -- L -- N -- M -- cycle);
draw(M -- N -- J -- I -- cycle);

draw (A -- B -- C -- D -- cycle);
draw (E -- F -- H -- G -- cycle);
draw (A -- B -- F -- E -- cycle);

draw(surface(B -- C -- I -- J -- H -- F -- cycle), verde*80);

draw(surface(C -- D -- K -- M -- I -- cycle), verde*80);
draw(surface(D -- K -- L -- G -- E -- A -- cycle), verde*80);
draw(surface(K -- L -- N -- M -- cycle), verde*80);
draw(surface(M -- N -- J -- I -- cycle), verde*80);

draw (surface(A -- B -- C -- D -- cycle), verde*80);
draw (surface(E -- F -- H -- G -- cycle), verde*80);
draw (surface(A -- B -- F -- E -- cycle), verde*80);

draw (surface(H -- J -- N --L -- G -- cycle), verde*80);


triple x = (A + (0,-.4,0));
triple w = (D + (0,-.4,0));
triple p = (E + (0,-.4,0));
triple q = (G + (0,-.4,0));
triple y = (K + (0,-.4,0));
triple z = (L + (0,-.4,0));
triple u = (I + (0,-.7,0));
triple v = (J + (0,-.7,0));

draw (x -- w -- y -- z --q -- p -- cycle);
draw (u -- v);


triple a = (0,0,0);
triple b = (1,0,0);
triple c = (1,1,0);
triple d = (0,1,0);

triple e = (0,0,1);
triple f = (1,0,1);
triple g = (1,1,1);
triple h = (0,1,1);



draw ((a -- b -- f -- e -- cycle));
draw ((a -- d -- h -- e -- cycle));
draw ((a -- b -- c -- d -- cycle));

dot(A, linewidth(2));
dot(B, linewidth(2));
dot(C, linewidth(2));
dot(D, linewidth(2));
dot(E, linewidth(2));
dot(F, linewidth(2));
dot(G, linewidth(2));
dot(H, linewidth(2));
dot(I, linewidth(2));
dot(J, linewidth(2));
dot(K, linewidth(2));
dot(L, linewidth(2));
dot(M, linewidth(2));
dot(N, linewidth(2));

//draw (a -- b -- c -- d -- cycle);
//draw (e -- f -- g -- h -- cycle);
//draw (a -- b -- f -- e -- cycle);
//draw (c -- d -- h -- g -- cycle);



draw((0,.9,.5) -- (0,.6,.5), arrow=Arrow3(TeXHead2));

path3 f1 =  ((.8,0,1.05) -- (.2,0,1.05));

string txt = "Plano Frontal";
draw(labelpath(txt, subpath(f1,0,1),angle=180));
\end{asy}
\\
(D)
\end{figure}

\begin{figure}[H]
\centering

\begin{asy}
size(4cm);

currentprojection=orthographic(1.5,1.5,1/2);

triple A = (.4,.4,.35);
triple B = (.4,.7,.35);
triple C = (.7,.7,.35);
triple D = (.7,.4,.35);

triple E = (A+(0,0,.1));
triple F = (B+(0,0,.1));
triple G = (E+(.1,0,0));
triple H = (F+(.1,0,0));

triple I = (C+(0,0,.3));
triple J = (H+(0,0,.2));

triple K = (D+(0,0,.4));
triple L = (G+(0,0,.3));

triple M = (K+(0,.1,0));
triple N = (L+(0,.1,0));

draw(B -- C -- I -- J -- H -- F -- cycle);

draw(C -- D -- K -- M -- I -- cycle);
draw(D -- K -- L -- G -- E -- A -- cycle);
draw(K -- L -- N -- M -- cycle);
draw(M -- N -- J -- I -- cycle);

draw (A -- B -- C -- D -- cycle);
draw (E -- F -- H -- G -- cycle);
draw (A -- B -- F -- E -- cycle);

draw(surface(B -- C -- I -- J -- H -- F -- cycle), verde*80);

draw(surface(C -- D -- K -- M -- I -- cycle), verde*80);
draw(surface(D -- K -- L -- G -- E -- A -- cycle), verde*80);
draw(surface(K -- L -- N -- M -- cycle), verde*80);
draw(surface(M -- N -- J -- I -- cycle), verde*80);

draw (surface(A -- B -- C -- D -- cycle), verde*80);
draw (surface(E -- F -- H -- G -- cycle), verde*80);
draw (surface(A -- B -- F -- E -- cycle), verde*80);

draw (surface(H -- J -- N --L -- G -- cycle), verde*80);


triple x = (A + (0,0,-.35));
triple w = (B + (0,0,-.35));
triple p = (C + (0,0,-.35));
triple q = (D + (0,0,-.35));
triple y = (G + (0,0,-.45));
triple z = (H + (0,0,-.45));
triple u = (M + (0,0,-.75));
triple v = (N + (0,0,-.75));

draw (x -- w -- p -- q -- cycle);
draw (y -- z);
draw (u -- v);

triple a = (0,0,0);
triple b = (1,0,0);
triple c = (1,1,0);
triple d = (0,1,0);

triple e = (0,0,1);
triple f = (1,0,1);
triple g = (1,1,1);
triple h = (0,1,1);

draw ((a -- b -- f -- e -- cycle));
draw ((a -- d -- h -- e -- cycle));
draw ((a -- b -- c -- d -- cycle));

dot(A, linewidth(2));
dot(B, linewidth(2));
dot(C, linewidth(2));
dot(D, linewidth(2));
dot(E, linewidth(2));
dot(F, linewidth(2));
dot(G, linewidth(2));
dot(H, linewidth(2));
dot(I, linewidth(2));
dot(J, linewidth(2));
dot(K, linewidth(2));
dot(L, linewidth(2));
dot(M, linewidth(2));
dot(N, linewidth(2));

//draw (a -- b -- c -- d -- cycle);
//draw (e -- f -- g -- h -- cycle);
//draw (a -- b -- f -- e -- cycle);
//draw (c -- d -- h -- g -- cycle);



draw((0,.5,.9) -- (0,.5,.6), arrow=Arrow3(TeXHead2));

path3 f1 =  ((0,0.2,1.05) -- (0,0.8,1.05));

string txt = "Plano Lateral";
draw(labelpath(txt, subpath(f1,0,1),angle=270));
\end{asy}
\\
(E)
\end{figure}

\begin{figure}[H]
\centering

\begin{asy}
size(4cm);

currentprojection=orthographic(1.5,1.5,1/2);

triple A = (.4,.4,.35);
triple B = (.4,.7,.35);
triple C = (.7,.7,.35);
triple D = (.7,.4,.35);

triple E = (A+(0,0,.1));
triple F = (B+(0,0,.1));
triple G = (E+(.1,0,0));
triple H = (F+(.1,0,0));

triple I = (C+(0,0,.3));
triple J = (H+(0,0,.2));

triple K = (D+(0,0,.4));
triple L = (G+(0,0,.3));

triple M = (K+(0,.1,0));
triple N = (L+(0,.1,0));

draw(B -- C -- I -- J -- H -- F -- cycle);

draw(C -- D -- K -- M -- I -- cycle);
draw(D -- K -- L -- G -- E -- A -- cycle);
draw(K -- L -- N -- M -- cycle);
draw(M -- N -- J -- I -- cycle);

draw (A -- B -- C -- D -- cycle);
draw (E -- F -- H -- G -- cycle);
draw (A -- B -- F -- E -- cycle);

draw(surface(B -- C -- I -- J -- H -- F -- cycle), verde*80);

draw(surface(C -- D -- K -- M -- I -- cycle), verde*80);
draw(surface(D -- K -- L -- G -- E -- A -- cycle), verde*80);
draw(surface(K -- L -- N -- M -- cycle), verde*80);
draw(surface(M -- N -- J -- I -- cycle), verde*80);

draw (surface(A -- B -- C -- D -- cycle), verde*80);
draw (surface(E -- F -- H -- G -- cycle), verde*80);
draw (surface(A -- B -- F -- E -- cycle), verde*80);

draw (surface(H -- J -- N --L -- G -- cycle), verde*80);


triple x = (C - (.7,0,0));
triple w = (D - (.7,0,0));
triple p = (K - (.7,0,0));
triple q = (M - (.7,0,0));
triple y = (I - (.7,0,0));
triple z = (E - (.4,0,0));
triple u = (F - (.4,0,0));

draw (x -- w -- p --q --y -- cycle);
draw (z -- u, dashed);

triple a = (0,0,0);
triple b = (1,0,0);
triple c = (1,1,0);
triple d = (0,1,0);

triple e = (0,0,1);
triple f = (1,0,1);
triple g = (1,1,1);
triple h = (0,1,1);

draw ((a -- b -- f -- e -- cycle));
draw ((a -- d -- h -- e -- cycle));
draw ((a -- b -- c -- d -- cycle));

dot(A, linewidth(2));
dot(B, linewidth(2));
dot(C, linewidth(2));
dot(D, linewidth(2));
dot(E, linewidth(2));
dot(F, linewidth(2));
dot(G, linewidth(2));
dot(H, linewidth(2));
dot(I, linewidth(2));
dot(J, linewidth(2));
dot(K, linewidth(2));
dot(L, linewidth(2));
dot(M, linewidth(2));
dot(N, linewidth(2));

//draw (a -- b -- c -- d -- cycle);
//draw (e -- f -- g -- h -- cycle);
//draw (a -- b -- f -- e -- cycle);
//draw (c -- d -- h -- g -- cycle);



draw((.9,0,.5) -- (.6,0,.5), arrow=Arrow3(TeXHead2));

path3 f1 =  ((0.9,1.15,0) -- (0.1,1.15,0));

string txt = "Plano Horizontal";
draw(labelpath(txt, subpath(f1,0,1),angle=100));
\end{asy}
\\
(F)
\end{figure}

\end{multicols}
\caption{Representação das vistas ortogonais}
\label{\detokenize{GE301-6:fig-proj-vistas-ortogonais-03}}
\end{figure}


Nas imagens (D), (E) e (F) da \hyperref[\detokenize{GE301-6:fig-proj-vistas-ortogonais-03}]{Figura \ref{\detokenize{GE301-6:fig-proj-vistas-ortogonais-03}}}, as setas indicam quais pontos serão projetados nos respectivos planos, da seguinte maneira: supondo-se que o objeto é opaco e que “raios de luz” chegam no sentido da seta, estes raios atingirão pontos do objeto e serão então bloqueados, não atingindo outros pontos. Apenas os pontos que recebem “raios de luz” serão projetados no plano. Estes pontos serão desenhados com uma linha sólida. Partes do objeto que ficam “escondidos” são desenhados com uma linha pontilhada, como acontece na imagem (F).

Na \hyperref[\detokenize{GE301-6:fig-proj-vistas-ortogonais-04}]{Figura \ref{\detokenize{GE301-6:fig-proj-vistas-ortogonais-04}}} as três projeções nos três planos estão desenhadas simultaneamente com o sólido 3D(imagem (A)) e sem ele (imagem (B)). As três projeções podem então ser dispostas lado a lado em um mesmo plano, gerando então a representação clássica das \index{vistas principais}vistas principais (imagem (C)): a \index{vista frontal}vista frontal (projeção ortogonal no plano frontal), a \index{vista lateral}vista lateral (projeção ortogonal no plano lateral) e a \index{vista superior}vista superior (projeção ortogonal no plano horizontal). Apesar do objeto 3D original ter as faces pintadas de vermelho, nas projeções apenas os contornos estão desenhados.

\phantomsection\label{\detokenize{GE301-6:fig-proj-vistas-ortogonais-04}}

\begin{figure}[H]
\centering
\begin{multicols}{2}

\begin{figure}[H]
\centering
\begin{asy}
size(6cm);

currentprojection=orthographic(1.5,1.5,1/2);

triple A = (.4,.4,.35);
triple B = (.4,.7,.35);
triple C = (.7,.7,.35);
triple D = (.7,.4,.35);

triple E = (A+(0,0,.1));
triple F = (B+(0,0,.1));
triple G = (E+(.1,0,0));
triple H = (F+(.1,0,0));

triple I = (C+(0,0,.3));
triple J = (H+(0,0,.2));

triple K = (D+(0,0,.4));
triple L = (G+(0,0,.3));

triple M = (K+(0,.1,0));
triple N = (L+(0,.1,0));

draw(B -- C -- I -- J -- H -- F -- cycle);

draw(C -- D -- K -- M -- I -- cycle);
draw(D -- K -- L -- G -- E -- A -- cycle);
draw(K -- L -- N -- M -- cycle);
draw(M -- N -- J -- I -- cycle);

draw (A -- B -- C -- D -- cycle);
draw (E -- F -- H -- G -- cycle);
draw (A -- B -- F -- E -- cycle);

draw(surface(B -- C -- I -- J -- H -- F -- cycle), verde*80);

draw(surface(C -- D -- K -- M -- I -- cycle), verde*80);
draw(surface(D -- K -- L -- G -- E -- A -- cycle), verde*80);
draw(surface(K -- L -- N -- M -- cycle), verde*80);
draw(surface(M -- N -- J -- I -- cycle), verde*80);

draw (surface(A -- B -- C -- D -- cycle), verde*80);
draw (surface(E -- F -- H -- G -- cycle), verde*80);
draw (surface(A -- B -- F -- E -- cycle), verde*80);

draw (surface(H -- J -- N --L -- G -- cycle), verde*80);


triple x = (C - (.7,0,0));
triple w = (D - (.7,0,0));
triple p = (K - (.7,0,0));
triple q = (M - (.7,0,0));
triple y = (I - (.7,0,0));
triple z = (E - (.4,0,0));
triple u = (F - (.4,0,0));

draw (x -- w -- p --q --y -- cycle);
draw (z -- u, dashed);

triple x1 = (A + (0,-.4,0));
triple w1 = (D + (0,-.4,0));
triple p1 = (E + (0,-.4,0));
triple q1 = (G + (0,-.4,0));
triple y1 = (K + (0,-.4,0));
triple z1 = (L + (0,-.4,0));
triple u1 = (I + (0,-.7,0));
triple v1 = (J + (0,-.7,0));

draw (x1 -- w1 -- y1 -- z1 --q1 -- p1 -- cycle);
draw (u1 -- v1);

triple x2 = (A + (0,0,-.35));
triple w2 = (B + (0,0,-.35));
triple p2 = (C + (0,0,-.35));
triple q2 = (D + (0,0,-.35));
triple y2 = (G + (0,0,-.45));
triple z2 = (H + (0,0,-.45));
triple u2 = (M + (0,0,-.75));
triple v2 = (N + (0,0,-.75));

draw (x2 -- w2 -- p2 -- q2 -- cycle);
draw (y2 -- z2);
draw (u2 -- v2);

triple a = (0,0,0);
triple b = (1,0,0);
triple c = (1,1,0);
triple d = (0,1,0);

triple e = (0,0,1);
triple f = (1,0,1);
triple g = (1,1,1);
triple h = (0,1,1);

draw ((a -- b -- f -- e -- cycle));
draw ((a -- d -- h -- e -- cycle));
draw ((a -- b -- c -- d -- cycle));

dot(A, linewidth(2));
dot(B, linewidth(2));
dot(C, linewidth(2));
dot(D, linewidth(2));
dot(E, linewidth(2));
dot(F, linewidth(2));
dot(G, linewidth(2));
dot(H, linewidth(2));
dot(I, linewidth(2));
dot(J, linewidth(2));
dot(K, linewidth(2));
dot(L, linewidth(2));
dot(M, linewidth(2));
dot(N, linewidth(2));

//draw (a -- b -- c -- d -- cycle);
//draw (e -- f -- g -- h -- cycle);
//draw (a -- b -- f -- e -- cycle);
//draw (c -- d -- h -- g -- cycle);


draw((0,.9,.2) -- (0,.6,.2), arrow=Arrow3(TeXHead2));
draw((.9,0,.2) -- (.6,0,.2), arrow=Arrow3(TeXHead2));
draw((0,.2,.9) -- (0,.2,.6), arrow=Arrow3(TeXHead2));

path3 f1 =  ((.8,0,1.05) -- (.2,0,1.05));

string txt = "Plano Frontal";
draw(labelpath(txt, subpath(f1,0,1),angle=180));

path3 f2 =  ((0,0.2,1.05) -- (0,0.8,1.05));

string txt = "Plano Lateral";
draw(labelpath(txt, subpath(f2,0,1),angle=270));

path3 f3 =  ((0.9,1.15,0) -- (0.1,1.15,0));

string txt = "Plano Horizontal";
draw(labelpath(txt, subpath(f3,0,1),angle=100));
\end{asy}
\\
(A)
\end{figure}

\begin{figure}[H]
\centering
\begin{asy}
size(6cm);

currentprojection=orthographic(1.5,1.5,1/2);

triple A = (.4,.4,.35);
triple B = (.4,.7,.35);
triple C = (.7,.7,.35);
triple D = (.7,.4,.35);

triple E = (A+(0,0,.1));
triple F = (B+(0,0,.1));
triple G = (E+(.1,0,0));
triple H = (F+(.1,0,0));

triple I = (C+(0,0,.3));
triple J = (H+(0,0,.2));

triple K = (D+(0,0,.4));
triple L = (G+(0,0,.3));

triple M = (K+(0,.1,0));
triple N = (L+(0,.1,0));

//draw(B -- C -- I -- J -- H -- F -- cycle);

//draw(C -- D -- K -- M -- I -- cycle);
//draw(D -- K -- L -- G -- E -- A -- cycle);
//draw(K -- L -- N -- M -- cycle);
//draw(M -- N -- J -- I -- cycle);

//draw (A -- B -- C -- D -- cycle);
//draw (E -- F -- H -- G -- cycle);
//draw (A -- B -- F -- E -- cycle);

//draw(surface(B -- C -- I -- J -- H -- F -- cycle), verde*80);

//draw(surface(C -- D -- K -- M -- I -- cycle), verde*80);
//draw(surface(D -- K -- L -- G -- E -- A -- cycle), verde*80);
//draw(surface(K -- L -- N -- M -- cycle), verde*80);
//draw(surface(M -- N -- J -- I -- cycle), verde*80);

//draw (surface(A -- B -- C -- D -- cycle), verde*80);
//draw (surface(E -- F -- H -- G -- cycle), verde*80);
//draw (surface(A -- B -- F -- E -- cycle), verde*80);

triple x = (C - (.7,0,0));
triple w = (D - (.7,0,0));
triple p = (K - (.7,0,0));
triple q = (M - (.7,0,0));
triple y = (I - (.7,0,0));
triple z = (E - (.4,0,0));
triple u = (F - (.4,0,0));

draw (x -- w -- p --q --y -- cycle);
draw (z -- u, dashed);

triple x1 = (A + (0,-.4,0));
triple w1 = (D + (0,-.4,0));
triple p1 = (E + (0,-.4,0));
triple q1 = (G + (0,-.4,0));
triple y1 = (K + (0,-.4,0));
triple z1 = (L + (0,-.4,0));
triple u1 = (I + (0,-.7,0));
triple v1 = (J + (0,-.7,0));

draw (x1 -- w1 -- y1 -- z1 --q1 -- p1 -- cycle);
draw (u1 -- v1);

triple x2 = (A + (0,0,-.35));
triple w2 = (B + (0,0,-.35));
triple p2 = (C + (0,0,-.35));
triple q2 = (D + (0,0,-.35));
triple y2 = (G + (0,0,-.45));
triple z2 = (H + (0,0,-.45));
triple u2 = (M + (0,0,-.75));
triple v2 = (N + (0,0,-.75));

draw (x2 -- w2 -- p2 -- q2 -- cycle);
draw (y2 -- z2);
draw (u2 -- v2);

triple a = (0,0,0);
triple b = (1,0,0);
triple c = (1,1,0);
triple d = (0,1,0);

triple e = (0,0,1);
triple f = (1,0,1);
triple g = (1,1,1);
triple h = (0,1,1);

draw ((a -- b -- f -- e -- cycle));
draw ((a -- d -- h -- e -- cycle));
draw ((a -- b -- c -- d -- cycle));



path3 f1 =  ((.8,0,1.05) -- (.2,0,1.05));

string txt = "Plano Frontal";
draw(labelpath(txt, subpath(f1,0,1),angle=180));

path3 f2 =  ((0,0.2,1.05) -- (0,0.8,1.05));

string txt = "Plano Lateral";
draw(labelpath(txt, subpath(f2,0,1),angle=270));

path3 f3 =  ((0.9,1.15,0) -- (0.1,1.15,0));

string txt = "Plano Horizontal";
draw(labelpath(txt, subpath(f3,0,1),angle=100));
\end{asy}
\\
(B)
\end{figure}

\end{multicols}

\begin{figure}[H]
\centering
\begin{asy}
size(9cm);

currentprojection=orthographic(1.5,1.5,1/2);

triple A = (.4,.4,.35);
triple B = (.4,.7,.35);
triple C = (.7,.7,.35);
triple D = (.7,.4,.35);

triple E = (A+(0,0,.1));
triple F = (B+(0,0,.1));
triple G = (E+(.1,0,0));
triple H = (F+(.1,0,0));

triple I = (C+(0,0,.3));
triple J = (H+(0,0,.2));

triple K = (D+(0,0,.4));
triple L = (G+(0,0,.3));

triple M = (K+(0,.1,0));
triple N = (L+(0,.1,0));

//draw(B -- C -- I -- J -- H -- F -- cycle);

//draw(C -- D -- K -- M -- I -- cycle);
//draw(D -- K -- L -- G -- E -- A -- cycle);
//draw(K -- L -- N -- M -- cycle);
//draw(M -- N -- J -- I -- cycle);

//draw (A -- B -- C -- D -- cycle);
//draw (E -- F -- H -- G -- cycle);
//draw (A -- B -- F -- E -- cycle);

//draw(surface(B -- C -- I -- J -- H -- F -- cycle), verde*80);

//draw(surface(C -- D -- K -- M -- I -- cycle), verde*80);
//draw(surface(D -- K -- L -- G -- E -- A -- cycle), verde*80);
//draw(surface(K -- L -- N -- M -- cycle), verde*80);
//draw(surface(M -- N -- J -- I -- cycle), verde*80);

//draw (surface(A -- B -- C -- D -- cycle), verde*80);
//draw (surface(E -- F -- H -- G -- cycle), verde*80);
//draw (surface(A -- B -- F -- E -- cycle), verde*80);

//triple x = (C - (.7,0,0));
//triple w = (D - (.7,0,0));
//triple p = (K - (.7,0,0));
//triple q = (M - (.7,0,0));
//triple y = (I - (.7,0,0));
//triple z = (E - (.4,0,0));
//triple u = (F - (.4,0,0));

//draw (x -- w -- p --q --y -- cycle);
//draw (z -- u, dashed);

//triple x1 = (A + (0,-.4,0));
//triple w1 = (D + (0,-.4,0));
//triple p1 = (E + (0,-.4,0));
//triple q1 = (G + (0,-.4,0));
//triple y1 = (K + (0,-.4,0));
//triple z1 = (L + (0,-.4,0));
//triple u1 = (I + (0,-.7,0));
//triple v1 = (J + (0,-.7,0));

//draw (x1 -- w1 -- y1 -- z1 --q1 -- p1 -- cycle);
//draw (u1 -- v1);

//pair A = (.4,.4);
//pair B = (.4,.5);
//pair C = (.3,.7);
//pair D = (.7,.4);

//draw(A--B--C--D);

//triple E = (A+(0,0,.1));
//triple F = (B+(0,0,.1));
//triple G = (E+(.1,0,0));
//triple H = (F+(.1,0,0));

//triple I = (C+(0,0,.3));
//triple J = (H+(0,0,.2));

//triple K = (D+(0,0,.4));
//triple L = (G+(0,0,.3));

//triple M = (K+(0,.1,0));
//triple N = (L+(0,.1,0));

//triple x2 = (A + (0,0,-.35));
//triple w2 = (B + (0,0,-.35));
//triple p2 = (C + (0,0,-.35));
//triple q2 = (D + (0,0,-.35));
//triple y2 = (G + (0,0,-.45));
//triple z2 = (H + (0,0,-.45));
//triple u2 = (M + (0,0,-.75));
//triple v2 = (N + (0,0,-.75));

//draw (x2 -- w2 -- p2 -- q2 -- cycle);
//draw (y2 -- z2);
//draw (u2 -- v2);

pair a = (0,0);
pair b = (1,0);
pair c = (1,1);
pair d = (0,1);

pair A = (.7,.25);
pair B = (.7,.25);
pair C = (.7,.35);
pair D = (.6,.35);

pair E = (.6,.75);
pair F = (.4,.75);
pair G = (.4,.25);

pair H = (E - (0,.1));
pair I = (F - (0,.1));

draw (shift(-.05)*(H -- I));
draw (shift(-.05)*(A -- B -- C -- D -- E -- F -- G -- cycle));

//pair e = (0,0,1);
//pair f = (1,0,1);
//pair g = (1,1,1);
//pair h = (0,1,1);

//draw ((a -- b -- f -- e -- cycle));
//draw ((a -- d -- h -- e -- cycle));
draw ((a -- b -- c -- d -- cycle));
draw (shift(1,0)*(a -- b -- c -- d -- cycle));

pair A1 = (.4,.25);
pair B1 = (.7,.25);
pair C1 = (.7,.6);
pair D1 = (.5,.75);
pair E1 = (.4,.75);
pair F1 = (A1+(0,.1));
pair G1 = (B1+(0,.1));

draw (shift(.95,0)*(A1 -- B1 -- C1 -- D1 -- E1) -- cycle);
draw (shift(.95,0)*(F1 -- G1), dashed);


draw (shift(0,-1)*(a -- b -- c -- d -- cycle));

pair A2 = (.3,.3);
pair B2 = (.7,.3);
pair C2 = (.7,.7);
pair D2 = (.3,.7);
pair E2 = (.6,.7);
pair F2 = (.6,.3);
pair G2 = (.3,.6);
pair H2 = (.6,.6);

draw (shift(0,-1)*(A2 -- B2 -- C2 -- D2) -- cycle);
draw (shift(0,-1)*(E2 -- F2));
draw (shift(0,-1)*(G2 -- H2));


path f1 =  ((0,1) -- (1,1));

label(scale(.8)*"Vista Frontal", (.5,1), align=S);
label(scale(.8)*"Vista Lateral", shift(1,0)*(.5,1), align=S);
label(scale(.8)*"Vista Superior", shift(0,-1)*(.5,1), align=S);

label(rotate(90)*scale(.8)*"Plano Frontal",(0,.5), align=W);
label(rotate(90)*scale(.8)*"Plano Horizontal",shift(0,-1)*(0,.5), align=W);
label(rotate(90)*scale(.8)*"Plano Lateral",(2.1,.5));
\end{asy}
\\
(C)

\end{figure}

\caption{Projeções ortogonais nos planos}
\label{\detokenize{GE301-6:fig-proj-vistas-ortogonais-04}}
\end{figure}
% \begin{figure}[H]
% \centering

% \noindent\includegraphics[width=300bp]{{vistas-ortogonais-04}.jpg}
% \caption{Projeções ortogonais nos planos}
% \label{\detokenize{GE301-6:fig-proj-vistas-ortogonais-04}}\end{figure}
\vspace{1em}

\textbf{Etapa 1.}

Desenhe as vistas principais ortogonais de cada um dos objetos geométricos apresentados a seguir. Dê sua resposta como em (C) em \hyperref[\detokenize{GE301-6:fig-proj-vistas-ortogonais-04}]{Figura \ref{\detokenize{GE301-6:fig-proj-vistas-ortogonais-04}}}.

\begin{enumerate}
\begin{multicols}{2}
\item
\adjustbox{valign=t}{
\begin{minipage}{\linewidth}
\begin{asy}
size(5cm);

currentprojection=perspective(2.53,1.60,1.27);

triple a = (0,0,0);
triple b = (1,0,0);
triple c = (1,1,0);
triple d = (0,1,0);

triple e = (0,0,1);
triple f = (1,0,1);
triple g = (1,1,1);
triple h = (0,1,1);

draw ((a -- b -- f -- e -- cycle));
draw ((a -- d -- h -- e -- cycle));
draw ((a -- b -- c -- d -- cycle));


triple A = (1/4,1/4,1/4);
triple B = (3/4,1/4,1/4);
triple C = (3/4,3/4,1/4);
triple D = (1/4,3/4,1/4);

triple E = (1/4,1/4,3/4);
triple F = (3/4,1/4,3/4);
triple G = (3/4,3/4,3/4);
triple H = (1/4,3/4,3/4);

draw (A -- B -- C -- D -- cycle);
draw (E -- F -- G -- H -- cycle);

draw (A -- E);
draw (B -- F);
draw (C -- G);
draw (D -- H);

draw (surface(A -- B -- C -- D -- cycle), verde*80);
draw (surface(E -- F -- G -- H -- cycle), verde*80);

draw (surface(A -- E -- F -- B -- cycle), verde*80);
draw (surface(B -- F -- G -- C -- cycle), verde*80);
draw (surface(C -- G -- H -- D -- cycle), verde*80);
draw (surface(D -- H -- E -- A -- cycle), verde*80);
\end{asy}
\end{minipage}}

\item
\adjustbox{valign=t}{
\begin{minipage}{\linewidth}
\begin{asy}
size(5cm);

currentprojection=perspective(2.53,1.60,1.27);



triple a = (0,0,0);
triple b = (1,0,0);
triple c = (1,1,0);
triple d = (0,1,0);

triple e = (0,0,1);
triple f = (1,0,1);
triple g = (1,1,1);
triple h = (0,1,1);

draw ((a -- b -- f -- e -- cycle));
draw ((a -- d -- h -- e -- cycle));
draw ((a -- b -- c -- d -- cycle));


triple A = rotate(45,(0,.5,.5), (1,.5,.5))*(1/4,1/4,1/4);
triple B = rotate(45,(0,.5,.5), (1,.5,.5))*(3/4,1/4,1/4);
triple C = rotate(45,(0,.5,.5), (1,.5,.5))*(3/4,3/4,1/4);
triple D = rotate(45,(0,.5,.5), (1,.5,.5))*(1/4,3/4,1/4);

triple E = rotate(45,(0,.5,.5), (1,.5,.5))*(1/4,1/4,3/4);
triple F = rotate(45,(0,.5,.5), (1,.5,.5))*(3/4,1/4,3/4);
triple G = rotate(45,(0,.5,.5), (1,.5,.5))*(3/4,3/4,3/4);
triple H = rotate(45,(0,.5,.5), (1,.5,.5))*(1/4,3/4,3/4);

draw (A -- B -- C -- D -- cycle);
draw (E -- F -- G -- H -- cycle);

draw (A -- E);
draw (B -- F);
draw (C -- G);
draw (D -- H);

draw (surface(A -- B -- C -- D -- cycle), verde*80);
draw (surface(E -- F -- G -- H -- cycle), verde*80);

draw (surface(A -- E -- F -- B -- cycle), verde*80);
draw (surface(B -- F -- G -- C -- cycle), verde*80);
draw (surface(C -- G -- H -- D -- cycle), verde*80);
draw (surface(D -- H -- E -- A -- cycle), verde*80);
\end{asy}
\end{minipage}}
\end{multicols}

\begin{multicols}{2}

\item
\adjustbox{valign=t}{
\begin{minipage}{\linewidth}
\begin{asy}
size(5cm);

currentprojection=perspective(2.53,1.60,1.27);



triple a = (0,0,0);
triple b = (1,0,0);
triple c = (1,1,0);
triple d = (0,1,0);

triple e = (0,0,1);
triple f = (1,0,1);
triple g = (1,1,1);
triple h = (0,1,1);

draw ((a -- b -- f -- e -- cycle));
draw ((a -- d -- h -- e -- cycle));
draw ((a -- b -- c -- d -- cycle));


draw(surface(sphere(c=(.5,.5,.5), r=1/3)), verde*80);
\end{asy}
\end{minipage}}

\item
\adjustbox{valign=t}{
\begin{minipage}{\linewidth}
\begin{asy}
size(5cm);

currentprojection=perspective(2.53,1.60,1.27);



triple a = (0,0,0);
triple b = (1,0,0);
triple c = (1,1,0);
triple d = (0,1,0);

triple e = (0,0,1);
triple f = (1,0,1);
triple g = (1,1,1);
triple h = (0,1,1);

draw ((a -- b -- f -- e -- cycle));
draw ((a -- d -- h -- e -- cycle));
draw ((a -- b -- c -- d -- cycle));

draw(cylinder((1/2,1/2,1/5), 1/3,3/5,(0,0,1)));
draw(surface(cylinder((1/2,1/2,1/5), 1/3,3/5,(0,0,1))), verde*80);
draw(surface(circle((1/2,1/2,4/5), r=1/3,Z)), verde*80);
//draw((circle((1/2,1/2,4/5), r=1/3,Z)));
draw(cylinder((1/2,1/2,1/5), 1/3,3/5,(0,0,1)), black+linewidth(.5));
\end{asy}
\end{minipage}}

\end{multicols}
\begin{multicols}{2}
\item
\adjustbox{valign=t}{
\begin{minipage}{\linewidth}
\begin{asy}
size(5cm);

currentprojection=perspective(2.53,1.60,1.27);



triple a = (0,0,0);
triple b = (1,0,0);
triple c = (1,1,0);
triple d = (0,1,0);

triple e = (0,0,1);
triple f = (1,0,1);
triple g = (1,1,1);
triple h = (0,1,1);

draw ((a -- b -- f -- e -- cycle));
draw ((a -- d -- h -- e -- cycle));
draw ((a -- b -- c -- d -- cycle));

draw(surface(cone((1/2,1/2,1/5), 1/3,3/5,(0,0,1))), verde*80);
\end{asy}
\end{minipage}}
\end{multicols}
\end{enumerate}


\textbf{Etapa 2.}

Em cada uma das figuras a seguir, desenhe um objeto cujas projeções ortogonais sobre os planos frontal, horizontal e lateral são aquelas apresentadas.

\begin{enumerate}
\begin{multicols}{2}

\item
\adjustbox{valign=t}{
\begin{minipage}{\linewidth}
\begin{asy}
size(5cm);

currentprojection=perspective(2.53,1.60,1.27);

triple a = (0,0,0);
triple b = (1,0,0);
triple c = (1,1,0);
triple d = (0,1,0);

triple e = (0,0,1);
triple f = (1,0,1);
triple g = (1,1,1);
triple h = (0,1,1);

draw ((a -- b -- f -- e -- cycle));
draw ((a -- d -- h -- e -- cycle));
draw ((a -- b -- c -- d -- cycle));

draw((1/2,1/4,0)--(1/2,3/4,0));
draw((0,1/4,1/2)--(0,3/4,1/2));

//draw (a -- b -- c -- d -- cycle);
//draw (e -- f -- g -- h -- cycle);
//draw (a -- b -- f -- e -- cycle);
//draw (c -- d -- h -- g -- cycle);
\end{asy}
\end{minipage}}


\item
\adjustbox{valign=t}{
\begin{minipage}{\linewidth}
\begin{asy}
size(5cm);

currentprojection=perspective(2.53,1.60,1.27);

triple a = (0,0,0);
triple b = (1,0,0);
triple c = (1,1,0);
triple d = (0,1,0);

triple e = (0,0,1);
triple f = (1,0,1);
triple g = (1,1,1);
triple h = (0,1,1);

draw ((a -- b -- f -- e -- cycle));
draw ((a -- d -- h -- e -- cycle));
draw ((a -- b -- c -- d -- cycle));

draw((1/2,1/4,0)--(1/2,3/4,0));
draw((1/2,0,1/4)--(1/2,0,3/4));
draw((0,1/4,1/4)--(0,3/4,3/4));

//draw (a -- b -- c -- d -- cycle);
//draw (e -- f -- g -- h -- cycle);
//draw (a -- b -- f -- e -- cycle);
//draw (c -- d -- h -- g -- cycle);
\end{asy}
\end{minipage}}
\end{multicols}

\begin{multicols}{2}

\item
\adjustbox{valign=t}{
\begin{minipage}{\linewidth}
\begin{asy}
size(5cm);

currentprojection=perspective(2.53,1.60,1.27);

triple a = (0,0,0);
triple b = (1,0,0);
triple c = (1,1,0);
triple d = (0,1,0);

triple e = (0,0,1);
triple f = (1,0,1);
triple g = (1,1,1);
triple h = (0,1,1);

draw ((a -- b -- f -- e -- cycle));
draw ((a -- d -- h -- e -- cycle));
draw ((a -- b -- c -- d -- cycle));

draw((3/4,0,1/4)--(1/4,0,3/4));
draw((1/2+sqrt(2)/8,1/4,0)--(1/2-sqrt(2)/8,3/4,0));
draw((0,1/2-sqrt(2)/8,1/4)--(0,1/2+sqrt(2)/8,3/4));


//draw (a -- b -- c -- d -- cycle);
//draw (e -- f -- g -- h -- cycle);
//draw (a -- b -- f -- e -- cycle);
//draw (c -- d -- h -- g -- cycle);

\end{asy}
\end{minipage}}


\item
\adjustbox{valign=t}{
\begin{minipage}{\linewidth}
\begin{asy}
size(5cm);

currentprojection=perspective(2.53,1.60,1.27);

triple a = (0,0,0);
triple b = (1,0,0);
triple c = (1,1,0);
triple d = (0,1,0);

triple e = (0,0,1);
triple f = (1,0,1);
triple g = (1,1,1);
triple h = (0,1,1);

draw ((a -- b -- f -- e -- cycle));
draw ((a -- d -- h -- e -- cycle));
draw ((a -- b -- c -- d -- cycle));

draw((1/4,0,1/2)--(3/4,0,1/2));
draw((0,1/4,1/2)--(0,3/4,1/2));

draw((1/4,1/4,0)--(3/4,1/4,0)--(3/4,3/4,0) --(1/4,3/4,0) -- cycle);

\end{asy}
\end{minipage}}
\end{multicols}
\begin{multicols}{2}

\item
\adjustbox{valign=t}{
\begin{minipage}{\linewidth}
\begin{asy}
size(5cm);
currentprojection=perspective(2.53,1.60,1.27);

triple a = (0,0,0);
triple b = (1,0,0);
triple c = (1,1,0);
triple d = (0,1,0);

triple e = (0,0,1);
triple f = (1,0,1);
triple g = (1,1,1);
triple h = (0,1,1);

draw ((a -- b -- f -- e -- cycle));
draw ((a -- d -- h -- e -- cycle));
draw ((a -- b -- c -- d -- cycle));


draw((0,1/4,1/2-sqrt(2)/8)--(0,3/4,1/2+sqrt(2)/8));

draw((1/4,0,1/2-sqrt(2)/8)--(3/4,0,1/2-sqrt(2)/8)--(3/4,0,1/2+sqrt(2)/8)--(1/4,0,1/2+sqrt(2)/8)--cycle);

draw((1/4,1/2-sqrt(2)/8,0)--(3/4,1/2-sqrt(2)/8,0)--(3/4,1/2+sqrt(2)/8,0)--(1/4,1/2+sqrt(2)/8,0)--cycle);


\end{asy}
\end{minipage}}


\item
\adjustbox{valign=t}{
\begin{minipage}{\linewidth}
\begin{asy}
size(5cm);

currentprojection=perspective(2.53,1.60,1.27);


triple a = (0,0,0);
triple b = (1,0,0);
triple c = (1,1,0);
triple d = (0,1,0);

triple e = (0,0,1);
triple f = (1,0,1);
triple g = (1,1,1);
triple h = (0,1,1);

draw ((a -- b -- f -- e -- cycle));
draw ((a -- d -- h -- e -- cycle));
draw ((a -- b -- c -- d -- cycle));

real f(real x){return -5*x^2+5*(x)-.5;}
path s = graph(f,0.15,.85);

path3 d = shift(0,.8,.2)*rotate(135,(0,0,0),(1,0,0))*path3(s);
//draw(d);

draw(planeproject(XY*unitsquare3)*d);
draw(planeproject(YZ*unitsquare3)*d);
draw(planeproject(ZX*unitsquare3)*d);
\end{asy}
\end{minipage}}
\end{multicols}
\begin{multicols}{2}

\item
\adjustbox{valign=t}{
\begin{minipage}{\linewidth}
\begin{asy}
settings.render=8;
size(5cm);

currentprojection=perspective(2.53,1.60,1.27);

real x(real t) {return (1+cos(2pi*t)+2)/6;}
real y(real t) {return (1+sin(2pi*t)+2)/6;}
real z(real t) {return (t+.5)/6;}

path3 p=graph(x,y,z,0,5,operator ..);

//draw(p,Arrow3);
draw(planeproject(XY*unitsquare3)*p);
draw(planeproject(YZ*unitsquare3)*p);
draw(planeproject(ZX*unitsquare3)*p);

triple a = (0,0,0);
triple b = (1,0,0);
triple c = (1,1,0);
triple d = (0,1,0);

triple e = (0,0,1);
triple f = (1,0,1);
triple g = (1,1,1);
triple h = (0,1,1);

draw ((a -- b -- f -- e -- cycle));
draw ((a -- d -- h -- e -- cycle));
draw ((a -- b -- c -- d -- cycle));
\end{asy}
\end{minipage}}
\end{multicols}
\end{enumerate}

\vspace{3em}
\textbf{Etapa 3.}

Desenhe \textbf{dois objetos diferentes} cujas projeções ortogonais sobre os planos frontal, horizontal e lateral são aquelas na figura a seguir. Nota: as projeções são congruentes e são formadas pelos quatro lados de um mesmo quadrado.

\begin{figure}[H]
\centering

\begin{asy}
size(6.5cm);
currentprojection=perspective(2.53,1.60,1.27);

triple a = (0,0,0);
triple b = (1,0,0);
triple c = (1,1,0);
triple d = (0,1,0);

triple e = (0,0,1);
triple f = (1,0,1);
triple g = (1,1,1);
triple h = (0,1,1);

draw ((a -- b -- f -- e -- cycle));
draw ((a -- d -- h -- e -- cycle));
draw ((a -- b -- c -- d -- cycle));

draw((1/4,1/4,0)--(3/4,1/4,0)--(3/4,3/4,0) --(1/4,3/4,0) -- cycle);
draw((0,1/4,1/4)--(0,3/4,1/4)--(0,3/4,3/4) --(0,1/4,3/4) -- cycle);
draw((1/4,0,1/4)--(3/4,0,1/4)--(3/4,0,3/4) --(1/4,0,3/4) -- cycle);


\end{asy}
\end{figure}

\textbf{Etapa 4.}
\begin{quote}
\end{quote}

\textbf{(Adaptado de um problema proposto por Martin Gardner)} A figura a seguir exibe as vistas frontal e superior de uma estrutura 3D de madeira. Como seria este objeto 3D e sua vista lateral?


\begin{figure}[H]
\centering

\begin{asy}
size(9cm);

currentprojection=orthographic(1.5,1.5,1/2);

pair a = (0,0);
pair b = (1,0);
pair c = (1,1);
pair d = (0,1);

draw ((a -- b -- c -- d -- cycle));

pair A = (.2,.2);
pair B = (.8,.2);
pair C = (.8,.8);
pair D = (.2,.8);

pair A1 = (.35,.35);
pair B1 = (.65,.35);
pair C1 = (.65,.65);
pair D1 = (.35,.65);

filldraw (A -- B -- C -- D -- cycle, fillpen=verde);
filldraw (A1 -- B1 -- C1 -- D1 -- cycle, fillpen=verde+white);

draw (shift(1,0)*(a -- b -- c -- d -- cycle));
draw ((2,0) -- (2,1));


draw (shift(0,-1)*(a -- b -- c -- d -- cycle));
filldraw (shift(0,-1)*(A -- B -- C -- D -- cycle), fillpen=verde);
filldraw (shift(0,-1)*(A1 -- B1 -- C1 -- D1 -- cycle),fillpen=verde+white);

label(scale(.8)*"Vista Frontal", (.5,1), align=S);
label(scale(.8)*"Vista Lateral", shift(1,0)*(.5,1), align=S);
label(scale(.8)*"Vista Superior", shift(0,-1)*(.5,1), align=S);
label(scale(.8)*"Objeto 3D", shift(1,-1)*(.5,1), align=S);


label(rotate(90)*scale(.8)*"Plano Frontal",(0,.5), align=W);
label(rotate(90)*scale(.8)*"Plano Horizontal",shift(0,-1)*(0,.5), align=W);
label(rotate(90)*scale(.8)*"Plano Lateral",(2.1,.5));

label(scale(3)*"?", (1.5,.5));
label(scale(3)*"?", (1.5,-.5));
\end{asy}
\end{figure}

Importante:
\begin{itemize}
\item {} 
A estrutura não tem segmentos ou linhas pintadas sobre ela.

\item {} 
Todos os contornos escondidos da estrutura devem ser desenhas com linhas tracejadas. Assim, em particular, as vistas frontal e superior não possuem contornos escondidos.

\item {} 
Uma vez que a estrutura é feita de madeira, isto significa que nenhuma de suas partes pode ter espessura zero.ave zero thickness.

\end{itemize}
\end{task}

\begin{observation}{}

É preciso ter atenção para o uso da palavra \index{vista}vista. Autores diferentes dão significados diferentes à palavra, significados estes que podem, inclusive, ser diferentes de sua interpretação comum (“aquilo que se apresenta ao olhar, que se vê”). Por exemplo, é comum encontrar em livros de arquitetura e engenharia exercícios que pedem para determinar a \emph{vista} de um objeto a partir de uma direção dada, como na \sphinxcode{fig proj-vistas-observacao}. A resposta esperada por estes livros é a imagem (B), ou seja, uma projeção ortogonal do cubo vazado. Contudo, (B) \emph{não é o que se é visto} a partir da direção indicada. O que se vê é melhor descrito por uma projeção em perspectiva, a saber, a imagem (C). Pegue o cubo vazado que você usou na \DUrole{xref,std,std-ref}{ativ-proj-luz-e-sombras} e veja por você mesmo.


\begin{figure}[H]
\centering
\begin{multicols}{3}
\begin{figure}[H]
\centering
\begin{asy}
size(5.2cm);
currentlight.background=box2;
currentprojection=perspective(.5,1.5,1.2);

triple a = (0,0,0);
triple b = (1,0,0);
triple c = (1,1,0);
triple d = (0,1,0);

triple e = (0,0,1);
triple f = (1,0,1);
triple g = (1,1,1);
triple h = (0,1,1);

draw (a -- b -- c -- d -- cycle);
draw (e -- f -- g -- h -- cycle);
draw (a -- b -- f -- e -- cycle);
draw (c -- d -- h -- g -- cycle);

triple a1 = (0.1,0.1,0.1);
triple b1 = (.9,0.1,0.1);
triple c1 = (.9,.9,0.1);
triple d1 = (0.1,.9,0.1);

triple e1 = (0.1,0.1,.9);
triple f1 = (.9,0.1,.9);
triple g1 = (.9,.9,.9);
triple h1 = (0.1,.9,.9);

//draw (a1 -- b1 -- c1 -- d1 -- cycle);
//draw (e1 -- f1 -- g1 -- h1 -- cycle);
//draw (a1 -- b1 -- f1 -- e1 -- cycle);
//draw (c1 -- d1 -- h1 -- g1 -- cycle);

triple a2 = (0.1,0.1,0);
triple b2 = (.9,0.1,0);
triple c2 = (.9,.9,0);
triple d2 = (0.1,.9,0);

draw (a2 -- b2 -- c2 -- d2 -- cycle);

triple e2 = (0.1,0.1,1);
triple f2 = (.9,0.1,1);
triple g2 = (.9,.9,1);
triple h2 = (0.1,.9,1);

draw (e2 -- f2 -- g2 -- h2 -- cycle);

triple i2 = (0.1,0,.9);
triple j2 = (.9,0,.9);
triple k2 = (.9,0,.1);
triple l2 = (0.1,0,.1);

draw (i2 -- j2 --k2 -- l2 -- cycle);

triple m2 = (0.1,1,.9);
triple n2 = (.9,1,.9);
triple o2 = (.9,1,.1);
triple p2 = (0.1,1,.1);

draw (m2 -- n2 -- o2 -- p2 -- cycle);

triple q2 = (0,0.1,0.1);
triple r2 = (0,0.9,0.1);
triple s2 = (0,.9,.9);
triple t2 = (0,.1,.9);

draw (q2 -- r2 -- s2 --t2 -- cycle);

triple u2 = (1,0.1,0.1);
triple v2 = (1,0.9,0.1);
triple w2 = (1,.9,.9);
triple x2 = (1,.1,.9);

draw (u2 -- v2 -- w2 --x2 -- cycle);

draw (a -- a2 -- b2 -- b -- c -- c2 -- d2 -- d);
draw (e -- e2 -- f2 -- f -- g -- g2 -- h2 -- h);
draw (b -- u2 -- v2 -- c -- g -- w2 -- x2 -- f);
draw (a -- q2 -- r2 -- d -- h -- s2 -- t2 -- e);
draw (a -- l2 -- k2 -- b -- f -- j2 -- i2 -- e);
draw (d -- p2 -- o2 -- c -- g -- n2 -- m2 -- h);

draw (a2 -- l2 -- k2 -- b2);
draw (a2 -- l2 -- p2 -- d2);
draw (c2 -- o2 -- p2 -- d2);
draw (l2 -- q2 -- r2 -- p2);
draw (b2 -- k2 -- o2 -- c2);

draw (e2 -- i2 -- m2 -- h2 -- cycle);
draw (e2 -- i2 -- j2 -- f2 -- cycle);
draw (i2 -- t2 -- s2 -- m2 -- cycle);
draw (f2 -- g2 -- n2 -- j2 -- cycle);
draw (j2 -- x2 -- u2 -- k2 -- cycle);
draw (v2 -- o2 -- n2 -- w2 -- cycle);

draw (surface(a2 -- l2 -- k2 -- b2 -- cycle), verde*80);
draw (surface(a2 -- l2 -- p2 -- d2 -- cycle), verde*80);
draw (surface(c2 -- o2 -- p2 -- d2 -- cycle), verde*80);
draw (surface(l2 -- q2 -- r2 -- p2 -- cycle), verde*80);
draw (surface(b2 -- k2 -- o2 -- c2 -- cycle), verde*80);



draw (surface(a -- l2 -- k2 -- b -- cycle), verde*80);
draw (surface(b -- u2 -- v2 -- c -- cycle), verde*80);
draw (surface(c -- o2 -- p2 -- d -- cycle), verde*80);
draw (surface(a -- q2 -- r2 -- d -- cycle), verde*80);

draw (surface(b -- u2 -- v2 -- c -- cycle), verde*80);
draw (surface(b -- f -- x2 -- u2 -- cycle), verde*80);
draw (surface(f -- g -- w2 -- x2 -- cycle), verde*80);
draw (surface(g -- c -- v2 -- w2 -- cycle), verde*80);

draw (surface(c -- g -- n2 -- o2 -- cycle), verde*80);
draw (surface(g -- h -- m2 -- n2 -- cycle), verde*80);
draw (surface(h -- d -- p2 -- m2 -- cycle), verde*80);

draw (surface(e -- f -- f2 -- e2 -- cycle), verde*80);
draw (surface(f -- f2 -- g2 -- g -- cycle), verde*80);
draw (surface(g -- g2 -- h2 -- h -- cycle), verde*80);
draw (surface(h -- h2 -- e2 -- e -- cycle), verde*80);

draw (surface(a -- a2 -- b2 -- b -- cycle), verde*80);
draw (surface(b -- b2 -- c2 -- c -- cycle), verde*80);
draw (surface(c -- c2 -- d2 -- d -- cycle), verde*80);
draw (surface(d -- d2 -- a2 -- a -- cycle), verde*80);

draw (surface(a -- e -- i2 -- l2 -- cycle), verde*80);
draw (surface(e -- f -- j2 -- i2 -- cycle), verde*80);
draw (surface(b -- f -- j2 -- k2 -- cycle), verde*80);

draw (surface(e -- h -- s2 -- t2 -- cycle), verde*80);
draw (surface(h -- d -- r2 -- s2 -- cycle), verde*80);
draw (surface(e -- a -- q2 -- t2 -- cycle), verde*80);

draw (u2 -- k2 -- j2 -- x2);

draw (surface(u2 -- k2 -- j2 -- x2 -- cycle), verde*80);
draw (surface(k2 -- u2 -- v2 -- o2 -- cycle), verde*80);
draw (surface(l2 -- q2 -- t2 -- i2 -- cycle), verde*80);
draw (surface(e2 -- i2 -- m2 -- h2 -- cycle), verde*80);
draw (surface(e2 -- i2 -- j2 -- f2 -- cycle), verde*80);
draw (surface(i2 -- t2 -- s2 -- m2 -- cycle), verde*80);
draw (surface(s2 -- m2 -- p2 -- r2 -- cycle), verde*80);
draw (surface(h2 -- g2 -- n2 -- m2 -- cycle), verde*80);
draw (surface(f2 -- g2 -- n2 -- j2 -- cycle), verde*80);
draw (surface(j2 -- n2 -- w2 -- x2 -- cycle), verde*80);
draw (surface(j2 -- x2 -- u2 -- k2 -- cycle), verde*80);
draw (surface(v2 -- o2 -- n2 -- w2 -- cycle), verde*80);

draw((.5,.5,1.5) -- (.5,.5,1.2), arrow=Arrow3,blue);
\end{asy}
\\
(A)

\end{figure}
\begin{figure}[H]
\centering
Projeção
\\
Orogonal

\begin{asy}
size(4cm);
currentlight.background=box2;
currentprojection=orthographic(0,0,1);

triple a = (0,0,0);
triple b = (1,0,0);
triple c = (1,1,0);
triple d = (0,1,0);

triple e = (0,0,1);
triple f = (1,0,1);
triple g = (1,1,1);
triple h = (0,1,1);

draw (a -- b -- c -- d -- cycle);
draw (e -- f -- g -- h -- cycle);
draw (a -- b -- f -- e -- cycle);
draw (c -- d -- h -- g -- cycle);

triple a1 = (0.1,0.1,0.1);
triple b1 = (.9,0.1,0.1);
triple c1 = (.9,.9,0.1);
triple d1 = (0.1,.9,0.1);

triple e1 = (0.1,0.1,.9);
triple f1 = (.9,0.1,.9);
triple g1 = (.9,.9,.9);
triple h1 = (0.1,.9,.9);

//draw (a1 -- b1 -- c1 -- d1 -- cycle);
//draw (e1 -- f1 -- g1 -- h1 -- cycle);
//draw (a1 -- b1 -- f1 -- e1 -- cycle);
//draw (c1 -- d1 -- h1 -- g1 -- cycle);

triple a2 = (0.1,0.1,0);
triple b2 = (.9,0.1,0);
triple c2 = (.9,.9,0);
triple d2 = (0.1,.9,0);

draw (a2 -- b2 -- c2 -- d2 -- cycle);

triple e2 = (0.1,0.1,1);
triple f2 = (.9,0.1,1);
triple g2 = (.9,.9,1);
triple h2 = (0.1,.9,1);

draw (e2 -- f2 -- g2 -- h2 -- cycle);

triple i2 = (0.1,0,.9);
triple j2 = (.9,0,.9);
triple k2 = (.9,0,.1);
triple l2 = (0.1,0,.1);

draw (i2 -- j2 --k2 -- l2 -- cycle);

triple m2 = (0.1,1,.9);
triple n2 = (.9,1,.9);
triple o2 = (.9,1,.1);
triple p2 = (0.1,1,.1);

draw (m2 -- n2 -- o2 -- p2 -- cycle);

triple q2 = (0,0.1,0.1);
triple r2 = (0,0.9,0.1);
triple s2 = (0,.9,.9);
triple t2 = (0,.1,.9);

draw (q2 -- r2 -- s2 --t2 -- cycle);

triple u2 = (1,0.1,0.1);
triple v2 = (1,0.9,0.1);
triple w2 = (1,.9,.9);
triple x2 = (1,.1,.9);

draw (u2 -- v2 -- w2 --x2 -- cycle);

draw (a -- a2 -- b2 -- b -- c -- c2 -- d2 -- d);
draw (e -- e2 -- f2 -- f -- g -- g2 -- h2 -- h);
draw (b -- u2 -- v2 -- c -- g -- w2 -- x2 -- f);
draw (a -- q2 -- r2 -- d -- h -- s2 -- t2 -- e);
draw (a -- l2 -- k2 -- b -- f -- j2 -- i2 -- e);
draw (d -- p2 -- o2 -- c -- g -- n2 -- m2 -- h);

draw (a2 -- l2 -- k2 -- b2);
draw (a2 -- l2 -- p2 -- d2);
draw (c2 -- o2 -- p2 -- d2);
draw (l2 -- q2 -- r2 -- p2);
draw (b2 -- k2 -- o2 -- c2);

draw (e2 -- i2 -- m2 -- h2 -- cycle);
draw (e2 -- i2 -- j2 -- f2 -- cycle);
draw (i2 -- t2 -- s2 -- m2 -- cycle);
draw (f2 -- g2 -- n2 -- j2 -- cycle);
draw (j2 -- x2 -- u2 -- k2 -- cycle);
draw (v2 -- o2 -- n2 -- w2 -- cycle);

draw (surface(a2 -- l2 -- k2 -- b2 -- cycle), verde*80);
draw (surface(a2 -- l2 -- p2 -- d2 -- cycle), verde*80);
draw (surface(c2 -- o2 -- p2 -- d2 -- cycle), verde*80);
draw (surface(l2 -- q2 -- r2 -- p2 -- cycle), verde*80);
draw (surface(b2 -- k2 -- o2 -- c2 -- cycle), verde*80);



draw (surface(a -- l2 -- k2 -- b -- cycle), verde*80);
draw (surface(b -- u2 -- v2 -- c -- cycle), verde*80);
draw (surface(c -- o2 -- p2 -- d -- cycle), verde*80);
draw (surface(a -- q2 -- r2 -- d -- cycle), verde*80);

draw (surface(b -- u2 -- v2 -- c -- cycle), verde*80);
draw (surface(b -- f -- x2 -- u2 -- cycle), verde*80);
draw (surface(f -- g -- w2 -- x2 -- cycle), verde*80);
draw (surface(g -- c -- v2 -- w2 -- cycle), verde*80);

draw (surface(c -- g -- n2 -- o2 -- cycle), verde*80);
draw (surface(g -- h -- m2 -- n2 -- cycle), verde*80);
draw (surface(h -- d -- p2 -- m2 -- cycle), verde*80);

draw (surface(e -- f -- f2 -- e2 -- cycle), verde*80);
draw (surface(f -- f2 -- g2 -- g -- cycle), verde*80);
draw (surface(g -- g2 -- h2 -- h -- cycle), verde*80);
draw (surface(h -- h2 -- e2 -- e -- cycle), verde*80);

draw (surface(a -- a2 -- b2 -- b -- cycle), verde*80);
draw (surface(b -- b2 -- c2 -- c -- cycle), verde*80);
draw (surface(c -- c2 -- d2 -- d -- cycle), verde*80);
draw (surface(d -- d2 -- a2 -- a -- cycle), verde*80);

draw (surface(a -- e -- i2 -- l2 -- cycle), verde*80);
draw (surface(e -- f -- j2 -- i2 -- cycle), verde*80);
draw (surface(b -- f -- j2 -- k2 -- cycle), verde*80);

draw (surface(e -- h -- s2 -- t2 -- cycle), verde*80);
draw (surface(h -- d -- r2 -- s2 -- cycle), verde*80);
draw (surface(e -- a -- q2 -- t2 -- cycle), verde*80);

draw (u2 -- k2 -- j2 -- x2);

draw (surface(u2 -- k2 -- j2 -- x2 -- cycle), verde*80);
draw (surface(k2 -- u2 -- v2 -- o2 -- cycle), verde*80);
draw (surface(l2 -- q2 -- t2 -- i2 -- cycle), verde*80);
draw (surface(e2 -- i2 -- m2 -- h2 -- cycle), verde*80);
draw (surface(e2 -- i2 -- j2 -- f2 -- cycle), verde*80);
draw (surface(i2 -- t2 -- s2 -- m2 -- cycle), verde*80);
draw (surface(s2 -- m2 -- p2 -- r2 -- cycle), verde*80);
draw (surface(h2 -- g2 -- n2 -- m2 -- cycle), verde*80);
draw (surface(f2 -- g2 -- n2 -- j2 -- cycle), verde*80);
draw (surface(j2 -- n2 -- w2 -- x2 -- cycle), verde*80);
draw (surface(j2 -- x2 -- u2 -- k2 -- cycle), verde*80);
draw (surface(v2 -- o2 -- n2 -- w2 -- cycle), verde*80);

draw((.5,.5,1.5) -- (.5,.5,1.2), arrow=Arrow3,blue);
\end{asy}
\\
(B)

\end{figure}
\begin{figure}[H]
\centering

Projeção
\\
em Perspectiva

\begin{asy}
size(4cm);
currentlight.background=box2;
currentprojection=perspective(.5,.5,1);

triple a = (0,0,0);
triple b = (1,0,0);
triple c = (1,1,0);
triple d = (0,1,0);

triple e = (0,0,1);
triple f = (1,0,1);
triple g = (1,1,1);
triple h = (0,1,1);

draw (a -- b -- c -- d -- cycle);
draw (e -- f -- g -- h -- cycle);
draw (a -- b -- f -- e -- cycle);
draw (c -- d -- h -- g -- cycle);

triple a1 = (0.1,0.1,0.1);
triple b1 = (.9,0.1,0.1);
triple c1 = (.9,.9,0.1);
triple d1 = (0.1,.9,0.1);

triple e1 = (0.1,0.1,.9);
triple f1 = (.9,0.1,.9);
triple g1 = (.9,.9,.9);
triple h1 = (0.1,.9,.9);

//draw (a1 -- b1 -- c1 -- d1 -- cycle);
//draw (e1 -- f1 -- g1 -- h1 -- cycle);
//draw (a1 -- b1 -- f1 -- e1 -- cycle);
//draw (c1 -- d1 -- h1 -- g1 -- cycle);

triple a2 = (0.1,0.1,0);
triple b2 = (.9,0.1,0);
triple c2 = (.9,.9,0);
triple d2 = (0.1,.9,0);

draw (a2 -- b2 -- c2 -- d2 -- cycle);

triple e2 = (0.1,0.1,1);
triple f2 = (.9,0.1,1);
triple g2 = (.9,.9,1);
triple h2 = (0.1,.9,1);

draw (e2 -- f2 -- g2 -- h2 -- cycle);

triple i2 = (0.1,0,.9);
triple j2 = (.9,0,.9);
triple k2 = (.9,0,.1);
triple l2 = (0.1,0,.1);

draw (i2 -- j2 --k2 -- l2 -- cycle);

triple m2 = (0.1,1,.9);
triple n2 = (.9,1,.9);
triple o2 = (.9,1,.1);
triple p2 = (0.1,1,.1);

draw (m2 -- n2 -- o2 -- p2 -- cycle);

triple q2 = (0,0.1,0.1);
triple r2 = (0,0.9,0.1);
triple s2 = (0,.9,.9);
triple t2 = (0,.1,.9);

draw (q2 -- r2 -- s2 --t2 -- cycle);

triple u2 = (1,0.1,0.1);
triple v2 = (1,0.9,0.1);
triple w2 = (1,.9,.9);
triple x2 = (1,.1,.9);

draw (u2 -- v2 -- w2 --x2 -- cycle);

draw (a -- a2 -- b2 -- b -- c -- c2 -- d2 -- d);
draw (e -- e2 -- f2 -- f -- g -- g2 -- h2 -- h);
draw (b -- u2 -- v2 -- c -- g -- w2 -- x2 -- f);
draw (a -- q2 -- r2 -- d -- h -- s2 -- t2 -- e);
draw (a -- l2 -- k2 -- b -- f -- j2 -- i2 -- e);
draw (d -- p2 -- o2 -- c -- g -- n2 -- m2 -- h);

draw (a2 -- l2 -- k2 -- b2);
draw (a2 -- l2 -- p2 -- d2);
draw (c2 -- o2 -- p2 -- d2);
draw (l2 -- q2 -- r2 -- p2);
draw (b2 -- k2 -- o2 -- c2);

draw (e2 -- i2 -- m2 -- h2 -- cycle);
draw (e2 -- i2 -- j2 -- f2 -- cycle);
draw (i2 -- t2 -- s2 -- m2 -- cycle);
draw (f2 -- g2 -- n2 -- j2 -- cycle);
draw (j2 -- x2 -- u2 -- k2 -- cycle);
draw (v2 -- o2 -- n2 -- w2 -- cycle);

draw (surface(a2 -- l2 -- k2 -- b2 -- cycle), verde*80);
draw (surface(a2 -- l2 -- p2 -- d2 -- cycle), verde*80);
draw (surface(c2 -- o2 -- p2 -- d2 -- cycle), verde*80);
draw (surface(l2 -- q2 -- r2 -- p2 -- cycle), verde*80);
draw (surface(b2 -- k2 -- o2 -- c2 -- cycle), verde*80);



draw (surface(a -- l2 -- k2 -- b -- cycle), verde*80);
draw (surface(b -- u2 -- v2 -- c -- cycle), verde*80);
draw (surface(c -- o2 -- p2 -- d -- cycle), verde*80);
draw (surface(a -- q2 -- r2 -- d -- cycle), verde*80);

draw (surface(b -- u2 -- v2 -- c -- cycle), verde*80);
draw (surface(b -- f -- x2 -- u2 -- cycle), verde*80);
draw (surface(f -- g -- w2 -- x2 -- cycle), verde*80);
draw (surface(g -- c -- v2 -- w2 -- cycle), verde*80);

draw (surface(c -- g -- n2 -- o2 -- cycle), verde*80);
draw (surface(g -- h -- m2 -- n2 -- cycle), verde*80);
draw (surface(h -- d -- p2 -- m2 -- cycle), verde*80);

draw (surface(e -- f -- f2 -- e2 -- cycle), verde*80);
draw (surface(f -- f2 -- g2 -- g -- cycle), verde*80);
draw (surface(g -- g2 -- h2 -- h -- cycle), verde*80);
draw (surface(h -- h2 -- e2 -- e -- cycle), verde*80);

draw (surface(a -- a2 -- b2 -- b -- cycle), verde*80);
draw (surface(b -- b2 -- c2 -- c -- cycle), verde*80);
draw (surface(c -- c2 -- d2 -- d -- cycle), verde*80);
draw (surface(d -- d2 -- a2 -- a -- cycle), verde*80);

draw (surface(a -- e -- i2 -- l2 -- cycle), verde*80);
draw (surface(e -- f -- j2 -- i2 -- cycle), verde*80);
draw (surface(b -- f -- j2 -- k2 -- cycle), verde*80);

draw (surface(e -- h -- s2 -- t2 -- cycle), verde*80);
draw (surface(h -- d -- r2 -- s2 -- cycle), verde*80);
draw (surface(e -- a -- q2 -- t2 -- cycle), verde*80);

draw (u2 -- k2 -- j2 -- x2);

draw (surface(u2 -- k2 -- j2 -- x2 -- cycle), verde*80);
draw (surface(k2 -- u2 -- v2 -- o2 -- cycle), verde*80);
draw (surface(l2 -- q2 -- t2 -- i2 -- cycle), verde*80);
draw (surface(e2 -- i2 -- m2 -- h2 -- cycle), verde*80);
draw (surface(e2 -- i2 -- j2 -- f2 -- cycle), verde*80);
draw (surface(i2 -- t2 -- s2 -- m2 -- cycle), verde*80);
draw (surface(s2 -- m2 -- p2 -- r2 -- cycle), verde*80);
draw (surface(h2 -- g2 -- n2 -- m2 -- cycle), verde*80);
draw (surface(f2 -- g2 -- n2 -- j2 -- cycle), verde*80);
draw (surface(j2 -- n2 -- w2 -- x2 -- cycle), verde*80);
draw (surface(j2 -- x2 -- u2 -- k2 -- cycle), verde*80);
draw (surface(v2 -- o2 -- n2 -- w2 -- cycle), verde*80);

draw((.5,.5,1.5) -- (.5,.5,1.2), arrow=Arrow3,blue);

\end{asy}
\\
(C)

\end{figure}
\end{multicols}

\caption{O que é uma \emph{vista}?}\label{\detokenize{GE301-6:fig-proj-vistas-observacao}}\label{\detokenize{GE301-6:id9}}
\end{figure}
% \begin{figure}[H]
% \centering
% \capstart

% \noindent\includegraphics[width=300bp]{{vistas-01_1}.jpg}
% \caption{O que é uma \emph{vista}?}\label{\detokenize{GE301-6:fig-proj-vistas-observacao}}\label{\detokenize{GE301-6:id9}}\end{figure}
\end{observation}

\begin{observation}{}

A Associação Brasileira de Normas Técnicas (ABNT) em seu documento NBR 10067 estabelece princípios gerais de representação em desenho técnico. Nele são consideradas seis vistas ortogonais simultâneas, três a mais ao que foi feito na \DUrole{xref,std,std-ref}{ativ-proj-vistas-ortogonais}. A \hyperref[\detokenize{GE301-6:fig-proj-vistas-ortogonais-06}]{Figura \ref{\detokenize{GE301-6:fig-proj-vistas-ortogonais-06}}} traz exatamente o exemplo apresentado neste documento. Uma versão interativa que pode ser acessada por meio de um navegador (inclusive o de seu celular) está disponível aqui: \textless{}\url{https://www.geogebra.org/m/SR7HrtkN}\textgreater{}.

\begin{figure}[H]
\centering
\begin{multicols}{2}
\begin{figure}[H]
\centering
\begin{asy}
size(5cm);

currentprojection=orthographic(.75,1,.5);
currentlight.background=box2;


triple A = (.4,.4,.35);
triple B = (.4,.7,.35);
triple C = (.7,.7,.35);
triple D = (.7,.4,.35);

triple E = (A+(0,0,.1));
triple F = (B+(0,0,.1));
triple G = (E+(.1,0,0));
triple H = (F+(.1,0,0));

triple I = (C+(0,0,.3));
triple J = (H+(0,0,.2));

triple K = (D+(0,0,.4));
triple L = (G+(0,0,.3));

triple M = (K+(0,.1,0));
triple N = (L+(0,.1,0));

draw(B -- C -- I -- J -- H -- F -- cycle);

draw(C -- D -- K -- M -- I -- cycle);
draw(D -- K -- L -- G -- E -- A -- cycle);
draw(K -- L -- N -- M -- cycle);
draw(M -- N -- J -- I -- cycle);

draw (A -- B -- C -- D -- cycle);
draw (E -- F -- H -- G -- cycle);
draw (A -- B -- F -- E -- cycle);

draw(surface(B -- C -- I -- J -- H -- F -- cycle), verde*80);

draw(surface(C -- D -- K -- M -- I -- cycle), verde*80);
draw(surface(D -- K -- L -- G -- E -- A -- cycle), verde*80);
draw(surface(K -- L -- N -- M -- cycle), verde*80);
draw(surface(M -- N -- J -- I -- cycle), verde*80);

draw (surface(A -- B -- C -- D -- cycle), verde*80);
draw (surface(E -- F -- H -- G -- cycle), verde*80);
draw (surface(A -- B -- F -- E -- cycle), verde*80);

draw (surface(H -- J -- N --L -- G -- cycle), verde*80);


triple x = (C - (.7,0,0));
triple w = (D - (.7,0,0));
triple p = (K - (.7,0,0));
triple q = (M - (.7,0,0));
triple y = (I - (.7,0,0));
triple z = (E - (.4,0,0));
triple u = (F - (.4,0,0));

draw (x -- w -- p --q --y -- cycle);
draw (z -- u, dashed);

draw (shift(1,0,0)*(x -- w -- p --q --y -- cycle));
draw (shift(1,0,0)*(z -- u));

triple x1 = (A + (0,-.4,0));
triple w1 = (D + (0,-.4,0));
triple p1 = (E + (0,-.4,0));
triple q1 = (G + (0,-.4,0));
triple y1 = (K + (0,-.4,0));
triple z1 = (L + (0,-.4,0));
triple u1 = (I + (0,-.7,0));
triple v1 = (J + (0,-.7,0));

draw (x1 -- w1 -- y1 -- z1 --q1 -- p1 -- cycle);
draw (u1 -- v1);

draw (shift(0,1,0)*(x1 -- w1 -- y1 -- z1 --q1 -- p1 -- cycle));
draw (shift(0,1,0)*(u1 -- v1), dashed);

triple x2 = (A + (0,0,-.35));
triple w2 = (B + (0,0,-.35));
triple p2 = (C + (0,0,-.35));
triple q2 = (D + (0,0,-.35));
triple y2 = (G + (0,0,-.45));
triple z2 = (H + (0,0,-.45));
triple u2 = (M + (0,0,-.75));
triple v2 = (N + (0,0,-.75));

draw (x2 -- w2 -- p2 -- q2 -- cycle);
draw (y2 -- z2);
draw (u2 -- v2);

draw (shift(0,0,1)*(x2 -- w2 -- p2 -- q2 -- cycle));
draw (shift(0,0,1)*(y2 -- z2), dashed);
draw (shift(0,0,1)*(u2 -- v2), dashed);

triple a = (0,0,0);
triple b = (1,0,0);
triple c = (1,1,0);
triple d = (0,1,0);

triple e = (0,0,1);
triple f = (1,0,1);
triple g = (1,1,1);
triple h = (0,1,1);

draw ((a -- b -- f -- e -- cycle));
draw ((a -- d -- h -- e -- cycle));
draw ((a -- b -- c -- d -- cycle));
draw ((b -- c -- g -- f -- cycle));
draw ((c -- d -- h -- g -- cycle));
\end{asy}

\end{figure}

\begin{figure}[H]
\centering

\begin{asy}
size(7cm);

currentprojection=orthographic(1.5,1.5,1/2);

triple A = (.4,.4,.35);
triple B = (.4,.7,.35);
triple C = (.7,.7,.35);
triple D = (.7,.4,.35);

triple E = (A+(0,0,.1));
triple F = (B+(0,0,.1));
triple G = (E+(.1,0,0));
triple H = (F+(.1,0,0));

triple I = (C+(0,0,.3));
triple J = (H+(0,0,.2));

triple K = (D+(0,0,.4));
triple L = (G+(0,0,.3));

triple M = (K+(0,.1,0));
triple N = (L+(0,.1,0));



pair a = (0,0);
pair b = (1,0);
pair c = (1,1);
pair d = (0,1);

pair A = (.7,.25);
pair B = (.7,.25);
pair C = (.7,.35);
pair D = (.6,.35);

pair E = (.6,.75);
pair F = (.4,.75);
pair G = (.4,.25);

pair H = (E - (0,.1));
pair I = (F - (0,.1));

draw ((a -- b -- c -- d -- cycle));
draw (shift(-.05)*(H -- I));
draw (shift(-.05)*(A -- B -- C -- D -- E -- F -- G -- cycle));

draw (reflect((2.5,0),(2.5,1))*shift(2,0)*(a -- b -- c -- d -- cycle));
draw (reflect((2.5,0),(2.5,1))*shift(2,0)*shift(-.05)*(H -- I), dashed);
draw (reflect((2.5,0),(2.5,1))*shift(2,0)*shift(-.05)*(A -- B -- C -- D -- E -- F -- G -- cycle));

pair A1 = (.4,.25);
pair B1 = (.7,.25);
pair C1 = (.7,.6);
pair D1 = (.5,.75);
pair E1 = (.4,.75);
pair F1 = (A1+(0,.1));
pair G1 = (B1+(0,.1));

draw (shift(1,0)*(a -- b -- c -- d -- cycle));

draw (shift(.95,0)*(A1 -- B1 -- C1 -- D1 -- E1) -- cycle);
draw (shift(.95,0)*(F1 -- G1), dashed);

draw (reflect((-.5,0),(-.5,1))*shift(-1,0)*(a -- b -- c -- d -- cycle));

draw (reflect((-.5,0),(-.5,1))*shift(-1.05,0)*(A1 -- B1 -- C1 -- D1 -- E1) -- cycle);
draw (reflect((-.5,0),(-.5,1))*shift(-1.05,0)*(F1 -- G1));

pair A2 = (.3,.3);
pair B2 = (.7,.3);
pair C2 = (.7,.7);
pair D2 = (.3,.7);
pair E2 = (.6,.7);
pair F2 = (.6,.3);
pair G2 = (.3,.6);
pair H2 = (.6,.6);

draw (shift(0,-1)*(a -- b -- c -- d -- cycle));

draw (shift(0,-1)*(A2 -- B2 -- C2 -- D2) -- cycle);
draw (shift(0,-1)*(E2 -- F2));
draw (shift(0,-1)*(G2 -- H2));

draw (shift(0,1)*(a -- b -- c -- d -- cycle));

draw (reflect((0,1.5),(1,1.5))*shift(0,1)*(A2 -- B2 -- C2 -- D2) -- cycle);
draw (reflect((0,1.5),(1,1.5))*shift(0,1)*(E2 -- F2), dashed);
draw (reflect((0,1.5),(1,1.5))*shift(0,1)*(G2 -- H2), dashed);


\end{asy}

\end{figure}
\end{multicols}
\begin{figure}[H]
\centering

\begin{asy}
size(8cm);

currentprojection=orthographic(.5,1,.5);
currentlight.background=box2;

triple a = (0,0,0);
triple b = (1,0,0);
triple c = (1,1,0);
triple d = (0,1,0);

triple e = (0,0,1);
triple f = (1,0,1);
triple g = (1,1,1);
triple h = (0,1,1);

triple ref1 = (rotate(45,a,e)*(d));
triple ref2 = (rotate(45,a,e)*(h));

triple A = (.4,.4,.35);
triple B = (.4,.7,.35);
triple C = (.7,.7,.35);
triple D = (.7,.4,.35);

triple E = (A+(0,0,.1));
triple F = (B+(0,0,.1));
triple G = (E+(.1,0,0));
triple H = (F+(.1,0,0));

triple I = (C+(0,0,.3));
triple J = (H+(0,0,.2));

triple K = (D+(0,0,.4));
triple L = (G+(0,0,.3));

triple M = (K+(0,.1,0));
triple N = (L+(0,.1,0));

draw(B -- C -- I -- J -- H -- F -- cycle);

draw(C -- D -- K -- M -- I -- cycle);
draw(D -- K -- L -- G -- E -- A -- cycle);
draw(K -- L -- N -- M -- cycle);
draw(M -- N -- J -- I -- cycle);

draw (A -- B -- C -- D -- cycle);
draw (E -- F -- H -- G -- cycle);
draw (A -- B -- F -- E -- cycle);

draw(surface(B -- C -- I -- J -- H -- F -- cycle), verde*80);

draw(surface(C -- D -- K -- M -- I -- cycle), verde*80);
draw(surface(D -- K -- L -- G -- E -- A -- cycle), verde*80);
draw(surface(K -- L -- N -- M -- cycle), verde*80);
draw(surface(M -- N -- J -- I -- cycle), verde*80);

draw (surface(A -- B -- C -- D -- cycle), verde*80);
draw (surface(E -- F -- H -- G -- cycle), verde*80);
draw (surface(A -- B -- F -- E -- cycle), verde*80);

draw (surface(H -- J -- N --L -- G -- cycle), verde*80);


triple x = (C - (.7,0,0));
triple w = (D - (.7,0,0));
triple p = (K - (.7,0,0));
triple q = (M - (.7,0,0));
triple y = (I - (.7,0,0));
triple z = (E - (.4,0,0));
triple u = (F - (.4,0,0));

draw (rotate(45,a,e)*(x -- w -- p --q --y -- cycle));
draw (rotate(45,a,e)*(z -- u), dashed);

draw (rotate(-45,b,f)*(shift(1,0,0)*(x -- w -- p --q --y -- cycle)));
draw (rotate(-45,b,f)*(shift(1,0,0)*(z -- u)));

triple x1 = (A + (0,-.4,0));
triple w1 = (D + (0,-.4,0));
triple p1 = (E + (0,-.4,0));
triple q1 = (G + (0,-.4,0));
triple y1 = (K + (0,-.4,0));
triple z1 = (L + (0,-.4,0));
triple u1 = (I + (0,-.7,0));
triple v1 = (J + (0,-.7,0));

draw (x1 -- w1 -- y1 -- z1 --q1 -- p1 -- cycle);
draw (u1 -- v1);

draw (rotate(100,ref1,ref2)*(shift(-sqrt(2)/2,-1+sqrt(2)/2,0)*shift(0,1,0)*(x1 -- w1 -- y1 -- z1 --q1 -- p1 -- cycle)));
draw (rotate(100,ref1,ref2)*(shift(-sqrt(2)/2,-1+sqrt(2)/2,0)*shift(0,1,0)*(u1 -- v1)), dashed);

triple x2 = (A + (0,0,-.35));
triple w2 = (B + (0,0,-.35));
triple p2 = (C + (0,0,-.35));
triple q2 = (D + (0,0,-.35));
triple y2 = (G + (0,0,-.45));
triple z2 = (H + (0,0,-.45));
triple u2 = (M + (0,0,-.75));
triple v2 = (N + (0,0,-.75));

draw (rotate(-45,a,b)*(x2 -- w2 -- p2 -- q2 -- cycle));
draw (rotate(-45,a,b)*(y2 -- z2));
draw (rotate(-45,a,b)*(u2 -- v2));

draw (rotate(45,e,f)*(shift(0,0,1)*(x2 -- w2 -- p2 -- q2 -- cycle)));
draw (rotate(45,e,f)*(shift(0,0,1)*(y2 -- z2)), dashed);
draw (rotate(45,e,f)*(shift(0,0,1)*(u2 -- v2)), dashed);



path3 Z1 = ((a -- b -- c -- d -- cycle));
path3 Z2 = ((e -- f -- g -- h -- cycle));
path3 X1 = ((a -- d -- h -- e -- cycle));
path3 X2 = ((b -- c -- g -- f -- cycle));
path3 Y1 = ((a -- b -- f -- e -- cycle));
path3 Y2 = ((c -- d -- h -- g -- cycle));



draw(rotate(-45, a, b)*Z1);
draw(rotate(45, e, f)*Z2);
draw(Y1);
draw(rotate(100,ref1,ref2)*(shift(-sqrt(2)/2,-1+sqrt(2)/2,0)*Y2));
draw(rotate(45,a, e)*X1);
draw(rotate(-45,b, f)*X2);

//draw (a -- b -- c -- d -- cycle);
//draw (e -- f -- g -- h -- cycle);
//draw (a -- b -- f -- e -- cycle);
//draw (c -- d -- h -- g -- cycle);
\end{asy}

\end{figure}

\caption{As seis vistas ortogonais do documento NBR 10067 da ABNT.}\label{\detokenize{GE301-6:fig-proj-vistas-ortogonais-06}}\label{\detokenize{GE301-6:id10}}
\end{figure}

% \begin{figure}[H]
% \centering
% \capstart

% \noindent\includegraphics[width=300bp]{{vistas-ortogonais-06}.jpg}
% \caption{As seis vistas ortogonais do documento NBR 10067 da ABNT.}\label{\detokenize{GE301-6:fig-proj-vistas-ortogonais-06}}\label{\detokenize{GE301-6:id10}}\end{figure}

Observamos que as definições das vistas ortogonais como definidas no documento da ABNT não são universais. Nos Estados Unidos, por exemplo, as projeções em cada par de planos paralelos no paralelepípedo da {\hyperref[\detokenize{GE301-6:fig-proj-vistas-ortogonais-06}]{\sphinxcrossref{\DUrole{std,std-ref}{As seis vistas ortogonais do documento NBR 10067 da ABNT.}}}} são permutados.
\end{observation}

\begin{knowledge}{}

Projeções ortogonais já foram um segredo militar!

A \index{Geometria Descritiva}Geometria Descritiva é o ramo da geometria que estuda a representação de objetos tridimensionais em duas dimensões através de um certo conjunto específico de procedimentos. As técnicas resultantes são importantes para a engenharia, a arquitetura, o design gráfico e as artes (\hyperref[\detokenize{GE301-6:fig-proj-geometria-descritiva-01}]{Figura \ref{\detokenize{GE301-6:fig-proj-geometria-descritiva-01}}}). A base teórica para a geometria descritiva é fornecida pelas projeções ortogonais.

\begin{figure}[H]
\centering
\capstart

\noindent\includegraphics[width=300bp]{{geometria-descritiva-01}.jpg}
\caption{Projeções ortogonais de um carro e de uma cabeça humana.}\label{\detokenize{GE301-6:fig-proj-geometria-descritiva-01}}\label{\detokenize{GE301-6:id11}}\end{figure}

O matemático francês Gaspard Monge (1746-1818) é considerado fundador da geometria descritiva. Ele a usou em engenharia militar (construção de fortificações) durante a época de Napoleão Bonaparte. De fato, geometria descritiva já foi considerada um segredo militar.

\begin{figure}[H]
\centering
\capstart

\noindent\includegraphics[width=200bp]{{gaspard-monge-01}.jpg}
\caption{Gaspard Monge (1746-1818).}\label{\detokenize{GE301-6:fig-proj-gaspar-monge-01}}\label{\detokenize{GE301-6:id12}}\end{figure}

Dennis Lieu e Sheryl Sorby, no excelente livro Visualization, Modeling, and Graphics for Engineering Design, apresentam o contexto histórico:
\begin{quote}

A pólvora começou a ser usada no mundo ocidental durante o Renascimento, assim como o canhão. Os canhões tornaram obsoletas a maioria das fortalezas construídas durante a era medieval. As muralhas não conseguiam suportar o impacto dos projéteis de canhão. Assim, as fortalezas precisavam ser remodeladas para suportar os tiros de canhão. Na França, um novo estilo de fortificação mais resistente foi então desenvolvido. A fortificação era construída com muros inclinados que ajudavam a defletir o tiro de canhão e não desmoronavam da mesma maneira que as muralhas planas verticais, quando atingidas diretamente. As novas fortalezas eram geometricamente mais complicadas de se construir do que suas predecessoras com muralhas verticais. Mais ainda, o perímetro da fortaleza evoluiu de um formato simples retangular para um formato pentagonal com uma extensão proeminente em cada ápice. Este formato de perímetro e o uso de muros inclinados resultaram em paredes que se justapunham em ângulos não usuais, os quais não podiam ser medidos facilmente ou diretamente. {[}…{]}

Felizmente, os franceses tinham Gaspard Monge, que desenvolveu uma técnica de análise gráfica chamada geometria descritiva. {[}…{]} As técnicas de geometria descritiva permitiram que os engenheiros da época criassem qualquer ponto de vista de um objeto geométrico a partir de dois pontos de vista existentes. Ao criar o ponto de vista apropriado, os engenheiros podiam observar e medir os atributos de um objeto. {[}…{]} A geometria complexa, os ângulos de interseção incomuns, e a altura das muralhas tinham a intenção de maximizar o fogo cruzado sobre um inimigo em aproximação sem revelar o interior da fortaleza. {[}…{]}

A astúcia dos franceses na construção de fortificações manteve a França como o principal poder europeu até o século XVIII. Na época, a geometria descritiva era considerada um segredo do estado francês, cuja divulgação era crime punível com a morte. Como resultado da aliança entre a França e o recém-constituído Estados Unidos, muitas fortificações dos EUA utilizaram projetos franceses. Como exemplo, temos o Forte McHenry que foi construído em 1806 e é primorosamente preservado em Baltimore, Maryland. O Forte McHenry sobreviveu ao bombardeamento inglês durante a Guerra de 1812 e tem importância porque ele inspirou Scott Key a escrever The Star Spangled Banner, o hino nacional dos EUA.
\end{quote}

\begin{figure}[H]
\centering
\capstart

\noindent\includegraphics[width=400bp]{{fig-fort-mchenry}.jpg}
\caption{Forte McHenry em Baltimore, Maryland, EUA (fonte: IAN Image and Video Library).}\label{\detokenize{GE301-6:fig-proj-forte-01}}\label{\detokenize{GE301-6:id13}}\end{figure}
\end{knowledge}

\begin{knowledge}{}

O cientista cognitivo americano Douglas Richard Hofstadter (1945-) concebeu, para a capa de seu livro “Gödel Escher Bach: Um Entrelaçamento de Gênios Brilhantes”, um objeto bem peculiar: suas projeções ortogonais em três planos produzem as letras “G” (de Gödel), “E” (de Escher) e “B” (de Bach).

\begin{figure}[H]
\centering
\capstart

\noindent\includegraphics[width=200bp]{{geb-01}.png}
\caption{GEB (fonte: \href{https://www.flickr.com/photos/maxbraun/3205365815}{Max Brown}).}\label{\detokenize{GE301-6:id14}}\end{figure}

Inspirado por esta ideia, o Projeto CDME da Universidade Federal Fluminense concebeu um jogo para praticar visualização espacial e vocabulário: para cada objeto, você deve identificar as letras formadas por projeções e dispô-las em uma ordem a fim de formar uma palavra sem acentos do dicionário. Além do Português, existem fases em Inglês, Espanhol e Francês! O jogo pode ser acessado de qualquer navegador, incluindo o do smartphone.

\begin{figure}[H]
\centering

\noindent\includegraphics[width=50bp]{{triplets-qr}.png}
\end{figure}

% \begin{figure}[H]
% \centering
% \capstart

% \noindent\includegraphics[width=300bp]{{triplets-exemplo}.png}
% \caption{Jogo \href{http://www.cdme.im-uff.mat.br/html5/triplets/triplets-html/triplets-br.html}{Trip-Lets} do Projeto CDME da UFF.}\label{\detokenize{GE301-6:id15}}\end{figure}
\end{knowledge}

\begin{knowledge}{}

Em softwares especializados de computação gráfica (Blender, Autocad, Autodesk 3DS Max, etc.), um recurso comum é o assim denominado \emph{quad view} que faz com que o programa exiba quatro janelas de visualização simultâneas: três janelas com as três vistas ortogonais principais mais uma quarta janela com a projeção em perspectiva. As janelas com as vistas ortogonais são usadas para uma interação mais precisa com o objeto 3D, enquanto que a janela com a projeção em perspectiva permite visualizar como o objeto será visto de uma posição arbitrária.
\begin{quote}

\begin{figure}[H]
\centering
\capstart

\noindent\includegraphics[width=400bp]{{blender-04}.jpg}
\caption{Sistema \emph{quadview} no software gratuito de computação gráfica Blender.}\label{\detokenize{GE301-6:id16}}\end{figure}
\end{quote}
\end{knowledge}

\begin{knowledge}{}

Existem perguntas sobre projeções ortogonais que a humanidade ainda não conhece as respostas. Vamos agora descrever uma dessas perguntas. Considere, a título de exemplo, um cubo que será projetado em um plano. Se você girar o cubo de forma que nenhuma de suas faces fique paralela à direção perpendicular a este plano de projeção, sua projeção será um polígono com 6 lados. Por este motivo, o cubo é denominado um \index{poliedro equiprojetivo}poliedro equiprojetivo de índice 6. Nesta construção interativa feita no GeoGebra \textless{}\url{https://www.geogebra.org/m/bF3y2m7K}\textgreater{} você pode constatar este fato.
\begin{quote}

\begin{figure}[H]
\centering

\noindent\includegraphics[width=250bp]{{poliedros-equiprojetivos-01}.png}
\end{figure}
\end{quote}

Mais geralmente, dizemos que um poliedro é equiprojetivo de índice \(k\) se ao girá-lo de forma que nenhuma de suas faces fique paralela à direção perpendicular ao plano de projeção, sua projeção sobre este plano sempre será um polígono com \(k\) lados.

Observe que existem poliedros que não são equiprojetivos. O tetraedro regular, por exemplo. A imagem a seguir mostra duas posições do tetraedro regular nas quais nenhuma de suas faces é paralela à reta perpendicular ao plano horizontal. Em uma dessas posições, sua projeção é um polígono de 3 lados. Na outra, a projeção é um polígono de 4 lados.

\begin{multicols}{2}
\begin{figure}[H]
\centering
\begin{asy}
size(5cm);

currentprojection=perspective(2.53,1.60,1.27);

triple a = (0,0,0);
triple b = (1,0,0);
triple c = (1,1,0);
triple d = (0,1,0);

triple e = (0,0,1);
triple f = (1,0,1);
triple g = (1,1,1);
triple h = (0,1,1);

draw ((a -- b -- f -- e -- cycle));
draw ((a -- d -- h -- e -- cycle));
draw ((a -- b -- c -- d -- cycle));


triple A = ((1-sqrt(3)/3)/2,1/6,0);
triple B = ((1-sqrt(3)/3)/2,5/6,0);
triple C = (sqrt(3)/3+(1-sqrt(3)/3)/2,1/2,0);

triple D = (sqrt(3)/3+(1-sqrt(3)/3)/2-sqrt(3)*2/9,1/2,0);

//draw(A -- D -- B -- D -- C);
//draw (A -- B -- C -- cycle);

triple Q = (0,0,(1-sqrt(24)/9)/2);


triple A1 = (A+Q);
triple B1 = (B+Q);
triple C1 = (C+Q);

draw(surface(A1 -- B1 -- C1 -- cycle), verde*80);

triple D1 = (sqrt(3)/3+(1-sqrt(3)/3)/2-sqrt(3)*2/9,1/2,(1-sqrt(24)/9)/2+sqrt(24)/9);

path3 p1 = (A1 -- B1 -- C1 -- cycle);
path3 p2 = (A1 -- D1 -- C1 -- cycle);
path3 p3 = (A1 -- D1 -- B1 -- cycle);
path3 p4 = (B1 -- D1 -- C1 -- cycle);

draw(p1);			
draw(p2);
draw(p3);
draw(p4);

draw(planeproject(XY*unitsquare3)*p1);
draw(planeproject(XY*unitsquare3)*p2);
draw(planeproject(XY*unitsquare3)*p3);

draw(surface(A1 -- D1 -- C1 -- cycle), verde*80+opacity(.9));
draw(surface(A1 -- D1 -- B1 -- cycle), verde*80+opacity(.9));
draw(surface(B1 -- D1 -- C1 -- cycle), verde*80+opacity(.9));

draw((A1 -- B1 -- C1 -- cycle));
draw((A1 -- D1 -- C1 -- cycle));
draw((A1 -- D1 -- B1 -- cycle));
draw((B1 -- D1 -- C1 -- cycle));
\end{asy}
\caption{A projeção do tetraedo regular no plano horizontal formando um polígono de 3 lados}
\end{figure}

\begin{figure}[H]
\centering
\begin{asy}
size(5cm);

currentprojection=perspective(2.53,1.60,1.27);

triple a = (0,0,0);
triple b = (1,0,0);
triple c = (1,1,0);
triple d = (0,1,0);

triple e = (0,0,1);
triple f = (1,0,1);
triple g = (1,1,1);
triple h = (0,1,1);

draw ((a -- b -- f -- e -- cycle));
draw ((a -- d -- h -- e -- cycle));
draw ((a -- b -- c -- d -- cycle));


triple A = ((1-sqrt(3)/3)/2,1/6,0);
triple B = ((1-sqrt(3)/3)/2,5/6,0);
triple C = (sqrt(3)/3+(1-sqrt(3)/3)/2,1/2,0);

triple D = (sqrt(3)/3+(1-sqrt(3)/3)/2-sqrt(3)*2/9,1/2,0);

//draw(A -- D -- B -- D -- C);
//draw (A -- B -- C -- cycle);

triple Q = (0,0,(1-sqrt(24)/9)/2);

triple r1 = (.5,0,.5);
triple r2 = (.5,1,.5);


triple A1 = rotate(-54.5,r1,r2)*(A+Q);
triple B1 = rotate(-54.5,r1,r2)*(B+Q);
triple C1 = rotate(-54.5,r1,r2)*(C+Q);

triple D1 = rotate(-54.5,r1,r2)*(sqrt(3)/3+(1-sqrt(3)/3)/2-sqrt(3)*2/9,1/2,(1-sqrt(24)/9)/2+sqrt(24)/9);

path3 p1 = (A1 -- B1 -- C1 -- cycle);
path3 p2 = (A1 -- D1 -- C1 -- cycle);
path3 p3 = (A1 -- D1 -- B1 -- cycle);
path3 p4 = (B1 -- D1 -- C1 -- cycle);

draw(shift(0,0,.1)*p1);			
draw(shift(0,0,.1)*p2);
draw(shift(0,0,.1)*p3);
draw(shift(0,0,.1)*p4);

draw(planeproject(XY*unitsquare3)*p1);
draw(planeproject(XY*unitsquare3)*p2);
draw(planeproject(XY*unitsquare3)*p3);
draw(planeproject(XY*unitsquare3)*p4);

draw(shift(0,0,.1)*surface(A1 -- B1 -- C1 -- cycle), verde*80+opacity(.5));

draw(shift(0,0,.1)*surface(A1 -- D1 -- C1 -- cycle), verde*80+opacity(.9));
draw(shift(0,0,.1)*surface(A1 -- D1 -- B1 -- cycle), verde*80+opacity(.9));
draw(shift(0,0,.1)*surface(B1 -- D1 -- C1 -- cycle), verde*80+opacity(.9));

//draw((A1 -- B1 -- C1 -- cycle));
//draw((A1 -- D1 -- C1 -- cycle));
//draw((A1 -- D1 -- B1 -- cycle));
//draw((B1 -- D1 -- C1 -- cycle));
\end{asy}
\caption{A projeção do tetraedo no plano horizontal formando um polígono de 4 lados}
\end{figure}

\end{multicols}

Dado um valor \(\geq 3\) para \(k\), quais são todos os poliedros equiprojetivos de índice \(k\)? Esta é uma pergunta que ninguém conseguiu responder até o presente momento. Ela é um entre muitos problemas em aberto em Matemática.
\end{knowledge}


\know{}
\label{\detokenize{GE301-A::doc}}\label{\detokenize{GE301-A:sec-proj-saber-mais-e-projetos}}\label{\detokenize{GE301-A:para-saber-mais-e-sugestoes-de-projetos}}

\subsection{Projeção mapeada}
\label{\detokenize{GE301-A:sub-projecaomapeada}}\label{\detokenize{GE301-A:projecao-mapeada}}
Projeção mapeada (\emph{mapping projection}, em inglês) é um termo empregado para designar um conjunto de técnicas usadas para projetar vídeos sobre superfícies diversas tais como fachadas de prédios, monumentos, corpo humano e etc. Com esta técnica é possível cobrir um objeto real com imagens que se adaptam perfeitamente às irregularidades da superfície de projeção, podendo assim criar ilusões de ótica, dar vida a objetos estáticos, obter texturas e efeitos tridimensionais, e etc. Para isso, são necessários projetores e softwares especializados que reconstroem o espaço real através da adição de um espaço virtual cuidadosamente projetado.

A projeção em perspectiva que você aprendeu neste capítulo é a utilizada nesta técnica. Neste caso, os vídeos são criados respeitando a perspectiva da superfície de projeção em relação ao projetor e, também, suas características como tamanho, forma e etc.

Durante a cerimônia de abertura dos jogos olímpicos de 2016 na cidade do Rio de Janeiro, a projeção mapeada foi utilizada em diversos momentos. 106 projetores foram usados para cobrir com imagens o campo do estádio do Maracanã (onde ocorreu o evento) e outros objetos cenográficos criando um cenário virtual rico em detalhes, cores e emoção.  As imagens abaixo, mostram dois momentos diferentes do belíssimo espetáculo proporcionado pelas projeções mapeadas.

\begin{figure}[H]
\centering
\capstart

\noindent\includegraphics[width=400bp]{{OpeningCerimonyRio2016_montagem}.jpg}
\caption{Imagens da cerimônia de abertura dos jogos olímpicos de 2016 na cidade do Rio de Janeiro (Fonte: \href{http://radiografico.com.br/projects/93}{site radiografico}).}\label{\detokenize{GE301-A:id2}}\end{figure}

Esta também é a técnica usada pela Walt Disney World em diversos shows e atrações de seus parques de diversão desde 1969. Um exemplo, é o show \emph{Happily ever after}, onde o Castelo da Cinderela (localizado no parque \emph{Magic Kingdon} em Orlando-EUA) se transforma em uma enorme tela de projeção que reproduz a história de diferentes personagens da Disney. Para saber sobre o processo de criação das projeções no castelo, recomendamos o vídeo \emph{Advanced Projection Mapping Tech Coming to ‘Happily Ever After’} disponibilizado pela Disney Parks em seu \href{https://www.youtube.com/watch?v=WGrKuosEkWw}{canal do YouTube}. O vídeo tem duração de 2 minutos e as legendas em inglês podem ser ativadas.

\begin{figure}[H]
\centering
\capstart

\noindent\includegraphics[width=350bp]{{CasteloDisney3}.jpg}
\caption{Vídeo \emph{Advanced Projection Mapping Tech Coming to ‘Happily Ever After’} disponível em \textless{}\url{https://www.youtube.com/watch?v=WGrKuosEkWw}\textgreater{}.}\label{\detokenize{GE301-A:id3}}\end{figure}

O Cristo Redentor, famoso ponto turístico da cidade do Rio de Janeiro, fechou os braços em um abraço simbólico à cidade através de projeção mapeada. Esse abraço foi parte da campanha de combate à violência e exploração sexual de crianças “Carinho de Verdade”, idealizada pelo cineasta brasileiro Fernando Salis em 2010. Embalados pelo som das Bachianas Brasileiras número 7 de Villa Lobos, oito projetores cobriram a estátua do Cristo com imagens do Rio de Janeiro e criaram uma ilusão de movimento dos braços da estátua. Um casamento perfeito entre Arte, Matemática e Tecnologia!

Para assistir a todas as projeções feitas no Cristo Redentor nesta campanha, sugerimos o vídeo \emph{Projeção do abraço do Cristo no Rio, de Fernando Salis 19/10/2010} disponibilizado pelo próprio cineasta em \href{https://www.youtube.com/watch?v=PNzi5JS46U8}{seu canal do YouTube}. Não se esqueça de ativar o som para assistir esse vídeo que vai te impressionar!

\begin{figure}[H]
\centering
\capstart

\noindent\includegraphics[width=300bp]{{AbracoCristo}.jpg}
\caption{Vídeo \emph{Projeção do abraço do Cristo no Rio, de Fernando Salis 19/10/2010} disponível em \textless{}\url{https://www.youtube.com/watch?v=PNzi5JS46U8}\textgreater{}.}\label{\detokenize{GE301-A:id4}}\end{figure}

Um outro exemplo de utilização desta técnica pode ser visto no vídeo \emph{Omote real times face tracking \& projection mapping}, também \href{https://www.youtube.com/watch?v=eVNDYgMrvUU}{disponível no YouTube}. Neste caso, a superfície de projeção escolhida é a face de uma pessoa, que se transforma através de diferentes texturas. O vídeo tem duração de 2:18 minutos e não precisa de legendas.

\begin{figure}[H]
\centering
\capstart

\noindent\includegraphics[width=350bp]{{FaceProjection}.jpg}
\caption{Vídeo \emph{Omote real times face tracking \& projection mapping} disponível em \textless{}\url{https://www.youtube.com/watch?v=eVNDYgMrvUU}\textgreater{}.}\label{\detokenize{GE301-A:id5}}\end{figure}

A seguir, listaremos uma série de vídeos da plataforma YouTube que ilustram o uso desta técnica impressionante:
\begin{enumerate}
\item {} 
\url{https://www.youtube.com/watch?v=lX6JcybgDFo}

\item {} 
\url{https://www.youtube.com/watch?v=D6EPGutC9Z0}

\item {} 
\url{https://youtu.be/PKMCB5v8pt0}

\item {} 
\url{https://www.youtube.com/watch?v=P1az8bbuOLg}

\end{enumerate}

Se você ficou com vontade de tentar utilizar a projeção mapeada, sugerimos acessar o site \url{http://projection-mapping.org/} e escolher um dos softwares disponíveis de acordo com seu sistema operacional. Divirta-se!


\subsection{Projeções no cinema}
\label{\detokenize{GE301-A:projecoes-no-cinema}}\label{\detokenize{GE301-A:sub-cinema}}
Para criar efeitos visuais nos filmes, é muito comum usar técnicas que envolvem projeção em perspectiva. Uma técnica bastante conhecida é chamada perspectiva forçada. A perspectiva forçada é uma ilusão de ótica que faz com que objetos pareçam maiores/menores do que são ou mais próximos/distantes uns dos outros.

Esta técnica foi amplamente utilizada nos filmes \emph{O Senhor dos Anéis}, uma trilogia baseada nos livros do britânico J. R. R. Tolkien. Em cenas onde aparecem os \emph{Hobbits} (criaturas pequenas se comparadas com humanos mas que foram encenadas por humanos), eles eram posicionados mais distantes da câmera enquanto os outros personagens da mesma cena eram posicionados mais próximos da câmera. Na perspectiva da câmera, quem está mais perto dela é maior do que quem está mais longe. E assim, os \emph{Hobbits} ficaram de tamanho reduzido no filme! Na cena mostrada na figura abaixo, duas mesas são utilizadas para haver afastamento dos personagens em relação à câmera, mas no filme essas duas mesas parecem uma só. Se você nunca assistiu essa trilogia, sugerimos que o faça e fique atento para os truques de perspectiva usados no filme. A Matemática realmente está presente onde menos esperamos!

\begin{figure}[H]
\centering
\capstart

\noindent\includegraphics[width=400bp]{{LordOfRings}.jpg}
\caption{Vídeo \emph{How Lord of the Rings used forced perspective shots with a moving camera} disponível em \textless{}\url{https://www.youtube.com/watch?v=QWMFpxkGO\_s}\textgreater{}.}\label{\detokenize{GE301-A:id6}}\end{figure}

Se você quiser entender todos os detalhes da perspectiva forçada, assista o vídeo \emph{The Math and Science of Forced Perspective} disponível em \textless{}\url{https://www.youtube.com/watch?v=pl4ah\_HvWkg\&t=187s}\textgreater{}. O vídeo tem cerca de 15 minutos e possui legendas em inglês.

Outro efeito utilizado no cinema que faz uso de projeções é o Efeito Vertigo (\emph{Dolly Zoom}, em inglês). Esse efeito é produzido quando aumentamos ou diminuímos o alcance da lente através do zoom enquanto alteramos a posição da câmera na direção oposta ao zoom. Assim, o personagem da cena permanece no foco, enquanto a perspectiva visual muda com a aproximação ou afastamento  do cenário.

% \begin{figure}[H]
% \centering
% \capstart

% \noindent\includegraphics[width=300bp]{{Vertigo}.png}
% \caption{Cena da torre do sino do filme Vertigo que utilizou a técnica (Fonte: \url{https://www.youtube.com/watch?v=sKJeTaIEldM}).}\label{\detokenize{GE301-A:id7}}\end{figure}

Utilizado no famoso filme \emph{Vertigo} dirigido por Alfred Hitchcock, que deu nome ao efeito aqui no Brasil, as cenas da torre do sino da capela e da perseguição policial que acabou com a morte de um policial são excelentes exemplos de uso dessa técnica. O movimento simultâneo da câmera e o uso do zoom causam uma distorção visual, gerando até mesmo náuseas em quem assiste a cena. O filme recebeu muitas críticas na ocasião de sua estréia, mas hoje é conhecido como uma obra de arte de Hitchcock.

\begin{figure}[H]
\centering
\capstart

\noindent\includegraphics[width=200bp]{{AlfredHitchcock}.jpg}
\caption{Alfred Hitchcock, o célebre diretor de cinema, o primeiro a utilizar o efeito Vertigo no cinema (Fonte: Wikimedia Commons).}\label{\detokenize{GE301-A:id8}}\end{figure}

Para saber mais, vamos listar alguns vídeos da plataforma YouTube que podem ajudá-lo a entender melhor o efeito criado com esta técnica:
\begin{enumerate}
\item {} 
\url{https://www.youtube.com/watch?v=neaOds5\_3js} disponível com áudio em português.

\item {} 
\url{https://www.youtube.com/watch?v=sKJeTaIEldM}

\item {} 
\url{https://www.youtube.com/watch?v=WIpMtL68G8w}

\end{enumerate}

Você poder tentar simular este efeito usando uma câmera. Se reúna com um colega, organize uma cena interessante e tente usar o \emph{zoom in} ao mesmo tempo que se afasta da cena, ou então, o contrário. Cuidado com o efeito vertigo!


\subsection{Jogos que utilizam projeções}
\label{\detokenize{GE301-A:jogos-que-utilizam-projecoes}}\label{\detokenize{GE301-A:sub-jogos}}
Vários jogos concretos ou digitais utilizam projeções, sejam elas apenas para criar a cena do jogo ou como uma componente da dinâmica do jogo. E, em alguns, as ambiguidades existentes nas projeções em perspectiva são o tema principal.  É claro que as projeções, neste caso, aparecem de uma forma mais informal do que a estudada neste capítulo, mas por trás da confecção do jogo, toda a Matemática aqui discutida certamente foi utilizada.

Os jogos para \emph{Playstation} chamados \emph{Echochrome} e \emph{Echochrome II} foram os pioneiros em utilizar efeitos de ilusão de ótica em sua dinâmica. Estes dois jogos inspiraram a criação de muitos jogos que conhecemos hoje. Em \emph{Echochrome}, o objetivo é levar o personagem através de um caminho do começo ao fim. Para isso, são necessárias mudanças de perspectiva que transformam caminhos impossíveis em factíveis.

% \begin{figure}[H]
% \centering
% \capstart

% \noindent\includegraphics[width=300bp]{{Echochrome3}.png}
% \caption{Jogo \emph{Echochrome} (Fonte: \url{https://www.youtube.com/watch?v=GybxIwfU4rI})}\label{\detokenize{GE301-A:id9}}\end{figure}

Já em \emph{Echochrome II}, para conduzir o personagem do início ao fim do caminho, são utilizadas luzes e sombras, que criam as ilusões de ótica por onde o personagem deve caminhar. Cada jogador enxerga de uma forma diferente os enigmas presentes no jogo, e portanto, existem diferentes possibilidades de condução do personagem.

% \begin{figure}[H]
% \centering
% \capstart

% \noindent\includegraphics[width=300bp]{{EchochromeII3}.png}
% \caption{Jogo \emph{Echochrome II} (Fonte: \url{https://www.youtube.com/watch?v=bWMSpmqVUOY})}\label{\detokenize{GE301-A:id10}}\end{figure}

A seguir, vamos listar alguns jogos que envolvem projeção. Escolha um jogo, utilize tudo que você aprendeu até aqui e se divirta!

\textbf{Jogos para vídeo-games:}
\begin{itemize}
\item {} 
Fez: este é um jogo do tipo quebra-cabeça onde o personagem principal, chamado Gomez, imagina viver em um mundo 2D, mas ao receber um chapéu \emph{Fez} percebe que o mundo é 3D. O jogador tem que ajudar Gomez a viver nesse mundo novo considerando as projeções do mundo 3D que possui. O objetivo é realinhar plataformas e resolver charadas para restaurar a ordem do universo.

\end{itemize}

Para entender melhor o funcionamento do jogo, assista o vídeo disponível em \textless{}\url{https://www.youtube.com/watch?v=HFNIH3m6i2s}\textgreater{}. Este vídeo possui áudio em português e cerca de 11 minutos.

Este jogo está disponível para \emph{Playstation} e \emph{Xbox}.

\begin{figure}[H]
\centering
\capstart

\noindent\includegraphics[width=450bp]{{Fez}.png}
\caption{Jogo Fez (Imagens de divulgação)}\label{\detokenize{GE301-A:id11}}\end{figure}
\begin{itemize}
\item {} 
The Bridge: neste jogo, o objetivo é levar o personagem principal Escher para a porta de saída. O caminho a ser percorrido é cheio de  enigmas que são inspirados nas obras do artista M. C. Escher. O jogador tem que desafiar a gravidade, girar a cena, abrir portas, subir escadas e etc para assim conduzir o personagem pelo caminho da saída.

\end{itemize}

% \begin{figure}[H]
% \centering
% \capstart

% \noindent\includegraphics[width=300bp]{{TheBridge2}.png}
% \caption{Jogo \emph{The Bridge} (Fonte: \url{https://www.youtube.com/watch?v=h8hOGbdoJdw})}\label{\detokenize{GE301-A:id12}}\end{figure}

Este jogo está disponível para \emph{Playstation} e \emph{Xbox}, assim como para computadores e celulares.

\textbf{Jogos para celular:}
\begin{itemize}
\item {} 
Monument Valley: neste jogo, o jogador deve resolver quebra-cabeças inspirados nas criações do artista M. C. Escher. É claro que os cenários são cheios de ilusões de ótica e arquiteturas que desafiam a lógica. O jogo está disponível para IOS e Android.

\end{itemize}

\begin{figure}[H]
\centering
\capstart

\noindent\includegraphics[width=200bp]{{MonumentValley}.png}
\caption{Jogo Monument Valley (Fonte: \url{https://www.monumentvalleygame.com/})}\label{\detokenize{GE301-A:id13}}\end{figure}
\begin{itemize}
\item {} 
Shadowmatic: o objetivo deste jogo é trabalhar com as sombras de objetos projetadas sobre uma parede provocadas por uma fonte de luz fora da tela. Os objetos que são projetados normalmente não se assemelham a nenhum objeto real e o jogador deve movimentá-lo até a sombra formar uma silhueta reconhecível. O jogo está disponível para IOS e Android.

\begin{figure}[H]
\centering
\capstart

\noindent\includegraphics[width=300bp]{{Shadowmatic}.png}
\caption{Jogo Shadowmatic (Fonte: \url{https://www.shadowmatic.com/})}\label{\detokenize{GE301-A:id14}}\end{figure}

\end{itemize}



\textbf{Jogo para computador:}
\begin{itemize}
\item {} 
Perspective: neste jogo, o jogador deve mover o avatar em um cenário 3D que muda de acordo com a perspectiva. Nele nada é impossível, e sim questão de perspectiva! O jogo é gratuito e pode ser acessado no site \textless{}\url{http://games.digipen.edu/games/perspective\#.WnW81ZOpmCQ.}\textgreater{}

\end{itemize}

% \begin{figure}[H]
% \centering
% \capstart

% \noindent\includegraphics[width=320bp]{{Perspective2}.png}
% \caption{Jogo Perspective (Fonte: \url{https://www.youtube.com/watch?v=SS4r9Fq3beU\&t=25s})}\label{\detokenize{GE301-A:id15}}\end{figure}

\textbf{Jogos concretos:}
\begin{itemize}
\item {} 
La Boca: neste jogo, o jogador escolhe um outro jogador dentre os demais competidores que o ajudará a construir um sólido geométrico com um conjunto de blocos menores disponibilizados pelo jogo. Cada um dos dois jogadores terá acesso à informações de apenas uma face do sólido (uma vista do sólido), sendo uma oposta à outra, e assim, cooperativamente eles devem fazer a construção.

\end{itemize}

\begin{figure}[H]
\centering
\capstart

\noindent\includegraphics[width=250bp]{{LaBocaJogo}.jpg}
\caption{Jogo La Boca produzido pela empresa alemã Kosmos (Imagem de divulgação)}\label{\detokenize{GE301-A:id16}}\end{figure}

O vídeo em português disponível em \textless{}\url{https://www.youtube.com/watch?v=n7yiM\_zak0Y}\textgreater{} vai te ajudar a entender melhor o andamento do jogo. São apenas 3 minutos de vídeo. Vale a pena assistir!

OBS.: Se você ficou curioso com o nome do jogo, saiba que ele é inspirado na rua \emph{Caminito} do bairro \emph{La Boca}, que fica localizado em Buenos Aires-Argentina. Esta é uma rua com muitas casas com fachadas coloridas e ponto turístico certo de quem visita a cidade.
\begin{itemize}
\item {} 
Papertown: neste jogo, dois jogadores que são adversários, devem construir uma cidade (chamada \emph{Paper Town}, que traduzindo para português seria \emph{Cidade de Papel}) com peças em papel que apresentam partes da cidade desenhadas em perspectiva. A perspectiva deve ser respeitada durante todo o decorrer da partida. Esse é um jogo que envolve criatividade, imaginação e muita geometria.

\end{itemize}

\begin{figure}[H]
\centering
\capstart

\noindent\includegraphics[width=300bp]{{Papertown}.png}
\caption{Jogo Papertown produzido pela editora brasileira RedBox (Fonte: \url{http://rodrigorego.com.br/papertown.html})}\label{\detokenize{GE301-A:id17}}\end{figure}

Assista a uma partida do jogo disponível em \textless{}\url{https://www.youtube.com/watch?v=b\_uhElq1sWM}\textgreater{}. O vídeo possui áudio em português e cerca de 33 minutos.

\textbf{Jogos em desenvolvimento:}
\begin{itemize}
\item {} 
Graybles: neste jogo são mostradas várias perspectivas de uma mesma cena e o jogador deve percorrer um caminho correto usando informações de todas as perspectivas. Veja o vídeo disponível em \textless{}\url{https://www.youtube.com/watch?v=ub3UM30-vcI}\textgreater{} para uma demonstração do jogo. O vídeo é bem rápido e possui legendas em inglês.

\item {} 
Pillow Castle: neste jogo é utilizada a projeção forçada (já discutida nesta seção) para criar ilusões de ótica e assim montar diferentes cenários. Este vídeo com legendas em inglês com duranção de cerca de 7 minutos vai te surpreender: \textless{}\url{https://www.youtube.com/watch?v=HOfll06X16c}\textgreater{}.

\end{itemize}


\subsection{Teatro de sombras}
\label{\detokenize{GE301-A:teatro-de-sombras}}\label{\detokenize{GE301-A:sub-teatrodesombras}}
Não é apenas em plataformas digitais que as projeções podem ser úteis. Vamos entender um pouco como as projeções podem ser utilizadas para criar espetáculos de teatro, chamados teatro de sombras, que são vistos como os precursores do cinema.

O \emph{teatro de sombras} é uma forma bem antiga de contar histórias com o auxílio de sombras, criadas por bonecos, que dão vida aos personagens da história. Uma fonte de luz incide sobre uma tela translúcida que oculta os bonecos, deixando visíveis apenas suas sombras. Os bonecos são controlados por pessoas que também ficam ocultas durante o espetáculo e que, normalmente, também confeccionam os bonecos. A história é contada utilizando as sombras geradas tanto pela movimentação dos bonecos quanto da fonte de luz.

No vídeo \emph{Traditional Chinese Shadow Puppet Performance, Bazhong, China} disponível neste \href{https://vimeo.com/41524173}{link} é possível assistir a um teatro de sombras tradicional da China. Neste caso, a fonte de luz é o próprio sol e os personagens são as sombras de bonecos confeccionados com tecido, papel e pequenos gravetos que são usados para sua manipulação. O vídeo possui o som do ambiente, produzido ao vivo por artistas da região.

\begin{figure}[H]
\centering
\capstart

\noindent\includegraphics[width=300bp]{{TradicionalChineseShadowPuppetPerformance}.png}
\caption{Vídeo \emph{Traditional Chinese Shadow Puppet Performance, Bazhong, China} \textless{}\url{https://vimeo.com/41524173}\textgreater{}.}\label{\detokenize{GE301-A:id18}}\end{figure}

Esta forma de arte remonta da pré-história quando o homem se encantava com suas sombras projetadas nas paredes de cavernas, mas não há um consenso sobre a origem exata do que chamamos de teatro de sombras. Há uma lenda que se passa no ano 121 na China que pode ser a origem dessa arte milenar. Segundo a lenda, o imperador Wu Ti, da dinastia Han, se desesperou com a morte de sua bailarina favorita e, então, ordenou ao mago da corte que a trouxesse de volta do “Reino das Sombras”, caso contrário ele seria decapitado. O mago então confeccionou a silhueta da bailarina com pele de peixe e, usando um lençol que deixava transparecer a luz do sol, ao som de uma flauta, recriou os movimentos leves e graciosos da bailarina no jardim do castelo do imperador. Neste momento, é possível que tenha surgido uma das mais antigas formas de projeção, que nos remete ao que estudamos neste capítulo.

% \begin{figure}[H]
% \centering
% \capstart

% \noindent\includegraphics[width=230bp]{{TeatroSombra_Lenda}.png}
% \caption{Vídeo \emph{Teatro de Sombras} \textless{}\url{https://www.youtube.com/watch?v=QXMlVgNquNs}\textgreater{} que mostra uma encenação da lenda.}\label{\detokenize{GE301-A:id19}}\end{figure}

Em 2011, o teatro de sombras da China foi denominado Patrimônio Cultural da Humanidade pela \emph{Unesco}. Este tipo de arte ainda é muito comum no país, onde a habilidade de manipular simultaneamente vários bonecos passa de pai para filho. Segundo a Unesco, este tipo de arte difunde conhecimento, promove valores culturais e entretém a comunidade. Sugerimos que você assista o \href{https://www.youtube.com/watch?v=8-mzqxZNp2g}{vídeo disponível} no canal da Unesco do YouTube. Você vai se impressionar com a habilidade dos manipuladores!

% \begin{figure}[H]
% \centering
% \capstart

% \noindent\includegraphics[width=230bp]{{Unesco}.png}
% \caption{Vídeo \emph{Chinese shadow puppetry} \textless{}\url{https://www.youtube.com/watch?v=8-mzqxZNp2g}\textgreater{}.}\label{\detokenize{GE301-A:id20}}\end{figure}

Em sua trajetória histórica, o teatro de sombras adquiriu características de acordo com as diversas culturas das regiões que o produziram e se popularizou especialmente na Ásia, em países como China, Indonésia, Malásia, Tailândia, Camboja, Índia e Nepal. Se você quiser saber um pouco mais sobre a história dessa arte e seus desdobramentos nos diversos países, sugerimos o livro \emph{Shadow Puppets and Shadow Play} de David Currel.

Nos dias atuais, algumas companhias de dança e teatro espalhadas pelo mundo continuam a criar espetáculos de sombra utilizando mais que bonecos e a luz do sol. É possível encontrar espaços destinados apenas a este tipo de arte, como o teatro russo \emph{Shadow Fireflies Theater} que possui um grupo de artistas fixos que encenam seus espetáculos. As sombras utilizadas nestes espetáculos são dos corpos dos próprios artistas que criam silhuetas humanas ou objetos diversos para contar suas histórias. Recomendamos, em especial, o vídeo \emph{Shadow Theatre “Fireflies” - New Year’s Dream} disponível do canal do teatro no \href{https://www.youtube.com/watch?v=AzS0VwXOlWs}{YouTube}.

% \begin{figure}[H]
% \centering
% \capstart

% \noindent\includegraphics[width=300bp]{{Russia}.png}
% \caption{Vídeo \emph{Shadow Theatre “Fireflies” - New Year’s Dream} \textless{}\url{https://www.youtube.com/watch?v=AzS0VwXOlWs}\textgreater{}.}\label{\detokenize{GE301-A:id21}}\end{figure}

No programa de TV inglês \emph{Britain’s Got Talent 2013}, o grupo de teatro de sombras húngaro chamado \emph{Attraction} emocionou os jurados com uma apresentação ao vivo de tirar o fôlego. Você pode assistir a esta apresentação na íntegra no canal do programa no \href{https://www.youtube.com/watch?v=JOZS\_Vq6eKw}{YouTube}. Vale a pena conferir!

% \begin{figure}[H]
% \centering
% \capstart

% \noindent\includegraphics[width=300bp]{{BGT}.png}
% \caption{Vídeo \emph{Attraction’s semi-final shadow theatre performance\textbar{} Semi-Final 5\textbar{}Britain’s Got Talent 2013} \textless{}\url{https://www.youtube.com/watch?v=JOZS\_Vq6eKw}\textgreater{}.}\label{\detokenize{GE301-A:id22}}\end{figure}

E se você ficou com vontade de encenar um espetáculo com seus amigos ou na sua escola, sugerimos que você assista os vídeos abaixo. Para garantir o efeito das sombras, você deve usar uma lanterna e uma parede como tela de projeção semelhante ao que foi feito na atividade da lanterna na seção X. Você também pode construir um mini teatro com materiais reciclados como mostra um dos vídeos a seguir e tentar usar até mesmo uma vela como fonte de luz.
\begin{enumerate}
\item {} 
\url{https://www.youtube.com/watch?v=BUqpQVbEX9M}

\item {} 
\url{https://www.youtube.com/watch?v=Gx7nw5QC0zQ}

\item {} 
\url{https://www.youtube.com/watch?v=Uv-MdaBfk8U}

\item {} 
\url{https://www.youtube.com/watch?v=-hL28SkHf1g}

\item {} 
\url{https://www.youtube.com/watch?v=gzAUIXu7-pY}

\end{enumerate}

% \begin{figure}[H]
% \centering
% \capstart

% \noindent\includegraphics[width=300bp]{{Maos}.png}
% \caption{Vídeo \emph{Como fazer SOMBRAS DE ANIMAIS com as mãos para crianças} \textless{}\url{https://www.youtube.com/watch?v=Gx7nw5QC0zQ}\textgreater{}.}\label{\detokenize{GE301-A:id23}}\end{figure}

No endereço \url{http://fabianaeaarte.blogspot.com/2012/06/teatro-de-sombras.html} você também pode encontrar um projeto pronto para construir seu próprio espetáculo com sombras. Convide um colega e mãos a obra!


\exercise
\label{\detokenize{GE301-E:exercicios}}\label{\detokenize{GE301-E::doc}}

\begin{enumerate}
\item {[}Texto e figuras retirados do site da OBM{]}(OBM2008 - Questão 6 da 1a fase-nível 1) Sobre uma mesa retangular de uma sala foram colocados quatro sólidos, mostrados no desenho. Uma câmera no teto da sala, bem acima da mesa, fotografou o conjunto.
\phantomsection\label{\detokenize{GE301-E:fig-projecoes-obm2008q6}}
\begin{figure}[H]
\centering

\noindent\includegraphics[width=250bp]{{OBM2008-fase1N1-Q6}.png}
\label{\detokenize{GE301-E:fig-projecoes-obm2008q6}}\end{figure}

Qual dos esboços a seguir representa melhor essa fotografia?
\phantomsection\label{\detokenize{GE301-E:fig-projecoes-obm2008q6-respostas}}
\begin{figure}[H]
\centering

\noindent\includegraphics[width=400bp]{{OBM2008-fase1N1-Q6-respostas-editadas}.png}
\label{\detokenize{GE301-E:fig-projecoes-obm2008q6-respostas}}\end{figure}

\item {[}Texto e figuras retirados da internet{]}(ENEM2009 - questão 69) A fotografia mostra uma turista aparentemente beijando a esfinge de Gizé, no Egito. A figura a seguir mostra como, na verdade, foram posicionadas a câmera fotográfica, a turista e a esfinge.

\begin{figure}[H]
\centering
\capstart

\noindent\includegraphics[width=300bp]{{QAnulada_ENEM2009}.png}
\caption{Questão 69 do ENEM 2009.}\label{\detokenize{GE301-E:fig-projecoes-enem2009-q69}}\label{\detokenize{GE301-E:id1}}\end{figure}

Medindo-se com uma régua diretamente na fotografia, verifica-se que a medida do queixo até o alto da cabeça da turista é igual a \(\frac23\) da medida do queixo da esfinge até o alto da sua cabeça. Considere que essas medidas na realidade são representadas por \(d\) e \(d’\), respectivamente, que a distância da esfinge à lente da câmera fotográfica, localizada no plano horizontal do queixo da turista e da esfinge, é representada por \(b\), e que a distância da turista à mesma lente, por \(a\). A razão entre \(b\) e \(a\) será dada por:

\begin{enumerate}
\item \(\frac{b}{a}=\frac{d'}{c}\)

\item \(\frac{b}{a}=\frac{2d}{3c}\)

\item \(\frac{b}{a}=\frac{3d'}{2c}\)

\item \(\frac{b}{a}=\frac{2d'}{3c}\)

\item \(\frac{b}{a}=\frac{2d’}{c}\)

\end{enumerate}

\item {[}Texto e figuras retirados do site do INEP{]}(ENEM2016 - Questão 178 caderno azul) A figura representa o globo terrestre e nela estão marcados os pontos \(A, B\) e \(C\). Os pontos \(A\) e \(B\) estão localizados sobre um mesmo paralelo, e os pontos \(B\) e \(C\), sobre um mesmo meridiano. É traçado um caminho do ponto \(A\) até \(C\), pela superfície do globo, passando por \(B\), de forma que o trecho de \(A\) até \(B\) se dê sobre o paralelo que passa por \(A\) e \(B\) e, o trecho de \(B\) até \(C\) se dê sobre o meridiano que passa por \(B\) e \(C\). Considere que o plano \(\alpha\) é paralelo à linha do equador na figura.
\phantomsection\label{\detokenize{GE301-E:fig-projecoes-enem2016q178}}
\begin{figure}[H]
\centering

\noindent\includegraphics[width=200bp]{{Q178_ENEM2016_1}.png}
\label{\detokenize{GE301-E:fig-projecoes-enem2016q178}}\end{figure}

A projeção ortogonal, no plano \(\alpha\), do caminho traçado no globo terrestre pode ser representada por:

\begin{figure}[H]
\centering

\noindent\includegraphics[width=135bp]{{ENEM-2016-Q178-respoaseditadas-versao2}.png}
\end{figure}

\item {[}Texto e figuras retiradas do site do INEP{]}(ENEM2007 - Questão 5) Representar objetos tridimensionais em uma folha de papel nem sempre é tarefa fácil. O artista holandês Escher (1898-1972) explorou essa dificuldade criando várias figuras planas impossíveis de serem construídas como objetos tridimensionais, a exemplo da litografia Belvedere, reproduzida ao lado. Considere que um marceneiro tenha encontrado algumas figuras supostamente desenhadas por Escher e deseje construir uma delas com ripas rígidas de madeira que tenham o mesmo tamanho.
\phantomsection\label{\detokenize{GE301-E:fig-projecoes-enem2007q5}}
\begin{figure}[H]
\centering

\noindent\includegraphics[width=150bp]{{Q5_ENEM2007}.png}
\label{\detokenize{GE301-E:fig-projecoes-enem2007q5}}\end{figure}

Qual dos desenhos a seguir ele poderia reproduzir em um modelo tridimensional real?
\phantomsection\label{\detokenize{GE301-E:fig-projecoes-enem2007q5-respostas}}
\begin{figure}[H]
\centering

\noindent\includegraphics[width=250bp]{{ENEM_2007_Q5_versao2}.png}
\label{\detokenize{GE301-E:fig-projecoes-enem2007q5-respostas}}\end{figure}

\item {[}Texto e figuras retirados do site do INEP{]}(ENEM2014 - Questão 160 caderno azul) O acesso entre os dois andares de uma casa é feito através de uma escada circular (escada caracol), representada na figura. Os cinco pontos \(A, B, C, D, E\) sobre o corrimão estão igualmente espaçados, e os pontos \(P, A\) e \(E\) estão em uma mesma reta. Nessa escada, uma pessoa caminha deslizando a mão sobre o corrimão do ponto \(A\) até o ponto \(D\).
\phantomsection\label{\detokenize{GE301-E:fig-projecoes-enem2014q160}}
\begin{figure}[H]
\centering

\noindent\includegraphics[width=200bp]{{Q160_ENEM2014}.png}
\label{\detokenize{GE301-E:fig-projecoes-enem2014q160}}\end{figure}

A figura que melhor representa a projeção ortogonal sobre o piso da casa (plano), do caminho percorrida pela mão dessa pessoa é:
\phantomsection\label{\detokenize{GE301-E:fig-projecoes-enem2014q160-respostas}}
\begin{figure}[H]
\centering

\noindent\includegraphics[width=110bp]{{ENEM-2014-Q160-respostaseditadas-versao2}.png}
\label{\detokenize{GE301-E:fig-projecoes-enem2014q160-respostas}}\end{figure}

\item {[}Texto e figuras retirados do site do INEP{]}(ENEM2013 - Questão 180 caderno azul) Gangorra é um brinquedo que consiste de uma tábua longa e estreita equilibrada e fixada no seu ponto central (pivô). Nesse brinquedo, duas pessoas sentam-se nas extremidade e, alternadamente, impulsionam-se para cima, fazendo descer a extremidade oposta, realizado, assim, o movimento da gangorra. Considere a gangorra representada na figura, em que os pontos \(A\) e \(B\) são equidistantes do pivô:
\phantomsection\label{\detokenize{GE301-E:fig-projecoes-enem2013q180}}
\begin{figure}[H]
\centering

\noindent\includegraphics[width=200bp]{{Q180_ENEM2013}.png}
\label{\detokenize{GE301-E:fig-projecoes-enem2013q180}}\end{figure}

A projeção ortogonal da trajetória dos pontos \(A\) e \(B\), sobre o plano do chão da gangorra, quando esta se encontra em movimento, é:
\phantomsection\label{\detokenize{GE301-E:fig-projecoes-enem2013q180-respostas}}
\begin{figure}[H]
\centering

\noindent\includegraphics[width=140bp]{{ENEM-2013-Q180-respostaseditadas-versao2}.png}
\label{\detokenize{GE301-E:fig-projecoes-enem2013q180-respostas}}\end{figure}

\item {[}Texto e figuras retirados do site da OBM{]}(OBM2014 - Questão 20 da 1a fase-nível 1) A figura abaixo mostra um bloco retangular montado com seis cubinhos pretos e seis cubinhos brancos, todos de mesmo tamanho.
\phantomsection\label{\detokenize{GE301-E:fig-projecoes-obm2014q20}}
\begin{figure}[H]
\centering

\noindent\includegraphics[width=110bp]{{OBM2014-fase1N1-Q20}.png}
\label{\detokenize{GE301-E:fig-projecoes-obm2014q20}}\end{figure}

Qual das figuras abaixo mostra o mesmo bloco visto por trás?
\phantomsection\label{\detokenize{GE301-E:fig-projecoes-obm2014q20-respostas}}
\begin{figure}[H]
\centering

\noindent\includegraphics[width=380bp]{{OBM2014-fase1N1-Q20-respostas-editadas}.png}
\label{\detokenize{GE301-E:fig-projecoes-obm2014q20-respostas}}\end{figure}

OBS.: Esta questão também está presente no nível 2.

\item {[}Texto e figuras retirados do site do INEP{]}(ENEM2009 - Questão 149 caderno azul) Em Florença, Itália, na Igreja de Santa Croce, é possível encontrar um portão em que aparecem os anéis de Borromeo. Alguns historiadores acreditavam que os círculos representavam as três artes: escultura, pintura e arquitetura, pois elas eram tão próximas quanto inseparáveis.
\phantomsection\label{\detokenize{GE301-E:fig-projecoes-enem2009q149}}
\begin{figure}[H]
\centering

\noindent\includegraphics[width=150bp]{{Q149_ENEM2009}.png}
\label{\detokenize{GE301-E:fig-projecoes-enem2009q149}}\end{figure}

Qual dos esboços a seguir melhor representa os anéis de Borromeo?
\phantomsection\label{\detokenize{GE301-E:fig-projecoes-enem2009q149-respostas}}
\begin{figure}[H]
\centering

\noindent\includegraphics[width=300bp]{{ENEM_2009_Q149_editadas}.png}
\label{\detokenize{GE301-E:fig-projecoes-enem2009q149-respostas}}\end{figure}

\item {[}Texto e figuras retirados do site da OBM{]}(OBM2013 - Questão 4 da 1a fase-nível 1) Esmeralda está construindo um paralelepípedo usando blocos menores iguais.
\phantomsection\label{\detokenize{GE301-E:fig-projecoes-obm2013q4}}
\begin{figure}[H]
\centering

\noindent\includegraphics[width=150bp]{{OBM2013-fase1N1}.png}
\label{\detokenize{GE301-E:fig-projecoes-obm2013q4}}\end{figure}

Para terminar sua tarefa, quantos blocos Esmeralda ainda deve colocar?

\begin{enumerate}
\item 12

\item 14

\item 16

\item 18

\item 20
\end{enumerate}

OBS.: Esta questão também está presente no nível 2.

\item {[}Texto e figuras retirados do site da OBM{]}(OBM2010 - Questão3 da 3a fase-nível 1) Dado um sólido formado por cubos de 1 cm de aresta, como mostra a figura abaixo da esquerda, podemos indicar a quantidade de cubos em cada direção, como mostra a figura abaixo da direita.
\phantomsection\label{\detokenize{GE301-E:fig-projecoes-obm2010q3}}
\begin{figure}[H]
\centering

\noindent\includegraphics[width=350bp]{{OBM2010-fase3N1-Q3}.png}
\label{\detokenize{GE301-E:fig-projecoes-obm2010q3}}\end{figure}

Esmeraldino montou um sólido com cubos de 1 cm de aresta e fez uma figura similar acima. Encontre os valores de \(a, b, c, d, e, f, x\) e \(m\).
\phantomsection\label{\detokenize{GE301-E:fig-projecoes-obm2010q3-2}}
\begin{figure}[H]
\centering

\noindent\includegraphics[width=180bp]{{OBM2010-fase3N1-Q3_2}.png}
\label{\detokenize{GE301-E:fig-projecoes-obm2010q3-2}}\end{figure}

\item {[}Texto e figuras retirados do site da OBM{]}(OBM2009 - Questão 20 da 1a fase-nível 1) Alguns cubos foram empilhados formando um bloco. As figuras abaixo representam a vista da esquerda e da frente desse bloco.
\phantomsection\label{\detokenize{GE301-E:fig-projecoes-obm2009q20}}
\begin{figure}[H]
\centering

\noindent\includegraphics[width=250bp]{{OBM2009-fase1N1-Q20}.png}
\label{\detokenize{GE301-E:fig-projecoes-obm2009q20}}\end{figure}

Olhando o bloco de cima, qual das figuras a seguir não pode ser vista?
\phantomsection\label{\detokenize{GE301-E:fig-projecoes-obm2009q20-respostas}}
\begin{figure}[H]
\centering

\noindent\includegraphics[width=450bp]{{OBM2009-fase1N1-Q20-respostas-editadas}.png}
\label{\detokenize{GE301-E:fig-projecoes-obm2009q20-respostas}}\end{figure}

\item {[}Texto e figuras retirados do site da OBM{]}(OBM2005 - Questão 25 da 1a fase-nível 2) Um bloco de dimensões \(1\times 2 \times 3\) é colocado sobre um tabuleiro \(8\times 8\), como mostra a figura, com a face X, de dimensões \(1\times 2\), virada para baixo. Giramos o bloco em torno de uma de suas arestas de modo que a face Y fique virada para baixo. Em seguida, giramos novamente o bloco, mas desta vez de modo que a face Z fique virada para baixo. Giramos o bloco mais três vezes, fazendo com que as faces X, Y e Z fiquem viradas para baixo, nessa ordem. Quantos quadradinhos diferentes do tabuleiro estiveram em contato com o bloco?
\phantomsection\label{\detokenize{GE301-E:fig-projecoes-obm2005q25}}
\begin{figure}[H]
\centering

\noindent\includegraphics[width=250bp]{{OBM2005-fase1N2-Q25}.png}
\label{\detokenize{GE301-E:fig-projecoes-obm2005q25}}\end{figure}

\begin{enumerate}
\item 18

\item 19

\item 20

\item 21

\item 22
\end{enumerate}

\item {[}Texto e figuras retirados do site da OBM{]}(OBM2004 - Questão 24 da 1a fase-nível 1) Observe a figura:
\phantomsection\label{\detokenize{GE301-E:fig-projecoes-obm2004q24}}
\begin{figure}[H]
\centering

\noindent\includegraphics[width=200bp]{{OBM2004-fase1N1-Q24}.png}
\label{\detokenize{GE301-E:fig-projecoes-obm2004q24}}\end{figure}

Duas das figuras abaixo representam o objeto acima colocado em outras posições.
\phantomsection\label{\detokenize{GE301-E:fig-projecoes-obm2004q24-respostas}}
\begin{figure}[H]
\centering

\noindent\includegraphics[width=400bp]{{OBM2004-fase1N1-Q24-respostas}.png}
\label{\detokenize{GE301-E:fig-projecoes-obm2004q24-respostas}}\end{figure}

Elas são:

\begin{enumerate}
\item I e II

\item I e IV

\item II e IV

\item I e III

\item II e III
\end{enumerate}

\item {[}Texto e figuras retirados do site da OBM{]}(OBM2004 - Questão 19 da 1a fase-nível 3) Dono de uma loja empilhou vários blocos medindo \(0{,}8~\text{m} \times 0{,}8~\text{m} \times 0{,}8~\text{m}\) no canto da loja e encostados numa parede de vidro que dá para a rua, conforme mostra a figura abaixo.
\phantomsection\label{\detokenize{GE301-E:fig-projecoes-obm2004q219}}
\begin{figure}[H]
\centering

\noindent\includegraphics[width=200bp]{{OBM2004-fase1N3-Q19}.png}
\label{\detokenize{GE301-E:fig-projecoes-obm2004q219}}\end{figure}

Quantos blocos no máximo, uma pessoa de \(1{,}80~\text{m}\) de altura que está do lado de fora da loja pode enxergar?

Obs. Consideramos que uma pessoa pode enxergar uma caixa se consegue ver uma pequena região de área positiva de sua superfície.

\begin{enumerate}
\item 13

\item 14

\item 15

\item 16

\item 17
\end{enumerate}

\item {[}Texto e figuras retirados do site da OBM{]}(OBM2000 - Questão 20 da 1a fase-nível 1) A figura abaixo foi desenhada em cartolina e dobrada de modo a formar um cubo.
\phantomsection\label{\detokenize{GE301-E:fig-projecoes-obm2000q20}}
\begin{figure}[H]
\centering

\noindent\includegraphics[width=200bp]{{OBM2000-fase1N1-Q20}.png}
\label{\detokenize{GE301-E:fig-projecoes-obm2000q20}}\end{figure}

Qual das alternativas mostra o cubo assim formado?
\phantomsection\label{\detokenize{GE301-E:fig-projecoes-obm2008-0q20-respostas}}
\begin{figure}[H]
\centering

\noindent\includegraphics[width=450bp]{{OBM2000-fase1N1-Q20-respostaseditadas}.png}
\label{\detokenize{GE301-E:fig-projecoes-obm2008-0q20-respostas}}\end{figure}

\item {[}Texto retirados do site da OBMEP{]}(OBMEP2017 - Questão 4 da 1a fase-nível 3)  Zequinha tem três dados iguais, com letras O, P, Q, R, S e T em suas faces. Ele juntou esses dados como na figura, de modo que as faces em contato tivessem a mesma letra. Qual é a letra na face oposta à que tem a letra T?
\phantomsection\label{\detokenize{GE301-E:fig-projecoes-obmep2017q4}}
\begin{figure}[H]
\centering

\noindent\includegraphics[width=200bp]{{OBMEP2017-fase1N3-Q4}.png}
\label{\detokenize{GE301-E:fig-projecoes-obmep2017q4}}\end{figure}

\begin{enumerate}
\item S

\item R

\item Q

\item P

\item O
\end{enumerate}

\item {[}Texto retirado do site da OBMEP{]}(OBMEP2015 - Questão 3 da 1a fase-nível 2) A peça da Figura 1 foi montada juntando-se duas peças, sem sobreposição.

\begin{figure}[H]
\centering
\capstart

\noindent\includegraphics[width=105bp]{{OBMEP2015_F1N2_Q3_1_1}.png}
\caption{Figura 1.}\label{\detokenize{GE301-E:fig-projecoes-obmep2015q3}}\label{\detokenize{GE301-E:id6}}\end{figure}

Uma das peças utilizadas foi a da Figura 2.

\begin{figure}[H]
\centering
\capstart

\noindent\includegraphics[width=80bp]{{OBMEP2015_F1N2_Q3_2_1}.png}
\caption{Figura 2.}\label{\detokenize{GE301-E:fig-projecoes-obmep2015q3-2}}\label{\detokenize{GE301-E:id7}}\end{figure}

Qual foi a outra peça utilizada?
\phantomsection\label{\detokenize{GE301-E:fig-projecoes-obmep2015q3-respostas}}
\begin{figure}[H]
\centering

\noindent\includegraphics[width=420bp]{{OBMEP2015_F1N2_Q3_respostas_1}.png}
\label{\detokenize{GE301-E:fig-projecoes-obmep2015q3-respostas}}\end{figure}

\item {[}UFSCar-2001{]} Considere um plano \(\alpha\) e um ponto \(P\) qualquer do espaço. Se por \(P\) traçarmos a reta perpendicular a \(\alpha\), a intersecção dessa reta com \(\alpha\) é um ponto chamado projeção ortogonal do ponto \(P\) sobre \(\alpha\). No caso de uma figura \(F\) do espaço, a projeção ortogonal de \(F\) sobre \(\alpha\) é definida pelo conjunto das projeções ortogonais de seus pontos.

Com relação a um plano \(\alpha\) qualquer fixado, pode-se dizer que:

\begin{enumerate}
\item a projeção ortogonal de um segmento de reta pode resultar numa semirreta.

\item a projeção ortogonal de uma reta sempre resulta numa reta.

\item a projeção ortogonal de uma parábola pode resultar num segmento de reta.

\item a projeção ortogonal de um triângulo pode resultar num quadrilátero.

\item a projeção ortogonal de uma circunferência pode resultar num segmento de reta.
\end{enumerate}

\item Vimos que uma projeção em perspectiva com relação a um centro \(O\) em um plano de projeção \(\pi\) pode ser considerada como uma função \(f\) de domínio \({\mathbb R}^{3} - \psi\) e contradomínio \(\pi\), com \(\psi\) o plano paralelo a \(\pi\) que passa por \(O\).
\begin{enumerate}
\item {} 
Por que ao se modelar uma projeção em perspectiva por meio de uma função, os pontos do plano \(\psi\) são excluídos do seu domínio?

\item {} 
Dado um ponto \(Y \in \pi\) quantos pontos \(X\) existem tais que \(f(X) = Y\)? Se você desenhasse todos os pontos \(X\) que satisfazem essa propriedade, o que apareceria desenhado?

\item {} 
Verdadeiro ou falso? Para cada ponto \(P \in {\mathbb R}^{3} - \psi\), vale que \(f(f(P)) = f(P)\). Justifique sua resposta.

\item {} 
Você conhece outras funções \(f\) tais que \(f(f(P)) = f(P)\) para todo ponto \(P\) do domínio de \(f\)? Quais?

\item {} 
Verdadeiro ou falso? Se \(f(P) = f(Q)\), então \(P = Q\). Justifique sua resposta.

\item {} 
Se o centro \(O\) é um ponto do plano de projeção \(\pi\), determine \(f(P)\) para todo \(P \in {\mathbb R}^{3} - \psi\).

\end{enumerate}

\item Vimos que uma projeção paralela com relação a uma direção \(d\) sobre um plano de projeção \(\pi\), com a direção \(d\) não paralela a \(\pi\), pode ser considerada como uma função \(f\) de domínio \({\mathbb R}^{3}\) e contradomínio \(\pi\).
\begin{enumerate}
\item {} 
Por que ao se modelar uma projeção paralela por meio de uma função, supõe-se que a direção \(d\) não seja paralela ao plano \(\pi\)?

\item {} 
Dado um ponto \(Y \in \pi\) quantos pontos \(X\) existem tais que \(f(X) = Y\)? Se você desenhasse todos os pontos \(X\) que satisfazem essa propriedade, o que apareceria desenhado?

\item {} 
Verdadeiro ou falso? Para cada ponto \(P \in {\mathbb R}^{3}\), vale que \(f(f(P)) = f(P)\). Justifique sua resposta.

\item {} 
Você conhece outras funções \(f\) tais que \(f(f(P)) = f(P)\) para todo ponto \(P\) do domínio de \(f\)? Quais?

\item {} 
Verdadeiro ou falso? Se \(f(P) = f(Q)\), então \(P = Q\). Justifique sua resposta.

\end{enumerate}

\item Considere a configuração geométrica da \hyperref[\detokenize{GE301-5:fig-proj-ladrilhamentos-01}]{Figura \ref{\detokenize{GE301-5:fig-proj-ladrilhamentos-01}}} da Etapa 1 da PARTE 2 da \DUrole{xref,std,std-ref}{ativ-proj-comprimentos}, ainda com os dados \(a = 3\), \(d = 5\) e \(h = 4\) e \(x = 6\). Suponha agora que vários segmentos congruentes a \(RS\) sejam desenhados no plano \(\gamma\): \(R_{7}S_{7}\), \(R_{8}S_{8}\), \(\ldots\), \(R_{n}S_{n}\), \(\ldots\) de tal modo que (1) para cada \(n \geq 7\), o quadrilátero \(RR_{n}S_{n}S\) é um retângulo e (2) para cada \(n \geq 7\), a distância de \(V\) até o segmento \(R_{n}S_{n}\) é \(n\). Assim, os segmentos \(R_{n}S_{n}\) estão uniformemente espaçados, com a distância entre dois segmentos consecutivos sendo igual a \(1\). Observe as alturas \(y_{n} = g(n)\) das projeções \(R_{n}'S_{n}'\) sobre o plano \(\pi\) com relação ao plano “do chão” \(\gamma\): elas aumentam a medida que \(n\) aumentam, mas a diferença entre duas consecutivas parece diminuir.
\phantomsection\label{\detokenize{GE301-E:fig-proj-perspectiva-ladrilhamentos-04-ex}}
\begin{figure}[H]
\centering

\noindent\includegraphics[width=300bp]{{perspectiva-ladrilhamentos-04}.jpg}
\label{\detokenize{GE301-E:fig-proj-perspectiva-ladrilhamentos-04-ex}}\end{figure}
\end{enumerate}