\renewcommand\chapterillustration{./abertura-investigacao}%Photo by Hoach Le Dinh on Unsplash, https://unsplash.com/photos/c8TWWQ5ZnUw?utm_source=unsplash&utm_medium=referral&utm_content=creditCopyText 
\def\chapterwhat{Taxas, índices e indicadores sociais, econômicos e ambientais. Compreensão de aspectos teóricos e práticos dessas informações, com uma metodologia que busca a análise de situações reais, tanto locais quanto globais.}

\def\chapterbecause{Na era da informação, somos inundados por dados sobre os mais diversos fenômenos da realidade. Compreender a obtenção e organização desses dados torna-se importante para planejarmos ações que busquem uma organização social mais justa e sustentável.} 
\chapter{Projetos de Investigação com Matemática}
\label{ladri-chap}

\mbox{}\thispagestyle{empty}\clearpage

\thispagestyle{empty}

\begin{center}
Projeto: LIVRO ABERTO DE MATEMÁTICA

\noindent \begin{tabular}{lcccr}
\includegraphics[scale=.15]{impa}& \quad\quad& \includegraphics[width=3cm]{logo} & \quad\quad& \includegraphics[scale=.24]{obmep} 
\end{tabular}
\end{center}

\vspace*{.3cm}

Cadastre-se como colaborador no site do projeto: \url{umlivroaberto.org}


% \begin{center}
%   \includegraphics[width=2cm]{canvas}
% \end{center}

\begin{tabular}{p{.15\textwidth}p{.7\textwidth}}
Título: & Projetos de Investigação com Matemática\\
\\
Ano/ Versão: & 2020 / versão 0.1 de 08 de junho de 2020\\
\\
Editora & Instituto Nacional de Matem\'atica Pura e Aplicada (IMPA-OS)\\
\\
Realização:& Olimp\'iada Brasileira de Matem\'atica das Escolas P\'ublicas (OBMEP)\\
\\
Produção:& Associação Livro Aberto\\
\\
Coordenação: & Fabio Simas, \\
			&  Augusto Teixeira (livroaberto@impa.br)\\
\\
  Autor: & Thiago Ferraiol, \\
         & Priscila Santos, \\
         & Rodrigo Belli \\
\\
Revisor: &  ---  \\
\\
Design: & Andreza Moreira (Tangentes Design) \\
\\
  Ilustrações: & --- \\ 
\\
Gráficos: & --- \\
\\
  Capa: & Foto de Clay Banks, no Unsplash \\
  		& https://unsplash.com/photos/U0-r0JMypE0 \\

\end{tabular}


\begin{figure}[b]
\begin{minipage}[l]{5cm}
\centering

{\large Licença:}

  \includegraphics[width=3.5cm]{cc-by-nc-sa}
\end{minipage}\hfill
\begin{minipage}[c]{5cm}
\centering
{\large Desenvolvido por}

\includegraphics[width=2.5cm]{logo-associacao.jpg}
\end{minipage}
\begin{minipage}[r]{5cm}
\centering

{\large Patrocínio:}
  \vspace{1em}
  \includegraphics[width=3.5cm]{itau}
\end{minipage}
\end{figure}

\mainmatter

\explore{Para começo de conversa}

As análises sobre a realidade social precisam se atentar àquilo que os especialistas chamam de conjuntura (Você Sabia: O que é conjuntura?). Nossa conjuntura compreende o período que se inicia a partir da segunda metade do século XX. Em nossa época, ocorre a terceira revolução industrial, também chamada de revolução microeletrônica ou digital. Essa revolução é caracterizada pela passagem de processos mecânicos e analógicos para processos digitais.

A partir do final do século XX deu-se também a massificação do uso dos computadores e da internet, permitindo que nos comuniquemos instantaneamente e inundando nosso mundo de informações. Este período mais recente representa uma quarta fase do desenvolvimento industrial, o da revolução 4.0, momento em que as tecnologias da informação e da comunicação estão completamente integradas em rede com outras dimensões da vida, operando instantaneamente à distância (SCHWAB, 2016). Por isso nossa conjuntura é conhecida por alguns como a era da informação.

No entanto, ao contrário do que poderíamos imaginar, quanto maior o fluxo de informações que nos chega, mais parecemos incapazes de interpretá-las. Entramos em um aparente paradoxo: percebemo-nos seres com cada vez mais informações e, ao mesmo tempo, cada vez menos informados. Existem algumas explicações para esse aparente paradoxo. Uma delas, dada pelo sociólogo polonês Zygmunt Bauman (2001), é que as informações na modernidade, da mesma forma que a produção e as relações sociais, estão cada vez mais líquidas. Isto significa que, assim como a água, elas se movem e se transformam muito rapidamente, dificultando nossa tarefa de sintetizá-las e utilizá-las objetivamente para interpretar o mundo e tomarmos decisões coletivas. Esse contexto de fartura, aliada à fragmentação, nos traz mais confusão do que formação, restringindo nossas ações aos aspectos individuais (daí a liquidez também manifestada nas relações sociais).

Neste sentido, o saber matemático pode se tornar uma ferramenta indispensável à análise sociológica, já que boa parte dessas informações pode ser medida, quantificada e traduzida em dados estatísticos. Assim, esses dados expressam, através dos números, dos índices, das tabelas e dos gráficos, partes das nossas condições atuais, das transformações ocorridas no mundo e servem de parâmetros para pensarmos em novas perspectivas de vida e de desenvolvimento.

Mas como podemos fazer isso? Afinal, se considerarmos o argumento de Bauman, nossa própria realidade social dificultaria o exercício de avaliação.

A chave para quebrarmos a reprodução desse entendimento está em partirmos do cotidiano - de suas expressões mais imediatistas - ao abstrato - as formulações pretensamente mais gerais -, e retornamos ao cotidiano numa nova condição, confrontando-o ao conhecimento socialmente construído pela prática científica.

\section{Proposta metodológica}

A proposta deste capítulo está calcada em cinco pontos, cada um indicando um momento das atividades no ambiente escolar que levam do cotidiano à abstração, retornando em seguida ao cotidiano, já numa nova condição

\begin{figure}[H]
\centering
\includegraphics[width=350bp]{investigacao1.jpg}

\end{figure}

\begin{itemize}
\item \hyperref[etapa1]{\textcolor{session1}{\textbf{Explorando: Atividades disparadoras - Etapa 1}}}: Etapa para conhecer materiais que lhe provoquem um estranhamento (Para saber +: O cotidiano e seu estranhamento) daquilo que acontece em sua rotina e lhe permita iniciar questionamentos sobre ela. São materiais como textos e/ou vídeos jornalísticos, materiais artísticos (como poesia, contos, crônicas, apresentações, instalações, performances), depoimentos de seus colegas ou de outras pessoas da comunidade escolar, etc. Estes materiais serão fornecidos por seu professor ou indicado por ele como acessá-los.

\item \hyperref[etapa2]{\textcolor{session1}{\textbf{Explorando: Elaboração de questões - Etapa 2}}}: Tem o objetivo de organizar as questões provenientes das atividades disparadoras em um problema geral que permita uma análise objetiva a partir do uso de indicadores. Além disso, serão formados grupos de trabalho, no qual cada um terá a sua questão investigativa.

\item \hyperref[etapa3]{\textcolor{session4}{\textbf{Organizando: Planejando a investigação - Etapa 3}}}: A partir dos problemas colocados, você e seus colegas deverão se aprofundar nas ferramentas necessárias para a execução da pesquisa. Esta é a terceira etapa da metodologia proposta.

\item \hyperref[etapa4]{\textcolor{session2}{\textbf{Praticando: Desenvolvendo a investigação - Etapa 4}}}: Nesta quarta etapa você e seus colegas utilizarão novos conhecimentos, construídos nas etapas anteriores, e provavelmente adquirirão mais alguns. É nesta fase do trabalho que os conceitos apreendidos no momento anterior serão realmente aplicados, ganhando profundidade. 

\item \hyperref[etapa5]{\textcolor{session2}{\textbf{Praticando: Comunicando as descobertas - Etapa 5}}}: Nesta quinta e última etapa cada grupo deve apresentar a pesquisa de análise matemática em um formato concreto, compartilhando seus resultados com todos os colegas da comunidade escolar.
\end{itemize}

Ainda sobre os produtos conclusivos das pesquisas, o que se espera alcançar ao final é um estímulo à pesquisa científica e ao exercício da intervenção prática sobre a realidade a partir de sua objetividade.

Todas as etapas de desenvolvimento do projeto serão apresentadas nas próximas seções por meio de um exemplo, que utiliza a pandemia do novo coronavírus para dar concretude às propostas de atividades e reflexões.

\know{O cotidiano e seu estranhamento}

De acordo com a filósofa húngara Agnes Heller, nossas vidas são marcadas profundamente pela dimensão do cotidiano. Como o próprio nome sugere, trata-se do momento da vida que nos ocupamos com nossa rotina diária. Quanto mais o tempo passa e, consequentemente, mantemos os elementos essenciais da rotina intocados, mais nos acostumamos a tratar tudo aquilo que nos acontece como algo “natural”, como algo que não possuísse outra alternativa de ser.

Mas a natureza, assim como a vida social, é bastante dinâmica. Basta prestarmos atenção ao nosso redor para identificarmos, tanto na natureza quanto na sociedade, formas bastante diferentes de manifestação da vida. Neste momento, podemos realizar um exercício de estranhamento, ou seja, de questionamento daquilo que está estabelecido e de proposição de algo diferente do corriqueiro. Deste modo surgem as diversas formas de arte, de ciência e de política. Todas criações humanas que ganham autonomia relativa diante da dimensão cotidiana.

\begin{knowledge}{O que isso, a conjuntura?}

O termo conjuntura se tornou recorrente na historiografia recente graças à figura de Fernand Braudel. Historiador francês vinculado à escola dos Annales, ele foi responsável por propor outro modelo de registro da  história.

De uma disciplina atrelada aos estudos do passado, que buscava cada vez mais compreender o chamado tempo presente, Braudel vincula seu ofício a uma forma de compreensão mais ampla, agregando recursos de análise das ciências humanas em geral.

De maneira bem resumida, para Braudel, seria possível realizar uma pesquisa historiográfica a partir de três níveis: estrutural, conjuntural e fatual. No primeiro seria possível observar fenômenos de longa duração, relações sociais e ambientais dos seres humanos que permanecem ativos por longos períodos. No conjuntural, são os fenômenos de média duração, que não consolidam estruturas de relação, mas que não se resumem a meras acontecimentos pontuais, algo restrito ao entendimento da história fatual (aquela que se contenta, como o nome sugere, aos fatos, a imediaticidade das relações).

Portanto, quando nos referimos à conjuntura, estamos considerando uma forma de análise histórica que, sem ignorar os acontecimentos pontuais e as grandes estruturas de relações que permanecem ativas por longos períodos de tempo, se concentra em abordar o movimento. Em outras palavras, de que maneira os acontecimentos pontuais são expressão das grandes estruturas de sociabilidade e em que medida sua concretização abre caminhos para uma alteração nessa mesma estrutura.

\end{knowledge}


\explore{Atividades disparadoras - Etapa 1}
\phantomsection\label{etapa1}

As atividades disparadoras tem o objetivo de te colocar em contato com o tema que será desenvolvido no projeto de investigação. Elas podem ser iniciadas por meio de filmes, documentários, notícias jornalísticas, músicas ou qualquer outro material que ajude a iniciar uma reflexão.

A seguir você encontrará as atividades disparadoras do exemplo proposto, que é a pandemia do novo coronavírus. 

\begin{example}{Primeiras informações sobre a pandemia}
\phantomsection\label{primeiras-informacoes}

Em dezembro de 2019, um novo coronavírus, chamado de Sars-Cov-2\footnote{Sars-Cov-2 é a abreviação, em inglês, para o coronavírus da síndrome respiratória aguda grave 2,}, causador da COVID-19\footnote{COVID-19 é a abreviação, em inglês, para a doença causada coronavírus, iniciada em  2019.}, foi identificado na cidade de Wuhan, na China. Desde então, o vírus se disseminou pelo mundo. O ritmo rápido do alastramento, o alto poder de contágio e a letalidade tem causado preocupações em todos. Em 11 de março de 2020, a Organização Mundial da Saúde (OMS) declarou que o mundo estava em uma pandemia, isto é, uma epidemia de proporções mundiais. Naquele momento, o relatório da OMS (Disponível em \url{https://www.who.int/emergencies/diseases/novel-coronavirus-2019/situation-reports/} apontava que o mundo tinha 118.319 casos confirmados e 4.292 mortes, sendo a maior parte na China, que tinha 80.955 casos confirmados e 3.162 mortes. Fora da China, a região mais afetada era a Itália, com 10.149 casos confirmados e 631 mortes. No Brasil tínhamos registrado oficialmente 34 casos e nenhuma morte.

Apesar do vírus não escolher quem ele contamina, a dinâmica da propagação e as consequências da doença acometeu regiões do mundo de formas muito distintas. Fatores sociais, econômicos e culturais, bem como as táticas adotadas no enfrentamento têm sido apontados como algumas das dimensões que impactam nessas diferenças. 

Tudo isso parece distante para você? No caso de nosso país, será que a maneira como a Covid-19 se disseminou e foi tratada ocorreu de maneira similar em todo o território? E em sua cidade, quais as medidas de enfrentamento à doença você pôde observar? Apresentaram eficiência? E como as pessoas mais próximas a você estão conseguindo sobreviver em condições de trabalho tão instáveis?

Vejamos algumas informações: No Brasil, até o dia 12 de maio de 2020, o ministério da saúde (\url{https://COVID.saude.gov.br/}) tinha registrado 177.589 casos, sendo 12.400 tinha falecido, 72.596 tinham se recuperado e 92.693 continuavam em acompanhamento.

Além dos números brutos, para dar uma dimensão da epidemia, auxiliar no acompanhamento da sua evolução e criar parâmetros para tomadas de decisão, o ministério da saúde calculava diversos indicadores. Alguns desses indicadores estão explicados a seguir e calculados a partir dos dados no Brasil em 12 de maio de 2020.

Além dos números brutos, para dar uma dimensão da epidemia, auxiliar no acompanhamento da sua evolução e criar parâmetros para tomadas de decisão, o ministério da saúde calcula diversos indicadores. No Brasil\footnote{Segundo estimativa do IBGE, em 2019, o Brasil tinha 210.147.125 de habitantes.}, alguns dos indicadores, calculados com base nos dados do dia 12 de maio de 2020, são:

\begin{itemize}
\item \textbf{Coeficiente de Incidência}: É o número de casos de COVID-19 para cada 100 mil habitantes, em um determinado período de tempo. Calcula-se assim:

\begin{equation*}
\frac{\text{número de casos}}{\text{população}} \times {100.000} = \frac{{177.589}}{{210.147.125}} \times {100.000} \approx 84{,}5
\end{equation*}

Esse número significa que, a cada 100 mil pessoas, 84,5 haviam sido acometidas pela COVID-19.

\item \textbf{Coeficiente de Mortalidade}: É o número de mortes para cada 100 mil habitantes, em um determinado período de tempo. Calcula-se assim:

\begin{equation*}
\frac{\text{número de óbitos}}{\text{população}} \times {100.000} = \frac{{12.000}}{{210.147.125}}\times{100.000}\approx 5,9
\end{equation*}

Esse número significa que, a cada 100 mil pessoas, 5,9 haviam morrido por causa da infecção pelo Sars-Cov-2.

\item \textbf{Taxa de Letalidade}: É o percentual, dentre todas as pessoas infectadas, que morreram por causa da doença. Calcula-se assim:

\begin{equation*}
\frac{\text{número de óbitos}}{\text{total de infectados}} = \frac{{12.400}}{{177.589}} \approx 0,698 = 6,98\%
\end{equation*}

Esse número significa que $6{,}98\%$ das pessoas infectadas até aquela data tinham morrido por causa da doença.

\end{itemize}
\end{example}

\textbf{O que representam as razões nestes cálculos?}

Uma razão entre grandezas --- neste caso, entre o número de pessoas --- é uma forma de fazer comparações da parte com um todo. Quando escrevemos a razão $\dfrac{\text{número de casos}}{\text{população}} = \dfrac{{177.589}}{{210.147.125}}$ estamos representando a parte de pessoas infectadas (177.589) dentro do total da população (210.147.125).

Supondo que a distribuição dos casos é homogênea dentro da população, podemos realizar algumas estimativas da quantidade de infectados em partes da população. Em situações simples, podemos até mesmo fazer contas de cabeça. Por exemplo, para estimar o número de pessoas infectadas em uma população de 100 milhões, podemos primeiro perceber que 100 milhões é um pouco menos da metade de 210 milhões. Assim, o número de pessoas infectadas deve ser um pouco menos da metade de 177 mil, ou seja, algo em torno de 85 mil. Essa é uma estimativa bem grosseira, mas que já nos dá uma dimensão da quantidade de infectados em uma população com 100 milhões de pessoas.


Se quisermos ser mais precisos nas comparações, ainda supondo que a proporção de infectados em cada população é a mesma, podemos fazer o cálculo usando uma regra de três simples. Por exemplo, para estimar o número de infectados na cidade de São Paulo, que tem 12,18 milhões de habitantes, fazemos o seguinte cálculo:
\begin{equation*}
\frac{177.589}{210.147.125}=\frac{x}{12.180.000}\implies x=\frac{177.589\times 12.180.000}{210.147.125}\implies x\approx 10.293
\end{equation*}

\textbf{Por que, ao calcular os coeficientes de incidência e de mortalidade, fez-se a multiplicação da razão por $100.000$}?

Basicamente, fez-se esta multiplicação para colocar os números em uma escala mais próxima da necessidade real e mais simples para realizar comparações e estimativas rápidas, tendo uma noção mais prática do quanto tais números representam.

Neste caso, a representação decimal da razão entre o número de mortes e o total da população resulta em $0{,}000059$. Isso é a proporção de mortes em um grupo. Se colocarmos essa razão em percentual, multiplicando a razão por 100, chegamos a um coeficiente de $0{,}0059\%$, indicando que de cada 100 pessoas, 0,0059 morreram de Covid-19. Neste caso, para fazer comparações e estimativas para cidades com dezenas de milhares de habitantes, teríamos que trabalhar com muitos zeros depois da vírgula e realizar muitas conversões para fazer as estimativas. No entanto, se multiplicarmos a razão por 100 mil, chegamos ao coeficiente de 5,9 pessoas a cada 100.000, e nossos caĺculos ficam simplificados.

Por exemplo, suponha que queremos ter uma noção da mortalidade em uma cidade como Curitiba, que tem cerca de 2 milhões de habitantes. Como o coeficiente indica que há 5,9 mortes a cada 100 mil pessoas, e como 2 milhões é 20 vezes 100 mil, então na população de 2 milhões o número estimado de mortes é em torno de $20\times5{,}9$, ou seja, cerca de 118 pessoas. Este foi um cálculo bem simples e rápido de ser feito.

Se, ao invés de trabalhar com o coeficiente na base de 100 mil habitantes, escolhêssemos trabalhar com percentuais, teríamos que calcular $0{,}0059\%$ de 2 milhões. Veja como os cálculos ficam mais complicados:
\begin{equation*}
0{,}0059\%\text{ de }1.000.000=\frac{0{,}0059}{100}\times 1.000.000=118
\end{equation*}

Outra possibilidade para realizar essa estimativa, mas que também demandaria mais cálculos, seria fazer a regra de três simples utilizando a razão $\dfrac{12.400}{210.147.125}$:
\begin{equation*}
\frac{12.400}{210.147.125}=\frac{x}{2.000.000}\implies x=\frac{2.000.000\times 12.400}{210.147.125}\implies x\approx 118
\end{equation*}

E na sua cidade, você consegue fazer uma estimativa rápida da mortalidade em 12 de maio de 2020, considerando uma taxa de 5,9 mortes a cada 100 mil habitantes?

\know{As formas de enfrentamento à pandemia}

No primeiro epicentro da doença, a cidade de Wuha, por exemplo, antes mesmo de ser classificada como pandemia, a medida adotada foi a de uma rigorosa quarentena.

Na República da Coreia, uma tática foi a utilização do rastreamento da população através de GPS associada a uma campanha massiva de testes.

Em outros territórios, como o Vietnã, houve fechamento de fronteiras.

Para além do continente asiático, o mundo apresentou formas replicadas das estratégias acima mencionadas, mas com resultados bastante diferentes.


\begin{task}{Calculando e interpretando dados da COVID-19 por região}

A tabela a seguir apresenta o tamanho da população e os dados brutos do número de casos e de óbitos pela COVID-19 em cada região do Brasil no dia 12 de maio de 2020.

\begin{enumerate}
\item Complete a tabela calculando os coeficientes de indicência, de mortalidade e a taxa de letalidade de cada Estado.
\item Qual região era, até aquele momento, a mais afetada pela COVID-19? Justifique.
\item Você percebeu vantages e desvantagens de se utilizar os coeficientes e taxas? Quais?
\item Como a sua região estava afetada pela COVID-19 naquele momento? Faça a comparação com os dados de outras regiões.
\item Compartilhe suas descobertas com seus colegas.
\end{enumerate}

\begin{table}[H]
\centering
\setlength\tabcolsep{4pt}
\begin{tabu} to \textwidth{|c|r|r|r|r|r|r|}
\hline
\thead
Região & População & Casos & Óbitos & \makecell{Coeficiente de \\ Incidência} & \makecell{Coeficiente de \\ Mortalidade} & \makecell{Taxa de \\ Letalidade} \\
\hline
Centro-Oeste & 16 297 074 & 5 090 & 129 & & & \\
\hline
Nordeste & 57 072 654 & 58 316 & 3 568 & & & \\
\hline
Norte & 18 430 980 & 30 900 & 2 190 & & & \\
\hline
Sudeste & 88 371 433 & 74 727 & 6 216 & & & \\
\hline
Sul & 29 975 984 & 8 556 & 297 & & & \\
\hline
\end{tabu}
\caption{Fonte: \href{https://COVID.saude.gov.br/}{Ministério da Saúde} (Consulta em 12 de Maio de 2020)}
\end{table}
\end{task}

\begin{example}{Outras dimensões da epidemia}
	
Antes da pandemia chegar em terras tupiniquins, acompanhávamos os impactos catastróficos causados em outras regiões do mundo. Receosos, especulávamos sobre como, quando e em que medida ela atingiria nossa nação? Seria também catastrófica ou seria apenas uma gripezinha? Nosso sistema de saúde público conseguiria dar conta? Todos os indivíduos teriam o mesmo acesso ao tratamento? E os impactos em nossa economia? Quais setores produtivos seriam considerados essenciais para nossa sociedade? E aqueles trabalhadores de setores considerados não essenciais, como poderiam garantir o seu sustento? Sobre a estratégia do isolamento social, como colocá-la em prática em locais com grandes aglomerados de pessoas, como nas favelas, nos presídios, etc.? Resumindo, considerando a nossa estrutura social, econômica e cultural, a pergunta geral era: como faríamos para enfrentar a pandemia?

Quando a doença deu seus primeiros sinais por aqui, essas questões começaram a ter respostas concretas. Pesquisadores de várias áreas, sobretudo nas universidades e institutos de pesquisa, trabalhavam arduamente para compreender o seu avanço.

O estudo da Funcação Oswaldo Cruz\footnote{Que pode ser acessado em \url{  https://agencia.fiocruz.br/estudo-aponta-maior-aceleracao-da-COVID-19-no-norte-e-nordeste}}, divulgado em 29 de maio, mostrou que o ritmo de aumento do avanço da covid-19 nas regiões Norte e Nordeste foi muito maior do que no restante do país. O estudo sugeriu a existência de uma relação entre esse aumento e os recursos de cada região.

\begin{quote}
"O Amazonas passou de uma taxa de 493 casos por milhão de habitantes em abril, para 5 300 em maio. O Amapá passou de 492 para 5 100. Roraima, de 366 para 3 266 e Ceará, d3 356 para 3 078. Na comparação, estados com mais recursos parecem menos atingidos pela pandemia, como por exemplo o Paraná, onde a taxa passou de 86 para 217, e o Rio Grande do Sul, de 75 para 329."
\flushleft

(Agência Fiocruz de notícias, 29/05/2020)
\end{quote}

Ou \href{https://drive.google.com/file/d/1tSU7mV4OPnLRFMMY47JIXZgzkklvkydO/view }{estudo}, do Núcleo de Operações e Inteligência em Saúde (NOIS), da PUC-Rio, divulgado no dia 28 demaio, realizou uma análise socioeconômica da taxa de letalidade da COVID-19 no Brasil buscando mostrar que as altas taxas são influenciadas pelas desigualdades no acesso ao tratamento. Para chegas às conclusões, foram cruzadas informações obtidas em duas plataformas públicas: o \href{https://opendatasus.saude.gov.br/}{OpenDataSUS}, que traz dados sobre a saúde no Brasil, e o \href{http://atlasbrasil.org.br}{Atlas de Desenvolvimento Humando do Brasil}, que traz indicadores sociais e econômicos, como o Índice de Desenvolvimento Humano (IDH), o índice de Gini, o Índice de Vulnerabilidade Social, a distribuição da população por classe, gênero e raça, por nível de escolaridade entre outros.

Um dos gráficos apresentados por esse estudo, por exemplo, mostrou a proporção de óbitos ou recuperados por escolaridade e por Raça/Cor.

\begin{figure}[H]
\centering
\includegraphics[width=\linewidth]{investigacao2}

\end{figure}

Impactos na economia também começaram a aparecer mais fortemente em indicadores como o Produto Interno Bruto (PIC) e as taxas de ocupação do trabalho. Em 29 de maio, o IBGE divulgou que "o PIB apresentou contração de 1,5\% na comparação do primeiro trimestre de 2020 contra o quarto trimestre de 2019, na série com ajuste sazonal. A Indústria (-1,4\%) e os Serviços (-1,6\%) apresentaram recuo, enquanto a Agropecuária (0,6\%) cresceu"\footnote{\url{https://agenciadenoticias.ibge.gov.br/agencia-sala-de-imprensa/2013-agencia-de-noticias/releases/27837-pib-cai-1-5-no-1-trimestre-de-2020}}. Quanto às taxas de ocupação do trabalho, dados da Pesquisa Nacional por Amostra de Domicílios Contínua (PNAD Contínua), revelaram que "a taxa de desocupação passou de 11,2\% para 12,6\% no trimstre terminado em abril, atingindo 12,8 milhões de desempregados"\footnote{\url{https://agenciadenoticias.ibge.gov.br/agencia-noticias/2012-agencia-de-noticias/noticias/27821-desemprego-atinge-12-6-no-trimestre-ate-abril-com-queda-recorde-na-ocupacao}}.

Como percebemos, os dados nos ajudam a compreender diversas dimensões de uma situação. Tratar de cada uma delas separadamente pode nos dar objetividade, mas certamente é insuficiente para propor soluções amplas. Por isso, podemos dividir tarefas para organizar essas informações, mas é sempre importante trocarmos experiências para compreender a situação de forma mais ampla e não cair em ilusões de soluções simplistas para problemas complexos.
\end{example}


\know{Pirâmide etária}

Não é um indicador na forma de um número, mas sim da distribuição de idades de uma população. A partir dela é possível obter informações sobre natalidade, longevidade e idade média da população.

Do ponto de vista matemático, este indicador é apenas uma contagem de quantas pessoas existem por faixa de idade.

No Brasil este indicador é produzido anualmente pelo IBGE, por meio da Pesquisa Nacional por Amostra de Domicílio Contínua (PNAD Contínua). Na imagem a seguir apresenta uma comparação entre os dados de pirâmide etária do Brasil de 2012 e 2019. Quais informações sobre natalidade, longevidade e idade média da população obter por meio dessa comparação?

\begin{figure}[H]
\centering

\includegraphics[width=\linewidth]{investigacao7}
\end{figure}

\know{Índice de atendimento total de esgoto}

Este é um indicador que tenta traduzir em um único número a quantidade de pessoas de um dado município, estado, região ou país atendidas pela rede de esgoto. Ele é calculado por meio de uma razão:


\begin{equation*}
\resizebox{\hsize}{!}{
$
\text{Índice de atendimento total de esgoto}=\frac
{\text{População atendida com esgoto na região de interesse (área rural e urbana)}}
{\text{População total da região de interesse}}
$
}
\end{equation*}


O índice varia entre $0$ e $1$, mas pode ser também apresentado em forma percentual. Quanto mais próximo de $1$ (ou de $100$, no caso da porcentagem), maior é a proporção de pessoas da região de interesse com atendimento de esgoto.

No Brasil o índice é calculado pelo Sistema Nacional de Informação sobre Saneamento (SNIS). Observe o gráfico abaixo produzido com os dados de 2018. A partir dele o que você observa sobre a coleta de esgoto no Brasil?



\begin{figure}[H]
\centering

\includegraphics[width=.8\textwidth]{investigacao8}

\caption{Fonte: Ministério do Desenvolvimento Regional - Sistema Nacional de Informação sobre Saneamento. Gráfico obtido do painel de informações sobre saneamento no Brasi \href{http://www.snis.gov.br/painel-informacoes-saneamento-brasil/web/painel-esgotamento-sanitario}{Sistema Nacional de Informação sobre Saneamento. Gráfico obtido do painel de informações sobre saneamento no Brasi}}
\end{figure}

\know{Leitos hospitalares por mil habitantes}

Indicador sobre a quantidade de leitos hospitalares disponíveis pelo sistema público de saúde para a população. Ele pode ser utilizado como uma medida de acesso à saúde. No Brasil, os dados são coletados e disponibilizados pelo Sistema Único de Saúde (SUS), por meio do sistema DataSUS. 

Seu cálculo é dado pela razão
\begin{equation*}
\resizebox{\hsize}{!}{
$
\text{Leitos hospitalares por mil habitantes}=\frac
{\text{Número médio anual de leitos hospitalares conveniados ou contratados pelo SUS}}
{\text{População total residente, ajustada para o meio do ano}}\times1.000
$
}
\end{equation*}

Para saber mais porque a razão é multiplicada por mil, volte e leia a discussão do \hyperref[primeiras-informacoes]{Exemplo 1: Primeiras informações sobre a pandemia}. 

Com este indicador é possível identificando situações de desigualdade entre regiões do país e tendências nas variações geográficas e temporais de leitos hospitalares ofertados pelo SUS. Ele também pode ser usado para subsidiar o planejamento de políticas públicas voltadas para a assistência hospitalar no país.

Na tabela a seguir é possível encontrar os dados temporais de disponibilidades de leitos. Quais desigualdades e tendências você observa?

\begin{figure}[H]
\centering

\includegraphics[width=.9\textwidth]{investigacao9}
\end{figure}


\know{Índice de vulnerabilidade social (IVS)}
Este indicador é formado pela composição de vários outros, é o que se chama de indicador multidimensional. Sua intenção é resumir em um único valor múltiplos parâmetros relacionados ao bem estar de uma população. Estes parâmetros são uma composição de indicadores de três grandes grupos:

\begin{itemize}
\item \textbf{Capital humano}: calculado por meio de uma média simples de 8 indicadores.
\item \textbf{Renda e trabalho}: calculado por meio de uma média simples de 5 indicadores.
\item \textbf{Infraestrutura urbana}: calculado por meio de uma média ponderada de 3 indicadores.
\end{itemize}

Por fim, o IVS é calculado por meio de uma média simples dos três indicadores que o compõe:

\begin{equation*}
\text{IVS}=\frac
{\text{IVS\sub{capital humano}+IVS\sub{renda e trabalho}+IVS\sub{infraestrutura urbana}}}
{3}
\end{equation*}

\begin{table}[H]
\centering

\begin{tabu} to \textwidth{|l|r|r|r|r|}
\hline
\tmcol{5}{|c|}{\parbox[c][1cm][c]{.6\textwidth}{ \centering Índice de Vulnerabilidade Social  Capitais da Região Norte do Brasil - 2010}} \\
\hline
\tmcol{1}{|c|}{Cidade} & \tmcol{1}{c|}{IVS Total} & \tmcol{1}{c|}{\makecell{IVS \\ Infraestrutura urbana}} & \tmcol{1}{c|}{\makecell{IVS Capital\\humano}} & \tmcol{1}{c|}{\makecell{IVS Renda e \\ trabalho}} \\
\tcolor{Rio Branco - AC} & $0{,}339$ & $0{,}276$ & $0{,}433$ & $0{,}307$ \\
\hline
\tcolor{Macapá - AP} & $0{,}339$ & $0{,}271$ & $0{,}433$ & $0{,}307$ \\
\hline
\tcolor{Manaus - AM} & $0{,}387$ & $0{,}458$ & $0{,}388$ & $0{,}314$ \\ 
\hline
\tcolor{Porto Velho - RO} & $0{,}322$ & $0{,}372$ & $0{,}364$ & $0{,}230$ \\
\hline
\tcolor{Boa Vista - RR} & $0{,}261$ & $0{,}157$ & $0{,}362$ & $0{,}265$ \\
\hline
\end{tabu}

\caption{Fonte: Atlas de Vulnerabilidade Social - IPEA}
\end{table}

Para saber quais os indicadores que compõem cada um dos grupos do IVS e seus respectivos pesos, você pode acessar a publicação do IPEA “\href{http://ivs.ipea.gov.br/images/publicacoes/Ivs/publicacao_atlas_ivs.pdf}{Atlas da vulnerabilidade social dos municípios brasileiros}”. Nele, na seção de “conceito e metodologia” você pode encontrar a lista completa dos indicadores que compõem o IVS.

\begin{task}{}

O texto acima trouxe a ideia de que podemos utilizar informações quantitativas e indicadores para compreender algumas dimensões dos problemas causados pela pandemia.

\begin{enumerate}
\item A seguir apresentamos as definições de alguns indicadores. Escreva como você imagina que cada um deles pode estar relacionado com a pandemia.

\begin{itemize}[itemsep=1em]
\setlength\parskip{-2pt}
\item \textbf{Pirâmide Etária}

\textbf{Descrição}: Quantidade de pessoas de uma região separadas por faixa etárias.

\textbf{Fonte}: IBGE - \href{https://www.ibge.gov.br/estatisticas/sociais/populacao/25089-censo-1991-6.html?=&t=o-que-e}{Censo} e \href{https://www.ibge.gov.br/estatisticas/sociais/populacao/9173-pesquisa-nacional-por-amostra-de-domicilios-continua-trimestral.html?t=destaques}{PNAD Contínua}

\item \textbf{Leitos hospitalares por habitante}

\textbf{Descrição}: Proporção do número de leitos hospitalares por habitante

\textbf{Fontes}: \href{https://datasus.saude.gov.br/}{Ministério da Saúde - DataSUS}; Secretarias de Saúde Estaduais e Municipais

\item \textbf{Índice de atendimento total de esgoto referido aos municípios atendidos com água}

\textbf{Descrição}: Percentual da população atendida por rede coletora de esgoto (com ou sem tratamento) em relação à população total.

\textbf{Fonte}:\href{http://www.snis.gov.br/painel-informacoes-saneamento-brasil/web/painel-setor-saneamento}{Sistema Nacional de Informações sobre Saneamento (SNIS)}


\item \textbf{Índice de Vulnerabilidade Social (IVS)}

\textbf{Descrição}: Indicador que sintetiza informações de infraestrutura, escolaridade e trabalho de grupos populacionais.

\textbf{Fonte}: \href{http://ivs.ipea.gov.br/index.php/pt/}{IPEA} (com dados do censo de IBGE)
\end{itemize}

\item Você conhece outras dimensões dos problemas causados pela pandemia? Que tipos de indicadores você acha que seriam importantes para analisar essas dimensões?

\end{enumerate}


\end{task}

\explore{Elaboração de Questões - etapa 2}
\phantomsection\label{etapa2}

\phantomsection\label{pergunta-cientifica}
\vspace{-.5\baselineskip}
\section{O que é uma boa pergunta científica?}

Nós, seres humanos, podemos ser muito diferentes uns dos outros. No entanto, existe algo que nos unifica: absolutamente, todos nós temos problemas. Decerto, eles podem variar muito. Algumas pessoas precisam se preocupar constantemente com o trabalho que precisam executar para conseguir alguma remuneração para comprar produtos que consideram necessários a sua sobrevivência. Outras passam longe desse tipo de preocupação, mas muitas vezes não encontram sentido profundo em suas vidas, deprimindo-se.

Em ambos os casos, não há escapatória: se não podemos solucionar esses problemas seremos soterrados por eles. Portanto, nosso cotidiano é uma dimensão da vida na qual precisamos constantemente dar respostas aos nossos problemas. De preferência, rapidamente, para que não se acumulem e nos prejudiquem.

Deste modo, a qualidade das nossas respostas interfere diretamente em nosso cotidiano. E conseguimos dar respostas melhores quando conseguimos entender melhor nossos problemas, quando conseguimos elaborá-los na forma de perguntas. Em outras palavras, conseguimos expressar nossa humanidade de maneira mais vívida quando somos curiosos.  

Mas a velocidade exigida pelo cotidiano dificulta a formulação das perguntas. A necessidade de respostas rápidas desorienta nossos esforços em busca de exercitar nossa curiosidade. É por isso que outra dimensão da vida é imprescindível: a ciência.

Independentemente da área do conhecimento, o desenvolvimento científico é todo baseada em perguntas. Quando querem investigar algum processo, fenômeno ou aparato, os cientistas elaboram uma pergunta relacionada ao seu objeto/tema de estudo e se concentram nela. 

A investigação desenvolvida neste capítulo tem limitações se comparada ao trabalho dos cientistas. Uma delas é a questão do tempo, dado que o tempo da escola é muito diferente do tempo do universo acadêmico. Assim, neste projeto não é possível realizar questionamentos muito complexos e investigações realmente aprofundadas. Para adequar a proposta ao tempo e objetivos da escola, as perguntas científicas trabalhadas aqui precisam seguir certos critérios:

\begin{itemize}
\item elas devem ser quantitativas, ist é, devem estar relacionadas a quantidades e/ou informações numéricas;
\item precisam ser exequíveis num período de tempo apropriado (e por isso não podem demandar anárlises muito longas ou complexas);
\item é necessário que sejam comparativas, ou seja, avaliar diferentes situações ou a mesma situação em localidades diferentes, etc;
\item busquem ser interessantes, de forma que não sejam respondidas com uma simples busca na internet;
\item e sjeam simples, restritas a situações pouco complexas ou olhar para apenas um fator de uma determinada situação.
\end{itemize}


% \begin{table}[H]

% \centering
% \begin{tabu} to \textwidth{|c|>{\vspace{3pt}}m{.3\textwidth}<{\vspace{3pt}}|m{.3\textwidth}|}
% \hline
% \thead
% Característica & Adequada & Inadequada \\
% \hline
% \cellcolor{white}\textcolor{black}{\textbf{Quantitativa}} & O índice de pessoas afetadas pela Covid em minha escola é maior ou menor do que na minha cidade? & Como se sentiram as pessoas infectadas em minha escola?\\
% \hline
% \cellcolor{white}\textcolor{black}{\textbf{\makecell{Exequível em \\ tempo apropriado}}} & Qual a taxa de ocupação das famílias de minha turma antes e depois da pandemia? & Qual a taxa de pessoas ocupadas em minha região antes e depois da pandemia? \\
% \hline
% \cellcolor{white}\textcolor{black}{\textbf{Comparativa}} & A taxa de letalidade da Covid é a mesma nas comunidades indígenas e nos grandes centros urbanos brasileiros? & A taxa de letalidade da Covid é a mesma para diferentes populações do Brasil?\\
% \hline
% \cellcolor{white}\textcolor{black}{\textbf{Sedutora/interessante}} & Qual foi o impacto da quarentena em minha região na contenção da pantemia? & Quam não tem acesso à rede de água e esgoto foi mais afetado pela Covid? \\
% \hline
% \cellcolor{white}\textcolor{black}{\textbf{Simples}} & Qual região da minha cidade apresentou a maior taxa de disseminação da Covid? & Qual a taxa de disseminação da Covid em minha turma? \\
% \hline
% \end{tabu}
% \end{table}

% Vejam que as perguntas da segunda coluna apresentam características que, embora não as tornem desnecessárias para a compreensão da realidade, as tornam inviáveis no contexto da sala de aula. Elas não permitem um aprofundamento metódico nos estudos matemáticos se estiverem orientadas a tentar compreender a subjetividade dos sujeitos envolvidos. Tampouco pode ser objetiva e, ao mesmo tempo, extremamtente abrangente, inviabilizando sua feitura nos prazos possíveis. O mesmo ponto sobre a abrangência vale para os casos de comparação possíveis: quanto mais genérica a comparação, mais difícil será efetivá-la. E, por fim, as perguntas não podem querer validar sua opinião, seja na confirmação de uma opinião amplamente aceita, seja na limitação rasteira do seu objeto.

% Você não precisa se preocupar em tentar solucionar os grandes problemas do mundo. Pelo menos não por agora. Afinal, estas grandes questões demandam um esforço que foge aos limites da pesquisa em sala de aula.

% Mas isso não significa que você deva abandonar esse desejo. Ao contrário, nossa intenção é que você seja cada vez mais capaz de alcançar essa possibilidade. Para isso, cabe um exercídio: dividir essa pergunta comiplexa em perguntas mais simples. Ao serem satisfatoriamente respondidas, nos enriquecem do conhecimento necessário para tratar de questões mais complexas, nos fazendo galgar degraus até nosso interesse final.

\begin{task}{Elaboração de boas perguntas científicas}

Elabore três perguntas investigadas sobre as pectos do seu tema. Tente seguir os critério mencionados anteriormente. Utilize as atividades do "\hyperref[primeiras-informacoes]{Explorando: Primeiras Informações Sobre a Pandemia}"{} como inspiração. Ao fazer as atividades e ler o texto daquela seção, quais curiosidades e questionamentos vieram à sua mente?

Uma dica: Ao pensar em perguntas tente imaginar como você faria para respondê-las. Que tipo de dados, medidas ou observações seriam necessárias? Se você não souber por onde começar, provavelmente você pensou em uma pergunta é complexa

Depois de verificar com sua professora ou seu professor se as três questões formuladas atendem todas as características para se enquadrarem como uma boa pergunta, será hora de se juntar com colegas que tenham interesse por recortes temáticos parecidos com o seu e definir qual será a pergunta de investigação de seu grupo.

Escreva em seu caderno a pergunta na qual você irá se aprofundar. Lembre-se que ela tem que atender a todas as características discutidas no texto "\hyperref[pergunta-cientifica]{O que é uma boa pergunta científica}".

\end{task}

\begin{task}{elaboração de questões do projeto sobre COVID-19}

Veja o diálogo entre o estudante Rodrigo e a sua professora.

\begin{quote}
\textbf{Rodrigo}: Professora, estive pensando em investigar como a pandemia afetou os nossos empregos.

\textbf{Professora}: Este me parece um questionamento interessante, Rodrigo. Mas vejamos se ele atende nossos critérios. Em primeiro lugar, como expressar esse problema de forma quantitativa?

\textbf{Rodrigo}: Não sei exatamente, professora.

\textbf{Professora}: Bem, penso que precisamos de indicadores de emprego. Você conhece algum?

\textbf{Rodrigo}: Eu vi matérias nos jornais dizendo que o desemprego estava alto. Falaram que $13\%$ das pessoas estavam desempregadas.

\textbf{Professora}: Exatamente, a taxa de desemprego é um indicador, que no Brasil é calculado pelo IBGE. É possível também olhar para a taxa de pessoas ocupadas, que são aquelas exercem algum trabalho. O que você prefere olhar? Como podemos então formular a pergunta?

\textbf{Rodrigo}: Acho que prefiro olhar para as pessoas ocupadas. A pergunta poderia então ser: “qual é a taxa de pessoas ocupadas durante a pandemia?”

\textbf{Professora}: De quais pessoas você está falando?

\textbf{Rodrigo}: Pensei em investigar entre os colegas de turma. Talvez perguntar sobre os familiares da minha classe. 

\textbf{Professora}: Legal. Você já tem uma pergunta que é simples, quantitativa e que me parece ser possível de respondermos em um tempo adequado. Mas ela não é comparativa, e também não responde à questão de como os empregos foram afetados pela pandemia.

\textbf{Rodrigo}: Podemos perguntar quais eram as taxas de ocupação dos familiares de minha classe antes da pandemia e compará-la com as atuais?
Professora: Creio que estamos num bom caminho. Tente formular a pergunta de forma objetiva.

\textbf{Rodrigo}: Qual era a taxa de ocupação dos familiares da minha classe antes da pandemia e qual é a taxa atual? Que tal?

\textbf{Professora}: Para deixar sua pergunta mais interessante, que acha de comparar essas taxas com as do país? Assim você teria uma ideia da relação entre a nossa turma e a realidade nacional.

\textbf{Rodrigo}: Gosto da ideia! Então ficaria assim: Qual era a taxa de ocupação dos familiares da minha classe antes da pandemia e qual é a taxa atual? E como essas taxas se comparam com as nacionais?

\end{quote}

Repare no diálogo entre Rodrigo e sua professora, apresentado no exemplo logo acima. Perceba que ele começou elaborando uma pergunta que não se encaixava em todos os critérios necessários para o projeto, mas que com ajuda foi reformulando a questão até ter algo que atendia a todos os critérios estabelecidos para o projeto.

As questões elaboradas pelo Rodrigo nesse processo, embora não sejam desnecessárias para a compreensão da realidade, são inviáveis no contexto de sala de aula. Elas não permitem um aprofundamento metódico nos estudos matemáticos se estiverem orientadas a tentar compreender a subjetividade dos sujeitos envolvidos. Tampouco pode ser objetiva e, ao mesmo tempo, extremamente abrangente, inviabilizando sua feitura nos prazos possíveis. O mesmo ponto sobre a abrangência vale para os casos de comparação possíveis: quanto mais genérica a comparação, mais difícil será efetivá-la. E, por fim, as perguntas elaboradas para o projeto não podem querer validar sua opinião, seja na confirmação de uma opinião amplamente aceita, seja na limitação rasteira do seu objeto.

Você não precisa se preocupar em tentar solucionar os grandes problemas do mundo. Pelo menos não por agora. Afinal, estas grandes questões demandam um esforço que foge aos limites da pesquisa em sala de aula.

Isso não significa que você deva abandonar esse desejo. Ao contrário, nossa intenção é que você seja cada vez mais capaz de alcançar essa possibilidade. Para isso, cabe um exercício: dividir essa pergunta complexa em perguntas mais simples.

\end{task}


\arrange{Planejando a investigação - etapa 3}
\phantomsection\label{etapa3}

\begin{task}{Roteiro de planejamento da investigação}

A seguir você encontrará algumas perguntas orientadoras para a formulação do planejamento. Em conjunto com seu grupo, você deve responder cada uma das questões, registrando suas respostas em uma ficha ou no caderno. 

\begin{enumerate}[label=\titem{\arabic*)}]
\item Escreva as hipóteses que o você e seu grupo tem sobre a pergunta que escolheram para investigar.

\item A partir de notícias e outras fontes de informação sobre o tema de sua pergunta investigativa façam uma pesquisa e registrem quais são os principais indicadores deste tema.

\item Quais dados vocês usaram para responder a pergunta proposta? Justifiquem explicando a relação entre os dados escolhidos por vocês e a relação deles com a pergunta investigativa do grupo.

\item Quais são as fontes de dados confiáveis para os dados escolhidos por vocês?

\item Expliquem como vocês utilizarão os dados para responder a pergunta investigativa do grupo. Dêem exemplos. Que cálculos vocês farão com esses dados e o que esperam obter? Não se esqueçam que ao final, a análise deve servir para testar a(s) sua hipótese(s) inicial(is).

\item Façam um esboço do(s) tipo(s) de gráfico(s) que vocês utilizarão para apresentar seus dados. Abusem das cores e criatividade em seu esboço. Não se esqueçam de identificar quais dados vocês esperam apresentar em cada gráfico que fizerem. 
\end{enumerate}
\end{task}

\subsection{Hipóteses}

Depois de elaborada a pergunta, é hora de pensar em como respondê-la. É provável que você, ao formular sua pergunta já tenha um palpite do que pode ser a resposta. Ele pode parecer bastante coerente na sua mente, mas é possível que você não consiga expressá-lo em palavras com a mesma desenvoltura. Trata-se mais de intuição do que de raciocínio.

Considerando que a ciência é uma atividade que trabalha com explicações baseadas em evidências, é necessário testarmos nossas ideias intuitivas por meio de observações ou tomadas e análises de dados. Um projeto científico possui esta finalidade: confrontar o pensamento intuitivo com a realidade, demonstrando o alcance de uma certa resposta ao problema colocado a partir de um conjunto de etapas passíveis de serem reproduzidas (verificadas).

A primeira etapa de uma investigação é entender o que você imagina  como resposta para sua pergunta, ou seja, materializar sua intuição em linguagem conceitual, formular aquilo que chamamos de hipótese. Observe o exemplo a seguir, no qual é apresentada uma hipótese para o projeto fictício sobre a COVID-19.


\begin{example}{Planejamento da investigação - COVID-19}

Depois da elaboração das perguntas, a Priscila e o Thiago se juntaram ao Rodrigo para investigar a questão proposta sobre as taxas de ocupação antes e depois da pandemia. Veja o diálogo entre eles:

\begin{quote}
\textbf{Rodrigo}: Olá pessoal, lembro que nossa questão diz respeito às taxas de ocupação dos familiares de nossa classe antes e depois da pandemia, bem como compará-las com as taxas nacionais. Quais são, neste caso, as nossas hipóteses?

\textbf{Thiago}: Bom, eu acho que não mudou muito. Os meus pais continuam trabalhando normalmente.

\textbf{Priscila}: Na minha família teve muita gente que não está conseguindo trabalhar. Meu pai mesmo é vendedor e não pode sair, pois é do grupo de risco. E minha mãe teve que abandonar o emprego pois precisa cuidar da minha irmã pequena em casa.

\textbf{Rodrigo}: Em casa também não mudou muito, mas direto a televisão mostra notícias sobre o aumento do desemprego. Então eu acho que deve ter mudado sim.

\textbf{Thiago}: É, talvez tenha mudado para a maioria. Acho que essa é nossa hipótese.

\textbf{Priscila}: Isso! A nossa hipótese é que muita gente perdeu o emprego. E no Brasil vai ser igual a escola, né?
\end{quote}

\end{example}


\subsection{Uso de indicadores}

As explicações que construímos sobre a realidade passam pelas ações de identificar e comparar. Percebemos a existência de diferenças e, em seguida, buscamos um jeito de compará-las, usando essas informações para então construir nossas explicações. Uma forma de fazer isso é através da utilização de indicadores. 


Um indicador é um recurso metodológico. Para nossos propósitos, nos interessa que ele seja empiricamente aferido, quantitativo, e que sintetize informações sobre aspectos da realidade ou sobre suas transformações. Um indicador com essas características procura dar objetividade a um conceito que pode ser, a princípio, abstrato.


Podemos representar sua construção em basicamente três etapas:

\begin{enumerate}[label=\titem{\arabic* -}]
\item percepção de eventos empíricos da realidade;
\item obtenção de dados brutos e estatísticas sobre o evento;
\item sintetização dos dados procurando dar-lhes um valor informacional sobre certos contextos.
\end{enumerate}

\begin{figure}[H]
\centering

\begin{tikzpicture}[every node/.style={scale=.9}]
\tikzstyle{quad}=[draw, rectangle, node distance=6cm, align=center, minimum height=2.5cm, minimum width=4cm, fill=\tikzcolor, font=\bfseries, text=white];

\node (a) [quad] {Eventos \\ empíricos \\ da realidade};
\node (b) [quad, right of=a] {Obtenção \\ de dados \\ brutos e \\ estatísticas};
\node (c) [quad, right of=b] {Sintetização e \\ agregação de valor \\ informacional para \\ analisar contextos};

\path[->,very thick]
(a) edge (b)
(b) edge (c);
\end{tikzpicture}

\end{figure}

Apesar da busca de objetividade, não podemos esquecer que a construção de qualquer indicador é realizada por grupos de indivíduos influenciados histórica e socialmente, e que portanto colocam nesses indicadores determinadas visões de mundo. Desta forma, um indicador pode ser objetivo, mas jamais é neutro ou imparcial.

Na maior parte das vezes, a matemática envolvida nos indicadores é bastante simples, se baseando conteúdos como razões, proporções, percentuais, médias, etc. Uma das partes mais difíceis é elaborar categorias e instrumentos de pesquisa que consigam dar conta de expressar a parte da realidade desejada.

A atividade a seguir busca mostrar como esse processo é complicado mesmo com questões aparentemente simples, como encontrar um indicador de subutilização da força de trabalho.

\begin{task}{indicadores da força de trabalho}
A ideia aqui é construir um indicador de quanta força de trabalho está sendo desperdiçada. Uma das possibilidades é simplesmente calcular, dentro da população, quantas estão desempregadas. No entanto, calcular isso desta forma seria muito simples e não captaria outras condições relacionadas à ocupação, como a questão dos indivíduos poderem ou não trabalhar, ou de estarem trabalhando menos do que poderiam, ou ainda sobre as ações tomadas pelos indivíduos na tentativa de realizar algum trabalho. Para dar conta de algumas dessas questões, o IBGE elabora a seguinte divisão das pessoas em idade de trabalhar:

\begin{figure}[H]
\centering

\begin{tikzpicture}[
level/.style={sibling distance = 5.5cm/#1,
level distance = 2cm, scale=1.25}, 
every node/.style={draw, rectangle, minimum width=3cm, minimum height=2cm, node distance=1., label distance=-.45cm, fill=\tikzcolor, text=white, font=\bfseries\small,},
align=center, rounded corners
				   ] 

\node (a) {Pessoas em idade \\de trabalhar \\(14 anos ou\\ mais de idade)}
	child {node [yshift=2cm]{Pessoas na \\ força de \\ trabalho}
			child {node [xshift=2.5cm]  {Pessoas \\ ocupadas}
					}
			child {node [yshift=-2.5cm]  {Subocupadas \\ por insuficiência \\ de horas \\ trabalhadas}
					}
			child {node [xshift=-2.5cm] {Pessoas \\ desocupadas}
					}
			}
	child {node  [yshift=2cm] {Pessoas fora \\ da força \\ de Trabalho}
			child {node [xshift=2.5cm] {Pessoas na\\  força de \\ trabalho \\ não potencial}
					}
			child {node [yshift=-2.5cm]{Pessoas na \\ força de \\ trabalho \\ potencial}
					}
			child {node [xshift=-2.5cm] {Pessoas \\ desalentadas}
					}
			}						
;
\end{tikzpicture}
\end{figure}
\end{task}

Em outras palavras, o IBGE divide as pessoas com 14 anos ou mais de idade em pessoas “na força de trabalho”{} e “fora da força de trabalho”. Dentro da força de trabalho, considera as categorias “ocupadas”{} e “desocupadas”. Dentro das ocupadas, considera ainda as “subocupadas”. Dentre as que estão fora da força de trabalho, o IBGE considera aquelas que formam uma “força de trabalho potencial”. E dentro dessa categoria, considera ainda as “pessoas desalentadas”.
\clearpage
A seguir colocamos algumas breves descrições de cada categoria:

\begin{table}[H]
\centering
\begin{tabu} to \textwidth{|>{\centering}m{.2\textwidth}|>{\vspace{2.5pt}}m{.6\textwidth}<{\vspace{2.5pt}}|}
\hline
\tcolor{Categoria} & \tmcol{1}{|c|}{Descrição} \\
\hline
Força de Trabalho & Pessoas disponíveis para trabalhar. \\
\hline
Ocupadas & Pessoas que exercem algum tipo de trabalho. \\
\hline
Desocupadas & Pessoas que não estão exercendo trabalho e que estão efetivamente em busca de conseguir um emprego. \\
\hline
Subocupadas & Pessoas que trabalham menos de 40 horas na semana e que estão dispostas a trabalhar mais. \\
\hline
Força de Trabalho Potencial & Pessoas que buscaram emprego mas que não podiam assumir no período considerado, ou pessoas que estavam disponíveis para trabalhar mas que não realizaram buscas por emprego. \\
\hline
Desalentadas & Pessoas desempregadas que gostariam de trabalhar, mas que não buscaram emprego pois não tinham esperanças de conseguir. \\
\hline
Força de Trabalho Ampliada & É a soma das pessoas na força de trabalho com as pessoas na força de trabalho potencial. \\
\hline
\end{tabu}
\end{table}

A tabela a seguir mostra os dados brutos da PNAD contínua do segundo trimestre de 2020 referente à essas categorias.

\begin{table}[H]
\centering

\begin{tabu} to \textwidth{|l|c|}
\hline
\tmcol{1}{|c|}{Categoria} & \tcolor{\makecell{Total\\(em milhares de pessoas)}} \\
\hline
Pessoas em idade de trabalhar & 173.918 \\
\hline
Força de Trabalho & 96.138 \\
\hline
Fora da Força de Trabalho & 77.781 \\
\hline
Ocupadas & 83.347 \\
\hline
Desocupadas & 12.791 \\
\hline
Subocupadas & 5.613 \\
\hline
Força de Trabalho Potencial & 13.542 \\
\hline
Desalentadas & 5.683 \\
\hline
Força de Trabalho Ampliada & 109.680 \\
\hline
\end{tabu}
\end{table}

\begin{enumerate}
\item Considerando as categorias e dados acima, qual cálculo você faria para representar a taxa de desemprego? Justifique.
\item Considerando as categorias e dados acima, qual cálculo você faria para representar a taxa de subutilização da força de trabalho? Justifique.
\item Discuta com seus colegas os cálculos que você propôs.
\item Consulte as taxas calculadas pelo IBGE e compare-os com os que foram criados por você e por seus colegas.
\end{enumerate}

\clearpage
\begin{example}{Planejamento da investigação - COVID-19}
Após compreender o que são indicadores e para que eles servem, o grupo de Thiago, Rodrigo e Priscila formulou as seguintes respostas para as questões \titem{2)} e \titem{3)} do roteiro de organização da investigação:

\begin{enumerate}[label=\titem{\arabic*)}]\setcounter{enumi}{1}
\item A partir de notícias e outras fontes de informação sobre o tema de sua pergunta investigativa façam uma pesquisa e registrem quais são os principais indicadores deste tema.

\titem{Resposta}: \textit{Por meio de nossa pesquisa descobrimos os seguintes indicadores:}

\begin{table}[H]
\centering

\begin{tabu} to \textwidth{|>{\vspace{2.5pt}}m{.3\textwidth}<{\vspace{2.5pt}}|>{\vspace{2.5pt}}m{.2\textwidth}<{\vspace{2.5pt}}|>{\vspace{2.5pt}}m{.4\textwidth}<{\vspace{2.5pt}}|}
\hline
\tmcol{1}{|c|}{Indicador} & \tmcol{1}{c|}{De onde ele é} & \tcolor{Qual informação ele traz} \\
\hline
Taxa de ocupação & PNAD-IBGE & Pessoas na força de trabalho que estão ocupadas \\
\hline
Taxa de de desocupação & PNAD-IBGE & Pessoas na força de trabalho que estão procurando emprego \\
\hline
Força de trabalho & PNAD-IBGE & Total de pessoas em idade de trabalhar que estão dispostas a trabalhar \\
\hline População subutilizada & PNAD-IBGE & É o total de pessoas desocupadas, subocupadas, desalentadas ou que não podem assumir um emprego\\
\hline
Indicador de Antecedente de Emprego & Fundação Getúlio Vargas & Sondagem das expectativas do mercado de trabalho.\\
\hline
\end{tabu}
\end{table}

\item Quais dados vocês usaram para responder à pergunta proposta? Justifiquem explicando a relação entre os dados escolhidos por vocês e a relação deles com a pergunta investigativa do grupo.

\titem{Resposta}: \textit{Vamos utilizar todos os indicadores da PNAD-IBGE na nossa tabela.}
\end{enumerate}
\end{example}

\subsection{Fontes confiáveis}


A pesquisa científica se diferencia do nosso conhecimento cotidiano pelo método e pela busca apurada de fontes de informação. Não há atividade de pesquisa sem coleta de dados, sejam quantitativos ou qualitativos. Estes ainda podem ser divididos em três grupos, segundo Antônio Gil (2002):

\begin{itemize}
\item \textbf{Fontes primárias}: também chamadas de originais, são objetos relacionados à temática da pesquisa que não apresentam nenhuma leitura anterior. Pode se tratar de um documento (como um manuscrito ou um texto de lei), imagem (fotografia), ou qualquer outra fonte de informação resultado de trabalho próprio ou de outrem reconhecido pela comunidade científica (como diários de pesquisa).
\item \textbf{Fontes secundárias}: buscam apresentar uma análise sobre um material até então original. Ou seja, quem as utiliza alcança seu objeto de pesquisa a partir da avaliação de outra pessoa (como livros ou artigos científicos), mesmo que esta possa ter se equivocado na coleta dos dados ou na análise das informações.
\item \textbf{Fontes terciárias}: podem ser consideradas a combinação das duas formas anteriores. Compilados que buscam confrontar as fontes primárias com as secundárias, facilitando a pesquisa de interessados (como nas coletâneas). No entanto, estes terão de lidar com os limites do trabalho do responsável pela compilação.
\end{itemize}

A utilização desse tipo de fonte variará de acordo com as possibilidades de coleta de informações por parte do sujeito pesquisador. Você já deve ter percebido pelas descrições acima, e como também afirma Umberto Eco (2008), que as fontes primárias devem ser sempre preferidas. Muito embora sejam difíceis de ser apuradas, dependendo do local de onde se realiza a pesquisa, do seu momento histórico e do acesso à tecnologia necessária para sua interpretação, elas possibilitam um contato mais fiel ao objeto investigado

A necessidade de fidelidade ao objeto está diretamente conectada ao fazer científico. Dados inconsistentes com a realidade afastam o pesquisador de seu objetivo, a lembrar, botar a prova uma hipótese calcada, a princípio, no senso-comum cotidiano. Gastar tempo e energia para realizar uma atividade metódica que chegará a um resultado que poderia ser obtido por meio de raciocínios rápidos de nossa rotina é um desperdício que não pode ser permitido.

Portanto, faz parte da pesquisa apurar o grau de confiabilidade de suas fontes. Caso você e sua equipe não sejam capazes de realizar o próprio processo de coleta de dados, existem outras fontes de dados oficiais que podem atender às suas expectativas.

A dica é sempre procurar instituições de pesquisa reconhecidas. Estas costumam estar vinculadas às universidades públicas (no caso brasileiro) e ao Estado, nas esferas municipal, estadual e federal. Sobre o tratamento de dados socioeconômicos, temos o  \href{https://www.ibge.gov.br/}{IBGE}. Já para dados relacionados à pandemia, o Ministério da Saúde elaborou um \href{https://covid.saude.gov.br/}{Painel Coronavírus}. Sua confiabilidade está ligada a dedicação exclusiva de seus trabalhadores nas atividades de pesquisa científica, no seu histórico de análises ao longo dos anos e na importância que os dados produzidos por ele possuem para o encaminhamento consequente das políticas públicas estatais.

Se não for possível averiguar a confiabilidade da fonte, é possível proceder a pesquisa, desde que na elaboração de seu produto final isto seja publicizado. Ser honesto sobre a característica das fontes utilizadas, bem como sobre o procedimento de análise delas, permitirá a comunidade escolar e científica que se interesse por sua pesquisa colocar a prova seus resultados replicando o seu procedimento.

\know{Senso-comum e "fake news"}

Como vimos anteriormente, o conhecimento científico é importante para colocar à prova nossas avaliações mais imediatas sobre a realidade. Podemos reconhecer que este conhecimento cotidiano, popularmente conhecido como senso-comum, ao ser exercitado em contraponto ao conhecimento científico, pode se tornar mais amparado neste último.

Não é exatamente o que se verifica desde há muito tempo, como já apontou o historiador irlandês John Desmond Bernal (1975-78). Poucas vezes o senso comum esteve alinhado com a ciência. Atividade realizada por grupos sociais específicos, geralmente as classes dominantes de uma determinada formação social, a ciência só se tornou mais próxima do cotidiano da maior parte da população com o espalhamento e a predominância da atividade industrial a nível mundial. E não necessariamente na atividade de pesquisa, mas no consumo de produtos desenvolvidos a partir de experiências laboratoriais.

Este afastamento entre ciência e senso-comum permitiu a fertilização do terreno de posturas e opiniões sobre a realidade baseadas em superstição e demagogia. Atualmente, a disseminação de ideias que não encontram respaldo nas pesquisas científicas ou na verificação apurada dos fatos leva o nome genérico de \textit{fake news}.

É bem verdade que o fenômeno das \textit{fake news} envolve relações que estão além da relação entre ciência e senso-comum, como afirma Michiko Kakutani (2018). Existe certo consenso sobre o quanto a atividade de disseminação de informações falsas serve a interesses políticos e também à manutenção de atividades econômicas diversas, a despeito do seu impacto negativo.

\know{adulteração de resultados e plágio}

Todo começo é difícil, em qualquer ciência. A necessidade de cumprir prazos e apresentar resultados parece dificultar ainda mais. Nada que a constante prática não seja capaz de superar. No entanto, muitas vezes, dadas as exigências de prestação de contas e de uma cultura de supervalorização do sucesso, pesquisadores se sintam tentados a adulterar os resultados para atender determinados interesses. Seja os daqueles que financiam suas atividades, seja os seus próprios, tentando fazer valer sua hipótese a despeito daquilo que a própria realidade apresenta, seja pela desqualificação em sua formação.

Como lembra Umberto Eco, a despeito das justificativas, a adulteração dos resultados bem como a prática do plágio constituem formas completamente avessas as potencialidades científicas, utilizando de seus espaços privilegiados para proveito pessoal, ao invés de constituírem um conhecimento objetivo sobre a realidade.

\begin{reflection}
\textbf{A objetividade do conhecimento científico}

O avanço do fenômeno das \textit{fake news} e a existência das adulterações e plágios na pesquisa científica parece homogeneizar as maneiras como nós temos de produzir conhecimento. Tudo parece permitido e, em certa medida, válido para compreender a realidade. Será mesmo?

O termo objetividade é utilizado para apontar aquilo que um objeto de estudo é, independente da vontade ou do desejo da pessoa sujeito da pesquisa. Muitas vezes, quando falamos em objetividade na ciência, estamos falando no exercício de afastarmos nossos pré-julgamentos da atividade da pesquisa, tentando garantir melhores condições para botarmos à prova nossas hipóteses, tal como formulado pelo filósofo positivista Auguste Comte em seu Curso de filosofia positiva

Mas, desde o início de nossas atividades nesta seção, priorizamos o exercício de estranharmos nossa realidade, partindo daquilo que há de mais pessoal em nossa rotina. Como podemos ser objetivos considerando todos os valores e conhecimentos do senso-comum vindos de nosso cotidiano?

É importante revermos a noção de objetividade como sinônimo de afastamento de nossos pressupostos. Como apontamos desde o início, a atividade da pesquisa precisa, para cumprir sua função de atividade metódica, confrontar o senso-comum, e não afastá-lo. Resumindo em uma palavra, ela precisa ser crítica.

Em todo o momento da pesquisa, desde a investigação até a sua apresentação, é importante informar quais foram os caminhos tratados, quais os interesses que justificam essa pesquisa e como o resultado final corresponde ou não a hipótese apresentada.

A objetividade do conhecimento científico, como lembra Mario Bunge, deve ser tratada como uma exercício amparado de avaliação coletiva, deveria primar não pela ideia de afastamento dos valores do pesquisador, mas pela exposição o mais honesta possível de seus fundamentos.

\end{reflection}

\begin{example}{Planejamento da investigação - COVID-19}

Para a questão \titem{4)} do roteiro de organização da investigação, Rodrigo, Priscila e Thiago elaboraram o seguinte registro:

\begin{enumerate}[label=\titem{\arabic*)}]\setcounter{enumi}{3}
\item Quais são as fontes de dados confiáveis para os dados escolhidos por vocês?

\titem{Resposta}: \textit{Vamos utilizar os dados do IBGE sobre o Brasil e faremos uma pesquisa com os familiares dos estudantes de nossa turma. }
\end{enumerate}
\end{example}

\explore{Gráficos e tabelas}

As tabelas e os gráficos são formas visuais de organizar informações quantitativas. Estes instrumentos são muito úteis para resumir informações e apresentá-las para uma leitura rápida. Entretanto, cada um desses instrumentos têm usos diferentes.

Por exemplo, as \textbf{tabelas} têm como principal função resumir uma grande quantidade de dados, de forma organizada.

Observe a imagem abaixo, de uma tabela do portal de dados \url{Brasil.io}. Ela apresenta oito tipos diferentes de informações, cada um para nove cidades brasileiras. Ou seja, esta tabela apresenta de forma compacta 72 dados.

\begin{table}[H]
\centering
\setlength\tabcolsep{2.5pt}
\setlength\tabulinesep{3pt}
\begin{tabu} to \textwidth{|l|l|c|r|r|r|r|r|}
\hline
\multicolumn{8}{|c|}{\cellcolor{\currentcolor!80}\textcolor{white}{\textbf{COVID-19 - Dados por Município}}} \\
\hline
\thead
Data & Município & UF & Confirmados & \makecell{Confirmados\\por 100k hab.} & Óbitos & Letalidade & \makecell{Óbitos por\\ 100k hab.}\\
\hline
30/05/2020 & São Paulo & SP & 58 619 & 478,44 & 4 239 & 7,23\% & 34,60 \\
\hline
29/05/2020 & Rio de Janeiro & RJ & 27 311 & 406, 28 & 3 430 & 12,56\% & 51,05 \\
\hline
30/05/2020 & Fortaleza & CE & 23 378 & 875,80 & 1 939 & 8,29\% & 72,64 \\
\hline
29/05/2020 & Manaus & AM & 17 492 & 801,37 & 1 349 & 7,71\% & 61,80 \\
\hline
30/05/2020 & Recife & PE & 14 832 & 901,24 & 1 040 & 7,01\% & 63,19 \\
\hline
29/05/2020 & Belém & PA & 11 509 & 771,00 & 1 311 & 11,39\% & 87,82 \\
\hline
29/05/2020 & Salvador & BA & 10 353 & 360,44 & 421 & 4,07\% & 14,66 \\
\hline
29/05/2020 & São Luís & MA & 8 882 & 806,07 & 521 & 5,87\% & 47,28 \\
\hline
29/05/2020 & Brasília & DF & 7 877 & 261,24 & 142 & 1,80\% & 4,71 \\
\hline
\end{tabu}

\caption*{Adaptado de \url{https://brasil.io/covid19/} - Acesso em 31/05/2020}
\end{table}

Um outro exemplo são os dados de disponibilidade de leitos de UTI e leitos clínicos apresentados nos boletins diários publicados pelo estado do Maranhão durante os primeiros meses de pandemia do novo coronavírus no Brasil.

Neste caso, a tabela tem a função de rapidamente passar informações importantes, como a taxa de ocupação de leitos na Grande Ilha, que inclui as cidades de São Luís e São José de Ribamar.

\begin{table}[H]
\centering
\begin{tabu} to \textwidth{|c|r|c|r|}
\hline
\multicolumn{4}{|c|}{\cellcolor{\currentcolor!80}{\textcolor{white}{\textbf{Leitos - Grande Ilha}}}} \\
\hline
\multicolumn{2}{|c}{\cellcolor{\currentcolor!80}{\textcolor{white}{\textbf{\makecell{Taxa de Ocupação de Leitos de UTI \\ Exclusivos Covid-19*}}}}} & \multicolumn{2}{|c|}{\cellcolor{\currentcolor!80}{\textcolor{white}{\textbf{\makecell{Taxa de Ocupação de Leitos de Clínicos \\ Exclusivos Covid-19*}}}}} \\
\hline
Total de Leitos & 240 & Total de Leitos & 752 \\
\hline
Leitos Ocupados & 230 & Leitos Ocupados & 304 \\
\hline
Leitos Livres & 10 & Leitos Ocupados & 448 \\
\hline
Taxa de ocupação & 95,83\% & Taxa de Ocupação & 40,43\% \\
\hline
\end{tabu}
\caption*{
\footnotesize
*Taxa de ocupação relativa aos leitos SUS disponíveis na rede SES/ME

Dados publicados pela Secretaria Estadual de Saúde do Maranhão

Boletim epidemiológico de 30/05/2020}
\end{table}
\needspace{5em}
Os \textbf{gráficos de linha} são utilizados para mostrar evolução. Ele é o instrumento mais adequaod para, por exemplo, mostrar a evolução do total de casos de COVID-19 no Brasil em função do empo, como no gráfico da imagem a seguir, publicado pelo Ministério da Saúde em 31/05/2020.

\begin{figure}[H]
\centering
\includegraphics[width=400bp]{investigacao3}

\end{figure}

Ou ainda, para mostrar como evoluiu o índice de isolamento no Estado de São Paulo entre os meses de março e maio de 2020.

\begin{figure}[H]
\centering
\includegraphics[width=400bp]{investigacao4}

\caption*{Fonte: Secretaria de Saúde do Estado de São Paulo. Acesso em 31/05/2020}
\end{figure}

Já os \textbf{gráficos de barras e colunas} são mais adequados para mostrar comparação entre quantidades. Um bom exemplo é o gráfico apresentado pelo Portal de Transparência de Registro Civil, que apresenta uma comparação da quantidade de óbitos com suspeita ou confirmação de COVID-19 por sexo e faixa etária.

\begin{figure}[H]
\centering
\includegraphics[width=400bp]{investigacao5}

\caption{Fonte: Central de Informações do Registro Civil - CRC Nacional}
\end{figure}

Os \textbf{gráficos de setores} dão noção de parte e todo, e por isso muitas vezes seus valores são apresentados na forma percentual. Esta foi a forma como o Estado de Santa Catarina escolheu para mostrar em seus boletins epidemiológicos a taxa de ocupação dos leitos de UTI. Repare como com este tipo de gráfico fica fácil entender se, comparado com o total de leitos, há muitos ou poucos livres.

\begin{figure}[H]
\centering
\includegraphics[width=300bp]{investigacao6}

\end{figure}


Existem muitas outras formas de mostrar dados quantitativos de formas visuais. Se você quiser se aprofundar um pouco mais neste assunto pode buscar também sobre as discussões a leitura e o uso de \textbf{infográficos} e \textbf{histogramas}, no capítulo introdutório do Livro de Estatística e Probabilidade do Livro Aberto, ou ainda uma discussão sobre a \textbf{representação gráfica de funções}, no capítulo introdutório do Livro de Funções do Livro Aberto. 

Com os programas de edição de imagem, hoje em dia até existe um profissional que é especializado em elaborar gráficos, tabelas e diagramas que ajudem os leitores a entenderem melhor as informações de uma certa notícia ou relatório. Para elaborar um bom gráfico é necessário conhecimento matemático, compreensão de qual informação quer-se enfatizar e muita criatividade!

\begin{example}{Planejamento da investigação - COVID-19}

Para as questões \titem{5} e \titem{6} do roteiro de organização da investigação, Rodrigo, Priscila e Thiago elaboraram os seguintes registros:

\begin{enumerate}[label=\titem{\arabic*)}]\setcounter{enumi}{4}
\item Expliquem como vocês utilizarão os dados para responder a pergunta investigativa do grupo. Dêem exemplos. Que cálculos vocês farão com esses dados e o que esperam obter? Não se esqueçam que ao final, a análise deve servir para testar a(s) sua hipótese(s) inicial(is).

\titem{Resposta}:\textit{Calcularemos as taxas de desocupação e iremos verificar a sua evolução ao longo do tempo. Por exemplo, no trimestre fev/mar/abr de 2019 a taxa era $12,5\%$, enquanto que em fev/mar/abr de 2020 era de $12,6\%$.  Assim, verificamos que a taxa de desocupação, nesta situação, não teve muita alteração. No entanto, nos questionamos sobre esses percentuais.}

\item Façam um esboço do(s) tipo(s) de gráfico(s) que vocês utilizarão para apresentar seus dados. Abusem das cores e criatividade em seu esboço. Não se esqueçam de identificar quais dados vocês esperam apresentar em cada gráfico que fizerem.

\begin{figure}[H]
\centering

\includegraphics[width=\textwidth]{investigacao10}
\end{figure}

\end{enumerate}

\end{example}


\practice{Desenvolvendo a investigação - etapa 4}
\phantomsection\label{etapa4}

Agora é hora de colocar em prática o seu planejamento!

Mas antes de iniciar o trabalho pense sobre como organizá-lo. Onde os dados dos índices ficarão guardados? Onde você fará suas contas e seus gráficos? Vai utilizar um caderno ou uma planilha? Onde anotará novas fontes bibliográficas que for encontrado pelo caminho, seus resultados parciais e conclusões? Em trabalhos em grupo é necessário também pensar em estratégias para que todos tenha acesso a todos os dados, contas, resultados e conclusões. 

Uma boa estratégia é manter um diário de bordo no qual você escreve um pequeno parágrafo contando o que fez na aula, os avanços obtidos e os próximos passos. Este diário pode ser feito num caderno próprio ou ser um relato coletivo. 

As questões a seguir são uma síntese do que deve ser o seu resultado final. Você só conseguirá respondê-las quando finalizar sua pesquisa.

Mão à obra!

\begin{enumerate}
\item Como os indicadores, gráficos e tabelas que você construiu respondem a sua pergunta de investigação?
\item Você considera que os indicadores que você e seu grupo escolheram usar foram realmente adequados? Eles ajudaram a responder a pergunta investigativa de vocês? Justifique sua resposta.
\item Por último, escreva suas conclusões. É o resumo do que você aprendeu e do que se pode saber através desta investigação. Pode ser apenas uma frase e algumas vezes pode haver mais de uma conclusão. É muito importante que na conclusão você analise a sua hipótese inicial. Ela estava correta? Se não estava, o que os dados mostraram de diferente das suas expectativas iniciais?
\end{enumerate}

\begin{example}{Realizando a investigação - COVID-19}

Rodrigo, Priscila e Thiago acessaram o site do IBGE e buscaram os dados da PNAD Contínua\footnote{\url{https://www.ibge.gov.br/estatisticas/sociais/trabalho/9173-pesquisa-nacional-por-amostra-de-domicilios-continua-trimestral.html}}. Durante a pesquisa, perceberam que os relatórios apresentavam dados brutos e que poderiam, eles próprios, realizar os cálculos e fazer interpretações dos indicadores. Eles consultaram diversas tabelas no site do IBGE e chegaram na tabela 4092 do sistema SIDRA do IBGE\footnote{\url{https://sidra.ibge.gov.br/tabela/4092}}, onde fizeram alguns filtros e geraram a tabela a seguir.

\begin{figure}[H]
\centering

\includegraphics[width=\textwidth]{investigacao11}
\end{figure}

Em seguida deu-se a uma discussão sobre o que encontram
\begin{quote}
\textbf{Rodrigo}: Pessoal, reparei que o número de pessoas desocupadas no segundo trimestre de 2019 e no segundo trimestre de 2020 foi quase o mesmo.

\textbf{Thiago}: É verdade. Em compensação, o número de pessoas ocupadas diminuiu cerca de 10 mil pessoas.

\textbf{Priscila}: Você quis dizer 10 milhões de pessoas, certo, Thiago?

\textbf{Thiago}: Não. Eu quis dizer 10 mil mesmo. Na tabela tá dizendo que a força de trabalho ocupada era cerca de 93 mil pessoas e passou para 83 mil.

\textbf{Priscila}: Thiago, você esqueceu de ver que os números da tabela são em “milhares de pessoas”, como tá escrito na legenda. Quando a tabela coloca 93.342, ela não tá dizendo “93 mil pessoas”, mas sim “93 mil milhares de pessoas”. Quer dizer 93.342$\times$1000, ou seja, 93.342.000, que é cerca de 93 milhões de pessoas.

\textbf{Thiago}: Que confusão, Priscila! Mas eu entendi, você tem razão. E 93 mil pessoas realmente parece pouco mesmo, pois só na cidade onde moramos, que não é grande, tem cerca de 100 mil habitantes.

\textbf{Rodrigo}: Pessoal, voltando aos indicadores, acho que se queremos um indicador de pessoas ocupadas, não devemos apenas calcular o percentual de pessoas ocupadas dentro do total da força de trabalho. Acho que precisamos calcular com relação ao total de pessoas em idade de trabalhar. Veja que o número de pessoas em idade de trabalhar aumentou de 171 milhões para 174 milhões, mas o número de pessoas na força de trabalho diminuiu de 106 milhões de pessoas para 96 milhões.

\textbf{Priscila}: Tem razão. A impressão é que mais de 10 milhões de pessoas deixaram de trabalhar e passaram a integrar o grupo de pessoas “fora da força de trabalho”{} e não o grupo de “pessoas desocupadas”.

\textbf{Thiago}: Mas qual a diferença entre uma pessoa desocupada e uma pessoa fora da força de trabalho? Se ela está fora de força de trabalho, então ela não está desocupada?

\textbf{Priscila}: De certa forma, sim. Mas pelo que consultei no glossário do IBGE\footnote{\url{ftp://ftp.ibge.gov.br/Trabalho_e_Rendimento/Pesquisa_Nacional_por_Amostra_de_Domicilios_continua/Mensal/glossario_pnadc_mensal.pdf}}, eles usam o termo “força de trabalho” para se referir ao conjunto de pessoas que estão dispostas a trabalhar.

\textbf{Rodrigo}: Isso mesmo, e “desocupadas” são as pessoas na força de trabalho, isto é, que querem trabalhar, mas que não estão trabalhando por algum motivo, que pode ser por ter buscado mas não ter conseguido um trabalho, por falta de perspectiva em conseguir um emprego, ou por impossibilidade de assumir um emprego no momento.

\textbf{Priscila}: Isso mesmo. Pelo que parece muita gente deixou de trabalhar neste momento de pandemia.

\textbf{Thiago}: O que era de se esperar, né Priscila? Como trabalhar se precisamos ficar em casa para não propagar o coronavírus?

\textbf{Priscila}: Mas como ficar em casa se precisamos trabalhar para pagar nossas contas?

\textbf{Thiago}: Ta aí um grande dilema! Em todo caso,  o que parece é que muita gente não apenas parou de trabalhar, como também passou a ficar indisponível para trabalhar. Ou seja, essas pessoas deixaram a categoria de “ocupadas” e passaram para a categoria “fora da força de trabalho”.

\textbf{Rodrigo}: Ótima colocação, Thiago. Por isso penso que se queremos saber indicadores de emprego, não devemos apenas dividir o número de pessoas ocupadas pelo número de pessoas na força de trabalho, mas sim pelo número total de pessoas em idade de trabalhar.

\textbf{Priscila}: Pelo que pesquisei, Rodrigo, o IBGE também faz esse cálculo. Eles chamam de “nível de ocupação”.

\textbf{Thiago}: Nossa, quantas contas! Então a gente vai calcular “taxa de ocupação”, que é o percentual de pessoas ocupadas dentro do total da força de trabalho, e também o “nível de ocupação”, que é o percentual de pessoas ocupadas dentro do total de pessoas em idade para trabalhar.

\textbf{Rodrigo}: Sim, Thiago. Acho que é isso. Neste caso, as taxas se referem às pessoas na força de trabalho, isto é, que estão dispostas a trabalhar. Já o nível se refere ao total de pessoas em idade de trabalhar.
\end{quote}

Com essa decisão, o grupo elaborou a tabela a seguir com esses indicadores:

\begin{table}[H]
\centering

\begin{tabu} to \textwidth{|l|c|c|}
\hhline{~|--|}
\multicolumn{1}{c|}{} & \tcolor{\makecell{2$^{\circ}$ trimestre \\ 2019}} & \tcolor{\makecell{2$^{\circ}$ trimestre \\ 2020}} \\
\hline
Total (pessoas com 14 anos ou mais de idade) & 170.864* & 173.918* \\
\hline
(1) Taxa de participação na Força de Trabalho & $62{,}1\%$ & $55,\%$ \\
\hline
(2) Taxa de ocupação & $88{,}0\%$ & $86{,}7\%$ \\
\hline
(3) Taxa de desocupação & $12{,}\%$ & $13{,}3\%$ \\
\hline
Nível de ocupação [(1)$\times$(2)] & $54{,}6\%$ & $47{,}9\%$ \\
\hline
\end{tabu}
\flushright

\scriptsize{* em milhares de pessoas}
\end{table}

Além disso, com base nesses estudos, o grupo elaborou um questionário (instrumento de coleta) para saber a situação do emprego dentre os familiares dos colegas de classe. Para expressar as condições de ocupação, o grupo decidiu por três categorias: (a) pessoas ocupadas; (b) pessoas desocupadas mas com intenções de trabalhar e; (c) pessoas desocupadas sem intenções de trabalhar. As pessoas com idade igual ou superior a 14 anos integrantes nas categorias (a) e (b) formam o grupo “força de trabalho”, e as da categoria (c) estão “fora da força de trabalho”

\begin{center}
\framebox[.8\textwidth][c]
{\parbox{.75\textwidth}{
	\vspace{1em}
	\centering 
	\textbf{\large Questionário sobre trabalho e condição de ocupação}
	\justify

	\begin{enumerate}[label=\textbf{\arabic*. }, itemsep=0cm]
	\item Nome:
	\item Sexo:
	\item Idade:
	\item Em qual situação você se encontrava no 2$^{\circ}$ trimestre de 2019?
	\begin{enumerate}[label=(\alph*)]
	\item Estava ocupado
	\item Estava desocupado, mas tinha a intenção de trabalhar.
	\item Estava desocupado e nem tinha a intenção de trabalhar.
	\end{enumerate}
	\item Em qual situação você se encontrava no 2$^{\circ}$ trimestre de 2020?
	\begin{enumerate}
	\item Estava ocupado
	\item Estava desocupado, mas tinha a intenção de trabalhar.
	\item Estava desocupado e nem tinha a intenção de trabalhar.
	\end{enumerate}
	\end{enumerate}
	\flushright

	\textbf{Elaboradores}: Priscila, Rodrigo e Thiago
	\vspace{.5em}
	}

}
\end{center}
\end{example}
\clearpage
\practice{Comunicando as descobertas - etapa 5}
\phantomsection\label{etapa5}

Agora é hora de materializar todo o aprendizado do grupo em um produto final. As possibilidades são ilimitadas e dependem da criatividade e de recursos (como tempo para execução, acesso a computadores e internet, disponibilidade de material papel e tinta, etc). 

Algumas possibilidades são:

\begin{itemize}
\item um revista "científica"{}com os artigos dos grupos;
\item um congresso com um seminário de cada "grupo de pesquisa";
\item uma série de podcasts no qual cada grupo faz o seu programa;
\item um telejornal ou um canal com vídeos de cada grupo;
\item uma página de internet para cada grupo apresentar suas descobertas sobre o tema estudado;
\item um fazine ou uma história em quadrinho (HQ);
\item um panfleto ou um folder;
\end{itemize}

\begin{task}{refletindo sobre o produto final}

Vamos pensar um pouco sobre o que é important que haja nesse produto final. Realize em uma ficha ou em seu caderno os registros para as questões a seguir.

\begin{itemize}
\item Pesquise e discuta com colegas quais as principais características do tipo de produto final escolhido para ser realizado como finalização deste projeto de investigação. Registre em uma ficha ou em seu caderno os pontos de atenção para produzir um bom produto final.
\item Quem é o público alvo de seu produto final?
\item Quais são as informações mais importantes de sua investigação. Isto é, o que seu público alvo precisa aprender com seu produto final?
\end{itemize}

Agora é hora de efetivamente dar vida ao produto final. Após ele ficar pronto e ser compartilhado com seus colegas. Volte para responder às últimas pergutnas desta Unidade Temática.

\begin{itemize}
\item Como foi ver o seu produto final concretizado? O que você sentiu? O que acha que ficou bom e o que poderia ser melhorado?
\item O que você sente que aprendeu durante todo esse processo de investigação?
\item O que você achou de toda essa experiência?
\end{itemize}

\end{task}