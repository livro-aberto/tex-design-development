\chapterillustration{./abertura-documentacao}{./abertura-documentacao-professor}

\chapterwhat{O Livro Aberto de Matemática é um projeto de construção de um material didático de matemática de distribuição livre e \textit{Open Source}, isto é, o código é aberto e pode receber submissões de qualquer um que deseje colaborar}

\chapterbecause{O Esforço para produzir livros didáticos de Matemática com licença aberta começou em 2016, com a elaboração do material de Frações para o ensino fundamental I. Desde então, novos elaboradores acreditaram e juntaram-se ao projeto para alcançarmos novos níveis e novos livros. Hoje, possuímos diversos capítulos escritos e outros 6 livros em produção. Tudo isso a partir de um trabalho colaborativo envolvendo matemáticos, professores universitários e professores da Educação Básica.

Um princípio fundamental desta iniciativa é que sua produção configure uma proposta pedagógica ancorada e acompanhada por pesquisa científica em Ensino de Matemática. O projeto tem também compromisso com a formação e o desenvolvimento profissional de professores. Em particular, pela composição característica da equipe, destaca-se o entendimento do potencial do projeto para enfrentar um reconhecido desafio: estreitar o diálogo entre a realidade e as demandas próprias da prática docente e a formação acadêmica do professor. Leia mais a respeito. Você também pode sugerir alterações no texto do projeto pelo \href{https://github.com/livro-aberto/tex-design-development}{Github}.}


\chapter{Documentação do Livro Aberto do Ensino Médio}

\autorum{Tarso Boudet Caldas}

\ilustracao{Miller  Guglielmo}

\versao{0.5}

\autordacapa{Bernd Klutsch}{Unplash}{https://unsplash.com/photos/nE2HV5AUXFo}


\mainmatter


\begin{apresentacao}{Introdução à Documentação do Material do Professor}
A versão do professor atual possui algumas diferenças em relação ao material do aluno, mas utilizando o mesmo arquivo do capítulo em \verb|.tex|. Isto quer dizer que para gerar o material, o texto do capítulo do aluno e do professor convivem no mesmo arquivo, apenas com a diferença de que se compila o arquivo \verb|professor.tex| em vez do \verb|aluno.tex|. De resto, as recomendação sobre como compilar esse texto \textbf{as mesmas} do \textcolor{session1}{Explorando: O que é necessário?} no início desta documentação. Com as seguintes diferenças:
\begin{itemize}
\item Ao adicionar a figura de abertura do capítulo, deve haver duas versões, uma para a do aluno e outra para a do professor. Como o tamanho da página do arquivo do aluno é diferente da do professor, as imagens de abertura possuem tamanhos diferentes. A abertura do aluno está no formato proporcional a \verb|2000 x 2924| pixels e do professor a \verb|2000 x 1646| pixels. É importante observar que as imagens não precisam estar nestes números, apenas deve ter o tamanho proporcional, caso contrário a imagem da capa ficará distorcida.

\item Há novos ambientes e comandos para os textos do professor, que serão melhor explicados adiantes.
\end{itemize}

Para gerar uma página como esta de apresentação, usamos o ambiente
\begin{verbatim}
\begin{apresentacao}{titulo}

\end{apresentacao}
\end{verbatim}
com o título de apresentação de sua escolha.  

\subsection{Limitações}
Por ser uma versão beta do modelo, alguns problemas podem surgir no uso. Este é feito com base em margens, isto é, o material do aluno é o texto principal e o material do professor as margens deste material. Por este motivo, há a dificuldade de alinhamento das caixas do professor nas páginas, por isso, algumas "gambiarras"{} devem ser feitas para que o alinhamento ocorra de modo desejado. Estas serão explicadas no máximo de suas possibilidades, mas alguns problemas podem surgir. 

Ao encontrar um erro, pedimos que este seja reportado na seção de \href{https://github.com/livro-aberto/tex-design-development/issues}{"issues"{} no GitHub do projeto}, para que estes possam ser visto pelo maior número de pessoas, ajudando aqueles que encontrarem um erro semelhante, e também para que os colaboradores que souberem como solucionar o problema tenham melhor acesso.

A seguir iremos descrever os ambientes novos criados e seguiremos a exemplificar nas próximas páginas.

\subsection{Habilidades}
Temos o ambiente \verb|habilities|, como no exemplo a seguir:

\begin{habilities}{EM13MAT305}
Resolver e elaborar problemas com funções logarítmicas nos quais seja necessário compreender e interpretar a variação das grandezas envolvidas, em contextos como os de abalos sísmicos, pH, radioatividade, Matemática Financeira, entre outros.
\tcbsubtitle{EM13MAT403}
Analisar e estabelecer relações, com ou sem apoio de tecnologias digitais, entre as representações de funções exponencial e logarítmica expressas em tabelas e em plano cartesiano, para identificar as características fundamentais (domínio, imagem, crescimento) de cada função.
\end{habilities}

Estas são as habilidades, do capítulo de Logaritmos. Este ambiente foi gerado pelo seguinte código:
\begin{verbatim}
\begin{habilities}{EM13MAT305}
texto da habilidade

\tcbsubtitle{EM13MAT403}
texto da habilidade
\end{habilities}
\end{verbatim}

A primeira habilidade deve estar entre chaves logo depois do início do ambiente. Se houver outras habilidades a serem colocadas no mesmo ambiente, o título deve estar no comando \verb|\tcbsubtitle{habilidade}|. Não há limite para o número de habilidades.

\subsection{Caixas do para o professor}

O conteúdo para a descrição dos objetivos, sugestões e soluções de atividades e afins deve ser colocado nas seguintes caixas

\paragraph{Objetivos Específicos}
Para descrever os objetivos específicos de uma atividade, utilizamos o ambiente \verb|objectives|, com os seguintes argumentos:
\begin{verbatim}
\begin{objectives}{título}
{
conteúdo
}{lado da página}{notas}
\end{objectives}
\end{verbatim}
Além do título e do conteúdo, o ambiente requer dois números, um para o lado da página, isto é, se é uma página par ou ímpar. No caso de página ímpar, usamos o número \verb|1| e para páginas ímpar o número \verb|2|. O segundo número indica se a caixa será colocada na página (\verb|1|) do conteúdo ou na seção das notas no final do capítulo (\verb|0|) (exemplificaremos mais à frente). Recomenda-se não colocar os objetivos específicos nas notas no final do capítulo, a menos que necessário.

\paragraph{Sugestões e discussões}
Nesta caixa escrevemos informações úteis à execução de uma atividade ou para chamar a atenção a algum conteúdo da página. O funcionamento é o mesmo da caixa de objetivos específicos, mas com o nome \verb|sugestions|:
\begin{verbatim}
\begin{sugestions}{título}
{
conteúdo
}{lado da página}{notas}
\end{sugestions}
\end{verbatim}
Neste caso, as notas têm três possibilidades: \verb|1|, \verb|0| ou \verb|9|. A diferença entre a caixa de objetivos específicos e de sugestões é a opção de quebra de página da caixa ao final do capítulo. Traremos exemplos mais tarde.

\paragraph{Respostas}
\begin{verbatim}
\begin{sugestions}{título}
{
conteúdo
}{notas}
\end{sugestions}
\end{verbatim}
O ambiente de respostas não possui diferenciação no lado da página, mas possui o mesmo argumento sobre as notas no final do capítulo que o ambiente de sugestões.

\paragraph{Texto fora da caixa}
Para escrever texto nas margens da página, que não as caixas citadas acima, pode-se utilizar o ambiente \verb|texto|:
\begin{verbatim}
\begin{texto}
{
conteúdo
}
\end{texto}
\end{verbatim}

\subsection{Posicionamento das caixas no código}

Por conflitos com as outras caixas do material do aluno, o conteúdo das margens tem sido colocado \textbf{antes} das seções Explorando e afins. A quebra de página das margens deve ser feita com o comando \verb|\clearmargin|, que tem a mesma funcionalidade de \verb|\clearpage|, mas apenas para a margem.

Por exemplo:
\begin{verbatim}
\def\currentcolor{session1}
\begin{objectives}{atividade 1}
{

}{1}{1}
\end{objectives}
\begin{sugestions}{atividade 1}
{

}{1}{1}
\end{sugestions}
\clearmargin
\begin{answer}{atividade 1}
{

}{1}
\end{answer}
\explore{título}
\end{verbatim}

Neste caso, teríamos a definição da cor da caixa usando \verb|\def\currentcolor{cor}|. Suas opções são 
\begin{itemize}
\item \verb|session1|, a cor do \textcolor{session1}{Explorando} 
\item \verb|session2|, a cor do \textcolor{session2}{Praticando}
\item \verb|session3|, a cor do \textcolor{session3}{Para Saber+}
\item \verb|session4|, a cor do \textcolor{session4}{Organizando} e
\item \verb|cor1|, a cor dos \textcolor{cor1}{Exercícios}
\item \verb|cor1|, a cor do \textcolor{cor2}{Projeto aplicado}
\end{itemize}


\subsection{Considerações}


\paragraph{Adjustbox}
As opções das notas no final do capítulo dizem respeito à quebra ou não da \verb|tcolorbox|. Quando permitido que as caixas quebrem, há a possibilidade da caixa ir para a próxima página enquando o marcador da nota fique na anterior. Por este motivo é usado o pacote e o comando \verb|adjustbox|. Com ele é possível alinhar a caixa ao marcador da nota. Entretanto, quando é usado este comando a caixa perde a possibilidade de quebrar, por isso é necessário atenção ao utilizar os comandos. 

Nos casos das três caixas de objetivos, sugestões e respostas, o último número a ser colocado é \verb|1| se é desejado que a caixa apareça na própria página; \verb|0| se é desejado que vá para as notas (usando \verb|adjustbox|) e \verb|9| quando se deseja que a caixa quebre (sem \verb|adjustbox|). Vamos mostrar casos das duas possibilidades.

\end{apresentacao}

\def\currentcolor{session1}
\begin{objectives}{Título da atividade}
{Esta caixa foi criada usando o comando 
\verb"
\begin{objectives}{Título da atividade}
{
Conteúdo
}{1}{1}
\end{objectives}"


Como no terceiro argumento da caixa usamos ``\verb|1|'', ela ficará na página no meio do texto. Se usarmos ``\verb|0|'', ela irá então para o final do capítulo, criando uma ``Nota'', como você pode observar abaixo desta caixa.
}{1}{1}
\end{objectives}


\begin{objectives}{Título da atividade}
{
Esta caixa foi criada usando o comando 
% \begin{verbatim}
% \begin{objectives}{Título da atividade}
% {
% Conteúdo
% }{0}{1}
% \end{objectives}
% \end{verbatim}

Valo notar que se a caixa está no final das notas, o quarto argumento perde função.
}{0}{1}
\end{objectives}
\begin{sugestions}{Título da atividade}
{

}{1}{1}
\end{sugestions}
\begin{answer}{Título da atividade}
{
\begin{enumerate}
\item 
\end{enumerate}
}{1}
\end{answer}


\explore{O que é necessário?}

Para a utilização deste pacote é necessário alguns arquivos auxiliares que estão incluídos na pasta \verb|tex-design-development| no \href{https://github.com/livro-aberto/tex-design-development/}{GitHub} do projeto. Os arquivos são:
\begin{itemize}
\item O diretório \verb|/chapters|, onde devem ser colocados os capítulos em produção (nele estão exemplos de outros capítulos já produzidos, mas estes arquivos não são necessários).
\item O diretório \verb|/Figuras|, onde devem ser colocadas as figuras dos capítulos (estão incluídas no GitHub uma pasta com as figuras de todos os capítulos produzidos até o momento e também uma pasta apenas com as figuras necessárias para a geração de um capítulo novo).
\item O diretório \verb|/Fontes| que possui todas as fontes usadas no livro (elas devem ser instaladas na sua máquina).
\item Os pacotes \verb|sphinx.sty|, \verb|livroaberto.sty| e \verb|livroaberto-professor.sty|.
\item O arquivo \verb|aluno.tex| para a compilação do material do aluno e o arquivo \verb|professor.tex| para compilar o material do professor.
\end{itemize}

Além disso, é necessário ter instalado \href{https://www.tug.org/texlive/}{\TeX{} Live} ou \href{https://miktex.org/}{Mik\TeX}{} e um editor de \LaTeX{} à sua escolha. Não pretendemos a ensinar o uso de \LaTeX, mas mostrar a estrutura e padronização utilizada nos capítulos já existentes do projeto. Para aqueles que possuem interesse em aprender indicamos \href{http://www.uft.edu.br/engambiental/prof/catalunha/arquivos/latex/latex_GilbertoSouto.pdf}{Curso de \LaTeX{}} de Gilberto Souto %provisório


\subsection{Início do capítulo}

Para começar o seu capítulo e poder compilá-lo, é necessário editar no arquivo \verb|aluno.tex| ou \verb|professor.tex|, dependendo de qual versão se deseje compilar, após o início do documento, incluindo o nome do seu arquivo pelo comando \verb|\include{chapters/seucapitulo}|(lembrando que o arquivo deve estar na pasta \verb|/chapters|). É necessário usar \verb|xelatex| para compilar propriamente.

No início do arquivo de seu capítulo é preciso colocar o que seria o "preâmbulo"{} do capítulo, isto é, antes do primeiro explorando é preciso colocar a abertura e os créditos (isto pode ser feito copiando de outro capítulo já pronto). A partir disso, é necessário:

\begin{itemize}
\item A ilustração da capa, que pode ser feito alterando o comando \spverb|\chapterillustration{./abertura-capitulo}{.abertura-capitulo-professor|. A capa deve estar na pasta \verb|/Figuras| e deve ter o nome no formato \spverb|abertura-nomedocapitulo|. A figura deve ter licença gratuita, e deve ser preferencialmente no formado de retrato. Figuras gratuitas podem ser encontradas no site \href{https://unsplash.com/}{Unsplash}.

\item O "O Quê?"{} pode ser colocado dentro comando \spverb|\chapterwhat{o que}|; o "Por Quê"{} em \spverb|\chapterbecause{por que}| e o nome do capítulo em \spverb|\chapter{capitulo}|

\item Alterar na página de créditos a partir dos seguintes comandos (:
\begin{enumerate}
\item \verb|\autorum|, \verb|autordois|, ..., \verb|autorcinco| para o nome dos autores (até cinco nomes).
\item \verb|\revisorum|, \verb|revisordois|, ..., \verb|revisorcinco| para o nome dos revisores (até cinco nomes).
\item \verb|\versao| a versão atual do material.
\item \verb|\graficos| para o nome do(a) autor(a) de figuras técnicas.
\item \verb|\ilustracao| para o nome do(a) ilustrador(a).
\item \verb|\autordacapa{Autor}{Fonte}{Link}| para os créditos da imagem de capa. Este comando coloca a frase ``Foto de Autor no Fonte'' na página de créditos, onde fonte é o site de retirada da imagem.
\end{enumerate}

Entre todos estes comandos, apenas o nome do primeiro autor e a versão são necessários para que não haja erro. Para então criar a página de créditos é necessário usar o comando \creditos após as definições acima.
\end{itemize}

Após isso, basta começar o Explorando após \verb|\mainmatter| usando \verb|\explore{nome da seção}|.

\begin{task}{Caixa de atividade}

A primeira parte de uma seção é o Explorando. Nela são colocadas atividades que introduzem o assunto que será discutido. Para criar uma caixa de atividade usamos
\begin{verbatim}
\begin{task}{Título da atividade}

Texto da atividade

\end{task}
\end{verbatim} 

O comando \verb|\label| também pode ser usado na caixa de atividade.
\end{task}


\begin{task}{Figuras}
Na sua atividade, você pode querer incluir figuras e tabelas. Temos um padrão que para essa inclusão.

\begin{figure}[H]
\centering

\includegraphics[width=250bp]{5_1.jpg}
\caption{Exemplo de figura}
\end{figure}

As figuras são em geral centralizadas e podem ou não possuir legenda. Por isso usamos o ambiente de figura dessa forma:

\begin{verbatim}
\begin{figure}[H]
\centering

\includegraphics[width=250bp]{suafigura.png}
\caption{Legenda}
\label{marcação para referência}
\end{figure}
\end{verbatim}

O ambiente de figura possui uma diferença crucial que é o uso da opção \verb|[H]|. Essa opção é fornecida pelo pacote \verb|float| (incluído no pacote \verb|livroaberto|) e é usada pois a figura não pode estar naturalmente dentro de \href{https://en.wikibooks.org/wiki/LaTeX/Floats,_Figures_and_Captions}{ambiente flutuante}, que é o caso da caixa de atividade, criada por uma \verb|tcolorbox|. Deixamos também a figura, em geral, centralizada.

Pedimos que ao se referir a uma atividade, figura ou tabela utilize \verb|\hyperref[label]{float \ref{label}}|, onde float é substituído por figura, atividade ou tabela, de acordo com a referência. 

\textbf{Atenção}: Os \textit{labels} devem sempre ficar após as \textit{captions} no código, senão o \LaTeX{} não as reconhece propriamente.
\end{task}

\begin{task}{Tabelas}

Tabelas requerem um pouco mais de atenção, pois possuem diversas opções e tipos de construções avançadas. Por isso, recomendamos a leitura do artigo sobre tabelas no \href{https://en.wikibooks.org/wiki/LaTeX/Tables}{Wikibooks}. 

Assim como nas figuras, deve-se usar o parâmetro \verb|H| para que ela funcione normalmente dentro da caixa de atividade. Na nossa padronização de tabela, buscamos um \textit{heading} que possua a mesma cor da parte da seção. Para conseguir isto temos três comandos:

\begin{itemize}
\item \verb|\tcolor{texto}|: Coloca a cor da célula na cor da seção atual e o texto branco em negrito.
\item \verb|\tmcol{#colunas}{alinhamento}{texto}|: Substitui o comando \verb|multicolumn| para ter as características que buscamos na célula.
\end{itemize}

\begin{table}[H]
\centering
\begin{tabular}{|c|c|c|c|}
\hline
\tcolor{Atividade Física} & \tcolor{Manhã} & \tcolor{Tarde} & \tcolor{Total} \\
\hline
\tcolor{Não pratica} & 140 & 130 & 270 \\
\hline
\tmcol{2}{|c|}{Futebol} & 80 & 240 \\
\hline
\end{tabular}
\caption{Versão da tabela usando tabular}
\label{tabela}
\end{table}

% \clearpage
\begin{verbatim}
\begin{table}[H]
\centering
\begin{tabular}{|c|c|c|c|}
\hline
\tcolor{Atividade Física} & \tcolor{Manhã} & \tcolor{Tarde} & 
\tcolor{Total} \\
\hline
\tcolor{Não pratica} & 140 & 130 & 270 \\
\hline
\tmcol{2}{|c|}{Futebol} & 80 & 240 \\
\hline
\end{tabular}
\caption{Versão da tabela usando tabular}
\label{tabela}
\end{table}
\end{verbatim}

Há também novos tipos de alinhamento de colunas, sendo estes \verb|d|, \verb|e| e \verb|f|.

Estes alinhamentos são implementados pelos seguintes comandos:
\begin{verbatim}
\newcolumntype{d}[1]{>{\vspace{2.5pt}}m{#1}<{\vspace{2.5pt}}}
\newcolumntype{e}[1]{>{\vspace{2.5pt}\centering}m{#1}<{\vspace{2.5pt}}}
\newcolumntype{f}{>$c<$}
\newcolumntype{g}[1]{>$e{#1}<$}
\end{verbatim}
\begin{itemize}
\item O alinhamento \verb|d| é o mesmo que o alinhamento \verb|m|, isto é, especificamos \verb|d{largura}|, onde entre chaves temos a largura de coluna desejada. A diferença dos dois alinhamentos é que é adicionado ao \verb|d| um espaçamento superior e inferior nas células desta coluna, o que é útil em casos que a célula é muito estreita para a entrada.

\item O alinhamento \verb|e| é o mesmo que o alinhamento \verb|d|, a única diferença é que o texto nas células é centralizado.

\item O alinhamento \verb|f| é o mesmo que \verb|c|, mas com todas as células na coluna com o modo matemático.

\item O alinhamento \verb|g| é uma mistura de \verb|e| e \verb|f|
\end{itemize}

Segue um exemplo do uso dos alinhamentos em uma tabela:

\begin{table}[H]
\centering

\begin{tabular}{|d{.15\linewidth}|e{.15\linewidth}|f|g{.15\linewidth}|}
\hline
\tcolor{Espaçamento da coluna \texttt{|d|}} & \tcolor{Espaçamento da coluna \texttt{|e|}} & $\tcolor{Modo matemático \texttt{|f|}}$ & $\tcolor{Modo matemático \texttt{|g|}}$ \tabularnewline
\hline
$\frac{\sqrt{3}}{\sqrt{3}}$ & $\frac{\sqrt{3}}{\sqrt{3}}$ &  \frac{\sqrt{3}}{\sqrt{3}} & \frac{\sqrt{3}}{\sqrt{3}} \tabularnewline
\hline
$\frac{\sqrt{3}}{\sqrt{3}}$ & $\frac{\sqrt{3}}{\sqrt{3}}$ &  \frac{\sqrt{3}}{\sqrt{3}} & \frac{\sqrt{3}}{\sqrt{3}} \tabularnewline
\hline
$\frac{\sqrt{3}}{\sqrt{3}}$ & $\frac{\sqrt{3}}{\sqrt{3}}$ &  \frac{\sqrt{3}}{\sqrt{3}} & \frac{\sqrt{3}}{\sqrt{3}} \tabularnewline
\hline
\end{tabular}
\end{table}

\begin{verbatim}
\begin{table}[H]
\centering
\begin{tabular}{|d{.15\linewidth}|e{.15\linewidth}|f|g{.15\linewidth}|}
\hline
\tcolor{Espaçamento da coluna \texttt{|d|}} & 
\tcolor{Espaçamento da coluna \texttt{|e|}} & 
$\tcolor{Modo matemático \texttt{|f|}}$ &
$\tcolor{Modo matemático \texttt{|g|}}$ \tabularnewline
\hline

$\frac{\sqrt{3}}{\sqrt{3}}$ & 
$\frac{\sqrt{3}}{\sqrt{3}}$ &  
\frac{\sqrt{3}}{\sqrt{3}} & 
\frac{\sqrt{3}}{\sqrt{3}} \tabularnewline
\hline

$\frac{\sqrt{3}}{\sqrt{3}}$ & 
$\frac{\sqrt{3}}{\sqrt{3}}$ &  
\frac{\sqrt{3}}{\sqrt{3}} & 
\frac{\sqrt{3}}{\sqrt{3}} \tabularnewline
\hline

$\frac{\sqrt{3}}{\sqrt{3}}$ & 
$\frac{\sqrt{3}}{\sqrt{3}}$ &  
\frac{\sqrt{3}}{\sqrt{3}} & 
\frac{\sqrt{3}}{\sqrt{3}} \tabularnewline
\hline
\end{tabular}
\end{table}
\end{verbatim}

Neste caso, tanto o alinhamento \verb|d| quanto o \verb|e| ficaram com a largura \verb|.3\linewidth|, isto é, $30\%$ da largura da linha. É possível observar que, no caso da coluna \verb|f|, o alinhamento vertical não é o mesmo das colunas \verb|d| e \verb|f|, não é necessário colocar \$ entre o texto matemático, apenas quando tivemos o texto na primeira linha. Ou seja, usamos os cifrões no sentido contrário, apenas quando queremos escrever textos não-matemáticos. Este modelo de coluna é útil quando se quer escrever colunas em que quase todas as entradas são textos matemáticos. 

\begin{observation}{Observação}
É importante sempre lembrar que ao usar os alinhamentos \verb|e| e \verb|g| como última coluna da tabela, deve-se pular a linha com \verb|\tabularnewline| em vez de \verb|\\|, pois como há o comando \verb|\centering| na sua definição, há problemas em como o \LaTeX{} gera a tabela.
\end{observation}

Por efeito de exemplo, segue abaixo a mesma tabela, mas usando os modelos de coluna original.

\begin{table}[H]
\centering

\begin{tabular}{|m{.15\linewidth}|c|}
\hline
\tcolor{Espaçamento da coluna \texttt{|m|}} & \tcolor{Modo matemático \texttt{|c|}} \\
\hline
$\frac{\sqrt{3}}{\sqrt{3}}$ & $\frac{\sqrt{3}}{\sqrt{3}}$ \\
\hline
$\frac{\sqrt{3}}{\sqrt{3}}$ & $\frac{\sqrt{3}}{\sqrt{3}}$ \\
\hline
\end{tabular}
\end{table}



\begin{verbatim}
\begin{table}[H]
\centering
\begin{tabular}{|m{.15\linewidth}|c|}
\hline

\tcolor{Espaçamento da coluna \texttt{|m|}} & 
\tcolor{Modo matemático \texttt{|c|}} \\
\hline

$\frac{\sqrt{3}}{\sqrt{3}}$ & 
$\frac{\sqrt{3}}{\sqrt{3}}$ \\
\hline

$\frac{\sqrt{3}}{\sqrt{3}}$ & 
$\frac{\sqrt{3}}{\sqrt{3}}$ \\
\hline

$\frac{\sqrt{3}}{\sqrt{3}}$ & 
$\frac{\sqrt{3}}{\sqrt{3}}$ \\
\hline

\end{tabular}
\end{table}
\end{verbatim}

\end{task}



Na caixa anterior temos o "Observação", que pode ser introduzido no texto no ambiente \verb|observation|:
\begin{verbatim}
\begin{observation}{titulo}
Sua observação não precisa necessariamente de um título
Para omiti-lo basta deixar as chaves de título vazias.
\end{observation}
\end{verbatim}

Além da caixa de atividade e observação temos mais quatro caixas:

\begin{research}
O "Para pesquisar", feito usando

\begin{verbatim}

\begin{research}
Seu o pesquisar não usa título.
\end{research}
\end{verbatim}
\end{research}

\begin{knowledge}
O "Você Sabia?", que usa o ambiente

\begin{verbatim}

\begin{knowledge}
Esta caixa também não possui título
\end{knowledge}
\end{verbatim}
\end{knowledge}

\begin{reflection}
O Para refletir

\begin{verbatim}

\begin{reflection}
Mais uma caixa que não tem título
\end{reflection}
\end{verbatim}
\end{reflection}

\begin{example}{Exemplo de uma caixa de exemplo}
Por fim, a última caixa é a de exemplo. Caso possua alguma atividade já com a resposta é recomendável usar essa caixa, que pode ser feita com
\begin{verbatim}

\begin{example}{Título}
Esta caixa pede um título, que pode possivelmente ser vazio.
\end{example}
\end{verbatim}
\end{example}

\begin{observation}
As caixas são feitas usando o pacote \verb|tcolorbox|, e é importante ressaltar que este pacote não permite sobrepor caixas, isto é, não se pode colocar uma caixa dentre de outra, e portanto não é possível colcoar uma observação no meio de uma atividade ou semelhantes.
\end{observation}

\arrange{Headers}

Além de boxes, também usamos headers (cabeçalho) para o início de seções. Assim como o explorando é feito com \verb|\explore{nome}|, temos:

\begin{itemize}
\item Organizando: \verb|\arrange{nome}|
\item Praticando: \verb|\practice{nome}|
\item Para saber+: \verb|\know{nome}|
\item Exercícios: \verb|\exercise|
\end{itemize}

Os headers não são ambientes, ou seja, não é necessário um \verb|\begin| e \verb|\end|, pois funcionam como seções novas. 

Uma característica interessante é a mudança de cores do texto em relação ao header, que faz justamente a mudança das cores de células das tabelas que vimos anteriormente

\begin{table}[H]
\centering
\begin{tabular}{|c|c|c|c|}
\hline
\tcolor{Atividade Física} & \tcolor{Manhã} & \tcolor{Tarde} & \tcolor{Total} \\
\hline
\tcolor{Não pratica} & 140 & 130 & 270 \\
\hline
\tmcol{2}{|c|}{Futebol} & 80 & 240 \\
\hline
\end{tabular}
\caption{Exemplo da mudança de cor}
\label{frequenciaatividade}
\end{table}
\clearpage
\practice{Outros tipos de padrões}

\begin{task}{Padrões no texto}
Colocamos sempre letras representando variáveis e conjuntos no modo de matemática. Números quando se referem a equações e unidades de medida, porcentagem e valores monetário também ficam em texto matemático. Porcentagens são escritas por \verb|$70\%$|, unidades de medida são do modo \verb|$70$mm|, e valores monetários \verb|R\$ $70$|, por exemplo.

O \LaTeX{} usa o sistema americano para casa decimais, e sempre coloca um espaço após a vírgula. Como aqui números decimais ficam após uma vírgula, é necessário usar \verb|{,}| para que não haja esse espaço (o espaço deve existir em outros casos, como pares ordenados e afins). Usamos também o padrão de pontos separando milhares, isto é, escrevemos 1.000.00 para representar um milhão.
\end{task}


\exercise

\begin{enumerate}
\item A seção de exercícios não possui título, e é feita usando o ambiente \verb|enumerate|.
\begin{enumerate}
\item É possível colocar enumerações
\begin{enumerate}
\item Em cima de numerações
\begin{enumerate}
\item Mas no momento apenas três níveis possuem \textit{labels}.
\end{enumerate}
\end{enumerate}
\end{enumerate}

\item E então pode continuar até onde for desejado.
\end{enumerate}

O ambiente a seguir é o de projeto aplicado
\begin{verbatim}
\begin{project}
o projeto aplicado não possui título
\end{project}
\end{verbatim}

\begin{project}
Este é ambiente para criação de um projeto aplicado. Ele cria uma nova página exclusiva para ele

\end{project}

\know{Sintaxe do Livro Aberto}

Aqui vamos falar um pouco da estrutura dos capítulos do Livro Aberto do Ensino Médio.

Todo capítulo começa em sua capa com 

\begin{itemize}
\item uma breve descrição do assunto do capítulo (\textbf{O quê?})
\item uma justificativa para o estudo no capítulo (\textbf{Por quê?}).
\end{itemize} 

\subsection{Para o professor (do capítulo)}
Após a página com os créditos, o capítulo começa com um "Para o professor" que versará sobre o capítulo inteiro, isto é, uma apresentação para os docentes sobre o que trata o capítulo e como deve ser trabalhado. O "Para o professor" do capítulo deve conter:

\begin{itemize}
\item Habilidades da BNCC ou extras que estão sendo contempladas.
\item Listar os conceitos e fatos matemáticos contidos no capítulo.
\item Destacar e justividar os conceitos e fatos que tradicionalmente são ensinados e que decidiu-se não ensinar no capítulo.
\item Destacar o percurso pedagógico a partir da estrutura de seções do capítulo.
\item Apresentar os pré requisitos para o capítulo.
\item O que há de inovador no capítulo em relação aos materiais didáticos correntes.
\item Citar a bibliografia utilizada e que serviu de base para as decisões na construção do material ao longo deste texto e as referências no final do mesmo.
\end{itemize}

\subsection{Seção}

Cada seção possui a seguinte estrutura:

\subsection{Para o professor (da seção)}
Este "Para o professor"{} deve conter:
\begin{itemize}
\item Lista de Objetivos Específicos de Aprendizagem da seção.
\item Conhecimento Pedagógico do assunto: Matemática subjacente, que eventualmente não seja explicitada aos estudantes, mas que seja utilizada como base da seção (opcional).
\item Apresentação do percurso pedagógico da seção, explicando a linguagem, o que espera-se ensinar nesta seção e o que espera-se que não seja ainda discutido.
\end{itemize}

\def\currentcolor{session1}
\subsection{Explorando: Título da seção}

Aqui são colocadas as atividades de exploração do assunto, ou seja, atividades que consigam refletir o que será tratado na seção. É opcional possuir um texto introdutório (mas desejado).

As atividade também devem possuir um "Para o professor", este que deve conter:
\begin{itemize}
\item Objetivos Específicos de Aprendizagem.
\item Sugestões e discussões, como: organização da turma; tempo estimado; conceitos abordados; linguagem; dificuldades previstas; sugestões gerais; enriquecimento da discussão; conexões; material necessário (os itens são opcionais e variam de acordo com a necessidade da atividade).
\item Solução completa.
\end{itemize}

\def\currentcolor{session4}
\subsection{Organizando: Título da seção}

No Organizando colocamos a sistematização do conteúdo da seção, formalizando os conceitos que foram investigado pelos alunos anteriormente. É recomendado colocar exemplos para a melhor compreensão do aluno.

\def\currentcolor{session2}
\subsection{Praticando: Título da Seção}

Aqui são colocadas mais atividades que complementem a seção. O praticando é opcional.

\def\currentcolor{primario}
\subsection{Exercícios}

Os exercícios no final da seção são também opcionais. Deve também conter um para o professor do exercício, que deve onter os Objetivos Específicos de Aprendizagem e a resposta ou solução (curta).


Ao fim de todas as seções do capítulo, é necessário que haja exercícios que reflitam o conteúdo do capítulo. Além dos exercícios, é opcional haver o \textcolor{session3}{\textbf{Para Saber+}} e o \textcolor{cor2}{\textbf{Projeto Aplicado}}.

\def\currentcolor{session3}
\subsection{Boxes}

Os boxes podem ser posicionados nas seções Explorando, Organizando as Ideias ou Praticando, mas é importante que respeitem as seguintes orientações:

\begin{observation}{Observação}
Esta caixa é essencial. Serve para destacar comentários fundamentais para o desenvolvimento do estudante ou para o aprendizado do conteúdo. Sua leitura é obrigatória.
\end{observation}
\clearpage
\begin{reflection}
A caixa "Para refletir"{} é importante, mas não essencial. Convida o estudante a uma reflexão adicional ou de aprofundamento. Pergunta, afirmação ou curiosidade. A leitura é recomendada.
\end{reflection}

\begin{knowledge}
Esta caixa é para aprofundamento. Não traz grande prejuízo para o cumprimento da habilidade proposta se o estudante pular estas caixas. É uma espécie de "Para saber mais"{ curto que pode ser usado nas demais seções}.

Curiosidades sobre conceitos, propriedades ou termos utilizados, usos dos conceitos em outros contextos, conexão com outras áres do conhecimento ou da própria matemática, etc.
\end{knowledge}