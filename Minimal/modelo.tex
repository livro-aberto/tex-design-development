%Define ilustração da capa do capítulo
\ifnum\aluno=1 
% Versão do aluno
\renewcommand\chapterillustration{abertura-capitulo}
\else
% Versão do professor
\renewcommand\chapterillustration{abertura-capitulo-professor}
\fi

% O quê
\renewcommand\chapterwhat{}

% Por quê
\renewcommand\chapterbecause{}


\chapter{Nome do capítulo}


% Página de créditos
\mbox{}\thispagestyle{empty}\clearpage

\thispagestyle{empty}

\begin{center}
Projeto: LIVRO ABERTO DE MATEMÁTICA

\noindent \begin{tabular}{lcccr}
\includegraphics[scale=.15]{impa}& \quad\quad& \includegraphics[width=3cm]{logo} & \quad\quad& \includegraphics[scale=.24]{obmep} 
\end{tabular}
\end{center}

\vspace*{.3cm}

Cadastre-se como colaborador no site do projeto: \url{umlivroaberto.org}

Versão digital do capítulo:

\url{https://www.umlivroaberto.org/BookCloud/Volume_1/master/view/GE301.html}


\begin{tabular}{p{.15\textwidth}p{.7\textwidth}}
Título: & Nome do capítulo\\
\\
Ano/ Versão: & \the\year / versão x.x de \today\\
\\
Editora & Instituto Nacional de Matem\'atica Pura e Aplicada (IMPA-OS)\\
\\
Realização:& Olimp\'iada Brasileira de Matem\'atica das Escolas P\'ublicas (OBMEP)\\
\\
Produção:& Associação Livro Aberto\\
\\
Coordenação:& Fabio Simas, \\
            & Augusto Teixeira (livroaberto@impa.br)\\
\\
  Autores: & Autor 1,\\
           & Autor 2,\\
             \\
Revisão: & Revisor \\
                
\\
Design: & Andreza Moreira (Tangentes Design) \\
\\
  Ilustrações: & --- \\ 
\\
Gráficos: & --- \\
		    

\\
  Capa: & Foto de ---, no --- \\
  		& \url{endereço da imagem} \\

\end{tabular}

\begin{figure}[b]
\begin{minipage}[l]{5cm}
\centering

{\large Licença:}

  \includegraphics[width=3.5cm]{cc-by-sa1}
\end{minipage}\hfill
\begin{minipage}[c]{5cm}
\centering
{\large Desenvolvido por}

\includegraphics[width=2.5cm]{logo-associacao.jpg}
\end{minipage}
\begin{minipage}[r]{5cm}
\centering

{\large Patrocínio:}
  \vspace{1em}
  \includegraphics[width=3.5cm]{itau}
\end{minipage}
\end{figure}


\mainmatter


%%%%%%%%%%% Comandos do material do professor

% Introdução no livro do professor
\begin{apresentacao}{Título}


\end{apresentacao}

\def\currentcolor{session1}

% Objetivos específicos
\begin{objectives}{}
{

}{}{}
\end{objectives}

% Sugestões e Discussões
\begin{sugestions}{}
{

}{}{}
\end{sugestions}

% Resposta
\begin{answer}{}
{

}{}
\end{answer}

%%%%%%%%%%%%%%
% Observação % As caixas do material do aluno devem ser colocadas antes do código da seção, usando o comando \clearmargin para mudar de páginas
%%%%%%%%%%%%%%


%%%%%%%%%%% Comandos do material do aluno

% Explorando
\explore{}

% Caixa de atividade
\begin{task}{}

\end{task}

% Caixa de exemplo
\begin{example}{}

\end{example}

% Caixa do Você sabia?
\begin{knowledge}

\end{knowledge}

% Caixa de Observação
\begin{observation}

\end{observation}

% Caixa de Observação com título
\begin{observationtitle}{}

\end{observationtitle}

% Caixa de Refletindo
\begin{reflection}

\end{reflection}


% Organizando
\arrange{}


% Praticando
\practice{}

% Para saber +
\know{}

% Exercícios
\exercise


\begin{enumerate}
\item 

\item 

\item 
\end{enumerate}