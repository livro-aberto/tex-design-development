\documentclass[extrafontsizes, twoside, 11pt, openright, final]{memoir}

\RequirePackage[skins,breakable]{tcolorbox}

\usepackage{cmap}
\usepackage[T1]{fontenc}
\usepackage{amsmath,amssymb,amstext}
\usepackage[brazil]{babel}
\usepackage{times}
%\usepackage[Sonny]{fncychap}% This inser
\usepackage{sphinx}%To recognize sphinxadmonition, sphinxincludegraphics, etc.
%\usepackage{tabulary}
\usepackage{geometry}
\usepackage{sphinxcontribtikz}
% Include hyperref last.
\usepackage{hyperref}
% Fix anchor placement for figures with captions.
\usepackage{hypcap}% it must be loaded after hyperref.


\usepackage{amsmath,amssymb,amsfonts,amsthm}
\usepackage{tikzmacros}
\usepackage{livroabertopage}
\usepackage{livroabertofonts}
\usepackage{livroabertocolors}
\usepackage{livroabertoheadings}
\usepackage{livroabertoboxes}
\usepackage{livroabertolists}
\usepackage{livroaberto}
\RequirePackage{multicol}


\begin{document}


% \setcounter{chapter}{0}
% \include{capa-funcoes}
% \setcounter{chapter}{1}
% \include{capa-afim}
% \setcounter{chapter}{2}
% \include{capa-quadraticas}
% \include{vetores-novo1}         
% \chapterillustration{./abertura-perspectiva1}{./abertura-perspectiva1-professor}

\chapterwhat{Conceito de volume (unidade, aditividade e conservação). Diversos usos de volumes e grandezas relacionadas, como área, densidade, concentração. Compressibilidade de materiais. Posições relativas de planos e planos e planos e retas. Planificações e cortes de sólidos. Cálculo de áreas e volumes de figuras clássicas e obtenção das fórmulas. Aproximação de áreas e volumes considerando o erro. Princípio de Cavalieri e aplicações.}

\chapterbecause{Áreas e volumes são conceitos elementares que estão presentes de maneira direta na vida cotidiana do cidadão e também são necessários para outras ciências, como a Química, a Biologia e a Física. O entendimento destes conceitos e de suas relações com outras grandezas são fundamentais e basta m para a maioria dos usos corriqueiros. Aqui, o estudo de áreas e volumes serve de plano de fundo para o aperfeiçoamento do entendimento de número real, de aproximação com erro, para o desenvolvimento da habilidade de visualização espacial e para a resolução de problemas.} 

\chapter{Medidas em Geometria Espacial}


%%%% Página de créditos

% Autores
\autorum{Augusto Teixeira}
\autordois{Fabio Simas}
\autortres{José Ezequiel Soto Sánchez}
\autorquatro{Letícia Rangel}


\graficos{Tarso Caldas}

% Revisores

\autordacapa{Luke Porter}{Unsplash}{https://unsplash.com/photos/ud6XcK_MUGI}
\versao{0.5}

\versaodigital{https://www.umlivroaberto.org/BookCloud/Volume_1/master/view/GE504.html}


\ccbysa

\creditos


\mainmatter

\label{\detokenize{GE504:medidas-em-geometria-espacial}}\label{\detokenize{GE504::doc}}

\def\currentcolor{session1}
\begin{objectives}{Volume de uma folha de papel}
{
\begin{itemize}
\item {} 
Reconhecer o conceito de volume (ideia intuitiva, medida do espaço ocupado por um determinado material incompressível), distinguindo-o da área, da densidade e da massa, por exemplo.

\item {} 
Aplicar o conceito de unidade (caixa de leite ou cubo de lado 1, que dará origem a uma unidade de medida) para comunicar e comparar volumes. Volume, área, comprimento, litro (e outras unidades de medida volumétrica do SI: cm\(^3\), m\(^3\), etc.).

\item {} 
Aplicar em contextos diversos o conceito de aditividade, isto é, que a área (e o volume) da união de conjuntos disjuntos é igual à soma de suas áreas (respectivamente volumes), estendendo essa propriedade à subtração quando se fazem “buracos” de formas conhecidas.

\end{itemize}

\textbf{Conceitos abordados:} Volume, área, gramatura.
}{1}{1}
\end{objectives}
\begin{sugestions}{Volume de uma folha de papel}
{
\textbf{Organização em sala de aula:} Nesta atividade, inicialmente o aluno deve estimar o volume de uma folha de papel. A discussão em grupos contribuirá para a avaliação das estimativas realizadas. Os próprios alunos avaliarão as estimativas uns dos outros. Portanto, recomenda-se que a atividade seja desenvolvida em grupos de 3 ou 4 estudantes. Recomendação análoga vale para os demais itens.

\textbf{Dificuldades previstas:} Não é improvável que os estudantes confundam volume e área, respondendo a área do papel como resultado para o volume. Nesse caso, muito provavelmente, o aluno não identificou a espessura do papel. Outra possibilidade é responder que o volume é zero. Aproveite esta atividade para discutir tal diferença. É importante que reconheçam que área é uma grandeza bidimensional e volume tridimensional.

\textbf{Sugestões gerais:} Espera-se que o item \titem{a)} seja respondido por comparação visual. Não se espera que o aluno realize cálculos organizados. A ideia é que aluno estime \(1\) cm$^3$ e relacione com a folha de papel.

Para o desenvolvimento da tarefa, recomenda-se ainda que sejam entregues folhas de papel aos alunos. Acreditamos que a distribuição das folhas pode enriquecer a atividade trazendo concretude e uma postura investigativa, ainda que bastasse ler as informações apresentadas no pacote ilustrado.

Uma estratégia possível para realizar o cálculo do volume é empilhar (bem compactadas) várias folhas. Essa estratégia se baseia na propriedade de aditividade do volume. Nesse caso, os alunos precisarão medir as 3 dimensões da pilha. Será necessário régua. Se possível, leve para a sala uma resma em pacote fechado para que os alunos possam obter tais medidas mais facilmente. O pacote tem 4,9cm de altura. No entanto, não os alerte para essa possibilidade. Deixe-os ter autonomia na condução da solução.

Caso não se tenha uma resma de papel sulfite A4 disponível, a tarefa pode ser realizada com uma pilha com uma quantidade menor de folhas de papel A4 ou com outro tipo de papel que se tenha disponível. Valem até a folha do caderno ou do livro. Outra possibilidade é recortar o papel. Nesse caso, os pedaços devem ser empilhados até que se possa medir a espessura da pilha. Mais importante do que o resultado é a discussão da estratégia utilizada para o cálculo. Se não for utilizar o papel sulfite tamanho A4 \textendash{} gramatura \(75\) g/m$^2$, será necessário ter as informações correspondentes para o papel que for utilizado (largura, comprimento e gramatura), que podem alterar as respostas.

\textbf{Enriquecimento da discussão:} Além do conceito de volume, na atividade, um dos itens trata da gramatura de papel, diferenciando-a de espessura. A gramatura do papel é a medida da massa pela área de um papel expressa em gramas por metro quadrado (g/m\(^2\)). A intenção desse item é que o aluno pesquise a resposta e tenha que interpretar e compreender a informação obtida, que é essencialmente matemática: uma medida. Discuta diferentes respostas obtidas visando ao entendimento dessa medida.

\textbf{Material necessário:} Folhas de papel sulfite tamanho A4 - gramatura \(75g/m^2\).
Régua milimetrada
}{0}{9}
\end{sugestions}
\begin{answer}{Volume de uma folha de papel}
{
\begin{enumerate}
\item {} 
Resposta pessoal.

\item {} 
Considerando uma folha de papel sulfite de tamanho A4 mais comum no mercado, aproximadamente, $6{,}1122$ cm\(^3\). Tal aproximação pode ser obtida a partir das medidas de uma resma. Uma resma desse papel, que é composta por 500 folhas, tem dimensões aproximadas $21{,}0\text{ cm}\times 29{,}7\text{ cm} \times 4{,}9{ cm}$. Portanto, 500 folhas têm volume \(3056,13cm^3\), e uma folha, $1/500$ desse valor. (Essa resposta pode variar dependendo do papel utilizado)

\item {} 
A gramatura corresponde à massa de \(1\) m$^2$ do papel. Assim, por exemplo, a gramatura \(90\) g/m$^2$ significa que a massa de \(1\) m$^2$ do papel é \(90\) g. Já a espessura é uma medida linear, que pode ser observada como a menor das dimensões da folha de papel. Há papéis de diferentes espessuras. Por exemplo, não é difícil observar que a folha de papel sulfite, comumente usada na escola, tem espessura diferente do papel utilizado para confeccionar caixas de sapatos, por exemplo. Espessura e gramatura, são, portanto, propriedades diferentes do papel. No entanto, em geral, papéis com maior gramatura têm maior espessura.

\item {} 
O peso da folha pode ser calculado a partir da gramatura do papel, informada no pacote:  \(75\) g/m$^2$. Nesse caso, é necessário ainda calcular a área da superfície da folha: \(623{,}7\text{ cm}^2 = 0{,}06237\text{ m}^2\). Portanto, o peso do papel é \(75 \times 0{,}06237 = 4{,}6\) g.

\end{enumerate}
}{1}
\end{answer}
\clearmargin
\begin{objectives}{Caminhonete de areia}
{
\begin{itemize}
\item {} 
Reconhecer o conceito de volume (ideia intuitiva, medida do espaço ocupado por um determinado material incompressível), distinguindo-o da área, da densidade e da massa, por exemplo.

\item {} 
Aplicar o conceito de unidade (caixa de leite ou cubo de lado 1, que dará origem a uma unidade de medida) para comunicar e comparar volumes.

\item {} 
Aplicar relações entre (área e) volume e outras grandezas em situações cotidianas.

\end{itemize}

\textbf{Conceitos abordados:} massa (peso), volume, densidade, unidades de medida.
}{1}{2}
\end{objectives}
\begin{sugestions}{Caminhonete de areia}
{
\textbf{Organização em sala de aula:} Recomenda-se que esta atividade seja realizada em duplas. A discussão com um colega pode ajudar a interpretar o enunciado, que é essencial neste caso.

\textbf{Dificuldades previstas:} A atividade oferece a revisão de alguns conceitos, como massa e densidade, que não costumam oferecer maior dificuldade. No entanto, a abordagem envolve mais a relação entre esses conceitos do que o cálculo. A dificuldade pode emergir daí.

\textbf{Enriquecimento da discussão:} Esta atividade aborda o conceito de volume aparente, que volta a ser discutido de forma mais específica em um “Você Sabia” seguinte.

Para determinar o volume de areia que precisa comprar, Gelson pode simplesmente levar um saco de cimento vazio à loja de materiais para construção e pedir 15 sacos daquele cheios de areia. Esta é uma discussão interessante para o estudante, porque o saco de cimento se torna uma unidade de volume. Na situação deste parágrafo, ela está sendo utilizada para medir o volume de areia a ser comprado.
}{1}{2}
\end{sugestions}


\explore{O Conceito de Volume}
\label{\detokenize{GE504-0:explorando-o-conceito-de-volume}}\label{\detokenize{GE504-0::doc}}

\begin{task}{volume de uma folha de papel}


\begin{enumerate}
\item {} 
Lembrando que um centímetro cúbico é o volume ocupado por um cubo de aresta 1cm, estime sem fazer cálculos o volume de uma folha de papel sulfite de tamanho A4.
\end{enumerate}

\begin{figure}[H]
\centering


\begin{asy}
size(5cm);
currentprojection=orthographic(3,1,.5);

draw(unitcube, azul*80+opacity(0.65));

draw((1,0,1) -- (1,0,0), verde+linewidth(1.25), L=Label("a",position=MidPoint));
draw((1,0,0) -- (1,1,0), laranja+linewidth(1.25), L=Label("b",position=MidPoint));
draw((1,1,0) -- (0,1,0), vinho+linewidth(1.25), L=Label("c",position=MidPoint));

draw((0,0,0) -- (1,0,0), dashed);
draw((0,0,0) -- (0,1,0), dashed);
draw((0,0,0) -- (0,0,1), dashed);

draw((0,0,1) -- (0,1,1));
draw((0,1,1) -- (1,1,1));
draw((1,1,1) -- (1,0,1));
draw((1,0,1) -- (0,0,1));
draw((0,1,1) -- (0,1,0));
draw((1,1,1) -- (1,1,0));
\end{asy}

\end{figure}
\begin{enumerate}
\item {} 
Avalie a sua estimativa no item anterior. Use uma estratégia de cálculo para obter o volume de uma folha de papel sulfite de tamanho A4.

\end{enumerate}

\begin{figure}[H]
\centering

\noindent\includegraphics[width=100bp]{{1}.png}
\end{figure}
\begin{enumerate}
\item {} 
O que é a gramatura do papel? Qual é a diferença e qual é a relação entre espessura e  gramatura de uma folha de papel? Pesquise.

\item {} 
Quanto pesa uma folha de papel da resma ilustrada no item \titem{b)}?

\end{enumerate}
\end{task}

\begin{knowledge}

A gramatura de uma folha de papel usada em escolas e escritórios costuma variar de $75$ a \(120\) g/m$^2$.  São as indicadas para impressoras domésticas, por exemplo. Para a confecção de cartões e impressão de fotos, são recomendados papéis de maior gramatura, em torno de \(200\) g/m$^2$.  As folhas de um jornal têm gramatura de $35$ a \(55\) g/m$^2$.  A escolha da gramatura é determinante para o uso do papel. Imagine as implicações de um jornal impresso em papel de maior gramatura. Seria mais pesado, o que além de ter impacto direto no manuseio e no custo do material, com certeza, influenciaria no transporte e na impressão, por exemplo. De maneira geral, quanto maior a gramatura, mais resistente é o papel. No entanto, não se deve confundir gramatura com espessura nem com volume. Ainda que papéis com gramaturas diferentes tendam a ter espessuras diferentes, a compactação das fibras e materiais que compõem o papel determinará se as espessuras serão ou não distintas. E, portanto, os volumes também.
\end{knowledge}

\begin{task}{caminhonete de areia}

\paragraph{Parte 1}

Gelson vai fazer um quarto novo para sua filhinha que está chegando. O tijolo e o cimento ele já tem, mas precisa comprar areia para misturar no cimento e começar a obra!

Quanta areia ele precisará comprar?

Gelson utilizará três sacos de cimento e sabe que a proporção recomendada para assentar tijolos é de cinco latas de areia para cada lata de cimento. Gelson avaliou que deveria comprar quinze sacos de areia. No entanto, a areia não é vendida em sacos como os de cimento. A areia fica armazenada em um galpão e é vendida por metro cúbico (m\(^3\)).

\begin{figure}[H]
\centering

\noindent\includegraphics[width=125bp]{{2}.png}
\end{figure}

Quantos metros cúbicos de areia Gelson deve comprar para realizar a obra evitando o desperdício de material?

\begin{figure}[H]
\centering

\noindent\includegraphics[height=130bp]{{3}.png}
\hspace{1em}
\includegraphics[height=130bp]{{4}.png}
\end{figure}

\begin{enumerate}
\item {} 
Avalie as perguntas a seguir e decida qual (ou quais) delas que, uma vez respondidas, permitiriam que Gelson comprasse a quantidade certa de areia:
\begin{itemize}
\item {} 
P1: Quantos sacos de cimento cheios de areia são necessários para se obter um metro cúbico de areia?

\item {} 
P2: Quantos metros cúbicos de areia são necessários para se misturar em um saco de cimento?

\item {} 
P3: Quantos metros cúbicos de cimento serão utilizados?

\item {} 
P4: Quanto pesa a areia que cabe em um saco de cimento?

\end{itemize}

\item {} 
Qual (ou quais) das perguntas do item anterior podem ser respondidas com procedimentos simples feitos em casa? Descreva tais procedimentos.

\item {} 
Para descobrir quantos metros cúbicos correspondem a quinze sacos de areia, Gelson despejou o cimento de um saco em um balde de \(20\)lL. Verificou que o cimento coube no balde enchendo-o completamente. Com essa informação, quantos metros cúbicos de areia Gelson precisa comprar?

\item {} 
O problema de Gelson agora é transportar a areia até a sua casa. Para isso, ele utilizará uma caminhonete como a da imagem a seguir. Gelson consegue transportar toda a areia em uma só viagem?

\end{enumerate}

\begin{figure}[H]
\centering

\noindent\includegraphics[width=300bp]{{5}.png}
\end{figure}

\paragraph{Parte 2}

Resolvido o problema do volume a ser carregado, Gelson passou a pensar de a caminhonete aguenta o peso deste tanto de areia. No manual da caminhonete está escrito que sua carga máxima é de \(530\) kg.

\begin{figure}[H]
\centering

\noindent\includegraphics[width=350bp]{{6}.png}
\end{figure}
\begin{enumerate}
\item {} 
Em uma estimativa grosseira, quanto você acha que pesa um metro cúbico de areia?

\item {} 
Gelson procurou na internet “Qual é o peso de um metro cúbico de areia?”. Ele achou várias respostas. Dependendo do tipo de areia, a densidade (isto é, massa / volume) pode variar de \(1200\) kg/m$^3$ até \(1700\) kg/m$^3$. Com essas informações, Gelson pode ter ceteza de que a viagem para transportar a areia comprada estará dentro das especificações da caminhonete?

\item {} 
Gelson decidiu pesar a areia comprada. Para isso, encheu o balde (de $20$ litros) com areia e o pesou, obtendo \(26\) kg. Com tal informação, que estimativa Gelson pode fazer para o peso total da areia que ele vai comprar?

\item {} 
No dia do transporte choveu e entrou água na caçamba. Gelson observou que aparentemente o nível de areia não havia se alterado, apesar da água. No entanto, se preocupou com o limite de peso, uma vez que o carro parecia perder estabilidade. Houve alteração no volume da carga transportada devido à chuva? Explique a sua resposta considerando a percepção de Gelson de que o nível de areia não se alterou.

\end{enumerate}
\end{task}

\begin{knowledge}

Em muitas situações do cotidiano as misturas são descritas por razões, como em:
\begin{itemize}
\item {} 
Misturamos o cimento com a areia na razão de $1$ para $5$.

\item {} 
Uma parte de farinha para três partes de leite.

\end{itemize}

Tais instruções podem ser imprecisas se não especificarem a que grandezas corresponde a razão indicada. Observe que as frases acima não diferenciam entre: para cada quilo de cimento usamos cinco quilos de areia, ou para cada litro de cimento utilizamos cinco litros de areia. Ou para cada quilo de farinha três litros de leite ou para cada colher de farinha três litros de leite.

Isso ocorre por exemplo na especificação do álcool para uso doméstico. Nas garrafas desse tipo de álcool a razão entre álcool e água é indicada em graus INPM. Assim, por exemplo, no álcool \(46^\circ\) INPM, há $46$ g de álcool em cada $100$ g do produto. Os $54$ g restantes são de água. Observe que se esta razão, especificada para massa, for considerada a mesma razão para volume, resultará em outra gradação INPM de álcool, já que álcool e água possuem densidades diferentes.
Em muitas situações do cotidiano vemos razões descritas na forma de razões, como em:

\begin{figure}[H]
\centering

\noindent\includegraphics[width=150bp]{{8_1}.jpg}
\end{figure}

Em alguns casos essa distinção pode ficar subentendida pelo contexto. Por exemplo, no caso da farinha e do leite é mais natural que essa razão esteja se referindo à volume, pois dificilmente medimos leite pelo peso, mas frequentemente medimos farinha em volume (copos, xícaras, colheres, etc.).

Tente inferir em cada um dos exemplos, se as razões se referem provavelmente a pesos ou a volumes:
\begin{enumerate}
\item {} 
Um alimento possui vinte vezes mais gordura do que fibra.

\item {} 
Uma tinta de tecidos deve ser misturada na água na razão de um para dez.

\item {} 
Um adubo deve ser misturado na razão de uma parte para cada oito partes de terra.

\end{enumerate}

Em outras situações, não faz tanta diferença se aplicamos a razão em termos de peso ou de volume. escolhemos a razão em termos de peso ou volume. Isso se dá quando a densidade dos materiais envolvidos é muito semelhante.

Tente inferir em quais situações é muito importante saber se as razões se referem a peso ou a volume:
\begin{enumerate}
\item {} 
Duas partes de leite para uma parte de óleo.

\item {} 
Uma parte de açúcar para cinco partes de chantili.

\item {} 
Uma parte de água para quatro partes de areia.

\end{enumerate}
\end{knowledge}

\clearpage
\begin{objectives}{Volume do paralelepípedo retângulo de arestas racionais}
{
Entender a demonstração da fórmula do volume de paralelepípedos retângulos (e áreas de retângulos) de lados racionais.

\textbf{Conceitos abordados:}
Volume de paralelepípedo retângulo. De modo indireto também são abordados funções e a subdivisão da unidade (frações).
}{1}{2}
\end{objectives}
\begin{sugestions}{Volume do paralelepípedo retângulo de arestas racionais}
{
\textbf{Organização em sala de aula:}
Sugerimos que a atividade seja individual ou em duplas. A reflexão mediada pelo aplicativo pode ser favorecida pela discussão com um colega. Grupos maiores podem gerar dispersão.

\textbf{Dificuldades previstas:}
Acreditamos que a atividade impõe desafios importantes, como lidar com uma fração da unidade de volume e compreender volume como uma função. No entanto, destacamos a compreensão da relação entre a variação dos lados e a variação do volume. Não é incomum que os alunos, apesar de reconhecerem na atividade que  \(V(n_1 x, n_2 y, n_3 z) =  n_1.n_2.n_3. V(x, y, z)\), não consigam aplicar tal resultado diretamente.
.. AQUI, ACHO, O CERTO SERIA INCLUIR INDICACAO DE ATIVIDADES NO MATERIAL QUE TRATEM DO ASSUNTO.

\textbf{Sugestões gerais:}
Esta atividade pretende levar o aluno a perceber o volume de um paralelepípedo para além da contagem de cubinhos, ou seja, extrapolar o universo dos números naturais. Espera-se que os alunos associem o volume do paralelepípedo (retângulo) à variação de suas dimensões como “medidas contínuas”, ou seja, que percebam de maneira intuitiva (não esperamos nem recomendamos a formalização) que o volume de um paralelepípedo é uma função contínua de três variáveis reais positivas (\(V(a,b,c) = V\)). Destacamos que,  nesta atividade, esse fato é explorado apenas para dimensões (variáveis) racionais. Observe que, no Ensino Fundamental, a fórmula de cálculo do volume do paralelepípedo retângulo é deduzida apenas para arestas naturais (contagem de cubinhos). Entendemos que no Ensino Médio o aluno poderá compreendê-la para arestas racionais, ficando a dedução da mesma para arestas de medidas reais apenas para alguns cursos superiores.

Ao longo de toda a atividade, recomenda-se que as justificativas sejam valorizadas porque o resultado em si é a mera aplicação da fórmula \(V(a,b,c) = abc\) em diversos itens.
}{1}{2}
\end{sugestions}

\clearmargin
\begin{sugestions}{Volume do paralelepípedo retângulo de arestas racionais}
{
A atividade foi planejada para ser realizada com o uso dos aplicativos recomendados, ainda que possa ser sem eles. Os aplicativos permitem visualização e interação dinâmica com os objetos  geométricos, contribuindo para a comunicação o ensino e a aprendizagem.

Na Parte 1, espera-se que o aluno compreenda a notação usada na atividade; reconheça que, com a troca dos comprimentos de duas das arestas de um paralelepípedo retângulo, obtém-se paralelepípedos congruentes; reconheça que paralelepípedos congruentes têm o mesmo volume e que paralelepípedos com medidas completamente diferentes podem ter o mesmo volume. Por fim, o item (d) convida o aluno para a reflexão conduzida na parte 2.

Você pode usar uma caixa em forma de paralelepípedo retângulo, destacando as arestas, para facilitar a compreensão da notação pelos estudantes. A manipulação permite observar que a troca, por exemplo, de \(a\) por \(b\) na expressão de V indica uma rotação da caixa e, portanto, uma caixa congruente à inicial.
No item a), não se preocupe se os estudantes não forem cuidadosos com as medidas das arestas ao desenhar os paralelepípedos porque eles provavelmente conseguirão perceber seus erros no item b).
No item b), espera-se que o estudante observe que o paralelepípedo de arestas 2, 3 e 4 é congruente ao de arestas 2, 4 e 3 (são iguais do ponto de vista da geometria) e, portanto, seus volumes são iguais.
No item c), espera-se que os estudantes percebam que volumes iguais podem ser obtidos por paralelepípedos não congruentes, inclusive bastante diferentes entre si.

Na Parte 2, espera-se que os estudantes percebam que, dado um paralelepípedo retângulo qualquer, se multiplicarmos uma de suas arestas por um número natural, então o volume do novo paralelepípedo (que não é semelhante ao primeiro) ficará multiplicado por esse mesmo número natural. O aplicativo permite que sejam gerados variados exemplos, o que ajuda o aluno na compreensão. Também aqui pode valer a pena usar caixas em forma de paralelepípedos para facilitar a visualização.
}{1}{1}
\end{sugestions}
\clearmargin
\begin{sugestions}{Volume do paralelepípedo retângulo de arestas racionais}
{
A Parte 3 é a parte mais delicada para o estudante, por isso recomendamos fortemente o uso dos aplicativos disponibilizados.
O item \textit{a)} tem o objetivo de familiarizar o estudante com a visualização de paralelepípedos que são frações do cubo unitário e relacioná-los com o próprio cubo unitário.
No item \titem{b)}, a parte \titem{v)} verifica se o estudante consegue generalizar a construção dos itens anteriores.

\textbf{Enriquecimento da discussão:}so
Ainda que não sejam discussões sugeridas, nem recomendadas, para a sala de aula, cabe observar que:
Na Parte 1, item c), é verificado que a função V não é injetiva uma vez que, por exemplo, \(V(1, 1, 72) = V(2, 4, 9)\).
Da Parte 2, pode-se concluir que a função volume é linear em cada uma de suas coordenadas.
Sobre a parte 2, destacamos que, nas aulas de função linear, talvez possa ser observado que o volume de um paralelepípedo retângulo de arestas fixadas é uma função linear da terceira aresta (por exemplo, se as arestas são 2, 3 e \(x\), então o volume é \(V(x) = 6x\)).
Além disso, também na parte 2, item c), observe que \(V(nx, ny, nz)\) é o volume de um paralelepípedo semelhante ao paralelepípedo de lados  \(x\), \(y\) e \(z\). Portanto, \(V(nx, ny, nz) = n^3V(x, y, z)\). Isto pode ser apresentado pelo professor como um desdobramento, caso o professor julgue pertinente.

\textbf{Links relacionados:}
Todos os aplicativos disponibilizados para esta atividade foram criados para serem facilmente utilizados em telas pequenas.

\textbf{Materiais necessários:}
A atividade foi planejada para o uso de aplicativos computacionais. Recomendamos que eles sejam utilizados, porque ampararão a construção de uma imagem representativa  A não utilização desses recursos não inviabiliza a realização da atividade, no entanto pode não atingir plenamente os objetivos.
}{1}{2}
\end{sugestions}

\begin{knowledge}

Lembrando da chuva que atrapalhou a viagem do Gelson, vamos pensar mais sobre o conceito de volume. Quando Gelson saiu do armazém, transportava apenas areia. Com a chuva, entrou água na caçamba sem que o nível da carga de areia se alterasse, ou seja, sugerindo que o volume de carga não se alterou.

Usualmente, quando medimos o volume de materiais granulados (como arroz, açúcar e areia) consideramos o volume do ar que fica entre os grãos. Assim, um copo com capacidade para $200$ ml cheio de açúcar, na verdade contém açúcar e ar. Logo, o volume real de açúcar é menor do que $200$ ml. Isso pode ser verificado, por exemplo, colocando-se, aos poucos, água em um copo cheio de açúcar e observando que o nível do conteúdo no copo não aumenta de início e não transborda imediatamente.
\end{knowledge}

\begin{task}{volume do paralelepípedo retângulo de arestas racionais}
\label{persp1-atividade-3}


A fórmula para o cálculo do volume de um paralelepípedo já é conhecida desde o Ensino Fundamental. Mas talvez você não saiba explicar por que essa fórmula vale. Esta atividade tem o objetivo de explorar o tema. Para isso, o cubo de aresta 1 será considerado como unidade e será chamado de \emph{cubo unitário}.

\begin{figure}[H]
\centering

\begin{asy}
size(5cm);
currentprojection=orthographic(2,0.5,1/2);

draw(unitcube, azul*80+opacity(0.65));

draw((1,0,1) -- (1,0,0));
draw((1,0,0) -- (1,1,0));
draw((1,1,0) -- (0,1,0));

draw((0,0,0) -- (1,0,0), dashed);
draw((0,0,0) -- (0,1,0), dashed);
draw((0,0,0) -- (0,0,1), dashed);

draw((0,0,1) -- (0,1,1));
draw((0,1,1) -- (1,1,1));
draw((1,1,1) -- (1,0,1));
draw((1,0,1) -- (0,0,1));
draw((0,1,1) -- (0,1,0));
draw((1,1,1) -- (1,1,0));
\end{asy}
\end{figure}

Como sabemos o paralelepípedo retângulo é determinado pelo conhecimento das medidas de suas três \emph{dimensões} indicadas na figura por \(a\), \(b\) e \(c\).

\begin{figure}[H]
\centering

\begin{asy}
size(7.5cm);
currentprojection=orthographic(1,2,.5);

draw(surface((0,0,0) -- (2,0,0) -- (2,0,1) -- (0,0,1) -- cycle), azul*80+opacity(0.65));
draw(surface((0,0,0) -- (2,0,0) -- (2,1,0) -- (0,1,0) -- cycle), azul*80+opacity(0.65));
draw(surface((0,0,1) -- (2,0,1) -- (2,1,1) -- (0,1,1) -- cycle), azul*80+opacity(0.65));
draw(surface((0,1,0) -- (0,1,1) -- (2,1,1) -- (2,1,0) -- cycle), azul*80+opacity(0.65));
draw(surface((0,0,0) -- (0,0,1) -- (0,1,1) -- (0,1,0) -- cycle), azul*80+opacity(0.65));
draw(surface((0,1,0) -- (0,1,1) -- (2,1,1) -- (2,1,0) -- cycle), azul*80+opacity(0.65));
draw(surface((2,0,0) -- (2,0,1) -- (2,1,1) -- (2,1,0) -- cycle), azul*80+opacity(0.65));

draw((0,1,0) -- (0,0,0) -- (2,0,0), dashed);
draw((0,0,0) -- (0,0,1), dashed);
draw((0,0,1) -- (2,0,1) -- (2,1,1) -- (0,1,1) -- cycle);
draw((0,1,0) -- (0,1,1));
draw((2,1,0) -- (2,1,1));

draw((2,0,1) -- (2,0,0), verde+linewidth(1.25), L=Label("a", position=MidPoint));
draw((2,0,0) -- (2,1,0), laranja+linewidth(1.25), L=Label("b", position=MidPoint));
draw((2,1,0) -- (0,1,0), vinho+linewidth(1.25), L=Label("c", position=MidPoint));
\end{asy}
\end{figure}

O volume \(V\) de um paralelepípedo depende de suas dimensões, \(a\), \(b\) e \(c.\) Assim, indicaremos \(V\) por \(V(a, b, c)\), ou seja, \(V(a,b,c)\) é o volume do paralelepípedo de dimensões \(a\), \(b\) e \(c\).

Desta forma, o volume do cubo de aresta $1$ é \(V(1, 1, 1) =1\) e o volume de um paralelepípedo retângulo de lados \(a = 2\), \(b = 3\) e \(c=5\) é \(V(2,3,5) = 30\) pois cabe 30 cubos de aresta $1$ no espaço ocupado por esse pararlelepípedo.

\begin{figure}[H]
\centering

\noindent\includegraphics[width=175bp]{{blocos}.png}
\end{figure}

\textbf{Afirmação}: Fixados três números reais positivos \(a\), \(b\) e \(c\). O volume do paralelepípedo retângulo de arestas \(a\), \(b\) e \(c\)  é dado pelo produto \(abc\).

Esta atividade vai justificar que \(V(a, b, c) = abc\) para \(a\), \(b\) e \(c\) números racionais positivos. Ela propõe construções que tratam a subdivisão do cubo unitário, encaminhando para a noção de “infinitamente pequeno”. Esse raciocínio tem um papel essencial na matemática e é importante no desenvolvimento do pensamento humano moderno. Comecemos com uma simples observação:

\paragraph{Parte 1}

Recomendamos o uso \href{https://ggbm.at/yk8bqdvz}{deste aplicativo} para o desenvolvimento da tarefa.
\begin{enumerate}
\item {} 
Seguindo o modelo da figura acima, desenhe um paralelepípedo retângulo cujas arestas sejam \(a=2\), \(b=3\) e \(c=4\) e outro cujas arestas sejam \(a=2\), \(b=4\) e \(c=3\).

\item {} 
Obtenha uma relação entre os volumes \(V(2, 3, 4)\) e \(V(2, 4, 3)\). Explique.

\item {} 
Desenhe um paralelepípedo retângulo cujo volume seja \(V(2, 4, 9)\), mas com arestas diferentes de $2$, $4$ e $9$.

\item {} 
Relacione os volumes \(V(2, 4, 3)\) e \(V(2, 4, 9)\).

\end{enumerate}

\paragraph{Parte 2} Considere um paralelepípedo retângulo de arestas \(x\), \(y\) e \(z\) de volume $12$, isto é, \(V(x, y, z) = 12\).

Recomendamos o uso \href{https://ggbm.at/uq2gd3ub}{deste aplicativo}  para o desenvolvimento desta tarefa.
\begin{enumerate}
\item {} 
Quanto valem \(V(2x, y, z)\), \(V(x, 3y, z)\) e \(V(x, y, 4z)\)? Justifique e faça uma figura para ilustrar cada uma de suas respostas? E  \(V(2x, 3y, z)\)?

\item {} 
Encontre todos os valores inteiros para \(n_1\leq n_2 \leq n_3\) de modo que \(V(n_1 x, n_2 y, n_3 z) = 144\).

\item {} 
Seja \(n\) um número natural. Quanto valem 

\begin{enumerate}
\item\(V(nx, y, z)\)
\item\(V(x, ny, z)\)
\item\(V(x, y, nz)\)
\item\(V(n x, n y, n z)\)
\end{enumerate}

\item {} 
Conclua que se \(a\), \(b\) e \(c\) são números naturais, então

\end{enumerate}
\begin{equation*}
\begin{split}V(a, b, c) = abc V(1,1,1) = abc.\end{split}
\end{equation*}
\paragraph{Parte 3} Caso \(a\), \(b\) e \(c\) sejam números racionais.

Recomendamos que seja usado \href{https://ggbm.at/zzdv6are}{este aplicativo} para nos itens \titem{a)} e \titem{b)}.
\begin{enumerate}
\item {} 
Desenhe o paralelepípedo retângulo de arestas $1$, $1$ e $\dfrac{1}{2}$. Relacione \(V(1,1,1/2)\) e \(V(1,1,1)\). Faça o mesmo para os paralelepípedos de arestas $1$, $1$ e $\dfrac{1}{4}$ e de arestas $1$, $\dfrac{1}{2}$ e 1/2.

\item {} 
Calcule os volumes a seguir. Explique as suas soluções.
\begin{enumerate}
\item {} 
\(V\left(1, 1, \frac{1}{2}\right)\)

\item {} 
\(V\left(1, 1, \frac{1}{7}\right)\)

\item {} 
\(V\left(1, 1, \frac{3}{7}\right)\)

\item {} 
\(V \left(1, 1, \frac{4}{3}\right)\)

\item {} 
\(V \left(1, 1, \frac{11}{17}\right)\)

\end{enumerate}

\item {} 
Explique com suas palavras a igualdade

\end{enumerate}
\begin{equation*}
\begin{split}\displaystyle{V \left( 1,1,\frac{m}{n} \right) = \frac{m}{n}}\end{split}
\end{equation*}
para quaisquer \(m/n\) com \(m\) e \(n\) naturais.

Recomendamos que seja usado \href{https://ggbm.at/zfaaqbr7}{este aplicativo} para nos itens a seguir.
\begin{enumerate}
\item {} 
Calcule os volumes a seguir. Explique as suas soluções e faça figuras para ilustrar a resposta.
\begin{enumerate}
\item {} 
\(V\left(1,\frac{1}{2},1\right)\)

\item {} 
\(V\left(\frac{3}{7}, 1, 1\right)\)

\item {} 
\(V\left(1,\frac{1}{5},\frac{1}{3}\right)\)

\item {} 
\(V\left(\frac{1}{2},\frac{1}{2},\frac{1}{2}\right)\)

\item {} 
\(V\left(\frac{1}{2},\frac{4}{3},\frac{2}{5}\right)\)

\item {} 
\(V\left(\frac{1}{2},\frac{37}{3},\frac{11}{17}\right)\)

\end{enumerate}

\item {} 
Explique a igualdade

\end{enumerate}
\begin{equation*}
\begin{split}V\left(1, \frac{p}{q}, \frac{m}{n}\right) = \frac{pm}{qn}\end{split}
\end{equation*}
para quaisquer números naturais \(p\), \(q\), \(m\) e \(n\) (sugestão: lembre-se que já verificamos que \(V(1, 1, m/n) = m/n\)).

De modo similar você pode explicar que
\begin{equation*}
\begin{split}\displaystyle{V\left(\frac{r}{s}, \frac{p}{q}, \frac{m}{n}\right) = \frac{rpm}{sqn}},\end{split}
\end{equation*}
para quaisquer \(r\), \(s\), \(p\), \(q\), \(m\) e \(n\) naturais.
\end{task}

\begin{observation}

A fórmula \(V(a,b,c) = abc\) explorada na \hyperref[persp1-atividade-3]{Atividade: volume do paralelepípedo retângulo} para \(a\), \(b\) e \(c\) racionais também vale para \(a\), \(b\) e \(c\) irracionais. Por exemplo, \(V(1,1,\pi) = \pi\), \(V(1,1,\sqrt{2}) = \sqrt{2}\), \(V(1,\pi,\sqrt{2} + \sqrt{3}) = (\sqrt{2} + \sqrt{3})\pi\), etc. No entanto, a justificativa extrapola os objetivos do Ensino Médio porque depende de argumentos do Cálculo Diferencial, geralmente estudado nos cursos de exatas na Universidade.
\end{observation}


\arrange{O Conceito de Volume}
\label{\detokenize{GE504-1:organizando-as-ideias-o-conceito-de-volume}}\label{\detokenize{GE504-1::doc}}
Para medir o volume de uma folha de papel, na Atividade: volume de uma folha de papel, foi necessário reconhecer que a folha, além de comprimento e largura, tem espessura, ou seja, é um objeto tridimensional.  De maneira intuitiva, volume diz respeito à quantidade de espaço que um objeto ocupa. Quando observamos uma única folha de papel a espessura parece não ser significativa diante das outras duas dimensões, que ressaltam a área da maior superfície da folha. No entanto, reunindo várias folhas a espessura fica evidente. Ou seja, a espessura da pilha de folhas é a soma das espessuras das folhas.

\begin{figure}[H]
\centering

\noindent\includegraphics[width=300bp]{{11}.png}
\end{figure}

Para calcular o volume de uma folha de papel também foi considerado que o volume da pilha é igual à soma dos volumes das folhas. Essa ideia é característica de medidas como comprimento, área e volume. Assim, o volume (assim como a área e o comprimento) da união de partes disjuntas é igual à soma dos volumes (das áreas e dos comprimentos, respectivamente) dessas partes.

\begin{figure}[H]
\centering

\noindent\includegraphics[width=350bp]{{12}.png}
\end{figure}

\begin{figure}[H]
\centering

\includegraphics[height=100bp]{{13}.png}\hspace{1em}
\includegraphics[height=100bp]{{14}.png}
\end{figure}

Estabelecer uma estratégia não basta para calcular o volume de um objeto tridimensional. É necessária uma unidade de medida para realizar a comparação e exprimir a medida como um número. O volume é comumente expresso em metro cúbico (m\(^3\)), seus submúltiplos (dm\(^3\), cm\(^3\)) ou em litro (l ou L). A escolha da unidade está relacionada ao que se quer medir e à quantidade medida. Por exemplo, para medir a folha de papel a unidade usada foi cm$^3$. Já para abastecer um carro com GNV usa-se metros cúbicos (m\(^3\)) e com gasolina, no Brasil, usa-se litro.

\begin{knowledge}

Outra propriedade importante do volume é que , em diversos casos, um mesmo material pode assumir diferentes formas sem que se altere o volume, como nos casos da massinha de modelar, da argila e da água. Um fato interessante a este respeito é que crianças de até 7 anos não têm esta noção clara, enquanto que crianças maiores de 9 anos já a possuem de forma bastante intuitiva.

\begin{figure}[H]
\centering

\noindent\includegraphics[width=250bp]{{15}.png}
\end{figure}

Teste de Piaget (\url{https://www.youtube.com/watch?v=h9ioMR8C9GI}) para assistir ao vídeo (opção de legenda em português traduzida automaticamente. Este é Teste de Piaget sobre conservação. Ginsburg, H. \& Opper, S. (1969). Piaget’s theory of intellectual development. Eaglewood Cliffs, New Jersey: Prentice-Hall, Inc).
\end{knowledge}

Ligado ao conceito de volume está o de capacidade. Por exemplo, quando se diz que o volume de uma xícara é $300$ ml não se está se referindo ao espaço ocupado pela xícara, ou seja, ao volume do objeto xícara, mas à quantidade de líquido que ela comporta, ou seja, à sua capacidade. Capacidade se refere ao volume de substância (líquido ou gás, por exemplo) que um recipiente pode conter e não à quantidade de espaço que o próprio recipiente ocupa. O tanque de combustível de um automóvel, garrafas térmicas, caixas d’água e geladeiras são identificados por sua capacidade e não pelo espaço que ocupam.



\begin{minipage}{.39\linewidth}
\centering
\includegraphics[height=150bp]{{16}.png} \hspace{5cm}
\end{minipage}
\begin{minipage}{.59\linewidth}
\centering
\vfill
\begin{tabular}{|c|c|}
\hline
\tmcol{2}{|c|}{Capacidade} \\
\hline
Capacidade geladeira & 265 litros \\
\hline
Capacidade freezer & 80 litros \\
\hline
Capacidade total de armazenamento & 345 litros \\
\hline
\tmcol{2}{|c|}{Dimensões} \\
\hline
Largura & 61,9 cm \\
\hline
Profundidade & 69 cm \\
\hline
Altura & 176 cm \\
\hline
\tcolor{Peso} & 72 kg \\
\hline
\end{tabular} 
\vfill
\end{minipage}


\begin{figure}[H]
\centering

\includegraphics[height=110bp]{{19}.png} \hspace{5em} \includegraphics[height=110bp]{{18}.png} 
\end{figure}

\begin{figure}[H]
\centering

\includegraphics[height=100bp]{20.png}
\end{figure}


Observamos que, além de volume, área, comprimento, largura e espessura, há outras medidas que podem caracterizar uma folha de papel: a massa e a gramatura. A gramatura exprime uma relação entre duas dessas medidas e permite classificar o papel para seus diversos fins. Gramatura é a razão da massa pela área de um papel, sendo comumente expressa em gramas por metro quadrado (g/m\(\sp{\text{2}}\)).

São diversas as medidas que podem ser observadas e caracterizam materiais, substâncias, corpos e objetos. Nem tudo de que se calcula o volume é sólido como um cubo de madeira, pode ser empilhado como a folha de papel ou tem uma forma “padrão” como uma caixa ou uma vela. Também pode não ser possível medir por acomodação em um recipiente cuja capacidade seja conhecida, como um copo ou um galão. Por exemplo, como calcular o volume de água usada na sua residência? Como quantificar a chuva que cai ao longo de um dia? Como entender a capacidade de um tanque de GNV e o consumo de um automóvel que use esse combustível? Qual o volume de um coração humano? Vamos explorar o assunto.

\begin{figure}[H]
\centering

\noindent\includegraphics[width=400bp]{{21}.png}
\end{figure}


\cleardoublepage
\def\currentcolor{session1}

\begin{objectives}{Loja de material de construção}
{
\begin{itemize}
\item {} 
Aplicar o conceito de unidade (caixa de leite ou cubo de lado 1, que dará origem a uma unidade de medida) para comunicar e comparar volumes. Volume, área, comprimento, litro (e outras unidades de medida volumétrica do SI: \(cm^3\), \(m^3\), etc.)

\item {} 
Analisar o uso de medidas linear, de área e de volume em situações do mundo real.

\item {} 
Aplicar relações entre (área e) volume e outras grandezas em situações cotidiano.

\end{itemize}
}{1}{1}
\end{objectives}


\explore{Dimensão}
\label{\detokenize{GE504-2::doc}}\label{\detokenize{GE504-2:explorando-dimensao}}
\begin{task}{loja de material de construção}



Gelson já construiu a alvenaria do quarto de sua filha, agora precisa cuidar das instalações elétricas, da pintura e do revestimento do piso. Para tanto, Gelson precisa comprar:
\begin{itemize}
\item {} 
Um ar condicionado.

\item {} 
Piso de cerâmica para cobrir o piso do quarto.

\item {} 
50m de fio.

\item {} 
Tinta para pintar as paredes e o teto.

\end{itemize}

Gelson precisa decidir alguns detalhes da compra, como a especificação do ar condicionado adequada ao tamanho do quarto, a quantidade de tinta para pintar as paredes etc. Para isso precisará das medidas do quarto, que são aproximadamente $3{,}60$ m por $4{,}80$ m e $2{,}80$ m de altura.

\paragraph{Parte 1}

Na hora de escolher o ar condicionado, Gelson encontrou na internet uma regrinha simples para identificar o aparelho recomendado: “Multiplique a área do cômodo em metros quadrados por 600 para obter o núḿero de BTU/h adequado ao ambiente.”
\begin{enumerate}
\item {} 
De acordo com a recomendação acima, quantos BTU/h seriam ideais para esse quarto?

\item {} 
Algumas instruções de como comprar ar condicionado usam outra fórmula: “Multiplicar $200$ pelo volume do ambiente em metros cúbicos para obter o número de BTU/h adequados”. Faça o cálculo com esse método.

\item {} 
Explique por que essas duas formas de cálculo têm resultados próximos para cômodos típicos de casas e apartamentos. Indique dimensões possíveis de um ambiente em que essas fórmulas não resultem em valores próximos, gerando dúvida entre comprar aparelho de $45.000$ BTU/h ou $60.000$ BTU/h. Nesse caso, qual fórmula deve ser usada?

\end{enumerate}

\paragraph{Parte 2}

Ao comprar o piso de cerâmica para o quarto, Gelson encontrou três tamanhos com preços parecidos: $30\text{ cm} \times 30\text{ cm}$, $60\text{ cm} \times 60\text{ cm}$ e $1\text{ m} \times 1\text{ m}$ (muitas vezes, esses pisos quadrados são identificados apenas pelo tamanho de um lado, como $30$ cm, $60$ cm e o de $1$ m).
\begin{enumerate}
\item {} 
Quais são as vantagens e desvantagens, em termos da quantidade de trabalho e da dificuldade de instalação, ao se escolher o piso maior (de $1$ m por $1$ m)?

\item {} 
Entre os pisos de 30cm e o de 60cm de lado, qual você entende que daria mais trabalho para instalar na mesma área? Se Gelson escolher a cerâmica de $60$ cm, ele deve ter que assentar aproximadamente quantas peças? E se ele escolher a de $30$ cm? Chamando de $x$ o número de peças (aproximado) que devem ser instaladas de $60$ cm e chamando de $y$ o número de peças de $30$ cm, quanto vale a razão \(\frac{y}{x}\)?

\item {} 
Você saberia encontrar o número \(\frac{y}{x}\) acima sem ter que calcular $x$ e $y$? E se cada peça de cerâmica tivesse $10$ cm por $10$ cm, qual seria a razão do número de peças em comparação a $30$ cm?

\end{enumerate}

\paragraph{Parte 3}

Quanto ao fio, Gelson não sabia qual era a grossura do fio de cobre que ele deveria comprar para as instalações elétricas. Decidiu então medir o diâmetro de um fio que já tinha instalado em seu quarto e obteve aproximadamente 2mm. Na hora de comprar o fio para o quarto da filha, percebeu que a classificação dos fios não era pelo diâmetro, mas pela área da secção do fio, ou seja, em mm\(\sp{\text{2}}\).

\begin{figure}[H]
\centering

\noindent\includegraphics[height=175bp]{{22}.jpg}\hspace{3em}\noindent\includegraphics[height=175bp]{{23}.png}
\end{figure}
\begin{enumerate}
\item {} 
Qual é aproximadamente a bitola (medida de área da seção do fio) em mm\(\sp{\text{2}}\) que Gelson deve comprar para o quarto da sua filha se quiser que seja como o que tem em seu quarto?

\item {} 
Gelson gostaria de saber o peso do fio para decidir como ir buscá-lo. Ele descobriu na internet que a densidade do cobre é $8890$ kg/m\(\sp{\text{3}}\). Desprezando o peso da borracha que reveste o fio, estime o peso da compra de fio que Gelson deve fazer?

\end{enumerate}

\paragraph{Parte 4}

Finalmente, Gelson agora precisa fazer o cálculo da quantidade de tinta. Para isso ele decidiu medir todo o quarto.

\begin{figure}[H]
\centering

\noindent\includegraphics[width=325bp]{{24}.jpg}
\end{figure}

Como já foi dito, o quarto mede aproximadamente $3{,}60$ m por $4{,}80$ m no piso e  tem  $2{,}8$ m de altura. No quarto há ainda uma porta de $72$ cm por $2{,}10$ m e uma janela de $1{,}9$ m por $90$ cm.
\begin{enumerate}
\item {} 
Considerando as informações apresentadas na lata de tinta a seguir e as dimensões do quarto, quantas latas dessa tinta Gelson deve comprar para passar duas demãos de tinta no quarto (as informações da lata se referem a apenas uma demão de tinta)?

\end{enumerate}

\begin{figure}[H]
\centering

\noindent\includegraphics[width=150bp]{{25}.jpg}
\end{figure}
\begin{enumerate}
\setcounter{enumi}{1}
\item {} 
Sabendo que a lata de tinta possui $3{,}6$ litros, estime a espessura da tinta fresca em cada cada demão, considerando o rendimento de uma demão apresentado na lata.

\end{enumerate}

Gelson pensou em decorar o quarto, cobrindo as menores paredes do quarto com papel de parede.
\begin{enumerate}
\setcounter{enumi}{2}
\item {} 
As menores paredes do quarto têm $3{,}60$ m de largura por $2{,}80$ m de altura. Um rolo do papel  que Gelson escolheu para cobrir as paredes tem 60cm de largura e contém $5$ m do papel. Quantos desses rolos devem ser comprados para cobrir uma das paredes menores, a que não possui porta?

\item {} 
E para cobrir a outra parede menor, em que  fica a porta do quarto?

\end{enumerate}
\end{task}

\begin{observation}

Sabe-se que, após a secagem, a espessura da tinta reduz em média para $70\%$ da inicial. Essa porcentagem é chamada razão Sólido por Volume (SV) da tinta e quanto maior ela for, maior é o rendimento da tinta.
\end{observation}


\arrange{Dimensão}
\label{\detokenize{GE504-3:organizando-as-ideias-dimensao}}\label{\detokenize{GE504-3::doc}}
Para decorar o quarto da filha, Gelson precisou determinar várias medidas. Dentre elas o comprimento necessário de fio, a área da parede a ser pintada e o volume da sala para a especificação do aparelho de ar condicionado. Comprimento, área e volume são grandezas que estão relacionadas à ideia de dimensão.

Mas o que é dimensão? Uma definição matematicamente rigorosa para dimensão pode não ser simples, no entanto, a ideia é bastante intuitiva. Vivemos em um mundo tridimensional. No mundo real objetos, corpos e seres são tridimensionais. Vimos que a folha de papel, em que duas dimensões se destacam, é tridimensional. Uma linha, muitas vezes associada apenas ao comprimento, tem espessura. Até um grão de areia é um sólido tridimensional.

A ideia de dimensão está associada à quantidade de informações necessárias para estabelecer a localização de um ponto. Por exemplo, para localizar uma casa em uma rua basta indicar o número dessa casa, ou seja, apenas uma informação ou uma coordenada. Já para localizar um ponto na superfície terrestre são necessárias duas informações: as coordenadas latitude e longitude. Para estabelecer a posição de um drone no espaço são necessárias três informações, além da latitude e da longitude, é preciso conhecer uma terceira coordenada, a altitude. A localização em uma rua, na superfície terrestre e no espaço aéreo são exemplos reais da ideia de dimensão. A rua está associada a ideia de uma dimensão, a superfície terrestre de duas e o espaço à três.

Em geometria, a ideia de dimensão está associada a conceitos elementares: uma linha é unidimensional, um plano é bidimensional e o espaço é tridimensional. Já o ponto é considerado adimensional, ou seja, sem dimensão.

A medida de uma linha, que é unidimensional, é o seu comprimento. Por exemplo, mede-se o comprimento do contorno do quadrado, ou seja, o perímetro do quadrado, De forma análoga, mede-se o comprimento de um segmento ou o comprimento da circunferência.

Já a área é a medida de uma forma bidimensional. Por exemplo, mede-se a área de um quadrado ou de uma forma abstrata como a da Figura XXX . É importante observar que uma linha não tem área,  pois área é uma medida de formas bidimensionais e uma linha é unidimensional.

No entanto, dada uma figura bidimensional é possível calcular medidas de elementos unidimensionais da figura. De um triângulo, por exemplo, calcula-se a área e também o perímetro (FIGURA XX). Já o volume é a medida do espaço ocupado por um objeto. Por exemplo, o volume de uma bola ou de um cilindro. Não se calcula o volume de um triângulo, que é uma forma bidimensional. No entanto, além do volume, é possível se calcular a altura de um paralelepípedo e a a área da superfície que o delimita (Figura YYY).
Assim um objeto é unidimensional (tem dimensão um) quando não têm área nem volume, são iguais a zero, mas tem comprimento diferente de zero. É bidimensional quando tem área diferente de zero, mas seu volume é zero, ou seja, não tem volume. E é tridimensional quando tem volume diferente de zero.

No mundo real, tridimensional, muitas vezes a medida observada é de um atributo unidimensional ou bidimensional dos objetos, corpos ou seres. Por exemplo, um fio elétrico: no momento da compra, é o comprimento, uma grandeza unidimensional, que determina a quantidade a ser adquirida. No entanto, são diferenciados pela área de sua secção reta, uma grandeza bidimensional.

\def\currentcolor{session2}
\begin{objectives}{GNV}
{
\begin{itemize}
\item {} 
Entender o conceito de incompressibilidade, ou seja, que o volume de objetos incompressíveis não se altera após aplicação de pressão ou mudança de forma (conservação).

\item {} 
Aplicar o conceito de unidade (caixa de leite ou cubo de lado 1, que dará origem a uma unidade de medida) para comunicar e comparar volumes. Volume, área, comprimento, litro (e outras unidades de medida volumétrica do SI: cm3, m3, etc.)

\end{itemize}
}{1}{1}
\end{objectives}
\begin{sugestions}{GNV}
{
Esta parte da atividade discute o conceito de compressibilidade de gases. Pretende-se chamar a atenção para o fato de que o volume de alguns materiais pode variar sob diversas condições (como por exemplo temperatura ou pressão). Na situação colocada no problema, o volume de GNV é função da temperatura ambiente e da pressão a que o gás está submetido.

Por outro lado, a maioria dos líquidos ou sólidos sofrem pouca ou nenhuma alteração perceptível em seus volumes quando submetidos a pequenas variações de temperatura ou pressão, esses são os materiais com que trabalhamos na maior parte desta Unidade.

\textbf{Organização em sala de aula:}
Especialmente se sua turma possuir mais de 20 estudantes, recomenda-se que os estudantes estejam dispostos em grupos de 4 ou 5 para que argumentem uns com os outros. Recomenda-se o estabelecimento de uma dinâmica de discussão no grupo. O fechamento da atividade está no Para refletir, é importante discuti-lo com os estudantes.
}{1}{1}
\end{sugestions}
\begin{answer}{GNV}
{
\begin{enumerate}
\item {} 
O valor de 16m\(\sp{\text{3}}\) não corresponde ao volume ocupado pelo tanque, mas sim à capacidade de armazenamento de GNV no tanque nas condições estipuladas pelo órgão de regulação (ANP). O tanque comporta uma capacidade tão grande de GNV porque o gás fica comprimido em seu interior.

\item {} 
Alguns dos motivos mais prováveis:

\begin{itemize}
\item {} 
O gás no tanque está submetido a maior pressão do que a indicada pela norma (220 bar) possibilitando uma capacidade maior no mesmo tanque.

\item {} 
A temperatura ambiente está menor do que convencionada para o estabelecimento da capacidade do tanque (13 ou 21º C).

\item {} 
A capacidade real do tanque é um pouco maior do que a capacidade nominal com que ela foi vendida.

\item {} 
A bomba não está bem calibrada.

\item {} 
O frentista ou o posto estão trapaceando.

\end{itemize}

\item {} 
A capacidade de água no tanque é de 29,7 litros = 0,0297m\(\sp{\text{3}}\) de água, muito menos do que a capacidade de 7,5m\(\sp{\text{3}}\) de GNV no tanque. Esta diferença de capacidades se dá pelo fato da água ser incompressível, já o GNV, como a maioria dos gases, é compressível assim o seu volume não é um valor absoluto, mas uma função das condições de temperatura e pressão do ambiente.

\end{enumerate}
}{1}
\end{answer}
\clearmargin
\begin{answer}{GNV}
{
\begin{enumerate}\setcounter{enumi}{3}
\item {} 
Os gases, assim como outros materiais, tendem a se comprimir quando têm sua temperatura reduzida. Isto explica o fenômeno. Por questões de segurança, em dias de baixa temperatura, o tanque deve ser abastecido com menos gás, de modo a manter a pressão mais baixa no tanque para que quando a temperatura subir, a pressão não ultrapasse o valor especificado pelas normas técnicas.

\item {} 
O volume de um gás, em geral, é estabelecido em condições normais de temperatura e pressão e varia quando as condições de temperatura ou de pressão se alteram. No caso do GNV, o tanque possui capacidade de 15m\(\sp{\text{3}}\) quando submetido a uma pressão de 220bar e a uma temperatura de 21ºC. Fora destas condições, o volume pode se alterar.
\end{enumerate}
}{1}
\end{answer}
\clearmargin
\begin{objectives}{Índice pluviométrico}
{
\begin{itemize}
\item {} 
Aplicar o conceito de unidade (caixa de leite ou cubo de lado 1, que dará origem a uma unidade de medida) para comunicar e comparar volumes. Volume, área, comprimento, litro (e outras unidades de medida volumétrica do SI: \(cm^3\), \(m^3\), etc.)

\item {} 
Analisar o uso de medidas linear, de área e de volume em situações do mundo real.

\item {} 
Aplicar relações entre (área e) volume e outras grandezas em situações cotidiano.

\end{itemize}
}{1}{2}
\end{objectives}
\begin{sugestions}{Índice pluviométrico}
{
Esta atividade, tem como objetivos explorar medidas em situações do mundo real e estabelecer relação entre volume e outras grandezas no cotidiano.

Para compreender o que é índice pluviométrico é preciso entender que não cabe medir a chuva em litros. Não há como coletar toda a chuva e medir o volume. A quantidade de chuva em determinada região e em determinado período é obtida a partir de uma medida linear.

A atividade depende da discussão de ideias entre os alunos. Portanto, recomendamos que inicialmente os alunos sejam organizados em grupos com 3 ou 4 alunos e que, após um período de reflexão, seja conduzida uma discussão geral visando ao fechamento dos conceitos e ideais tratados.

Os itens \titem{a)} e \titem{b)} têm como objetivo levar os alunos a perceberem a relação entre o volume de água de chuva coletada e a forma dos recipientes. Em particular, que percebam que, quanto maior a área de coleta, maior volume de água será armazenado. E que o nível que a água alcança em cada recipiente depende da sua forma.

Os demais itens têm como objetivo verificar a compreensão do aluno sobre o que é e como se mede o índice pluviométrico.
}{1}{2}
\end{sugestions}
\clearmargin
\begin{sugestions}{Índice pluviométrico}
{
Avalie a possibilidade de realizar a experiência descrita no item (a), ou seja, colocar recipientes de vidro ou plástico transparente (que podem ser copos) de formatos diversos na chuva para verificar o nível da água após um período estabelecido de tempo. Realizar a experiência enriquecerá muito a atividade. Observe que, nesse caso, é importante que os recipientes tenham formatos variados e que pelo menos dois sejam prismas de bases diferentes.

No item \titem{c)}, a imagem que apresenta as etapas de construção de um pluviômetro artesanal a partir de uma garrafa de refrigerante é bastante ilustrativa. Também não é difícil obter tutoriais na internet. Por exemplo, em XXXXX. Se houver oportunidade de realizar a construção com seus alunos, recomendamos. Lembramos que é importante verificar a previsão de chuva na região da sua escola para que a construção seja adequadamente aproveitada.
}{1}{2}
\end{sugestions}
\clearmargin
\begin{answer}{Índice pluviométrico}
{
\begin{enumerate}
\item {} 
Não. O Volume de chuva coletado em cada recipiente depende da área aberta no recipiente, ou seja, a área de captação. Quanto maior for essa área, maior será o volume de água coletado.

\item {} 
Recipiente 1: Igual (ou muito próximo); Recipiente 2: Igual (ou muito próximo); Recipiente 3: maior; Recipiente 4: Menor. (Recipiente para oficinas com professores:  Nâo é possível responder sem saber os raios das secções e do recipiente)

\item {} 
Sim, o índice pluviométrico não é exatamente uma medida de volume, mas uma medida linear que permite estimar a quantidade de chuva em determinada região em um dado período de tempo.
\end{enumerate}
}{1}
\end{answer}
\clearmargin
\begin{answer}{Índice pluviométrico}
{
\begin{enumerate}\setcounter{enumi}{3}
\item {} 
Como a base da garrafa não é plana, as pedras ajudam a preencher o fundo da garrafa. A faixa com a escala é então colocada a partir das pedras. A variação da água é observada a partir de um ponto que não corresponde ao fundo, mas que está na parte cilíndrica da garrafa. Observe que no lugar de pedras poderiam ser colocadas bolinhas de gude, areia ou mesmo nada. Se nada fosse colocado, dever-se-ia  esperar que a água de chuva coletada alcançasse a marca zero da escala, o que demoraria mais.

\item {} 
Como o primeiro coletor é cilíndrico, alturas iguais correspondem a volumes iguais, portanto a escala pode ser com subdivisões iguais para representar uma unidade. Já o segundo coletor tem a forma de um tronco de cone.Nesse caso, alturas iguais não correspondem a volumes iguais. A escala precisa ser adequada a essa variação.

\item {} 
Sabemos que \(1\text{ L} = 1\text{ dm}^3 = 0{,}001\text{ m}^3\). Portanto, como a base da caixa tem \(1\text{ m}^2\), \(1\text{ L}\) de água nessa caixa alcançará \(0{,}001\text{ m} = 1\text{ m}\).

\end{enumerate}

A completar. Item f
}{1}
\end{answer}
\clearmargin
\begin{objectives}{A coelha e o cervo}
{
\begin{itemize}
\item {} 
Reconhecer dimensões de segmentos de reta, regiões planas e sólidos no espaço e a relação com suas unidades de medidas (e.g., \(m\), \(m^2\) e \(m^3\)).

\item {} 
Reconhecer a relação entre a dimensão intrínseca e os conceitos de comprimento, área e volume, distinguindo um sólido de sua fronteira.

\end{itemize}

\textbf{Conceitos abordados:} dimensão.
}{1}{2}
\end{objectives}
\begin{sugestions}{A coelha e o cervo}
{
\begin{itemize}
\item {} 
Você, professor, não precisa aplicar todas as questões aqui sugeridas. Dependendo do tempo disponível e da turma, escolhas ou modificações devem ser feitas. Sinta-se livre para fazê-las!

\item {} 
Parece óbvio, mas vale o conselho: sempre assista ao vídeo antes de trabalhar com ele em sala de aula.

\item {} 
Antes dos alunos assistirem ao vídeo, sugerimos que eles leiam as questões que serão trabalhadas.

\item {} 
Nossa experiência mostra que os alunos ficam sempre mais motivados quando as atividades desenvolvidas fazem parte do sistema de avaliação.

\end{itemize}
}{1}{2}
\end{sugestions}
\begin{sugestions}{A coelha e o cervo}
{
\textbf{Enriquecimento da discussão:}
\begin{itemize}
\item {} 
Este curta-metragem já ganhou mais de 100 prêmios internacionais.

\item {} 
No vídeo, o cubo colorido que aparece várias vezes é conhecido como o cubo mágico ou cubo de Rubik. Este quebra-cabeça 3D foi inventado em 1974 pelo escultor e professor de arquitetura húngaro Ernő Rubik. Resolvê-lo consiste em deixar cada uma de suas seis faces com uma única cor. Para isto, o usuário pode girar seus mecanismos. O matemático português Rogério Martins fala um pouco mais sobre o cubo mágico no vídeo \url{https://goo.gl/eQhDXo} da série Isto é Matemática. Uma curiosidade: existem  43 252 003 274 489 856 000 posições diferentes para o cubo de Rubik (Bandelow, 1982). Uma versão interativa virtual do cubo de Rubik que pode ser executada em dispositivos modernos (incluindo smartphones e tablets) pode ser encontrada em \url{http://goo.gl/sc2qUL}.
\begin{quote}

Figura: Ernő Rubik (1944-)
Fonte: Wikimedia Commons.

Figura: Cubo mágico.
Fonte: Wikimedia Commons.
\end{quote}

\item {} 
Segundo o diretor Peter Vacz, em seu blog \url{http://vaczpeter.blogspot.com}, os protagonistas da animação foram inspirados nele mesmo (que segundo um amigo se parecia com um cervo) e em sua ex-namorada (que se parecia com uma coelha). Vacz começou a fazer ilustrações com esses dois animais com base nos momentos que compartilharam juntos e, então, percebeu que o que tornou os personagens tão especiais foram seus momentos felizes e suas brigas tolas, cenas de sua vida cotidiana.

\item {} 
Existem várias outras animações que tratam da questão da dimensão e que podem ser exibidas junto com “A Coelha e O Cervo”:  “Homer” do seriado “Os Simpsons”,   “2-D Blacktop” do seriado “Futurama”, “Planolândia  - O Filme”  e  “Dimensões” (\url{http://goo.gl/dgYi6S}).

\end{itemize}

\textbf{Link para o vídeo:} \url{https://www.youtube.com/watch?v=\_IEvklgjC-U}. Tem apenas 16‘25’‘. Página web oficial: \url{http://www.rabbitanddeer.com}.

É necessário que os estudantes assistam ao vídeo no link da atividade, então ou eles precisarão ter assistido em casa, ou ou professor pode projetar o filme no quadro.
}{1}{1}
\end{sugestions}

\begin{knowledge}

Em diversas áreas das ciências são necessárias mais do que três dimensões para que sejam descritos alguns fenômenos. Assistam ao vídeo

\href{https://www.youtube.com/watch?v=4TnMMdT3VGw}{Tudo é Matemática T05E07:  A Quarta Dimensão}

\begin{figure}[H]
\centering

\noindent\includegraphics[width=.45\linewidth]{{26}.png}
\noindent\includegraphics[width=.45\linewidth]{{27}.png}

\noindent\includegraphics[width=.45\linewidth]{{28}.png}
\noindent\includegraphics[width=.45\linewidth]{{29}.png}
\end{figure}
\end{knowledge}


\practice{Dimensão}
\label{\detokenize{GE504-4::doc}}\label{\detokenize{GE504-4:praticando}}
\begin{task}{GNV}



A frota de veículos movido a Gás Natural Veicular (GNV) no Brasil no início de 2017 era de 1.859.300 veículos (Fonte: Instituto Brasileiro de Petróleo, Gás e Biocombustíveis - IBP). Metade desta frota está no Rio de Janeiro, seguido por São Paulo com $21{,}5\%$. Isto significa que aproximadamente $4{,}5\%$ dos automóveis, picapes ou caminhonetes do país usam GNV (Fontes: \href{http://www.fecombustiveis.org.br/relatorios/relatorio-anual-da-revenda-de-combustiveis-2017/}{Relatório Anual de Revenda de Combustíveis 2017} e \href{https://g1.globo.com/carros/noticia/frota-brasileira-de-veiculos-cresce-12-em-2017-diz-sindipecas.ghtml}{G1 automóveis}, para o tamanho da frota). Esta é a terceira maior frota de veículos movidos a GNV do mundo (carece de fontes confiáveis).

Os instaladores do kit gás nos veículos anunciam os tanques com capacidades variadas como $7{,}5$ m\(\sp{3}\), $14$ m\(\sp{3}\), $15$ m\(\sp{3}\), $15{,}5$ m\(\sp{3}\) e $16$ m\(\sp{3}\). Os tanques são geralmente posicionados no porta malas do carro e é necessário conferir o modelo do carro para saber se o botijão vendido cabe no porta malas.

\begin{observationtitle}{Observação matemática}

Lembre-se que $1$ metro cúbico ($1$ m\(\sp{3}\)) é o volume de uma caixa na forma de um cubo de lado $1$ metro (veja  a figura $1$). Deste modo, $16$ m\(\sp{3}\) correspondem a $16$ caixas deste tamanho (veja a figura 2).

\begin{table}[H]
\centering
\begin{tabular}{|c|c|}
\hline
Figura 1 & Figura 2\\
\hline
\end{tabular}
\end{table}
\end{observationtitle}

\begin{enumerate}
\item {} 
Como você explica tantos carros pequenos com tanques para $16$ m\(\sp{3}\) de gás em seus portas malas, levando em consideração a observação matemática?

\item {} 
Outra situação comum para os usuários de GNV é que ao encher o tanque, o volume apresentado na bomba do posto (em metros cúbicos), ultrapassa a capacidade nominal do tanque (veja \href{https://br.answers.yahoo.com/question/index?qid=20120927220946AAlMbWD}{aqui} ou \href{https://br.answers.yahoo.com/question/index?qid=20061006192226AAvKvOl}{aqui}), levando as pessoas a desconfiarem do posto em que abastecem. Após refletir um pouco, apresente os motivos mais prováveis, em sua opinião, para esta aparente contradição.


\begin{figure}[H]
\centering

\noindent\includegraphics[width=200bp]{{30}.png}
\end{figure}

\item {} 
Num tanque de GNV vendido como de $7{,}5$ m\(\sp{3}\) (veja a figura a seguir), está especificado $29{,}7$ litros. Na oficina de instalação, explicam que $29{,}7$ litros é o “volume de água” do tanque e que para obter o volume em metros cúbicos, é necessário dividir a capacidade de água em litros por $4$ para obter a capacidade em metros cúbicos de GNV. Compare os volumes de gás e de água no tanque, busque argumentar o significado desta diferença e aparente contradição.


\begin{figure}[H]
\centering

\noindent\includegraphics[width=300bp]{{31}.png}
\end{figure}

\item {} 
É muito comum, por exemplo, em fóruns de discussão e em blogs na internet (veja a discussão no \href{https://www.youtube.com/watch?v=i5QJ0C-qXjw}{vídeo} ou no \href{https://br.answers.yahoo.com/question/index?qid=20120927220946AAlMbWD}{fórum do Yahoo}) as pessoas notificarem que em dias frios, cabe mais GNV no tanque. Como isso é possível?

\item {} 
Levando em consideração toda a discussão desta parte da atividade, tente explicar qual é o significado da capacidade do tanque do combustível ser, digamos, $15$ m\(\sp{3}\). Quais são os fatores relevantes para que o tanque realmente contenha $15$ m\(\sp{3}\) de GNV quando completo?

\end{enumerate}
\end{task}

\begin{knowledge}

Combustíveis fósseis como diesel, gasolina, etanol e GNV emitem CO\(_2\) na atmosfera quando queimados no motor dos veículos. A emissão deste gás na atmosfera é tema de discussões e tratados internacionais (Por exemplo o protocolo de Kyoto de 1997) devido ao seu potencial causador do efeito estufa. Nestes tratados é comum que os países se comprometam a reduzir as emissões de gás carbônico em um dado intervalo de tempo.

Veículos híbridos, em que um motor elétrico auxilia um motor a gasolina, reduzem a aproximadamente $92$ gramas de CO\(_2\) por quilômetro rodado (aproximadamente $82\%$ do que emite um veículo movido a GNV) e veículos inteiramente elétricos não emitem gás carbônico pois seus motores não funcionam a base de combustão.
\end{knowledge}

\begin{reflection}

Conforme visto na Atividade: GNV, o volume de alguns materiais pode ser alterado consideravelmente devido a variações nas condições de temperatura e pressão. Isto é especialmente fácil de se verificar para gases.

A Lei dos Gases Ideais afirma que para um gás ideal em um sistema isolado as grandezas $P$, $V$ e $T$ (respectivamente pressão, volume e temperatura) satisfazem
\begin{equation*}
\begin{split}PV=nRT,\end{split}
\end{equation*}
onde $n$ e $R$ são constantes do sistema. Assim
\begin{observationtitle}{Transformação isobárica} 
Se mantivermos a pressão constante, \textbf{o volume torna-se diretamente proporcional à temperatura}: \(V=\frac{nR}{P}T\), ou seja, existe uma constante k (neste caso \(k=\frac{nR}{P}\)), tal que
\begin{equation*}
\begin{split}V=kT\end{split}
\end{equation*}
\end{observationtitle}
Isto significa que, nesse tipo de sistema, se a temperatura for multiplicada por algum valor, o volume será multiplicado pelo mesmo valor.

Em dias em que a temperatura ambiente está baixa ($T$ \(\downarrow\)), usando a mesma pressão coloca-se mais GNV no tanque pois o volume por ele ocupado é menor ($V$ \(\downarrow\)).
\begin{observationtitle}{Transformação isotérmica} 
Se mantivermos a temperatura constante, o volume torna-se inversamente proporcional à pressão: $V = nRT / P$, ou seja, existe uma constante $k’$ tal que
\begin{equation*}
\begin{split}V=\frac{k'}{P}\end{split}
\end{equation*}
\end{observationtitle}
Isto significa, por exemplo, que se aumentarmos a pressão o volume diminui à mesma taxa.

Quando coloca-se o GNV no tanque é exercida uma pressão sobre o gás ($P$ \(\uparrow\)) de modo que o volume por ele ocupado fica bastante reduzido ($V$ \(\downarrow\))como se pode ver na Atividade: GNV. Na prática, você perceberá que o tanque também aquece um pouco.

Existe também a transformação isovolumétrica, em que o volume é mantido constante e variam a pressão e a temperatura. Este é aproximadamente o caso da panela de pressão em que aumenta-se a pressão do interior da panela para que o alimento, com volume constante, tenha sua temperatura também aumentada (em relação à panela aberta). Imagine o que acontece quando a panela é aberta antes da hora: a pressão baixa muito de forma abrupta ($P$ \(\downarrow\)), a temperatura varia muito pouco, então o volume aumenta muito ($V$ \(\uparrow\)), também de forma abrupta, o que provoca uma espécie de explosão na cozinha (não tente reproduzir isso em casa! Perigo de morte!).
\end{reflection}

\begin{task}{índice pluviométrico}



\textbf{Você sabe o que é índice pluviométrico? O que esse índice mede?}

Medir a quantidade de chuva é importante para a agricultura, influenciando, por exemplo, a decisão do quê e quando plantar.  Também é importante para avaliar a necessidade de medidas que possam evitar tragédias determinadas por grandes quantidades de chuva, como enchentes ou deslizamentos.

\emph{O que significa dizer que, em determinada região, em determinado período, choveu $30$ mm?}

Pode parecer estranho medir chuva como comprimento: “choveu $5$ mm”. Afinal, chuva é água! Mas, de fato, considerando um determinado período de tempo, é uma medida linear que permite quantificar a chuva que cai em dada região. O índice pluviométrico mede a quantidade a chuva em unidades de comprimento por unidade de tempo \textendash{} por exemplo, em milímetros por hora. O “comprimento” corresponde ao “nível” de água da chuva que se acumulara em uma superfície plana, horizontal e impermeável durante um determinado período de tempo, por exemplo uma hora. Essa forma de medir não considera a chuva que escorre ou se infiltra no solo, apenas a chuva que cai.

Na prática, essa medida é feita com um aparelho próprio, o pluviômetro. Há vários modelos diferentes, mas o instrumento constitui-se, basicamente de recipiente de captação e de um recurso para medir o volume coletado de água. Para chegar ao índice pluviométrico de um determinada região (estado ou cidade, por exemplo) em um determinado período, há diversas estações meteorológicas espalhadas, cada uma com o seu pluviômetro. Com base nos dados coletados por essas estações é possível chegar à média da precipitação observada na região. Essa média é o índice pluviométrico da região. Assim, a informação de que choveu, por exemplo, 5 milímetros na cidade ao longo do dia, significa que essa é a altura média alcançada pela água a partir do chão, na área total da cidade ao longo desse dia, se não houvesse escoamento ou infiltração no solo.

\begin{figure}[H]
\centering

\noindent\includegraphics[height=.45\linewidth]{{32}.png} \hfill
\noindent\includegraphics[height=.45\linewidth]{{33}.png}
\end{figure}
\begin{enumerate}
\item {} 
Enquanto chovia, durante uma hora, foram colocados cinco recipientes lado a lado para coletar a água da chuva. O volume de água coletado nos cinco recipientes é o mesmo? Explique a sua resposta. De que característica dos recipientes depende a quantidade de água que será coletada em cada um?


\begin{multicols}{2}

\begin{figure}[H]
\centering
\capstart

\noindent\includegraphics[height=100bp]{{34}.png}
\caption{Recipiente 1 (Cúbico)}\label{\detokenize{GE504-4:id8}}\end{figure}

\begin{figure}[H]
\centering
\capstart

\noindent\includegraphics[height=100bp]{{35}.png}
\caption{Recipiente 2 (Cilindrico)}\label{\detokenize{GE504-4:id9}}\end{figure}

\begin{figure}[H]
\centering
\capstart

\noindent\includegraphics[height=100bp]{{36}.png}
\caption{Recipiente 3 (Cone)}\label{\detokenize{GE504-4:id10}}\end{figure}

\begin{figure}[H]
\centering
\capstart

\noindent\includegraphics[height=100bp]{{37}.png}
\caption{Recipiente para oficinas com professores}\label{\detokenize{GE504-4:id11}}\end{figure}
\end{multicols}

\item {} 
Considerando que o índice pluviométrico da chuva no local em que os recipientes foram colocados foi de $35$ mm em uma hora e que os recipientes ficaram por esse período na chuva, avalie o nível de água em cada um deles: será igual (ou muito próximo), maior ou menor do que $35$ mm? Explique.

\item {} 
Nos recipientes 1 e 2, a chuva coletada alcançará o mesmo nível. No entanto, o volume pode não ser o mesmo. Explique.

\item {} 
Na construção de um pluviômetro caseiro foi utilizada uma garrafa plástica como ilustra a sequência de imagens a seguir:


\begin{figure}[H]
\centering

\noindent\includegraphics[width=350bp]{{38_1}.png}
\end{figure}

Explique:   
\begin{enumerate}
\item qual o objetivo das pedras e da água colocadas no fundo da garrafa cortada. Se fosse uma garrafa de fundo plano, as pedras seriam necessárias?  
\item por que a faixa com a escala de leitura do nível da chuva coleta é colocada a partir do nível da água colocada com as pedras?
\end{enumerate}

\item {} 
Observe os coletores ilustrados nas figuras a seguir. Um é cilíndrico e o outro tem o formato de um tronco de cone. Explique a diferença entre as escalas de leitura do nível da chuva coletada em função do formato do coletor.

\end{enumerate}

\begin{figure}[H]
\centering

\noindent\includegraphics[height=200bp]{{39}.png}\hspace{5em}
\noindent\includegraphics[height=200bp]{{40}.png}
\end{figure}
\begin{enumerate}
\setcounter{enumi}{5}
\item {} 
Se em uma caixa com base de área igual a \(1\) m\super{2}  for depositado $1$ $\ell$ de água que nível a água alcançará?

\item {} 
Em determinada cidade do sudeste do Brasil, foi registrado que o índice pluviométrico da chuva atingiu $1236$ mm por hora. De acordo com os dados estatísticos apresentados a seguir, essa chuva justificaria que manchete:
\begin{enumerate}
\item {} 
Chuva do final da tarde de ontem confirma a média esperada para o período.

\item {} 
Chuva recorde deixa estragos e desalojados.

\item {} 
A chuva que caiu durante todo o domingo não foi suficiente para atrapalhar o carnaval na cidade.

\end{enumerate}

\end{enumerate}

\begin{figure}[H]
\centering

\noindent\includegraphics[width=420bp]{{41}.png}
\end{figure}
\begin{enumerate}
\setcounter{enumi}{7}
\item {} 
(ENEM 2015 - adaptado) O índice pluviométrico é utilizado para mensurar a precipitação da água da chuva, em milímetros, em determinado período de tempo. Seu cálculo é feito de acordo com o nível de água da chuva acumulada em  \(1\) m$^2$ , ou seja, se o índice for de 10mm, significa que a altura do nível de água acumulada em um tanque aberto, em formato de um cubo com \(1\) m$^2$ de área de base, é de 10mm. Em uma região, após um forte temporal, verificou-se que a quantidade de chuva acumulada em uma lata de formato cilíndrico, com raio 300mm e altura 1 200mm, era de um terço da sua capacidade. (Se necessário, utilize 3,0 como aproximação para \(\pi\)) .


O índice pluviométrico da região, durante o período do temporal, em milímetros, é de

\begin{multicols}{6}
\begin{enumerate}
\item $10{,}8$. 
\item $12{,}0$.  
\item $32{,}4$.  
\item $108{,}0$.  
\item $324{,}0$.  
\item $400{,}0$.
\end{enumerate}
\end{multicols}

\end{enumerate}
\end{task}

\begin{task}{a coelha e o cervo}

\textit{(Atividades desenvolvidas por Hamanda de Aguiar Pereira, André de Carvalho Rapozo sob a orientação do Professor Humberto Bortolossi (UFF).)}

As questões a seguir referem-se ao vídeo “A coelha e o cervo” disponível \href{https://www.youtube.com/watch?v=\_IEvklgjC-U}{neste link}.

\paragraph{Sinopse}

A coelha e o cervo vivem juntos e felizes em um universo plano, até que o cervo fica intrigado com um cubo mágico que aparece em sua TV quando esta se quebra. Com isso, o cervo fica obcecado em descobrir o mundo tridimensional. Um acidente o projeta para este universo e ele então se vê separado de sua amiga coelha. Veja como estes dois personagens resolvem essa situação nesse encantador curta metragem de Péter Vácz.

\begin{figure}[H]
\centering

\noindent\includegraphics[width=\linewidth]{{42434445464748}.png}
\end{figure}

\paragraph{Parte 1 - Questões gerais}
\begin{enumerate}
\item {} 
Na sua opinião, o vídeo quer transmitir alguma mensagem? Qual?

\item {} 
No mundo bidimensional em que vivem a coelha e o cervo no início do vídeo, os personagens passam uns pelos outros, pela frente e por trás dos objetos. Supondo que, mesmo no mundo bidimensional, dois corpos não podem ocupar a mesma posição ao mesmo tempo, isto seria realmente possível em um mundo plano? E passar um braço por sobre o corpo? Como você acha que eles deveriam fazer para passar por alguma coisa que estivesse em seu caminho? E no mundo tridimensional?

\item {} 
No mundo bidimensional em que vivem a coelha e o cervo no início do vídeo, como eles veem um ao outro?

\item {} 
Na animação existem várias cenas com as quais se procura diferenciar características geométricas dos elementos que fazem parte da história quando estes estão em duas e em três dimensões. Destaque algumas destas características.

\item {} 
Após um sonho, o cervo começa uma pesquisa frenética em busca de algo. Qual objeto o instiga a pesquisar? O que ele busca?

\item {} 
Depois que o cervo e a coelha vão para o mundo tridimensional, em uma das cenas, aparece uma borboleta pousada na coelha. No vídeo, você diria que a borboleta está representada mais como um objeto semelhante a coelha bidimensional ou ao cervo tridimensional? Por quê?

\item {} 
Na sua pesquisa, o cervo consultou vários livros e se deparou com um desenho e as letras x, y e z. Por que, na sua opinião, o cineasta decidiu usar essas duas representações nesse ponto da história?

\end{enumerate}

\begin{figure}[H]
\centering

\noindent\includegraphics[width=200bp]{{48_1}.png}
\end{figure}
\begin{enumerate}
\setcounter{enumi}{7}
\item {} 
O que você mais gostou no filme?

\item {} 
Se você fosse o diretor desta animação, você faria algo diferente? O quê?

\end{enumerate}

\paragraph{Parte 2 - Questões específicas} 
\begin{enumerate}
\item {} 
No momento em que a televisão quebra, surge uma imagem na tela (01:47-01:56). Na sua opinião, que objeto o cineasta quis representar?

\item {} 
Em seu sonho, o cervo interage com um quadrado (02:40-02:45). O que você acha que ele está fazendo com o quadrado?  Na sua opinião, qual é o objetivo do cineasta com esta cena?

\end{enumerate}

\begin{figure}[H]
\centering

\noindent\includegraphics[width=200bp]{{49}.png}
\end{figure}
\begin{enumerate}
\setcounter{enumi}{2}
\item {} 
Que figuras começam a surgir do chão depois que o cervo joga o quadrado no chão? Em que elas se transformam? (02:48-02:59)

\item {} 
O que o cervo acha em um dos livros que está estudando? Por que você acha que a letra z está destacada? (03:21-03:41)

\item {} 
No vídeo (04:15-04:19), o cervo desenha um círculo, com um “cervo vitruviano” no seu interior, usando um compasso. Em um mundo bidimensional, seria possível o cervo desenhar um círculo fazendo os movimentos que ele fez com o compasso, como mostra o vídeo? Na sua opinião, como seria possível fazer um desenho circular estando em duas dimensões? Como deveria ser o compasso e quais movimentos seriam possíveis?

\end{enumerate}

\begin{figure}[H]
\centering

\noindent\includegraphics[width=200bp]{{50}.png}
\end{figure}

\begin{figure}[H]
\centering

\noindent\includegraphics[width=200bp]{{51}.png}
\end{figure}
\begin{enumerate}
\setcounter{enumi}{5}
\item {} 
Depois que a bebida cai no computador do cervo, ele recebe uma descarga elétrica e desaparece. Ao reaparecer, qual é o primeiro objeto que ele vê? Que diferenças você consegue ver no cervo antes e depois deste acontecimento? (05:28-06:06)

\end{enumerate}


\end{task}

\cleardoublepage
\def\currentcolor{session1}
\clearmargin
\begin{objectives}{Reconhecimento de elementos}
{
{[}Identificar elementos{]} \textbf{OE10. Reconhecer (identificar e nomear)} elementos básicos da geometria espacial que são necessários para volumes e relacioná-los entre eles (e.g., posições relativas de planos (ver o que é realmente necessário aqui)).  (vértices de sólidos, planos paralelos, perpendicularidade entre reta plano, distância de ponto a plano e retas reversas no espaço)

\textbf{Conceitos abordados:} Reta, plano, posições relativas de retas e retas, retas e planos e, planos e planos, colinearidade, coplanaridade.
}{1}{2}
\end{objectives}
\begin{sugestions}{Reconhecimento de elementos}
{
\textbf{Organização em sala de aula:} Para esta atividade espera-se que os estudantes estejam divididos em grupos de mais do que 2 estudantes (4 ou 5 é o ideal).

\textbf{Dificuldades previstas:} A percepção e a representação de objetos de geometria espacial oferece diversos desafios. Por exemplo, comumente um plano é representado por um retângulo ou um paparalelogramo. Não é raro que os estudantes confundam essas figuras, não reconhecendo que o plano é infinito, por exemplo. O papel do professor nas discussões será fundamental para diferenciar a ideia abstrata de plano de suas representações.

Uma sugestão é que o professor apresente uma folha cortada de forma irregular (com as bordas não retilíneas, por exemplo) e pergunte aos estudantes se a folha ainda poderia representar um plano. Isso deve ajudar o estudante a perceber que o plano é ilimitado, logo se dois planos distintos se intersectam em um ponto, então eles se intersectam em uma reta inteira.

\textbf{Sugestões gerais:} Deixe os materiais concretos que você trouxe à disposição dos estudantes para que os utilizem para desenvolver a sua intuição de reta e plano.

Considere não solicitar que seus estudantes escrevam as suas soluções, mas que a atividade seja um guia de discussões.

Evite deixar o fechamento para o final da atividade. Os estudantes costumam se concentrar por pouco tempo no que você está dizendo, então falar pouco tempo em cada interação pode ajudar na compreensão do que está sendo dito.

\textbf{Materiais necessários:} Canudos ou lápis e folhas de papel à vontade (melhor se houver folhas de cores diferentes)
}{1}{2}
\end{sugestions}
\clearmargin
\clearmargin
\begin{objectives}{Agrupando sólidos}
{
{[}Identificar elementos{]} OE11. Entender (analisar, segundo Van Hiele) os sólidos clássicos por meio de suas propriedades e não apenas por associação e semelhança (visualização, segundo Van Hiele).
}{1}{2}
\end{objectives}
\begin{sugestions}{Agrupando sólidos}
{
\textbf{Organização em sala de aula:} Espera-se que sejam realizadas discussões críticas sobre as classificações dos objetos.

\textbf{Dificuldades previstas:} Os estudantes podem querer juntar os  “corpos redondos” entre si. Não há problema neste tipo de agrupamento.

\textbf{Sugestões gerais:} Os estudantes não precisam realmente lembrar os nomes dos sólidos aqui apresentados para desenvolverem a atividade. Espera-se aqui que eles realmente sejam criativos e observadores sobre as características comuns aos objetos.
}{1}{2}
\end{sugestions}


\explore{Elementos de Geometria Espacial}
\label{\detokenize{GE504-5:explorando-elementos-de-geometria-espacial-e-volumes}}\label{\detokenize{GE504-5::doc}}
Nesta seção, exploraremos elementos básicos de geometria espacial e suas relações com os objetivos de estimular a percepção espacial e de construir a linguagem necessária para estudar alguns sólidos clássicos.

\begin{task}{motivação}

\paragraph{Parte 1}

Observe as figuras a seguir e decida se, em cada uma delas, os segmentos destacados em vermelho têm o mesmo comprimento. Explique a sua resposta.

\begin{figure}[H]
\centering

\begin{tikzpicture}[scale=1]


\draw [fill=\currentcolor!50](0,0) rectangle (2,4.5);
\draw (0,1.5) -- (2,1.5);
\draw (0,3) -- (2,3);

\draw [fill = \currentcolor!50] (4,1.4) rectangle (5,3.1);
\draw (4,1.9666666) -- (5,1.9666666);
\draw (4,2.533332) -- (5,2.533332);

\draw [fill=\currentcolor!80] (2,4.5) -- (4,3.1) -- (4,1.4) -- (2,0) -- cycle;

\draw (2,1.5) -- (4,1.9666666);
\draw (2,3) --  (4,2.533332);

\draw [destacado, thick] (4,3.1) -- (4,1.4);
\draw [destacado, thick] (2,1.5)-- (2,3);
\end{tikzpicture}\hspace{5em}
\begin{tikzpicture}[scale=1]


\draw [fill=\currentcolor!50](0,0) rectangle (2,4.5);
\draw (0,1.5) -- (2,1.5);
\draw (0,3) -- (2,3);

\draw [fill = \currentcolor!50] (4,1.4) rectangle (5,3.1);
\draw (4,1.9666666) -- (5,1.9666666);
\draw (4,2.533332) -- (5,2.533332);

\draw [fill=\currentcolor!50] (2,4.5) -- (4,3.1) -- (4,1.4) -- (2,0) -- cycle;

\draw (2,1.5) -- (4,1.9666666);
\draw (2,3) --  (4,2.533332);

\draw [destacado, thick] (4,3.1) -- (4,1.4);
\draw [destacado, thick] (2,1.5)-- (2,3);
\end{tikzpicture}\end{figure}

\paragraph{Parte 2}

A figura sugere a imagem de um cubo do qual um pedaço foi retirado gerando uma superfície plana circular.

\begin{figure}[H]
\centering

\noindent\includegraphics[width=175bp]{{53}.png}
\end{figure}

Isso é possível? Ou seja, é possível retirar um pedaço de um cubo por meio de um único corte, como sugere a figura, gerando uma superfície plana circular? Argumente para justificar a sua resposta ao item anterior a partir da relação entre os pontos os \(A\), \(B\) e \(C\), que, na figura, estão na intersecção da face do cubo e da superfície plana circular.
\end{task}

\begin{task}{reconhecimento de elementos}



\paragraph{Parte 1}

Para responder às perguntas, procure imaginar pontos, retas e planos no espaço. Se necessário, use desenhos ou material concreto, tais como folhas de papel, para representar planos, e lápis, canetas ou canudos para representar retas. Lembre-se: pontos são adimensionais, retas unidimensionais, planos bidimensionais e o espaço tridimensional.
\begin{enumerate}
\item {} 
Considere um ponto \(A\) no espaço. Quantas retas no espaço contêm \(A\)?

\item {} 
Existe reta no espaço que não contenha \(A\)? Se sim, quantas?

\item {} 
Considere agora dois pontos distintos \(A\) e \(B\) no espaço. Quantas são as retas que contêm \(A\) e \(B\)?

\item {} 
Considere agora duas retas \(r\) e \(s\) paralelas. Existe algum plano no espaço que contenha \(r\) e contenha \(s\), ou seja, essas retas são coplanares?

\item {} 
Considere duas retas \(r\) e \(t\) no espaço tais que \(r\) e \(t\) não têm ponto comum, ou seja, não se intersectam. As retas \(r\) e \(t\) são necessariamente paralelas? Explique a sua resposta. Pode ser com um desenho.

\item {} 
As retas \(r\) e \(t\) do item anterior são coplanares?

\item {} 
Explique por que dadas duas retas distintas no espaço, elas necessariamente são concorrentes, paralelas ou reversas (ou seja, não coplanares).

\end{enumerate}

\paragraph{Parte 2}

Considere três pontos \(A\), \(B\) e \(C\) no espaço. Suponha que os pontos \(A\), \(B\) e \(C\) não são colineares, isto é, que nenhuma reta contenha todos os três pontos.
\begin{enumerate}
\item {} 
Quantas são as retas que contêm ao menos dois destes pontos?

\item {} 
Estamos considerando três pontos não colineares. Esse pontos são coplanares? Isto é, existe um plano que contenha todos os três pontos?

\item {} 
Existem dois planos diferentes que contenham os mesmos três pontos não colineares, \(A\), \(B\) e \(C\)?

\item {} 
Considere um quarto ponto \(D\) no espaço. Este ponto é necessariamente coplanar com os pontos \(A\), \(B\) e \(C\)?

\item {} 
Em uma folha de papel, faça um desenho que represente quatro pontos não coplanares e faça os segmentos de reta ligando cada dois destes pontos. Busque deixar claro o que está na frente e o que está atrás na sua figura.

\end{enumerate}

\paragraph{Parte 3}

Use \href{https://ggbm.at/ar9et3rv}{este aplicativo} ou folhas de papel para visualizar e responder às perguntas:
\begin{enumerate}
\item {} 
Em quantas regiões um plano divide o espaço?

\item {} 
Quais são as possibilidades para o conjunto interseção de dois planos no espaço?

\item {} 
Em quantas regiões dois planos dividem o espaço?

\item {} 
É possível que a interseção de dois planos seja exatamente um ponto? Por quê?

\end{enumerate}

\paragraph{Parte 4}

Posições relativas de retas e planos.
\begin{enumerate}
\item {} 
É possível que uma reta intersecte um plano em exatamente dois pontos? Por quê?

\item {} 
Pode haver uma reta e um plano que não se intersectam no espaço? Faça uma figura para ilustrar a sua resposta.

\item {} 
Represente por desenho as possíveis posições relativas entre um plano e uma reta no espaço.

\end{enumerate}

Você deve se lembrar que no plano duas retas são perpendiculares quando se intersectam em um ponto e dividem o plano em quatro regiões congruentes (iguais). Precisamos dizer aqui quando uma reta é perpendicular a um plano. Mas antes vejamos se você tem uma boa intuição e já consegue identificar perpendicularismo entre reta e plano.

Em cada um dos casos a seguir diga se a reta \(r\) parece ou não parece perpendicular ao plano \(\alpha\), na sua opinião.

\begin{figure}[H]
\centering

\noindent\includegraphics[width=450bp]{{54555657}.png}
\end{figure}

\begin{observationtitle}{Definição}
Dizemos que uma reta é perpendicular a um plano quando existirem duas retas desse plano que sejam concorrentes e perpendiculares a ela.
\end{observationtitle}

\begin{figure}[H]
\centering

\noindent\includegraphics[width=450bp]{{58596061}.png}
\end{figure}

Observe que nos casos em que a reta \(r\) não é perpendicular ao plano alfa, existe apenas uma reta de \(\alpha\) que é perpendicular à reta \(r\).
\end{task}

\begin{task}{agrupando sólidos}


Descreva três critérios para agrupar os sólidos apresentados nas figuras de modo que a quantidade de diferentes grupos obtidas a partir de cada um deles seja diferente. Qual critério determinou a menor quantidade de grupos?



\begin{enumerate}
\begin{multicols}{4}

\item
\adjustbox{valign=t}
{
\begin{minipage}{3cm}
\begin{asy}
currentprojection=orthographic(2,0.5,1/2);
size(3cm,3cm);

draw(unitcube, azul*80+opacity(0.65));

draw((1,0,1) -- (1,0,0));
draw((1,0,0) -- (1,1,0));
draw((1,1,0) -- (0,1,0));

draw((0,0,0) -- (1,0,0), dashed);
draw((0,0,0) -- (0,1,0), dashed);
draw((0,0,0) -- (0,0,1), dashed);

draw((0,0,1) -- (0,1,1));
draw((0,1,1) -- (1,1,1));
draw((1,1,1) -- (1,0,1));
draw((1,0,1) -- (0,0,1));
draw((0,1,1) -- (0,1,0));
draw((1,1,1) -- (1,1,0));
\end{asy}
\end{minipage}
}

\item 
\adjustbox{valign=t}
{
\begin{minipage}{3cm}
\begin{asy}
size(3cm,3cm);
currentprojection=orthographic(1,2,.5);

draw(surface((0,0,0) -- (2,0,0) -- (2,0,3) -- (0,0,3) -- cycle), azul*80+opacity(0.65));
draw(surface((0,0,0) -- (2,0,0) -- (2,1,0) -- (0,1,0) -- cycle), azul*80+opacity(0.65));
draw(surface((0,0,3) -- (2,0,3) -- (2,1,3) -- (0,1,3) -- cycle), azul*80+opacity(0.65));
draw(surface((0,1,0) -- (0,1,3) -- (2,1,3) -- (2,1,0) -- cycle), azul*80+opacity(0.65));
draw(surface((0,0,0) -- (0,0,3) -- (0,1,3) -- (0,1,0) -- cycle), azul*80+opacity(0.65));
draw(surface((0,1,0) -- (0,1,3) -- (2,1,3) -- (2,1,0) -- cycle), azul*80+opacity(0.65));
draw(surface((2,0,0) -- (2,0,3) -- (2,1,3) -- (2,1,0) -- cycle), azul*80+opacity(0.65));

draw((0,1,0) -- (0,0,0) -- (2,0,0), dashed);
draw((0,0,0) -- (0,0,3), dashed);
draw((0,0,3) -- (2,0,3) -- (2,1,3) -- (0,1,3) -- cycle);
draw((0,1,0) -- (0,1,3));
draw((2,1,0) -- (2,1,3));
\end{asy}
\end{minipage}
}

\item
\adjustbox{valign=t}
{
\begin{minipage}{3cm}
\begin{asy}
size(3cm,3cm);
currentprojection=orthographic(2,0.5,1/2);

draw(surface((0,0,0) -- (0,1,0) -- (1,1,0) -- (1,0,0)-- cycle), azul*80+opacity(0.65));
draw(surface((0,.5,2) -- (0,1.5,2) -- (1,1.5,2) -- (1,.5,2)-- cycle), azul*80+opacity(0.65));

draw(surface((0,0,0) -- (0,.5,2) -- (1,.5,2) -- (1,0,0)-- cycle), azul*80+opacity(0.65));
draw(surface((1,0,0) -- (1,1,0) -- (1,1.5,2) -- (1,.5,2)-- cycle), azul*80+opacity(0.65));
draw(surface((0,0,0) -- (0,1,0) -- (0,1.5,2) -- (0,.5,2)-- cycle), azul*80+opacity(0.65));
draw(surface((0,1,0) -- (0,1.5,2) -- (1,1.5,2) -- (1,1,0)-- cycle), azul*80+opacity(0.65));

draw((0,1,0) -- (0,0,0) -- (1,0,0), dashed);
draw((0,0,0) -- (0,0.5,2), dashed);

draw((0,.5,2) -- (0,1.5,2) -- (1,1.5,2) -- (1,.5,2)-- cycle);
draw((0,1,0) -- (0,1.5,2) -- (1,1.5,2) -- (1,1,0)-- cycle);
draw((1,0,0) -- (1,1,0) -- (1,1.5,2) -- (1,.5,2)-- cycle);
\end{asy}
\end{minipage}
}

\item
\adjustbox{valign=t}
{
\begin{minipage}{3cm}
\begin{asy}
size(3cm,3cm);
currentprojection=orthographic(-1.25,.2,1/2);

draw(surface((0,0,0) -- (1,1,0) -- (0.35796,2.60007,0) -- (-1.03884,2.03884,0) -- (-1.26007,0.64203,0) -- cycle), azul*80+opacity(0.65));
draw(surface((0,0,4) -- (1,1,4) -- (0.35796,2.60007,4) -- (-1.03884,2.03884,4) -- (-1.26007,0.64203,4) -- cycle), azul*80+opacity(0.65));

draw((0,0,4) -- (1,1,4) -- (0.35796,2.60007,4) -- (-1.03884,2.03884,4) -- (-1.26007,0.64203,4) -- cycle);

draw(surface((0,0,4) -- (1,1,4) -- (1,1,0) -- (0,0,0) -- cycle), azul*80+opacity(0.65));
draw((0,0,0) -- (1,1,0), dashed);
draw((1,1,0) -- (1,1,4), dashed);

draw(surface((1,1,0) -- (0.35796,2.60007,0) -- (0.35796,2.60007,4) -- (1,1,4) -- cycle), azul*80+opacity(0.65));
draw((1,1,0) -- (0.35796,2.60007,0), dashed);

draw(surface((0.35796,2.60007,0) -- (-1.03884,2.03884,0) -- (-1.03884,2.03884,4) -- (0.35796,2.60007,4) -- cycle), azul*80+opacity(0.65));
draw((0.35796,2.60007,0) -- (-1.03884,2.03884,0) -- (-1.03884,2.03884,4) -- (0.35796,2.60007,4) -- cycle);

draw(surface((-1.03884,2.03884,0) -- (-1.26007,0.64203,0) -- (-1.26007,0.64203,4) -- (-1.03884,2.03884,4) -- cycle), azul*80+opacity(0.65));
draw((-1.03884,2.03884,0) -- (-1.26007,0.64203,0) -- (-1.26007,0.64203,4) -- (-1.03884,2.03884,4) -- cycle);

draw(surface((0,0,0) -- (-1.26007,0.64203,0) -- (-1.26007,0.64203,4) -- (0,0,4) -- cycle), azul*80+opacity(0.65));
draw((0,0,0) -- (-1.26007,0.64203,0) -- (-1.26007,0.64203,4) -- (0,0,4) -- cycle);
\end{asy}
\end{minipage}
}
\end{multicols}


\begin{multicols}{4}
\item 
\adjustbox{valign=t}
{
\begin{minipage}{3cm}
\begin{asy}
size(3cm,3cm);
currentprojection=orthographic(1/2,6,0.5);

triple a = (0,0,0);
triple b = (1,0,0);
triple c = (1.5,0.8662,0);
triple d = (1,1.73205,0);
triple e = (0,1.73205,0);
triple f = (-.5,0.85502,0);

triple A = (-1,0,3);
triple B = (0,0,3);
triple C = (.5,0.8662,3);
triple D = (0,1.73205,3);
triple E = (-1,1.73205,3);
triple F = (-1.5,0.85502,3);

draw(surface(a -- b -- c -- d -- e -- f -- cycle), azul*80+opacity(0.65));
draw(surface(A -- B -- C -- D -- E -- F -- cycle), azul*80+opacity(0.65));

draw(surface(a -- b -- B -- A -- cycle), azul*80+opacity(0.65));
draw((A --a -- b -- B), dashed);

draw(surface(b -- B -- C -- c -- cycle), azul*80+opacity(0.65));
draw((A -- B -- C -- c));
draw((c -- b), dashed);

draw(surface(c -- d -- D -- C -- cycle), azul*80+opacity(0.65));
draw((c -- d -- D -- C -- cycle));

draw(surface(d -- D -- E -- e -- cycle), azul*80+opacity(0.65));
draw((d -- d -- D -- E -- e -- cycle));

draw(surface(e -- E -- F -- f -- cycle), azul*80+opacity(0.65));
draw((e --E -- F -- f -- cycle));

draw(surface(f -- F -- A -- a -- cycle), azul*80+opacity(0.65));
draw((a -- f), dashed);
draw((A -- F));
\end{asy}
\end{minipage}
}


\item 
\begin{minipage}{3cm}
\begin{asy}
size(3cm,3cm);
currentprojection=orthographic(1/2,6,0.5);


// Draw cylinder
// cylinder(startpoint3d, radius, length, along_this_axis)
triple start = (0,0,0);
real length = 3.5;
real radius = 1;
triple ax = (0,0,1);
revolution r = cylinder(start,radius,length,ax);
draw(surface(r),azul*80+opacity(0.65));
draw(r, black+linewidth(.5));
draw(surface(circle(c=(0,0,0), r=1, Z)), azul*80+opacity(0.65));
draw(surface(circle(c=(0,0,3.5), r=1, normal=Z)), azul*80+opacity(0.65));
\end{asy}
\end{minipage}

\item 
\adjustbox{valign=t}
{
\begin{minipage}{3cm}
\begin{asy}
size(3cm,3cm);
currentprojection=orthographic(2,5,-1);


path3 p = circle(c=(0,0,0), r=1, normal=Z);;

triple extAlong = Z + .5Y;
real h = 3;
draw (p);
draw (surface(p), azul*80+opacity(.5));
draw (surface(shift (h * extAlong) * (p)), azul*80+opacity(.5));
draw (shift (h * extAlong) * (p));
draw (extrude(reverse (p), h * extAlong), azul*80+opacity (.5));
\end{asy}
\end{minipage}
}

\item 
\adjustbox{valign=t}
{
\begin{minipage}{3cm}
\begin{asy}
size(3cm,3cm);
currentprojection=orthographic(2,5,-1);


path p = (0, 0) .. (1, -2) .. (3, 0) .. (3, -1) .. (3, 2) .. (1, 1) .. (1, 0) .. cycle;

triple extAlong = Z + .5Y;
real h = 6;
draw (path3 (p));
draw (surface(p), azul*80+opacity(.5));
draw (surface(shift (h * extAlong) * path3 (p)), azul*80+opacity(.5));
draw (shift (h * extAlong) * path3 (p));
draw (extrude(reverse (p), h * extAlong), azul*80+opacity (.5));
\end{asy}
\end{minipage}
}
\end{multicols}

\begin{multicols}{4}
\item
\adjustbox{valign=t}
{
\begin{minipage}{3cm}
\begin{asy}
size(3cm,3cm);
currentprojection=orthographic(1/2,2,0.5);

triple a = (0,0,0);
triple b = (1,0,0);
triple c = (1,1,0);
triple d = (0,1,0);


triple A = (.5,.5,1.5);

draw(surface(a--b--c--d--cycle), azul*80+opacity(.5));

draw(surface(a--b--A--cycle), azul*80+opacity(.5));
draw(surface(b--c--A--cycle), azul*80+opacity(.5));
draw(surface(c--d--A--cycle), azul*80+opacity(.5));
draw(surface(a--d--A--cycle), azul*80+opacity(.5));

draw(A--b--c--A--d--c);
draw(b--a--d, dashed);
draw(a--A, dashed);
\end{asy}
\end{minipage}
}

\item 
\adjustbox{valign=t}
{
\begin{minipage}{3cm}
\begin{asy}
size(3cm,3cm);
currentprojection=orthographic(3,10,3);

triple a = (0,0,0);
triple b = (1,0,0);
triple c = (.5,.86602,0);
triple d = (0,1,0);

triple A = (.5,0.288675,1.5);

draw(surface(a -- b -- c -- a -- cycle), azul*80+opacity(0.65));
draw(surface(a -- b -- A -- cycle), azul*80+opacity(0.65));
draw(surface(b -- c -- A -- cycle), azul*80+opacity(0.65));
draw(surface(c -- a -- A -- cycle), azul*80+opacity(0.65));

draw((a -- b), dashed);
draw(b -- c);
draw(c -- a);
draw(a -- A -- b -- c -- A);

\end{asy}
\end{minipage}
}

\item 
\adjustbox{valign=t}
{
\begin{minipage}{3cm}
\begin{asy}
size(3cm,3cm);
currentprojection=orthographic(-1.25,.2,1/2);

triple a = (0,0,0);
triple b = (1,1,0);
triple c = (0.35796,2.60007,0);
triple d = (-1.03884,2.03884,0);
triple e = (-1.26007,0.64203,0);

triple A =(2,2,2);

draw(surface(a -- b -- c -- d -- e -- cycle), azul*80+opacity(0.65));

draw(surface(a -- b -- A -- cycle), azul*80+opacity(0.65));
draw(surface(b -- c -- A -- cycle), azul*80+opacity(0.65));
draw(surface(c -- d -- A -- cycle), azul*80+opacity(0.65));
draw(surface(d -- e -- A -- cycle), azul*80+opacity(0.65));
draw(surface(e -- a -- A -- cycle), azul*80+opacity(0.65));

draw(a -- b, dashed);
draw(b -- c, dashed);
draw(b -- A, dashed);
draw(c -- d -- e --a);
draw(c -- A -- d -- e -- A -- a);
\end{asy}
\end{minipage}
}

\item 
\adjustbox{valign=t}
{
\begin{minipage}{3cm}
\begin{asy}
size(3cm,3cm);
currentprojection=orthographic(-1.25,.2,1/2);

triple a = (0,0,0);
triple b = (1,0,0);
triple c = (1.5,0.8662,0);
triple d = (1,1.73205,0);
triple e = (0,1.73205,0);
triple f = (-.5,0.85502,0);

triple A =(2,-1,2);

draw(surface(a -- b -- c -- d -- e -- f -- cycle), azul*80+opacity(0.65));

draw(surface(a -- b -- A -- cycle), azul*80+opacity(0.65));
draw(surface(b -- c -- A -- cycle), azul*80+opacity(0.65));
draw(surface(c -- d -- A -- cycle), azul*80+opacity(0.65));
draw(surface(d -- e -- A -- cycle), azul*80+opacity(0.65));
draw(surface(e -- f -- A -- cycle), azul*80+opacity(0.65));
draw(surface(f -- e -- A -- cycle), azul*80+opacity(0.65));


draw(b -- c, dashed);
draw(c -- d, dashed);
draw(c -- A, dashed);
draw(b -- A, dashed);

draw(d -- e -- f -- a -- b);

draw(d -- A -- e -- f-- A -- a);
\end{asy}
\end{minipage}
}
\end{multicols}

\begin{multicols}{4}
\item 
\adjustbox{valign=t}
{
\begin{minipage}{3cm}
\begin{asy}
size(3cm,3cm);
currentprojection=orthographic(1/2,6,0.5);


// Draw cylinder
// cylinder(startpoint3d, radius, length, along_this_axis)
triple start = (0,0,0);
real length = 3.5;
real radius = 1;
triple ax = (0,0,1);
revolution r = cone(start,radius,length,ax);
draw(surface(r),azul*80+opacity(0.65));
draw(circle(c=start, r=1, Z), black+linewidth(.5));
draw(surface(circle(c=(0,0,0), r=1, Z)), azul*80+opacity(0.65));
draw((0,0,3.5) -- (1,0,0));
draw((0,0,3.5) -- (-1,0,0));
//draw(surface(circle(c=(0,0,3.5), r=1, normal=Z)), azul*80+opacity(0.65));
\end{asy}
\end{minipage}
}

\item 
\adjustbox{valign=t}
{
\begin{minipage}{3cm}
\begin{asy}
size(3cm,3cm);
currentprojection=orthographic(0,7,0.5);


path3 p = circle(c=(0,0,0), r=1, Z);

triple extAlong = Z + .5Y;
real h = 6;
draw (p);
draw (surface(p), azul*80+opacity(.5));
draw (extrude(p, (-1.25,0,3) -- cycle), azul*80+opacity (.5));
draw ((-1.25,0,3) -- (1,0,0));
draw ((-1.25,0,3) -- (-1,0,0));
\end{asy}
\end{minipage}
}

\item 
\adjustbox{valign=t}
{
\begin{minipage}{3cm}
\begin{asy}
size(3cm,3cm);
currentprojection=orthographic(4,7,0.5);

path p = (-2,0) .. (0, -3) .. (3, -2) .. (3, -1) .. (3, 2) .. (1, 1) .. (0, 1) .. cycle;

triple c = (5,4,7);
dot((c), linewidth(.000001));
draw (path3 (p));
draw (surface(path3(p)), azul*80+opacity(.5));
draw (extrude(path3 (p), (c) -- cycle), azul*80+opacity (.5));
\end{asy}
\end{minipage}
}
\end{multicols}
\end{enumerate}


\end{task}


\arrange{Elementos de Geometria Espacial}
\label{\detokenize{GE504-6:organizando-as-ideias-elementos-de-geometria-espacial-e-volumes}}\label{\detokenize{GE504-6::doc}}
As atividades do início desta seção devem tê-lo levado a reconhecer alguns fatos da geometria espacial que são bastante intuitivos como:
\begin{itemize}
\item {} 
por dois pontos passa uma única reta,

\item {} 
por três pontos não colineares passa um único plano,

\item {} 
se uma reta possui dois de seus pontos em um plano, então esta reta está contida no plano,

\item {} 
se dois planos possuem um ponto em comum, então eles possuem uma reta inteira em comum.

\end{itemize}

Assim como as noções de ponto, reta e plano, esses fatos são considerados intuitivos e assumidos como verdadeiros, portanto, não serão justificados. Em uma construção mais rigorosa, afirmações como essas são consideradas como postulados e delas seriam provadas todas as demais afirmações da teoria.

Na \DUrole{xref,std,std-ref}{Atividade: reconhecendo elementos}, você percebeu que duas retas diferentes no espaço podem ser coplanares ou não-coplanares. Coplanares significa que existe um plano que contém as duas retas. Nessa situação, as retas ou são paralelas ou são concorrentes, como você já conhece da Geometria Plana. Naturalmente, duas retas são não-coplanares quando nenhum plano contém as duas. Retas não-coplanares são também chamadas de retas reversas.

As posições relativas de retas e planos podem ser evidenciada na observação de alguns sólidos clássicos. Por exemplo, no cubo \(ABCD-EFGH\) da figura, as retas que contêm os segmentos \(AB\) e \(DE\) são reversas. A reta \(AB\) não tem ponto em comum com o plano determinado por \(E\), \(F\) e \(G\) (atenção que estamos realmente falando da reta \(AB\) e não apenas do segmento de reta \(AB\)).

\begin{figure}[H]
\centering

\begin{asy}
size(5cm,5cm);
currentprojection=orthographic(3,1/2,.5);

draw(unitcube, laranja*80+opacity(0.65));

triple a = (0,0,0);
triple b = (0,1,0);
triple c = (1,1,0);
triple d = (1,0,0);

triple e = (0,0,1);
triple f = (0,1,1);
triple g = (1,1,1);
triple h = (1,0,1);

draw(h -- d);
draw(d -- c);
draw(c -- b);

draw(a -- d, dashed);
draw(a -- b, dashed);
draw(a -- e, dashed);

draw(e -- f);
draw(f -- g);
draw(g -- h);
draw(h -- e);
draw(f -- b);
draw(g -- c);

label ("D", (a), align=NW);
label ("C", (b), align=E);
label ("B", (c), align=S);
label ("A", (d), align=SW);

label ("H", (e), align=NW);
label ("G", (f), align=NE);
label ("F", (g), align=N);
label ("E", (h), align=NW);

\end{asy}
\end{figure}

Nesta situação, dizemos que a reta \(AB\) é paralela ao plano \(EFG\).

\begin{observationtitle}{Definição}
Dizemos que um plano é paralelo a uma reta (ou que uma reta é paralela ao plano) quando não possui ponto em comum com a reta, ou seja, quando eles não se intersectam.
\end{observationtitle}

A reta \(AE\) é perpendicular ao plano \(ABC\). Como \(ABFE\) e \(ADHE\) são quadrados, a reta \(AE\) é perpendicular às retas concorrentes \(AB\) e \(AD\) do plano \(ABC\), logo \(AE\) é perpendicular ao plano \(ABC\).

O primeiro sólido de nossa lista na verdade é uma ampla categoria que inclui quase todos os demais.

\subsection{Cilindro}

Considere um plano alfa e uma região plana \(R\) (veja a figura). Considere um plano alfa’, paralelo a alfa e uma reta s secante aos planos alfa e alfa’. Por cada ponto \(P\) da figura \(R\), tomamos uma reta paralela a \(s\), que intersecta alfa’ em \(P’\). A união dos pontos \(P’\) assim definidos forma uma figura \(R’\) contida em alfa’. Esta figura é congruente a \(R\). Chamaremos de cilindro de bases \(R\) e \(R’\) à união dos segmentos \(PP’\) como acima.

\begin{figure}[H]
\centering

\includegraphics[width=200bp]{{75}.png}
\end{figure}

Observações:
\begin{enumerate}
\item {} 
Dois casos são de especial importância para este texto:
\begin{enumerate}
\item {} 
Caso em que R é um círculo. Neste caso o cilindro será chamado de cilindro circular.

\item {} 
Caso em que R é uma região poligonal. Neste caso o cilindro será chamado de prisma.

\end{enumerate}

\item {} 
Quando a reta s é perpendicular aos planos alfa e alfa’, dizemos que o cilindro é reto. A maioria dos exemplos discutidos neste texto serão de cilindros circulares retos ou prismas retos.

\item {} 
A altura de um cilindro é a distância entre os planos de suas bases.
\end{enumerate}

% \setlength\columnsep{2.5cm}
\begin{multicols}{4}
\begin{asy}
size(3cm,3.5cm);
currentprojection=orthographic(3,.3,.5);


triple a = (0,0,0);
triple b = (1,0,0);
triple c = (.5,.86602,0);

triple d = (0,0,1);
triple e = (1,0,1);
triple f = (.5,.86602,1);

draw(surface(a--b--c--a--cycle), laranja*80+opacity(0.65));
draw(surface(d--e--f--d--cycle), laranja*80+opacity(0.65));
draw(surface(a--d--e--b--cycle), laranja*80+opacity(0.65));
draw(surface(b--e--f--c--cycle), laranja*80+opacity(0.65));
draw(surface(c--f--d--a--cycle), laranja*80+opacity(0.65));

draw(d--e--f--cycle);
draw(d--a--b, dashed);
draw(a--c,dashed);
draw(e--b--c--f);
\end{asy}

\begin{asy}
size(3.5cm,3.5cm);
currentprojection=orthographic(3,1/2,.5);

draw(unitcube, laranja*80+opacity(0.65));

triple a = (0,0,0);
triple b = (0,1,0);
triple c = (1,1,0);
triple d = (1,0,0);

triple e = (0,0,1);
triple f = (0,1,1);
triple g = (1,1,1);
triple h = (1,0,1);

draw(h -- d);
draw(d -- c);
draw(c -- b);

draw(a -- d, dashed);
draw(a -- b, dashed);
draw(a -- e, dashed);

draw(e -- f);
draw(f -- g);
draw(g -- h);
draw(h -- e);
draw(f -- b);
draw(g -- c);
\end{asy}

\begin{asy}
size(3cm,3.5cm);
currentprojection=orthographic(1,2,.5);

draw(surface((0,0,0) -- (2,0,0) -- (2,0,3) -- (0,0,3) -- cycle), laranja*80+opacity(0.65));
draw(surface((0,0,0) -- (2,0,0) -- (2,1,0) -- (0,1,0) -- cycle), laranja*80+opacity(0.65));
draw(surface((0,0,3) -- (2,0,3) -- (2,1,3) -- (0,1,3) -- cycle), laranja*80+opacity(0.65));
draw(surface((0,1,0) -- (0,1,3) -- (2,1,3) -- (2,1,0) -- cycle), laranja*80+opacity(0.65));
draw(surface((0,0,0) -- (0,0,3) -- (0,1,3) -- (0,1,0) -- cycle), laranja*80+opacity(0.65));
draw(surface((0,1,0) -- (0,1,3) -- (2,1,3) -- (2,1,0) -- cycle), laranja*80+opacity(0.65));
draw(surface((2,0,0) -- (2,0,3) -- (2,1,3) -- (2,1,0) -- cycle), laranja*80+opacity(0.65));

draw((0,1,0) -- (0,0,0) -- (2,0,0), dashed);
draw((0,0,0) -- (0,0,3), dashed);
draw((0,0,3) -- (2,0,3) -- (2,1,3) -- (0,1,3) -- cycle);
draw((0,1,0) -- (0,1,3));
draw((2,1,0) -- (2,1,3));
\end{asy}

\begin{asy}
size(3cm,3.5cm);
currentprojection=orthographic(-1.25,.2,1/2);

draw(surface((0,0,0) -- (1,1,0) -- (0.35796,2.60007,0) -- (-1.03884,2.03884,0) -- (-1.26007,0.64203,0) -- cycle), laranja+opacity(0.65));
draw(surface((0,0,4) -- (1,1,4) -- (0.35796,2.60007,4) -- (-1.03884,2.03884,4) -- (-1.26007,0.64203,4) -- cycle), laranja+opacity(0.65));

draw((0,0,4) -- (1,1,4) -- (0.35796,2.60007,4) -- (-1.03884,2.03884,4) -- (-1.26007,0.64203,4) -- cycle);

draw(surface((0,0,4) -- (1,1,4) -- (1,1,0) -- (0,0,0) -- cycle), laranja+opacity(0.5));
draw((0,0,0) -- (1,1,0), dashed);
draw((1,1,0) -- (1,1,4), dashed);

draw(surface((1,1,0) -- (0.35796,2.60007,0) -- (0.35796,2.60007,4) -- (1,1,4) -- cycle), laranja+opacity(0.65));
draw((1,1,0) -- (0.35796,2.60007,0), dashed);

draw(surface((0.35796,2.60007,0) -- (-1.03884,2.03884,0) -- (-1.03884,2.03884,4) -- (0.35796,2.60007,4) -- cycle), laranja+opacity(0.65));
draw((0.35796,2.60007,0) -- (-1.03884,2.03884,0) -- (-1.03884,2.03884,4) -- (0.35796,2.60007,4) -- cycle);

draw(surface((-1.03884,2.03884,0) -- (-1.26007,0.64203,0) -- (-1.26007,0.64203,4) -- (-1.03884,2.03884,4) -- cycle), laranja+opacity(0.65));
draw((-1.03884,2.03884,0) -- (-1.26007,0.64203,0) -- (-1.26007,0.64203,4) -- (-1.03884,2.03884,4) -- cycle);

draw(surface((0,0,0) -- (-1.26007,0.64203,0) -- (-1.26007,0.64203,4) -- (0,0,4) -- cycle), laranja+opacity(0.65));
draw((0,0,0) -- (-1.26007,0.64203,0) -- (-1.26007,0.64203,4) -- (0,0,4) -- cycle);

\end{asy}
\end{multicols}


\subsection{Cone}

Considere uma região \(R\) num plano alfa e um ponto \(V\) não pertencente a alfa. Chamaremos de cone de base \(R\) e vértice \(V\) ao conjunto formado pela união dos segmentos \(PV\) onde \(P\) pertence à região \(R\).

\begin{figure}[H]
\centering

\ifnum\aluno=1
\noindent\includegraphics[width=200bp]{{77}.png}
\else
\noindent\includegraphics[width=175bp]{{77}.png}
\fi
\end{figure}

Observações:

1. Novamente os casos especiais para este texto são os casos em que:
\(R\) é um círculo.
\begin{enumerate}
\item {} 
Neste caso o cone de base \(R\) será chamado simplesmente de cone ou de cone circular.

\item {} 
\(R\) é uma região poligonal. Neste caso o cone de base \(R\) será chamado de pirâmide de base \(R\).

\end{enumerate}
\begin{enumerate}
\setcounter{enumi}{1}
\item {} 
Quando o cone for circular e a reta perpendicular a alfa passando por \(V\) for o centro do círculo da base, dizemos que o cone é reto.

\item {} 
A altura de um cone é a distância do seu vértice ao plano da base. Isto é, o comprimento do segmento perpendicular ao plano da base, que passa pelo vértice do cone.

\end{enumerate}

\begin{reflection}
\begin{enumerate}
\item {} 
Qual é o prisma que qualquer face pode ser tomada como base do prisma?

\item {} 
Qual é a pirâmide que qualquer face pode ser tomada como base da pirâmide?

\end{enumerate}
\end{reflection}

\begin{observation}

O que é um triângulo? Os três segmentos de  reta ou a região por eles delimitada?

A resposta para esta segunda pergunta não é realmente relevante na maioria das situações. Podemos usar a palavra triângulo em ambos os casos e sermos mais específicos dizendo região triangular ou linha poligonal triangular, quando necessário. Por outro lado, É comum dizermos circunferência para indicar o contorno de uma região circular e círculo para referir  à região.

Com os cilindros e cones não será diferente. Usaremos as expressões cilindro e cone tanto para identificar o sólido como para a superfície que o delimita. A diferença fica determinada pelo contexto. Dizemos que uma vela e um copo têm formato cilíndrico. Também nos referimos ao bloco de concreto e à caixa de papel como paralelepípedos. Isso simplifica a comunicação.
\end{observation}



\practice{Elementos de Geometria Espacial}
\label{\detokenize{GE504-7::doc}}\label{\detokenize{GE504-7:praticando}}
\begin{task}{posições relativas de retas e planos}



Decida se as afirmações a seguir são verdadeiras ou falsas, justificando as falsas com argumentos ou desenhos.
\begin{enumerate}
\item {} 
Se uma reta \(r\) não está contida no plano \(α\) e é paralela à reta \(s\) que está contida no plano \(α\), então a reta \(r\) e o plano \(α\) são paralelos.

\item {} 
Se uma reta \(r\) não está contida no plano \(α\) e é perpendicular à reta \(s\) que está contida no plano \(α\), então a reta \(r\) e o plano \(α\) são perpendiculares.

\item {} 
Se uma reta \(r\) é paralela ao plano \(α\), então \(r\) é paralela a todas as retas do plano \(α\).

\item {} Elementos de Geometria Espacial
Veja as imagens abaixo e reveja as suas respostas nos itens a), b) e c).

\end{enumerate}

\begin{figure}[H]
\centering

\noindent\includegraphics[width=400bp]{{78798081}.png}
\end{figure}
\end{task}

\begin{reflection}

Agora você vai rever como se calculam as áreas de paralelogramos e triângulos a partir da área de retângulos e também que o volume de um prisma de base poligonal é dado por área da base vezes altura, assim como o volume do prisma de base retangular (paralelepípedo).

Em matemática é muito importante construir as novas ideias a partir do que tem sido definido e demonstrado previamente, assim, é relevante mostrar como o cálculo da área do retângulo pode ser estendido para justificar os métodos de cálculo das áreas de: paralelogramos, triângulos e outros polígonos em geral.

\textbf{Área do paralelogramo e volume de prisma cuja base é um paralelogramo}

A partir fórmula da área do retângulo, é possível deduzir a fórmula da área de um paralelogramo.

Observe que, dado um paralelogramo qualquer, sempre podemos projetar um dos seus vértices no lado oposto.

\begin{figure}[H]
\centering

\noindent\includegraphics[width=150bp]{{82}.png}
\end{figure}

Cortando o paralelepípedo nessa projeção e movendo o triângulo retângulo formado até que a hipotenusa coincida com o lado oposto, obtemos um retângulo.

\begin{figure}[H]
\centering

\noindent\includegraphics[width=150bp]{{83}.png}
\end{figure}

\begin{figure}[H]
\centering

\noindent\includegraphics[width=150bp]{{84}.png}
\end{figure}

Observe que o retângulo obtido e o paralelogramo original coincidem em base e altura, por tanto, a área de qualquer paralelogramo pode ser calculada da mesma forma: \(A\) \(=\) base \(x\) altura.

Já vimos que o volume de um paralelepípedo retângulo pode ser calculado multiplicando-se a área da base (\(A\)) pela altura (\(h\)) relativa a este lado.
\begin{equation*}
\begin{split}V = A . h\end{split}
\end{equation*}
Mostraremos que esta mesma expressão serve para qualquer prisma reto de base poligonal. Comecemos pelos prismas cujas bases são paralelogramos.

\begin{figure}[H]
\centering

\noindent\includegraphics[width=300bp]{{85_1}.png}
\end{figure}

A atividade está no link: \url{https://www.geogebra.org/classic/mfnxkdqa}

Aproveite e acesse também a demonstração sem palavras da área do paralelogramo do \href{http://www.cdme.im-uff.mat.br/dsp/dsp-html/dsp-br.html}{CDME da UFF}.

\textbf{Área do triângulo e volume de prisma de base triangular}

Após justificar a expressão para o cálculo da área de um paralelogramo qualquer a partir da fórmula para a área de um retângulo (paralelogramo especial), mostraremos como obter a fórmula da área do triângulo a partir da expressão para o cálculo da área de um paralelogramo.

Considere um triângulo \(ABC\) qualquer. Trace uma reta paralela a \(BC\) por \(A\) e uma reta paralela a \(AB\) por \(C\). Chame de \(D\) o ponto de interseção destas duas retas traçadas.

\begin{figure}[H]
\centering

\noindent\includegraphics[width=150bp]{{86}.png}
\end{figure}

Como os lados opostos do polígono \(ABCD\) são paralelos, este quadrilátero é um paralelogramo. Observe que os triângulo \(ABC\) e \(ADC\) são congruentes pelo caso \(LLL\) de congruências de triângulos. Assim, as áreas destes dois triângulos são iguais. Portanto,  Área(\(ABC\)) = Área(\(ADC\)) = (base x altura)/2.

Do mesmo modo como fizemos antes, podemos calcular o volume do prisma de base triangular como área da base vezes a altura do prisma com um argumento similar ao usado para calcular a área do triângulo. Basta definir um prisma congruente ao prisma triangular dado que quando posicionado de maneira conveniente transforme o prisma triangular em um prisma cuja base é um paralelogramo. Por um lado, o volume deste novo prisma é o dobro do volume do prisma triangular. Por outro, o volume do novo prisma já foi calculado antes e, sabe-se, que é dado por área da base vezes altura. Conclusão: o volume do prisma de base triangular também é dado por área da base vezes altura. Resumindo em linguagem matemática temos:
\begin{align*}
V(\text{prisma triangular}) &= V(\text{prisma paralelogramo})/2 \\
&=(\text{área do paralelogramo})\times(\text{altura do prisma})/2 \\
&=(\text{área do paralelogramo}/2)\times(\text{altura}) \\
&= (\text{área do triângulo})\times(\text{altura}).
\end{align*}
\begin{figure}[H]
\centering

\noindent\includegraphics[width=400bp]{{87888990}.png}
\end{figure}

Aproveite e acesse também a demonstração sem palavras da fórmula para a área do triângulo do \href{http://www.cdme.im-uff.mat.br/dsp/dsp-html/dsp-br.html}{CDME da UFF}.

\textbf{Área de um polígono qualquer e volume de prisma de base poligonal qualquer}

Considere um polígono plano qualquer. Observe que sempre é possível decompor este polígono em triângulos como no exemplo da figura.

\begin{figure}[H]
\centering

\noindent\includegraphics[width=300bp]{{91}.png}
\end{figure}

Então a área deste polígono é a soma das áreas dos triângulos formados. Do mesmo modo, dado um prisma de base poligonal qualquer, podemos decompor este prisma em prismas de bases triangulares, cuja soma dos volumes é o volume do prisma original. Isto é,
\begin{equation*}
\text{Volume do prisma}= \text{Área da base}\times\text{altura}.
\end{equation*}

Por exemplo, se o prisma original tem sua base decomposta em 4 triângulos, digamos \(T_1\), \(T_2\), \(T_3\) e \(T_4\) como o da figura, então o volume do prisma original é dado por:
\begin{align*}
V(\text{prisma hexagonal})&=V(\text{prisma de base \(T_1\)}) + V(\text{prisma de base \(T_2\)}) \\ 
&+V(\text{prisma de base \(T_3\)}) + V(\text{prisma de base \(T_4\)})\\
&=A(T_1)\times h + A(T_2) \times h + A(T_3) \times h + A(T_4) \times h\\
&=(A(T_1) + A(T_2) + A(T_3) + A(T_4)) \times h\\
&=A(\text{hexágono}) \times h.
\end{align*}
\end{reflection}

\begin{knowledge}

Como qualquer cilindro reto pode ser tão bem aproximado quanto se deseje por prismas, o volume de qualquer cilindro pode ser calculado por meio da expressão
\begin{equation*}
\text{Volume}=(\text{Área da base})\times\text{altura}.
\end{equation*}
\begin{figure}[H]
\centering

\noindent\includegraphics[width=100bp]{{92}.png}
\end{figure}

Na Seção Princípio de Cavalieri, você verá uma justificativa para esta expressão no caso em que a base é um círculo e que esta mesma expressão serve para calcular o volume de qualquer cilindro, seja ele reto ou oblíquo.
\end{knowledge}

\clearpage
\def\currentcolor{session2}
\begin{objectives}{Um decímetro cúbico é igual a um litro e Secções no cubo}
{
\begin{enumerate}\setcounter{enumi}{10}
\item Reconhecer (identificar e nomear) elementos básicos da geometria espacial que são necessários para volumes e relacioná-los entre eles (e.g., posições relativas de planos (ver o que é realmente necessário aqui)).  (vértices de sólidos, planos paralelos, perpendicularidade entre reta plano, distância de ponto a plano e retas reversas no espaço)

\item Entender (analisar, segundo Van Hiele) os sólidos clássicos por meio de suas propriedades e não apenas por associação e semelhança (visualização, segundo Van Hiele).

\item Reconhecer um objeto sólido a partir de suas projeções ou cortes.

\item Entender a relação entre a planificação de um objeto e o sólido gerado por ela e vice-versa.
\end{enumerate}
}{1}{2}
\end{objectives}
\begin{sugestions}{Um decímetro cúbico é igual a um litro e Secções no cubo}
{
Nesta atividade pretende-se desenvolver a capacidade de transferência dos estudantes entre as diversas representações planas de sólidos no espaço e os próprios objetos. Assim, materiais para a construção de objetos em três dimensões são incluídos como suporte às atividades nas quais se indagam propriedades dos objetos.

Diversos elementos dos sólidos serão destacados e identificados em cada uma das diversas representações com a intenção de desenvolver no estudante uma familiaridade com os objetos sólidos e suas representações projetivas, o que permitirá que ele reconheça e descreva os elementos corretamente e resolva de forma correta problemas enunciados na geometria espacial.

Links relacionados: \url{https://www.korthalsaltes.com/}
}{1}{2}
\end{sugestions}
\begin{sugestions}{Um decímetro cúbico é igual a um litro e Secções no cubo}
{
\textbf{Representação de objetos:} textual, plana (desenho projetivo), espacial.

\textbf{Operações construtivas:} pĺanificação, decomposição em faces.

\textbf{Outras operações:} cortes, rotação.

\textbf{Materiais necessários:} Para esta atividade será necessário um material que permita preencher volumes, com a finalidade de medir a capacidade de sólidos geométricos utilizando dito material. Os modelos que serão utilizados para medir as suas capacidades estão abertos. No caso de escolher trabalhar com água, eles precisam ser construídos em material plástico (ou em material plastificado reciclado, e.g. tetra-brick).
\begin{itemize}
\item {} 
Fluido: Água, bolinhas de isopor, areia, arroz, feijão, lentilha, fubá, etc. No pior dos casos: bolinhas de gude, massinha (usar as formas como moldes).

\item {} 
Recipientes com escalas em ml.

\item {} 
Planificações de sólidos com a escala adequada na qual se quer a construção.

\item {} 
Semi-esfera de isopor + Cone e Cilindro (ambos do mesmo diâmetro e altura igual ao raio do interior da semi-esfera).

\item {} 
Régua graduada em cm.

\item {} 
Tesoura.
\end{itemize}
}{0}{0}
\end{sugestions}
\clearmargin
\clearmargin
\clearmargin
\begin{objectives}{Cilindro de GNV}
{
OE16. Aplicar a decomposição de um sólido dado em sólidos de volume conhecido para calcular seu volume, propondo suas próprias decomposições e incluindo a subtração.

OE17. Analisar o erro de diferentes procedimentos de aproximação por meio do conceito de cota inferior e cota superior.

OE9 - (A) Volume \& outras grandezas (Praticando) - Aplicar relações entre (área e) volume e outras grandezas em situações cotidiana.
}{1}{1}
\end{objectives}
\begin{sugestions}{Cilindro de GNV}
{
\textbf{Organização em sala de aula:} Especialmente se sua turma possuir mais de 20 estudantes, recomenda-se que os estudantes estejam dispostos em grupos de 4 ou 5 para que argumentem uns com os outros. Recomenda-se o estabelecimento de uma dinâmica de discussão no grupo. O fechamento da atividade está no Para refletir, é importante discuti-lo com os estudantes.

\textbf{Sugestões gerais:}

PARTE I. Esta parte da atividade tem como objetivo o cálculo aproximado de volume por inclusão de um sólido em outro. Assim pretende-se que os estudantes usem recursos diversos para aproximarem o melhor que puderem o volume de um objeto do mundo real, obtendo também uma margem de erro. Isso costuma ter muito mais utilidade em situações reais do que o cálculo de volumes exatos.

Ao final desta etapa é importante que o estudante tenha reconhecido que a discussão envolve três medidas: a capacidade do tanque, o volume ocupado pelo tanque e o volume do carro ou do porta-malas do carro..

PARTE II. Esta parte não trata realmente de volumes, você pode prescindir dela para o aprendizado deste tema. O objetivo desta parte é conectar o tema combustíveis com sustentabilidade e economia. A parte usa raciocínio lógico, interpretação de texto e domínio de razões e proporções. Por outro lado, é algo que pode despertar o interesse dos estudantes pela matemática por estar diretamente relacionada a situações com as quais ele ou pessoas próximas a ele podem se deparar.

\textbf{Atividade relacionada:} Atividade: GNV, na Seção: o conceito de volume.
}{1}{1}
\end{sugestions}
\begin{sugestions}{Cilindro de GNV}
{
\textbf{Links relacionados:} Sobre queima do metano, da gasolina e do etanol:
\begin{itemize}
\item {} 
\url{http://www.usp.br/qambiental/combustao\_energiaExperimento.html}

\item {} 
\url{https://www.educabras.com/enem/materia/quimica/aulas/reacao\_de\_combustao\_combustao\_completa\_e\_incompleta}

\item {} 
\url{http://www.fem.unicamp.br/~em672/GERVAP1.pdf}

\item {} 
\url{http://www.usp.br/qambiental/combustao\_energiaExperimento.html}

\item {} 
\url{https://www.infoescola.com/quimica/gas-natural-veicular-gnv/}

\item {} 
\url{http://ecoscore.be/en/info/ecoscore/co2}

\end{itemize}

\textbf{Materiais necessários:} Para o desenvolvimento desta atividade o aluno pode precisar das fórmulas para o cálculo do volume do cilindro, da esfera e do cone. Se seus alunos ainda não conhecem tais fórmulas, recomendamos que você a escreva no quadro ou permita que os estudantes busquem na internet em seus celulares.

Volume do cilindro = Área da base x altura.

Volume da esfera = (4/3) pi raio\(\sp{\text{3}}\)

Volume do cone = ( Área da base x altura )/ 3

Isto pode reforçar que o conhecimento da fórmula não é o foco. Os estudantes podem precisar de ajuda no item b) para determinar onde termina o cilindro do tanque, estimule o uso da régua e de proporções, tome cuidado para não dar dicas demais e tornar a atividade uma mera aplicação de fórmulas.
}{0}{1}
\end{sugestions}
\begin{answer}{Cilindro de GNV}
{
\paragraph{Parte I}  
Usando uma régua para comparar a largura da imagem do tanque (dimensão de $35$ cm) com a ponta, parte em que ele deixa de ter a forma de um cilindro, vemos que a ponta mede aproximadamente metade, portanto, aproximadamente $17{,}5$ cm em cada ponta. Então o volume do tanque é aproximadamente o volume de um cilindro de altura $70$ cm e raio $17{,}5$ cm. Uma expressão para o volume do cilindro é \(\pi r^2 h\), isto é aproximadamente $0{,}067314$ m\(\sp{3}\) o que corresponde a $67{,}314$ litros.

Observe que a escolha para a altura do cilindro poderia ser qualquer valor entre $85$ e $50$ cm (alturas aproximadas do menor cilindro que contém o tanque e do maior cilindro que está contido no tanque).

FIGURA

O tanque contém um cilindro \(C_1\) de altura \(h_1 = 50\text{ cm}\) e raio da base $17{,}5$ cm, ou seja, o volume do tanque é maior que o volume deste cilindro \(C_1\). Por outro lado, o tanque cabe (está contido) em um cilindro \(C_2\) de altura \(h_2 = 85\text{ cm}\) e raio da base \(17{,}5\) cm, de onde concluímos que o volume do tanque é menor que o volume de \(C_2\). Finalmente o volume do tanque certamente está contido no intervalo aberto 
$]Vol(C_1), Vol(C_2)[$.

FIGURA
}
{1}
\end{answer}

\begin{answer}{Cilindro de GNV}
{
Efetuando os cálculos obtemos:
\begin{equation*}
\begin{split}Vol(C_1) = \pi \cdot r^2 \cdot h_1 = \pi\cdot0{,}175^2\cdot 0{,}5 = 0{,}0153125\cdot\pi \approx 0{,}048105627\text{ m}^3
\end{split}
\end{equation*}
o que corresponde a $48{,}106$ litros, aproximadamente. Procedendo de maneira análoga com o cilindro \(C_2\), obtemos \(Vol(C_2) = \pi\cdot r^2 \cdot h_2 = 0{,}0263125 \cdot \pi \approx 0{,}0817796\text{ m}^3 \approx 81,780\) litros.

Finalmente, o volume $Vol(T)$ ocupado pelo tanque está nos intervalos $0{,}048106\text{ m}^3 < Vol(T) < 0{,}081780\text{ m}^3$ ou $48{,}106\text{ litros} < Vol(T) < 81{,}780$ litros.

Uma possibilidade de aproximação para o volume do tanque é considerá-lo como a composição de dois hemisférios com um cilindro como na figura a seguir.

FIGURA

Então teremos \(Vol(T) \approx Vol(C_1) + Vol\) (bola de raio $17{,}5$ cm). O volume do cilindro \(C_1\) já foi calculado no item anterior e após uma busca na internet ou na seção X deste capítulo, o estudante pode descobrir que o volume de uma bola de raio r é \((4 \cdot \pi \cdot r^3)/3\). Efetuando os cálculos obtemos
\begin{align*}
Vol(T) & \approx Vol(C_1) + Vol(\text{bola de raio } 17{,}5\text{ cm}) \\
& \approx 0{,}048106 + (4 \pi \cdot 0{,}175^3)/3 \approx 0{,}070555293\text{ m}^3
\end{align*}
o que corresponde a aproximadamente $70{,}555$ litros.

Pode causar estranhamento que o $15$ m$^3$ de GNV caibam em um tanque que ocupa um volume aproximado de $0{,}07$ m\(\sp{\text{3}}\). Mas como discutimos na Parte I, o GNV é armazenado no tanque sob pressão e por isso seu volume é reduzido.

\paragraph{Parte II}

Para decidir que combustível é mais vantajoso, precisamos saber qual anda mais quilômetros com a mesma quantia em dinheiro, isto é, basta obter quantos quilômetros o carro faz por real. Seja G preço do litro da gasolina obtida no link e E o preço do litro do etanol, para a gasolina temos

$14$ km/Litro  dividido por $G$ litros/R\$, obtemos $\frac{14}{G}$ km/R\$.

O cálculo para o rendimento do etanol é análogo. No Rio de Janeiro em 13 de setembro de 2018 obtivemos os seguintes preços $G = \text{R\$ }4,992$/litro e $E = \text{R\$ }3,338$/litro, fazendo as contas obtemos um rendimento de $2{,}804$ km/R\$ para a gasolina e $2{,}996$ km/R\$ para o etanol de modo que o etanol é mais vantajoso que a gasolina, nas condições dadas porque anda mais quilômetros com $1$ real.

O rendimento do etanol por litro de combustível é $10$ km/L. O rendimento da gasolina é de $14$ km/L. Assim, o rendimento do etanol é $10/14 \approx 0{,}7143$ o rendimento da gasolina. Logo se o preço do etanol estiver até $71{,}43\%$ do valor da gasolina, ele será mais vantajoso, caso contrário a gasolina será mais vantajosa. O valor $70\%$ é uma aproximação. Deve-se ao fato de que os rendimentos dependem dos veículos considerados.

Digamos que você rode $x$ quilômetros por mês. Vamos calcular o custo mensal com combustível usando etanol e subtrair do custo mensal com combustível usando GNV e efetuar a diferença, esta diferença é a economia mensal em função do número $x$ de quilômetros rodados num mês. Igualando a expressão obtida a 420 obtemos o número mínimo de quilômetros que precisaremos dirigir num mês para que a economia com o combustível seja igual à prestação. Uma vez que obtivermos $x$, basta dividir por 30 para obter a média diária de quilômetros necessários para que o GNV compense nas condições do problema. Vamos às contas.

Se $R_\text{etanol}$ é o número de quilômetros rodados com $1$ real, isto é, o  rendimentos em km/R\$ do etanol obtido no item a). Rodando $x$ km, serão gastos
\begin{equation*}
\begin{split}R_\text{etanol}&\text{ km} \longrightarrow 1\text{ real}\\
x&\text{ km} \longrightarrow \text{? reais}
\end{split}
\end{equation*}
Obtemos um custo de $x/R_\text{etanol}$ reais no mês. Analogamente, o custo mensal com GNV será de$ x/R_\text{gnv}$ reais e a economia será dada pela diferença do custo com etanol pelo custo com GNV em um mês:

$$x/R_\text{etanol} - x/R_\text{gnv}$$

Igualando este valor a $420$ obtemos o valor de $x$ para que a economia seja igual à prestação da instalação do kit gás.

Usando os valores do Rio de Janeiro como no item \titem{a)} obtemos

$x/2{,}996 - x/5{,}797 = 420$, acarreta em $x\approx2604{,}24$ km, isto é, aproximadamente $87$ quilômetros por dia, em média.

Gasolina: considerando que o veículo tenha um rendimento de $14$ km/L, como são $2392$ gramas de CO$_2$ por litro de gasolina, obtemos
\begin{equation*}
2392/14 \text{ (gramas de CO\(_2\)}/\text{L})/(\text{km}/\text{L})\approx170{,}86\text{ gramas de CO\(_2\)/km}.
\end{equation*}
GNV: considerando o rendimento de $16$ km/m\(\sp{3}\) de GNV, como são $2252$ gramas de CO\(_2\)/kg de GNV, precisamos da emissão de CO\(_2\) por metro cúbico de GNV. Para isso multiplicamos pela densidade do GNV
\begin{align*}
&2252\text{ gramas de CO\(_2\)/kg de GNV} \times 0{,}8\text{ kg/m\(\sp{3}\)} =\\
&2252\times0{,}8\text{ (gramas de CO\(_2\)/kg de GNV)} \times \text{ (kg de GNV/m\(\sp{3}\) de GNV)}  =\\
&1801{,}6\text{ gramas de CO\(_2\)/m\(\sp{3}\) de GNV}.
\end{align*}
Finalmente, dividimos para obter a emissão de CO\(_2\) por m\(\sp{3}\) de GNV.
\begin{equation*}
1801{,}6/16\text{ (gramas de CO\(_2\)/L)/(km/L)} = 112{,}6\text{ gramas de CO\(_2\)/km.}
\end{equation*}
Conclusão, o GNV emite $112{,}6/170{,}86=65{,}9\%$ do CO$_2$ emitido pela gasolina a cada.
}{9}
\end{answer}
\begin{objectives}{Aproximando pi}
{
OE17. Analisar o erro de diferentes procedimentos de aproximação por meio do conceito de cota inferior e cota superior.
}{1}{2}
\end{objectives}
\begin{sugestions}{Aproximando pi}
{
\textbf{Organização em sala de aula:} Especialmente se sua turma possuir mais de 20 estudantes, recomenda-se que os estudantes estejam dispostos em grupos de 4 ou 5 para que argumentem uns com os outros. Recomenda-se o estabelecimento de uma dinâmica de discussão no grupo. O fechamento da atividade está no Para refletir, é importante discutí-lo com os estudantes.

\textbf{Sugestões gerais:} Nessa atividade o estudante estará reforçando a idéia de aproximar a área de uma figura, obtendo cotas inferiores e superiores para o valor exato. Também será nessa atividade que daremos um sabor da seção seguinte, em que aproximações sucessivamente melhores levam a um resultado exato no limite.
A qualidade das aproximações não é o foco central dessa atividade e sim o procedimento empregado.

\textbf{Materiais utilizados:} Papel, caneta e régua. Também é recomendado (mas não necessário) o uso de papel milimetrado e um compasso.
}{1}{1}
\end{sugestions}

\begin{task}{um decímetro cúbico é igual a um litro}

Construa (em material resistente) um cubo de $10$ cm de lado, quer dizer $1$ dm de aresta, este cubo mede \(1\) dm$^3$ = $1.000$ cm$^3$.

(\emph{Material:} planificação de cubo de $10$ cm de lado)
\begin{enumerate}
\item {} 
Preencha ele com o fluido, repasse o fluido para o seu recipiente graduado em ml ou litros. Qual é a capacidade do cubo?

\item {} 
Quantos litros tem em 1 m\super{$3$}?

$1$ m\super{$3$} = ($1$\text{ m})$ \times$ ($1$\text{ m}) $\times$ ($1$\text{ m}) = ( \rule{3em}{.5pt} dm) $\times$ ( \rule{3em}{.5pt} dm) $\times$ ( \rule{3em}{.5pt} dm) = \rule{3em}{.5pt} dm\super{$3$} = \rule{3em}{.5pt} L

\end{enumerate}
\end{task}

\begin{task}{secções no cubo}



Construa 5 pequenos cubos em papel.
\begin{enumerate}
\item {} 
Em cada um dos cubos montados por você, reproduza os desenhos dos modelos a seguir.

\end{enumerate}

\begin{figure}[H]
\centering

\noindent\includegraphics[width=350bp]{{93}.png}
\end{figure}
\begin{enumerate}
\item {} 
Para cada modelo, use uma planificação e reproduza, na planificação, as linhas traçadas nos cubos montados. Existe uma única forma de fazer isso? Discuta com seus colegas.

\end{enumerate}
\end{task}

\clearpage

\begin{task}{altura de prismas e pirâmides}



Construa as pirâmides e os prismas seguindo as instruções no \href{https://docs.google.com/document/d/12ERHynaBYMapyryZgVE3RKWWpxqF3BUgNvVhK5ywLmc/edit}{material para reprodução}.
\begin{enumerate}
\item {} 
Identifique os prismas cujas alturas coincidem com suas arestas laterais. O que esses prismas têm em comum? E os demais, o que têm em comum?

\item {} 
Em todo prisma reto a altura coincide com suas arestas laterais. Já se o prisma for oblíquo a altura será menor do que suas arestas laterais.

\item {} 
Identifique as pirâmides cujas alturas coincidem com uma de suas arestas laterais.

\item {} 
É possível que a altura de uma pirâmide coincida com duas de suas arestas laterais? Explique.

\item {} 
Identifique a pirâmide e o cilindro que têm a mesma altura e a mesma base. Preencha com um fluido a pirâmide e despeje o conteúdo no prisma de mesmas base e altura. Repita o processo até que consiga encher o prisma. Quantas vezes foi necessário repetir a operação para preencher o prisma?

\item {} 
Faça o mesmo experimento com o outro par pirâmide/prisma. O resultado é se repetiu?

\item {} 
Tente com o cone e o cilindro. Acontece a mesma coisa?

\end{enumerate}
\end{task}

\begin{task}{intuição sobre volume da esfera}



Coloca o cone dentro do cilindro com as suas bases coincidindo e ao lado coloca a semi-esfera aberta para cima, como se mostra na figura.

\begin{figure}[H]
\centering

\noindent\includegraphics[width=200bp]{{94}.png}
\end{figure}
\begin{enumerate}
\item {} 
Separa pequenas medidas de arroz com o recipiente medidor de forma sucessiva e coloca a mesma quantidade em dentro de cada figura.

\item {} 
A cada passo, observa a altura que o conteúdo adquire em cada figura.

\item {} 
Continua até que uma das duas figuras fique cheia. O que aconteceu com a outra?

\item {} 
Você pode concluir que o volume de ambas as figuras é a mesma?

\item {} 
Partindo da fórmula do volume do cilindro e do cone, calcule o volume da semi-esfera. Ela coincide com a fórmula dada na literatura?

\end{enumerate}
\end{task}

\begin{task}{planificações do cilindro e do cubo}


\begin{enumerate}
\item {} 
Quais das seguintes são planificações possíveis de um cubo?

\begin{figure}[H]
\centering

\noindent\includegraphics[width=400bp]{{95}.png}
\end{figure}

\item {} 
Os retângulos a seguir são planificações da região lateral de um cilindro circular reto. Desenhe como ficarão os cilindros depois de fechados.

\begin{figure}[H]
\centering

\begin{tikzpicture}
\begin{scope}
\draw (0,0) rectangle (5,3.5) node [below left, xshift=-5 cm,\currentcolor] {i)};;
\draw (0,0) -- (5,3.5);
\end{scope}


\begin{scope} [xshift=6cm]
\draw (0,0) rectangle (5,3.5) node [below left, xshift=-5 cm,\currentcolor] {ii)};
\draw [dashed, help lines] (0,3.5*3/4) -- (5,3.5*3/4); 
\draw (5,3.5) -- (0,3.5*3/4);
\draw [dashed, help lines] (0,3.5*2/4) -- (5,3.5*2/4);
\draw (5,3.5*3/4) -- (0,3.5*2/4);
\draw [dashed, help lines] (0,3.5*1/4) -- (5,3.5*1/4);
\draw (5,3.5*2/4) -- (0,3.5*1/4);
\draw (5,3.5*1/4) -- (0,0);
\end{scope}

\begin{scope} [yshift=-4.5cm]
\draw (0,0) rectangle (5,3.5) node [below left, xshift=-5 cm, \currentcolor] {iii)};;
\draw (.5,.5) rectangle (4.5,3);
\draw (.5+1.7/4,1.75) -- (2.5,.5+1.7/4) -- (4.5-1.7/4,1.75) -- (2.5,3-1.7/4) -- cycle;
\draw (2.5,1.75) circle (0.4375);
\end{scope}

\begin{scope}[xshift =6cm, yshift=-4.5cm]
\draw (0,0) rectangle (5,3.5) node [below left, xshift=-5 cm,\currentcolor] {iv)};;
\draw plot [domain=0:5, smooth] (\x,{1.75*sin (2*pi*\x/3 r)+1.75});
\end{scope}
\end{tikzpicture}
\end{figure}

\end{enumerate}
\end{task}

\begin{task}{cilindro de GNV}

Na Atividade: GNV discutimos a capacidade do tanque de GNV e a compressibilidade do gás nele colocado. Agora vamos aproveitar esta situação para falar do volume ocupado pelo tanque.

\paragraph{Parte 1 - Aproximação: cota inferior e cota superior}

\begin{figure}[H]
\centering

\noindent\includegraphics[width=200bp]{{100}.png}
\end{figure}
\begin{enumerate}
\item {} 
Aproxime o volume ocupado (em metros cúbicos), por um tanque de GNV de capacidade \(15{,}5\) m$^3$ como o da figura.

\item {} 
Encontre o menor número que você conseguir que seja certamente maior do que o volume ocupado pelo tanque de GNV. Apresente sua resposta em metros cúbicos e em litros. Use os recursos que julgar conveniente.

\item {} 
Agora encontre o maior número que você conseguir que seja certamente menor que o volume ocupado pelo tanque de GNV. Apresente sua resposta em metros cúbicos e em litros. Use os recursos que julgar conveniente.

\item {} 
Encontre o menor número que você conseguir que seja certamente maior que o volume ocupado pelo tanque de GNV. Apresente sua resposta em metros cúbicos e em litros. Novamente você pode usar os recursos que julgar conveniente para resolver a atividade.

\item {} 
Crie um modelo aproximado do tanque usando figuras conhecidas, busque as fórmulas para o cálculo do volume destas figuras e obtenha uma nova aproximação para o volume do tanque.

\item {} 
Discuta o significado da diferença entre as suas aproximações para o volume do tanque e a capacidade especificada pelo vendedor, de \(15{,}5\) m$^3$.

\end{enumerate}

\paragraph{Parte 2}

Dentre os carros populares, os mais econômicos na cidade fazem \(14\) km/L usando gasolina e \(10\) km/L usando etanol como combustível (Veja a edição de 2018 do \href{http://www.inmetro.gov.br/consumidor/pbe/veiculos\_leves\_2018.pdf}{Programa Brasileiro de Etiquetagem Veicular} do INMETRO).
\begin{enumerate}
\item {} 
Verifique se é financeiramente mais vantajoso usar gasolina ou etanol com os preços atuais no seu município. Use o preço médio para este cálculo*.
\begin{itemize}
\item {} 
Os preços atuais do GNV, da Gasolina e do Etanol no seu município estão no \href{http://anp.gov.br/preco/prc/Resumo\_Por\_Municipio\_Index.asp}{Sistema de Levantamento de Preços da ANP}.

\end{itemize}

\item {} 
É corrente entre usuários de carros com motor FLEX utilizar a regra dos $70\%$ para saber se é mais vantajoso usar etanol ou gasolina (Veja por exemplo: \href{http://www.calculoexato.net/calculadora-flex-gasolina-x-alcool/}{calculadora}). A regra funciona assim: se o preço do etanol for até $70\%$ do preço da gasolina, deve-se comprar etanol, acima desse percentual deve-se comprar gasolina. É claro que o valor $70\%$ é aproximado. Explique como ela foi obtida.

\end{enumerate}

Estes mesmos veículos fariam aproximadamente 16 km/m\(\sp{\text{3}}\) de GNV. Digamos que você saiba que o serviço de instalação pode ser financiado em até 10 vezes de R\$ $420{,}00$ para o seu carro.
\begin{enumerate}
\item {} 
Quantos quilômetros você precisaria rodar em média por dia para que você consiga pagar a instalação com a economia de combustível proveniente do uso do GNV. Use o preço médio para este cálculo?

\item {} 
Os motores dos veículos a base destes combustíveis fósseis, como GNV, gasolina e etanol funcionam a base de combustão. Isto é, consomem oxigênio e liberam gás carbônico e água. Um motor regulado libera aproximadamente $2392$ gramas de CO$_2$ por de gasolina e $2252$ gramas de CO$_2$ por quilograma de GNV. Qual dos dois combustíveis emite menos CO$_2$ por cada quilômetro rodado? Considerando que a densidade do GNV no motor do veículo é de aproximadamente $0{,}8$ kg/m\(\sp{\text{3}}\).

\end{enumerate}
\end{task}

\begin{task}{aproximando pi}

\paragraph{Parte 1}

Existe uma constante muito importante na matemática chamada  \(\pi\) (lê-se pi), que aparece nas fórmulas do comprimento da circunferência, área do círculo, volume da esfera, funções trigonométricas e muitas outras.

O valor dessa constante fundamental na matemática é definido como: a metade do comprimento da circunferência de raio um. Nesta atividade usaremos a fórmula da área do círculo para obter uma aproximação de \(\pi\).

Lembre-se que a fórmula para a área de um círculo de raio \(r\) é  \(\pi.r^2\). Dessa forma, para obter uma aproximação de \(\pi\), basta aproximar a área de um círculo e dividir o resultado obtido por \(r^2\). É o que faremos. Mas como aproximar a área de um círculo? Vamos fazer isso usando a área de figuras já conhecidas.
\begin{enumerate}
\item {} 
Desenhe um círculo em um papel em branco (usando uma forma circular ou um compasso) e meça o seu diâmetro. Feito isso, use uma régua para quadricular o papel com quadrados de tamanho um décimo do diâmetro do círculo, como ilustrado a seguir.

\begin{figure}[H]
\centering

\begin{tikzpicture}[scale=2.5]

\draw [\currentcolor, thick](0,0) circle (1cm);
\foreach \x in {-1.2,-1,-0.8,-0.6,-0.4} \draw (-1.3,\x) -- (1,\x);
\foreach \x in {-1.2,-1,-0.8,-0.6,-0.4} \draw (\x,-1.3) -- (\x,1);

\end{tikzpicture}\end{figure}

\item {} 
Qual a medida da área de um quadradinho do quadriculado que você fez?

\item {} 
Com essa figura, pinte todos os quadradinhos que estão inteiramente contidos no círculo. Qual é a área da região colorida? Agora pinte todos os quadradinhos que têm intersecção não vazia com o círculo. Qual é a área da nova região colorida. Com esses cálculos é possível concluir que a área do círculo está entre que números? Dê uma estimativa para a área do círculo.

\item {} 
Sabendo que a fórmula da área do círculo é \(\pi.r^2\) e usando os cálculos realizados no item anterior, apresente estimativas para \(\pi\), uma menor e outra maior do que o valor de \(\pi\). Algo como: “\(\pi\) é maior do que \rule{3em}{.5pt}  e menor do que \rule{3em}{.5pt}”.

\item {} 
Compare os resultados obtidos neste experimento com o valor de \(\pi\) que você conhece.

\item {} 
Que alteração poderia ser feita no processo desse experimento para melhorar a aproximação obtida para \(\pi\)?

\end{enumerate}

\paragraph{Parte 2}

A imprecisão do método utilizado na Parte 1 está na limitação do desenho, que, especialmente à medida que os quadradinhos diminuem, dificulta a decisão sobre alguns quadradinhos terem ou não interseção com o círculo. Desta vez, usaremos outro procedimento para aproximar o valor de \(\pi\), sem incorrer neste tipo de imprecisão.
\begin{enumerate}
\item {} 
Calcule o lado do quadrado e do octógono regular inscritos em um círculo de raio um (como na figura abaixo).

\begin{figure}[H]
\centering

\begin{tikzpicture} [scale=3.5, every path/.style={very thick}]


\draw (0,0) circle (1cm);
\draw [color=atento!80] (0.7071,0.7071) -- ++(-90:2*0.7071) -- ++(-180:1.4142) -- ++(-270:1.4142) -- cycle;
\draw [color=destacado!80] (0.7071,0.7071) -- ++(157.5:0.76535) -- ++(202.5:0.76535) -- ++(247.5:0.76535) -- ++(292.5:0.76535) -- ++(337.5:0.76535) -- ++(22.5:0.76535) -- ++(67.5:0.76535) -- cycle;
\draw [color=primario!80] (0.7071,0.7071) -- ++(-56.25:0.39017) -- ++(-56.25-22.5:0.39017) -- ++(-56.25-22.5-22.5:0.39017) -- ++(-56.25-22.5-22.5-22.5:0.39017) -- ++(-56.25-22.5-22.5-22.5-22.5:0.39017) -- ++(-56.25-22.5-22.5-22.5-22.5-22.5:0.39017) -- ++(-56.25-22.5-22.5-22.5-22.5-22.5-22.5:0.39017) -- ++(-56.25-22.5-22.5-22.5-22.5-22.5-22.5-22.5:0.39017) -- ++(-56.25-22.5-22.5-22.5-22.5-22.5-22.5-22.5-22.5:0.39017) -- ++(-56.25-22.5-22.5-22.5-22.5-22.5-22.5-22.5-22.5-22.5:0.39017) -- ++(-56.25-22.5-22.5-22.5-22.5-22.5-22.5-22.5-22.5-22.5-22.5:0.39017)--++(-56.25-22.5-22.5-22.5-22.5-22.5-22.5-22.5-22.5-22.5-22.5-22.5:0.39017)--++(-56.25-22.5-22.5-22.5-22.5-22.5-22.5-22.5-22.5-22.5-22.5-22.5-22.5:0.39017)--++(-56.25-22.5-22.5-22.5-22.5-22.5-22.5-22.5-22.5-22.5-22.5-22.5-22.5-22.5:0.39017)--++(-56.25-22.5-22.5-22.5-22.5-22.5-22.5-22.5-22.5-22.5-22.5-22.5-22.5-22.5-22.5:0.39017)-- cycle;
\foreach  \x in {0,22.5,45,67.5,90,112.5,135,157.5,180,202.5,225,247.5,270,292.5,315,337.5,360} \draw [very thin,help lines] (0,0) -- ++(\x:1);


\end{tikzpicture}

\end{figure}

\item {} 
Calcule a área dessas figuras e use os resultados para estimar o valor de \(\pi\). O valor encontrado é maior ou menor do que \(\pi\)? Compare a aproximação obtida neste item com a obtida na Parte 1 desta atividade. Avalie qualitativamente a melhora da aproximação.

\item {} 
Faça o mesmo agora considerando um hexágono circunscrito ao círculo unitário e estime o valor de \(\pi\) por cima.

\end{enumerate}

Outra forma de estimar o valor de \(\pi\) usando áreas pode ser encontrada no \href{https://www.geogebra.org/m/v2aqzkce}{aplicativo Geogebra: Aproximação de Pi}
\end{task}

\cleardoublepage
\def\currentcolor{session1}
\begin{objectives}{Princípio de Cavalieri em 2D}
{
OE23. Entender o Princípio de Cavalieri.

\textbf{Conceitos abordados:} Área de figuras planas, Princípio de Cavalieri
}{1}{1}
\end{objectives}
\begin{sugestions}{Princípio de Cavalieri em 2D}
{
\textbf{Organização em sala de aula:} O professor pode levar um projetor de multimídia para apresentar os aplicativos ou pode permitir que os estudantes manipulem os aplicativos em seus telefones celulares ou computadores.

\textbf{Materiais necessários:} Projetor de multimídia e computador com acesso à internet (o professor pode baixar as aplicações antes da aula e usar offline durante a aula) OU telefones celulares dos estudantes com conexão com a internet OU computadores com conexão com a internet.

\textbf{Links relacionados:} Aplicativos do GeoGebra utilizados na atividade \url{https://ggbm.at/bxrxatwv}, \url{https://ggbm.at/gkh7g4y5} e \url{https://ggbm.at/rqpdcc33}
}{0}{1}
\end{sugestions}
\begin{objectives}{Distância percorrida dada a velocidade instantânea}
{
OE22. Entender que refinamentos no processo de aproximação podem levar a erros cada vez menores.

OE23. Entender o Princípio de Cavalieri.
}{1}{1}
\end{objectives}
\begin{sugestions}{Distância percorrida dada a velocidade instantânea}
{
\textbf{Organização em sala de aula:} Manter os estudantes em grupos pode facilitar o controle do professor sobre o caminho da atividade e permitir que eles discutam suas estratégias de solução com outros estudantes.

\textbf{Sugestões gerais:} Esteja atento às estratégias dos estudantes durante o desenvolvimento da atividade para não permitir que os estudantes se afastem demais dos objetivos propostos e, especialmente, que atinjam os objetivos esperados.

Cuidado para não se antecipar às dificuldades dos estudantes e transformar a atividade em mero exercício de calcule. Deixe os estudantes testarem as suas estratégias e discutirem seus métodos. Valorize a diversidade de ideias.
}{1}{1}
\end{sugestions}
\clearmargin
\begin{objectives}{Volume de concreto de uma barragem}
{
\textbf{Objetivos específicos:}

OE22. Entender que refinamentos no processo de aproximação podem levar a erros cada vez menores.

\textbf{Conceitos abordados:} Volume, logaritmos, aproximações sucessivas.
}{1}{2}
\end{objectives}
\begin{sugestions}{Volume de concreto de uma barragem}
{
\textbf{Organização em sala de aula:} Recomenda-se que os estudantes estejam em pequenos grupos para que discutam as suas tentativas com outros estudantes.

\textbf{Dificuldades previstas:} O item a) não trata de volume, antes trata de funções exponencial e logaritmo. Caso você prefira saltar esta parte, dê a resposta aos seus estudantes.

\textbf{Sugestões gerais:} No enunciado não é indicada uma forma para que os estudantes realizem a aproximação. Estimule que eles reflitam sobre soluções diversas. Mas ao final recomenda-se que você apresente a solução deste aplicativo com um projetor.

A observação final no texto da atividade serve para que os estudantes busquem cotas superiores (e não inferiores) para o volume da barragem.

\textbf{Enriquecimento da discussão:} As ideias desta e da atividade anterior são aquelas que servirão de pano de fundo para a noção de integral que os estudantes que continuarem seus estudos na área de exatas verão. No aplicativo colocamos o símbolo de integral, embora ele seja desnecessário para ajudar a despertar a curiosidade do estudante e, eventualmente, criar a oportunidade do professor discutir temas mais avançados com a turma.

\textbf{Links relacionados:} Versão digital desta atividade \url{https://ggbm.at/nxtbehpa}.

\textbf{Materiais necessários:} Para o item a) é necessário calculadora científica para calcular exponencial e logaritmos. Mas você pode deixar os estudantes usarem seus celulares ou mesmo apresentar para eles os valores necessários.
}{1}{2}
\end{sugestions}
\begin{answer}{Volume de concreto de uma barragem}
{
\begin{enumerate}
\item {} 
Seja \(f: (0, \infty) \to \mathbb{R}\), a função procurada. Como ela é do tipo exponencial, podemos escrever \(f(z) = c \cdot a^z\). Sabendo que \(f(0) = 16\), obtemos \(c = 16\). Como \(f(24) = 4,67\), temos \(4,67 = 16 \cdot a^z\). Tomando o logaritmo natural em ambos os lados obtemos \(\ln 4,67 = \ln 16 + 24 \ln a\) e, portanto, \(\ln a = (\ln 4,67 - \ln 16)/24 \approx  -0,0513\), elevando ambos os membros a  \(e\) , obtemos \(a \approx 0,95\). Portanto,  \(f(z) = 16 \cdot 0,95^z\).

\item {} 
Como não pode faltar concreto, é melhor procurar uma cota superior. Um paralelepípedo de lados \(92m \cdot 16m x 24m\) certamente tem volume superior ao da barragem e seu volume é dado por \(92 \cdot 16 \cdot 24m^3 = 35,328m^3\).

\end{enumerate}
}{0}
\end{answer}

\explore{Princípio de Cavalieri}
\label{\detokenize{GE504-8:explorando-principio-de-cavalieri}}\label{\detokenize{GE504-8::doc}}
\begin{task}{Princípio de Cavalieri em 2D}



O Princípio de Cavalieri, objeto de estudo desta seção, apresenta condições para que duas regiões planas (dois sólidos) tenham mesma área ( mesmo volume).
\begin{enumerate}
\item {} 
Use o aplicativo \href{https://ggbm.at/bxrxatwv}{deste link} e o aplicativo \href{https://ggbm.at/gkh7g4y5}{deste link} e tente descrever com as suas palavras o que vem a ser o Princípio de Cavalieri. Registre por escrito a sua descrição.

\item {} 
Use o aplicativo \href{https://ggbm.at/rqpdcc33}{deste link} e veja se a descrição do item anterior ainda serve para este exemplo.

\end{enumerate}
\end{task}

\begin{task}{distância percorrida dada a velocidade instantânea}



A velocidade de um veículo em metros por segundo no intervalo de tempo {[}1,4{]} segundos é dada pela expressão \(v(t) = t^2 - 4t + 5\) cujo gráfico está esboçado na figura.

\begin{figure}[H]
\centering

\begin{tikzpicture}[scale=1, every node/.style={scale=1.5}]

\draw [->] (-.1,0) -- (5.5,0) node [above left, scale=0.6] {$t$};
\draw [->] (0,-.1) -- (0,5.5) node [below right, scale=0.6] {$v$};

\foreach \x in {1,...,5} {\draw (\x,.05) -- (\x,-.05)  node [scale=0.5, below] {\x}; \draw (.05,\x) -- (-.05,\x) node [scale=0.5, left] {\x};
}
\node [below left, scale=0.5] at  (-.05,-.05)  {0};

\draw [color=\currentcolor!80, domain=1:4, semithick, smooth] plot  (\x,{(\x)^2 -4*\x+5}) node [color=\currentcolor!80, left, xshift =-0.5cm, yshift=-1cm , scale=0.6] {$v(t)=t^2 - 4t +5$};
\node [ponto, fill=\currentcolor!80, scale=0.7] at (1,2) {} node at (1,2) [above right, scale=0.5] {(1,2)};
\node [ponto, fill=\currentcolor!80, scale=0.7] at (4,5) {} node at (4,5) [above right, scale=0.5] {(4,5)};



\end{tikzpicture}\hspace{5em}
\begin{tikzpicture}[scale=1, every node/.style={scale=1.5}]

\fill [\currentcolor!50, opacity=0.3, domain =1:4, variable=\x] (1,0) -- plot  (\x,{(\x)^2 -4*\x+5}) -- (4,0) -- cycle;


\draw [->] (-.1,0) -- (5.5,0) node [above left, scale=0.6] {$t$};
\draw [->] (0,-.1) -- (0,5.5) node [below right, scale=0.6] {$v$};

\foreach \x in {1,...,5} {\draw (\x,.05) -- (\x,-.05)  node [scale=0.5, below] {\x}; \draw (.05,\x) -- (-.05,\x) node [scale=0.5, left] {\x};
}
\node [below left, scale=0.5] at  (-.05,-.05)  {0};

\draw [color=\currentcolor!80, domain=1:4, semithick] plot  (\x,{(\x)^2 -4*\x+5}) node [color=\currentcolor!80, left, xshift =-0.5cm, yshift=-1cm , scale=0.6] {$v(t)=t^2 - 4t +5$};
\node [ponto, fill=\currentcolor!80, scale=0.7] at (1,2) {} node at (1,2) [above right, scale=0.5] {(1,2)};
\node [ponto, fill=\currentcolor!80, scale=0.7] at (4,5) {} node at (4,5) [above right, scale=0.5] {(4,5)};


\end{tikzpicture}\end{figure}

É um fato conhecido da física que a distância percorrida pelo veículo entre os instantes \(t = 1\) segundo e \(t = 4\) segundos é dado pela área da região limitada pelo gráfico da função velocidade, pelo eixo \(t\) e pelas retas verticais \(t = 1\) e \(t = 4\) (região hachurada na figura).
\begin{enumerate}
\item {} 
Obtenha um método para aproximar a distância percorrida com erro tão pequeno quanto desejado.

\item {} 
Reveja a atividade anterior e busque argumentar pela validade do Princípio de Cavalieri usando a estratégia adotada no item a) desta atividade.

\end{enumerate}
\end{task}

\begin{task}{volume de concreto de uma barragem}

(Derivada da atividade Staumauer de \href{https://www.geogebra.org/u/lindner}{Andreas Lindner} disponível nos materiais do GeoGebra)



A figura mostra um modelo de barragem que pretende-se construir em concreto. Para isso será necessário conhecer o volume da barragem pronta. O responsável pelo projeto informou que ela tem 92m de comprimento, 24m de altura, que todas as seções transversais são retângulos, que ao pé da parede a largura é de 16m, e no topo é de 4,67m de largura e que a espessura da parede aumenta exponencialmente para cima. Garanta que não faltará concreto para a construção! Melhor sobrar do que faltar.

\begin{figure}[H]
\centering

\noindent\includegraphics[width=400bp]{{105}.png}
\end{figure}
\begin{enumerate}
\item {} 
Obtenha a função que a cada altura z da parede associa a largura da parede naquela altura.

\item {} 
Obtenha uma estimativa inicial para o volume da barragem. A ordem de grandeza basta aqui.

\item {} 
Obtenha uma aproximação melhor que a anterior. Tente fazer de modo que possa ser estabelecido um algoritmo que sirva para o cálculo de aproximações sucessivas.

\end{enumerate}
\end{task}


\arrange{Princípio de Cavalieri}
\label{\detokenize{GE504-9:organizando-as-ideias-principio-de-cavalieri}}\label{\detokenize{GE504-9::doc}}
A imagem apresenta duas retas tracejadas paralelas, um retângulo (à esquerda), um paralelogramos não retângulo ao centro e uma outra figura formada por duas linhas curvas e dois segmentos de reta sobre as retas tracejadas. Em todas elas, se traçarmos uma reta paralela às retas tracejadas obteremos segmentos de comprimento 3cm na região por elas limitada.

\begin{figure}[H]
\centering

\begin{tikzpicture}%[scale=, every node/.style={scale=}]

\draw (0,1) -- (3,1) node [midway, scale=0.8, below] {3};
\draw [xshift=4cm] (0.5,1) -- (3.5,1) node [midway, scale=0.8, below] {3};
\draw [xshift =10cm](0.745,1) -- (3.745,1) node [midway, scale=0.8, below] {3};

\draw [, color=\currentcolor](0,0) rectangle (3,4);
\draw [xshift=4cm, color=\currentcolor] (0,0) -- (3,0) -- (5,4) -- (2,4) -- cycle;
\draw [xshift =10cm, color=\currentcolor] (0,0) -- (3,0) .. controls (4,0.5) and (4,1.5) .. (3,2) .. controls (2,2.5) and (2,3.5) .. (3.5,4) -- (0.5,4) .. controls (-1, 3.5) and (-1, 2.5) .. (0,2) .. controls (1,1.5) and (1,0.5) .. (0,0)
;



\end{tikzpicture}
\end{figure}

As áreas dos dois quadriláteros são iguais pois ambos são paralelogramos de mesma base e mesma altura. Como você já deve imaginar da discussão as atividades iniciais desta seção a área da terceira região também coincide com as duas anteriores. Para esta e outras situações usamos o Princípio de Cavalieri.

\begin{observationtitle}{Princípio de Cavalieri (versão do plano):} Suponha que duas regiões em um plano estão compreendidas entre duas retas paralelas. Se toda reta paralela a essas duas retas intersecta as regiões em segmentos de comprimentos iguais, então as duas regiões têm áreas iguais.
\end{observationtitle}

A versão tridimensional é inteiramente análoga, mas trata das áreas das seções e volumes dos sólidos onde a versão plana trata de comprimentos das seções e áreas das regiões.

Todos concordamos que pilhas de formas diferentes formadas com as mesmas peças, têm o mesmo volume. O Princípio de Cavalieri é a situação limite deste argumento quando as alturas das peças tende a zero. \href{https://ggbm.at/kdzfw7xd}{Neste aplicativo} você pode manipular os sólidos e melhorar a sua visualização.

\begin{figure}[H]
\centering

\noindent\includegraphics[width=300bp]{{107}.png}
\end{figure}

\begin{observationtitle}{Princípio de Cavalieri (versão do espaço):} Suponha que dois sólidos no espaço estão compreendidos entre planos paralelos. Se todo plano paralelo a estes dois planos intersectar os sólidos em regiões de áreas iguais, então os dois sólidos têm volumes iguais.
\end{observationtitle}

Uma primeira aplicação do Princípio de Cavalieri é o cálculo do volume de cilindro oblíquos. Em um cilindro oblíquo todas as seções por um plano paralelo às bases resultam em regiões congruentes às bases. Por isso, e pelo Princípio de Cavalieri, o volume de qualquer cilindro oblíquo coincide com o volume do cilindro reto de mesma área da base e mesma altura. Portanto, o volume dos cilindros oblíquos também são dados por área da base vezes altura. A figura mostra o caso particular em que este cilindro oblíquo tem base triangular.

\begin{figure}[H]
\centering

\noindent\includegraphics[width=300bp]{{108}.png}
\end{figure}

\textbf{Atenção:} Não confunda a altura do prisma oblíquo com o comprimento de suas arestas laterais.

\clearpage
\def\currentcolor{session2}
\begin{objectives}{Reflexões sobre perímetros e áreas de triângulos}
{
\textbf{Objetivos específicos:}

OE23. Entender o princípio de Cavalieri.

\textbf{Conceitos abordados:} semelhança de triângulos, perímetro de triângulos, área de triângulos, relação entre áreas de triângulos semelhantes, retas paralelas, Princípio de Cavalieri.
}{1}{1}
\end{objectives}
\begin{sugestions}{Reflexões sobre perímetros e áreas de triângulos}
{
\textbf{Sugestões gerais:} Se possível, mostre o aplicativo do link usando um projetor ou permita que os estudantes acessem o aplicativo de seus smartphones. Nele será possível arrastar os pontos azuis das figuras e experimentar diferentes posições antes de resolver a atividade.

A atividade ainda pode ser desenvolvida em sala de aula mesmo sem o uso deste recurso.

\textbf{Atividade relacionada:} Embora esta atividade tenha um interesse próprio do ponto de vista do pensamento matemático, ela serve de preparação e estabelecimento da linguagem e ideias necessárias para a Atividade: volume de pirâmides.

\textbf{Links relacionados:} Versão digital desta atividade - \href{https://ggbm.at/vhxy7mp3}{Parte II}
}{1}{1}
\end{sugestions}
\clearmargin
\begin{objectives}{Volume da pirâmide}
{
\textbf{Objetivos específicos:}

OE24. Entender como utilizar o princípio de Cavalieri no cálculo de volumes de prismas e cilindros oblíquos.
}{1}{2}
\end{objectives}
\begin{sugestions}{Volume da pirâmide}
{
\textbf{Dificuldades previstas:} Os estudantes provavelmente terão dificuldades em mostrar a semelhança \(XYZ \sim ABC\). Caso decida dar alguma dica, sugerimos que vá oferecendo conforme a necessidade dos estudantes o seguinte passo a passo:
\begin{enumerate}
\item {} 
Observe que \(XY\), \(YZ\) e \(ZX\) são paralelas a \(AB\), \(BC\) e \(CA\), respectivamente.

\item {} 
Explique que \(VXY \sim VAB\), \(VYZ \sim VBC\) e \(VZX \sim VCA\).

\item {} 
Explique por que \(VWX \sim VDA\) com razão de semelhança \(h / H\).

\item {} 
Conclua que as razões de semelhança do item b) também são h / H.

\item {} 
Conclua que \(VX = (h/H) VA\), \(VY = (h/H) VB\) e \(VZ = (h/H) VC\), de onde segue a semelhança.

\end{enumerate}

\textbf{Enriquecimento da discussão:} O conhecimento do Teorema de Tales no espaço poderia simplificar um pouco as contas do item a) da atividade, veja A Matemática do Ensino Médio, vol. 2, SBM.

\textbf{Links relacionados:} Versão digital desta atividade - \href{https://ggbm.at/zzcvwr9m}{Parte I} e  \href{https://ggbm.at/sq7h5tjr}{Parte II}
}{1}{2}
\end{sugestions}
\clearmargin
\clearmargin
\clearmargin
\begin{objectives}{Volume da esfera}
{
OE21. Entender a demonstração da expressão para cálculo do volume da esfera usando o Princípio de Cavalieri.
}{1}{1}
\end{objectives}
\begin{sugestions}{Volume da esfera}
{
\textbf{Enriquecimento da discussão:} O conhecimento do Teorema de Tales no espaço poderia simplificar um pouco as contas do item a) da atividade, veja A Matemática do Ensino Médio, vol. 2, SBM.

\textbf{Links relacionados:} \href{https://ggbm.at/dcetmq2g}{Versão digital} desta atividade Vídeo da UNICAMP \url{https://www.youtube.com/watch?time\_continue=566\&v=2pP9aR4nkQc}
}{1}{1}
\end{sugestions}


\practice{Princípio de Cavalieri}
\begin{task}{reflexões sobre perímetros e áreas de triângulos}

\paragraph{Parte 1}

Na figura, as retas \(r\) e \(s\) são paralelas e os segmentos \(BC\) e \(B'C'\) são congruentes.

\begin{figure}[H]
\centering

\begin{tikzpicture}[scale=1.35, every node/.style={scale=2}]

\draw [fill=\currentcolor!80, color=\currentcolor, opacity=0.4] (1,0) -- (1.75,3) -- (2.5,0) -- cycle;

\draw [fill=\currentcolor!80, color=\currentcolor, opacity=0.4] (3,0) -- (4.75,3) -- (4.5,0) -- cycle;

\foreach \x in {(1,0),(1.75,3),(2.5,0),(3,0),(4.5,0)} \node [ponto] at \x {};

\foreach \x/\y/\z in {(1,0)/B/below,(1.75,3)/A/above, (2.5,0)/C/below,(3,0)/B'/below,(4.5,0)/C'/below} \node  [\z, scale=0.5]  at \x {$\y$};

\node [ponto, color=destacado] at (4.75,3) {} node [color=destacado, above, scale=0.5] at (4.75,3) {$A'$};

\draw (0,0) -- (5,0) node [above right, pos=0, scale=0.5] {$s$};
\draw (0,3) -- (5,3) node [above right, pos=0, scale=0.5] {$r$};

\end{tikzpicture}
\end{figure}
\begin{enumerate}
\item {} 
Qual dos triângulos têm a maior área, \(ABC\) ou \(A'B'C'\)? Explique a sua resposta.

\item {} 
Dentre todos os triângulos que se pode formar movendo A’  sobre a reta \(r\), qual deles tem menor perímetro? Justifique. E o de perímetro máximo?

\end{enumerate}

\paragraph{Parte 2}

Seja \(H\) a distância entre as retas \(r\) e \(s\) e considere uma reta \(t\) paralela a \(r\) e a \(s\), que dista \(h\) de \(r\) e intersecta os lados dos triângulos em \(XY\) e \(X'Y'\) como na figura.

\begin{figure}[H]
\centering

\begin{tikzpicture}[scale=1.35, every node/.style={scale=2}]

\draw [fill=\currentcolor!80, color=\currentcolor, opacity=0.4] (1,0) -- (1.75,3) -- (2.5,0) -- cycle;

\draw [fill=\currentcolor!80, color=\currentcolor, opacity=0.4] (3,0) -- (4.75,3) -- (4.5,0) -- cycle;

\foreach \x in {(1,0),(1.75,3),(2.5,0),(3,0),(4.5,0)} \node [ponto] at \x {};

\foreach \x/\y/\z in {(1,0)/B/below,(1.75,3)/A/above, (2.5,0)/C/below,(3,0)/B'/below,(4.5,0)/C'/below} \node  [\z, scale=0.5]  at \x {$\y$};

\node [ponto, color=destacado] at (4.75,3) {} node [color=destacado, above, scale=0.5] at (4.75,3) {$A'$};

\draw (0,0) -- (7,0) node [above right, pos=0, scale=0.5] {$s$};
\draw (0,3) -- (7,3) node [above right, pos=0, scale=0.5] {$r$};

\draw [help lines] (0,2) -- (7,2);
\draw [help lines] (5.5,3) -- (5.5,2) node [midway, right, scale=0.5, black] {$h$};
\draw [help lines] (6.5,3) -- (6.5,0) node [midway, right, scale=0.5, black] {$H$};
\node [ponto, atento, scale=0.7] at (6.5,2) {};

\draw [destacado, thick] (1.5,2) -- (2,2) node [pos=0, ponto] {} node [pos=0, above left, scale=0.5, black] {$X$} node [pos=1, ponto] {} node [pos=1, above right, scale=0.5, black] {$Y$};

\draw [destacado, thick] (4.17,2) -- (4.67,2) node [pos=0, ponto] {} node [pos=0, above left, scale=0.5, black] {$X'$} node [pos=1, ponto] {} node [pos=1, above right, scale=0.5, black] {$Y'$};
\end{tikzpicture}
\end{figure}
\begin{enumerate}
\item {} 
Mostre que, seja lá qual for a distância \(h\), os segmentos \(XY\) e \(X'Y'\) são congruentes.

\item {} 
Seja \(\mathcal{A}\) = Área(\(ABC\)), use o Princípio de Cavalieri para calcular a Área(\(A'B'C'\)).

\item {} 
Calcule Área(\(AXY\)) / Área(\(ABC\)) em termos de \(h\) e de \(H\).

\end{enumerate}
\end{task}

\begin{task}{volume da pirâmide}

\paragraph{Parte 1}

O tetraedro de vértice \(V\) e base \(ABC\) da figura possui arestas de comprimentos \(AB\), \(BC\), \(CA\), \(VA\), \(VB\) e \(VC\) respectivamente iguais a 8, 9, 10, 18, 18, 19 cm e altura \(H\) cm. Foi traçado um plano paralelo ao plano \(ABC\) a uma distância \(h\) do ponto \(V\) intersectando o tetraedro no triângulo \(XYZ\) como na figura.

\begin{figure}[H]
\centering

\noindent\includegraphics[width=350bp]{{112}.png}
\end{figure}
\begin{enumerate}
\item {} 
Explique por que o triângulo \(XYZ\) é semelhante ao triângulo \(ABC\) com razão de semelhança \(h / H\).

\item {} 
Calcule a razão Área(\(XYZ\)) / Área(\(ABC\)) em função de \(h\) e \(H\).

\end{enumerate}

\paragraph{Parte 2}

Dois tetraedros com áreas iguais em suas bases e alturas iguais têm volume iguais.

Os tetraedros da figura têm bases \(ABC\) e \(A'B'C'\) de mesma área e possuem alturas iguais. Nesta atividade você vai justificar que eles possuem volumes iguais mesmo sem conhecer a fórmula para o volume de um tetraedro.

\begin{figure}[H]
\centering

\noindent\includegraphics[width=350bp]{{113}.png}
\end{figure}
\begin{enumerate}
\item {} 
Assim como na parte anterior, os triângulos \(XYZ\) e \(X'Y'Z'\) são determinados pela interseção do tetraedro original por um plano paralelo aos planos das bases que dista \(h\) dos vértices. Explique por que os triângulos \(XYZ\) e \(X'Y'Z'\) têm áreas iguais.

\item {} 
Use o Princípio de Cavalieri para explicar por que os tetraedros \(V-ABC\) e \(V-A'B'C'\) têm volumes iguais.

\end{enumerate}

\paragraph{Parte 3}

Qualquer tetraedro é parte de um prisma triangular formado por outros dois tetraedros de mesmo volume que o tetraedro original.
\begin{enumerate}
\item {} 
Use o aplicativo \href{https://ggbm.at/shmk9dkj}{deste link} para montar o prisma triangular com os três sólidos apresentados.

\item {} 
Como você nomearia estes sólidos dados?

\item {} 
Explique por que os sólidos dados têm mesmo volume (reveja os resultados dos itens anteriores, se necessário).

\end{enumerate}

\paragraph{Parte 4}

A sequência de figuras, apresenta uma demonstração sem palavras de um fato matemático.

\begin{figure}[H]
\centering

\noindent\includegraphics[width=430bp]{{114-115---121}.png}
\end{figure}

Portanto,

\begin{figure}[H]
\centering

\noindent\includegraphics[width=400bp]{{122123}.png}
\end{figure}
\begin{enumerate}
\item {} 
Que fato matemático está sendo justificado na sequência de figuras? Em que sentido as igualdades são verdadeiras?

\item {} 
Explique a construção realizada em cada um dos passos. Explique com cuidado especial as “igualdades” entre os tetraedros.

\end{enumerate}
\end{task}

\clearpage
\begin{task}{volume da esfera}

A figura mostra um hemisfério de raio \(r\) (metade de uma bola) e um cilindro de raio e altura iguais a \(r\) de onde foi removido um cone de mesma base e altura que o cilindro (chamaremos este sólido de anticlépsidra).

\begin{figure}[H]
\centering

\noindent\includegraphics[width=200bp]{{124}.png}
\end{figure}
\begin{enumerate}
\item {} 
Descreva a figura formada na seção da bola por um plano que está a uma distância \(h\) do centro.

\item {} 
Descreva a figura formada na seção da anticlépsidra por um plano que está a uma distância \(h\) do plano da base.

\item {} 
As seções têm mesma área?

\item {} 
Explique por que o volume da esfera de raio \(r\) é \(4/3 \pi.r^3\). Você pode usar aqui que o volume do cone é (1/3).(Área da base) x (altura) e que o volume do cilindro é (Área da base) x (altura).

\end{enumerate}
\end{task}


\exercise{}
\label{\detokenize{GE504-E:exercicios}}\label{\detokenize{GE504-E::doc}}\begin{enumerate}
\item {} 
(OBMEP 2018) Alice colocou um litro (1000 \(cm^3\)) de água em uma jarra  e  mediu  o  nível  da  água.  Depois  ela  colocou  um  objeto  maciço  de  prata  na  jarra  e  mediu  novamente  o  nível  da  água, conforme a figura. A massa de um centímetro cúbico de prata é 10,5 gramas. Qual é a massa desse objeto?

\end{enumerate}

\begin{figure}[H]
\centering

\noindent\includegraphics[width=200bp]{{Screenshot_from_2018-12-07_21-01-39}.png}
\end{figure}
\begin{enumerate}
\item {} 
Considere duas garrafas, uma com água e outra com óleo, e dois cubos visualmente idênticos (com as mesmas dimensões), um de aço e outro de chumbo. Ao submergir os cubos, um em cada garrafa, qual líquido desloca mais, a água ou o óleo?

\item {} 
Um copo possui marcas de arroz e de farinha com numerações em níveis diferentes. Porém os números não possuem unidades associadas. Esses números podem corresponder a medidas de volume? E de massa?

\item {} 
Use um copo medida graduado de cozinha para estimar o volume de um ovo. Descreva a sua estratégia, justificando sua validade. Essa experiência também permite estimar a massa do ovo.

\item {} 
Um objeto qualquer que tenha seu volume alterado terá necessariamente também sua massa alterada? Justifique sua resposta.

\item {} 
Um conhecido quebra cabeça é feito a partir de 27 pequenos cubos presos por um fio (Figura x) que podem ser organizados como um único cubo maior (Figura y). Relacione o volume do quebra cabeça nas duas configurações apresentadas: desmontado e montado.

Figura Figura

\item {} 
Um cilindro de gás do tamanho de uma pessoa foi capaz de encher balões suficientes para preencher um cômodo inteiro. Por que não houve conservação de volume?

\item {} 
Fechamos duas garrafas de refrigerante consumidas até a metade por um longo período. Uma delas foi fechada somente colocando a tampa e outra amassando a garrafa antes de tampá-la. Após alguns dias em qual das garrafas o líquido restante perdeu mais gás?

\item {} 
Na feira há duas barracas que vendem feijão de corda. Na primeira barraca vende-se a medida de uma lata vazia de leite condensada por R\$ 2,00. Na segunda o preço de cada medida é de R\$ 3,00, mas neste caso a medida é uma cambuca de plástico. Escolha qual das seguintes estratégias permitiria determinar com precisão o preço de um kilo de feijão em cada barraca.
\#. Comprar R\$ 10,00 em cada barraca e comparar o tamanho das porções.
\#. Comprar uma medida em cada barraca e pesar as porções numa balança e subtrair os resultados.
\#. Comprar uma medida em cada barraca e pesar as porções numa balança e dividir cada resultado pelo preço.
\#. Comprar uma medida em cada barraca, pesar as porções numa balança e dividir o preço pelo peso de cada porção.
Fonte: \url{https://www.directoalpaladar.com.mx/ingredientes-y-alimentos/lo-que-necesitas-saber-del-tofu}

\item {} 
Ligamos para dois fornecedores de feijões e queremos decidir de qual deles compraremos. O primeiro nos fornece o preço medido em termos de garrafas pet de refrigerante. O segundo, contudo nos informa o preço por sacos de feijão. Qual pergunta poderíamos fazer ao segundo fornecedor para tomar a decisão: quantos quilos tem cada saco? ou quantos litros tem cada saco?

\item {} 
A figura a seguir representa um cubo formado por cubos menores. Quantos cubos menores são necessários para formar o cubo da figura?
FIGURA

\item {} 
Qual é a(s) dimensão(ões) relevante para a compra de uma corda? E para a compra de um tecido? E para a compra de gás ou gasolina?

\item {} 
Quantos cubos foram usados para o arranjo tridimensional da figura a seguir?
FIGURA

\item {} 
Uma tinta deve ser aplicada com espessura de 0.1 mm em uma parede retangular de 3m de altura e 6m de largura. Qual a quantidade mínima de tinta que deve ser comprada?

\item {} 
(OBMEP 2017) Vários quadrados foram dispostos um ao lado do outro, em ordem crescente de tamanho, formando uma figura com 100cm de base. O lado do maior quadrado mede 20cm. Qual é o perímetro da figura formada por esses quadrados?

FIGURA

\item {} 
São dadas peças de tamanho 2 por 3 para cobrir um retângulo 5 por 7, como na figura.

FIGURA
\begin{enumerate}
\item {} 
Faça uma figura da cobertura sem sobreposição indicando os cortes necessários em cada peça.

\item {} 
Preencha a tabela com os cortes

\end{enumerate}

\begin{table}[H]
\centering
\begin{tabular}{|c|c|c|c|}
\hline
\tcolor{Tipo de corte} & \tcolor{Área da peça} & \tcolor{núm. de peças usadas} & \tcolor{área acumulada} \\
\hline
figura & & & \\
\hline
figura & & & \\
\hline
figura & & & \\
\hline
figura & & & \\
\hline
\end{tabular}
\end{table}

\item {} 
A partir do desenho da planta de organização de caixas de produtos num estante industrial, a altura da pilha e o número de produtos em cada caixa; deduzir o número total de produtos transportados na estante.

FIGURA

\item {} 
Em quais das figuras a seguir o volume destacado é de \(\frac{1}{2}m^3\)?

\item {} 
Um jogo de blocos de montar possui blocos em forma de paralelepípedos de lados: 2, 4 e 5. Qual é o lado do menor cubo que podemos construir com esses blocos?

\item {} 
Assinale as características mínimas que precisamos conhecer para determinarmos um prisma a menos de congruência.
* altura,
* número de lados do polígono da base,
* área da base,
* uma aresta lateral,
* todas as arestas laterais,
* comprimentos dos lados do polígono da base,
* ângulo entre uma aresta lateral e o plano de uma das bases.
* ângulos internos do polígono da base.
* polígono da base.

\item {} 
Quais dos cilindros a seguir estão determinados a menos de congruência.
\begin{enumerate}
\item {} 
Figura com cilindro circular reto com raio e altura dados.

\item {} 
Figura com cilindro oblíquo com ângulo do eixo com o plano e altura dados.

\item {} 
Figura com cilindro circular reto com área da base e área lateral dados.

\item {} 
Figura com cilindro circular reto com planificação e as dimensões do retângulo formado na planificação.

\item {} 
Figura com cilindro circular oblíquo com raio e comprimento  do eixo dados.

\end{enumerate}

\item {} 
(Enem 2001)  Um fabricante de brinquedos recebeu o projeto de uma caixa que deverá conter cinco pequenos sólidos, colocados na caixa por uma abertura em sua tampa. A figura representa a planificação da caixa, com as medidas dadas em centímetros.

\end{enumerate}


\ifnum\aluno=1
\clearpage
\else
\notasfinais
\fi

% 
% \bibliography{../Bibliografia/perspectiva1_bibliografia.bib}

\nocite{*}
% \documentclass[extrafontsizes, twoside, 11pt, openright, final]{memoir}


\usepackage{../../livroaberto-html}


\begin{document}

% \chapterillustration{abertura-funcoes}{abertura-funcoes-professor}

% \chapterwhat{
% 	Funções e suas diferentes representações (numérica, algébrica e gráfica); domínio, contradomínio e imagem; aplicações em situações envolvendo a análise, interpretação e resolução de problemas em contextos diversos.
% }

% \chapterbecause{
% 	Funções são objetos matemáticos que nos permitem compreender como a variação de uma grandeza influencia na variação de outra. Por isso elas são ferramentas essencias para a compreensão, análise e tomada de decisão em diversas situações do nosso dia a dia.

% 	O estudo de funções nos permite, por exemplo, relacionar a área de um polígono com o comprimento de seus lados, a distância percorrida por um objeto com o intervalo de tempo gasto no percurso e o valor da conta de energia elétrica com o consumo de energia.

% 	De um modo mais geral, funções são úteis para o estudo do crescimento populacional, disseminação de doenças, lançamento de foguetes e satélites, interpretação de exames médicos, etc.
% }

\chapter{Introdução às Funções\label{chap-funcoes}}



\explore{Conceito de Função}

O que o nosso batimento cardíaco, um terremoto ou a variação das ações de uma empresa na bolsa de valores possuem em comum? Os batimentos cardíacos podem ser monitorados a partir de um sinal bioelétrico cujo gráfico é representado em um eletrocardiograma, as ondas sísmicas produzidas por um terremoto podem ser observadas a partir do registro de um sismógrafo e as variações dos valores das ações de uma empresa percebidas ao longo do tempo podem ser facilmente visualizadas em um gráfico.

\begin{figure}[H]
	\begin{center}
		\centering

		\noindent\includegraphics[width=\textwidth]{sismografo_2.png}
	\end{center}
\end{figure}

Como nos fenômenos descritos acima, muitas situações e decisões do dia a dia dependem do reconhecimento de uma relação entre duas grandezas e da análise de como a variação de uma delas influencia na variação da outra (Por exemplo, a distância percorrida e o tempo transcorrido, a área de um polígono e o comprimento de seus lados, a absorção de um medicamento pelo organismo humano e o tempo desde a sua ingestão, valor da conta de energia elétrica e consumo, quantidade de vereadores e a população etc). O tema funções trata da relação entre grandezas, identificando um tipo especial de relação. Funções são uma ferramenta matemática importante para descrever, analisar e tomar decisões em diversas situações.

As funções, de maneira geral, conectam grandezas, medidas, conjuntos numéricos e até variáveis que não podem ser quantificadas, ou seja, não numéricas, como, por exemplo, as variáveis qualitativas estudadas pela Estatística (classe social, cor dos olhos, local de nascimento, gênero etc).

Função é um dos conceitos centrais da Matemática, e sua importância transcende os limites dessa ciência, sendo fundamental para descrever fenômenos em diversas áreas do conhecimento, não só nas mais próximas, como a Física, a Química, ou as Engenharias como também em Biologia, Geografia, Sociologia, e em situações cotidianas diversas, como será exemplificado nas atividades a seguir.

A noção de função não surgiu ao acaso. É um instrumento matemático indispensável para o estudo quantitativo dos fenômenos naturais, tendo sua origem nos estudos desenvolvidos por Kepler (1571--1630) e Galileu (1564--1642) sobre os movimentos dos planetas e a queda dos corpos pela ação da força da gravidade, respectivamente.  Nesses estudos era preciso medir grandezas, identificar regularidades e obter relações que oferecessem uma descrição matemática simples.

A aplicação da Matemática nas mais diversas áreas é feita, na maioria das vezes, por meio da noção de modelo matemático. Um modelo matemático permite representar uma determinada situação ou fenômeno a partir de variáveis e de relações entre essas variáveis. Portanto, funções são fundamentais tanto na concepção e construção de um modelo matemático como no estudo desses modelos.


\begin{task}{ Pluviometria no Sistema Cantareira}
	\label{\detokenize{AF106-0:atividade-pluviometria-no-sistema-cantareira}}\label{\detokenize{AF106-0:ativ-funcoes-pluviometria}}

	As chuvas são a principal fonte de água para os reservatórios que abastecem as grandes cidades. Com base em dados passados, constrói-se uma média mensal esperada de chuvas. Em períodos em que a chuva real é menor do que o esperado pode-se observar uma diminuição da quantidade de água armazenada no sistema.

	O gráfico a seguir apresenta a variação pluviométrica (em milímetros) da chuva real e da chuva esperada no Sistema Cantareira, que abastece a região metropolitana de São Paulo, no período de dezembro de 2013 (2013-12) a novembro de 2016 (2016-11).

	\begin{figure}[H]
		\begin{center}
			\includegraphics[width=\textwidth]{funcoesaluno-figure0.pdf}
		\end{center}
	\end{figure}

	De acordo com o gráfico acima:
	\begin{enumerate}
		\item Que grandezas estão sendo relacionadas?
		\item Em que mês e ano houve a maior incidência de chuvas? E a menor?
		\item Em que período(s) a diferença entre a quantidade de chuva esperada e a quantidade real de chuva superou $100$mm?
		\item Houve algum mês em que não foi registrada chuva na região do Sistema Cantareira?
		\item O que pode ser observado nos meses de agosto de 2015 e março de 2016?
	\end{enumerate}
\end{task}


\begin{task}{ Números triangulares}
	\label{numeros-triangulares-funcoes}
	\begin{center}
		\includegraphics[width=.8\textwidth]{funcoesaluno-figure1.pdf}
	\end{center}


	Considere a sequência de números ilustrada acima. Ela é conhecida como a sequência dos \emph{números triangulares}. O $n$-ésimo número triangular, $T_n$, é igual a quantidade total de círculos congruentes necessários para formar um triângulo equilátero cujo lado tem $n$ círculos. Por exemplo, o quarto número triangular é $T_4=10$, porque são necessários $10$ círculos congruentes para formar um triângulo cujo lado tem, $4$ desses círculos.
	\begin{enumerate}
		\item Determine o 6º, o 7º e o 8º números triangulares.

		\item Descreva o procedimento que você usou para determinar $T_6$, $T_7$ e $T_8$ no item anterior.

		\item Determine o milésimo número triangular, $T_{1000}$.

		\item Descreva um procedimento que permita determinar qualquer número triangular a partir da sua ordem na sequência? Explique.

		\item Quais são as variáveis relacionadas?

	\end{enumerate}
\end{task}

\clearpage
\begin{task}{Arranha-céu}\label{ativ-arranha}

	Imagine um arranha-céu de $40$ andares cujas diferentes alturas que correspondem a alguns andares estão representadas na tabela abaixo.

	\begin{table}[H]
		\centering

		\begin{tabular}{|c|c|*{9}{>{\begin{center}\arraybackslash}p{.5cm}<{\end{center}}@{}|}}
			\hline
			\hline
			\tcolor{Número do Andar} & Garagem (0) & 1 & 2 & 3  & 4  & … & 10 & … &  &    \\
			\hline
			\tcolor{Altura (metros)} & -1          & 3 & 7 & 11 & 15 & … &    & … &  & 91 \\
			\hline
		\end{tabular}
	\end{table}


	Considere que a altura de um andar é medida a partir do nível da rua até o piso desse andar e que a altura entre os andares seja sempre a mesma, conforme o esquema abaixo.

	\begin{figure}[H]
		\begin{center}
			\centering

			\noindent\includegraphics[width=.4\textwidth]{{Arranha-ceu_1}.png}
		\end{center}
	\end{figure}
	\begin{enumerate}
		\item Qual a altura entre os andares?

		\item Qual a altura  do 10º andar?

		\item O que significa o sinal negativo do andar da garagem?

		\item A que andar corresponde a altura de 91 m?

		\item Qual é a altura total desse prédio?

		\item Realize uma pesquisa na internet e descubra o maior arranha-céu brasileiro atualmente. Dividindo a altura total desse arranha-céu pela quantidade de andares, determine a altura média de um andar.

	\end{enumerate}
\end{task}



\arrange{Conceito de Função}
\label{\detokenize{AF106-1:sec-funcao-organizando-ideias-conceito}}\label{\detokenize{AF106-1::doc}}\label{\detokenize{AF106-1:organizando-as-ideias-conceito-de-funcao}}
Vamos identificar juntos quais são as características comuns presentes em cada uma das situações anteriores. Em todas elas há pelo menos dois conjuntos bem determinados cujos elementos estão sendo relacionados. Nessa relacão, \textbf{cada} elemento de um desses conjuntos está associado a um \textbf{único} elemento do outro conjunto.

Na {\hyperref[\detokenize{AF106-0:ativ-funcoes-pluviometria}]{Atividade: Pluviometria no Sistema Cantareira}}, um dos conjuntos se refere ao tempo e é determinado pelos meses do ano, no período de dezembro de 2015 a novembro de 2016. O outro é um conjunto numérico que deve conter todos os possíveis valores para o índice pluviométrico do Sistema Cantareira em milímetros. A relação representada no gráfico pela linha azul associa a cada ano-mês o índice de chuva real naquele período. Já a relação representada pela linha vermelha associa a cada mês-ano o índice de chuva esperada naquele período. Observe que, em ambos os casos, para cada mês-ano é associado um único índice pluviométrico.

Na {\hyperref[\detokenize{AF106-4:ativ-funcoes-numeros-triangulares}]{Atividade: números triangulares no plano}}, um dos conjuntos tem como elementos as ordens dos termos da sequência, indicadas de maneira geral por $n$. O outro conjunto deve conter todos os possíveis números triangulares $T_n$. Assim, a cada ordem $n$ está associado, sem ambiguidade, o número triangular $T_n$.

Por fim, na {\hyperref[\detokenize{ativ-arranha}]{Atividade: Arranha-céu}} temos cada andar do prédio sendo relacionado com sua altura até o nível da rua.

Nas três relações apresentadas, \textbf{cada} elemento de um conjunto $A$ está associado a um \textbf{único} elemento de um conjunto $B$. Uma relação com essas propriedades é chamada \textbf{função}.
\begin{description}
	\item[{Função\index{Função|textbf}}] \leavevmode\phantomsection\label{\detokenize{AF106-1:term-funcao}}
		Dizemos que uma relação $f$ entre os elementos de dois conjuntos não vazios, $A$ e $B$, é uma função de $A$ em $B$ se \emph{todo} elemento do conjunto $A$ estiver relacionado a um \emph{único} elemento do conjunto $B$.

\end{description}

Assim, para cada $x\in A$ deve existir um único elemento $y\in B$ que está associado a $x$ pela função $f$. Esse elemento $y$ é também denotado por $f(x)$:

\begin{figure}[H]
	\begin{center}
		\includegraphics[width=.4\textwidth]{funcoesaluno-figure2.pdf}
	\end{center}
\end{figure}

O conjunto $A$ é chamado \index{domínio da função}domínio da função $f$, o conjunto $B$ é chamado \index{contradomínio}contradomínio de $f$ e o subconjunto de $B$ formado pelas imagens de todos os elementos de $A$ é chamado \index{conjunto imagem}conjunto imagem da função $f$.

\begin{figure}[H]
	\begin{center}
		\includegraphics[width=.5\textwidth]{funcoesaluno-figure3.pdf}
	\end{center}
\end{figure}
De maneira geral, escreve-se:
\begin{equation*}
	\begin{split}f:A \to B \\
		x \mapsto f(x)\end{split}
\end{equation*}

Por exemplo, na {\hyperref[\detokenize{AF106-0:ativ-funcoes-pluviometria}]{Atividade: Pluviometria no Sistema Cantareira}}, se $f$ é a função que associa a cada ano-mês o índice de chuva real naquele período, $f(2014-3)=200$ nos informa que o índice de chuva real observada na região do sistema Cantareira no mês de março do ano de 2014 foi de $200$ milímetros.

Em uma função $f$ de $A$ em $B$, a dependência estabelecida entre as variáveis $x \in A$ e $y \in B$ permite que $y$ seja identificada como “variável dependente” e $x$ como  “variável independente”, uma vez que os valores assumidos por $y$ são determinados em função da variação de $x$ no domínio. Na atividade “Arranha-céu” por exemplo, a variável independente é aquela que representa os andares e a variável dependente é a altura do andar.

\begin{observation}
	A definição de uma função $f$ de $A$ em $B$ exige que a cada elemento $x\in A$ corresponda uma imagem $y=f(x)\in B$ e que não haja ambiguidade na determinação dessa imagem, ou seja, que ela seja única. Asssim, nem toda relação de $A$ em {\color{red}\bfseries{}{}`}B é uma função. Por exemplo, a relação que associa a cada pessoa o número de seu telefone não é função, pois a imagem pode não ser única, ou seja, há ambiguidade: algumas pessoas têm mais de um número de telefone. E além disso, nem todas as pessoas têm telefone.
\end{observation}

\begin{reflection}
	Junto com seus colegas, reflita sobre a definição que acabamos de ver. Vocês conseguem pensar em outros exemplos de relações do seu dia a dia que possam ser consideradas funções? Descrevam algumas delas e compartilhem com o restante da turma, destacando os conjuntos domínio e contradomínio dessas funções.
\end{reflection}
\newpage


\practice{Conceito do Função}
\vspace{-2\parskip}
\begin{task}{Colorindo o mapa}
	\label{\detokenize{AF106-2:atividade-colorindo-o-mapa}}\label{\detokenize{AF106-2:ativ-funcoes-colorindo-o-mapa}}


	A imagem a seguir, que foi retirada do aplicativo Google Maps, exibe o trânsito na ponte Rio-Niterói e seus acessos em um determinado dia e hora. Várias informações podem ser observadas a partir dos elementos apresentados. Por exemplo, as cores nas vias informam a velocidade média dos veículos que trafegam por elas, conforme a legenda na parte inferior; a distância entre dois pontos quaisquer do mapa pode ser estimada usando a escala exibida no canto inferior direito. Gráficos como esse são produzidos a partir das relações entre diversas informações coletadas.

	\begin{figure}[H]
		\begin{center}
			\centering

			\noindent\includegraphics[width=\linewidth]{{rio_niteroi_maps}.png}
		\end{center}
	\end{figure}

	A tabela a seguir mostra os dados coletados sobre o tempo gasto pelos veículos (em média) para atravessar a ponte, ao longo de um dia.

	\begin{table}[H]
		\centering

		\begin{tabular}{|c|c|>{\centering}m{.1\textwidth}|c|}
			\hline
			\hline
			\tcolor{Período do Dia} & \tcolor{Tempo (min)} & \tcolor{Cor} & \tcolor{Velocidade Média (km/min)} \\
			\hline
			5:00 - 7:00             & 13                   &              &                                    \\
			\hline
			7:00 - 9:00             & 18                   &              &                                    \\
			\hline
			9:00 - 11:00            & 15                   &              &                                    \\
			\hline
			11:00 - 13:00           & 15                   &              &                                    \\
			\hline
			13:00 - 15:00           & 16                   &              &                                    \\
			\hline
			15:00 - 17:00           & 16                   &              &                                    \\
			\hline
			17:00 - 19:00           & 23                   &              &                                    \\
			\hline
			19:00 - 21:00           & 14                   &              &                                    \\
			\hline
			21:00 - 23:00           & 13                   &              &                                    \\
			\hline
		\end{tabular}
	\end{table}

	\begin{enumerate}
		\item Tomando como referência a ilustração anterior e utilizando a escala de cores a seguir, complete a terceira coluna da tabela com a cor que a ponte deveria estar colorida em cada período do dia destacado. Descreva os critérios que você utilizou na escolha de cada uma das cores e compare com os critérios dos seus colegas.
		      \begin{center}
			      \includegraphics[width=.5\linewidth]{funcoesaluno-figure4.pdf}
		      \end{center}
		\item Você precisou associar uma mesma cor para para períodos diferentes do dia. Por que?

		\item Sabendo que a ponte Rio-Niterói tem aproximadamente $13$ km de extensão complete a quarta coluna da tabela com a velocidade média registrada em cada um dos períodos do dia.

		\item É possível que uma mesma velocidade média esteja associada a dois tempos de travessia diferentes? Por quê?
	\end{enumerate}
\end{task}

Na atividade anterior, observam-se diferentes relações entre os dados. Por exemplo, para cada tempo de travessia é possível associar uma única cor e uma única velocidade média. Da mesma maneira, a cada velocidade média está associada uma única cor e um único tempo de travessia. No entanto, a uma mesma cor é possível associar tempos diferentes e velocidades médias diferentes.

\begin{task}{ é função?}
	\label{\detokenize{AF106-2:atividade-e-funcao}}\label{\detokenize{AF106-2:ativ-funcoes-e-funcao}}

	No contexto da atividade anterior são observados diferentes conjuntos de dados: O conjunto dos tempos de travessia da ponte, $A=\{13, 14, 15, 16, 18, 23\}$; O conjunto das cores que compoõem a escala, $B=\{$Verde, Laranja, Vermelho, Vinho$\}$; e o conjunto de velocidades obtidas,{}`C{}`. Considere as diferentes relações de dependências estabelecidas entre esses conjuntos. Quais são funções?

	\begin{table}[H]
		\centering
		\begin{tabular}{|c|c|>{\centering}m{6cm}<{\arraybackslash}|}
			\hline
			\hline
			\tcolor{Relação} & \tcolor{É função?} & \tcolor{Se não, por que?} \tabularnewline
			\hline
			De A em B
			                 &                    & \tabularnewline
			\hline
			De B em A
			                 &                    & \tabularnewline
			\hline
			De A em C
			                 &                    & \tabularnewline
			\hline
			De C em A
			                 &                    & \tabularnewline
			\hline
			De B em C
			                 &                    & \tabularnewline
			\hline
			De C em B
			                 &                    & \tabularnewline
			\hline
		\end{tabular}
	\end{table}
\end{task}

Toda relação de um conjunto $A$ em um conjunto $B$ pode ser identificada por um conjunto de pares ordenados. Nesse caso, cada associação entre elementos do conjunto $A$ e elementos do conjunto $B$ fica representada por um par ordenado tal que o elemnto do conjunto $A$ ocupa a primeira posição do par e o correspondente elemento do conjunto $B$ a segunda posição.

Por exemplo, se consideramos a relação dos números reais em si mesmo que, a cada número real, associa o seu quadrado, os pares ordenados $(1,1), (2,4), (\sqrt{3},3), (-\pi,\pi^2)$ indicam elementos que estão relacinados. Já os pares ordenados $(9,5)$ e $(4,2)$, $(\sqrt{2},-2)$ formados por números reais, não indicam números associados pela mesma relação, uma vez que $5$ não é quadrado de $9$, $2$ não é quadrado de $4$ e $-2$ não é o quadrado de $\sqrt{2}$.

Como funções são um tipo especial de relação, a mesma ideia se estende para representação das funções. Assim, os pares ordenados de uma função $f:A\to B$ serão da forma $(x,y)$ em que $x\in A$ e $y=f(x)\in B$.


\begin{task}{ não é função!}
	\label{\detokenize{AF106-2:atividade-nao-e-funcao}}\label{\detokenize{AF106-2:ativ-funcoes-nao-e-funcao}}

	Considere a relação formada por todos $(a,b)$ de números naturais tais que $b$ é múltiplo de $a$. Assim, $(2,4)$, $(2,6)$, $(3,6)$ e $(9, 9)$ são pares ordenado dessa relação, pois $4$ é múltiplo de $2$, $6$ é múltiplo de $2$ e de $3$ e $9$ é múltiplo de $9$ . No entanto, $(4,2)$ e $(7,17)$ são pares ordenados de números naturais, mas não são pares dessa relação.
	\begin{enumerate}
		\item Exiba outros quatro pares ordenados dessa relação.

		\item Explique porque essa relação não é uma função.

		\item $(5, 405)$ é um par ordenado dessa relação. Quantos outros pares ordenados dessa relação têm 5 como primeiro elemento?

		\item Dê exemplo de uma ou mais relações que não sejam funções. Não precisam ser exemplos numéricos.

	\end{enumerate}
\end{task}

\begin{task}{ a família}
	\label{\detokenize{AF106-2:atividade-a-familia}}

	Cada ponto do gráfico a seguir representa uma das seguintes pessoas.

	\begin{figure}[H]
		\begin{center}
			\centering

			\noindent\includegraphics[width=.5\textwidth]{{familia}.png}
			\label{\detokenize{AF106-2:fig-altura-idade}}
		\end{center}
	\end{figure}

	\begin{enumerate}
		\item Associe cada ponto do gráfico à pessoa correspondente.

		\item A relação expressa pelos pares ordenados (idade, altura) apresentados no gráfico é função? Por que?
	\end{enumerate}


	\begin{center}
		\includegraphics[width=.5\textwidth]{funcoesaluno-figure5.pdf}
	\end{center}




	{\color{red}\bfseries{}*}Adaptado de The Language of Functions and Graphs, Shell Centre for Mathematical Education Publications Ltd., 1985.
\end{task}

Quando nos deparamos com uma função é fundamental identificarmos os conjuntos domínio e contradomínio, e a maneira como os elementos desses conjuntos estão relacionados. Tal maneira pode ser muito variada, no entanto, principalmente quando os conjuntos envolvidos são numéricos, é comum considerar como contradomínio o conjunto $\mathbb{R}$. Por isso, daqui por diante, quando estivermos considerando funções numéricas, o contradomínio será igual a $\mathbb{R}$.

Em muitos casos, a forma de associação entre os elementos é dada por uma expressão analítica. Vejamos alguns exemplos.

$(I)$ Para calcular o perímetro de um quadrado de lado $\ell$ usa-se a expressão $P=4\ell$. Percebe-se então que o perímetro está relacionado com o lado. A partir daí pode-se definir a função perímetro:
\begin{equation*}
	\begin{split}P: ]0,+\infty[\to \mathbb{R} \quad ; \quad P(\ell)=4\ell.\end{split}
\end{equation*}
Da mesma forma a área de um quadrado de lado $\ell$ é dada por $A=\ell^2$, que permite definir a função:
\begin{equation*}
	\begin{split}A: ]0,+\infty[\to \mathbb{R} \quad ; \quad A(\ell)=\ell^2.\end{split}
\end{equation*}
A variável $\ell$ pode assumir qualquer valor dentro do intervalo $]0,+\infty[$ que é o domínio da função $P$ . Se quisermos saber o valor do perímetro do quadrado de lado $5$cm, basta substituirmos $\ell$ por 5 na expressão de  $P(\ell)$. Ficamos assim com
\begin{equation*}
	\begin{split}P(5)=4\times 5 = 20\mathrm{cm}.\end{split}
\end{equation*}
A área do quadrado de lado $9$cm é
\begin{equation*}
	\begin{split}A(9)=9^2=81\text{cm}^2.\end{split}
\end{equation*}
$(II)$ A fórmula de Lorentz já foi muito utilizada pelos médicos para o cálculo do “peso ideal” $p$, em kg, em função da altura $h$, em centímetros, do paciente.
\begin{equation*}
	\begin{split}p:]0,300[\to \mathbb{R}\quad ; \quad p(h)=h-100-\dfrac{h-150}{k}\end{split}
\end{equation*}
em que $k$ vale 4 para homens e vale 2 para mulheres.

Que tal usar a fórmula acima para calcular o seu peso ideal?

$(III)$ Imagine que um objeto é solto, a partir do repouso, de uma altura de $10$ metros e percorre uma trajetória vertical em queda livre. Da Física, sabemos que sua altura $h$ em metros medida a partir do solo, em função do tempo $t$ em segundos, quando desprezamos a resistência do ar, é dada por
\begin{equation*}
	\begin{split}h:[0,+\infty[\to \mathbb{R}\quad ; \quad h(t)=10-\dfrac{gt^2}{2},\end{split}
\end{equation*}
em que $g$ representa a aceleração da gravidade em $m/s^2$.metros por segundo ao quadrado.

Fazer a variável tempo assumir o valor $t=0$ segundos na expressão de $h(t)$ significa que estamos medindo a altura no início da contagem do tempo, ou seja a altura inicial do corpo. Nesse caso teremos
\begin{equation*}
	\begin{split}h(0)=10-\dfrac{g\ 0^2}{2}=10.\end{split}
\end{equation*}
\emph{Se por exemplo, quisermos saber em quanto tempo o corpo chegará ao solo, o que devemos fazer?} Como a medição é feita a partir do solo, dizer que o objeto chegou ao solo é o mesmo que dizer que sua altura é igual a 0. Portanto, precisamos descobrir o valor da variável $t$, de maneira que $h(t)=0$. A partir da expressão de $h(t)$ e aproximando $g$ por $10 m/s^2$, obtemos $10-5t^2=0$, donde concluímos que  $t=\sqrt{2}$ aproximadamente.


\begin{task}{ praticando a notação}
	\label{\detokenize{AF106-2:atividade-praticando-a-notacao}}\label{\detokenize{AF106-2:ativ-praticando-notacao}}

	Considere as funções $f$, $g$, $k$ e $h$, todas de domínio $\mathbb{R}$, tais que:
	\begin{equation*}
		\begin{split}f(x)=3x^2+5x\quad ; \quad g(x)=\frac{x-1}{x^3+3}\quad ; \quad k(x)=(x-2)^2+6\quad ; \quad h(x)=2x-7\end{split}
	\end{equation*}
	Determine o valor de:


	\begin{table}[H]
		\centering
		\begin{tabular}{|l|c|}
			\hline
			\hline
			\tcolor{Função}       & \tcolor{Valor} \\
			\hline
			$f(3)$                &                \\
			\hline
			$g(-1)$               &                \\
			\hline
			$k(2)$                &                \\
			\hline
			$f(1)+g(1)$           &                \\
			\hline
			$g(2)-k(-1)$          &                \\
			\hline
			$k(0).f(-2)$          &                \\
			\hline
			$f(0)+h(0)-1$         &                \\
			\hline
			$f(-2).g(-2)+k(2)$    &                \\
			\hline
			$\dfrac{f(-3)}{k(0)}$ &                \\
			\hline
			$x$ quando $h(x)=0$   &                \\
			\hline
			$x$ quando $h(x)=3$   &                \\
			\hline
		\end{tabular}
	\end{table}

\end{task}

\begin{task}{ enchendo o cone}
	\label{\detokenize{AF106-2:atividade-enchendo-o-cone}}\label{\detokenize{AF106-2:ativ-funcoes-enchendo-o-cone}}

	O reservatório representado a seguir tem a forma de um cone cuja altura é $6 m$ e a base é um círculo de raio $3 m$. O volume $V$ em litros de água no reservatório pode ser estimado a partir altura do nível da água $h$ (em metros) de acordo com a seguinte expressão:
	\begin{equation*}
		\begin{split}V(h)=250h^3\end{split}
	\end{equation*}
	\begin{center}
		\includegraphics[width=.3\textwidth]{funcoesaluno-figure6.pdf}
	\end{center}
	\begin{enumerate}
		\item Determine $V(2), V(3)$ e $V(4)$ e explique os seus significados no contexto.

		\item Quais os volumes de água, mínimo e máximo, que o reservatório comporta?

		\item A que altura do nível da água corresponde o volume igual a $3 456$ litros?

	\end{enumerate}
\end{task}

\begin{task}{ uniformemente variado}
	\label{\detokenize{AF106-2:atividade-uniformemente-variado}}\label{\detokenize{AF106-2:ativ-funcoes-uniformemente-variado}}

	A posição $S$ (em quilômetros), medida a partir de um referencial, de um veículo que se desloca segundo um movimento retilíneo uniformemente variado (MRUV) é dada em função do tempo $t$ (medido em horas) pela seguinte expressão:
	\begin{equation*}
		\begin{split}S(t)=2t^2-4t+2\end{split}
	\end{equation*}\begin{enumerate}
		\item Determine a posição inicial do veículo. Explique o significado desse resultado a partir do contexto.

		\item Após quanto tempo o veículo estará a $18$km da origem?

	\end{enumerate}
\end{task}


\know{}
\label{\detokenize{AF106-3::doc}}\label{\detokenize{AF106-3:sec-aprofundando}}\label{\detokenize{AF106-3:para-saber-mais}}

\begin{task}{ por que não é função?}
	\label{\detokenize{AF106-3:ativ-nao-funcao}}\label{\detokenize{AF106-3:atividade-por-que-nao-e-funcao}}

	Vimos que para que uma relação de $A$ em $B$ seja uma função não pode haver:

	$(I)$ Elementos no conjunto $A$ sem correspondente em $B$;
	$(II)$ Ambiguidade na determinação de correspondente em $B$.

	Determine se cada uma das relações apresentadas a seguir é função. Justifique suas respostas a partir das condições $(I)$ e $(II)$.
	\begin{enumerate}
		\item Seja $\mathcal{P}$ o conjunto de todas as pessoas e considere a relação de $\mathcal{P}$ em $\mathcal{P}$, que a cada “pessoa” associa “irmão da pessoa”.

		\item Seja $\mathbb{R}$  o conjunto dos números reais e considere a relação de $\mathbb{R}$ em $\mathbb{R}$, que a cada “número real $x$ ” associa “raiz quadrada do número real $x$ “.

		\item Sejam $\mathbb{R}^+$ o conjunto dos números reais positivos e $\mathcal{T}$ o conjunto de todos os triângulos. Considere a relação de $\mathbb{R}^+$ em $\mathcal{T}$ que a cada “número real positivo $x$ ” associa “triângulo de área $x$ “.

	\end{enumerate}

\end{task}

\begin{task}{ domínio e imagem}
	\label{\detokenize{AF106-3:ativ-qual-e-imagem}}\label{\detokenize{AF106-3:atividade-dominio-e-imagem}}

	Considere a seguinte lista de expressões algébricas.
	% \begin{multicols}{2}
	\begin{enumerate}
		\item $f(x)=\sqrt{x}$
		\item $G(z)=\sqrt{z-5}$
		\item $h(s)=\frac{1}{3-s}$
		\item $J(t)=\frac{1}{t+8}$
		\item $T(x)=\frac{1}{\sqrt{x}}$
		\item $R(x)=(x-2)^2+7$
		\item $g(u)=5u^2+8$
		\item $F(x)=(x+1)^2-3$
	\end{enumerate}
	% \end{multicols}

	Veja que, em algumas das expressões, a variável independente não pode assumir alguns valores, por exemplo, na letra a) $x$ não pode assumir valores negativos. Complete a tabela abaixo com o maior conjunto domínio possível que cada uma das funções pode ter e o correspondente conjunto imagem.


	\begin{table}[H]
		\centering
		\begin{tabular}{|c|c|c|}
			\hline
			\hline
			\tcolor{Expressão} & \tcolor{Domínio $A$}         & \tcolor{Imagem}             \\
			\hline
			$(a)$              & $\mathbb{R}^+$               &                             \\
			\hline
			$(b)$              &                              &                             \\
			\hline
			$(c)$              &                              & $\mathbb{R}\setminus \{0\}$ \\
			\hline
			$(d)$              & $\mathbb{R}\setminus \{-8\}$ &                             \\
			\hline
			$(e)$              &                              &                             \\
			\hline
			$(f)$              &                              & $[7,+\infty[$               \\
			\hline
			$(g)$              &                              &                             \\
			\hline
			$(h)$              &                              &                             \\
			\hline
		\end{tabular}
	\end{table}


\end{task}

\exercise


\begin{enumerate}
	\item  Assim como os números triangulares (ver {\hyperref[\detokenize{AF106-4:ativ-funcoes-numeros-triangulares}]{Atividade: números triangulares no plano}}), fala-se nos números quadrados perfeitos, pentagonais, hexagonais, inspirados, respectivamente, pelas sequências abaixo.
	      \phantomsection\label{\detokenize{AF106-E1:fig-figurados}}
	      \begin{figure}[H]
		      \begin{center}
			      \includegraphics[width=.4\linewidth]{funcoesaluno-figure7.pdf}

			      \includegraphics[width=.4\linewidth]{funcoesaluno-figure8.pdf}

			      \includegraphics[width=.4\linewidth]{funcoesaluno-figure9.pdf}
		      \end{center}
	      \end{figure}

	      \begin{enumerate}
		      \item       Para cada uma destas sequências, represente as próximas duas figuras;

		      \item       Escreva uma sequência de números que possa estar associada a cada sequência de figuras;

		      \item       Descreva a regra de formação de cada uma dessas sequências de números.

	      \end{enumerate}

	      \clearpage
	\item Observe as duas sequências que se seguem:
	      \begin{equation*}
		      \begin{split}1, 1, 2, 3, 5, 8, 13, \dots\end{split}
	      \end{equation*}\begin{equation*}
		      \begin{split}1000, 100, 10, \dots\end{split}
	      \end{equation*}\begin{enumerate}
		      \item       Descreva, em palavras ou em linguagem simbólica, uma regra de formação que você percebe em cada uma das sequências apresentadas.

		      \item       Baseado na regra que você identificou no item anterior, descubra qual é o 20º termo de cada uma das sequências anteriores.

	      \end{enumerate}

	\item Cada prisma obtém-se empilhando cubos do mesmo tamanho, brancos e cinzas, segundo uma regra sugerida na figura.
	      \phantomsection\label{\detokenize{AF106-E1:fig-prismas}}

	      \begin{figure}[H]
		      \begin{center}
			      \centering

			      \includegraphics[width=.5\linewidth]{funcoesaluno-figure10.pdf}
		      \end{center}
	      \end{figure}
	      \begin{enumerate}
		      \item       Descreva, em palavras ou em linguagem simbólica, uma regra de formação sugerida pela figura.

		      \item       Para construir o prisma $4$ dessa sequência, segundo o padrão por você descrito, quantos cubos cinzas são necessários?

		      \item       Justifique a afirmação: “O número total de cubos cinzas necessários para construir qualquer prisma desta sequência é par.”

		      \item       Segundo o padrão por você descrito, quantos cubos cinzas terá o prisma 200?

		      \item       Explicite uma expressão numérica que permita determinar o número de cubos cinzas do Prisma $n$ em função de $n$, isto é, uma expressão que de forma geral associe a ordem da figura à quantidade de cubos cinzas em sua composição.

		      \item       Justifique novamente a afirmação do item (c), agora a partir da expressão que você explicitou no ítem anterior.

		      \item       Se $x$ representar o número total de cubos (brancos e cinzas) de um prisma desta sequência, qual das expressões seguintes representará o número de cubos cinzas desse prisma. Justifique sua escolha.

	      \end{enumerate}
	      \begin{equation*}
		      \begin{split}\square \ x-8 \quad \quad \square \ 2x-4 \quad \quad \square \ x-4 \quad \quad \square \ 4x\end{split}
	      \end{equation*}
	\item  Ao final de um treino para a prova de 100 metros rasos, uma corredora recebe de seu treinador a seguinte tabela com as marcas intermediárias da sua melhor corrida.

	      \begin{table}[H]
		      \centering

		      \begin{tabular}{|c|c|}
		      	  \hline
			      \hline
			      \tcolor{Tempo (s)} & \tcolor{Distância (m)} \\
			      \hline
			      5                  & $25$                   \\
			      \hline
			      10                 & $50$                   \\
			      \hline
			      15                 & $75$                   \\
			      \hline
			      20                 & $100$                  \\
			      \hline
		      \end{tabular}
	      \end{table}

	      Considerando que a velocidade da atleta é constante ao longo dos 100 metros responda as seguintes perguntas.
	      \begin{enumerate}
		      \item       Quanto tempo ela gastou para percorrer os primeiros $30$ metros?

		      \item       Pensando em uma estratégia para melhorar a preformance da atleta, seu treinador resolve detalhar a tabela com os tempos correspondentes a cada $10$ metros. Construa essa tabela.

	      \end{enumerate}

	\item Hoje de manhã a Ana saiu de casa e dirigiu-se para a escola. Fez uma parte do percurso andando e a outra parte correndo. O gráfico a seguir mostra a distância percorrida pela Ana, em função do tempo que decorreu desde o instante em que ela saiu de casa até ao instante em que chegou à escola.
	      \begin{center}
		      \includegraphics[width=.5\linewidth]{funcoesaluno-figure11.pdf}
	      \end{center}
	      Apresentam-se, a seguir, quatro afirmações. De acordo com o gráfico, apenas uma é verdadeira. Assinale-a com X, explicando por que motivo cada uma das demais opções é falsa.

	      ( { } ) A Ana percorreu metade da distância andando e a outra metade correndo.

	      ( { } ) A Ana percorreu maior distância andando do que correndo.

	      ( { } ) A Ana esteve mais tempo correndo do que andando.

	      ( { } ) A Ana iniciou o percurso correndo e terminou-o andando.

	\item Em Janeiro, o Vitor, depois de ter vindo do barbeiro, decidiu estudar o comprimento do seu cabelo, registando todos os meses a sua medida. O gráfico seguinte representa o crescimento do cabelo do Vitor, desde o mês de Janeiro (mês 0), até ao mês de Junho (mês 5).
	      \phantomsection\label{\detokenize{AF106-E1:fig-cabelo}}

	      \begin{center}
		      \includegraphics[width=.5\linewidth]{funcoesaluno-figure12.pdf}
	      \end{center}\begin{enumerate}
		      \item       A partir dos dados apresentados no gráfico, complete a tabela acima.

		      \item       Em cada mês, quantos centímetros cresceu o cabelo do Vitor?

		      \item       Escreva uma expressão geral que represente o Comprimento (C) do cabelo do Vitor, em função do número de meses (M) passados após o corte de cabelo inicial.

		      \item       Considerando o comportamento indicado no gráfico, se o cabelo do Vitor crescer $19,8 \ cm$, se que haja cortes no período, quantos meses terão se passado desde o último corte de cabelo? Justifique.

	      \end{enumerate}

	\item Considere a função $g:\mathbb{R}\to\mathbb{R}\quad$ tal que $\quad g(x)=9-x^2$.
	      \begin{enumerate}
		      \item       Coloque em ordem crescente os números $g(\sqrt{2})$, $g(\sqrt{5})$ e  $g(\sqrt{10})$.

		      \item       Determine todos os possíveis valores de $x$ do domínio que têm imagem igual a 8.

		      \item       Existe algum $x\in \mathbb{R}$ cuja imagem é igual a $10$? Por que?

		      \item       Que condição deve satisfazer um número real $b$ para que seja a imagem de algum número real $x$, isto é, $b=g(x)$ ?

	      \end{enumerate}

	\item Considere o processo que associa \emph{cada número natural à soma de seus algarismos}.
	      \begin{enumerate}
		      \item       Por meio do processo descrito acima o número natural $13717$ será associado a que número?

		      \item       Proponha um número cujo resultado do processo seja $22$.

		      \item       Quantos números entre $1$ e $10000$ nos levam ao resultado $3$?

		      \item       É possível obter qualquer número natural como resultado desse processo? Explique.

	      \end{enumerate}
\end{enumerate}


\explore{Gráficos}
Segundo informações do \href{http://www.bigdatabusiness.com.br/visualizacao-de-dados-por-que-transformar-big-data-em-graficos/}{Big Data Business}, as palavras estimulam o lado esquerdo do cérebro e são um recurso essencial para a manutenção da memória. No entanto, as imagens são ainda mais eficazes, porque elas conseguem ativar os dois lados do cérebro simultaneamente e, assim, permitem o resgate de ideias e informações com maior precisão e agilidade. Especialmente quando se quer analisar grande quantidade de dados, apresentá-los em uma imagem ou em um gráfico, pode favorecer a comunicação.

\begin{figure}[H]
	\begin{center}
		\centering


		\noindent\includegraphics[width=.8\linewidth]{{grafico-final}.png}
		\caption{Alguns exemplos de representações gráficas}\label{\detokenize{AF106-4:id1}}\end{center}
\end{figure}

Represe25ntar graficamente conjuntos de dados e suas relações pode fazer toda a diferença para transmitir informações. Há vários tipos de gráficos, cada um tem a sua particularidade e serve para transmitir as informações de forma específica. Nesta seção iremos estudar a representação gráfica de funções.

Vamos considerar a seguinte situação:


\begin{task}{ ação promocional}
	\label{\detokenize{AF106-4:atividade-acao-promocional}}

	Uma empresa resolve lançar uma ação promocional na internet usando uma \href{https://pt.wikipedia.org/wiki/Hashtag}{hashtag}. Um mês após o lançamento, o presidente dessa empresa resolve analisar o impacto da ação na rede. Para isso ele pede a um de seus funcionários que prepare um relatório sobre o número de vezes que a \emph{hashtag} foi mencionada nas redes sociais em cada dia durante aquele mês. O funcionário resolveu apresentar os dados das seguintes duas formas:
	\begin{figure}[H]
		\begin{center}
			\centering

			\includegraphics[width=.5\linewidth]{funcoesaluno-figure13.pdf}
		\end{center}
	\end{figure}

	\begin{enumerate}
		\item Quantas vezes a \emph{hashtag} foi mencionada mais de 1500 vezes em um dia?

		\item Em que dia a \emph{hashtag} foi mais citada?

		\item Identifique todos os períodos em que houve crescimento no número de citações.

		\item Faça o mesmo para o decrescimento.

		\item Escreva um parágrafo explicando o comportamento global do gráfico, apontando possíveis causas para as variações observadas.
	\end{enumerate}

\end{task}

Uma função, essencialmente, relaciona duas ou mais grandezas ou variáveis, de forma que são obtidos pares $(x,y)$, em que $x$ pertence ao domínio da função e $y=f(x)$. Perceba que a ordem em que os termos que compõem o par são apresentados é importante. Em matemática, chamamos esse tipo de objeto de \emph{par ordenado}, eles são objetos fundamentais para a compreensão do gráfico de uma função.

No caso de funções reais de variável real, isto é, cujos domínio e contradomínio são o conjunto dos números reais (ou subconjuntos dele) tanto $x$ como $y$ serão números reais.

A representação geométrica mais comum para esses pontos, e que você provavelmente já conhece, é no plano cartesiano\index{plano cartesiano}. Essa representação tem como base duas retas numéricas perpendiculares que se intersectam em suas origens conforme a figura abaixo.

\begin{center}
	\includegraphics[width=.4\linewidth]{funcoesaluno-figure14.pdf}
\end{center}

As retas que compõem um sistema cartesiano são chamadas de eixos\index{eixos coordenados} do plano cartesiano. O eixo em que são registradas as primeiras coordenadas do par é chamado de eixo das abscissas\index{eixo das abscissas}. O outro eixo, em que são registradas as segundas coordenadas do par é chamado de eixo das ordenadas\index{eixo das ordenadas}.

Já vimos alguns exemplos de funções em atividades anteriores, vamos explorá-los um pouco mais.


\begin{task}{ do mapa para o gráfico}
	\label{\detokenize{AF106-4:ativ-funcoes-do-mapa-para-grafico}}\label{\detokenize{AF106-4:atividade-do-mapa-para-o-grafico}}

	\begin{enumerate}
		\item A partir das colunas \emph{Tempo de travessia} e \emph{Cor} da {\hyperref[\detokenize{AF106-2:ativ-funcoes-colorindo-o-mapa}]{Atividade: colorindo o mapa}}, escreva o conjunto de pares ordenados da forma (tempo, cor) respeitando o critério que você escolheu para a determinação das cores.

		\item Represente graficamente este conjunto de pares ordenados.

		\item Para colorir as vias de todo o mapa, precisamos distribuir as cores para outros valores de tempo. Como você faria a distribuição para o intervalo de $0$ a $25$ minutos considerando um trecho qualquer de $13$ km (a mesma extensão da ponte)?

		\item Encontre outra maneira de representar graficamente a associação entre os tempos e as cores.

	\end{enumerate}

\end{task}

\begin{task}{ números triangulares no plano}
	\label{\detokenize{AF106-4:atividade-numeros-triangulares-no-plano}}\label{\detokenize{AF106-4:ativ-funcoes-numeros-triangulares}}

	Represente, no plano cartesiano, o conjunto de pontos que correspondem aos pares ordenados $\{(n,T_n)\ ;\ n\in\{1,2,...,8\}\}$, em que $T_n$ é o $n$-ésimo número triangular.

\end{task}

\begin{task}{ jornada até a escola}
	\label{\detokenize{AF106-4:atividade-jornada-ate-a-escola}}\label{\detokenize{AF106-4:ativ-funcoes-jornada-ate-a-escola}}

	Leonardo mora a $6$ km da escola onde estuda e utiliza o transporte escolar, que o busca na porta de sua casa. Em um certo dia, o percurso de Leonardo até sua escola foi assim: Ele estava na porta de casa às $7$ horas, como de costume, mas o transporte escolar atrasou, passando em sua casa somente às $7h05min$. Leonardo entrou na van e sentou no penúltimo lugar vago. Ainda faltava Marina. “Ela mora a $3$ km da minha casa!”, lembrou Leonardo. Às $7h10min$ em ponto, o transporte escolar chegou à casa de Marina, que já estava pronta aguardando para embarcar. Para tentar compensar o atraso, o motorista resolveu tomar um atalho, mas a estratégia não funcionou. Às $7h15min$ precisou ficar parado por $5$ minutos em frente a uma cancela aguardando um trem de carga passar. Finalmente, às $7h25min$ chegaram à escola, $5$ minutos antes do sinal tocar.

	No plano cartesiano a seguir, o eixo horizontal indica o tempo em minutos e o eixo vertical a distância percorrida em quilômetros. Os pontos marcados correspondem às distâncias percorridas por diversos estudantes da escola a cada $5$ minutos no período das $7h$ às $7h30min$ da mesma manhã descrita na situação acima.

	\phantomsection\label{\detokenize{AF106-4:fig-pontos-jornada}}
	\begin{figure}[H]
		\begin{center}
			\centering

			\includegraphics[width=.5\linewidth]{funcoesaluno-figure15.pdf}
		\end{center}
	\end{figure}


	\begin{enumerate}
		\item Conecte os pontos que correspondem à jornada de Leonardo, desde a porta da sua casa até a chegada à escola, no dia descrito acima.

		\item Faça uma estimativa da distância a que Leonardo estará de sua casa às $7h07min$.

		\item Escolha um conjunto de pontos que possa representar a jornada de um outro estudante da sua casa à escola e descreva essa jornada.

	\end{enumerate}
\end{task}

\arrange{Gráficos}
\label{\detokenize{AF106-5:sec-organizando-graficos}}\label{\detokenize{AF106-5:organizando-as-ideias-graficos}}\label{\detokenize{AF106-5::doc}}
É hora de organizar as ideias sobre representação gráfica de uma função. Vimos que, para representar graficamente as funções, os pares ordenados são fundamentais. Cada par identifica as grandezas ou variáveis relacionadas e a ordem no par distingue o papel de cada uma delas: elemento do domínio, abscissa, e imagem, ordenada. Sendo assim, a representação gráfica de uma função exige: a identificação das variáveis do problema e a identificação da relação estabelecida entre as variáveis.

Para funções reais de variável real, isto é, funções cujo domínio é um subconjunto de $\mathbb{R}$ e o contradomínio é $\mathbb{R}$, sua representação gráfica no plano cartesiano será o conjunto dos pares ordenados $(x,f(x))$ em que $x$ pertence ao domínio da função.

\begin{figure}[H]
	\begin{center}
		\centering

		\includegraphics[width=.5\linewidth]{funcoesaluno-figure16.pdf}
	\end{center}
\end{figure}

\begin{reflection}

	Os conjuntos domínio e imagem ficam evidenciados na representação gráfica de uma  função a partir dos eixos coordenados. Observe a representação gráfica a seguir, em que estão destacados conjuntos sobre os eixos. Qual deles você identifica como domínio? A que conjunto corresponde o outro?
	\begin{figure}[H]
		\begin{center}
			\centering

			\includegraphics[width=.6\linewidth]{funcoesaluno-figure17.pdf}
		\end{center}
	\end{figure}
\end{reflection}

\practice{Gráficos}
\phantom{M}
\vspace{-1em}
%\label{\detokenize{AF106-5:sec-praticando-grafico}}\label{\detokenize{AF106-5:praticando}}

\begin{task}{Indo para escola}
	\label{\detokenize{AF106-5:ativ-indo-para-escola}}\label{\detokenize{AF106-5:atividade-indo-para-escola}}

	Arthur, Caetano, Gael, Levi e Pedro utilizam a mesma avenida para ir à escola a cada manhã. Levi vai com seu pai de carro, Arthur de bicicleta e Gael caminhando. Os demais variam, a cada dia, a forma como percorrem o trajeto. O mapa a seguir mostra a posição da casa de cada um em relação à escola.
	\phantomsection\label{\detokenize{AF106-5:fig-mapa-escola}}

	\begin{figure}[H]
		\begin{center}
			\centering

			\includegraphics[width=.5\linewidth]{funcoesaluno-figure18.pdf}
		\end{center}
	\end{figure}

	Os pontos marcados no plano cartesiano abaixo fornecem informações sobre a jornada de cada criança na última segunda-feira.
	\phantomsection\label{\detokenize{AF106-5:fig-grafico-jornada}}

	\begin{figure}[H]
		\begin{center}
			\centering

			\includegraphics[width=.5\linewidth]{funcoesaluno-figure19.pdf}
		\end{center}
	\end{figure}

	\begin{enumerate}
		\item Associe cada ponto do gráfico com o nome da criança que ele representa.

		\item Como Pedro e Caetano foram para a escola na última segunda-feira? Por que?

	\end{enumerate}

	{\color{red}\bfseries{}{}`\emph{{}`Adaptado de *The Language of Functions and Graphs}, Shell Centre for Mathematical Education Publications Ltd., 1985.}

\end{task}

\begin{task}{ qual é o gráfico?}
	\label{\detokenize{AF106-5:ativ-qual-e-o-grafico}}\label{\detokenize{AF106-5:atividade-qual-e-o-grafico}}

	Dentre os gráficos apresentados a seguir identifique aquele que melhor descreve os dados apresentados em cada uma das tabelas seguintes.


	\begin{enumerate}
		\item  Café esfriando
		      \begin{table}[H]
			      \centering
			      \begin{tabular}{|c|c|c|c|c|c|c|c|}
			      	  \hline
				      \hline
				      \tcolor{Tempo (minutos)}           & 0  & 5  & 10 & 15 & 20 & 25 & 30 \\
				      \hline
				      \tcolor{Temperatura ($^{\circ}$C)} & 90 & 79 & 70 & 62 & 55 & 49 & 44 \\
				      \hline
			      \end{tabular}
		      \end{table}

		\item Preparando a ceia

		      \begin{table}[H]
			      \centering
			      \begin{tabular}{|c|c|c|c|c|c|c|c|}
			      	  \hline
				      \hline
				      \tcolor{Peso (quilos)} & 3   & 4 & 5   & 6 & 7   & 8 & 9   \\
				      \hline
				      \tcolor{Tempo (horas)} & 2,5 & 3 & 3,5 & 4 & 4,5 & 5 & 5,5 \\
				      \hline
			      \end{tabular}
		      \end{table}

		\item Depois de três canecas de cerveja…

		      \begin{table}[H]
			      \centering
			      \begin{tabular}{|c|c|c|c|c|c|c|c|}
			          \hline
				      \hline
				      \tcolor{Tempo (horas)}               & 1  & 2  & 3  & 4  & 5  & 6  & 7 \\
				      \hline
				      \tcolor{Álcool no sangue (mg/100ml)} & 90 & 75 & 60 & 45 & 30 & 15 & 0 \\
				      \hline
			      \end{tabular}
		      \end{table}

		\item Como um bebê cresce antes do nascimento

		      \begin{table}[H]
			      \centering
			      \begin{tabular}{|c|c|c|c|c|c|c|c|c|}
			      	  \hline
				      \hline
				      \tcolor{Tempo de gestação (meses)} & 2 & 3 & 4  & 5  & 6  & 7  & 8  & 9  \\
				      \hline
				      \tcolor{Comprimento do bebê (cm)}  & 4 & 9 & 16 & 24 & 30 & 34 & 38 & 42 \\
				      \hline
			      \end{tabular}
		      \end{table}
	\end{enumerate}

	\begin{figure}[H]
		\begin{center}
			\centering

			\includegraphics[width=.5\linewidth]{funcoesaluno-figure20.pdf}

			\includegraphics[width=.5\linewidth]{funcoesaluno-figure21.pdf}

			\includegraphics[width=.5\linewidth]{funcoesaluno-figure22.pdf}
		\end{center}
	\end{figure}

	$*$ Adaptado de \emph{The Language of Functions and Graphs}, Shell Centre for Mathematical Education Publications Ltd., 1985.
\end{task}

\begin{task}{ imaginando gráficos}
	%\label{\detokenize{AF106-5:atividade-imaginando-graficos}}

	Associe cada uma das situações apresentadas a seguir a um dos gráficos dados abaixo. Explique sua escolha e escreva, em cada um dos eixos, o que eles representam.
	\begin{figure}[H]
		\begin{center}
			\centering

			\includegraphics[width=.6\linewidth]{funcoesaluno-figure23.pdf}
		\end{center}
	\end{figure}

	\begin{enumerate}[label=($\Roman*$)]
		\item Após um concerto houve um grande silêncio. Então uma pessoa na platéia começou a aplaudir. Gradualmente, as pessoas à sua volta também começaram a apludir de forma que rapidamente todos estavam aplaudindo.

		\item Se o preço cobrado pelo ingresso de um cinema for muito baixo, seu prorietário irá perder dinheiro. Por outro lado, se o valor cobrado for muito alto, poucas pessoas irão pagar e novamente o proprietário vai perder dinheiro. Um cinema deve portanto cobrar um preço moderado por seu ingresso de forma que seja lucrativo.

		\item Preços estão agora subindo mais lentamente do que em qualquer época nos últimos cinco anos.
	\end{enumerate}
	\begin{itemize}
		\item Adaptado do artigo \emph{Michal Ayalon \& Anne Watson \& Steve Lerman (2015). Progression Towards Functions: Students’ Performance on Three Tasks About Variables from Grades 7 to 12.}
	\end{itemize}
\end{task}

\clearpage
\begin{reflection}
	Observe as figuras abaixo
	\begin{figure}[H]
		\begin{center}
			\centering

			\includegraphics[width=.5\linewidth]{funcoesaluno-figure24.pdf}
		\end{center}
	\end{figure}

	O que os gráficos da primeira linha têm em comum? E as da segunda linha?

	Agora observe-os por coluna. Você consegue identificar algo em comum?
\end{reflection}

\begin{description}
	\item[{Função crescente e função decrescente\index{Função crescente e função decrescente|textbf}}] \leavevmode\phantomsection\label{\detokenize{AF106-5:term-funcao-crescente-e-funcao-decrescente}}
		Uma função $f: \mathbb{R} \to \mathbb{R}$ é dita \emph{crescente} quando os valores das imagens, $f(x)$, aumentam à medida em que os valores de $x$ aumentam, ou seja, para $x_2>x_1$ tem-se $f(x_2)>f(x_1)$.

		\begin{figure}[H]
			\begin{center}
				\centering

				\includegraphics[width=.4\linewidth]{funcoesaluno-figure25.pdf}
			\end{center}
		\end{figure}

		E é dita \emph{decrescente} quando os valores das imagens, $f(x)$, diminuem à medida em que os valores de $x$ aumentam, ou seja, para $x_2>x_1$ tem-se $f(x_2)<f(x_1)$.
		\begin{figure}[H]
			\begin{center}
				\centering

				\includegraphics[width=.4\linewidth]{funcoesaluno-figure26.pdf}
			\end{center}
		\end{figure}

\end{description}


\begin{task}{ leia no gráfico!}
	\label{\detokenize{AF106-5:atividade-leia-no-grafico}}\label{\detokenize{AF106-5:ativ-praticando-notacao}}

	Seja $f$ a função real cuja representação gráfica é apresentada a seguir.

	\begin{figure}[H]
		\begin{center}
			\centering

			\includegraphics[width=.4\linewidth]]{funcoesaluno-figure27.pdf}
		\end{center}
	\end{figure}

	A partir da representação gráfica calcule os seguintes valores:

	\begin{table}[H]
		\centering
		\begin{tabular}{|l|c|}
			\hline
			\hline
			\tcolor{Notação}                    & \tcolor{Valor} \\
			\hline
			$f(1)-f(0)$                         &                \\
			\hline
			$4\cdot f(3)$                       &                \\
			\hline
			$f(4)/f(2)$                         &                \\
			\hline
			$f(6)\cdot f(2)$                    &                \\
			\hline
			$x$ quando $f(x)=-2$                &                \\
			\hline
			$x$ quando $f(x)=0$                 &                \\
			\hline
			$f(3\cdot 2)-4\cdot f(\sqrt{81})+1$ &                \\
			\hline
		\end{tabular}
	\end{table}

\end{task}

\begin{reflection}
	Observe o gráfico da função real dada pela expressão $f(x)=3x^2-15x+18$. Veja que ele possui interseções com o eixo das abscissas e com o eixo das ordenadas. Qual procedimento você utilizaria para determinar esses pontos de interseção?
	\begin{center}
		\includegraphics[width=.2\linewidth]{funcoesaluno-figure28.pdf}
	\end{center}
	Os valores de $x$ para os quais há interseção com o eixo das abscissas são chamados de \emph{zeros} da função.
\end{reflection}

\begin{task}{Imposto de renda}

	A seguinte tabela é utilizada para o cálculo do Imposto de Renda para Pessoa Física (IRPF).

	\begin{table}[H]
		\centering

		\large{\textbf{Tabela do IRF - Vigência a partir de 01/04/2015}}

		(Medida Provisória 670/2015 convertida na Lei 13.149/2015)
		\begin{tabular}{|l|c|r|}
			\hline
			\hline
			\tcolor{Base de cálculo (R\$)}   & \tcolor{Alíquota (\%)} & \tcolor{Parcela a deduzir do IR (R\$)} \\
			\hline
			Até $1.903{,}98$                 & -                      & -                                      \\
			\hline
			De $1.903{,}99$ até $2.826{,}65$ & 7,5                    & $142{,}80$                             \\
			\hline
			De 2$.825{,}55$ até $3.751{,}05$ & 15                     & $354{,}80$                             \\
			\hline
			$3.751{,}06$ até $4.664{,}68$    & 22,5                   & $636{,}13$                             \\
			\hline
			Acima de $4.664{,}68$            & 27,5                   & $869{,}36$                             \\
			\hline
		\end{tabular}
		\caption{Fonte: \url{http://www.portaltributario.com.br}}
	\end{table}

	Por esta tabela, um trabalhador cujo rendimento é inferior a R\$ $1.903{,}98$ está isento do imposto de renda. Já um trabalhador com rendimento de R\$ $3.000{,}00$ tem um desconto, em reais, de $15\%$ de $3.000{,}00$ (450,00) menos a dedução de 354,80, isto é, deverá pagar de importo de renda o valor R\$ $450-354{,}80=95{,}20$ .

	\clearpage
	\begin{enumerate}
		\item Com os dados apresentados na tabela acima construímos a seguinte função que fornece o valor de importo de renda a ser pago, a partir do rendimento informado:
		      $$f(x)=
			      \begin{cases}
				      0, \text{ se } x\leq1.903{,}98                             \\
				      0{,}075x-142{,}90, \text{ se } 1.903{,}98<x<2.826{,}65     \\
				      0{,}15x-354{,}90, \text{ se } 2.826{,}65\leq x<3.751{,}05  \\
				      0{,}225x-636{,}13 \text{ se } 3.751{,}05 \leq x<4.664{,}68 \\
				      0{,}275x-869{,}36 \text{ se } 4.664{,}68\leq x
			      \end{cases}
		      $$

		      Determine o imposto que deverá ser pago por um trabalhador cujo rendimento seja:
		      \begin{enumerate}
			      \item R\$ $1.750{,}00$
			      \item R\$ $2.680{,}00$
			      \item R\$ $4.060{,}00$
			      \item R\$ $5.500{,}00$
		      \end{enumerate}

		\item Observe o gráfico a seguir. Nele estão destacados os impostos de renda pago por três trabalhadores, indicados pelas letras $A$, $B$ e $C$.

		      \begin{figure}[H]
			      \begin{center}
				      \centering

				      \includegraphics[width=.5\linewidth]{funcoesaluno-figure29.pdf}
			      \end{center}
		      \end{figure}

		      Segundo a tabela IRF, determine as alíquotas de desconto que estão sendo aplicadas a cada um destes trabalhadores e qual o salário de cada um deles.
	\end{enumerate}

\end{task}

\clearpage
\begin{task}{Planos telefônicos}

	Você deseja trocar o plano do seu telefone e ao consultar a sua operadora tem a opção de escolher entre dois planos: plano Prata e plano Ouro. No seu plano atual, você paga R\$ $70{,}00$ por 500MB de internet e os dados além disso custam R\$ $0{,}20$ por MB.

	O plano Ouro cobra R\$ $140{,}00$ por dados ilimitados e o plano Prata tem a mesma estrutura do seu plano atual. Os valores cobrados pelo plano Prata estão representados no gráfico a seguir.

	\begin{figure}[H]
		\begin{center}
			\centering


			\includegraphics[width=.5\linewidth]{funcoesaluno-figure30.pdf}
		\end{center}
	\end{figure}

	\begin{enumerate}
		\item Qual o valor fixo cobrado no plano Prata e que quantidade de dados ele cobre?
		\item Qual o valor por MB excedente do valor estipulado?
		\item A partir de que quantidade de dados consumidos o plano Ouro passa a ser mais vantajoso?
		\item Represente no sistema de coordenadas acima o gráfico do preço a pagar pelo plano Ouro.
	\end{enumerate}

\end{task}

\clearpage

\begin{task}{Bandeiras tarifárias}

	Desde o ano de 2015, as contas de energia passaram a trazer uma novidade: o Sistema de Bandeiras Tarifárias, que apresenta as seguintes modalidades: verde, amarela e vermelha - as mesmas cores dos semáforos - e indicam se haverá ou não acréscimo no valor de energia a ser repassada ao consumidor final, em função das condições de geração de eletricidade. Cada modalidade apresenta as seguintes características:


	\begin{itemize}
		\item \textbf{Bandeira verde}: condições favoráveis de geração de energia. A tarifa não sofre nenhum acréscimo;

		\item \textbf{Bandeira amarela}: condições de geração menos favoráveis. A tarifa sobre acréscimo de R\$ $0{,}01343$ para cada quilowatt-hora (kWh) consumidos;

		\item \textbf{Bandeira vermelha --- patamar 1}: condições mais custosas de geração. A tarifa sofre acréscimo de R\$ $0{,}04169$ para cada quilowatt-hora (kWh) consumido.

		\item \textbf{Bandeira vermelha --- patamar 2}: condições mais custosas de geração. A tarifa sofre acréscimo de R\$ $0{,}06243$ para cada quilowatt-hora (kWh) consumido.
	\end{itemize}

	\flushright{\small

		Texto extraído da página da ANEEL em 28/03/2020 \\ \url{https://www.aneel.gov.br/bandeiras-tarifárias}}

	\justify
	O sistema de coordenadas abaixo contém os gráficos para as funções que relacionam o preço a pagar pela energia em relação ao consumo em quilowatt-hora (kWh) para cada uma das bandeiras tarifárias, em uma cidade vizinha. Com base nas informações do gráfico a seguir, responda:

	\begin{figure}[H]
		\begin{center}
			\centering

			\includegraphics[width=.5\linewidth]{funcoesaluno-figure35.pdf}
		\end{center}
	\end{figure}

	\begin{enumerate}
		\item Qual o preço da tarifa básica por quilowatt-hora nessa cidade?
		\item Considerando que uma residência tenha registrado um consumo de 350KWh em maio, e que seja um mês de bandeira amarela, qual o valor a pagar?
		\item Sabendo que em junho a bandeira será vermelha (patamar 1) e que uma família possa gastar no máximo R\$ $300{,}00$ com a conta de energia elétrica, qual deve ser o consumo máximo nessa residência?
	\end{enumerate}

\end{task}


\know{}
\label{\detokenize{AF106-A::doc}}\label{\detokenize{AF106-A:para-saber-mais}}\label{\detokenize{AF106-A:sec-aprofundando-grafico}}

\begin{task}{ Todo mundo tem \emph{Facebook}?}
	\label{\detokenize{AF106-A:atividade-todo-mundo-tem-facebook}}\label{\detokenize{AF106-A:ativ-todo-mundo-tem-facebook}}

	A rede social virtual \emph{Facebook} é um grande sucesso. O Facebook criado por Mark Zuckerberg em outubro de 2003, com o nome de \emph{Facemash}, quando ele era  um estudante do segundo ano em Harvard. Inicialmente $450$ visitantes geraram $22.000$ visualizações de fotos em suas primeiras $4$ horas online. Em fevereiro de $2004$, agora com o nome de \emph{Thefacebook}, ele já contava com a participação de mais da metade dos alunos de Harvard, e um mês depois, estudantes das Universidades de Stanford, Columbia, Yale, Boston, Nova Iorque e MIT tiveram acesso à rede social criada por Mark Zuckerberg. A partir de setembro de $2005$, funcionários de várias empresas, dentre elas \emph{Apple} e \emph{Microsoft}, puderam ter acesso ao \emph{Facebook} e no final de $2006$ o serviço ficou disponível para qualquer pessoa maior de $13$ anos e com um endereço válido de \emph{e-mail}.

	A tabela a seguir mostra o número de usuários ativos do \emph{Facebook} em janeiro dos anos de $2004$ a $2015$.

	\begin{table}[H]
		\centering
		\begin{tabular}{|c|l|c|}
			\hline
			\hline
			\tcolor{Ano} & \tcolor{Número de usuários} & \tcolor{Crescimento percentual} \\
			\hline
			2004         & 5                           & \textendash{}                   \\
			\hline
			2005         & 1.000.000                   &                                 \\
			\hline
			2006         & 5.500.000                   & 450\%                           \\
			\hline
			2007         & 12.000.000                  &                                 \\
			\hline
			2008         & 70.000.000                  &                                 \\
			\hline
			2009         & 150.000.000                 &                                 \\
			\hline
			2010         & 370.000.000                 &                                 \\
			\hline
			2011         & 600.000.000                 &                                 \\
			\hline
			2012         & 800.000.000                 &                                 \\
			\hline
			2013         & 1.056.000.000               &                                 \\
			\hline
			2014         & 1.228.000.000               &                                 \\
			\hline
			2015         & 1.317.000.000               &                                 \\
			\hline
		\end{tabular}
	\end{table}


	Imagine que queremos investigar o crescimento anual do número de usuários. E, a partir da investigação formular um modelo que nos permita fazer previsões sobre a base de usuários para os próximos anos.
	\begin{enumerate}
		\item Vamos começar investigando o crescimento percentual, preenchendo as lacunas da terceira coluna da tabela acima.

		\item Marque no plano cartesiano os pontos correspondentes aos dados fornecidos pelas duas primeiras colunas da tabela, usando a seguinte escala: no eixo das abscissas $1$ cm corresponde a $1$ ano e no eixo das ordenadas $1$ cm corresponde a $200$ milhões de usuários ativos.

		\item Como você descreveria o crescimento do número de usuários ativos do \emph{Facebook}? Você acha que o crescimento está com tendência a diminuir, a aumentar ou a permanecer estável?

		\item Baseado no item c), faça uma previsão para o número de usuários para os anos de 2016 e 2017.

		\item Usando os dados da tabela e a representação gráfica feita no item b), faça uma previsão para o futuro do \emph{Facebook}. Você acha que os números continuarão a aumentar? Se sim, quando ele atingirá a marca de $2$ bilhões de usuários? Explique seu raciocínio.

		\item Um modelo matemático que fornece uma aproximação para a relação entre os dados das duas primeiras colunas da tabela é dado por uma função $f$ que tem a seguinte expressão
		      \begin{equation*}
			      \begin{split}f(x)=\dfrac{980}{0,7+670 \cdot 0,45^{(x+1)}}\end{split}
		      \end{equation*}
		      em que $x$ representa o tempo decorrido desde $2004$, isto é, para $2010$ tem-se $x=6$, e $f(6)$ é o valor em milhões de usuários ativos no \emph{Facebook} naquele ano. Com a ajuda de uma calculadora científica, use a expressão acima para calcular a estimativa do número de usuários nos anos de $2013$ e de $2014$, e em seguida compare com a tabela.

		\item Use a expressão anterior e calcule a estimativa para os anos de $2016$ e $2017$ e compare com as suas previsões do item (d).

	\end{enumerate}

	Os dados reais para os meses de janeiro de $2016$ e $2017$ são $1.654.000.000$ e $1.936.000.000$, respectivamente. Isso significa que apesar do modelo descrever de forma satisfatória o comportamento do crescimento do número de usuários até o ano de $2015$, para os anos seguintes ele não se mostra adequado. Existia de fato uma tendência para diminuição do crescimento, no entanto essa trajetória foi possivelmente modificada por ações que foram tomadas pela empresa ao perceber tal comportamento.

	Situações como essa são bastante comuns em Modelagem Matemática. O modelo se mostra adequado sob certas condições, mas quando outras variáveis são consideradas (investimento em propaganda, alteração no algoritmo que escolhe as atualizações que serão exibidas para cada usuário, etc) ele pode perder sua acurácia, momento em que se fazem necessárias revisões.

\end{task}

\begin{task}{ Decodificando a mensagem}
	\label{\detokenize{AF106-A:atividade-decodificando-a-mensagem}}\label{\detokenize{AF106-A:ativ-decodificando}}

	Um dos conceitos mais importantes para a segurança na \emph{internet} nos dias de de hoje é o que chamamos de \textbf{criptografia} (do grego \emph{criptos} = escondido, \emph{grafia} = escrita). Segundo o site \emph{wikipedia} ela é o estudo dos princípios e técnicas pelas quais a informação pode ser transformada da sua forma original para outra codificada, de forma que possa ser conhecida apenas por seu destinatário (detentor da “chave secreta”), o que a torna difícil de ser decifrada por alguém não autorizado. Em outras palavras, cria-se um código que pode ser facilmente desfeito (decodificado) mas apenas por aqueles que conhecem a codificação.

	Considere a seguinte maneira de codificar o alfabeto

	\begin{table}[H]
		\centering
		\setlength\tabcolsep{3pt}
		\begin{tabular}{|c|c|c|c|c|c|c|c|c|c|c|c|c|c|c|c|c|c|c|c|c|c|c|c|c|c|c|}
			\hline
			\hline
			\cellcolor{\tikzcolor}{\textcolor{white}{\textbf{Original}}} & A & B & C & D & E & F & G & H & I & J & K & L & M & N & O & P & Q & R & S & T & U & V & W & X & Y & Z \\
			\hline
			\cellcolor{\tikzcolor}{\textcolor{white}{\textbf{Código}}}   & P & Q & R & S & T & U & V & W & X & Y & Z & A & B & C & D & E & F & G & H & I & J & K & L & M & N & O \\
			\hline
		\end{tabular}
	\end{table}

	\begin{enumerate}
		\item Use o código acima para codificar a palavra IMAGEM.

		\item Se você recebesse uma mensagem com a expressão RGXEIDVGPUPG, como faria para decodificá-la?

		      A codificação acima pode também ser representada em um gráfico em que no eixo horizontal estão as letras originais e no vertical os seus respectivos códigos.

		      \begin{figure}[H]
			      \begin{center}
				      \centering

				      \includegraphics[width=.6\linewidth]{funcoesaluno-figure36.pdf}
			      \end{center}
		      \end{figure}
		\item Usando ainda o código acima escreva uma mensagem codificada com duas ou três palavras e troque com algum colega seu de classe. Decodifique a mensagem que recebeu.

		      Você deve ter percebido que a codificação é uma função do conjunto das letras do alfabeto em si mesmo: todas as letras precisam ter um código e uma mesma letra não pode ter mais de um código associada a si.

		\item Seja $X$ o conjunto dos números naturais de $1$ a $26$. Fazendo a correspondência, $A \mapsto 1, B \mapsto 2, C \mapsto 3$, e assim por diante até $Z \mapsto 26$, determine uma função $f:X\to X$ que corresponda ao código acima. Observe que por exemplo, $f(1)=16$.

		\item Usando a expressão $f(x)=x^2$ crie um novo código entre as letras, representando-o no gráfico. O que devemos fazer quando os valores são  maiores que 26?

		\item Considerando o código do gráfico abaixo, tente decodificar a palavra APQGJXV.

		      \begin{figure}[H]
			      \begin{center}
				      \centering

				      \includegraphics[width=.6\linewidth]{funcoesaluno-figure37.pdf}
			      \end{center}
		      \end{figure}

		\item Quais letras do código acima são impossíveis de decodificar e por quê?

		\item Que propriedades deve ter um código para que seja possível decodificá-lo?

	\end{enumerate}

\end{task}

\begin{project}

	\label{\detokenize{AF106-A:projeto-aplicado}}\label{\detokenize{AF106-A:sec-projeto-aplicado}}

	\textbf{Como construir uma caixa de volume máximo?}

	Vamos utilizar uma folha de cartolina quadrada de lado $40$ cm para construir uma caixa sem tampa. Para isso, cortamos quadrados nos quatro cantos da cartolina e dobramos as partes retangulares restantes, para formar os lados da caixa. O objetivo é obter a caixa com o maior volume possível.

	\begin{figure}[H]
		\begin{center}
			\centering

			\includegraphics[width=.6\linewidth]{funcoesaluno-figure40.png}
		\end{center}
	\end{figure}
	\begin{enumerate}
		\item Discuta com seus colegas de grupo a melhor estratégia para se obter a caixa de volume máximo. Em seguida construa a caixa e calcule o seu volume.

		\item Faça uma comparação com os volumes das caixas construídas pelos demais grupos. Organize os dados em uma tabela que relacione a medida do lado $x$ do quadrado recortado com o volume $V(x)$ da caixa obtida.

		      \begin{figure}[H]
			      \begin{center}
				      \begin{table}[H]
					      \centering
					      \begin{tabular}{|c|*{10}{p{.5cm}|}}
					      	  \hline
						      \hline
						      \cellcolor{\tikzcolor}{\textcolor{white}{\textbf{x}}}    &  &  &  &  &  &  &  &  &  & \\
						      \hline
						      \cellcolor{\tikzcolor}{\textcolor{white}{\textbf{V(x)}}} &  &  &  &  &  &  &  &  &  & \\
						      \hline
						  \end{tabular}
				      \end{table}
			      \end{center}
		      \end{figure}

		\item Encontre a expressão que fornece o volume $V(x)$ da caixa em função do lado $x$ do quadrado recortado.

		\item No contexto do problema, em que intervalo real a variável independente $x$ pode ser considerada?

		\item Baseado nos itens anteriores, faça uma conjectura sobre qual o valor de $x$ fornece o volume máximo.

		\item Utilize um software ou uma calculadora gráfica para visualizar a representação gráfica da função $V(x)$. A partir dessa representação gráfica determine, aproximadamente, o valor de $x$ que fornece o volume máximo.

	\end{enumerate}
\end{project}

\exercise


\label{\detokenize{AF106-E2:sec-exercicios-grafico}}\label{\detokenize{AF106-E2:exercicios}}\label{\detokenize{AF106-E2::doc}}

\begin{enumerate}
	\item O gráfico abaixo mostra a altura do nível de água em uma piscina com vazamento. Identifique as variáveis na situação descrita e representada a partir do gráfico. Observe a relação apresentada no gráfico e indique possíveis causas para o comportamento observado.
	      \begin{center}
		      \includegraphics[width=.5\linewidth]{funcoesaluno-figure38.pdf}
	      \end{center}

	\item Garrafas de água potável são vendidas em vários tamanhos e preços. Cada ponto no gráfico abaixo representa uma garrafa de água.
	      \begin{center}
		      \includegraphics[width=.5\linewidth]{funcoesaluno-figure39.pdf}
	      \end{center}\begin{enumerate}
		      \item       Qual garrafa armazena a maior quantidade de água?

		      \item       Qual garrafa é vendida pelo preço mais alto?

		      \item       Identifique dois pontos que estejam sobre uma mesma reta paralela ao eixo das abscissas (reta horizontal) e interprete o que isso significa.

		      \item       Identifique dois pontos que estejam sobre uma mesma reta paralela ao eixo das ordenadas (reta vertical) e interprete o que isso significa.

		      \item       Entre as garrafas $A$ e $E$, qual tem o melhor custo-benefício? Por que? E entre $B$ e $E$? Por que?

	      \end{enumerate}
\end{enumerate}

\end{document}
% \renewcommand\chapterillustration{abertura-afim}
\renewcommand\chapterwhat{Funções linear e afim e suas representações algébrica e gráfica, taxa de variação, proporcionalidade direta, funções afins de domínio discreto (Progressões Aritméticas), Aplicações.}
\renewcommand\chapterbecause{O modelo de variação constante é um dos modelos mais presentes em observações científicas e até mesmo em nosso cotidiano. Ele pode ser identificado por exemplo, em relações que modelam a compra/consumo e venda, o esvaziamento de um recipiente por um ralo em função do tempo ou na relação entre distância e tempo do “movimento uniforme” estudado pela Cinemática, entre tantos outros. Um caso particularmente importante é aquele em que há proporcionalidade entre as grandezas envolvidas na modelagem. As ideias desenvolvidas neste capítulo servem como base para aplicações em diversas áreas.}
\chapter{Função afim}



\mbox{}\thispagestyle{empty}\clearpage

\thispagestyle{empty}

\begin{center}
Projeto: LIVRO ABERTO DE MATEMÁTICA

\noindent \begin{tabular}{lcccr}
\includegraphics[scale=.15]{impa}& \quad\quad& \includegraphics[width=3cm]{logo} & \quad\quad& \includegraphics[scale=.24]{obmep} 
\end{tabular}
\end{center}

\vspace*{.3cm}

Cadastre-se como colaborador no site do projeto: \url{umlivroaberto.org}

Versão digital do capítulo:

\url{https://www.umlivroaberto.org/BookCloud/Volume_1/master/view/AF107.html}


\begin{tabular}{p{.15\textwidth}p{.7\textwidth}}
Título: & Função Afim\\
\\
Ano/ Versão: & 2020 / versão 1.0 de 24 de março de 2020\\
\\
Editora & Instituto Nacional de Matem\'atica Pura e Aplicada (IMPA-OS)\\
\\
Realização:& Olimp\'iada Brasileira de Matem\'atica das Escolas P\'ublicas (OBMEP)\\
\\
Produção:& Associação Livro Aberto\\
\\
Coordenação:& Fabio Simas, \\
            & Augusto Teixeira (livroaberto@impa.br)\\
\\
  Autores: & Gladson Antunes (UNIRIO),\\
        & Michel Cambrainha (UNIRIO),\\
             & Bruno Vianna (Colégio Pedro II).\\
\\
Revisão: &  Cydara Ripoll  \\
         &  Letícia Rangel \\
\\
Design: & Andreza Moreira (Tangentes Design) \\
\\
  Ilustrações: & Miller  Guglielmo \\ 
\\
Gráficos: & Beatriz Cabral e Tarso Caldas (Licenciandos da UNIRIO)\\
\\
  Capa: & Foto de Scott Webb, no Unplash \\
        & https://unsplash.com/photos/fMUIVein7Ng \\

\end{tabular}



\begin{figure}[b]
\begin{minipage}[l]{5cm}
\centering

{\large Licença:}

  \includegraphics[width=3.5cm]{cc-by-sa1}
\end{minipage}\hfill
\begin{minipage}[c]{5cm}
\centering
{\large Desenvolvido por}

\includegraphics[width=2.5cm]{logo-associacao.jpg}
\end{minipage}
\begin{minipage}[r]{5cm}
\centering

{\large Patrocínio:}
  \vspace{1em}
  \includegraphics[width=3.5cm]{itau}
\end{minipage}
\end{figure}

\mainmatter

\explore{função linear}\label{funcaolinear}
No capítulo de Funções, você foi apresentado ao conceito de função, uma relação entre duas grandezas que atende determinadas condições. Neste capítulo, pretendemos colocá-lo em contato com um dos modelos mais presentes em observações científicas e até mesmo em nosso cotidiano, o modelo de variação constante, aqui representado pelo conceito de função afim. Este modelo pode ser identificado por exemplo, em relações que modelam a compra/consumo e venda, o esvaziamento de um recipiente por um ralo em função do tempo ou na relação entre distância e tempo do “movimento uniforme” estudado pela Cinemática, entre tantos outros. Antes de estudarmos a função afim propriamente dita, vamos entender um de seus casos particulares, a função linear.

Comecemos analisando os seguintes problemas:

\begin{figure}[H]
\centering
\noindent\includegraphics[width=400bp]{naoprop.png}
\end{figure}


Os três problemas acima fornecem três informações e propõem que, a partir delas, determinemos uma quarta informação. Esse é o enunciado típico de problemas cujo método de solução é conhecido como “regra de três”. Entretanto, nenhum dos problemas apresentados, como você deve ter percebido, pode ser resolvido aplicando a tal regra. Você consegue imaginar por quê?

Isso se deve ao fato de que as grandezas relacionadas não são proporcionais entre si. Não é verdade que se \(5\) músicos tocam uma peça de música em \(10\) minutos, \(35\) músicos (que é \(7 \times 5\)) tocarão a peça em \(7 \times 10\) minutos, afinal trata-se da mesma música, logo \(35\), \(45\) ou \(179\) músicos tocarão a tal peça nos mesmos \(10\) minutos. Isto é, o tempo de execução não é diretamente proporcional ao número de músicos que executam a música.

\begin{description}
\item[Teorema]
\leavevmode\phantomsection\label{\detokenize{AF107-0:term-grandezas-diretamente-proporcionais}}
Diz-se que \textbf{duas grandezas são diretamente proporcionais} quando elas se correspondem de tal modo que, multiplicando-se uma quantidade de uma delas por um número real, a quantidade correspondente da outra fica multiplicada pelo mesmo número, sempre que os resultados dessas multiplicações fizerem sentido no contexto observado.
\[\begin{array}{ccc}
X\quad &\overline{\quad \quad \quad}& \quad Y \\
k\cdot X \quad &\overline{\quad \quad \quad}& \quad k\cdot Y
\end{array}\]
\end{description}

É muito comum encontrarmos situações no nosso dia a dia em que as grandezas envolvidas são diretamente proporcionais, e você certamente já resolveu muitos problemas, na escola e fora dela, usando a “regra de três”.

\begin{task}{Na piscina}
\label{ativ-na-piscina}

Duas piscinas de 1000 litros cada estão sendo enchidas simultaneamente. A piscina 1 leva 5 horas para ficar completamente cheia e a piscina 2, 8 horas. A cada hora, o volume total de água em cada piscina foi sendo registrado em dois gráficos


\begin{figure}[H]
\centering

\begin{tikzpicture}[scale=1.5, every node/.style={scale=1}]
\tikzstyle{ponto}=[circle, minimum size=3pt, inner sep=0, draw=\currentcolor!80, fill=\currentcolor!80, shift only]
\begin{scope}[yscale=.5]
\draw[lightgray](0,0)grid[xstep=.25,ystep=.25](6,10);
\draw[gray](0,0)grid(6,10);
\draw[thick, ->](0,0)--(6,0)node [below, shift={(-0.55,-0.2)}]{tempo(horas)};
\draw[thick, ->](-0,0)--(0,10);
\node[right, rotate=90]at (-.9,6){capacidade(litros)};
\foreach \x in{0,1, 2, 3, 4, 5}
\node[ponto]at(\x,2*\x){};
\foreach\x in{0,1, 2, 3, 4, 5, 6}
\node[below] at(\x, 0){ \x};
\foreach\y in{100, 200, 300, ..., 1000}
\node[left]at(0,.01*\y){ \y};
\end{scope}
\end{tikzpicture}
\caption{Piscina 1}
\label{piscina1}
\end{figure}
\begin{figure}[H]
\centering

\begin{tikzpicture}[scale=1.5, every node/.style={scale=1}]
\tikzstyle{ponto}=[circle, minimum size=3pt, inner sep=0, draw=\currentcolor!80, fill=\currentcolor!80, shift only]
\begin{scope}[yscale=.5]
\draw[lightgray](0,0)grid[xstep=.25,ystep=.25](8,10);
\draw[gray](0,0)grid(8,10);
\draw[thick, ->](0,0)--(8,0)node [below, shift={(-0.55,-0.3)}]{tempo(horas)};
\draw[thick, ->](-0,0)--(0,10);
\node[right, rotate=90]at (-.9,6){ capacidade(litros)};
\node[ponto]at(0,0){};
\node[ponto]at(1,1.5){};
\node[ponto]at(2,2){};
\node[ponto]at(3,3){};
\node[ponto]at(4,5){};
\node[ponto]at(5,8){};
\node[ponto]at(6,9){};
\node[ponto]at(7,9.5){};
\node[ponto]at(8,10){};
\foreach\x in{0,1, 2, 3, 4, 5, 6, 7, 8}
\node[below] at(\x, 0){\x};
\foreach\y in{100, 200, 300, ..., 1000}
\node[left]at(0,.01*\y){\y};
\end{scope}
\end{tikzpicture}
\caption{Piscina 2}
\label{piscina2}

\end{figure}

\begin{enumerate}
\item {} 
Construa uma tabela com os dados de cada gráfico.

\item {} 
As grandezas volume total de água e tempo de enchimento da piscina 1 são diretamente proporcionais? Explique.

\item {} 
As grandezas volume total de água e tempo de enchimento da piscina 2 são diretamente proporcionais? Explique.

\end{enumerate}
\end{task}

\begin{reflection}

Suponha que os dados numéricos fossem omitidos dos eixos nos dois gráficos. Ainda assim seria possível determinar a proporcionalidade ou não entre as grandezas? Como?
\end{reflection}


\arrange{ função linear}
\label{\detokenize{AF107-1::doc}}\label{\detokenize{AF107-1:organizando-as-ideias-funcao-linear}}
Considere duas grandezas diretamente proporcionais que podem assumir quaisquer valores reais e vamos representá-las pelas letras \(x\) e \(y\). Então, sempre que multiplicarmos \(x\) por qualquer número real \(k\), o valor correspondente da grandeza \(y\) também fica multiplicado pelo mesmo valor. Isto é
\[\begin{array}{ccc}
x\quad &\overline{\quad \quad \quad}& \quad y \\
k\cdot x \quad &\overline{\quad \quad \quad}& \quad k\cdot y
  \end{array}\]
Vamos agora, traduzir a propriedade acima para a linguagem de função. Consideremos que a grandeza \(y\) é expressa como função da grandeza \(x\), isto é, \(y=f(x)\). Vamos supor que quando a variável $x$ vale 1, o valor correspondente de $y$ é $a$, ou seja, $f(1)=a$. Assim, teremos

\[\begin{array}{ccc}
x\quad &\overline{\quad \quad \quad}& \quad f(x) \\
1 \quad &\overline{\quad \quad \quad}& \quad a \\
2 \quad &\overline{\quad \quad \quad}& \quad 2a \\
3 \quad &\overline{\quad \quad \quad}& \quad 3a \\
\vdots \quad &\overline{\quad \quad \quad}& \quad \vdots\\
  \end{array}\]

\begin{observation}{}
Usando a ``regra de três'' nas duas primeiras linhas fica assim
\[\begin{array}{ccc}
x \quad &\overline{\quad \quad \quad}& \quad f(x)\\
1\quad &\overline{\quad \quad \quad}& \quad a \end{array}\]
O que nos leva a
\[\dfrac x1 = \dfrac {f(x)}a \Longrightarrow f(x) = a\cdot x\]
\end{observation}

Toda função real que pode ser expressa na forma $f(x)=ax$ para algum número real $a$, é chamada de \textbf{função linear}. Neste caso, as grandezas representadas pelas variáveis $x$ e $y=f(x)$ são diretamente proporcionais. O número $a=f(1)$ é chamado de \textbf{constante de proporcionalidade} entre as grandezas.


Na \DUrole{xref,std,std-ref}{Atividade Na piscina} você deve ter percebido que as grandezas relacionadas eram diretamente proporcionais apenas no caso da piscina 1. Naquele caso, a função que fornece o volume de água na piscina em função do tempo é dada por \(V:\{1,2,3,4,5\}\to \mathbb{R}\),   \(V(t)=V(1)\cdot t=200\cdot t\).

\begin{reflection}

Suponha que duas grandezas \(x\) e \(y\) se relacionem de maneira que \(y\) seja uma função linear de \(x\).
\begin{enumerate}
\item {} 
Essas duas grandezas são proporcionais?

\item {} 
Podemos afirmar também que \(x\) é uma função linear de \(y\)?

\end{enumerate}
\end{reflection}


\practice{função linear}
\label{\detokenize{AF107-1:praticando}}

\begin{task}{Taxa de câmbio}
\label{ativ-cambio}



Segundo o \href{http://www.bcb.gov.br/pre/bc\_atende/port/taxCam.asp}{site do Banco Central do Brasil}, a \emph{taxa de câmbio} é o preço de uma moeda estrangeira medido em unidades ou frações (centavos) da moeda nacional. Em um determinado dia as taxas de câmbio do dólar americano e do euro eram respectivamente \(R\$ 3,20\) e \(R\$ 4,00\).
\begin{enumerate}
\item {} 
Nesse mesmo dia você deseja comprar \(100\) dólares. Qual seria o valor em reais necessário para realizar essa compra?

\item {} 
Para adquirir nesse mesmo dia \(200\) euros, qual o valor em reais deverá ser desembolsado?

\item {} 
A partir da taxa praticada nesse dia, apresente uma função que converta dólar americano para reais. Qual o conjunto domínio mais adequado a ser considerado para essa função? Justifique.

\item {} 
Com a taxa de câmbio que está sendo praticada nesse dia, quantos dólares americanos podem ser comprados com \(R\$ 2000,00\). Com os mesmos \(R\$ 2000,00\), quantos euros podem ser adquiridos?

\end{enumerate}
\end{task}



\begin{task}{Proporcionalidade na construção de retângulos}
\label{ativ-prop-retangulo}

Considere o retângulo \(R\) abaixo, de lados \(3\) e \(1,5\), e responda as questões propostas.

\begin{figure}[H]
\centering

\begin{tikzpicture}
\tikzstyle{ponto}=[circle, minimum size=2pt, inner sep=0, draw=black, fill=black, shift only]
\draw[thick,black,fill=\currentcolor!80] (0.,0.) -- (3.,0.) -- (3.,1.5) -- (0.,1.5) -- cycle;
\draw (0.2,0.) -- (0.2,0.2) -- (0.,0.2) -- (0.,0.);
\draw (0.,1.3) -- (0.2,1.3) -- (0.2,1.5) -- (0.,1.5);
\draw (2.8,1.5) -- (2.8,1.3) -- (3.,1.3) -- (3.,1.5);
\draw(3.,0.2) -- (2.8,0.2) -- (2.8,0.) -- (3.,0.);
\node[ponto]at(0,0){};
\node[ponto]at(3,0){};
\node[ponto]at(3,1.5){};
\node[ponto]at(0,1.5){};
\node[below ]at(0,0){$$};
\node[below ]at(3,0){$$};
\node[above ]at(3,1.5){$$};
\node[above ]at(0,1.5){$$};
\node[above]at(1.7,-.7){$3$};
\node[right]at(3,.75){$1.5$};
\end{tikzpicture}

\end{figure}

\begin{enumerate}
\item {} 
Observe o retângulo da figura a seguir e determine se ele é semelhante ou não ao retângulo \(R\).

\begin{figure}[H]
\centering

\begin{tikzpicture}
\tikzstyle{ponto}=[circle, minimum size=2pt, inner sep=0, draw=black, fill=black, shift only]
\draw[thick,black,fill=\currentcolor!80] (0.,0.) -- (6.,0.) -- (6.,1) -- (0.,1.)-- cycle;
\draw (0.2,0.) -- (0.2,0.2) -- (0.,0.2) -- (0.,0.);
\draw (0.,.8) -- (0.2,.8) -- (0.2,1) -- (0.,1);
\draw (5.8,1) -- (5.8,.8) -- (6.,.8) -- (6.,1);
\draw(6.,0.2) -- (5.8,0.2) -- (5.8,0.) -- (6.,0.);
\node[ponto]at(0,0){};
\node[ponto]at(6,0){};
\node[ponto]at(6,1){};
\node[ponto]at(0,1){};
\node[below ]at(0,0){$$};
\node[below ]at(6,0){$$};
\node[above ]at(6,1){$$};
\node[above ]at(0,1){$$};
\node[above]at(3,-.7){$6$};
\node[right]at(6,.5){$1.5$};
\end{tikzpicture}

\end{figure}

\item {} 
Na figura a seguir temos a medida base de um retângulo em destaque, qual deve ser a medida de sua altura para que o retângulo gerado seja semelhante a \(R\)? Qual a função linear que relaciona esses dois retângulos?

\begin{figure}[H]
\centering


\begin{tikzpicture}
\tikzstyle{ponto}=[circle, minimum size=2pt, inner sep=0, draw=black, fill=black, shift only]
\fill[bottom color=\currentcolor!80,top color =white] (0.,0.) -- (6.,0.) -- (6.,.5) -- (0.,.5) -- cycle;
\draw (0.2,0.) -- (0.2,0.2) -- (0.,0.2) -- (0.,0.);
\draw(6.,0.2) -- (5.8,0.2) -- (5.8,0.) -- (6.,0.);
\draw(0.,.5)--(0.,0.) -- (6.,0.) -- (6.,.5);
\node[ponto]at(0,0){};
\draw[fill](0,.6)circle(.5pt);
\draw[fill](0,.7)circle(.5pt);
\draw[fill](0,.8)circle(.5pt);
\node[ponto]at(6,0){};
\draw[fill](6,.6)circle(.5pt);
\draw[fill](6,.7)circle(.5pt);
\draw[fill](6,.8)circle(.5pt);
\node[below ]at(0,0){$$};
\node[below ]at(6,0){$$};
\node[above ]at(6,1.5){$$};
\node[above ]at(0,1.5){$$};
\node[above]at(3,-.7){$6$};
\end{tikzpicture}

\end{figure}
\item {} 
Seguindo a mesma ideia do item anterior, qual deve ser a medida da altura desse novo retângulo de base \(5\), para que ele seja semelhante a \(R\)? E neste caso, qual a função linear entre os retângulos?

\begin{figure}[H]
\centering


\begin{tikzpicture}
\tikzstyle{ponto}=[circle, minimum size=2pt, inner sep=0,   draw=black, fill=black, shift only]
\fill[bottom color=\currentcolor!80,top color =white] (0.,0.) -- (5.,0.) -- (5.,.5) -- (0.,.5) -- cycle;
\draw (0.2,0.) -- (0.2,0.2) -- (0.,0.2) -- (0.,0.);
\draw(5.,0.2) -- (4.8,0.2) -- (4.8,0.) -- (5.,0.);
\draw(0.,.5)--(0.,0.) -- (5.,0.) -- (5.,.5);
\node[ponto]at(0,0){};
\draw[fill](0,.6)circle(.5pt);
\draw[fill](0,.7)circle(.5pt);
\draw[fill](0,.8)circle(.5pt);
\node[ponto]at(5,0){};
\draw[fill](5,.6)circle(.5pt);
\draw[fill](5,.7)circle(.5pt);
\draw[fill](5,.8)circle(.5pt);
\node[below ]at(0,0){$$};
\node[below ]at(5,0){$$};
\node[above ]at(5,1.5){$$};
\node[above ]at(0,1.5){$$};
\node[above]at(2.5,-.7){$5$};
\end{tikzpicture}

\end{figure}
\item {} 
Já na figura a seguir, apresentamos um retângulo de altura \(4\), qual deve ser a medida da base desse novo retângulo, para que ele seja semelhante a \(R\)?

\begin{figure}[H]
\centering

\begin{tikzpicture}
\tikzstyle{ponto}=[circle, minimum size=2pt, inner sep=0, draw=black, fill=black, shift only]
\fill[left color = white, right color =\currentcolor!80,] (2.,0.) -- (3.,0.) -- (3.,2.5) -- (2.,2.5) -- cycle;
\draw[thick] (2.,0.) -- (3.,0.) -- (3.,2.5) -- (2.,2.5) ;
\draw (2.8,2.5) -- (2.8,2.3) -- (3.,2.3) -- (3.,2.5);
\draw(3.,0.2) -- (2.8,0.2) -- (2.8,0.) -- (3.,0.);
\node[ponto]at(0,0){};
\node[ponto]at(3,0){};
\node[ponto]at(3,2.5){};
\node[ponto]at(0,2.5){};
\node[below ]at(0,0){$$};
\node[below ]at(3,0){$$};
\node[above ]at(3,2.5){$$};
\node[above ]at(0,2.5){$$};
\node[right]at(3,1.25){$4$};
\draw[fill](1.7,0)circle(.5pt);
\draw[fill](1.8,0)circle(.5pt);
\draw[fill](1.9,0)circle(.5pt);
\draw[fill](1.7,2.5)circle(.5pt);
\draw[fill](1.8,2.5)circle(.5pt);
\draw[fill](1.9,2.5)circle(.5pt);
\end{tikzpicture}

\end{figure}
\item {} 
Na figura a seguir, apresentamos um retângulo cuja base tem a mesma medida da base de \(R\) (igual a \(3\)), e cuja altura coincide com a de um triângulo equilátero de lado medindo \(3\). Esse retângulo é semelhante a \(R\)?

\begin{figure}[H]
\centering


\begin{tikzpicture}
\tikzstyle{ponto}=[circle, minimum size=2pt, inner sep=0, draw=black, fill=black, shift only]
\draw[fill=\currentcolor!80,very thick](0,0)--(4,0)--(4,3.46)--(0,3.46)--cycle;
\draw[fill=terciario,very thick](0,0)--(4,0)--(2,3.46)--cycle;
\node[ponto]at(0,0){};
\node[ponto]at(4,0){};
\node[ponto]at(4,3.46){};
\node[ponto]at(0,3.46){};
\node[ponto]at(2,3.46){};
\node[below]at(0,0){$$};
\node[below]at(4,0){$$};
\node[above]at(4,3.46){$$};
\node[above]at(0,3.46){$$};
\node[above]at(2,3.46){$$};
\node[above]at(2,-.8){$3$};
\end{tikzpicture}

\end{figure}
\item {} 
Se utlizarmos a altura do retângulo da figura anterior na construção de um novo retângulo, qual deve ser a medida de sua base para que seja semelhante a \(R\)?

\end{enumerate}
\end{task}


\begin{task}{Qual é a área?}
\label{ativ-qual-area}

Caso tenha disponibilidade, sugerimos o uso da construção GeoGebra disponível \href{https://www.geogebra.org/m/Xjjym4e7}{neste link}, que é a versão eletrônica dessa atividade.

\begin{figure}[H]
\centering

\noindent\includegraphics[width=100bp]{{codigo}.png}
\end{figure}



\begin{enumerate}
\item {}
Cada círculo representado a seguir tem área total \(20\). Um   dos setores circulares destacados em verde nesses círculos tem área \(14\). Qual é esse setor?

\begin{figure}[H]
\centering


  \begin{tikzpicture}
    \draw(0,0)circle(1);
    \draw[fill=\currentcolor!80] (1,0)--(0,0) --(210:1) arc (210:0:1);
    \node at(-1,1){(a)};
\end{tikzpicture}\quad\quad\quad
 \begin{tikzpicture}
    \draw(0,0)circle(1);
    \draw[fill=\currentcolor!80] (1,0)--(0,0) --(250:1) arc (250:0:1);
    \node at(-1,1){(b)};
 \end{tikzpicture}\quad\quad\quad
      \begin{tikzpicture}
    \draw(0,0)circle(1);
    \draw[fill=\currentcolor!80] (1,0)--(0,0) --(270:1) arc (270:0:1);
    \node at(-1,1){(c)};
 \end{tikzpicture}
\end{figure}

\item{}
Agora, um dos setores circulares em verde tem área \(18\). Qual é esse setor?

\begin{figure}[H]
\centering


\begin{tikzpicture}
  \draw(0,0)circle(1);
  \draw[fill=\currentcolor!80] (1,0)--(0,0) --(330:1) arc (330:0:1);
  \node at(-1,1){(a)};
 \end{tikzpicture}\quad\quad\quad
      \begin{tikzpicture}
  \draw(0,0)circle(1);
  \draw[fill=\currentcolor!80] (1,0)--(0,0) --(250:1) arc (250:0:1);
  \node at(-1,1){(b)};
 \end{tikzpicture}\quad\quad\quad
      \begin{tikzpicture}
  \draw(0,0)circle(1);
  \draw[fill=\currentcolor!80] (1,0)--(0,0) --(300:1) arc (300:0:1);
  \node at(-1,1){(c)};
\end{tikzpicture}

\end{figure}

 \item{}
Explique a estratégia matemática que você utilizou para resolver os itens anteriores? Dentre os setores circulares apresentados a seguir, um deles tem área \(7\). Aplique sua estratégia para determinar qual é esse setor.

\begin{figure}[H]
\centering


 \begin{tikzpicture}

   \draw(0,0)circle(1);
   \draw[fill=\currentcolor!80] (1,0)--(0,0) --(110:1) arc (110:0:1);
   \node at(-1,1){(a)};
 \end{tikzpicture}\quad\quad\quad
      \begin{tikzpicture}
        \draw(0,0)circle(1);\draw[fill=\currentcolor!80] (1,0)--(0,0) --(126:1) arc (126:0:1);\node at((-1,1){(b)};
         \end{tikzpicture}\quad\quad\quad
      \begin{tikzpicture}
        \draw(0,0)circle(1);
        \draw[fill=\currentcolor!80] (1,0)--(0,0) --(142:1) arc (142:0:1);
        \node at(-1,1){(c)};
      \end{tikzpicture}
 \end{figure}
      
\item{}
Possivelmente você encontrou alguma dificuldade para determinar a resposta correta no item anterior. Que tal acrescentarmos uma informação a mais para ajudar na decisão?

\begin{figure}[H]
\centering


\begin{tikzpicture}
  \draw(0,0)circle(1);
  \draw[fill=\currentcolor!80] (1,0)--(0,0) --(110:1) arc (110:0:1);
  \node at(-1,1){(a)};
  \draw[atento] (.2,0) arc (0:110:.2);
  \node at(.4,.3){\tiny $ 110^\circ$};
   \end{tikzpicture}\quad\quad\quad
      \begin{tikzpicture}
        \draw(0,0)circle(1);
        \draw[fill=\currentcolor!80] (1,0)--(0,0) --(126:1) arc (126:0:1);
        \node at(-1,1){(b)};\draw[atento] (.2,0) arc (0:126:.2);
        \node at(.4,.3){\tiny $126^\circ$};
 \end{tikzpicture}\quad\quad\quad
      \begin{tikzpicture}
        \draw(0,0)circle(1);
        \draw[fill=\currentcolor!80] (1,0)--(0,0) --(142:1) arc (142:0:1);
        \node at(-1,1){(c)};
        \draw[atento] (.2,0) arc (0:142:.2);
        \node at(.4,.3){\tiny $ 142^\circ$};
 \end{tikzpicture}
 \end{figure}

\item{}
E agora? Como você usou a medida do ângulo que determina o setor circular para ajudar no cálculo da área? Vamos fazer mais uma vez! Um dos setores apresentados a seguir tem área \(4\). Determine esse setor.

\begin{figure}[H]
\centering


 \begin{tikzpicture}
\draw(0,0)circle(1);\draw[fill=\currentcolor!80] (1,0)--(0,0) --(72:1) arc (72:0:1);\node at(-1,1){(a)}; \draw[atento] (.2,0) arc (0:72:.2);\node at(.4,.2){\tiny $ 72^\circ$}; \end{tikzpicture}\quad\quad\quad
      \begin{tikzpicture}
\draw(0,0)circle(1);\draw[fill=\currentcolor!80] (1,0)--(0,0) --(60:1) arc (60:0:1);\node at(-1,1){(b)};\draw[atento] (.2,0) arc (0:60:.2);\node at(.5,.2){\tiny $ 60^\circ$}; \end{tikzpicture}\quad\quad\quad
      \begin{tikzpicture}
\draw(0,0)circle(1);\draw[fill=\currentcolor!80] (1,0)--(0,0) --(45:1) arc (45:0:1);\node at(-1,1){(c)};\draw[atento] (.2,0) arc (0:45:.2);\node at(.5,.2){\tiny $ 45^\circ$};
 \end{tikzpicture}%\label{fig-setor5}
\end{figure}

\item{}
Determine a função que relaciona a área do setor circular com o seu ângulo central, especificando seu domínio.

\end{enumerate}
\end{task}

\begin{reflection}

Em uma circunferência, podemos relacionar a área \(A\) e o raio \(r\) por meio da função\linebreak \(A(r)=\pi r^2\). Aumentando o raio da circunferência, sua área também aumenta. Isso nos indica que a função \(A\) é crescente. Reflita um pouco e responda: Essa função é linear? Ou seja, a área de um círculo é proporcional ao seu raio?

\textbf{Pense no seguinte caso:} A área de um círculo de raio \(2r\) é igual ao dobro da área de um círculo de raio \(r\)? Ou ainda, é possível encontrar um número real (fixo) tal que \(A(r)=k\cdot r\)?


\begin{figure}[H]
\centering

\begin{tikzpicture}
\fill[\currentcolor!80](-2,0)circle(2cm);
\node[right]at(0.2,0){\Huge =};
\fill[\currentcolor!80](2,0)circle(1cm);
\node[right]at(3.2,0){\Huge +};
\fill[\currentcolor!80](5,0)circle(1cm);
\node[right]at(6.2,0){\Huge ?};
\draw(-2,0)--+(50:2);
\node[left]at(-1.4,.8){$2r$};
\draw(2,0)--+(50:1);
\node[left]at(2.4,.5){$r$};
\draw(5,0)--+(50:1);
\node[left]at(5.4,.5){$r$};
\end{tikzpicture}

\end{figure}
\end{reflection}


\explore{ taxa de variação média}
\label{\detokenize{AF107-2:explorando-taxa-de-variacao-media}}\label{\detokenize{AF107-2::doc}}
\begin{task}{Teor de álcool sanguíneo}

\label{ativ-alcool}

De acordo com o site \href{https://pt.wikihow.com/Calcular-o-N\%C3\%ADvel-de-\%C3\%81lcool-no-Sangue}{wikiHow} o Teor Alcoólico Sanguíneo, ou TAS, é a medida da proporção de álcool no sangue de uma pessoa. Um TAS de \(0,08\) indica que há \(80mg\) de álcool por \(100ml\) de sangue. O álcool é absorvido de forma diferente pelos homens e pelas mulheres. O corpo masculino geralmente tem mais água (\(61\%\) \emph{versus} \(52\%\)) e, portanto, dilui melhor o álcool, gerando TAS mais baixos.

O TAS é proporcional ao número de doses de bebida consumidas, de maneira que para um homem de \(75kg\), a função linear \(h(x)\) que relaciona o TAS com o número de doses \(x\) de bebida é dada pela expressão
\begin{equation*}
\begin{split}h(x)=0,0205 \cdot x.\end{split}
\end{equation*}
Para uma mulher que pesa \(60 kg\), a mesma relação é dada pela função linear
\begin{equation*}
\begin{split}m(x)=0,0307 \cdot x.\end{split}
\end{equation*}\begin{enumerate}
\item {} 
Complete a tabela a seguir que relaciona os valores de \(h(x)\) e de \(m(x)\) correspondentes a valores inteiros de \(x\), de \(0\) a \(5\).

\begin{table}[H]
\centering
\begin{tabu} to \textwidth{|l|c|c|}
\hline
\thead
\(x\) & \(h(x)\) & \(m(x)\) \\
\hline
0 & & \\
\hline
1 & & \\
\hline
2 & & \\
\hline
3 & & \\
\hline
4 & & \\
\hline
5 & & \\
\hline
\end{tabu}
\end{table}

\item {} 
Calcule, para a função \(h(x)\), as taxas de variação médias nos seguintes intervalos de valores de \(x\):

b.1) entre \(x=0\) e \(x=1\);

b.2) entre \(x=1\) e \(x=3\);

b.3) entre \(x=2\) e \(x=5\);

\item {} 
Repita o item anterior para a função \(m(x)\) nos intervalos:

c.1) entre \(x=2\) e \(x=3\);

c.2) entre \(x=1\) e \(x=4\);

c.3) entre \(x=0\) e \(x=5\);

\item {} 
A partir dos itens anteriores, faça uma conjectura sobre as taxas de variação médias de uma função linear qualquer.

\end{enumerate}
\end{task}

\begin{task}{Câmara frigorífica}
\label{ativ-camara}


Uma câmara frigorífica está programada para diminuir sua temperatura segundo uma taxa constante em \(^\circ C\) por hora. Na primeira observação constata-se que ela está a \(0^\circ C\). Após \(8\) horas, realiza-se uma nova observação e seu visor mostra a temperatura de \(-24^\circ C\) e também o seguinte gráfico para a evolução da temperatura em função do tempo.

\begin{figure}[H]
\centering

\begin{tikzpicture}[yscale=.75, xscale=1.25]
\tikzstyle{ponto}=[circle, minimum size=3pt, inner sep=0, draw=black, fill=black, shift only]
\draw(-3,-.05) grid(5,.05);
\draw(-.05,-12) grid(.05,2);
\draw[->, thick](-3,0)--(5,0);
\draw[->, thick](0,-12)--(0,2);
\draw[\currentcolor!80, thick](0,0)--(4,-12);
\draw[dashed](0,-12)--(4,-12)--(4,0);
\node[ponto]at (0,0){};
\node[ponto]at (4,-12){};
\node[above,rotate=90] at(-1,-10){Temperatura $^\circ$C};
\node[above] at(4,0){Tempo (h)};
\foreach\x in{-4, -2, 0, 2, 4, 6, 8}
\node[below left] at (.5*\x, 0){\x};
\foreach \y in{-24, -22, -20, ..., -2}
\node[left]at(0,.5*\y){\y};
\node[left]at(0,1){2};
\node[left]at(0,2){4};
\end{tikzpicture}
\end{figure}
\begin{enumerate}
\item {} 
Qual a temperatura da câmara \(1\) hora após a primeira observação? E \(5\) horas após a primeira observação? E \(t\) horas após a primeira observação?

\item {} 
Qual o valor da taxa (de variação média) constante segundo a qual a temperatura diminui?

\item {} 
Determine a função que relaciona temperatura e tempo nesse contexto, considerando para seu domínio o intervalo de números reais \([0,8]\). Ela é uma função crescente ou decrescente? Por que?

\item {} 
Como seria o gráfico se a temperatura, no mesmo intervalo de tempo, ao invés de diminuir, estivesse aumentando \(1,5^\circ C/h\)? Qual seria a expressão da função, nesse caso? Teríamos uma função crescente ou decrescente? Por que?

\end{enumerate}
\end{task}

\begin{task}{Hora de carregar o celular}
\label{ativ-celular}

\begin{figure}[H]
\centering

\noindent\includegraphics[width=100bp]{{celular}.jpg}
\end{figure}

O tempo total de recarga da bateria (de \(0\%\) a \(100\%\)) de um determinado modelo de telefone celular é  de \(2\) horas e \(5\) minutos. Supondo que o carregamento ocorre segundo uma taxa constante:

\begin{enumerate}
\item {} 
Faça uma tabela que forneça o percentual de carga na bateria a cada \(25\) minutos, a partir de zero.

\item {} 
Represente em um plano cartesiano os pontos da tabela do item anterior.

\item {} 
Descreva uma estratégia que permita, a partir da representação gráfica obtida no item anterior, determinar o percentual de carga na bateria após \(40\) minutos de carregamento.

\item {} 
Determine a função que modela o carregamento desse modelo de telefone, especificando seus domínio e conjunto imagem.

\item {} 
Qual é a taxa de carregamento desse modelo de telefone celular.
\end{enumerate}
\end{task}

\arrange{taxa de variação média}
\label{\detokenize{AF107-2:organizando-as-ideias-taxa-de-variacao-media}}
No capítulo de Taxa de variação, você aprendeu a calcular a taxa de variação média de uma função em um determinado intervalo. É um número expresso em forma de uma razão que fornece diversas informações sobre o comportamento da função no intervalo considerado.

Relembrando, se um intervalo \([x_1,x_2]\) está contido no domínio de uma função \(f\), então a taxa de variação média dessa função nesse intervalo é a razão
\begin{equation*}
\begin{split}\dfrac{f(x_2)-f(x_1)}{x_2-x_1}\end{split}.
\end{equation*}
Como você deve ter percebido na \DUrole{xref,std,std-ref}{Atividade Teor de álcool sanguíneo}, o valor obtido para as taxas de variação médias nos diversos intervalos foi sempre o mesmo para cada função considerada. Essa é uma propriedade importante das funções lineares, que provaremos agora.

Considere uma função linear \(\ell:\mathbb{R}\to\mathbb{R}\), dada por \(\ell(x)=a\cdot x\), e também dois números reais distintos \(x_1<x_2\). A taxa de variação média de \(\ell\) no intervalo  \([x_1,x_2]\) pode ser calculada assim
\begin{equation*}
\begin{split}\dfrac{\ell(x_2)-\ell(x_1)}{x_2-x_1}=\dfrac{a x_2- a x_1}{x_2-x_1}=\dfrac{a(x_2-x_1)}{x_2-x_1}=a.\end{split}
\end{equation*}
Podemos destacar duas coisas sobre a conclusão deste último cálculo:
\begin{itemize}
\item {} 
o valor final para a taxa de variação média não depende dos valores de \(x_1\) e \(x_2\). Isso significa que podemos escolher qualquer intervalo  de números reais e chegaremos ao mesmo resultado.

\item {} 
o resultado coincide com o coeficiente de \(x\) na expressão da função, e também pode ser obtido calculando-se a imagem de \(x=1\). Sendo assim, podemos afirmar que a função \(y=7x\) tem taxa de variação média constante igual a \(7\), enquanto que a função \(y=-\frac {3x}5\) tem taxa de variação média constante igual a \(-\frac {3}5\).

\end{itemize}

\begin{description}
\item[Teorema]
Toda função linear \(f\) tem taxa de variação média constante igual a \(f(1)\), e pode ser representada pela expressão \(f(x)=f(1)\cdot x\).
\end{description}


Usando essas ideias no contexto da \DUrole{xref,std,std-ref}{Atividade Câmara frigorífica}, podemos afirmar que a expressão da temperatura em função do tempo, mostrada pelo gráfico pode ser dada por \(f(t)=-3t\), uma vez que \(f(1)=-3\). A cada hora a temperatura decresce \(3^\circ C\), gerando portanto uma função decrescente.

De uma maneira geral, se a taxa de variação média \(a\) de uma função linear é um número real \textbf{negativo}, então essa função é decrescente, pois, para \(a<0\)
\begin{equation*}
\begin{split}x_1<x_2 \Longleftrightarrow ax_1>ax_2 \Longleftrightarrow f(x_1)>f(x_2).\end{split}
\end{equation*}
Por outro lado, se a taxa de variação média \(a\) de uma função linear é um número real \textbf{positivo}, então essa função é crescente, pois, nesse caso \(a>0\) e
\begin{equation*}
\begin{split}x_1<x_2 \Longleftrightarrow ax_1<ax_2 \Longleftrightarrow f(x_1)<f(x_2).\end{split}
\end{equation*}
Vamos agora entender como é a representação gráfica de uma função com taxa de variação média constante. Para isso, consideremos uma função \(f:\mathbb{R}\to\mathbb{R}\) que tenha essa propriedade, isto é, para qualquer intervalo a taxa de variação média de \(f\) neste intervalo é igual a \(a\).

Na figura a seguir, os pontos \(A=(x_1,f(x_1))\) e \(B=(x_2,f(x_2))\) pertencem ao gráfico da função \(f\). O segmento \(BC\) mede \(f(x_2)-f(x_1)\) e o segmento \(AC\) mede \(x_2-x_1\). Dessa forma o quociente \(\dfrac{\overline{BC}}{\overline{AC}}\) é igual à taxa de variação média da função nesse intervalo, e portanto podemos conluir que \(\overline{BC}=a\cdot \overline{AC}\).

\begin{figure}[H]
\centering


\begin{tikzpicture}
\tikzstyle{ponto}=[circle, minimum size=2pt, inner sep=0, draw=black, fill=black, shift only]
\draw[->, thick](-.5,0)--(4,0);
\draw[->, thick](0,-.5)--(0,3.5);
\draw[dashed](0,1)--(3,1);
\draw[dashed](0,2.5)--(3,2.5);
\draw[dashed](1,0)--(1,1)--(3,2.5)--(3,0);
\node[below] at(1,0){$x_1$};
\node[below] at(3,0){$x_2$};
\node[below] at(2,1){$x_2-x_1$};
\node[right, rotate=0] at(3,1.75){$f(x_2)-f(x_1)$};
\node[left] at(0,1){$f(x_1)$};
\node[left] at(0,2.5){$f(x_2)$};
\node[ponto]at(1,1){};
\node[above]at(1,1){$A$};
\node[ponto]at(3,1){};
\node[right]at(3,1){$C$};
\node[ponto]at(3,2.5){};
\node[right]at(3,2.5){$B$};
\end{tikzpicture}
\end{figure}

\begin{equation*}
\begin{split}\dfrac{\overline{BC}}{\overline{AC}}= \dfrac{f(x_2)-f(x_1)}{x_2-x_1}=a \Longrightarrow \overline{BC}=a\cdot \overline{AC}.\end{split}
\end{equation*}
Por isso, quaisquer dois pontos do gráfico de \(f\), sempre serão extremidades da hipotenusa de um triângulo retângulo cujos catetos são paralelos aos eixos e suas medidas se relacionam conforme a seguinte figura.

\begin{figure}[H]
\centering

\begin{tikzpicture}
\tikzstyle{ponto}=[circle, minimum size=2pt, inner sep=0, draw=black, fill=black, shift only]
\draw[->, thick](-.5,0)--(4,0);
\draw[->, thick](0,-.5)--(0,3.5);
\draw[dashed](1,1)--(3,1);
\draw[dashed](1,1)--(3,2.5)--(3,1);
\node[below] at(2,1){$d$};
\node[right, rotate=0] at(3,1.75){$a\cdot d$};
\node[ponto]at(1,1){};
\node[ponto]at(3,1){};
\node[ponto]at(3,2.5){};
\end{tikzpicture}
\end{figure}

Consideremos agora três pontos do gráfico de \(f\) com os respectivos triângulos retângulos da construção anterior.
\begin{figure}[H]
\centering

\begin{tikzpicture}
\tikzstyle{ponto}=[circle, minimum size=2pt, inner sep=0, draw=black, fill=black, shift only]
\draw[->, thick](-.5,0)--(4,0);
\draw[->, thick](0,-.5)--(0,3.5);
\draw[dashed](.5,1)--(3,3.5)--(3,2)--(1.5,2)--(1.5,1)--cycle;
\node[ponto]at(.5,1){};
\node[ponto]at(1.5,2){};
\node[ponto]at(3,3.5){};
\node[below]at(1,1){$d$};
\node[right]at(1.5,1.4){$a\cdot d$};
\node[above]at(2.3,2){$D$};
\node[right]at(3,2.7){$a\cdot D$};
\draw(1.5,1)rectangle++(-1mm, 1mm);
\draw(3,2)rectangle++(-1mm, 1mm);
\end{tikzpicture}
\end{figure}
Como os triângulos são semelhantes e têm um ponto em comum, podemos concluir que os três pontos pertencem a uma mesma reta. A conclusão é válida quaisquer que sejam os três pontos considerados, logo acabamos de justificar a seguinte propriedade.

\begin{description}
\item[Teorema]

Se uma função tem taxa de variação média constante então seu gráfico está contido em uma reta.

Em particular, como a função linear tem taxa de variação média constante, seu gráfico está contido em uma reta.
\end{description}


\begin{observation}{Algumas propriedades da função linear}


\begin{itemize}
\item {} 
Sempre que fizer sentido calcular a imagem de \(x=0\), teremos \(f(0)=a \cdot 0 = 0\), isto é, a origem \((0,0)\) do plano cartesiano pertencerá ao gráfico de \(f\). Em qualquer caso, o gráfico de uma função linear está contido em uma reta que passa pela origem (mesmo quando não fizer sentido calcular a imagem de \(x=0\)).

\item {} 
A taxa de variação da função linear \(f(x)=ax\) também pode ser calculada fazendo-se a diferença entre as imagens de dois valores que distam \(1\) entre si da seguinte maneira:

\end{itemize}
\begin{equation*}
\begin{split}f(x+1)-f(x)=a(x+1)-ax=ax+a-ax=a\end{split}
\end{equation*}

\begin{figure}[H]
\centering

\begin{tikzpicture}
\tikzstyle{ponto}=[circle, minimum size=2pt, inner sep=0, draw=black, fill=black, shift only]
\draw[thick,->](-1,0)--(4,0);
\draw[thick,->](0,-1)--(0,4);
\draw[domain=-.5:2, thick, \currentcolor!80]plot(\x,2*\x);
\draw[dashed, thick](.5,0)--(.5,1)--(0,1);
\draw[dashed, thick](1.5,0)--(1.5,3)--(0,3);
\draw[dashed, thick](.5,1)--(1.5,1);
\node[ponto] at(.5,1){};
\node[ponto] at(1.5,3){};
\node[right] at(1.5,2){$a$};
\node[below] at(1,1){$1$};
\node[below] at(1.5,.1){$x+1$};
\node[below] at(.5,0){$x$};
\node[left] at(0,3){$f(x+1)$};
\node[left] at(0,1){$f(x)$};
\end{tikzpicture}
\end{figure}

\begin{itemize}
\item {} 
%Para taxas de variação médias positivas, quanto maior for o valor de \(a\), mais inclinada será a reta que contém o gráfico da função linear associada.
A função linear $f(x)=x$ é chamada \textbf{função identidade}.

\begin{figure}[H]
\centering

\begin{tikzpicture}[scale=0.5, every node/.style={scale=0.7}]

\draw [gray!50](-6.5,-6.5) grid (6.5,6.5);
\draw [->] (-6.5,0) -- (6.5,0);
\draw [->] (0,-6.5) -- (0,6.5);
\foreach \x in {-6,...,-1,1,2,...,6}{
\node [below] at (\x,0) {\x};
\node [left] at (0,\x) {\x};
}

\draw [very thick, \currentcolor!80] (-6.5,-6.5) -- (6.5,6.5);
\node [above left] at (0,0) {0};
\end{tikzpicture}
\end{figure}

\end{itemize}

% \begin{figure}[H]
% \centering

% \noindent\includegraphics[width=400bp]{{aumenta_a}.png}
% \end{figure}

Para uma visualização do comportamento da representação gráfica com taxa de variação média também negativa, sugerimos o uso da construção GeoGebra disponível \href{https://www.geogebra.org/m/FSnzt9vC}{neste link} .

\begin{figure}[H]
\centering

\noindent\includegraphics[width=100bp]{{codigo2_2}.png}
\end{figure}

%\begin{figure}[H]
%\centering
%
%\noindent\includegraphics[width=400bp]{{taxa_linear}.png}
%\end{figure}
\begin{itemize}
\item {} 
Se uma reta contém a origem do plano cartesiano e o ponto \((x_0,y_0)\) com \(x_0\neq 0\), então ela é o gráfico da função linear \(f:\mathbb{R}\to\mathbb{R}\), dada por \(f(x)=ax\), em que \(a=\dfrac{y_0}{x_0}\).

\end{itemize}

Para verificar isso, basta observarmos uma reta nas condições dadas e os dois  triângulos retângulos destacados da figura a seguir a partir da origem e dos pontos \((x_0,y_0)\) e \((x,y)\). Observe que, qualquer que seja o ponto \((x,y)\) escolhido diferente da origem, esses triângulos são semelhantes, portanto,
\begin{equation*}
\begin{split}\dfrac{f(x)}{x}=\dfrac{y_0}{x_0} \Longrightarrow f(x)=\dfrac{y_0}{x_0} \cdot x\end{split}
\end{equation*}

\begin{figure}[H]
\centering

\begin{tikzpicture}
\tikzstyle{ponto}=[circle, minimum size=2pt, inner sep=0, draw=black, fill=black, shift only]
\usetikzlibrary[patterns]
\fill[color=terciario](2,0)--(2,1.5)--(0,0)-- cycle;
\fill[pattern color=secundario, pattern =north east lines](3,0)--(3,2.25)--(0,0)-- cycle;
\draw[dashed](2,0)--(2,1.5)--(0,1.5);
\draw[dashed](3,0)--(3,2.25)--(0,2.25);
\draw[thick, ->](-1.5,0)--(3.5,0);
\draw[thick, ->] (0,-1.5)--(0,3.5);
\draw[thick,\currentcolor!80,domain=-1.5:4]plot(\x,.75*\x);
\node[ponto]at(2, 1.5){};
\node[ponto]at(3, 2.25){};
\node[below] at(2,0){$x$};
\node[below] at(3,0){$x_0$};
\node[left] at(0,1.5){$f(x)$};
\node[left] at(0,2.25){$y_0$};
\end{tikzpicture}

\end{figure}
Assim, por exemplo, a reta que contém a origem e o ponto \((3,8)\) é o gráfico da função \(f(x)=\dfrac 83 x\). Se a reta contém a origem e o ponto \((-5,2)\) ela será o gráfico da função \(g(x)=\dfrac{2}{-5} x=-\dfrac{2}{5}x\).
\begin{figure}[H]
\centering

\begin{tikzpicture}
\tikzstyle{ponto}=[circle, minimum size=2pt, inner sep=0, draw=black, fill=black, shift only]
\draw[thick, ->](-2,0)--(2.5,0);
\draw[thick, ->] (0,-1.5)--(0,3.5);
\draw[thick,\currentcolor!80,samples=100,domain=-.5:1.15]plot(\x, 3*\x);
\draw[dashed](.5,0)--(.5,1.5)--(0,1.5);
\node[ponto]at(.5,1.5){};
\node[right]at(.5,1.5){$(3,8)$};
\node[below]at(.5,0){$3$};
\node[left]at(0,1.5){$8$};
\end{tikzpicture}

\end{figure}
Gráfico da função \(f(x)=\dfrac 83 x\)
\begin{figure}[H]
\centering

\begin{tikzpicture}[scale=1.5]
\tikzstyle{ponto}=[circle, minimum size=2pt, inner sep=0, draw=black, fill=black, shift only]
\draw[thick, ->](-3,0)--(3,0);
\draw[thick, ->] (0,-1.5)--(0,2);
\draw[dashed](-1.5,0)--(-1.5,.5)--(0,.5);
\node[above] at(-1.5,.5){$(-5,2)$};
\node[below] at(-1.5,0){$-5$};
\node[right] at(0,.5){$2$};
\draw[domain=-3:3,thick,\currentcolor!80,samples=100]plot(\x,{-\x/3});
\node[ponto] at(-1.5,.5){};
\end{tikzpicture}
\end{figure}
Gráfico da função \(g(x)=-\dfrac{2}{5}x\).

Concluímos, assim, que toda reta não vertical que contém a origem é o gráfico de uma função linear.
\end{observation}


\practice{taxa de variação média}
\label{\detokenize{AF107-3::doc}}\label{\detokenize{AF107-3:praticando}}

\begin{task}{Quando trocar o filtro do purificador?}
\label{quando-trocar-o-filtro-do-purificador}

Há \(1\) ano você adquiriu um purificador de água com capacidade de refrigeração, e deseja saber quanto tempo falta para realizar a troca do filtro interno. No manual do fabricante do seu purificador, você encontra o seguinte quadro:

%\centering
\setlength\tabulinesep{1mm}
\begin{longtabu} to \textwidth{|c|c|c|c|c|c|}
\hline\endfirsthead
%\thead
\cellcolor{\currentcolor!80}&\cellcolor{\currentcolor!80}{\textcolor{black}{\textbf{FIT}}}&\cellcolor{\currentcolor!80}{\textcolor{black}{\textbf{FLAT}}}&\cellcolor{\currentcolor!80}{\textcolor{black}{\textbf{PLUS}}}&\cellcolor{\currentcolor!80}{\textcolor{black}{\textbf{SLIM}}}&\cellcolor{\currentcolor!80}{\textcolor{black}{\textbf{STAR}}}\\
\hline
\makecell{Dimensões \\ Altura \\ Largura \\ Profundidade} &\makecell{ 27cm \\ 29cm \\ 36cm} & \makecell{29cm \\ 36cm \\36cm} & \makecell{40cm \\ 30cm \\ 45cm} & \makecell{36cm \\ 25cm \\ 41cm} & \makecell{40cm \\ 30cm \\ 36cm}\\
\hline
Peso bruto & 13kg & 12kg &14kg & 13kg & 13kg \\
\hline
\parbox{2cm}{\centering Capacidade de refrigeração com ambiente a $32^{\circ}$C e água a $27^{\circ}$C} & \parbox{2cm}{\centering 1,1 litros/hora (atende até 15 pessoas)} & \parbox{2cm}{\centering 1,5 litros/hora (atende até 10 pessoas)} & \parbox{2cm}{\centering 4,4 litros/hora (atende até 30 pessoas)} & \parbox{2cm}{\centering 1,5 litros/hora (atende até 10 pessoas)} & \parbox{2cm}{\centering 2,2 litros/hora (atende até 15 pessoas)}\\ 
\hline 
\parbox{2cm}{\centering Capacidade de armazenamento de águal gelada} & 1,2 litros & 1.5 litros & 2 litros & 1.5 litros & 2 litros \\
\hline
\makecell{Gás \\ refrigerante} & R134a & R134a & R134a & R134a & R134a \\
\hline
Carga de gás & 36g & 32g & 32g & 32g & 36g \\
\hline
Tensão & \parbox{2cm}{\centering 127V ou 220V-60Hz} & \parbox{2cm}{\centering 127V ou 220V-60Hz} & \parbox{2cm}{\centering 127V ou 220V-60Hz} &  \parbox{2cm}{\centering 127V ou 220V-60Hz} &  \parbox{2cm}{\centering 127V ou 220V-60Hz} \\
\hline 
Potência & 100W & 100W & 100W & 100W & 100W \\
\hline
Pressão nominal & \multicolumn{5}{c|}{0,196MPa (30 metros de coluna de água)} \\
\hline
\parbox{3cm}{\centering Temperatura min/max de rede hidráulica} & \multicolumn{5}{c|}{0,29MPa a 0,392MPa (3 a 40 metros de coluna de água)}\\
\hline
\parbox{3cm}{\centering Temperatura min/max de trabalho}&\multicolumn{5}{c|}{$5^{\circ}$C a $42^{\circ}$C}\\
\hline
\parbox{2cm}{\centering Vazão elemento filtrante} & \multicolumn{5}{c|}{4.000 litros} \\
\hline
\parbox{2cm}{\centering Vazão máxima recomendada} & \multicolumn{5}{c|}{0,75 litros/minuto} \\
\hline
\parbox{2cm}{\centering Volume interno do aparelho} & 1,6 litros & 2 litros & 2,5 litros & 2 litros & 2,5 litros \\
\hline
\parbox{2cm}{\centering Volume de referência para ensio de particulado} &\multicolumn{5}{c|}{4.000 litros}\\
\hline
\end{longtabu}
\begin{enumerate}
\item Quais informações do quadro são relevantes para responder à sua dúvida?

\item Explique com suas palavras o significado da vazão 0,75 litros/minuto.

\item Para calcular a vida útil do seu filtro interno, é necessário estimar a quantidade de água consumida diariamente na sua casa. Suponha, então, que você observou que o purificador é acionado ao longo de um dia o equivalente ao tempo total de 12 minutos. Quantos litros de água são consumidos em um dia, nessas condições? (assuma que o purificador foi regulado para funcionar com a vazão máxima recomendada pelo fabricante)

\item Assumindo que o consumo estimado no item anterior seja o mesmo para todos os dias, qual foi o consumo de água do purificador ao final do primeiro dia de uso? E entre o 10º e o 11º dias de uso?

\item Qual o aumento do consumo de água observado para cada dia de uso do purificador?

\item Calcule a vida útil do filtro interno do seu aparelho e, supondo que você tenha utilizado o seu purificador todos os dias desde a instalação, determine em quanto tempo você deverá solicitar a troca do seu filtro interno.

\item Com base nas informações que você possui, encontre uma expressão matemática que relacione o consumo de água do purificador em função do tempo de uso em dias e represente-a graficamente.
\end{enumerate}

\end{task}


\explore{ função afim}
\label{\detokenize{AF107-4:sec-funcao-afim}}\label{\detokenize{AF107-4::doc}}\label{\detokenize{AF107-4:explorando-funcao-afim}}

\begin{task}{distância x tempo}
\label{ativ-dist-tempo}

O gráfico a seguir mostra a variação da distância a um ponto de referência de uma pessoa que caminha em linha reta durante \(4\) segundos. Algumas informações sobre o movimento podem ser extraídas diretamente dessa representação gráfica.
\phantomsection\label{\detokenize{AF107-4:fig-tempo-distancia}}
\begin{figure}[H]
\centering

\begin{tikzpicture}[yscale=.75]

\draw [dashed, help lines, thin] (0,0) grid (4.5,9.5);
\draw [->] (-1,0) -- (4.5,0);
\draw [->] (0,-1) -- (0,9.5);

\draw [\currentcolor, very thick] (0,1) -- (4,9) node [above right] {$f$};
\draw [thick, dashed] (0,1) -- (4.5,1);
\draw [thick, dashed] (4,0) -- (4,9);
\draw [thick, dashed] (0,9) -- (4,9);

\foreach \x in {1,...,4}{
\draw [ thick] (\x,-.1) -- (\x,.1);
\node [below] at (\x,-.1) {\x};}

\foreach \x in {2,4,...,8} \node [left] at (-.1,\x) {\x};
\foreach \x in {1,...,9} \draw [thick] (-.1,\x) -- (.1,\x);

\node [below] at (2,1) {tempo decorrido};
\node [left] at (0,1) {início};
\node [rotate=90, above] at (-.5,8) {distância ($m$)};
\node [rotate=90, below] at (4.5,5) {distância percorrida};
\node [right] at (4.5,0) {tempo ($s$)};
\end{tikzpicture}


\label{\detokenize{AF107-4:fig-tempo-distancia}}\end{figure}
\begin{enumerate}
\item {} 
A que distância do ponto de referência estava a pessoa no início da contagem do tempo? E ao final de \(4\) segundos?

\item {} 
Qual a distância total percorrida?

\item {} 
É possível saber se ele está se afastando ou se aproximando do ponto de referência? Como?

\item {} 
Qual a velocidade média da pessoa no intervalo de tempo de \(0\) a \(4\) segundos?

\item {} 
A que distância a pessoa estava do ponto de referência após \(1\) segundo do início da caminhada? Qual a distância percorrida por ela nesse intervalo de tempo?

\item {} 
Qual a distância percorrida pela pessoa entre \(1\) e \(2\) segundos? E entre \(3\) e \(4\) segundos?

\item {} 
O que se pode concluir, a partir das suas respostas nos itens (e) e (f), sobre a variação da distância percorrida a cada minuto de caminhada?

\item {} 
As grandezas relacionadas pelo gráfico são proporcionais? Porque?

\end{enumerate}

\end{task}


\arrange{ função afim}
\label{\detokenize{AF107-4:organizando-as-ideias-funcao-afim}}
O gráfico da atividade \hyperref[ativ-dist-tempo]{distância x tempo} apresenta a distância até o ponto de referência como uma função do tempo. Como a ação acontece no intervalo de tempo de \(0\) a \(4\) segundos, podemos modelar essa relação com uma função \(f\) cujo domínio é o intervalo \([0,4]\), isto é, \(f:[0,4] \to \mathbb{R}\). Vamos juntos determinar uma expressão para \(f\)?

Apesar do gráfico ser um segmento de reta, \(f\) não pode ser uma função linear, pois a reta não contém a origem do plano cartesiano. Como no início da contagem do tempo a pessoa estava na posição \(1\), devemos ter \(f(0)=1\). Em particular, as grandezas posição e tempo, nesse caso, não são proporcionais.

Vamos considerar, como na atividade, a relação de outra grandeza com o tempo: a distância percorrida do início até o instante $ t $. Nesse caso temos proporcionalidade e vamos ver por quê.

\begin{minipage}{0.6\textwidth}
Para facilitar a discussão, chamemos essa função de $ d(t) $. Seu domínio coincide com o domínio de $ f $ e podemos dizer que \(d(t)\) é a distância percorrida no intervalo de tempo \([0,t]\). Por exemplo, como fazemos para calcular \(d(2)\) ? Basta saber em que posições a pessoa estava nos tempos \(t=0\) e \(t=2\) e fazer a diferença. Neste caso \(d(2)=f(2)-f(0)=5-1=4\). Isso nos dá a dica de como calcular \(d(t)\) para qualquer \(t\in[0,4]\).
\end{minipage}
\begin{minipage}{0.3\textwidth}
\begin{table}[H]
\centering
\begin{tabu} to \textwidth{|l|c|}
\hline
\thead
$t$ & $d(t)$ \\ 
\hline 
$0$ & $ 1-1=0 $ \\ 
\hline 
$1$ & $ 3-1=2 $ \\ 
\hline 
$2$ & $ 5-1=4 $ \\ 
\hline 
$3$ & $ 7-1=6 $ \\ 
\hline 
$4$ & $ 9-1=8 $ \\ 
\hline
\end{tabu}
\end{table}
\end{minipage}

\[
d(t)=f(t)-f(0) \Longrightarrow d(t)= f(t)-1.
\]

Por causa dessa última relação, o gráfico de $ d $ será exatamente igual ao gráfico de $ f $, deslocado uma unidade para baixo. Ou seja, uma reta paralela a que é o gráfico de $ f $ e que passa pela origem. Conclusão, $d(t)$ é uma função linear, e, como $ d(1)=2 $, podemos afirmar que $ d(t)=2t $.
\begin{figure}[H]
	\centering
\begin{tikzpicture}[yscale=.75]

\draw [dashed, help lines, thin] (0,0) grid (4.5,9.5);
\draw [->] (-1,0) -- (4.5,0);
\draw [->] (0,-1) -- (0,9.5);

\draw [dashed, very thick] (0,1) -- (4,9) node [above right] {$f$};
\draw [\currentcolor, very thick] (0,0) -- (4,8) [above right] node {$d$};

\fill (2.5,5) circle (2.pt);
\fill (2.5,6) circle (2pt);
\draw [dashed, thick] (2.5,5) -- (2.5,6);

\foreach \x in {1,...,4}{
\draw [ thick] (\x,-.1) -- (\x,.1);
\node [below] at (\x,-.1) {\x};}

\foreach \x in {2,4,...,8} \node [left] at (-.1,\x) {\x};
\foreach \x in {1,...,9} \draw [thick] (-.1,\x) -- (.1,\x);

\node [below] at (2,1) {tempo decorrido};
\node [left] at (0,1) {início};
\node [rotate=90, above] at (-.5,8) {distância ($m$)};
\node [rotate=90, below] at (4.5,5) {distância percorrida};
\node [right] at (4.5,0) {tempo ($s$)};
\end{tikzpicture}
\end{figure}

Pelo que tínhamos visto, $ d(t)=f(t)-1 $, o que nos leva à expressão
\[
f(t)=2t+1
\]

\begin{observation}{}
Chamamos qualquer função que pode ser escrita dessa forma de uma \textbf{função afim}, isto é, toda função real que pode ser escrita da forma
\[
f(x)=ax+b
\]
para quaisquer números reais $ a\neq 0 $ e $ b$. O gráfico de uma função afim é sempre uma reta, paralela à reta gráfico da função linear $ \ell(x)=ax $ deslocada $b$ unidades na vertical.

O número $ a $ é a \textbf{taxa de variação média} da função afim $ f $ (e da função linear também) em qualquer intervalo.
\[
\dfrac{\Delta f}{\Delta x}= \dfrac{f(x_2)-f(x_1)}{x_2-x_1}= \dfrac{ax_2+b-(ax_1+b)}{x_2-x_1}= \dfrac{a(x_2-x_1)}{x_2-x_1}=a.
\]


\begin{figure}[H]
\centering

\begin{tikzpicture}

\draw [->] (-2,0) -- (5,0);
\draw [->,thick] (0,-1) -- (0,6);
\draw [|-|, thick] (1,1) -- (1,3.5) node [midway, left] {$b$};
\draw [domain=-1:5, very thick, dashed] plot (\x,{\x}) node [rotate=45, above, shift={(-1.5,0)}] {$\ell(x)=ax$};
\draw [domain= -2:3.5, \currentcolor!80, very thick] plot (\x,{\x+2.5}) node [rotate=45, above, shift={(-1.5,0)}] {$f(x)=ax+b$} ;

\end{tikzpicture}

\end{figure}

\end{observation}

\begin{example}{}

\begin{figure}[H]
\centering
\capstart

\begin{tikzpicture}[scale=.75]

\draw [ thin, step=2, dashed, gray!50] (-4.5,-4.5) grid (8.5,7.5);
\draw [->] (-4.5,0) -- (8.5,0);
\draw [->] (0,-4.5) -- (0,7.5);
\clip (-4.5,-4.5) rectangle (8.5,7.5);

\foreach \x in {-3,...,-1,1,2,...,8} \node [below, scale=.75] at (\x,0) {\x};
\foreach \x in {-4,...,-1,1,2,...,7} \node [left, scale=.75] at (0,\x) {\x};
\node [above left, scale=.75] at (0,0) {0};

\draw [destacado, very thick] plot (\x,{2*\x+5}) node [rotate =63.43494, pos=0.7, shift={(2,2.1)}] {$f(x)=2x +5$};;; 
\draw [very thick, dashed] plot (\x,{2*\x}) node [rotate =63.43494, pos=0.7, shift={(5,0)}, above] {$\ell(x)=2x$};
\draw [domain=-5: 12, \currentcolor, very thick] plot (\x,{2*\x-3}) node [rotate =63.43494, pos=0.7, shift={(5,-1.7)}, above] {$g(x)=2x -3$};;

\end{tikzpicture}

\caption{Exemplos: \(\ell(x)=2x\), \(f(x)=2x+5\) e \(g(x)=2x-3\)}\label{\detokenize{AF107-4:id1}}\end{figure}

\begin{figure}[H]
\centering
\capstart

\begin{tikzpicture}[scale=.75]

\draw [ thin, step=2, dashed, gray!50] (-4.5,-4.5) grid (8.5,7.5);
\draw [->] (-4.5,0) -- (8.5,0);
\draw [->] (0,-4.5) -- (0,7.5);
\clip (-4.5,-4.5) rectangle (8.5,7.5);

\foreach \x in {-3,...,-1,1,2,...,8} \node [below, scale=.75] at (\x,0) {\x};
\foreach \x in {-4,...,-1,1,2,...,7} \node [left, scale=.75] at (0,\x) {\x};
\node [above left,scale=.75] at (0,0) {0};

\draw [destacado, very thick, domain=-2:8.5] plot (\x,{-\x+6}) node [rotate =-45, pos=0.7, shift={(1,3.3)}, above] {$g(x)=-x+6$};

\draw [very thick, dashed] plot (\x,{-\x}) node [rotate =-45, pos=0.7, shift={(-3,0)}, above] {$\ell(x)=-x$};

\draw [domain=-5: 12, \currentcolor, very thick] plot (\x,{-\x-2}) node [rotate =-45, pos=0.7, shift={(0,-2)}, above] {$f(x)=-x-2$};;

\end{tikzpicture}

\caption{Exemplos: \(\ell(x)=-x\), \(f(x)=-x-2\) e \(g(x)=-x+6\)}\label{\detokenize{AF107-4:id2}}\end{figure}
\end{example}

\textbf{A inclinação de uma reta e a taxa de variação média}

\begin{figure}[H]
\centering
\capstart

\noindent\includegraphics[width=200bp]{{1024px-Planalto_Palace_ramp_and_parlatorium}.jpg}
\caption{Flickr: \href{https://commons.wikimedia.org/w/index.php?curid=18864847}{Palácio do Planalto, Brasília, Brasil}}\label{\detokenize{AF107-4:id3}}\end{figure}

Em muitas construções, quando se deseja fazer o acesso entre dois espaços que estão em níveis diferentes, usa-se a rampa como recurso. A rampa é uma superfície, em geral plana, que liga dois níveis diferentes. Uma característica importante de uma rampa é a sua inclinação (ou declive). Rampas muito íngremes podem se tornar obstáculos intransponíveis para cadeirantes, por exemplo. Como medir, então, a inclinação de uma rampa?
A inclinação de uma rampa é o número obtido quando dividimos a altura (diferença entre os níveis) pelo deslocamento horizontal
\begin{figure}[H]
\centering

\noindent\includegraphics[width=300bp]{rampa.png}
\end{figure}
Uma rampa que tem o deslocamento horizontal muito menor que a altura, terá pela fórmula uma inclinação grande, enquanto uma rampa em que o deslocamento horizontal é bem maior que a altura, terá uma inclinação pequena. Rampas de inclinação pequena são as mais desejáveis.

A Associação Brasileira de Normas Técnicas (ABNT) tem normas para a construção de rampas (a NBR 9050:2015 que pode ser acessada \href{http://www.ufpb.br/cia/contents/manuais/abnt-nbr9050-edicao-2015.pdf}{neste link}) que contém o seguinte artigo:


\begin{quote}
6.6.2.1 As rampas devem ter inclinação de acordo com os limites estabelecidos na Tabela. Para inclinação entre 6,25\% e 8,33\% é recomendado criar áreas de descanso (6.5.) nos patamares, a cada 50 m de percurso. Excetuam-se deste requisito as rampas citadas em 10.4 (plateia e palcos), 10.12 (piscinas) e 10.14 (praias).
\end{quote}

\begin{table}[H]
\centering
\begin{tabu} to \textwidth{|c|c|c|}
\hline
\thead
\parbox[1cm]{3.5cm}{\centering\vspace{.3em} Desníveis máximos de cada segmento de rampa\newline$h$} & \parbox[1cm]{3.5cm}{\centering\vspace{.3em} Inclinação admissível em cada segmento de rampa $i$} & \parbox{4cm}{\centering Número máximo de segmentos de rampa} \\[.25cm]
\hline
1,50 & $5,00 (1:20)$ & Sem limite \\
\hline
1,00 & $5,00 (1:20) < i \leq 6,25 (1:16)$ & Sem limite \\
\hline
0,80 & $6,25 (1:16) < i \leq 8,33 (1:12)$ & 15\\
\hline
\end{tabu}
\end{table}

Por exemplo, para uma rampa ter inclinação \(5\% (1:20)\) ela precisa de \(20\) metros de deslocamento horizontal para cada metro de subida, ou dito de outra forma, para cada metro de deslocamento horizontal, a rampa “sobe” \(0,05\) metros. Veja se as rampas ilustradas abaixo estão dentro das especificações da ABNT.

Da mesma forma, a inclinação de uma reta é a medida do quão íngreme ela é em relação ao sistema de coordenadas onde ela está inserida. Dada uma reta em um plano cartesiano, para determinarmos sua inclinação basta fazer a divisão entre a variação das ordenadas e a variação das abscissas para quaisquer dois pontos da reta
\begin{equation*}
\begin{split}\text{inclinação}=\dfrac{y_2-y_1}{x_2-x_1}\end{split}
\end{equation*}
\begin{figure}[H]
\centering
\capstart

\begin{tikzpicture}[scale=.7, every node/.style={scale=1}]

\draw [->] (-2,0) -- (7,0);
\draw [->] (0,-1) -- (0,5);

\draw [dashed] (0,2) -- (1.6,2) -- (1.6,0);
\draw [dashed] (0,4) -- (5.6,4) -- (5.6,0);
\draw [dashed] (1.6,2) -- (5.6,2);

\draw [domain=-2:7, thick] plot (\x,{.5*\x+1.2});

\fill (1.6,2) circle (2pt);
\fill (5.6,4) circle (2pt);

\node [left] at (0,2) {$y_1$};
\node [left] at (0,4) {$y_2$};

\node [below] at (1.6,0) {$x_1$};
\node [below] at (5.6,0) {$x_2$};

\node [above left, shift={(.5,.1)}] at (1.6,2) {$(x_1,y_1)$};
\node [above left, shift={(.5,.1)}] at (5.6,4) {$(x_2,y_2)$};

\node [right] at (5.6,3) {$y_2-y_1$};
\node [below] at (3.6,2) {$x_2-x_1$};
\end{tikzpicture}
\caption{Inclinação de uma reta}\label{\detokenize{AF107-4:fig-inclina}}\label{\detokenize{AF107-4:id4}}\end{figure}

Por exemplo, a reta que contém os pontos \((1,2)\) e \((5,7)\) tem inclinação \(\dfrac{7-2}{5-1}=\dfrac 54\), enquanto a reta que contém os pontos \((-1,3)\) e \((2,-6)\) tem inclinação \(\dfrac{-6-3}{2-(-1)}=\dfrac{-9}{3}=-3\).

\begin{figure}[H]
\centering
\capstart

\begin{tikzpicture}[scale=.7, every node/.style={scale=1}]

\draw [->] (-2,0) -- (7,0);
\draw [->] (0,-1) -- (0,8);

\draw [dashed] (0,2) -- (1,2) -- (1,0);
\draw [dashed] (0,7) -- (5,7) -- (5,0);
\draw [dashed] (1,2) -- (5,2);

\draw [domain=-1:5.5, thick] plot (\x,{(5/4)*\x+0.75});

\fill (1,2) circle (2pt);
\fill (5,7) circle (2pt);

\node [left] at (0,2) {$2$};
\node [left] at (0,7) {$7$};

\node [below] at (1,0) {$1$};
\node [below] at (5,0) {$5$};

\node [right, scale=1] at (5.5,4.5) {$\displaystyle \frac{7-2}{5-1}=\frac{5}{4}$};
\end{tikzpicture}

\caption{Inclinação da reta que contém os pontos \((1,2)\) e \((5,7)\)}\label{\detokenize{AF107-4:fig-inclina-1}}\label{\detokenize{AF107-4:id5}}\end{figure}

\begin{figure}[H]
\centering
\capstart

\begin{tikzpicture}[scale=.6,remember picture, shift={(1,0)}]

\draw [->] (-2.5,0) -- (3.5,0);
\draw [->] (0,-7.5) -- (0,4.5);

\draw [dashed] (0,3) -- (-1,3) -- (-1,0);
\draw [dashed] (0,-6) -- (2,-6) -- (2,0);
\draw [dashed] (-1,3) -- (2,3) -- (2,0);

\draw [domain=-1.5:2.5, thick] plot (\x,{-3*\x});

\fill (-1,3) circle (2pt);
\fill (2,-6) circle (2pt);

\node [above left] at (0,3) {$3$};
\node [left] at (0,-6) {$-6$};

\node [below] at (-1,0) {$-1$};
\node [below right] at (2,0) {$2$};

\node [right, scale=1] at (2.5,1.5) {$\displaystyle \frac{-6-3}{2-(-1)}=-3$};
\end{tikzpicture}
\caption{Inclinação da reta que contém os pontos \((-1,3)\) e \((2,-6)\)}\label{\detokenize{AF107-4:fig-inclina-2}}\label{\detokenize{AF107-4:id6}}\end{figure}

Se considerarmos uma reta no plano e a função afim \(f\) que a tem como gráfico, a inclinação da reta coincide com a taxa de variação da função \(f\).

Por exemplo, a reta da \hyperref[AF107-4:fig-inclina-1]{Figura \ref{AF107-4:fig-inclina-1}} acima é o gráfico da função afim \(f(x)=\dfrac 54 x +b\). Para determinarmos o valor de \(b\), basta substituirmos um dos pontos que estão sobre a reta. Como \((1,2)\) pertence ao gráfico de \(f\), podemos afirmar que \(f(1)=2\), logo
\begin{equation*}
\begin{split}2=\dfrac 54 \cdot 1 + b \Longrightarrow b=2-\dfrac 54 = \dfrac 34\end{split}
\end{equation*}
Isto é, a função \(f\) é dada pela expressão \(f(x)=\dfrac 54 x + \dfrac 34\).

Já a reta da \hyperref[AF107-4:fig-inclina-2]{Figura \ref{AF107-4:fig-inclina-2}} tem expressão \(f(x)=-3x\).

Retas horizontais não possuem inclinação, de fato não há deslocamento vertical quando nos deslocamos horizontalmente sobre a reta, ou seja, sua inclinação será igual a zero.



\practice{função afim}
\label{\detokenize{AF107-6::doc}}\label{\detokenize{AF107-6:praticando}}

\begin{task}{Antecipando o pagamento de uma dívida}
\label{\detokenize{AF107-6:atividade-antecipando-o-pagamento-de-uma-divida}}\label{\detokenize{AF107-6:ativ-titulo-da-atividade}}


O cliente de um banco foi à sua agência de relacionamento para negociar a antecipação de uma dívida no valor de \(R\$ 5.000,00\), que venceria no prazo de 8 meses. Lá, foi informado pelo gerente que a modalidade praticada era a de desconto comercial, no qual o valor a ser descontado é calculado a partir do valor da dívida e é proporcional ao tempo de antecipação. Para o caso em questão, o gerente gerou a seguinte quadro:

\begin{table}[H]
\centering
\begin{tabu} to \textwidth{|l|c|c|c|c|c|}
\hline
\thead
Antecipação (meses) & 0 & 1 & 2 & 3 & 4 \\
\hline
Valor a pagar (R\$) & 5.000 & 4.750 & 4.500 & 4.250 & 4.000 \\
\hline
\end{tabu}
\end{table}

\begin{enumerate}
\item {} 
Observando o padrão de desconto, complete a tabela até o 8º mês de antecipação.

\item {} 
Descreva a variação sofrida pelo valor a pagar à medida que aumenta o tempo de antecipação. O que se pode afirmar sobre a variação observada no valor a pagar entre dois meses consecutivos?

\item {} 
Descreva a variação sofrida pelo valor do desconto à medida que aumenta o tempo de antecipação. O que se pode afirmar sobre a variação observada no valor de desconto entre dois meses consecutivos?

\item {} 
Qual é, então, a taxa mensal de desconto comercial praticada pelo banco?

\item {} 
No plano cartesiano a seguir estão representados os pares ordenados \((n,D(n))\) em que \(n\) é o tempo de antecipação e \(D(n)\) o valor do desconto correspondente. Represente nele os pontos que correspondem aos pares ordenados \((n,V(n))\) em que \(V(n)\) é o valor a pagar no tempo \(n\).

\end{enumerate}
\end{task}


\begin{task}{Abastecendo a caixa}
\label{\detokenize{AF107-6:atividade-abastecendo-a-caixa}}\label{\detokenize{AF107-6:id1}}

Uma caixa de água é abastecida por uma torneira cujo fluxo de água é constante e igual a \(10\) litros por minuto e, simultaneamente, seu conteúdo escoa, por um ralo, cujo fluxo de água é controlado à razão constante de \(15\) litros por minuto. Em certo instante, o volume de água dentro da caixa é de \(100\) litros, estando a torneira e o ralo ambos abertos.
\begin{enumerate}
\item {} 
Sendo V(t) o volume de água na caixa após t minutos do instante citado. Exiba uma sentença matemática para V(t).

\item {} 
Complete a tabela abaixo com os valores correspondentes ao volume de água na caixa.

\begin{table}[H]
\centering
\begin{tabu} to \textwidth{|l|c|c|c|c|c|c|c|c|}
\hline
\thead
Tempo (minutos) & 0 & 1 & 2 & 3 & 4 & 5 & 10 & 20 \\
\hline
Volume (litros) & & 4 & & & & & & \\
\hline
\end{tabu}
\end{table}

\item {} 
À medida que os valores do tempo aumentam, o que ocorre com os valores correspondentes ao volume de água da caixa?

\item {} 
Quando os valores do tempo aumentam de \(t=1\) a \(t=2\), o quanto variam os valores correspondentes ao volume de água da caixa? E quando estes valores aumentam de \(t=12\) a \(t=13\)?

\item {} 
Quando os valores do tempo aumentam em uma unidade, a partir de um instante qualquer, o quanto variam os valores correspondentes ao volume de água da caixa?

\end{enumerate}
\end{task}


\begin{task}{Temperatura Controlada}
\label{\detokenize{AF107-6:atividade-temperatura-controlada}}\label{\detokenize{AF107-6:id2}}
Num laboratório, um químico conseguiu controlar a variação de temperatura de dois compostos. A variação de ambos está associada às funções afins \(f\) e \(g\), de maneira que a taxa de variação das temperaturas de cada um dos compostos seja constante. Observe o gráfico, onde o eixo das ordenadas indica a temperatura (em graus Celsius) de cada composto em função do tempo \(t\), em minutos. O gráfico da figura a seguir modela a situação:

O gráfico da função \(f\) passa pelos pontos \(A=(0,-4)\) e \(B=(4,0)\), indicando que o composto associado à \(f\) está com uma temperatura de \(-4\,^{\circ}\mathrm{C}\) no início da medição e após \(4\) minutos a temperatura atinge \(0\,^{\circ}\mathrm{C}\).

O gráfico da função \(g\) passa pelos pontos \(C=(0,2)\) e \(D=(2,0)\), indicando que o composto associado à \(g\) está com uma temperatura de \(2\,^{\circ}\mathrm{C}\) no início da medição e após \(2\) minutos a temperatura atinge \(0\,^{\circ}\mathrm{C}\).

Com base nas informações do texto responda as perguntas a seguir:
\begin{enumerate}
\item {} 
Determine as expressões das funções afins \(f\) e \(g\).

\item {} 
A temperatura do composto associado à função \(f\) estão aumentando ou diminuindo? E do composto associado à função \(g\)?

\item {} 
Em quanto tempo cada composto atinge a temperatura de

\begin{enumerate}
\item \(1\,^{\circ}\mathrm{C}\)?

\item \(-3\,^{\circ}\mathrm{C}\)?

\item \(-8\,^{\circ}\mathrm{C}\)?

\item \(10\,^{\circ}\mathrm{C}\)?
\end{enumerate}
\item {} 
Após quantos minutos os dois compostos terão a mesma temperatura? E que temperatura é essa?

\end{enumerate}
\end{task}


\begin{task}{Frio nas alturas}
\label{\detokenize{AF107-6:atividade-frio-nas-alturas}}\label{\detokenize{AF107-6:id3}}

Mesmo em pleno verão um avião, precisa lidar com temperaturas muito baixas. Quando uma aeronave opera em baixas temperaturas, com umidade presente, há a possibilidade de formação de gelo que virá a se acumular na sua estrutura ou em seu grupo moto-propulsor. O gelo se forma quando um avião voa através de uma nuvem ou de um ambiente contendo gotículas de água super-resfriadas. O principal problema causado pela formação de gelo é a modificação do fluxo de ar sobre as superfícies das asas, prejudicando assim o desempenho da nave e acarretando, eventualmente, em mais gastos de combustível. Para evitar problemas como esses as aeronaves contam com um sistema anti-gelo que diminui a formação de camadas de gelo em sua fuselagem, produzindo os chamados “rastros de condensação” como na imagem.

\begin{figure}[H]
\centering

\noindent\includegraphics[width=300bp]{frio-aviao.jpg}
\end{figure}

A temperatura na troposfera (primeira camada da atmosfera que tem aproximadamente \(40\ 000\) pés de altitude) diminui \(2^\circ C\) a cada aumento de \(1\ 000\) pés na altitude. Suponha que, em um determinado dia, a temperatura em um aeroporto seja de \(30^\circ C\), e que a água congela a \(0^\circ C\).
\begin{enumerate}
\item {} 
Qual a taxa de variação, em \(^\circ C/\text{pé}\), da temperatura da atmosfera, \(T\), em função da altitude, \(h\).

\item {} 
A função \(T(h)\) é crescente ou decrescente? Como isso se reflete na taxa de variação?

\item {} 
Determine uma expressão para \(T(h)\) e represente seu gráfico.

\item {} 
Qual a temperatura a \(37\ 200\) pés de altitude?

\item {} 
A partir de que altitude o piloto deverá acionar o sistema anti-gelo da aeronave?

\item {} 
Em outro dia, a temperatura no mesmo aeroporto era de \(25^\circ C\). Qual a altitude de acionamento do sistema anti-gelo, nesse caso?

\item {} 
Estabeleça uma maneira de calcular a altitude de acionamento do sistema anti-gelo quando a temperatura do aeroporto é igual a \(T_0\).

\end{enumerate}
\end{task}

\begin{observation}{}

Seja \(f:\mathbb{R}\to\mathbb{R}\) uma função afim, \(f(x)=ax+b\), cuja taxa de variação não é nula (isto é, o gráfico de \(f\) não é uma reta horizontal).

A reta que é o gráfico de \(f\), certamente terá interseção com o eixo das abscissas. O valor de \(x\) onde ocorre essa interseção é o zero da função afim.

Seu valor pode ser calculado de duas formas:
\begin{itemize}
\item {} 
Diretamente pela expressão de \(f\):

\end{itemize}
\begin{equation*}
\begin{split}f(k)=0 \Longleftrightarrow ak+b=0 \Longleftrightarrow ak=-b \Longleftrightarrow k=-\dfrac ba\end{split}
\end{equation*}\begin{itemize}
\item {} 
Pelo gráfico, usando semelhança de triângulos:

\end{itemize}

\begin{figure}[H]
\centering

\begin{tikzpicture}
\filldraw [pattern color=\currentcolor!80, pattern=north west lines] (-1.5,0) -- (0,0) -- (0,3);
\filldraw [pattern color=\currentcolor!80, pattern=north west lines] (0,3) -- (1,3) -- (1,5) -- cycle;
\draw [->] (-3,0) -- (3,0);
\draw [->] (0,-2) -- (0,6);

\draw [domain=-2.5:1.3, thick] plot (\x,{2*\x+3});
\fill (0,3) circle (2pt) node [above left] {$(0,b)$};
\fill (-1.5,0) circle (2pt) node [above left] {$(k,0$)};

\node [below] at (-0.75,0) {$-k$};
\node [right] at (0,1.5) {$b$};
\node [below] at (0.5,3) {1};
\node [right] at (1,4) {$a$};

\node [left] at (-.5,5) {$\displaystyle \frac{-k}{1}=\frac{b}{a}\Longrightarrow k=\frac{-b}{a}$};
\end{tikzpicture}
\end{figure}

Por exemplo, o zero da função \(f(x)=5x-21\) é o valor \(k=\dfrac{-(-21)}5\).

Já o zero da função \(g(x)=7-\dfrac x2\) é \(k=\dfrac{-7}{-\frac 12}=14\) .
\end{observation}


\begin{task}{Qual a frequência?}
\label{\detokenize{AF107-6:atividade-qual-a-frequencia}}\label{\detokenize{AF107-6:id4}}

A Agência Nacional de Telecomunicações (ANATEL) determina que as emissoras de rádio FM utilizem as frequências de 87,9 a 107,9 MHz, e que haja uma diferença de 0,2 MHz entre emissoras com frequências vizinhas.
\begin{enumerate}
\item {} 
Em uma determinada região as frequências entre a 70ª e 86ª são reservadas rádios comunitárias. Determine a frequência mínima e máxima para uma rádio comunitária.

\item {} 
Determine quantas emissoras FM podem funcionar em uma mesma região.

\item {} 
Lembre-se da frequência da rádio que você costuma ouvir e determine a posição dela na sequência das frequências do problema.

\end{enumerate}
\end{task}

\explore{domínios discretos}

\begin{task}{quadros na parede}
Caso tenha disponibilidade, sugerimos o uso da versão digital desta atividade disponível neste \href{https://teacher.desmos.com/activitybuilder/custom/5e7ba1a876309d7f9879af12}{link}

\begin{figure}[H]
\centering
\includegraphics[width=100bp]{qr_code_quadros}

\end{figure}

Você deseja pedurar três quadros que têm as mesmas dimensões na parede acima do seu sofá. A linha tracejada da figura indica a altura que você deve pedurar os ganhchos. Sem auxílio de instrumentos de medição marque as posições aproximadas dos ganchos sobre a linha tracejada de maneira que os quadros fiquem igualmente espaçados entre si.

\begin{figure}[H]
\centering
\includegraphics[width=300bp]{quadros_na_parede1}

\caption{Fonte: \href{freepik.com}{Freepik}}

\end{figure}

\begin{enumerate}
\item Marcar as posições "de olho"{}pode gerar imprecisões. Para que fique perfeito é necessário medir e distribuir uniformemente os ganchos. Com o auxílio de uma régua meça a parece acima ($1$cm=$1$m) e marque as posições para os três ganchos de pendurar quadros, precisamente. Use uma caneta de outra cor para comparar com as posições anteriores. Explique sua estratégia.

\item Agora, suponha que você tem uma outra parede representada abaixo em que deseja pendurar cinco quadros equidistantes. Onde você deve posicionar os ganchos? Use o desenho abaixo para indicar e explique a sua estratégia.

\begin{figure}[H]
\centering
\includegraphics[width=300bp]{quadros_na_parede2}


\end{figure}

\item A tabela a seguir indica duas das distâncias (até o lado esquerdo da parede) de ganchos equidistantes em uma parede. Complete os espaços em branco e explique sua estratégia.

\begin{table}[H]
\centering
\begin{tabu} to \textwidth{|c|c|}
\hline
\thead
Gancho & Distância (m) \\
\hline
1 & \\
\hline
2 & 11 \\
\hline
3 & \\
\hline
4 & \\
\hline
5 & 23 \\
\hline
\end{tabu}
\end{table}

\item Supona agora que você queira posicionar mais 15 quadros na parede do item anterior, respeitando as distâncias estabelecidas. Como saber a distância, até o lado esquerdo da parede, que se encontra o gancho de número 10? Descreva pelo menos duas estratégias diferentes.

\item Para a parede do item anterior, escreva uma expressão que forneça a distância $d$ dos ganchos até o lado esquerdo da parede em função do número $n$ do gancho.

\item Em uma outra parede a distância dos ganchos \textit{em metros} em função da sua posição é dada por $d(n)=3+2n$. A que distância está o primeiro gancho? E o sexto gancho? O que representa o número $d(13)$?

\item Júlia e Camila usaram as seguintes expressões para representar a tabela a seguir:

\begin{table}[H]
\centering
\begin{tabu} to \textwidth{l>{$}l<{$}}
\text{Júlia} & d(n)=7+3n \\
\text{Camila} & d(n)=10+3(n-1)
\end{tabu}
\end{table}

\begin{table}[H]
\centering
\begin{tabu} to \textwidth{|c|c|}
\hline
\thead
Gancho & Distância (m) \\
\hline
1 & 10 \\
\hline
2 & 13 \\
\hline
3 & 16 \\
\hline
4 & 19 \\
\hline
5 & 22 \\
\hline
6 & 25 \\ 
\hline
\end{tabu}
\end{table}

Quem das duas se expressou corretamente? Por quê? Como cada uma delas pode ter pensado para chegar nas expressões?

\end{enumerate}

\end{task}

\arrange{domínios discretos}

Neste capítulo estamos trabalhando com funções afins e, como você deve ter percebido, na atividade anterior a relação entre as variáveis disância e posição do gancho na fila também estavam relacionadas por uma função afim. O que queremos destacar nesta seção é uma outra maneira de representar a \textbf{função afim} quando seu domínio é um \textbf{conjunto discreto} - em que é possível contar ou enumerar seus elementos, dizer quem é o primeiro, o segundo, o terceiro e assim por diante. Em muitos casos e atividades anteriores, de fato, nos ativemos apenas a um conjunto finito de elementos do domínio, e para esses casos é possível traduzi-los para essa nova notação.

No caso de funções definidas em cojuntos discretos, é possível adotar como domínio padrão o conjunto (ou um subonjunto) dos números naturais $\mathbb{N}=\{1,2,3,4,...\}$. Dessa forma, podemos apenas representar o conjunto imagem como uma sucessão ou sequência de valores ordenados: 
\begin{equation*}
(a_1, a_2,a_3,a_4,...)
\end{equation*}
em que, necessariamente, o termo de índice $n$ está na $n$-ésima posição da lista.

Quando a função em questão é \textit{afim}, chamamos a sequência de valores da imagem de \textbf{progressão aritmética} ou simplesmente uma \textbf{P.A.}. Assim, em uma progressão aritmética, o número que está na $n$-ésima posição da lista pode ser expresso como uma função afim da variável $n$.
\begin{equation*}
a_n=f(n)=\alpha\cdot n+\beta
\end{equation*}
em que $\alpha$ e $\beta$ representam números reais e $\alpha\neq0$

Na atividade anterior, conforme o número de quadros foi amentando, vimos que foi necessário desenvolver uma estratégia para que eles pudessem ser distribuídos de modo que ficassem igualmente espaçados na parede. Se retornarmos ao item \textbf{\textcolor{session1}{c})} e escrevermos lado a lado os valores presentes na segunda coluna da tabela obtemos a sequência (7,11,15,19,23).

Na sequência $(7,11,15,19,23)$ percebemos que a expressão
\begin{equation*}
a_n=4n+3,
\end{equation*}
para $n=1,2,3,4,5$ fornece cada um dos termos da sequência.

A exemplo que foi observado na atividade dos quadros, a diferença entre dois termos consecutivos em uma progressão aritmética é sempre a mesma. Na sequência de vinte termos $(7,11,15,...,83)$ a diferença entre termos consecutivos é igual a 4. Dito de outra forma, cada número é obtido somando 4 ao anterior.
\begin{equation*}
7\xrightarrow{+4} 11 \xrightarrow{+4} 15 \xrightarrow{+4} 19 \xrightarrow{+4} 23 \xrightarrow{+4} \cdots \xrightarrow{+4} 83
\end{equation*}

Esse valor constante que somamos para obter o próximo termo da progressão é chamado de \textbf{razão} da P.A. e coincide com a taxa de variação da função afim que a define.

\begin{equation*}
a_{k+1}=a_k=\alpha(k+1)+\beta-(\alpha+\beta)=\alpha
\end{equation*}
\begin{equation*}
a_1 \xrightarrow{+\alpha} a_2 \xrightarrow{+\alpha} a_3 \xrightarrow{+\alpha} a_4 \xrightarrow{+\alpha}\cdots   
\end{equation*}

\begin{align*}
a_2=&a_1+\alpha\\
a_3=&a_2+\alpha=a_1+2\alpha\\
a_4=&a_3+\alpha=a_2+2\alpha=a_1+3\alpha\\
&\vdots\\
a_n=&a_{n-1}+\alpha=\cdots=a_1+(n-1)\alpha
\end{align*}

Observe então que, para determinar todos os termos de uma progressão aritmética, basta conhecer
\begin{enumerate}
\item dois termos e suas posições, ou
\item um termo em posição e a razão.
\end{enumerate}
Veja os exemplos:

\begin{example}{}
Determinar os elementos de uma P.A. de 10 termos sabendo que $a_2=0{,}9$ e que $a_7=1{,}9$.

\textit{Solução}: Neste caso, para ir do segundo para o sétimo termo, temos que somar 5 vezes a razção. Assim,
\begin{equation*}
1{,}9=0{,}9+5\alpha\Rightarrow\alpha=0{,}2
\end{equation*}

A progressão é $(0{,}7;0{,}9;1{,}1;1{,}3;1{,}5;1{,}7;1{,}9;2{,}1;2{,}3;2{,}5)$
\end{example}
\begin{example}{}
Qual é o trigésimo quarto termo de uma progressão aritmética de razão $\displaystyle\frac{-\pi}{2}$ e cujo termo de posição $12$ é igual a $0$?

\textit{Solução}: Do $a_12$ até o $a_34$ somamos $22$ vezes a razão. Logo,
\begin{equation*}
a_{34}=a_12+22\cdot(\frac{-\pi}{2})=0+22\cdot(\frac{-\pi}{2})=-11\pi.
\end{equation*}
\end{example}

\begin{observation}{}
De maneira análoga à função afim, se uma progressão aritmética tem razão positiva ela será \textbf{crescente}, e se a razão for um número negativo, ela será \textbf{decrescente}.
\end{observation}

\practice{domínios discretos}

\begin{task}{termo geral de uma P.A.}
A função afim que relaciona um termo genérico de uma progressão aritmética com o primeiro termo e a razão é comumente chamada de \textbf{fórmula do termo geral} da progressão. Ou seja, para a P.A. $(a_1,a_2,a_3,...)$ de razão $r$ a fórmula do termo geral é
\begin{equation*}
a_n=a_1+(n-1)r
\end{equation*}

Complete a tabela abaixo com as progressões ou as fórmulas dos termos gerais.

\begin{table}[H]
\setlength\tabulinesep{5pt}
\centering
\begin{tabu} to \textwidth{|>{$\displaystyle}l<{$}|>{\centering $}m{2cm}<{$}|>{$}c<{$}|>{$\displaystyle}l<{$}|}
\hline
$\centering P.A.$ & $Primeiro termo$ & $Razão$ & $Termo geral$ \\
\hline
(a_1,a_2,a_3) & a_1 & r & a_n=1+2(n-1)=2n-1 \\
\hline
(1,3,5,7,9,...) & 1 & 2 & a_n=1+2(n-1)=2n-1 \\
\hline
(2,4,6,8,10,...) & & & \\
\hline
& 3 & -1 & \\
\hline
& & & a_n=10-\frac{n}{5} \\
\hline
\Big(\pi,\frac{5\pi}{4},\frac{9\pi}{4},...\Big) & & & \\
\hline
& 4 & & a_n=2+2n \\
\hline
\end{tabu}
\end{table}
\end{task}

\begin{task}{Quantos múltiplos?}

Os múltiplos de um número inteiro positivo $m$, quando representados na reta numérica ficam igualmente espaçados entre si.
\begin{enumerate}
\item Quantos múltiplos de 13 há entre 100 e 200? Explique sua estratégia.
\item Quantos múltiplos de 7 há entre 1000 e 2000?
\end{enumerate}
\end{task}

\begin{task}{Plantas companheiras}
No cultivo de produtos orgânicos, é comum o plantio de Plantas Companheiras. "Plantas Companheiras são plantas pertencentes a espécies ou famílias que se ajudam e complementam mutuamente, não apenas na ocupação do espaço e utilização de água, luz e nutrientes, mas também por meio de interações bioquímicas chamadas de Efeitos Alelopáticos. Estes podem ser tanto de natureza estimuladora quanto inibidora, não somente entre plantas, mas também em relação a insetos e outros animais." Disponível \href{https://permacoletivo.files.wordpress.com/2008/06/manual-horta-organica-domestica.pdf}{neste link} - Acesso em 25/11/2017.

Uma empresa especializada em consultoria e plantio de produtos orgânicos apresenta alguns modelos de plantio de um determinado vegetal, representado na figura a seguir por ({\Large $\bullet$}) e sua respectiva planta companheira ({\Large$\diamondsuit$}), cada modelo é adequado para o tamano do plantio e tem como objetivo criar uma barreira natural contra pragas, observe a figura a seguir que apresenta os modelos de plantio, identificados sequencialmente por Modelo 1, Modelo 2, Modelo 3, ..., Modelo $n$.


\begin{figure}[H]
\centering

\begin{tikzpicture}[node distance=10pt, every node/.style={scale=2.5}]

\begin{scope}[local bounding box=scope1]
\node (a1) {$\diamondsuit$};
\node (a2) [right of=a1] {$\diamondsuit$};
\node (a3) [above of=a2] {$\diamondsuit$};
\node (a4) [above of=a1] { $\diamondsuit$};

\node (b) [yshift=5pt] at ($(a1)!.5!(a2)$) { $\bullet$};

\node (c) at ($(a1)!.5!(a2)$) {};

\node [below of=c, scale=.5] {Modelo 1};

\end{scope}

\begin{scope}[xshift=70]
\node (a1) {$\diamondsuit$};
\node (a2) [right of=a1] {$\diamondsuit$};
\node (a3) [right of=a2] {$\diamondsuit$};

\node (a4) [above of=a1] {$\diamondsuit$};
\node (a5) [above of=a4] {$\diamondsuit$};
\node (a6) [right of=a5] {$\diamondsuit$};
\node (a7) [right of=a6] {$\diamondsuit$};
\node (a8) [above of=a3] {$\diamondsuit$};

\node (b1) [yshift=5pt] at ($(a1)!.5!(a2)$) { $\bullet$};
\node (b2) [right of=b1] {$\bullet$};
\node (b3) [above of=b1] {$\bullet$};
\node (b4) [above of=b2] {$\bullet$};

\node [below of=a2, scale=.5] {Modelo 2};
\end{scope}

\begin{scope}[xshift=170]
\node (a1) {$\diamondsuit$};
\node (a2) [right of=a1] {$\diamondsuit$};
\node (a3) [right of=a2] {$\diamondsuit$};
\node (a4) [right of=a3] {$\diamondsuit$};

\node (a5) [above of=a1] {$\diamondsuit$};
\node (a6) [above of=a5] {$\diamondsuit$};
\node (a7) [above of=a6] {$\diamondsuit$};
\node (a8) [right of=a7] {$\diamondsuit$};
\node (a9) [right of=a8] {$\diamondsuit$};
\node (a10) [right of=a9] {$\diamondsuit$};
\node (a11) [above of=a4] {$\diamondsuit$};
\node (a12) [above of=a11] {$\diamondsuit$};


\node (b1) [yshift=5pt] at ($(a1)!.5!(a2)$) { $\bullet$};
\node (b2) [right of=b1] {$\bullet$};
\node (b3) [right of=b2] {$\bullet$};

\node (b4) [above of=b1] {$\bullet$};
\node (b5) [above of=b4] {$\bullet$};
\node (b6) [right of=b5] {$\bullet$};
\node (b7) [right of=b6] {$\bullet$};
\node (b8) [above of=b3] {$\bullet$};
\node (b9) [above of=b2] {$\bullet$};

\node (c) at ($(a2)!.5!(a3)$) {};

\node [below of=c, scale=.5] {Modelo 3};
\end{scope}

\begin{scope}[xshift=300]
\node (a1) {$\diamondsuit$};
\node (a2) [right of=a1] {$\diamondsuit$};
\node (a3) [right of=a2] {$\diamondsuit$};
\node (a4) [right of=a3] {$\diamondsuit$};
\node (a5) [right of=a4] {$\diamondsuit$};

\node (a6) [above of=a1] {$\diamondsuit$};
\node (a7) [above of=a6] {$\diamondsuit$};
\node (a8) [above of=a7] {$\diamondsuit$};
\node (a9) [above of=a8] {$\diamondsuit$};

\node (a10) [right of=a9] {$\diamondsuit$};
\node (a11) [right of=a10] {$\diamondsuit$};
\node (a12) [right of=a11] {$\diamondsuit$};
\node (a13) [right of=a12] {$\diamondsuit$};

\node (a14) [above of=a5] {$\diamondsuit$};
\node (a15) [above of=a14] {$\diamondsuit$};
\node (a16) [above of=a15] {$\diamondsuit$};
\node (a17) [above of=a16] {$\diamondsuit$};


\node (b1) [yshift=5pt] at ($(a1)!.5!(a2)$) { $\bullet$};
\node (b2) [right of=b1] {$\bullet$};
\node (b3) [right of=b2] {$\bullet$};
\node (b4) [right of=b3] {$\bullet$};

\node (b5) [above of=b1] {$\bullet$};
\node (b6) [above of=b5] {$\bullet$};
\node (b7) [above of=b6] {$\bullet$};
\node (b8) [right of=b7] {$\bullet$};
\node (b9) [right of=b8] {$\bullet$};
\node (b10) [right of=b9] {$\bullet$};
\node (b11) [above of=b4] {$\bullet$};
\node (b12) [above of=b11] {$\bullet$};

\node (b13) [right of=b5] {$\bullet$};
\node (b14) [right of=b13] {$\bullet$};
\node (b15) [right of=b14] {$\bullet$};

\node (b16) [right of=b6] {$\bullet$};
\node (b17) [right of=b16] {$\bullet$};
\node (b18) [right of=b17] {$\bullet$};

\node [below of=a3, scale=.5] {Modelo 4};
\end{scope}

\end{tikzpicture}
\end{figure}

\begin{enumerate}
\item Preencha o quadro a seguir, que nos informa a quantidade de cada tipo de planta em cada um dos modelos.

\begin{table}[H]
\centering
\setlength\tabulinesep{2.5pt}
\begin{tabu} to \textwidth{|l|>{\centering}m{2cm}|>{\centering}m{2cm}|}
\hline
\thead
& ({\Large$\bullet$}) & ({\Large$\diamondsuit$}) \\
\hline
Modelo 1 & & \\
\hline
Modelo 2 & & \\
\hline
Modelo 3 & & \\
\hline
Modelo 4 & & \\
\hline
\end{tabu}
\end{table}
\item Descreva textualmente qual a relação entre a quatidade de vegetais ({\Large$\bullet$}) e o número $n$ que identifica o modelo na sequência.

\item Descreva textualmente qual a relação entre a quantidade de plantas companheiras ({\Large$\diamondsuit$}) e o número $n$ que identifica o modelo na sequência.
\item Exiba uma expressão algébrica que relacione a quantidade $V$ de vegetais ({\Large$\bullet$}) em função do número $n$ que identifica o $n$-ésimo modelo na sequência.
\item xiba uma expressão algébrica que relacione a quantidade $V$ de vegetais ({\Large$\diamondsuit$}) em função do número $n$ que identifica o $n$-ésimo modelo na sequência.
\item No plano cartesiano a seguir estão representados os pares ordenados $(n,V(n))$ em que $n$ é o "número"{} que representa o $n$-ésimo modelo e $V(n)$ a quantidade $V$ de vegetais ({\Large$\bullet$}). Represente nele os pontos que correspondem aos pares ordenados $(n,C(n0)$ em que $C(n)$ é a quantidade $C$ de plantas companheiras ({\Large$\diamondsuit$}) em função de $n$.

\begin{figure}[H]
\centering
\begin{tikzpicture}[scale=.5]
\draw [->] (-.5,0) -- (11,0);
\draw [->] (0,-.5) -- (0,17);
\draw [help lines] (0,0) grid (11,17);

\foreach \x in {1,...,10} \node [below] at (\x,0) {\x};
\foreach \x in {1,...,16} \node [left] at (0,\x) {\x};
\node [below left] {0};

\foreach \x/\y in {1/1, 2/4, 3/9, 4/16} \node [fill, circle, inner sep=2pt] at (\x,\y) {};

\end{tikzpicture}

\end{figure}
\item Qual a quantidade $C$ de plantas companheiras ({\Large$\diamondsuit$}) utilizadas no décimo modelo?
\item Qual o valor de $n$ para um modelo que utilize $144$ plantas companheiras ({\Large$\diamondsuit$})?
\end{enumerate}
\end{task}

\begin{task}{Imagem da P.A.}

Considere as funções afins $f,g$ definidas por $f(x)=3x+1$ e $g(x)=-5x+2$.
\begin{enumerate}
\item Complete a tabela abaixo com as imagens pedidas

\begin{table}[H]
\centering
\begin{tabu} to \textwidth{|>{$}c<{$}|*{9}{c|}}
\hline
\cellcolor{\currentcolor!80} x & & & & & & & & & \\
\hline
\cellcolor{\currentcolor!80} f(x) & & & & & & & & & \\
\hline
\cellcolor{\currentcolor!80} g(x) & & & & & & & & & \\
\hline
\end{tabu}
\end{table}
\item Qual a razão da P.A. da primeira linha da tabela?
\item As imagens pela função $f$ formam tabém uma P.A.? Caso positivo, qual a razão?
\item E as imagens pela função $g$? Caso positivo, qual a razão?
\item Faça uma conjectura sobre o que acontece com as imagens de uma P.A. por uma função afim.
\end{enumerate}
\end{task}

\know{ função afim por partes}
\label{\detokenize{AF107-A::doc}}\label{\detokenize{AF107-A:para-saber-mais-funcao-afim-por-partes}}
\begin{task}{Afim de um passeio}

Você caminha por \(12\) minutos a uma taxa de \(1\) km por hora,  ao encontrar um amigo permanece parado conversando por \(3\) minutos, voltando logo em seguida  a caminhar por mais \(6\) minutos a uma taxa de \(2\) km por hora.
\begin{enumerate}
\item {} 
Como você representaria no plano cartesiano, o período em que você permaneceu parado conversando com seu amigo? Considere no eixo das abscissas o tempo em minutos e no eixo das ordenadas a distância percorrida em metros.

\item {} 
Represente no plano cartesiano um gráfico que ilustra toda a situação descrita.

\item {} 
Obtenha expressões para as funções afins cujos gráficos são os segmentos de reta que você representou no item anterior.

\end{enumerate}
\end{task}

\begin{task}{Comprando vinho}

Em uma vinícola podemos comprar vinho por litro. Neste caso, o vinho é colocado em garrafões com capacidade de \(5\) litros. O vinho é vendido a R\$ \(10,00\) por litro e cada garrafão é vendido a R\$ \(5,00\).
\begin{enumerate}
\item {} 
Calcule o preço que um cliente deverá pagar por \(2\) litros, por \(5\) litros e por \(7\) litros. Explique seus cálculos.

\item {} 
Determine uma expressão para o preço \(p\) (em reais) em função do volume \(x\) (expresso em litros) de vinho adquirido. Considere \(x\) compreendido entre \(0\) e \(15\).

\item {} 
Trace a curva que representa a função \(p\) no plano cartesiano. Utilize a escala de \(1\) cm para \(1\) litro no eixo das abscissas e \(1\) cm para \(10\) reais nas ordenadas.

\end{enumerate}
\end{task}

\exercise
\label{\detokenize{AF107-E:exercicios}}\label{\detokenize{AF107-E::doc}}
\begin{enumerate}
\item \textbf{(Uerj-1998)} A promoção de uma mercadoria em um supermercado está representada, no gráfico abaixo, por \(6\) pontos de uma mesma reta.
\begin{figure}[H]
\centering

\begin{tikzpicture}[scale=.5]
Questão 1
\tikzstyle{ponto}=[circle, minimum size=5pt, inner sep=0, draw=black, fill=black, shift only, label={}]
\draw [thick, <->] (17,0) -- (0,0) -- (0,17);
     \node [above right, ] at (-1,17) {Valor da compra (R\$)};
     \node [below right] at (16,0) {Quantidade};
       \node [left] at (0,15) {150};
       \node [left] at (0,5) {50};
     \node [below] at (2.5,0) {5};
       \node [below] at (10,0) {20};
     \node [below] at (15,0) {30};
     \node [ponto,color=\currentcolor!80] at (2.5,15) {};
       \node [ponto,color=\currentcolor!80] at (15,5) {};
     \node [ponto,color=\currentcolor!80] at (10,9) {};
       \node [ponto,color=\currentcolor!80] at (12.5,7) {};
       \node [ponto,color=\currentcolor!80] at (7.5,11) {};
       \node [ponto,color=\currentcolor!80] at (5,13) {};
       \draw [dashed,color=secundario] (0,15) -- (2.5,15) -- (2.5,0);
     \draw [dashed,color=secundario] (0,9) -- (10,9) -- (10,0);
     \draw [dashed,color=secundario] (0,5) -- (15,5) -- (15,0);
\end{tikzpicture}
\end{figure}
Quem comprar \(20\) unidades dessa mercadoria, na promoção, pagará por unidade, em reais, qual valor?

\item De cada usuário de energia elétrica é cobrada uma taxa mensal de acordo com o seu consumo no período, desde que esse consumo ultrapasse determinado nível. Caso contrário, o consumidor deve pagar uma taxa mínima referente ao custo de manutenção. Em certo mês, o gráfico consumo (em kWh) x preço em (R\$) foi o apresentado abaixo:

\begin{figure}[H]
\centering


	\begin{tikzpicture}[scale=.7]
\tikzstyle{ponto}=[circle, minimum size=5pt, inner sep=0, draw=black, fill=black, shift only, label={}]
\draw [thick,<->] (0,13.25) -- (0,0) -- (12,0);
       \node [above] at (-0.5,13.25) {R\$};
       \node [below right] at (12,0) {kWh};
\node [below] at (-0.5,0.25) {0};
       \node [left] at (0,1.25) {250};
       \node [left] at (0,3.75) {750};
       \node [left] at (0,11.25) {2250};
       \node [below] at (2.5,0) {50};
       \node [below] at (5,0) {100};
       \node [below] at (10,0) {200};
       \draw [very thick,color=\currentcolor!80] (0,1.25) -- (2.5,1.25) -- (5,3.75) -- (10,11.25);
       \draw [dashed,color=secundario] (2.5,1.25) -- (2.5,0);
       \draw [dashed,color=secundario] (0,3.75) -- (5,3.75) -- (5,0);
       \draw [dashed,color=secundario] (0,11.25) -- (10,11.25) -- (10,0);
     \end{tikzpicture}
\end{figure}
     \begin{enumerate}
\item {} 
Determine entre que valores de consumo em kWh é cobrada taxa mínima.

\item {} 
Determine o consumo correspondente à taxa de R\$ \(1.950,00\).

\end{enumerate}

\item \textbf{(UFRJ-98-PNE)} - O gráfico a seguir descreve o crescimento populacional de certo vilarejo desde \(1910\) até \(1990\). No eixo das ordenadas, a população é dada em milhares de habitantes.

\begin{figure}[H]
\centering

\begin{tikzpicture}[scale=.5]
\draw [thick, <->] (22,0) -- (0,0) -- (0, 13);
       \node [below] at (21,0) {ano};
       \node [left] at (0,12) { popula\c{c}\~{a}o};
       \node [left] at (0,2) {2};
       \node [left] at (0,3) {3};
     \node [left] at (0,4) {4};
       \node [left] at (0,5) {5};
     \node [left] at (0,6) {6};
       \node [left] at (0,7) {7};
       \node [left] at (0,8) {8};
       \node [left] at (0,9) {9};
       \node [left] at (0,10) {10};
       \node [below] at (2,0) {1910};
     \node [below] at (4,0) {1920};
       \node [below] at (6,0) {1930};
       \node [below] at (8,0) {1940};
     \node [below] at (10,0) {1950};
       \node [below] at (12,0) {1960};
       \node [below] at (14,0) {1970};
       \node [below] at (16,0) {1980};
       \node [below] at (18,0) {1990};
     \draw [thick, dashed, color=secundario!60] (4,0) -- (4,2) -- (0,2);
       \draw [thick, dashed, color=secundario!60] (6,0) -- (6,3) -- (0,3);
     \draw [thick, dashed, color=secundario!60] (8,0) -- (8,4.2) -- (0,4.2);
       \draw [thick, dashed, color=secundario!60] (10,0) -- (10,6) -- (0,6);
       \draw [thick, dashed, color=secundario!60] (12,0) -- (12,7) -- (0,7);
     \draw [thick, dashed, color=secundario!60] (14,0) -- (14,8.5) -- (0,8.5);
       \draw [thick, dashed, color=secundario!60] (16,0) -- (16,10) -- (0,10);
     \draw [thick, dashed, color=secundario!60] (18,0) -- (18,11.5) ;
     \draw [thick] (0.1,2) -- (-0.1,2);
       \draw [thick] (0.1,3) -- (-0.1,3);
       \draw [thick] (0.1,4) -- (-0.1,4);
     \draw [thick] (0.1,5) -- (-0.1,5);
       \draw [thick] (0.1,6) -- (-0.1,6);
       \draw [thick] (0.1,7) -- (-0.1,7);
       \draw [thick] (0.1,8) -- (-0.1,8);
       \draw [thick] (0.1,9) -- (-0.1,9);
       \draw [thick] (0.1,10) -- (-0.1,10);
     \draw [thick] (2,0.1) -- (2,-0.1);
       \draw [thick] (4,0.1) -- (4,-0.1);
       \draw [thick] (8,0.1) -- (8,-0.1);
       \draw [thick] (10,0.1) -- (10,-0.1);
       \draw [thick] (12,0.1) -- (12,-0.1);
       \draw [thick] (14,0.1) -- (14,-0.1);
     \draw [thick] (16,0.1) -- (16,-0.1);
       \draw [thick] (18,0.1) -- (18,-0.1);
       \node [ponto, color=\currentcolor!80] at (4,2) {};
       \node [ponto, color=\currentcolor!80] at (6,3) {};
       \node [ponto, color=\currentcolor!80] at (8,4.2) {};
       \node [ponto, color=\currentcolor!80] at (10,6) {};
       \node [ponto, color=\currentcolor!80] at (12,7) {};
       \node [ponto, color=\currentcolor!80] at (14,8.5) {};
       \node [ponto, color=\currentcolor!80] at (16,10) {};
       \node [ponto, color=\currentcolor!80] at (18,11.5) {};
\draw[very thick, color=\currentcolor!80] (2,0.8) -- (4,2) -- (6,3) -- (8,4.2) -- (10,6) -- (12,7) -- (14, 8.5) -- (16,10) -- (18, 11.5);
\end{tikzpicture}

\end{figure}
\begin{enumerate}
\item {} 
Determine em que década a população atingiu a marca de \(5 000\) habitantes.

\item {} 
Observe que a partir de \(1960\) o crescimento da população em cada década tem se mantido constante. Suponha que esta taxa se mantenha inalterada no futuro. Determine em que década o vilarejo terá \(20 000\) habitantes.

\end{enumerate}

\item Suponha que um determinado veículo de transporte de passageiro tem seu valor comercial depreciado linearmente, isto é, seu valor comercial sofre desvalorização constante por ano. Veja a figura seguinte.
\begin{figure}[H]
\centering

\begin{tikzpicture}[scale=.5]
Questão 4.png
\draw [thick,<->] (0,12) -- (0,0) -- (17,0);
     \node [below left] at (0,12) {Valor (R\$)};
     \node [below] at (17,0) {Tempo (anos)};
       \node [left] at (0,-0.5) {0};
       \node [below] at (7.5,0) {20};
       \draw [very thick,color=\currentcolor!80] (0,10) -- (7.5,2) -- (15,2);
       \draw [dashed,color=secundario] (7.5,2) -- (7.5,0);
\tikzstyle{ponto}=[circle, minimum size=5pt, inner sep=0, draw=black, fill=black, shift only, label={}]
\end{tikzpicture}
\end{figure}

Esse veículo foi vendido pelo seu primeiro dono, após \(5\) anos de uso, por R\$ \(24.000,00\). Sabendo-se que o valor comercial do veículo atinge seu valor mínimo após \(20\) anos de uso, e que esse valor mínimo corresponde a \(20\%\) do valor que tinha quando era novo. Responda:
\begin{enumerate}
\item {} 
Qual o valor de fábrica do veículo (valor quando era novo)?

\item {} 
Qual o função \(f(x)=ax+b\) que está definida no gráfico acima no intervalo \([0,20]\) ?

\item {} 
Qual o valor comercial do carro quando atinge \(2\) anos de uso?

\end{enumerate}

\item \textbf{(UERJ 2016)} O resultado de um estudo para combater o desperdício de água, em certo município, propôs que as companhias de abastecimento pagassem uma taxa à agência reguladora sobre as perdas por vazamento nos seus sistemas de distribuição. No gráfico, mostra-se o valor a ser pago por uma companhia em função da perda por habitante.
\begin{figure}[H]
\centering

\begin{tikzpicture}[scale=.3]
\tikzstyle{ponto}=[circle, minimum size=5pt, inner sep=0, draw=black, fill=black, shift only, label={}]
\draw [thick, -] (22,0) -- (0,0) -- (0, 26);
     \node [below] at (15,-0.9) {perda por habitante (litros)};
       \node [above right, rotate=90] at (-1.2,10) { valor por habitante (reais)};
       \node [left] at (0,0) {0};
       \node [left] at (0,2.5) {5};
       \draw [thick] (0,2.5) -- (-0.1,2.5);
       \node [left] at (0,9.5) {20};
       \draw [thick] (0,9.5) -- (-0.1,9.5);
       \node [left] at (0,25) {V};
       \draw [thick] (0,25) -- (-0.1,25);
       \node [below] at (4,0) {100};
     \draw [thick] (4,0) -- (4,-0.1);
     \node [below] at (9,0) {200};
     \draw [thick] (9,0) -- (9,-0.1);
     \node [below] at (20,0) {500};
       \draw [thick] (20,0) -- (20,-0.1);
       \draw [thick, dashed, color=secundario] (4,0) -- (4,2.5);
       \draw [thick, dashed, color=secundario] (9,0) -- (9,9.5) -- (0,9.5);
       \draw [thick, dashed, color=secundario] (20,0) -- (20,25) -- (0,25);
     \draw [very thick, color=\currentcolor!80] (0,2.5) -- (4,2.5) -- (20,25);
     \node [ponto, color=\currentcolor!80] at (4,2.5) {};
       \node [ponto, color=\currentcolor!80] at (9,9.5) {};
       \node [ponto, color=\currentcolor!80] at (20,25) {};
\end{tikzpicture}
\end{figure}
Calcule o valor V, em reais, representado no gráfico, quando a perda for igual a \(500\) litros por habitante.

\item \textbf{(ENEM-2012)} Certo vendedor tem seu salário mensal calculado da seguinte maneira: ele ganha um valor fixo de R\$ \(750,00\), mais uma comissão de R\$ 3,00 para cada produto vendido. Caso ele venda mais de \(100\) produtos, sua comissão passa a ser de R\$ \(9,00\) para cada produto vendido, a partir do \(101º\) produto vendido. Com essas informações, o gráfico que melhor representa a relação entre salário e o número de produtos vendidos é:
\begin{figure}[H]
\centering

\begin{tikzpicture}[scale=.5]
\draw [help lines, dashed, secundario!30, xscale=2] (0,0) grid (11,11);
  \draw [thick, <->, xscale=2] (11,0) -- (0,0) -- (0, 11);
  \node [below, ] at (19,-1) {Produtos vendidos};
  \node [above, rotate=90,] at (-2, 9) {Sal\'ario em R\$};
  \node [below,] at (0,0) {0};
  \node [below,] at (2,0) {25};
  \node [below] at (4,0) {50};
  \node [below] at (6,0) {75};
\node [below] at (8,0) {100};
\node [below] at (10,0) {125};
  \node [below] at (12,0) {150};
  \node [below] at (14,0) {175};
  \node [below] at (16,0) {200};
\node [below] at (18,0) {225};
  \node [below] at (20,0) {250};
  \node [left] at (0,0) {0};
\node [left] at (0,1) {250};
\node [left] at (0,2) {500};
\node [left] at (0,3) {750};
  \node [left] at (0,4) {1000};
  \node [left] at (0,5) {1250};
  \node [left] at (0,6) {1500};
\node [left] at (0,7) {1750};
  \node [left] at (0,8) {2000};
  \node [left] at (0,9) {2250};
\node [left] at (0,10) {2500};
  \draw [ultra thick, color=\currentcolor!80] (0,3) -- (16,6);
\end{tikzpicture}

\end{figure}
\begin{figure}[H]
\centering

\begin{tikzpicture}[scale=.5]
\tikzstyle{ponto}=[circle, minimum size=5pt, inner sep=0, draw=black, fill=black, shift only, label={}]
\draw [help lines, dashed, secundario!30, xscale=2] (0,0) grid (11,11);
       \draw [thick, <->, xscale=2] (11,0) -- (0,0) -- (0, 11);
  \node [below, ] at (19,-1) {Produtos vendidos};
  \node [above, rotate=90,] at (-2, 9) {Sal\'ario em R\$};
     \node [below,] at (0,0) {0};
       \node [below,] at (2,0) {25};
       \node [below] at (4,0) {50};
       \node [below] at (6,0) {75};
       \node [below] at (8,0) {100};
       \node [below] at (10,0) {125};
       \node [below] at (12,0) {150};
     \node [below] at (14,0) {175};
     \node [below] at (16,0) {200};
       \node [below] at (18,0) {225};
       \node [below] at (20,0) {250};
     \node [left] at (0,0) {0};
       \node [left] at (0,1) {250};
       \node [left] at (0,2) {500};
       \node [left] at (0,3) {750};
       \node [left] at (0,4) {1000};
     \node [left] at (0,5) {1250};
       \node [left] at (0,6) {1500};
       \node [left] at (0,7) {1750};
       \node [left] at (0,8) {2000};
       \node [left] at (0,9) {2250};
       \node [left] at (0,10) {2500};
       \draw [ultra thick, color=\currentcolor!80] (0,3) -- (6,6.5) -- (16,8);
\end{tikzpicture}

\end{figure}

\begin{figure}[H]
\centering

\begin{tikzpicture}[scale=.5]
\tikzstyle{ponto}=[circle, minimum size=5pt, inner sep=0, draw=black, fill=black, shift only, label={}]
\draw [help lines, dashed, secundario!30, xscale=2] (0,0) grid (11,11);
       \draw [thick, <->, xscale=2] (11,0) -- (0,0) -- (0, 11);
  \node [below, ] at (19,-1) {Produtos vendidos};
  \node [above, rotate=90,] at (-2, 9) {Sal\'ario em R\$};
       \node [below,] at (0,0) {0};
       \node [below,] at (2,0) {25};
       \node [below] at (4,0) {50};
       \node [below] at (6,0) {75};
       \node [below] at (8,0) {100};
       \node [below] at (10,0) {125};
       \node [below] at (12,0) {150};
       \node [below] at (14,0) {175};
     \node [below] at (16,0) {200};
       \node [below] at (18,0) {225};
       \node [below] at (20,0) {250};
     \node [left] at (0,0) {0};
       \node [left] at (0,1) {250};
       \node [left] at (0,2) {500};
       \node [left] at (0,3) {750};
       \node [left] at (0,4) {1000};
       \node [left] at (0,5) {1250};
       \node [left] at (0,6) {1500};
       \node [left] at (0,7) {1750};
       \node [left] at (0,8) {2000};
       \node [left] at (0,9) {2250};
       \node [left] at (0,10) {2500};
       \draw [ultra thick, color=\currentcolor!80] (0,3) -- (8,3) -- (16,7);
\end{tikzpicture}

\end{figure}
\begin{figure}[H]
\centering

\begin{tikzpicture}[scale=.5]
Questão 6 item d
\tikzstyle{ponto}=[circle, minimum size=5pt, inner sep=0, draw=black, fill=black, shift only, label={}]
\draw [help lines, dashed, secundario!30, xscale=2] (0,0) grid (11,11);
       \draw [thick, <->, xscale=2] (11,0) -- (0,0) -- (0, 11);
  \node [below, ] at (19,-1) {Produtos vendidos};
  \node [above, rotate=90,] at (-2, 9) {Sal\'ario em R\$};
     \node [below,] at (0,0) {0};
       \node [below,] at (2,0) {25};
       \node [below] at (4,0) {50};
       \node [below] at (6,0) {75};
       \node [below] at (8,0) {100};
       \node [below] at (10,0) {125};
       \node [below] at (12,0) {150};
     \node [below] at (14,0) {175};
     \node [below] at (16,0) {200};
       \node [below] at (18,0) {225};
     \node [below] at (20,0) {250};
     \node [left] at (0,0) {0};
       \node [left] at (0,1) {250};
     \node [left] at (0,2) {500};
       \node [left] at (0,3) {750};
       \node [left] at (0,4) {1000};
     \node [left] at (0,5) {1250};
       \node [left] at (0,6) {1500};
       \node [left] at (0,7) {1750};
       \node [left] at (0,8) {2000};
       \node [left] at (0,9) {2250};
       \node [left] at (0,10) {2500};
       \draw [ultra thick, color=\currentcolor!80] (0,3) -- (16,3);
\end{tikzpicture}
\end{figure}
\begin{figure}[H]
\centering

\begin{tikzpicture}[scale=.5]
\tikzstyle{ponto}=[circle, minimum size=5pt, inner sep=0, draw=black, fill=black, shift only, label={}]
\draw [help lines, dashed, secundario!30, xscale=2] (0,0) grid (11,11);
       \draw [thick, <->, xscale=2] (11,0) -- (0,0) -- (0, 11);
       \node [below, ] at (19,-1) {Produtos vendidos};
       \node [above, rotate=90,] at (-2, 9) {Sal\'ario em R\$};
       \node [below,] at (0,0) {0};
       \node [below,] at (2,0) {25};
       \node [below] at (4,0) {50};
       \node [below] at (6,0) {75};
     \node [below] at (8,0) {100};
     \node [below] at (10,0) {125};
       \node [below] at (12,0) {150};
       \node [below] at (14,0) {175};
     \node [below] at (16,0) {200};
       \node [below] at (18,0) {225};
       \node [below] at (20,0) {250};
       \node [left] at (0,0) {0};
       \node [left] at (0,1) {250};
       \node [left] at (0,2) {500};
       \node [left] at (0,3) {750};
       \node [left] at (0,4) {1000};
       \node [left] at (0,5) {1250};
       \node [left] at (0,6) {1500};
       \node [left] at (0,7) {1750};
     \node [left] at (0,8) {2000};
     \node [left] at (0,9) {2250};
     \node [left] at (0,10) {2500};
     \draw [ultra thick, color=\currentcolor!80] (0,3) -- (8,4.25) -- (16,8);
\end{tikzpicture}
\end{figure}
\item \textbf{(UERJ-2014 - 2ª F)} O reservatório A perde água a uma taxa constante de 10 litros por hora, enquanto o reservatório B ganha água a uma taxa constante de 12 litros por hora. No gráfico, estão representados, no eixo y, os volumes, em litros, da água contida em cada um dos reservatórios, em função do tempo, em horas, representado no eixo x.
\begin{figure}[H]
\centering

\begin{tikzpicture}[scale=.5]
\draw [thick, <->] (14,0) -- (0,0) -- (0, 14);
     \node [below] at (13.5,0) {$x$};
     \node [left] at (0,13.5) {$y$};
     \node [left] at (0,2) {60};
     \draw [thick] (0,2) -- (-0.1,2);
     \node [left] at (0,12) {720};
       \draw [thick] (0,12) -- (-0.1,12);
       \node [below] at (5,0) {$x_{0}$};
       \draw [thick] (5,0) -- (5,-0.1);
     \node [left] at (10,11.5) {$B$};
       \node [right] at (9,2.5) {$A$};
     \draw[very thick, color= \currentcolor!80]  (0,12) -- (11,0);
     \draw[very thick, color= \currentcolor!80]  (0,2) -- (11,12);
     \draw[thick, dashed, color= secundario]  (5,0) -- (5,6.5);
\end{tikzpicture}
\end{figure}
Determine o tempo \(x_{0}\), em horas, indicado no gráfico.

\item Numa estrada existem dois telefones instalados no acostamento: um no quilometro 3 e outro no quilometro 88. Entre eles serão colocados mais 16 telefones, mantendo-se entre dois telefones consecutivos sempre a mesma distância. Determine em quais marcos quilométricos deverão ficar esses novos telefones.

\item (UNIRIO) O fichário da clínica médica de um hospital possui 10.000 clientes cadastrados, em fichas numeradas de 1 a 10.000. Um médico pesquisador, desejoso de saber a incidência de hipertensão arterial entre pessoas que procuravam o setor, fez um levantamento, analisando as fichas que tinham números múltiplos de 15. Quantas fichas NÃO foram analisadas ?
\begin{enumerate}
\item {} 
666

\item {} 
1500

\item {} 
1666

\item {} 
8334

\item {} 
9334

\end{enumerate}

\item (UERJ-2003-1ª fase) Uma seqüência de cinco átomos está organizada por ordem crescente de seus números atômicos, cujos valores são regidos por uma progressão aritmética de razão 4. Já o número de nêutrons desses mesmos átomos é regido por uma progressão aritmética de razão 5.
Se o átomo mais pesado pertence ao elemento ferro e o mais leve possui o número de prótons igual ao número de nêutrons, o número de massa do terceiro átomo da série é:
\begin{enumerate}
\item {} 
18

\item {} 
20

\item {} 
26

\item {} 
38

\end{enumerate}

\item (UERJ 2015-1º ex qualif)
\phantomsection\label{\detokenize{AF107-E:fig-charge}}
\begin{figure}[H]
\centering

\noindent\includegraphics[width=300bp]{{charge}.png}
\label{\detokenize{AF107-E:fig-charge}}\end{figure}

Na situação apresentada nos quadrinhos, as distâncias, em quilômetros, \(d(A,B)\), \(d(B,C)\) e \(d(C,D)\) formam, nesta ordem, uma progressão aritmética.
O vigésimo termo dessa progressão corresponde a:
\begin{enumerate}
\item {} 
−50

\item {} 
−40

\item {} 
−30

\item {} 
− 20

\end{enumerate}

\item (ENEM-2011) O número mensal de passagens de uma determinada empresa aérea aumentou no ano passado nas seguintes condições: em janeiro foram vendidas \(33 000\) passagens; em fevereiro, \(34 500\); em março, \(36 000\). Esse padrão de crescimento se mantém para os meses subsequentes. Quantas passagens foram vendidas por essa empresa em julho do ano passado?
\begin{enumerate}
\item {} 
\(38 000\)

\item {} 
\(40 500\)

\item {} 
\(41 000\)

\item {} 
\(42 000\)

\item {} 
\(48 000\)

\end{enumerate}

\item (UNICAMP) A ANATEL determina que as emissoras de rádio FM utilizem as frequências de \(87,9\) a \(107,9\) MHz, e que haja uma diferença de \(0,2\) MHz entre emissoras com frequências vizinhas. A cada emissora, identificada por sua freqüência, é associado um canal, que é um número natural que começa em \(200\). Desta forma, à emissora cuja frequência é de \(87,9\) MHz corresponde o canal \(200\); à seguinte, cuja frequência é de \(88,1\) MHz, corresponde o
canal \(201\), e assim por diante. Pergunta-se:
\begin{enumerate}
\item {} 
Quantas emissoras FM podem funcionar {[}na mesma região{]}, respeitando-se o intervalo de frequências permitido pela ANATEL? Qual o número do canal com maior frequência?

\item {} 
Os canais \(200\) e \(285\) são reservados para uso exclusivo das rádios comunitárias. Qual a frequência do canal \(285\), supondo que todas as frequências possíveis são utilizadas?

\end{enumerate}

\item (UFRJ-2000) Mister MM, o Mágico da Matemática, apresentou-se diante de uma platéia com \(50\) fichas, cada uma contendo um número. Ele pediu a uma espectadora que ordenasse as fichas de forma que o número de cada uma, excetuando-se a primeira e a última, fosse a média aritmética do número da anterior com o da posterior. Mister MM solicitou a seguir à espectadora que lhe informasse o valor da décima sexta e da trigésima primeira ficha, obtendo como resposta \(103\) e \(58\) respectivamente. Para delírio da platéia, Mister MM adivinhou então o valor da última ficha.

Determine você também este valor.

\end{enumerate}
% %!TEX root = ../aluno.tex

\ifnum\aluno=1
\renewcommand\chapterillustration{./abertura-funcao-quadratica}
\else
\renewcommand\chapterillustration{abertura-funcao-quadratica-professor}
\fi
\renewcommand\chapterwhat{Função quadrática com enfoque na representação algébrica ou gráfica, entendendo suas aplicações tanto em problemas de otimização quanto na utilização das propriedades da curva (parábola) nas diversas áreas do conhecimento.}
\renewcommand\chapterbecause{As funções quadráticas apresentam-se especialmente em problemas que chamamos de otimização, onde o objetivo é determinar em que condições uma grandeza assume valores máximo ou mínimo, como por exemplo, o lucro máximo de uma empresa, área máxima de uma região plana, o preço mínimo de um determinado produto sujeito a condições específicas e assim por diante. Além disso, servem de modelo para os estudos físicos do movimento onde há aceleração constante, e possui aplicação em diversas áreas como: engenharia, economia, administração, ciência da computação etc.}
\chapter{Função Quadrática}
\label{\detokenize{AF209::doc}}\label{\detokenize{AF209:funcao-quadratica}}

\ifdefined\funcoeschap
\else
\label{chap-funcoes}
\label{numeros-triangulares-funcoes}
\fi
\def\estchapdois{}

\mbox{}\thispagestyle{empty}\clearpage

\thispagestyle{empty}

\begin{center}
Projeto: LIVRO ABERTO DE MATEMÁTICA

\noindent \begin{tabular}{lcccr}
\includegraphics[scale=.15]{impa}& \quad\quad& \includegraphics[width=3cm]{logo} & \quad\quad& \includegraphics[scale=.24]{obmep} 
\end{tabular}
\end{center}

\vspace*{.3cm}

Cadastre-se como colaborador no site do projeto: \url{umlivroaberto.org}

Versão digital do capítulo:

\url{https://www.umlivroaberto.org/BookCloud/Volume_1/master/view/AF209.html}


\begin{tabular}{p{.15\textwidth}p{.7\textwidth}}
Título: & Função Quadrática\\
\\
Ano/ Versão: & 2020 / versão 1.1 de \today\\
\\
Editora & Instituto Nacional de Matem\'atica Pura e Aplicada (IMPA-OS)\\
\\
Realização:& Olimp\'iada Brasileira de Matem\'atica das Escolas P\'ublicas (OBMEP)\\
\\
Produção:& Associação Livro Aberto\\
\\
Coordenação:& Fabio Simas, \\
			& Augusto Teixeira (livroaberto@impa.br)\\
\\
  Autores: & Luiz Amorim (Colégio Pedro II),\\
           & Bruno Vianna (Colégio Pedro II).\\

\\
Revisão &  Cydara Ripoll,  \\
        &  Letícia Rangel \\
\\
Design: & Andreza Moreira (Tangentes Design) \\
\\
  Ilustrações: & --- \\ 
\\
Gráficos: & Beatriz Cabral e Tarso Caldas (Licenciandos da UNIRIO)\\
\\
  Capa: & Foto de Ashkan Forouzani, no Unsplash\\
  		& https://unsplash.com/photos/J4idEoFc8k8 \\

\end{tabular}
\vspace{.5cm}


\begin{figure}[b]
\begin{minipage}[l]{5cm}
\centering

{\large Licença:}

  \includegraphics[width=3.5cm]{cc-by-sa1}
\end{minipage}\hfill
\begin{minipage}[c]{5cm}
\centering
{\large Desenvolvido por}

\includegraphics[width=2.5cm]{logo-associacao.jpg}
\end{minipage}
\begin{minipage}[r]{5cm}
\centering

{\large Patrocínio:}
  \vspace{1em}
  \includegraphics[width=3.5cm]{itau}
\end{minipage}
\end{figure}

\mainmatter

\begin{apresentacao}{Introdução}
Neste capítulo contemplam-se as seguintes habilidades da segunda versão da Base Nacional Comum Curricular (BNCC):

\begin{habilities}{EM12MT09}
Reconhecer função quadrática e suas representações algébrica e gráfica, compreendendo o modelo de variação determinando domínio, imagem, máximo e mínimo, e utilizar essas noções e representações para resolver problemas como os de movimento uniformemente variado.
\end{habilities}

\paragraph{Pré-requisitos}
\begin{habilities}{EF08MT15}
Resolver e elaborar problemas que possam ser representados por equações polinomiais de 2\super{o} grau do tipo $ax^2=b$.

\tcbsubtitle{EF09MT18}
Compreender os processos de fatoração de expressões algébricas, a partir de suas relações com os produtos notáveis, para resolver e elaborar problemas que possam ser representados por equações polinomiais de 2\super{o} grau.
\end{habilities}

\section{Objetivos gerais}

\begin{itemize}
\item {} 
Motivar o conceito de função quadrática por meio do movimento uniformemente variado (M.U.V.).

\item {} 
Explorar de modo intuitivo, as principais propriedades da função quadrática, com ênfase no seu gráfico.

\item {} 
Explorar o conceito de otimização, por meio da forma canônica, em situações que possam ser modeladas por funções quadráticas.

\item {} 
Inferir que o gráfico de toda função quadrática é uma parábola e que pode ser obtido por translações da função real \(f\) definida por \(f(x)=ax^2\) (\(a \in \mathbb{R}\)), definindo formalmente suas formas polinomiais e canônicas.

\item {} 
Apresentar situações modeladas por funções quadráticas de domínio discreto.

\item {} 
Determinar os zeros de uma função quadrática bem como seus intervalos de crescimento e decrescimento, tanto por meio de sua expressão algébrica como de sua representação gráfica.

\item {} 
Reconhecer o eixo de simetria do gráfico de uma função quadrática.

\item {} 
Determinar a lei de formação de uma função quadrática apresentando seu gráfico.

\end{itemize}

Prezado colega, neste capítulo buscamos contemplar os conceitos, propriedades, definições e aplicações relacionadas ao estudo das funções quadráticas de modo gradativo e por meio de atividades que guiarão os alunos para os objetivos deste capítulo, mencionados anteriormente. Optamos, por influência da habilidade do BNCC, introduzir os conceitos mais básicos por uma atividade que estuda o movimento de queda livre de um objeto. De posse desses conceitos básicos, partimos para a segunda atividade que tem como objetivo geral de familiarizar os alunos com o gráfico de uma função real \(f\) definida por \(f(x)=x^2\), chamando a atenção para suas caracteríticas e propriedades.

Nas atividades seguintes exploramos problemas de otimização. Optamos por abordar esse assunto utilizando a forma canônica da função quadrática, pois é fato que a mesma já exibe claramente as coordenadas do vértice da parábola, facilitando assim a descoberta desse valor (máximo ou mínimo). Além disso, a forma canônica permite, de modo simples, apresentar ao aluno que todos os gráficos das funções quadráticas podem ser obtidos por translações do gráfico da função real \(f\) definida por \(f(x)=ax^2\) apresentado nas atividades iniciais do capítulo.

Para concluir o capítulo, aplicamos os conhecimentos adquiridos em problemas nos quais a modelagem faz uso de uma aproximação por parábolas e, nesses casos, o estudante precisa determinar a lei de formação da função quadrática por meio de informações gráficas inicialmente apresentadas.

Por fim, na seção “Você sabia?”, abordamos dois assuntos de extrema importância: A utilização prática da parábola que vem como consequência da sua propriedade refletora, e desfazemos alguns equívocos frequentes que ocorrem ao se admitir que algumas curvas ou situações podem ser modeladas por funções quadráticas.

\subsection{Dificuldades típicas dos alunos (distratores)}

\begin{itemize}

\item {} 
Os alunos conhecem a denominação correta do gráfico apresentado pela função quadrática, porém, não conseguem distingui-lo de outros gráficos curvilíneos. \citep{alexandre2009}

Distrator trabalhado na atividade \hyperref[\detokenize{AF209-2:ativ-funcao-quadratica-investigando-x-a-2}]{\textit{Em busca de padrões em \(f(x)=x^2\)}} e em \hyperref[\detokenize{AF209-11:sub-funcao-quadratica-voce-sabia-catenaria}]{Será que é parábola?}.

\item {} 
Os alunos sabem, conceitualmente, a relação existente entre os eixos das abscissas e ordenadas na função quadrática, mas não possuem habilidades de diferenciá-los durante o processo da resolução de uma questão contextualizada envolvendo função quadrática. \citep{alexandre2009}

Distrator trabalhado em: \hyperref[\detokenize{AF209-0:ativ-funcao-quadratica-lancamento-vertical-em-dubai}]{\textit{Lançando objetos das nuvens em Dubai}}, \hyperref[\detokenize{AF209-3:sub-ativ-funcao-quadratica-perimetro-fixo}]{\textit{Perímetro fixo}}, \hyperref[\detokenize{AF209-7:ativ-funcao-quadratica-aumento-passagem}]{\textit{Aumento da passagem}}, \hyperref[\detokenize{AF209-9:sec-funcao-quadratica-obtendo-lei-do-grafico}]{Explorando: determinando a função quadrática através do gráfico}.

\item {} 
Os alunos compreendem a qual eixo está relacionado, genericamente, o domínio e a imagem, porém não conseguem particularizá-lo à função quadrática. \citep{alexandre2009}

Distrator trabalhado em: \hyperref[\detokenize{AF209-0:ativ-funcao-quadratica-lancamento-vertical-em-dubai}]{\textit{Lançando objetos das nuvens em Dubai}}, \hyperref[\detokenize{AF209-3:sub-ativ-funcao-quadratica-perimetro-fixo}]{\textit{Perímetro fixo}}, \hyperref[\detokenize{AF209-7:ativ-funcao-quadratica-aumento-passagem}]{\textit{Aumento da passagem}}, \hyperref[\detokenize{AF209-9:sec-funcao-quadratica-obtendo-lei-do-grafico}]{Explorando: determinando a função quadrática através do gráfico}.

\item {} 
Há uma grande dificuldade em utilizar processos simples de fatoração para representar uma função quadrática em sua forma fatorada, consequentemente na busca dos zeros da função. \citep{parent2015}

Distrator trabalhado em: \hyperref[\detokenize{AF209-3:sub-ativ-funcao-quadratica-perimetro-fixo}]{\textit{Perímetro fixo}}, \hyperref[\detokenize{AF209-7:ativ-funcao-quadratica-aumento-passagem}]{\textit{Aumento da passagem}}, \hyperref[\detokenize{AF209-8:sec-funcao-quadratica-org-ideias-intersecoes-com-eixos}]{Organizando as ideias: interseção com os eixos coordenados}, \hyperref[\detokenize{AF209-9:sec-funcao-quadratica-obtendo-lei-do-grafico}]{Explorando: determinando a função quadrática através do gráfico}.

\item {} 
“{[}…{]}os estudantes ficam confusos quando as equações quadráticas são apresentadas de maneira não usual pois não são exatamente como estes estão acostumados a vê-las. Por o exemplo, ao apresentar \(x^2 + 3x + 1 = x + 4\) que não está em forma padrão, vários alunos apresentam dificuldades quando solicitados a realizarem várias tarefas. \citep{kotsopoulos2007}

Distrator trabalhado em: \hyperref[\detokenize{AF209-3:sec-funcao-quadratica-org-ideias-quad-max-min-na-quadratica}]{Organizando as ideias: máximos ou mínimos} , \hyperref[\detokenize{AF209-3:sub-ativ-funcao-quadratica-perimetro-fixo}]{\textit{Perímetro fixo}}, \hyperref[\detokenize{AF209-7:ativ-funcao-quadratica-aumento-passagem}]{\textit{Aumento da passagem}}.

\item {} 
Ao fazer alusão com a função afim alguns alunos acreditam equivocadamente que o coeficiente “a” da forma polinomial ou canônica representa a taxa de variação da função ou a “inclinação” de uma função quadrática. \citep{parent2015}

Distrator trabalhado em: \hyperref[\detokenize{AF209-3:sec-funcao-quadratica-org-ideias-quad-max-min-na-quadratica}]{Organizando as ideias: máximos ou mínimos}, \hyperref[\detokenize{AF209-5:sec-funcao-quadratica-parametros-grafico}]{Explorando: os parâmetros da forma canônica e o gráfico da função quadrática}.

\item {} 
Alguns alunos não associam a ideia de máximo ao \(a<0\) e ao mínimo ao \(a>0\), associam apenas ao valor numérico da expressão \(\frac{-\Delta}{4a}\), sem ao menos se preocupar se o domínio é um intervalo e se a ordenada do vértice está contida na imagem.

Distrator trabalhado em: \hyperref[\detokenize{AF209-3:sec-funcao-quadratica-org-ideias-quad-max-min-na-quadratica}]{Organizando as ideias: máximos ou mínimos}.

\item {} 
Há uma grande tendência dos alunos associarem a imagem da função quadrática ao gráfico da parábola e não a um conjunto de valores reais do eixo das ordenadas.

Distrator trabalhado em: \hyperref[\detokenize{AF209-0:ativ-funcao-quadratica-lancamento-vertical-em-dubai}]{\textit{Lançando objetos das nuvens em Dubai}}, \hyperref[\detokenize{AF209-2:sec-funcao-quadratica-org-ideias-em-x-a-2}]{Organizando as ideias: características da função real}, \hyperref[\detokenize{AF209-3:sub-ativ-funcao-quadratica-perimetro-fixo}]{\textit{Perímetro fixo}}, \hyperref[\detokenize{AF209-7:ativ-funcao-quadratica-aumento-passagem}]{\textit{Aumento da passagem}}.

\end{itemize}
\end{apresentacao}

\def\currentcolor{session1}
\begin{objectives}{Lançando objetos das nuvens em Dubai}
{
\begin{itemize}
\item Reconhecer que a relação matemática entre a distância percorrida por um objeto em queda livre e o tempo de queda não pode ser modelada por uma função afim.
\item Relacionar o movimento de queda livre de um objeto a existência de uma aceleração na velocidade de queda.
\item Inferir que o tempo é uma grandeza contínua, mesmo sendo finito o número de dados coletados.
\item Reconhecer que o movimento pode ser descrito por uma curva e não por um conjunto de pontos desconectos.
\end{itemize}
}{1}{1}
\end{objectives}
\begin{sugestions}{Lançando objetos das nuvens em Dubai}
{
\begin{itemize}
\item Sugerimos resolver a atividade anteriormente para definir o tempo necessário de sua aplicação.
\item Orientamos que seja feito um acompanhamento por parte do professor, durante a confecção da tabela apresentada no item a, com a finalidade de ter a certeza que os estudantes estejam compreendendo o significado dos valores gerados por ela.
\item Caso seja necessário, reforce as principais caracteríticas da função afim, como por exemplo: a sua taxa de variação constante.
\item No item \titem{d}, recomendamos que o professor chame a atenção dos estudantes para o fato de que, o gráfico seja apenas um conjunto de sete pontos, partindo da origem, e não uma curva contínua.
\item Para o item e, orientamos que o professor enfatize aos alunos que o registro fotográfico foi feito em intervalos de $1$ s, mas que o fenômeno continuou mesmo sem os registros.
\end{itemize}
}{1}{1}
\end{sugestions}
\clearmargin
\marginpar{\vspace{.5em}}
\begin{answer}{Lançando objetos das nuvens em Dubai}
{
\begin{enumerate}
\item $d_0=0\text{ m}; d_1=5\text{ m}; d_2=20\text{ m}; d_3=45\text{ m}; d_4=80\text{ m}; d_5=125\text{ m}; d_6=180\text{ m}$.
\item Não. Para verificar, basta calcular a razão entre a variação das distâncias em dois intervalos distintos de um segundo, por exemplo: $\frac{5−0}{1−0}=5\neq\frac{20−5}{2−1}=15$, pois a função afim é caracterizada por uma variação constante.
\item Não, pois a taxa de variação não é constante.
\end{enumerate}
}{1}
\end{answer}
\clearmargin
\begin{answer}{Lançando objetos das nuvens em Dubai}
{
\begin{enumerate}\setcounter{enumi}{3}
\item \adjustbox{valign=t}
{
\begin{tikzpicture}[yscale=.8, xscale=1.2, scale=.75]
\tikzstyle{ponto}=[circle, minimum size=5pt, inner sep=0, draw=black, fill=black, shift only, label={}]
\draw [help lines, secundario!10, step=0.2] (0,0) grid (7,11);
\draw [help lines, secundario!40] (0,0) grid (7,11);
\draw [very thick, <->] (7.1,0) -- (0,0) -- (0, 11.1);
\node [below ] at (6.3,-0.5) {Tempo (s)};
\node [left] at (-0.5, 10.7) {Dist\^ancia (m)};
\node [below] at (0,0) {0};
\node [below] at (1,0) {1};
\node [below] at (2,0) {2};
\node [below] at (3,0) {3};
\node [below] at (4,0) {4};
\node [below] at (5,0) {5};
\node [below] at (6,0) {6};
\node [left] at (0,1) {20};
\node [left] at (0,2) {40};
\node [left] at (0,3) {60};
\node [left] at (0,4) {80};
\node [left] at (0,5) {100};
\node [left] at (0,6) {120};
\node [left] at (0,7) {140};
\node [left] at (0,8) {160};
\node [left] at (0,9) {180};
\node [left] at (0,10) {200};
\node [ponto, color=primario] at (1,0.25) {};
\node [ponto, color=primario] at (2,1) {};
\node [ponto, color=primario] at (3,2.25) {};
\node [ponto, color=primario] at (4,4) {};
\node [ponto, color=primario] at (5,6.25) {};
\node [ponto, color=primario] at (6,9) {};
\end{tikzpicture}
}
\item Sim, pois o tempo é contínuo.
\item Curva
\item $d(t)=5t^2$
\end{enumerate}
}{1}
\end{answer}
\clearmargin
\begin{objectives}{Distância segura entre os carros}
{
\begin{itemize}
\item Relacionar a frenagem com a existência da desaceleração.
\item Registrar que mesmo o texto indicando uma proporcionalidade, que a relação entre as grandezas discutidas na atividade não é uma função afim.
\item Reforçar a ideia de que a função afim não modela a variação do deslocamento para movimentos acelerados.
\item Expressar matematicamente uma informação dada na forma de texto.
\item Perceber que a desaceleração é mais intensa no seco do que no molhado, desenvolvendo as noções intuitivas necessárias à compreensão dos movimentos uniformemente variados.
\end{itemize}
}{1}{2}
\end{objectives}
\begin{answer}{Distância segura entre os carros}
{
\begin{enumerate}
\item $80\div3{,}6=2009\approx22$. Assim, o carro se desloca aproximadamente $22$ m nesse segundo.
\item $57−22=35$ m.
\item $D=kv2$
\item $k=\frac{D}{v^2}\iff k=\frac{35}{80^2}\iff k=\frac{7}{1280}\implies k\approx0{,}0055$.
\item Não.
\item $a=25\text{ m}; b\approx45 \text{ m}; c=70 \text{ m}; f\approx28 \text{ m}; g=55 \text{ m}; h=83 \text{ m}.$ Os valores a serem preenchidos na faixa azul de pista molhada exigem uma outra relação de $D$ e $v$: $\frac{D}{v^2}=\frac{71}{80^2}\iff\frac{D}{v^2}=\frac{71}{6400}\implies D=0{,}01\cdot v^2$, aproximadamente. Assim, $d=81 \text{ m}; e=106 \text{ m}; i=100 \text{ m}; j=128 \text{ m}$.
\end{enumerate}
}{1}
\end{answer}

\explore{Movimentos com Velocidade Variável}
\label{\detokenize{AF209-0:sec-funcao-quadratica-movimento-com-velocidade-variavel-queda-vertical}}\label{\detokenize{AF209-0::doc}}\label{\detokenize{AF209-0:explorando-movimentos-com-velocidade-variavel}}\phantomsection\label{\detokenize{AF209-0:ativ-funcao-quadratica-lancamento-vertical-em-dubai}}

Vamos agora conhecer um {}novo tipo de função real: as \textbf{funções quadráticas}. Também conhecidas como funções polinomiais do segundo grau, ela aparece em diversas situações do cotidiano, especialmente em problemas que chamamos de otimização, onde o objetivo é determinar em que condições uma grandeza assume valores máximos ou mínimos, como por exemplo, o lucro máximo de uma empresa, área máxima de uma região plana, o preço mínimo de um determinado produto e assim por diante. Assim como nos outros capítulos do Livro Aberto, vamos apresentar conceitos, definições e propriedades por meio de atividades e aprofundar esses conhecimentos na seção “Organizando Ideias”. Esperamos que você desfrute, se aproprie e aplique esses conceitos que serão úteis em diversas áreas do conhecimento, não só nos estudos físicos do movimento, mas em áreas como da engenharia, economia, administração, ciência da computação etc.

\begin{task}{Lançando objetos das nuvens em Dubai}

No topo do hotel Burj Al Arab, em Dubai, encontra-se a quadra de tênis mais alta do mundo, com aproximadamente \(200\) metros de altura. Em 2005, os campeões Roger Federer e Andre Agassi disputaram uma partida de exibição. Considere que por um descuido, uma das bolinhas usadas nesse jogo caiu \(200\)m, verticalmente e em queda livre. Vamos aproveitar essa situação para investigar a matemática por trás desse fenômeno físico. A imagem a seguir traduz a situação no início da queda da bola.

\begin{figure}[H]
\centering
\capstart

\noindent\includegraphics[width=140bp]{{fig_1}.jpg}
\caption{Hotel e a bolinha de tênis (\textbf{credito da imagem aqui}).}\label{\detokenize{AF209-0:id125}}\end{figure}

Um observador registra com seu equipamento fotográfico a queda da bolinha, disparando fotos a cada intervalo de \(1\) segundo, até a mesma atingir o solo. Os registros fotográficos encontram-se agrupados e animados na simulação da queda, que pode ser visualizada no Geogebra: Bola de Tenis (\url{https://ggbm.at/hvnNHMY2})

A tabela a seguir descreve a altura da bolinha ao longo do tempo.

\begin{table}[H]
\centering
\begin{tabu} to \textwidth{|c|c|c|}
\hline
\thead
\(t\) & Tempo (s) & Altura (m) \\
\hline
\(t_0\) & \(0\) & \(200\) \\ 
\hline
\(t_1\) & \(1\) & \(195\) \\
\hline
\(t_2\) & \(2\) & \(180\) \\
\hline
\(t_3\) & \(3\) & \(155\) \\
\hline
\(t_4\) & \(4\) & \(120\) \\
\hline
\(t_5\) & \(5\) & \(75\) \\
\hline
\(t_6\) & \(6\) & \(20\) \\
\hline
\end{tabu}
\end{table}

\begin{enumerate}
\item {} 
Numa folha de papel ou similar, reproduza a tabela a seguir e preencha o que falta, informando a distância total percorrida pela bolinha na queda, a partir de \(t_0\).

\begin{table}[H]
\centering
\begin{tabu} to \textwidth{|c|l|}
\hline
\thead
Tempo de Queda & Distância percorrida pela bolinha \\
\hline
De \(t_0\) a \(t_0 = 0\)s & \(d_0 = 200 - 200 = 0\)m \\
\hline
De \(t_0\) a \(t_1  = 1\)s & \(d_1 = 200 - 195 = 5\)m \\
\hline
De \(t_0\) a \(t_2 = 2\)s & \(d_2 =\) \\
\hline
De \(t_0\) a \(t_3 = 3\)s & \(d_3 =\) \\
\hline
De \(t_0\) a \(t_4 = 4\)s & \(d_4 =\) \\
\hline
De \(t_0\) a \(t_5 = 5\)s & \(d_5 =\) \\
\hline
De \(t_0\) a \(t_6 = 6\)s & \(d_6 =\) \\
\hline
\end{tabu}
\end{table}


\item {} 
As distâncias percorridas pela bolinha ao longo do tempo de queda aumentam com a mesma taxa de variação?

\item {} 
É possível obter uma função afim que relaciona a distância percorrida \(d_n\) (em metros) com o tempo de queda \(t\) (em segundos)? Justifique.

\item {} 
Em uma folha de papel ou similar, copie o plano cartesiano abaixo e, em seguida, represente os pares ordenados \((t;d_n)\) em que \(t\) representa o tempo de queda em segundos e \(d_n\) a distância, em metros, percorrida pela bolinha na queda:



\begin{figure}[H]
\centering

\begin{tikzpicture}[yscale=.8, xscale=1.2]
Gráfico
\tikzstyle{ponto}=[circle, minimum size=5pt, inner sep=0, draw=black, fill=black, shift only, label={}]
\draw [help lines, secundario!10, step=0.2] (0,0) grid (7,11);
\draw [help lines, secundario!40] (0,0) grid (7,11);
\draw [very thick, <->] (7.1,0) -- (0,0) -- (0, 11.1);
\node [below ] at (6.3,-0.5) {Tempo (s)};
\node [left] at (-0.5, 10.7) {Distância (m)};
\node [below] at (0,0) {0};
\node [below] at (1,0) {1};
\node [below] at (2,0) {2};
\node [below] at (3,0) {3};
\node [below] at (4,0) {4};
\node [below] at (5,0) {5};
\node [below] at (6,0) {6};
\node [left] at (0,1) {20};
\node [left] at (0,2) {40};
\node [left] at (0,3) {60};
\node [left] at (0,4) {80};
\node [left] at (0,5) {100};
\node [left] at (0,6) {120};
\node [left] at (0,7) {140};
\node [left] at (0,8) {160};
\node [left] at (0,9) {180};
\node [left] at (0,10) {200};
\end{tikzpicture}
\end{figure}

\item {} 
O domínio da função que descreve a queda da bolinha ao longo do tempo é \(D = \{0 ; 1 ; 2 ; 3 ; 4 ; 5 ; 6 \}\). A mesma situação poderia ser descrita por uma função de domínio contínuo?

\item {} 
Neste caso, ao ligarmos todos os pontos do gráfico do item \(d\) teríamos um segmento de reta ou uma curva?

\item {} 
Dentre as alternativas a seguir, qual relação atende aos valores descritos no gráfico sendo \(d(t)\) a distância percorrida pela bolinha na queda (em metros) com o tempo de queda \(t\) (em segundos).

\(\Box \; d(t)= -t^2\)

\(\Box \; d(t)= 10t+10\)

\(\Box \; d(t)= 20t\)

\(\Box \; d(t)= 5t^2\)

\(\Box \; d(t)= 10t^2\)

\end{enumerate}
\end{task}

\phantomsection\label{\detokenize{AF209-0:ativ-funcao-quadratica-distancia-frenagem}}
\clearpage
\begin{task}{Distância segura entre os carros}

Uma noção importante sobre a direção defensiva trata do fato de que \textit{“Ao pisar no freio do veículo, ele não para instantaneamente. Entre o momento que o motorista observa um obstáculo à sua frente e decide acionar os freios até o instante que o carro realmente para, ele se desloca vários metros”} [\href{https://www.jcnet.com.br/noticias/geral/2013/02/367699-direcao-defensiva--saiba-como-a-velocidade-influi-na-frenagem-do-veiculo.html}{JCNET-2013}]. Esse fato gera a chamada \textbf{distância de frenagem}, que precisa ser conhecida, para a segurança de todo motorista.

Como essa distância depende de muitos fatores, logo que um veículo é lançado, revistas especializadas tratam de divulgar tabelas com as relações entre as velocidades e as distâncias de frenagem para estes veículos. A análise experimental e cuidadosa de qualquer uma dessas tabelas revela que a distância percorrida por um veículo após o acionamento dos freios é proporcional ao quadrado da sua velocidade [\href{http://rpm.org.br/cdrpm/12/5.htm}{Avila}].

No artigo [\href{https://www.jcnet.com.br/noticias/geral/2013/02/367699-direcao-defensiva--saiba-como-a-velocidade-influi-na-frenagem-do-veiculo.html}{JCNET-2013}] encontramos que um veículo a \(80\) Km/h, ao considerarmos os tempos de percepção, de reação e de parada, vai percorrer em média \(57\) metros em pista seca até parar totalmente, assim que o motorista observar o obstáculo e decidir frear.

\begin{enumerate}
\item {} 
Considere que o tempo de reação entre a percepção do obstáculo e a pisada no freio para um motorista seja de um segundo. Nesse tempo, quantos metros o seu carro se desloca, se inicialmente está a 80Km/h? {[}Se necessário, utilize que \(\upsilon\)Km/h = \(( \upsilon \div 3\text{,}6 )\)m/s{]}.

\item {} 
A distância de \(57\)m descrita no texto considera duas distância juntas: a que o móvel percorre no segundo anterior ao acionamento do freio, e a distância de frenagem. Sendo assim, quanto é somente a distância de frenagem desse móvel a \(80\)Km/h e que percorreu um total de \(57\)m antes de parar?

\item {} 
Sendo \(k\) uma constante de proporcionalidade, exiba uma relação algébrica entre a distância de frenagem e a velocidade do móvel antes do acionamento do freio, descrita no segundo parágrafo do texto.

\item {} 
Para os valores considerados no item \titem{b)}, qual o valor da constante de proporcionalidade \(k\)?

\item {} 
A relação algébrica obtida no item \titem{c)} é uma função afim?

Observe a figura a seguir. Ela exibe, na placa o número \(80\), referente a velocidade do carro antes de perceber o obstáculo e decidir freiar. Logo abaixo da placa há um Sol e uma nuvem de chuva. Isso é para indicar que a faixa vermelha revere-se a situação de frenagem com a pista seca, e a faixa azul a frenagem com pista molhada.

\begin{figure}[H]
\centering
\capstart

\noindent\includegraphics[width=400bp]{{frenagem1}.jpg}
\caption{Exemplo preenchido}\label{\detokenize{AF209-0:id127}}\end{figure}

\begin{figure}[H]
\centering
\capstart

\noindent\includegraphics[width=200bp]{{frenagem2}.jpg}
\caption{Significado das bandeiras nas figuras}\label{\detokenize{AF209-0:id128}}\end{figure}

\item {} 
Conforme o exemplo acima, determine todos os valores que estão faltando e que estão representados pelas letras de ‘a’ até ‘j’, observando a mudança nas placas de velocidade do carro antes de perceber o obstáculo e decidir freiar.

\end{enumerate}

\begin{figure}[H]
\centering

\noindent\includegraphics[width=400bp]{{frenagem3}.jpg}
\end{figure}
\end{task}


Na prática, para manter uma distância segura entre os carros e evitar o “engavetamento”, aconselha-se seguir a regra dos dois segundos:
\begin{itemize}
\item {} 
\textit{Observe a estrada à sua frente e escolha um ponto fixo de referência (à margem) como uma árvore, placa, poste, casa, etc.}

\item {} 
\textit{Quando o veículo que está à sua frente passar por este ponto, comece a contar pausadamente: cinqüenta e um, cinqüenta e dois. (mais ou menos dois segundos).}

\item {} 
\textit{Se o seu veículo passar pelo ponto de referência antes de contar (cinqüenta e um e cinqüenta e dois), deve aumentar a distância, diminuindo a velocidade, para ficar em segurança.}

\end{itemize}
% \begin{enumerate}
% \item {} 
% Quanto deu os resultados de cada uma dessas somas?

% \item {} 
% Com base nos itens ‘g’ e ‘h’, determine o resultado da soma de todos os números da primeira e da segunda linhas.

% \item {} 
% Lembrando do que você marcou no item ‘f’, determine o resultado obtido por \textit{Gauss}.

% \item {} 
% Você seria de capaz de refazer as etapas, porém desta vez encontrando uma expressão para o resultado da soma dos \(n\) primeiros números naturais? Ou seja, tente expressar em função de \(n\), o resultado de \(1+2+3+4+5+ \cdots +(n-3)+(n-2)+(n-1)+n\).

% \end{enumerate}

\arrange{Queda Vertical}
\label{\detokenize{AF209-1:sec-org-ideias-galileu-muv}}\label{\detokenize{AF209-1::doc}}\label{\detokenize{AF209-1:organizando-as-ideias-queda-vertical}}
\textbf{Aristóteles} (\(\star\) Estagira, \(384\) a.C. — \(\dagger\) Atenas, \(322\) a.C.) e \textbf{Galileu Galilei} \((\star 1564 , \dagger 1642)\) \textbf{no estudo da Queda Livre}

Aristóteles, discípulo de Platão e professor de Alexandre o Grande, foi um grande filósofo grego e seus estudos, nas mais diversas áreas, impactaram significativamente a história da humanidade.

Em sua época, a escola Platônica dominava a produção científica ocidental, e boa parte dela era influenciada pelo conceito de “natureza” de um objeto, onde se afirmava que todo objeto era composto da combinação dos quatro elementos: “terra, ar, água e fogo”. O estudo dos Movimentos de Aristóteles apresenta alguns equívocos que perpassaram por alguns séculos, devido a essas definições.

Aristóteles afirmava que havia dois tipos de movimentos: os naturais e os violentos. Os naturais, para ele, decorre diretamente da natureza do objeto e os violentos decorrem de  forças aplicadas a esses objetos, ou seja um movimento imposto.

No caso dos movimentos naturais, Aristóteles afirmava que cada objeto tem um lugar próprio na natureza e que esses se “esforçam” para retornar ao seu lugar de origem. Por exemplo um vaso de cerâmica, ao ser abandonado cai procurando seu lugar ao solo, já que em sua composição o elemento terra é o mais presente, e pelo mesmo motivo um sopro, ou “baforada” acaba se misturando com o ar.

Além disso, Aristóteles, afirmava que um objeto mais pesado (com maior massa) cai em direção ao solo, mais rapidamente que um objeto mais leve, ou seja, para ele, objetos ao serem lançados, caíam com rapidez proporcional ao seu peso. Suas afirmações sobre o movimento, pautaram o pensamento científico por mais de dois mil anos, sendo a base da ciência na era Medieval e Renascentista. Sendo corrigidas por seus sucessores, como Galileu Galilei.

Galileu Galilei filósofo italiano e também considerado físico, matemático e astrônomo, teve também um papel importante na revolução científica, e no contexto histórico mundial, sendo considerado um dos maiores cientistas de sua época.

Galileu abandonou a faculdade de Medicina dedicando-se aos estudos de física e matemática na Universidade de Pisa, uma das mais conceituadas da época. Foi na Universidade de Pisa, já como professor de matemática que Galileu, realizou experiências públicas sobre a queda dos corpos.

Certos historiadores, relatam que perante uma multidão de professores, estudantes e religiosos, Galileu, ao alto da torre de Pisa deixou cair dois pedaços de metal, um deles com o peso dez vezes maior que o outro e os dois chegaram ao solo praticamente no mesmo instante, contrariando assim Aristóteles. Ele prosseguiu realizado outras experiências laboratoriais que acabaram criando o conceito de resistência do ar, o que explicaria por que uma pena e uma bola de metal abandonadas de uma mesma altura, não chegam ao chão ao mesmo tempo. Mesmo assim Galileu não conseguiu convencer as autoridades universitárias de Pisa, que o acusaram de sacrilégio e acabaram tornando sua continuidade em Pisa um tanto desagradável. Por isso, no ano seguinte Galileu aceita uma cadeira na Universidade de Pádua, onde foi muito bem recebido e, por quase dezoito anos continuou realizando diversas experiências e ganhando prestígio com suas publicações e suas famosas aulas magnas.

Em 2014, a BBC publicou na internet a experiência da bola e da pena feita no vácuo: \href{https://www.youtube.com/watch?v=E43-CfukEgs}{Human Universe: Episode 4 Preview - BBC Two}, verificando na prática a veracidade das conclusões de Galileu.

Uma das conclusões mais importantes de Galileu, foram com bases em experiências laboratoriais com planos inclinados. Como retratam as figuras a seguir:
\begin{quote}

\end{quote}

\begin{figure}[H]
\centering
\capstart

\begin{tikzpicture}[every node/.style={scale=1.25}, scale=1.25]

%\draw [ ] (0,0) rectangle (5,5);
%\draw [ ] (0,0.8) -- (2.5,0.8) -- (2.5,4.2) -- (0,4.2);
%\draw [ ] (0,1.5) -- (1,1.5) -- (1,3.5) -- (0,3.5);
%\draw [ , fill=white] (-1,2.5) rectangle (4.4,2.2);
%\draw [fill=white]  (4.4,2.2) -- (5.26602,2.7) -- (5.26602,3) -- (4.4,2.5);
%\draw [fill=white] (-1,2.5) -- (-0.13397,3) -- (5.26602,3) -- (4.4,2.5) -- cycle ;

\draw  (0,3.18675) -- (0,2.88675) -- (5,0) -- (5,0.3)  ;
\draw (5,0) -- (6.73205,1) -- (6.73205,1.3) -- (6.08283,0.925);
\draw (5.649835,0.675) -- (5,0.3);
\draw(1.08283,3.81175) -- (1.73205,4.18675) -- (6.73205,1.3);
\draw (1.5,0.93782) -- (1.5,2.02072) -- (2.5,1.44336) -- (2.5,0.66987) -- cycle;
\draw (2.5,0.66987) --(3.1717,1.06) ;
\draw  (0.649835,3.56175) -- (5.649835,0.675);
\draw  (1.08283,3.81175) -- (6.08283,0.925);
\draw (6.08283,0.925) -- (6.08283,0.825) -- (5.649835,0.575) -- (5.649835,0.675);
\draw  (1.08283,3.81175) -- (1.08283,3.71175) --  (0.727,3.51);
\draw (1.08283,3.71175) -- (6.08283,0.825);
\shade[ball color = \currentcolor!40,] (0.866335,3.58675) circle (0.25cm);
\draw (0.866335,3.58675) circle (0.25cm);
\shade[ball color = \currentcolor!40,] (2.116335,2.86506) circle (0.25cm);
\draw (2.116335,2.86506) circle (0.25cm);
\shade[ball color = \currentcolor!40,] (3.366335,2.14337) circle (0.25cm);
\draw (3.366335,2.14337) circle (0.25cm);
\shade[ball color = \currentcolor!40,] (5.64982,0.825) circle (0.25cm);
\draw (5.64982,0.825) circle (0.25cm);
\draw [fill=white] (0,3.18675) -- (0.649835,3.56175) -- (5.649835,0.675) -- (5,0.3) -- cycle;
\draw [fill=white, color=white] (5.649835,0.575) -- (5.649835,0.675) --  (5,0.2) -- cycle;
\draw (5.649835,0.575) -- (5.649835,0.675) -- (5,0.3) -- (5,0);

\begin{scope}[]
\draw [densely dashed] (1.309345,3.83675) --(2.397185,4.46175) -- (3.6373425,3.74006) -- (2.5495025,3.11506);
\draw [densely dashed] (2.397185,4.46175) -- (3.283205,4.96175)  -- (5.783205,3.51837) --(3.809345,2.39337);
\draw [densely dashed] (3.283205,4.96175) -- (4.14954,5.46175) --   (8.933025,2.7) -- (6.09283,1.075);
\node [above, scale=0.8,rotate=-30] at (3.01726375,4.100905) {$\frac{t}{2}$} node [below, scale=0.8,,rotate=-30] at (3.01726375,4.100905) {$\frac{\Delta p}{4}$};
\node [above, scale=0.8, rotate=-30] at (4.533205,4.24006) {$\frac{t}{\sqrt{2}}$} node [below, scale=0.8, rotate=-30] at (4.533205,4.24006) {$\frac{\Delta p}{2}$};
\node [above, scale=0.8, rotate=-30] at (6.5412825,4.080875) {${t}$} node [below, scale=0.8, rotate=-30] at (6.5412825,4.080875) {${\Delta p}$};
\end{scope}

\end{tikzpicture}\end{figure}

\begin{table}[H]
\centering
\setlength\tabulinesep{2mm}
\begin{tabu} to \textwidth{|c|c|c|}
\hline
\thead
Medição & Espaço percorrido & Tempo Gasto \\
\hline
1ª & \(\displaystyle \frac{\Delta p}{4}\) & \(\displaystyle\frac{t}{2}\) \\
\hline
2ª & \(\displaystyle\frac{\Delta p}{2}\) & \(\displaystyle\frac{t}{\sqrt{2}}\) \\
\hline
3ª & \(\Delta p\) & \(t\) \\
\hline
\end{tabu}
\end{table}


Galileu notou que:

Na primeira medição obtemos a razão:\(\displaystyle\frac{\frac{\Delta p}{4}}{(\frac{t}{2})^2}=\frac{\Delta p}{4}.\frac{4}{t^2}=\frac{\Delta p}{t^2}\)

Na segunda medição obtemos a razão:\(\displaystyle\frac{\frac{\Delta p}{2}}{(\frac{t}{\sqrt{2}})^2}=\frac{\Delta p}{2}.\frac{2}{t^2}=\frac{\Delta p}{t^2}\)

Na terceira medição obtemos a razão: \(\displaystyle\frac{\Delta p}{t^2}\)

Após obter esses dados ele concluiu que se dividirmos, em cada caso, o espaço percorrido pelo quadrado do tempo gasto, obteremos uma razão constante. Em outras palavras, Galileu constatou que \textbf{a distância percorrida é proporcional ao quadrado do tempo gasto nesse percurso}. A partir dessas conclusões chegamos as fórmulas: \(\frac{d}{t^2} = \frac{g}{2}\) e portanto \(d(t)=\frac{gt^2}{2}\), onde \(g\) é a aceleração da gravidade; \(d(t)\) é a distância percorrida pelo objeto durante as \(t\) unidades de tempo.

\subsection{O movimento uniformemente variado}

Nos casos em que a velocidade de um móvel varia de modo constante por todo o intervalo de tempo deste movimento, o movimento é chamado de \textbf{uniformemente variado} e temos que a função que dá a velocidade em cada instante de tempo do movimento é uma função afim, logo
\begin{equation*}
\begin{split}v(t)=v_0 + a \cdot t,\end{split}
\end{equation*}
onde \(v_0\) é a velocidade no início da observação do movimento e \(a\) é o fator que vai alterando a velocidade, chamado de aceleração.

São as leis de Newton \((\star 1643, \dagger 1727)\) que estabelecem como a aceleração se relaciona com os objetos de um modo geral, permitindo o uso das relações obtidas por Galileu para situações além da queda livre dos corpos, que vão desde jogadas feitas por atletas com bolas, discos, flechas ou dardos, passando pela aceleração e frenagem de automóveis, e chegando até ao lançamento de foguetes e satélites.

Nas palavras de Pietrocola et al (2016), \textit{“Para um corpo permanecer em movimento uniformemente variado (MUV) - com aceleração constante -, é preciso que uma força atue nele. No caso dos corpos em queda, a força gravitacional Terra-corpo se encarrega dessa tarefa. Para o lançamento vertical, é necessária a ação de uma força gravitacional. Já no caso do movimento variado no plano horizontal, uma força deve ser constantemente aplicada no corpo”}.

Essas constatações conduzem ao fato de que a relação existente entre a posição de um corpo em cada instante de tempo, no caso de um \textit{MUV}, conforme já constatamos, não é uma função afim. Prova-se que as grandezas posição e tempo podem ser descrita por uma equação do tipo \(p(t)=p_0+v_0 \cdot t + \frac{a}{2} \cdot t^2\); onde \(p(t)\) é a posição no tempo \(t\), \(p_0\) é a posição inicial em relação ao referencial estabelecido na análise, \(v_0\) é a velocidade no início da observação e \(a\) é a aceleração.

\cleardoublepage
\def\currentcolor{session1}
\begin{objectives}{Em busca de padrões em \(f(x)=x^2\)}
{
\begin{itemize}[itemsep=0pt]
\item Inferir, através da análise das imagens da função $f:\R\to\R$ definida por $f(x)=x^2$, experimental e formalmente, as propriedades:

\begin{enumerate}[leftmargin=10pt]
\item de simetria axial em relação ao eixo vertical, ou seja, que $f(x)=f(−x)$, para todo $x$ real;

\item de que $f$ possuí mínimo absoluto, ou seja, que $f(x)\geq0$, para todo x real.
\end{enumerate}

\item Inferir que os pontos do gráfico de f não podem ser conectados por segmentos de reta.

\item Inferir que as variações das imagens geradas por elementos do domínio em progressão aritmética, estão também em progressão aritmética.

\item Observar que o comportamento crescente ou descrescente de $f$ não é proporcional a $x$.

\item Relacionar as constatações feitas sobre $f$ com possíveis gráficos, concluindo o que não pode ocorrer nesta representação.

\item Representar o gráfico de $f$.
\end{itemize}
}{1}{1}
\end{objectives}
\begin{sugestions}{Em busca de padrões em \(f(x)=x^2\)}
{
Esta atividade, mesmo não inserida em uma contextualização, representa uma excelente oportunidade de investigação através do experimento. Aqui o estudante terá a oportunidade de perceber, em um ambiente de pouca complexidade de conhecimento matemático, o que acontece ou não no comportamento da função quadrática. Sendo assim, recomendamos que:

\begin{itemize}[itemsep=0pt]
\item O professor faça a atividade antes de aplicá-la com os alunos, com a finalidade de conhecer o tempo de aplicação do mesmo.
\item Ao final de cada item que o professor faça uma espécie de resumo para a turma das respostas dadas pelos alunos com ênfase na característica que aquele item procura revelar.
\item Para o item \titem{c)}, será necessário o conhecimento da distância de um ponto a uma reta, que é o segmento gerado a partir do trajeto da projeção ortogonal do ponto na reta. Sem esse conhecimento a ideia de simetria não pode ser efetivada. Assim, investigue se os alunos tem essa noção antes de aplicar a atividade.
\item A escolha, em alguns itens, por valores fracionários ou irracionais, prezam pelo fortalecimento da continuidade da função mesmo que a marcação desses pontos não seja feita.
\item Recomendamos que o estudante seja estimulado a argumentar com os outros de sua turma sobre as razões que descartam cada gráfico do item \titem{i)} como candidato ao gráfico de $f$.
\item Consideramos que os casos mais difíceis de serem descartados no item \titem{i)} sejam os Gráficos 2, 5 e 6. Portanto, leia as respostas destes em particular e perceba que eles foram gerados pelas seguintes equações
\end{itemize}
}{0}{1}
\end{sugestions}
\begin{answer}{Em busca de padrões em \(f(x)=x^2\)}
{
\begin{enumerate}
\item As posições referentes ao \(-2\) e ao \(5\) deste gabarito poderiam ter sido ocupadas, respectivamente, pelo \(2\) e pelo \(-5\).

\begin{table}[H]
\centering
\resizebox{.95\linewidth}{!}
{
\begin{tabular}{|*{12}{>$e{.075\linewidth}<$|}}
\hline
\tmat{x} & -5 & -3 & -2 & -1 & 0 & 1 & 2 & 3 & 5 & \dfrac{10}{3} & \sqrt{123} \tabularnewline
\hline
\tmat{f(x)} & 25 & 9 & 4 & 1 & 0 & 1 & 4 & 9 & 25 & \dfrac{100}{0} & 123 \tabularnewline
\hline
\end{tabular}
}
\end{table}

\setcounter{enumi}{2}

\item \((-3,9)\) e \((3,9)\);

\((-2,4)\) e \((2,4)\);

\((-1,1)\) e \((1,1)\).

\end{enumerate}
}{1}
\end{answer}
\begin{answer}{Em busca de padrões em \(f(x)=x^2\)}
{
  \begin{enumerate}\setcounter{enumi}{1}
  \item \adjustbox{valign=t}
  {
  \begin{tikzpicture}[every node/.style={scale=3},scale=.75]
  
  \draw [help lines, secundario!10, step=0.2] (0,0) grid (7,11);
  \draw [help lines, secundario!40] (0,0) grid (7,11);
  \draw [<->] (7.1,0) -- (0,0) -- (0, 11.1);
  \node [below,scale=.3] at (6.3,-0.5) {Tempo (s)};
  \node [left,scale=.3] at (-0.5, 10.7) {Dist\^ancia (m)};
  \node [below,scale=.3] at (0,0) {0};
  \node [below,scale=.3] at (1,0) {1};
  \node [below,scale=.3] at (2,0) {2};
  \node [below,scale=.3] at (3,0) {3};
  \node [below,scale=.3] at (4,0) {4};
  \node [below,scale=.3] at (5,0) {5};
  \node [below,scale=.3] at (6,0) {6};
  \node [left,scale=.3] at (0,1) {20};
  \node [left,scale=.3] at (0,2) {40};
  \node [left,scale=.3] at (0,3) {60};
  \node [left,scale=.3] at (0,4) {80};
  \node [left,scale=.3] at (0,5) {100};
  \node [left,scale=.3] at (0,6) {120};
  \node [left,scale=.3] at (0,7) {140};
  \node [left,scale=.3] at (0,8) {160};
  \node [left,scale=.3] at (0,9) {180};
  \node [left,scale=.3] at (0,10) {200};
  \end{tikzpicture}
  }
  \end{enumerate}
}{0}
\end{answer}
\clearmargin
\begin{answer}{Em busca de padrões em \(f(x)=x^2\)}
{
\begin{enumerate}\setcounter{enumi}{3}
\item 
\adjustbox{valign=t}
{
\resizebox{\linewidth}{!}
{
\setlength\tabcolsep{2.5pt}
\begin{tabular}{|>$e{.15\linewidth}<$|*{7}{f|}}
\hline
\tmat{(x,y)\in f} & (7,49) & (-5,25) & \bigg(\dfrac{2}{5},\dfrac{4}{25}\bigg) & \bigg(-\dfrac{6}{7},\dfrac{36}{49}\bigg) & (\sqrt{3},3) & \bigg(\sqrt\dfrac{1}{2}, \dfrac{1}{2}\bigg) & (-\pi,\pi^2) \tabularnewline
\hline
$\tcolor{Ponto Equidistante do eixo $y$}$ & (-7,49) & (5,25) & \bigg(-\dfrac{2}{5},\dfrac{4}{25}\bigg) & \bigg(\dfrac{6}{7},\dfrac{36}{49}\bigg) & (-\sqrt{3},3) & \bigg(-\sqrt\dfrac{1}{2}, \dfrac{1}{2}\bigg) & (\pi,\pi^2) \tabularnewline
\hline 
\end{tabular}
}
}

\item \((0,0)\); Esse ponto pertence ao eixo \(y\), logo dista zero deste eixo. Outra argumentação boa é que o zero é o único número simétrico de si mesmo.

\item Não.

\item Decrescente; Crescente.

\item Não. \(\displaystyle\frac{f(5)-f(4)}{1} \neq \frac{f(4) - f(3)}{1} \neq \frac{f(3)-f(2)}{1} \neq \frac{f(2)-f(1)}{1} \neq \frac{f(1)-f(0)}{1}\).
\item \adjustbox{valign=t}
{
\resizebox{.9\linewidth}{!}
{
\begin{tabular}{|e{.15\linewidth}|e{.8\linewidth}|}
\hline
Gráfico \(1\) & As imagens dos números no intervalo \([-2,2]-{0}\) não correspondem ao que foi calculado no item a. \tabularnewline
\hline
Gráfico \(2\) & As imagens de \({-1, 1}\) estão incorretas. Perceba ainda que, por exemplo, para \(x>2\) as variações nas imagens não aparentam ter o crescimento calculado no item h. \tabularnewline
\hline
Gráfico \(3\) & Conforme visto no capítulo de função afim, esse gráfico só pode corresponder a uma função real do tipo \(f(x)=ax+b\). Outra razão é o gráfico não ser simétrico em relação ao eixo y. \tabularnewline
\hline
Gráfico \(4\) & A parte crescente não satisfazer o teorema fundamental da proporcionalidade. \tabularnewline
\hline
\end{tabular}
}
}
\end{enumerate}
}{1}
\end{answer}

\clearmargin
\begin{answer}{Em busca de padrões em \(f(x)=x^2\)}
{
\begin{enumerate}\setcounter{enumi}{7}
\item \adjustbox{valign=t}
{
\resizebox{.9\linewidth}{!}
{
\begin{tabular}{|e{.15\linewidth}|e{.8\linewidth}|}
\hline
Gráfico \(5\) & As imagens de \(-5\) e \(5\) parecem já ter aparecido para algum outro elemento do domínio no intervalo \([-5,5]\) e isso não ocorre. \tabularnewline
\hline
Gráfico \(6\) & A sessão Para saber mais do capítulo de função afim evidencia que um gráfico deste tipo, composto por vários segmentos de reta, apresenta, para intervalos diferentes do eixo \(x\), funções afins diferentes. \tabularnewline
\hline
Gráfico \(7\) & Existe nesse gráfico imagens que são negativas e isso não é possível, pois \(f(x) \geq 0\). \tabularnewline
\hline
Gráfico \(8\) & Todas as imagens se concentram de zero a oito, mas a imagem de \(f\) tem, por exemplo, os valores \(9\) e \(16\). \tabularnewline
\hline
\end{tabular}

}}

\item Resposta livre, mas as representações devem devem ficar o mais próxima possível desta:
\end{enumerate}
}{1}
\end{answer}

\explore{\texorpdfstring{A Função Real $\bm{f}$ Definida por $\bm{f(x)=x^2}$}{A Função Real \textit{f} Definida por \textit{f(x)=x²}}}
\label{\detokenize{AF209-2:sec-funcao-quadratica-propriedades-de-x-a-2}}\label{\detokenize{AF209-2::doc}}\label{\detokenize{AF209-2:explorando-a-funcao-real-definida-por}}\phantomsection\label{\detokenize{AF209-2:ativ-funcao-quadratica-investigando-x-a-2}}
\begin{task}{Em busca de padrões em \(\bm{f(x)=x^2}\)}

No capítulo anterior foi estudado o modelo matemático para funções afins. Lá, constatou-se que as funções afins são do tipo \(f(x)=ax+b\). Contudo, no \hyperref[\detokenize{AF209-0:sec-funcao-quadratica-movimento-com-velocidade-variavel-queda-vertical}]{Explorando: movimentos com velocidade variável} aparece o termo \(\alpha \cdot x^2\), com \(\alpha \in \mathbb{R}\) e \(\alpha \neq 0\). Isso revela uma situação nova em relação à função afim. A atividade que segue tem a finalidade de destacar algumas das características de funções como estas que apareceram na seção anterior. Para isso, passaremos a investigar a função real definida por \(f(x)=x^2\).

Dada a função \(f: \mathbb{R} \to \mathbb{R}\) definida por \(f(x)=x^2\), faça o que se pede:
\begin{enumerate}
\item {} 
Complete a tabela a seguir com os valores que faltam.

\begin{table}[H]
\setlength
\tabulinesep{1mm}
\centering
\begin{tabu} to \textwidth{|c|c|c|c|c|c|c|c|c|c|c|c|}
\hline
\cellcolor{\currentcolor!80}\textcolor{white}{\(\bm{x}\)} & \(-5\) & \(-3\) & & \(-1\) & & \(1\) & \(2\) & \(3\) & & \(\frac{10}{3}\) & \(\sqrt{123}\) \\
\hline
\cellcolor{\currentcolor!80}\textcolor{white}{$\bm{f(x)}$} & & & \(4\) & & \(0\) & & & & \(25\) & & \\
\hline
\end{tabu}
\end{table}


\item {} 
Em uma folha de papel ou similar, faça a figura do plano cartesiano conforme a indicada a seguir.
\begin{figure}[H]
\centering

\begin{tikzpicture}[yscale=0.8, xscale=1.2, scale=.8]
\draw [help lines, dashed, secundario!70] (0,0) grid (8,13);
\draw [->] (0,3) -- (8.1,3);
\draw [->] (4,0) -- (4,13.1);
\node [above right] at (8,3) {$x$};
\node [below right] at (4,13) {$y$};
\draw [thick] (1,2.9) -- (1,3.1);
\draw [thick] (2,2.9) -- (2,3.1);
\draw [thick] (3,2.9) -- (3,3.1);
\draw [thick] (5,2.9) -- (5,3.1);
\draw [thick] (6,2.9) -- (6,3.1);
\draw [thick] (7,2.9) -- (7,3.1);
\draw [thick] (3.9,1) -- (4.1,1);
\draw [thick] (3.9,2) -- (4.1,2);
\draw [thick] (3.9,4) -- (4.1,4);
\draw [thick] (3.9,5) -- (4.1,5);
\draw [thick] (3.9,6) -- (4.1,6);
\draw [thick] (3.9,7) -- (4.1,7);
\draw [thick] (3.9,8) -- (4.1,8);
\draw [thick] (3.9,9) -- (4.1,9);
\draw [thick] (3.9,10) -- (4.1,10);
\draw [thick] (3.9,11) -- (4.1,11);
\draw [thick] (3.9,12) -- (4.1,12);
\node [below,] at (1,3) {-3};
\node [below,] at (2,3) {-2};
\node [below,] at (3,3) {-1};
\node [below,] at (5,3) {1};
\node [below] at (6,3) {2};
\node [below] at (7,3) {3};
\node [left] at (4,1) {-2};
\node [left] at (4,2) {-1};
\node [left] at (4,4) {1};
\node [left] at (4,5) {2};
\node [left] at (4,6) {3};
\node [left] at (4,7) {4};
\node [left] at (4,8) {5};
\node [left] at (4,9) {6};
\node [left] at (4,10) {7};
\node [left] at (4,11) {8};
\node [left] at (4,12) {9};
\node [below left] at (4,3) {0};
\end{tikzpicture}
\caption{Gráfico 1}
\end{figure}
Represente os pontos da tabela do item ‘a’ nesse plano cartesiano, desprezando as coordenadas cujo valor de \(x\) não aparece destacado no que você fez no papel.

\item {} 
Destaque os pares de pontos que estão a mesma distância do eixo \(y\).

\item {} 
Caso seja possível, forneça o ponto da função \(f\) que está a mesma distância do eixo \(y\) que cada um dos pontos de \(f\) já listados a seguir. {[}Mesma distância = equidistante{]}

\begin{table}[H]
\centering
\setlength
\tabulinesep{1mm}
\setlength
\tabcolsep{2.5pt}
\begin{tabu} to \textwidth{|c|c|c|c|c|c|c|c|}
\hline
\thead
$\bm{(x,y) \in f}$ & $\bm{(7,49)}$ & $\bm{(-5,25)}$ & $\bm{\big(\frac{2}{5},\frac{4}{25}\big)}$ & $\bm{\big(-\frac{6}{7},\frac{36}{49}\big)}$ & $\bm{\big(\sqrt{3},3\big)}$ & $\bm{\big(\sqrt{\frac{1}{2}},\frac{1}{2}\big)}$ & $\bm{(- \pi , \pi^{2})}$ \\
\hline
\cellcolor{\currentcolor!80}\makecell{\textcolor{white}{\textbf{Ponto equidistante}} \\ \textcolor{white}{\textbf{do eixo} $\bm{y}$}} & & & & & & & \\
\hline
\end{tabu}
\end{table}


\item {} 
De todos os pontos que podemos obter com a função \(f\), existe um que não tem correspondente equidistante do eixo \(y\). Que ponto é esse? Tente descrever as características que esse ponto tem em relação aos outros da função \(f\) ou em relação aos eixos coordenados.

\item {} 
Existe algum ponto da imagem de \(f\) que seja menor do que zero?

\item {} 
Considerando os pontos do domínio de \(f\) entre \(-4\) e \(0\), a melhor classificação para esta função é crescente ou decrescente? E entre \(0\) e \(4\)?

\item {} 
Considerando os elementos \(\{ 0; 1; 2; 3; 4; 5 \}\) do domínio de \(f\), pode-se afirmar que a razão em que as imagens variam é a mesma para cada unidade de variação do domínio?

\item {} 
Agora serão apresentados alguns gráficos e, para cada um deles, você deve afirmar com alguma justificativa, se é ou não o gráfico de \(f\). Para isso, use o que você experimentou nos itens da atividade até aqui.

\begin{multicols}{2}
\begin{figure}[H]
\centering

\begin{tikzpicture}[scale=.35, every node/.style={scale=.75}]

  \draw [help lines, dashed, thin, color=secundario!50] (-6,-2) grid  (6,10); 
  \draw [very thick, ->] (-6,0) -- (6,0) node [above left] {$x$};
  \draw [very thick, ->] (0,-2) -- (0,10) node [below right] {$y$};
  \foreach \x in {-5,...,5}  \draw [thick] (\x,0.1) -- (\x,-0.1);
  \foreach \y in {-1,...,9}  \draw [thick] (0.1,\y) -- (-0.1,\y);
  \node [left] at (0,2) {2};
  \node [left] at (0,4) {4};
  \node [left] at (0,6) {6};
  \node [left] at (0,8) {8};
  \node [below] at (-5,0) {-5};
  \node [below] at (-4,0) {-4};
  \node [below] at (-3,0) {-3};
  \node [below] at (-2,0) {-2};
  \node [below] at (-1,0) {-1};
  \node [below] at (1,0) {1};
  \node [below] at (2,0) {2};
  \node [below] at (3,0) {3};
  \node [below] at (4,0) {4};
  \node [below] at (5,0) {5};
  \draw [very thick, domain=0:6, smooth] plot (\x,{sqrt(16*\x)});
  \draw [very thick, domain=-6:0, smooth] plot (\x,{sqrt(-16*\x)});

\end{tikzpicture}
\caption{Gráfico 1}
\end{figure}

\begin{figure}[H]
\centering

\begin{tikzpicture}[scale=.35, every node/.style={scale=.75}]

  \draw [help lines, dashed, thin, color=secundario!50] (-6,-2) grid  (6,10); 
  \draw [very thick, ->] (-6,0) -- (6,0) node [above left] {$x$};
  \draw [very thick, ->] (0,-2) -- (0,10) node [below right] {$y$};
  \foreach \x in {-5,...,5}  \draw [thick] (\x,0.1) -- (\x,-0.1);
  \foreach \y in {-1,...,9}  \draw [thick] (0.1,\y) -- (-0.1,\y);
  \node [left] at (0,2) {2};
  \node [left] at (0,4) {4};
  \node [left] at (0,6) {6};
  \node [left] at (0,8) {8};
  \node [below] at (-5,0) {-5};
  \node [below] at (-4,0) {-4};
  \node [below] at (-3,0) {-3};
  \node [below] at (-2,0) {-2};
  \node [below] at (-1,0) {-1};
  \node [below] at (1,0) {1};
  \node [below] at (2,0) {2};
  \node [below] at (3,0) {3};
  \node [below] at (4,0) {4};
  \node [below] at (5,0) {5};
  \foreach \x/\y in {0/0,1/1,2/4,3/9,-1/1,-2/4,-3/9} \fill (\x,\y) circle (5pt);

\end{tikzpicture}
\caption{Gráfico 2}
\end{figure}
\end{multicols}
\begin{multicols}{2}
\begin{figure}[H]
\centering

\begin{tikzpicture}[scale=.35, every node/.style={scale=.75}]
  \draw [help lines, dashed, thin, color=secundario!50] (-6,-6) grid  (6,6);  
  \draw [very thick, ->] (-6,0) -- (6,0) node [above left] {$x$};
  \draw [very thick, ->] (0,-6) -- (0,6) node [below right] {$y$};
  \foreach \x in {-5,...,5}  \draw [thick] (\x,0.1) -- (\x,-0.1);
  \foreach \y in {-5,...,5}  \draw [thick] (0.1,\y) -- (-0.1,\y);
  \node [left] at (0,-4) {4};
  \node [left] at (0,-2) {-2};
  \node [left] at (0,2) {2};
  \node [left] at (0,4) {4};
  \node [below] at (-5,0) {-5};
  \node [below] at (-4,0) {-4};
  \node [below] at (-3,0) {-3};
  \node [below] at (-2,0) {-2};
  \node [below] at (-1,0) {-1};
  \node [below] at (1,0) {1};
  \node [below] at (2,0) {2};
  \node [below] at (3,0) {3};
  \node [below] at (4,0) {4};
  \node [below] at (5,0) {5};

  \draw [very thick, domain=-3:3)] plot (\x,{(2)*(\x)});
  \node [ponto] at (-2,-4) {};
  \node [ponto] at (0,0) {};
  \node [ponto] at (2,4) {};



\end{tikzpicture}
\caption{Gráfico 3}
\end{figure}

\begin{figure}[H]
\centering

\begin{tikzpicture}[scale=.35, every node/.style={scale=.75}]

  \draw [help lines, dashed, thin, color=secundario!50] (-6,-2) grid  (6,10); 
  \draw [very thick, ->] (-6,0) -- (6,0) node [above left] {$x$};
  \draw [very thick, ->] (0,-2) -- (0,10) node [below right] {$y$};
  \foreach \x in {-5,...,5}  \draw [thick] (\x,0.1) -- (\x,-0.1);
  \foreach \y in {-1,...,9}  \draw [thick] (0.1,\y) -- (-0.1,\y);
  \node [left] at (0,2) {2};
  \node [left] at (0,4) {4};
  \node [left] at (0,6) {6};
  \node [left] at (0,8) {8};
  \node [below] at (-5,0) {-5};
  \node [below] at (-4,0) {-4};
  \node [below] at (-3,0) {-3};
  \node [below] at (-2,0) {-2};
  \node [below] at (-1,0) {-1};
  \node [below] at (1,0) {1};
  \node [below] at (2,0) {2};
  \node [below] at (3,0) {3};
  \node [below] at (4,0) {4};
  \node [below] at (5,0) {5};

  \draw [very thick, domain=-5:5] plot (\x,(abs(2*\x);

  \node [ponto] at (-2,4) {};
  \node [ponto] at (2,4) {};
  \node [ponto] at (0,0) {};

\end{tikzpicture}
\caption{Gráfico 4}
\end{figure}
\end{multicols}
\begin{multicols}{2}
\begin{figure}[H]
\centering

\begin{tikzpicture}[scale=.35, every node/.style={scale=.75}]

  \draw [help lines, dashed, thin, color=secundario!50] (-6,-2) grid  (6,10); 
  \draw [very thick, ->] (-6,0) -- (6,0) node [above left] {$x$};
  \draw [very thick, ->] (0,-2) -- (0,10) node [below right] {$y$};
  \foreach \x in {-5,...,5}  \draw [thick] (\x,0.1) -- (\x,-0.1);
  \foreach \y in {-1,...,9}  \draw [thick] (0.1,\y) -- (-0.1,\y);
  \node [left] at (0,2) {2};
  \node [left] at (0,4) {4};
  \node [left] at (0,6) {6};
  \node [left] at (0,8) {8};
  \node [below] at (-5,0) {-5};
  \node [below] at (-4,0) {-4};
  \node [below] at (-3,0) {-3};
  \node [below] at (-2,0) {-2};
  \node [below] at (-1,0) {-1};
  \node [below] at (1,0) {1};
  \node [below] at (2,0) {2};
  \node [below] at (3,0) {3};
  \node [below] at (4,0) {4};
  \node [below] at (5,0) {5};

  \draw [very thick, smooth] (0,0)   to [out=160, in=300] (-1,1) to [out=110, in=290] (-3,6) to [out=110, in = 310] (-4,7.9) to [out=140, in=50] (-5,7.9);
  \draw [very thick, xscale=-1, smooth] (0,0)  to [out=160, in=300] (-1,1) to [out=110, in=290] (-3,6) to [out=110, in = 310] (-4,7.9) to [out=140, in=50] (-5,7.9);

\end{tikzpicture}
\caption{Gráfico 5}
\end{figure}

\begin{figure}[H]
\centering

\begin{tikzpicture}[scale=.35, every node/.style={scale=.75}]

  \draw [help lines, dashed, thin, color=secundario!50] (-6,-2) grid  (6,10); 
  \draw [very thick, ->] (-6,0) -- (6,0) node [above left] {$x$}; 
  \draw [very thick, ->] (0,-2) -- (0,10) node [below right] {$y$};
  \foreach \x in {-5,...,5}  \draw [thick] (\x,0.1) -- (\x,-0.1);
  \foreach \y in {-1,...,9}  \draw [thick] (0.1,\y) -- (-0.1,\y);
  \node [left] at (0,2) {2};
  \node [left] at (0,4) {4};
  \node [left] at (0,6) {6};
  \node [left] at (0,8) {8};
  \node [below] at (-5,0) {-5};
  \node [below] at (-4,0) {-4};
  \node [below] at (-3,0) {-3};
  \node [below] at (-2,0) {-2};
  \node [below] at (-1,0) {-1};
  \node [below] at (1,0) {1};
  \node [below] at (2,0) {2};
  \node [below] at (3,0) {3};
  \node [below] at (4,0) {4};
  \node [below] at (5,0) {5};
  \draw [very thick,domain=-4:-1] plot (\x,{-3*\x-2}) to (0,0);
  \draw [very thick,domain=4:1] plot (\x,{3*\x-2}) to (0,0);
  \node [ponto] at (-1,1) {};
  \node [ponto] at (0,0) {};
  \node [ponto] at (1,1) {};
  \node [ponto] at (-2,4) {};
  \node [ponto] at (2,4) {};

  
\end{tikzpicture}
\caption{Gráfico 6}
\end{figure}
\end{multicols}
\begin{multicols}{2}
\begin{figure}[H]
\centering

\begin{tikzpicture}[scale=.35, every node/.style={scale=.75}]
  \draw [help lines, dashed, thin, color=secundario!50] (-6,-6) grid  (6,6);  
  \draw [very thick, ->] (-6,0) -- (6,0) node [above left] {$x$};
  \draw [very thick, ->] (0,-6) -- (0,6) node [below right] {$y$};
  \foreach \x in {-5,...,5}  \draw [thick] (\x,0.1) -- (\x,-0.1);
  \foreach \y in {-5,...,5}  \draw [thick] (0.1,\y) -- (-0.1,\y);
  \node [left] at (0,-4) {4};
  \node [left] at (0,-2) {-2};
  \node [left] at (0,2) {2};
  \node [left] at (0,4) {4};
  \node [below] at (-5,0) {-5};
  \node [below] at (-4,0) {-4};
  \node [below] at (-3,0) {-3};
  \node [below] at (-2,0) {-2};
  \node [below] at (-1,0) {-1};
  \node [below] at (1,0) {1};
  \node [below] at (2,0) {2};
  \node [below] at (3,0) {3};
  \node [below] at (4,0) {4};
  \node [below] at (5,0) {5};

  \draw [very thick, domain=-1.81712:1.81712)] plot (\x,{((\x)^3)});
  \node [ponto] at (1,1) {};
  \node [ponto] at (0,0) {};



\end{tikzpicture}
\caption{Gráfico 7}
\end{figure}

\begin{figure}[H]
\centering

\begin{tikzpicture}[scale=.35, every node/.style={scale=.75}]

  \draw [help lines, dashed, thin, color=secundario!50] (-6,-2) grid  (6,10); 
  \draw [very thick, ->] (-6,0) -- (6,0) node [above left] {$x$}; 
  \draw [very thick, ->] (0,-2) -- (0,10) node [below right] {$y$};
  \foreach \x in {-5,...,5}  \draw [thick] (\x,0.1) -- (\x,-0.1);
  \foreach \y in {-1,...,9}  \draw [thick] (0.1,\y) -- (-0.1,\y);
  \node [left] at (0,2) {2};
  \node [left] at (0,4) {4};
  \node [left] at (0,6) {6};
  \node [left] at (0,8) {8};
  \node [below] at (-5,0) {-5};
  \node [below] at (-4,0) {-4};
  \node [below] at (-3,0) {-3};
  \node [below] at (-2,0) {-2};
  \node [below] at (-1,0) {-1};
  \node [below] at (1,0) {1};
  \node [below] at (2,0) {2};
  \node [below] at (3,0) {3};
  \node [below] at (4,0) {4};
  \node [below] at (5,0) {5};
  \draw [very thick] (0,4) ellipse (1.4 and 4);
  
\end{tikzpicture}
\caption{Gráfico 8}
\end{figure}
\end{multicols}

\item No mesmo papel em que você marcou alguns dos pontos da função \(f\), lá no item \titem{b)}, construa o gráfico que você acha que representa a função \(f\) e compare com o de seus colegas. Se houver discondâncias, tentem argumentar e aprimorar os gráficos uns dos outros com base nas argumentações.

\end{enumerate}
\end{task}

\arrange{\texorpdfstring{Características da Função Real $\bm{f(x)=x^{2}}$}{Características da função real \textit{f(x)=x²}}}
\label{\detokenize{AF209-2:organizando-as-ideias-caracteristicas-da-funcao-real}}\label{\detokenize{AF209-2:sec-funcao-quadratica-org-ideias-em-x-a-2}}
Na atividade isolamos o termo \(x^{2}\) que apareceu no início deste capítulo e motivamos algumas experimentações que devem ter provocado algumas conjecturas e também conduziu a algumas certezas. Será que sua atenção recaiu nesses fatos que listamos a seguir?

\subsection{Simetria axial de \(f\)}

Os itens de \titem{b)} a \titem{d} esclarecem que, na função \(f\), valores simétricos do domínio geram imagens iguais, ou seja, \(f(-x) = f(x)\), para qualquer \(x \in \mathbb{R}\). Basta perceber que \(f(-x) = (-x)^{2} = (-x)(-x) = x^{2} = f(x)\). Isso faz com que o eixo \(y\) seja mediatriz do segmento que une esses pares de pontos do tipo \((x,x^{2})\) e \((-x,x^{2})\) que destacamos, ou para qualquer outro elemento do domínio de \(f\). A única exceção é \(x=0\) pois 0 é simétrico de si mesmo. Assim, podemos afirmar que, para o gráfico da função \(f\), o eixo \(y\) é eixo de simetria.

\begin{figure}[H]
\centering

\begin{tikzpicture}[xscale=1.2, yscale=.8]

\draw [thin,help lines, dotted, secundario!70] (-1,0) grid (7,13.5);
\draw [->] (-0.5,3) -- (7.3,3);
\draw [->] (3,1) -- (3,13.1);
\node [above] at (7,3.2) {$x$};
\node [right] at (3.2,13) {$y$};
\foreach \y in { 1,2, ...,9} \node [right] at (3,\y+3) {\y};
\foreach \x in {-3, -2, -1,...,3} \node [below] at (\x+3.2,3) {\x};
\draw [color=\currentcolor!80, thick] (3,3) parabola (0,12);
\draw [color=\currentcolor!80, thick] (3,3) parabola (6,12);
\draw[very thin,dashed, secundario!50] (0,12) -- (0,3);
\draw[very thin,dashed, secundario!50] (6,12) -- (6,3);
\draw[very thin,dashed, secundario!50] (1,7) -- (1,3);
\draw[very thin,dashed, secundario!50] (5,7) -- (5,3);
\draw[very thin,dashed, secundario!50] (2,4) -- (2,3);
\draw[very thin,dashed, secundario!50] (4,4) -- (4,3);
\draw [color=\currentcolor!80]  (2.7,11.6) rectangle (3,12);
\draw [color=\currentcolor!80]  (2.7,6.6) rectangle (3,7);
\draw [color=\currentcolor!80]  (2.7,3.6) rectangle (3,4);
\draw [color=terciario] (0,12)--(6,12);
\draw [color=terciario] (1,7)--(5,7);
\draw [color=terciario] (2,4)--(4,4);
\draw [color=terciario] (1.3, 11.8) -- (1.3, 12.2);
\draw [color=terciario] (1.5, 11.8) -- (1.5, 12.2);
\draw [color=terciario] (1.7, 11.8) -- (1.7, 12.2);
\draw [color=terciario] (4.3, 11.8) -- (4.3, 12.2);
\draw [color=terciario] (4.5, 11.8) -- (4.5, 12.2);
\draw [color=terciario] (4.7, 11.8) -- (4.7, 12.2);
\draw [color=terciario] (1.9, 6.8) -- (1.9, 7.2);
\draw [color=terciario] (2.1, 6.8) -- (2.1, 7.2);
\draw [color=terciario] (3.9, 6.8) -- (3.9, 7.2);
\draw [color=terciario] (4.1, 6.8) -- (4.1, 7.2);
\draw [color=terciario] (3.5, 3.8) -- (3.5, 4.2);
\draw [color=terciario] (2.5, 3.8) -- (2.5, 4.2);
\node [ponto, color=\currentcolor!80] at (0,12) {};
\node [ponto, color=\currentcolor!80] at (6,12) {};
\node [ponto, color=\currentcolor!80] at (3,12) {};
\node [ponto, color=\currentcolor!80] at (1,7) {};
\node [ponto, color=\currentcolor!80] at (5,7) {};
\node [ponto, color=\currentcolor!80] at (3,7) {};
\node [ponto, color=\currentcolor!80] at (2,4) {};
\node [ponto, color=\currentcolor!80] at (4,4) {};
\node [ponto, color=\currentcolor!80] at (3,4) {};
\end{tikzpicture}
\caption{Eixo de simetria do gr\'afico de $f$}
\end{figure}

\begin{description}
\item[Teorema 1]

A função real \(f\) definida por \(f(x)=x^2\) é simétrica em relação ao eixo \(y\).
\end{description}

\subsection{A imagem de \(f\)}

O item ‘e’ nos leva a refletir sobre um fato muito importante no estudo que estamos desenvolvendo aqui. Não importa qual o valor real do domínio que seja utilizado, a menor imagem é zero, pois sendo x um número real, só existem três possibilidades para x:
\begin{equation*}
\begin{split}& x<0 \Rightarrow x \cdot x = x^{2}>0 \Rightarrow f(x)>0;\\
& x=0 \Rightarrow x \cdot x = 0 \cdot 0 =0 \Rightarrow f(x)=0;\\
& x>0 \Rightarrow x \cdot x = x^{2}>0 \Rightarrow f(x)>0.\\\end{split}
\end{equation*}
\begin{description}
\item[Teorema 2]

A função real \(f\) definida por \(f(x)=x^2\) é tal que para qualquer \(x \in \mathbb{R}\), \(f(x) = x^{2} \ge 0\), ou seja, o menor valor de \(f\) é zero e \(Im(f) = [0, +\infty[\).
\end{description}

\begin{figure}[H]
\centering

\begin{tikzpicture}[xscale=1.2, yscale=.8]
\draw [->] (3,1) -- (3,13.1);
\draw [very thin,dashed, color= secundario!70] (0,12) -- (6,12);
\draw [very thin,dashed, color= secundario!70] (0,7) -- (6,7);
\draw [very thin,dashed, color= secundario!70] (0,4) -- (6,4);
\draw [very thin,dashed, color= secundario!70] (0,3) -- (6,3);
\draw [color=\currentcolor!80, thick] (3,3) parabola (0,12);
\draw [color=\currentcolor!80, thick] (3,3) parabola (6,12);
\node [ponto, color=\currentcolor!80] at (0,12) {};
\node [ponto, color=\currentcolor!80] at (6,12) {};
\node [ponto, color=\currentcolor!80] at (1,7) {};
\node [ponto, color=\currentcolor!80] at (5,7) {};
\node [ponto, color=\currentcolor!80] at (2,4) {};
\node [ponto, color=\currentcolor!80] at (4,4) {};
\node [ponto, color=\currentcolor!80] at (3,3) {};
\node [right] at (6,12)  {$f(x)=(9)$};
\node [right,] at (6,7)  {$f(x)=(4)$};
\node [right,] at (6,4.3)  {$f(x)=(1)$};
\node [right, align=center, xshift=-.25cm] at (6,2.8)  {$f(x)=(0)$ \\ valor m\'inimo};
\node [left] at (0,12)  {\phantom{$f(x)=(9)$}};
\node [left,] at (0,7)  {\phantom{$f(x)=(4)$}};
\node [left,] at (0,4.3)  {\phantom{$f(x)=(1)$}};

\node [left, align=center, xshift=-.25cm] at (0,2.8)  {\phantom{valor m\'inimo}};

\node [below,] at (7.8,2.3)  {};
\end{tikzpicture}
\caption{A não proporcionalidade no crescimento de \(f\)}
\end{figure}



Como o gráfico da função \(f\) é simétrico em relação ao eixo \(y\), a análise gráfica que se faz em uma das metades da figura fica espelhada para compor a outra metade. Assim, vamos analisar o que ocorre na parte crescente de \(f\) quando aumentamos em uma unidade um elemento \(x\) do seu domínio:
Se \(x \in ]0,+\infty[\), temos que \(f(x) = x^{2}\) e \(f(x+1)=(x+1)^{2}=x^{2}+2x+1\). Assim, \(f(x+1)-f(x)=2x+1\), ou seja, as variações das imagens dependem do \(x\) escolhido. Mais especificamente, neste caso elas formam uma progressão aritmética de razão \(2\) e, com isso, as variações analisadas são crescentes. Graficamente,

\begin{observation}{}

A função real \(f\) definida por \(f(x)=x^2\) não é função afim.
\end{observation}

\subsection{\(f\) e as progressões aritméticas}

Muito provavelmente, as características anteriores de \(f\), ou mesmo os itens da atividade, tenham transmitido alguma ideia da existência de uma progressão aritmética nessa função real. A tabela a seguir exibe elementos do domínio em progressão aritmética, suas imagens e as diferenças consecutivas dessas imagens:

\begin{table}[H]
\centering
\setlength\tabulinesep{.5mm}
\begin{tabu} to \textwidth{|l|l|l|}
\hline
\thead
$\bm{x \in f}$ & $\bm{f(x)}$ & $\bm{f(x+1)-f(x)}$ \\
\hline
\(0\) & \(0\) & \(1-0=1\) \\
\hline
\(1\) & \(1\) & \(4-1=3\) \\
\hline
\(2\) & \(4\) & \(9-4=5\) \\
\hline
\(3\) & \(9\) & \(16-9=7\) \\
\hline
\(4\) & \(16\) & \(25-16=9\) \\
\hline
\(5\) & \(25\) & \(36-25=11\) \\
\hline
\multicolumn{1}{|c|}{\(\vdots\)} & \multicolumn{1}{c|}{\(\vdots\)} & \multicolumn{1}{c|}{\(\vdots\)} \\
\hline
\end{tabu}
\end{table}


Escolhendo \(x\) do domínio de \(f\) e um \(r \in \mathbb{R}\) constante, podemos analisar a situação da tabela acima de uma forma mais geral:

\begin{table}[H]
\centering
\setlength\tabcolsep{3.5pt}
\setlength\tabulinesep{1mm}
\begin{tabu} to \textwidth{|l|l|l|}
\hline
\thead
$\bm{x\in f}$ & \multicolumn{1}{c|}{\cellcolor{\currentcolor!80}\textcolor{white}{$\bm{f(x)}$}} & \multicolumn{1}{c|}{\cellcolor{\currentcolor!80}\textcolor{white}{$\bm{f(x+r)-f(x)}$}} \\
\hline
$x$ & $x^2$ & $(x+r)^2-x^2=2xr+ 1 \cdot r^2$ \\
\hline
$x+r$ & $(x+r)^2=x^2+2xr+r^2$ & $(x+2r)^2-(x+r)^2=2xr+3r^2=(2xr+r^2)+2r^2$ \\
\hline
$x+2r$ & $(x+2r)^2=x^2+4xr+4r^2$ & $(x+3r)^2-(x+2r)^2=2xr+5r^2=(2xr+r^2)+2 \cdot 2r^2$ \\
\hline
$x+3r$ & $(x+3r)^2=x^2+6xr+9r^2$ & $(x+4r)^2-(x+3r)^2=2xr+7r^2=(2xr+r^2)+3 \cdot 2r^2$ \\
\hline
$x+4r$ & $(x+4r)^2=x^2+8xr+16r^2$ & $(x+5r)^2-(x+4r)^2=2xr+9r^2=(2xr+r^2)+4 \cdot 2r^2$ \\
\hline
$x+5r$ & $(x+5r)^2=x^2+10xr+25r^2$ & \multicolumn{1}{c|}{$\vdots$} \\
\hline
\multicolumn{1}{|c|}{$\vdots$} & \multicolumn{1}{c|}{$\vdots$} & \multicolumn{1}{c|}{$\vdots$} \\
\hline
\end{tabu}
\end{table}


E esse padrão continua, nos permitindo perceber que

\begin{observation}{}

Na função real \(f\) definida por \(f(x)=x^2\) as diferenças entre imagens consecutivas, geradas por uma parte do domínio cujos elementos estejam em progressão aritmética, formam também uma progressão aritmética com primeiro termo igual a \(2xr+r^2\) e razão \(2r^2\).
\end{observation}

\cleardoublepage
\def\currentcolor{session1}
\begin{objectives}{Perímetro Fixo}
{
Prezado colega esta atividade tem como objetivo aplicar o conceito de otimização em função quadrática num contexto geométrico, sem a utilização do gráfico da função nem muito menos da curva denominada parábola, para isso pretendemos:

\begin{itemize}
\item explorar a situação através do uso, já corriqueiro, de preenchimento de um quadro.
\item modelar a situação utilizando álgebra de maneira simples e guiada.
\item apresentar e explorar a técnica de completar quadrados para passarmos a função quadrática encontrada da forma polinomial para a forma canônica, sem obrigatoriamente citar esses termos.
\item utilizar a apresentação da forma canônica para identificarmos os valores de área máxima e os valores que maximizam essa área, convidando seu aluno à fazer inferências apenas aritméticas na forma encontrada.
\end{itemize}
}{1}{1}
\end{objectives}
\begin{answer}{Perímetro Fixo}
{
\begin{enumerate}
\item {} 
No retângulo temos as medidas de: \textbf{perímetro}, \textbf{área}, \textbf{base} e \textbf{altura}.

\item {} 
O perímetro não varia, e a área, a base e a altura variam.

\item {} 
Segue o quadro preenchido:

\begin{table}[H]
\centering
\begin{tabular}{|c|c|c|}
\hline
\tcolor{Base} & \tcolor{Altura} & \tcolor{Área} \\
\hline
$0$ & $6$ & $0$ \\
\hline
$2$ & $4$ & $8$ \\
\hline
$4$ & $2$ & $8$ \\
\hline
$6$ & $0$ & $0$\\
\hline
\end{tabular}
\end{table}

\item {} 
A área foi nula. Eles não devem ser considerados, pois não existem retângulos cujas medidas dos lados sejam nulas.

\item {} 
Considerando a base como \(x\) temos \(x=2\) ou \(x=4\).

\item {} 
\(h(x)=6-x\) . Sim, com: \(h:]0,6[\to]0,\infty[\).

\item {} 
\(A(x)=x(6-x) \to A(x)=-(x^2-6x)\).

\item {} 
Verificação.
\end{enumerate}
}{1}
\end{answer}
\clearmargin
\begin{answer}{Perímetro Fixo}
{
\begin{enumerate}\setcounter{enumi}{8}
\item {} 
Não por vários motivos, seguem alguns:

\begin{enumerate}[label=\arabic*)]
\item a função afim é sempre monótona (sempre crescente ou sempre decrescente), os valores da última coluna do quadro nos mostram que ora \(A(x)\) é crescente ora é decrescente.

\item a função afim apresenta taxa de variação constante, já \(A(x)\) não apresenta, pois: \(\frac{5-0}{1-0}=5\) e \(\frac{8-5}{2-1}=3\).
\end{enumerate}

\item {} 
\(9\).

\item {} 
\(A(x)=-(x^2-6x+9-9)\).

\item {} 
\(A(x)=-(x^2-6x+9-9)=-(x^2-6x+9)+9=-(x-3)^2+9\) , com \(a=-1\) ; \(p=3\) e \(q=9\).

\item {} 
Não existe. \(x=p\).

\item {} 
Não. Pois para quaisquer valores de \(x\), \((x-3)^2\) sempre será positivo, e consequentemente \(-(x-3)^2\) será sempre negativo, e se esse valor negativo for somado com \(9\) o resultado obrigatoriamente será menor que \(9\).

\item {} 
\(9cm^2\).

\item {} 
\(3cm\).

\end{enumerate}
}{1}
\end{answer}

\explore{Um Caso de Otimização}
\label{\detokenize{AF209-3:sec-funcao-quadratica-vertex}}\label{\detokenize{AF209-3::doc}}\label{\detokenize{AF209-3:explorando-um-caso-de-otimizacao}}\phantomsection\label{\detokenize{AF209-3:sub-ativ-funcao-quadratica-perimetro-fixo}}
\begin{task}{Perímetro fixo}

Imagine que você tenha um pedaço de barbante de \(12\) cm de comprimento e queira moldar um retângulo com ele e calcular sua área. A figura abaixo ajuda a ilustrar a situação.

\begin{figure}[H]
\centering

\noindent\includegraphics[width=150bp]{{maos}.jpg}
\end{figure}
\begin{enumerate}
\item {} 
A situação em questão envolve quatro grandezas, aponte quais são.

\item {} 
Quais grandezas descritas acima variam e quais não variam?

\item {} 
Numa folha de papel ou similar, copie a tabela a seguir e complete-a.

\begin{table}[H]
\centering
\begin{tabu} to \textwidth{|c|c|c|}
\hline
\thead
Base & Altura & Área \\
\hline
$0$ & & \\
\hline
$2$ & & \\
\hline
$4$ & & \\
\hline
$6$ & & \\
\hline
\end{tabu}
\end{table}


\item {} 
O que ocorreu com a área para os valores da base iguais a \(0\) e \(6\)?  Esses valores devem ser considerados em nossa análise da situação?

\item {} 
Quais as medidas da base do retângulo que apresentaram área máxima no quadro acima?

\item {} 
Assumindo a base do retângulo como \(x\), e sua altura como \(h(x)\), exiba uma expressão algébrica que representa a medida da altura desse retângulo em função de \(x\). A expressão \(h(x)\), encontrada pode ser considerada uma função afim? Com que domínios e imagens?

\item {} 
Assumindo a base do retângulo como \(x\), a altura \(h(x)\) encontrada no item anterior e sua área como \(A(x)\), exiba uma expressão que apresente a área deste retângulo em função de \(x\).

\item {} 
Verifique se a relação encontrada pode ser dada por \(A(x)=-(x^2-6x)\), caso contrário refaça os itens anteriores.

\item {} 
A expressão \(A(x)\), encontrada pode ser considerada uma função afim? Por quê?

\item {} 
Observe que a relação apresentada no item \titem{h)}, possui dentro do parênteses um binômio que pode ser parte de um trinômio quadrado perfeito, qual seria o terceiro termo que faria o binômio se transformar num trinômio quadrado perfeito?

\item {} 
Agora repita a relação: \(A(x)=-(x^2-6x+\Box -\Box)\) acrescentando e retirando o número encontrado no item anterior.

\item {} 
Ao fatorar a relação do item anterior podemos recair na forma: \(A(x)=a(x-p)^2+q\), quais os valores de a, p e q, que foram encontrados neste processo de fatoração?

\item {} 
Levando em consideração a forma apresentada no item anterior, e ao analisarmos apenas o termo \((x-p)^2\), Existe algum valor de \(x\) que torne a expressão negativa? e qual valor de \(x\) torna a expressão nula?

\item {} 
Ao analisarmos \(A(x)=-(x-3)^2+9\), existe algum valor de \(x\) que faça \(A(x)\) ser maior que \(9\)? Por quê?

\item {} 
Qual a área máxima do Retângulo?

\item {} 
Qual o valor de \(x\), que gera a área máxima?

\end{enumerate}
\end{task}

\arrange{Máximos ou Mínimos}
\label{\detokenize{AF209-3:sec-funcao-quadratica-org-ideias-quad-max-min-na-quadratica}}\label{\detokenize{AF209-3:organizando-as-ideias-maximos-ou-minimos}}
Na atividade \hyperref[\detokenize{AF209-3:sub-ativ-funcao-quadratica-perimetro-fixo}]{\textit{Perímetro fixo}} você foi auxiliado na transformação da lei de formação da função \(A\) descrita por \(A(x)=6x-x²\) para \(A(x)=-(x-3)²+9\). Qual o objetivo dessa transformação? Que vantagem há nisso?

Sabe-se que uma função real do tipo \(f(x)=x^2\) tem a propriedade \(f(x) \geq 0\), para todo \(x \in \mathbb{R}\). Ou seja, qualquer variável real que esteja elevada ao quadrado tem resultado mínimo igual a zero e pode crescer tanto quanto se queira. Imagine agora que esse quadrado seja multiplicada por um número negativo, os resultados que podiam crescer o quanto se quisesse, agora ficam negativos e, na verdade, passam a diminuir tanto quanto se queira e o zero passa a ser o seu maior valor. As tabelas abaixo evidenciam isso:

\begin{multicols}{2}
\begin{table}[H]
\raggedleft
\setlength\tabulinesep{.8mm}
\begin{tabu} to \textwidth{|c|c|}
\hline
\thead
$\bm{x}$ & $\bm{x^2}$ \\
\hline
$0$ & $0$ \\
\hline
$\pm\sqrt{3}$ & $3$ \\
\hline
$\displaystyle\pm\frac{13}{3}$ & $\displaystyle\frac{169}{9}$ \\
\hline
$\pm8$ & $64$ \\
\hline
$\pm10$ & $100$ \\
\hline
$\pm100$ & $10000$ \\
\hline
$\pm1000$ & $1000000$ \\
\hline
\end{tabu}
\end{table}
\columnbreak
\begin{table}[H]
\raggedright
\setlength\tabulinesep{.5mm}
\begin{tabu} to \textwidth{|c|c|}
\hline
\thead
$\bm{x}$ & $\bm{-x^2}$ \\
\hline
$0$ & $-0$ \\
\hline
$\pm\sqrt{3}$ & $-3$ \\
\hline
$\displaystyle\pm\frac{13}{3}$ & $\displaystyle-\frac{169}{9}$ \\
\hline
$\pm8$ & $-64$ \\
\hline
$\pm10$ & $-100$ \\
\hline
$\pm100$ & $-10000$ \\
\hline
$\pm1000$ & $-1000000$ \\
\hline
\end{tabu}
\end{table}
\end{multicols}
\needspace{10em}
A análise feita gera a regra que segue.

\begin{observation}{}

Para \(f(x)=ax^2\) temos:

\(a > 0\), \(f\) tem resultado \textbf{mínimo} em \(x^2 = 0\);

\(a = 0\), \(f\) é constante e nula, ou seja \(f(x)=0\);

\(a < 0\), \(f\) tem resultado \textbf{máximo} em \(x^2 = 0\).
\end{observation}

A forma \(A(x) = 6x – x^2\) tem duas variações simultâneas: \(6x\) e \(-x^2\), o que torna mais difícil a determinação de um possível resultado máximo de \(A\). Já a forma \(A(x)=-(x-3)^2 +9\) só tem uma variação: \(-(x-3)^2\), que pela regra descrita acima tem um resultado máximo que ocorre em \((x-3)^2=0\), logo o resultado máximo de \(A\) é \(0+9=9\). Destacamos com isso o quanto fica simples a determinação de um resultado máximo ou mínimo em situações em que podemos reduzir as variações a um único termo ao quadrado.

Diante do que conhecemos até aqui, podemos finalmente estabelecer que toda função real \(f\) do tipo \(f(x)=ax^{2}+bx+c\), onde \(a\), \(b\) e \(c\) são números reais e \(a \neq 0\), pode ser transformada em sua forma equivalente \(f(x)=a(x-p)^{2}+q\). Em ambos os formatos, chamaremos \(f\) de \textbf{função quadrática}. Denominando \(f(x)=ax^{2}+bx+c\) de \textbf{forma polinomial} e \(f(x)=a(x-p)^{2}+q\) de \textbf{forma canônica} da função quadrática.

A forma \(f(x)=a(x-p)^{2}+q\) permite identificar rapidamente  qual é o resultado máximo ou mínimo da função conforme \(a\) seja positivo ou negativo.
Considere, como exemplo do que foi concluído, que o tamanho do barbante seja de \(14\) cm. Sua área \(A(x)\) em função da base \(x\) será \(A(x)=7x-x^{2}\). Fatorando \(A(x)\), teremos:
\begin{equation*}
\begin{split}& A(x)= 7x-x^{2}\\
& A(x)=-x^{2}+2 \cdot \frac{7}{2}x\\
& A(x)=-x^{2}+2 \cdot \frac{7}{2}x - \frac{49}{4} + \frac{49}{4}\\
& A(x)=-\left(x^{2}-2 \cdot \frac{7}{2}x + \frac{49}{4}\right) + \frac{49}{4}\\
& A(x)=- \left(x - \frac{7}{2} \right )^{2}+ \frac{49}{4}\\\end{split}
\end{equation*}
A função tem um resultado máximo, pois \(a=-1<0\) e este valor aparece quando \(\displaystyle\left(x-\frac{7}{2}\right)^{2}=0\), ou seja, \(x=\frac{7}{2}\). Assim, o valor máximo da função é \(\displaystyle A(x)=0+\frac{49}{4}=\frac{49}{4}\).

De modo geral, \(f(x)=ax^{2}+bx+c\) equivale a \(f(x)=a(x-p)^{2}+q\) e, avaliado se existe o resultado máximo ou o mínimo para a função real, esse resultado é o ponto \((p,q)\) que passaremos a chamar de ponto de máximo ou ponto de mínimo, dependendo do valor \(a\).

\subsection{Obtendo o ponto de máximo ou de mínimo através da forma geral}

Você já deve ter percebido que a forma geral modificada para a forma canônica, exibe imediatamente o ponto \((p,q)\). No entanto, podemos usar essa técnica no sentido inverso para que a mudança para a forma canônica não seja o único modo de obter \((p,q)\). Assim, vamos desenvolver a \textit{forma canônica} de \(f\):
\begin{equation*}
\begin{split}& a(x-p)^2+q= \\
& a(x^2-2px+p^2)+q \\
& ax^2-2apx+ap^2+q \\
& ax^2-2apx+(ap^2+q) \\\end{split}
\end{equation*}
Comparando esse resultado com sua forma equivalente \textit{forma geral} \(ax^2+bx+c\), que é a \textit{forma geral} temos:
\begin{equation*}
\begin{split}& ax^2=ax^2 \Rightarrow a=a \;\;\;\;\;\;\;\;\;\;\;\;\;\;\; (1) \\
& -2apx=bx \Rightarrow p=-\frac{b}{2a} \;\;\;\;\; (2)\\
& ap^2+q=c \Rightarrow q=c-ap^2 \;\;\;\;\; (3)\\\end{split}
\end{equation*}
A conclusão \((1)\) não traz novidade, a \((2)\) nos mostra como determinar \(p\) a partir da \textit{forma geral} e \((3)\) revelará quem é \(q\), mas precisaremos simplificar um pouco mais a expressão. Para isso, usaremos \((2)\) em \((3)\):
\begin{equation*}
\begin{split}q &=c-a \cdot \left(- \frac{b}{2a} \right)^{2} =c \cdot 1-a \cdot \left( \frac{b^2}{4a^2} \right) \\
& =c \cdot \frac{4a}{4a} - \frac{b^2}{4a}= \frac{4ac-b^2}{4a} \\
& = - \frac{b^2-4ac}{4a}\end{split}
\end{equation*}
Lembrando, que em equações do segundo grau \(ax^2+bx+c=0\), a expressão “\(b^2-4ac\)” é representada pela letra grega \(\Delta\), ou seja, \(\Delta = b^2-4ac\), temos que \(q = - \displaystyle\frac{\Delta}{4a}\).

\begin{description}
\item[Teorema 3]

Seja a função quadrática \(f\), de domínio real, definida por  \(f(x)=ax^2+bx+c\) ou pela sua forma equivalente \(f(x)=a(x-p)^2+q\), temos 

$$(p,q)= \left( -\frac{b}{2a}, -\frac{\Delta}{4a} \right)$$

e uma das situações a seguir é verdadeira:

\((i)\;(p,q)\) é o \textbf{ponto de mínimo}, se \(a>0\);

\((ii)\;(p,q)\) é o \textbf{ponto de máximo}, se \(a<0\).
\end{description}

\clearpage
\def\currentcolor{session3}
\marginpar{\vspace{.5em}}
\begin{objectives}{Menino Gauss}
{
  \begin{itemize}
  \item Reconhecer a função quadrática na expressão que dá a soma dos primeiros termos de uma progressão aritmética.
  \item Resolver o problema de somar os primeiros termos de uma progressão aritmética com as ferramentas da função quadrática.
  \end{itemize}
}{1}{1}
\end{objectives}
\begin{answer}{Menino Gauss}
{
\begin{enumerate}
\item {} 
Sim.

\item {} 
\(150\); \(133\) e \(76\).
\end{enumerate}
}{1}
\end{answer}
\clearmargin
\marginpar{\vspace{.5em}}
\begin{answer}{Menino Gauss}
{
\begin{enumerate}\setcounter{enumi}{2}
\item {} 
\begin{enumerate}[label=\Roman*)]
\item \(1+150=151\);

\item \(15+136=151\);

\item \(31+120=151\);

\item \(49+102=151\);

\item \(75+76=151\).
\end{enumerate}

\item {} 
\(151\).

\item {} 
\(150 \cdot 151=22650\).

\item {} 
Não.

\item {} 
Devemos dividir a soma obtida por \(2\); \(22650 \div 2=11325\).

\item {} 
As somas de cada número com seu correspondente no verso dá, agora, \(101\). Com isso, a soma de todos os números de ambos os lados dessa fita será \(100 \cdot 101\) e \((100 \cdot 101) \div 2 = 10100 \div 2 = 5050\).

\item {} 
As somas de cada número com seu correspondente no verso dá, agora, \(n+1\). Com isso, a soma de todos os números de ambos os lados dessa fita será \(n \cdot (n+1)\) e
\begin{gather*}
1+2+3+4+5+ \cdots +(n-3)+(n-2)+(n-1)+n=\\=\frac{n \cdot (n+1)}{2}
=\frac{n^2 + n}{2}=\frac{n^2}{2} + \frac{n}{2}
\end{gather*}
\end{enumerate}
}{1}
\end{answer}
\clearmargin
% \marginpar{\vspace{.5em}}
\begin{objectives}{Números triangulares}
{
\begin{itemize}
\item Reconhecer a função quadrática na expressão que dá a soma dos primeiros termos de uma progressão aritmética.
\item Resolver o problema de somar os primeiros termos de uma progressão aritmética com as ferramentas da função quadrática.
\end{itemize}
}{1}{1}
\end{objectives}
\begin{answer}{Números triangulares}
{
\begin{enumerate}
\item {} 
\((1,3,6,10,15,21)\)

\item {} 
Não; \(3-1 \neq 6-3 \neq 10-6 \neq 15-10 \neq 21-15\).

\item {} 
\(1\)

\(1+2\)

\(1+2+3\)

\(1+2+3+4\)

\(1+2+3+4+5\)
\end{enumerate}
}{1}
\end{answer}
\clearmargin
\begin{answer}{Números triangulares}
{
\begin{enumerate}\setcounter{enumi}{3}
\item Um número triangular é soma dos primeiros números naturais, tal como o episódio do menino \textit{Guass}.

\item {} 
Sim; \(T_{100}=1+2+3+ \cdots +98+99+100=5050\).

\item {} 
\(T_{n}= \frac{n \cdot (n+1)}{2}= \frac{n^2}{2} + \frac{n}{2}\).
\end{enumerate}
}{1}
\end{answer}

\know{Soma de uma Progressão Aritmética}
\label{\detokenize{AF209-4::doc}}\label{\detokenize{AF209-4:para-saber-mais}}

\begin{task}{menino Gauss}
No livro \textit{Antologia Matemática} de Malba Tahan, conta um episódio cuja personagem principal seria o “príncipe da matemática” Carl Frederick \textbf{Gauss} (\(\star 1777- \dagger 1855\)). Não se sabe se o episódio é real, mas conta-se que aos sete anos de idade, chegando para mais um dia de aula, \textit{Gauss} e seus colegas teriam encontrado o professor com pouca paciência. Assim, o professor, com o intuito de entreter seus alunos por longo tempo e não precisar dar-lhes qualquer atenção, pediu para que todos somassem os números naturais desde \(1\) até \(100\). Contudo, o jovem \textit{Gauss} em pouco tempo levou o resultado do exercício para o professor e este, incrédulo do feito, teria mandado \textit{Gauss} para a direção. Mais tarde, tudo se esclareceu e o professor reconheceu o acerto no método e no resultado dado pelo jovem e desculpou-se.

Como o jovem \textit{Gauss} teria obtido este resultado por um método aparentemente desconhido do enfurecido professor e com tanta rapidez?

Com a finalidade de responder a essa pergunta sugerimos uma atividade. Ela necessitará de uma fita métrica.

\begin{figure}[H]
\centering
\capstart

\noindent\includegraphics[width=200bp]{{plastic-tape-measure}.jpg}
\caption{Imagem de \href{https://commons.wikimedia.org/wiki/File:Plastic\_tape\_measure.jpg}{Pastorius} CC-BY}\label{\detokenize{AF209-4:id1}}\end{figure}

Como as fitas métricas comercializadas tem um tamanho padrão, em nossa atividade vamos entender como o jovem \textit{Gauss} fez a soma começando por somar os números da fita métrica, ou seja, vamos começar resolvendo a expressão
\begin{equation*}
\begin{split}1+2+3+4+5+ \cdots +147+148+149+150\end{split}
\end{equation*}\begin{enumerate}
\item {} 
De posse da fita métrica, perceba que ela tem os dois lados numerados. Cada um desses lados tem todos os números que queremos somar?

\item {} 
Qual o número que corresponde ao verso (outro lado da fita) do número \(1\)? E quais são os números dos versos correspondentes de \(18\) e \(75\)?
\newpage

\item {} 
Agora, vamos fazer algumas somas de um número com o seu correspondente no verso da fita. Faça:

\(1 + \,\;\) seu correspondente;

\(15 + \;\) seu correspondente;

\(31 + \;\) seu correspondente;

\(49 + \;\) seu correspondente;

\(75 + \;\) seu correspondente.

\item {} 
Qual o resultado obtido sempre que se soma um número com o seu correspondente no verso desta fita?

\item {} 
Com base na resposta do item anterior, qual o resultado da soma de todos os números dos dois lados dessa fita?

\item {} 
A soma de todos os números em ambos os lados da fita é o resultado que queríamos obter?

\item {} 
Que operação devemos fazer com a soma de todos os números da fita para que ele seja o resultado da expressão
\begin{equation*}
\begin{split}1+2+3+4+5+ \cdots +147+148+149+150 \text{ ?}\end{split}
\end{equation*}
Qual é o valor dessa expressão?

\item {} 
Imagine agora uma outra fita que tenha em cada lado, todos os números de 1 até 100.

\begin{figure}[H]
\centering

\noindent\includegraphics[width=250bp]{{fita1}.jpg}
\end{figure}

Utilizando o mesmo raciocício, tente responder a mesma pergunta feita para a turma do jovem \textit{Gauss}, ou seja, quanto dá \(1+2+3+ \cdots +97+98+99+100\)?

\item {} 
E se a fita fosse até o número natural \(n\)?

\begin{figure}[H]
\centering

\noindent\includegraphics[width=250bp]{{fita2}.jpg}
\end{figure}

Com o que foi aprendido, obtenha uma expressão para o resultado da soma dos \(n\) primeiros números naturais. Ou seja, tente expressar em função de \(n\), o resultado de \(1+2+3+4+5+ \cdots +(n-3)+(n-2)+(n-1)+n\).

\end{enumerate}
\end{task}

\begin{task}{números triangulares}

No capítulo  \hyperref[chap-funcoes]{\textbf{Introdução às Funções}}, uma das \hyperref[numeros-triangulares-funcoes]{atividades} sugere que você determine a relação entre uma sequência de figuras e a quantidade de pontos usados para compor cada figura.

\begin{figure}[H]
\centering

\begin{tikzpicture}
\begin{scope}
\draw [fill=black] (0.,0.) circle (1.0pt);
\draw [fill=black] (0.5,0.) circle (1.0pt);
\draw [fill=black] (0.5,0.5) circle (1.0pt);
\draw [fill=black] (0.,0.5) circle (1.0pt);
\draw [fill=black] (1.5,0.) circle (1.0pt);
\draw [fill=black] (2.,0.) circle (1.0pt);
\draw [fill=black] (2.,0.5) circle (1.0pt);
\draw [fill=black] (1.5,0.5) circle (1.0pt);
\draw [fill=black] (2.5,0.) circle (1.0pt);
\draw [fill=black] (2.5,1.) circle (1.0pt);
\draw [fill=black] (1.5,1.) circle (1.0pt);
\draw [fill=black] (2.,1.) circle (1.0pt);
\draw [fill=black] (2.5,0.5) circle (1.0pt);
\draw [fill=black] (3.5,0.) circle (1.0pt);
\draw [fill=black] (4.,0.) circle (1.0pt);
\draw [fill=black] (4.,0.5) circle (1.0pt);
\draw [fill=black] (3.5,0.5) circle (1.0pt);
\draw [fill=black] (4.5,0.) circle (1.0pt);
\draw [fill=black] (4.5,1.) circle (1.0pt);
\draw [fill=black] (3.5,1.) circle (1.0pt);
\draw [fill=black] (5.,0.) circle (1.0pt);
\draw [fill=black] (5.,1.5) circle (1.0pt);
\draw [fill=black] (3.5,1.5) circle (1.0pt);
\draw [fill=black] (4.,1.) circle (1.0pt);
\draw [fill=black] (4.5,0.5) circle (1.0pt);
\draw [fill=black] (5.,0.5) circle (1.0pt);
\draw [fill=black] (5.,1.) circle (1.0pt);
\draw [fill=black] (4.5,1.5) circle (1.0pt);
\draw [fill=black] (4.,1.5) circle (1.0pt);
\draw [fill=black] (-1.,0.) circle (1.0pt);
\end{scope}
\end{tikzpicture}\hfill
\begin{tikzpicture}
\begin{scope}
\draw [fill=black] (-1.,0.) circle (1.0pt);
\draw [fill=black] (0.,0.) circle (1.0pt);
\draw [fill=black] (0.5,0.) circle (1.0pt);
\draw [fill=black] (0.6545084971874737,0.4755282581475766) circle (1.0pt);
\draw [fill=black] (0.25,0.7694208842938133) circle (1.0pt);
\draw [fill=black] (-0.15450849718747367,0.4755282581475768) circle (1.0pt);
\draw [fill=black] (2.,0.) circle (1.0pt);
\draw [fill=black] (2.5,0.) circle (1.0pt);
\draw [fill=black] (2.6545084971874737,0.4755282581475766) circle (1.0pt);
\draw [fill=black] (2.25,0.7694208842938133) circle (1.0pt);
\draw [fill=black] (1.8454915028125263,0.4755282581475768) circle (1.0pt);
\draw [fill=black] (3.,0.) circle (1.0pt);
\draw [fill=black] (3.3090169943749475,0.9510565162951532) circle (1.0pt);
\draw [fill=black] (2.5,1.5388417685876266) circle (1.0pt);
\draw [fill=black] (1.6909830056250525,0.9510565162951536) circle (1.0pt);
\draw [fill=black] (4.,0.) circle (1.0pt);
\draw [fill=black] (3.1545084971874737,0.4755282581475766) circle (1.0pt);
\draw [fill=black] (2.9045084971874737,1.2449491424413899) circle (1.0pt);
\draw [fill=black] (2.0954915028125263,1.24494914244139) circle (1.0pt);
\draw [fill=black] (4.5,0.) circle (1.0pt);
\draw [fill=black] (4.654508497187473,0.4755282581475766) circle (1.0pt);
\draw [fill=black] (4.25,0.7694208842938133) circle (1.0pt);
\draw [fill=black] (3.8454915028125263,0.4755282581475768) circle (1.0pt);
\draw [fill=black] (5.,0.) circle (1.0pt);
\draw [fill=black] (5.3090169943749475,0.9510565162951532) circle (1.0pt);
\draw [fill=black] (4.5,1.5388417685876266) circle (1.0pt);
\draw [fill=black] (3.6909830056250525,0.9510565162951536) circle (1.0pt);
\draw [fill=black] (5.154508497187473,0.4755282581475766) circle (1.0pt);
\draw [fill=black] (4.904508497187473,1.2449491424413899) circle (1.0pt);
\draw [fill=black] (4.095491502812527,1.24494914244139) circle (1.0pt);
\draw [fill=black] (5.5,0.) circle (1.0pt);
\draw [fill=black] (5.963525491562422,1.42658477444273) circle (1.0pt);
\draw [fill=black] (4.75,2.3082626528814396) circle (1.0pt);
\draw [fill=black] (3.5364745084375793,1.4265847744427305) circle (1.0pt);
\draw [fill=black] (5.654508497187474,0.4755282581475767) circle (1.0pt);
\draw [fill=black] (5.837430563354646,0.9408663263400823) circle (1.0pt);
\draw [fill=black] (5.557545365872711,1.7215466012812657) circle (1.0pt);
\draw [fill=black] (5.174750541804863,2.046031522986149) circle (1.0pt);
\draw [fill=black] (4.3461569097741055,2.014853473191221) circle (1.0pt);
\draw [fill=black] (3.942313819548215,1.7214442935010055) circle (1.0pt);
\end{scope}
\end{tikzpicture}
\end{figure}

\begin{figure}[H]
\centering

\begin{tikzpicture}
\begin{scope}
\draw [fill=black] (-1.,0.) circle (1.0pt);\draw [fill=black] (0.,0.) circle (1.0pt);\draw [fill=black] (-0.5,0.) circle (1.0pt);\draw [fill=black] (0.25,0.43301270189221935) circle (1.0pt);\draw [fill=black] (0.,0.8660254037844388) circle (1.0pt);\draw [fill=black] (-0.5,0.8660254037844389) circle (1.0pt);\draw [fill=black] (-0.75,0.43301270189221974) circle (1.0pt);\draw [fill=black] (1.,0.) circle (1.0pt);\draw [fill=black] (1.5,0.) circle (1.0pt);\draw [fill=black] (2.,0.) circle (1.0pt);\draw [fill=black] (2.5,0.8660254037844387) circle (1.0pt);\draw [fill=black] (2.,1.7320508075688776) circle (1.0pt);\draw [fill=black] (1.,1.7320508075688779) circle (1.0pt);\draw [fill=black] (0.5,0.8660254037844395) circle (1.0pt);\draw [fill=black] (1.75,0.43301270189221935) circle (1.0pt);\draw [fill=black] (1.5,0.8660254037844388) circle (1.0pt);\draw [fill=black] (1.,0.8660254037844389) circle (1.0pt);\draw [fill=black] (0.75,0.43301270189221974) circle (1.0pt);\draw [fill=black] (2.25,0.43301270189221935) circle (1.0pt);\draw [fill=black] (2.25,1.2990381056766582) circle(1.0pt);\draw [fill=black] (1.5,1.7320508075688776) circle (1.0pt);\draw [fill=black] (0.75,1.2990381056766587) circle (1.0pt);\draw [fill=black] (3.5,0.) circle (1.0pt);\draw[fill=black] (4.,0.) circle (1.0pt);\draw [fill=black] (4.5,0.) circle (1.0pt);\draw [fill=black] (5.,0.) circle (1.0pt);\draw [fill=black] (5.75,1.299038105676658) circle (1.0pt);\draw[fill=black] (5.,2.5980762113533165) circle (1.0pt);\draw [fill=black] (3.5,2.598076211353317) circle (1.0pt);\draw [fill=black] (2.75,1.2990381056766593) circle (1.0pt);\draw [fill=black] (5.,0.8660254037844387) circle (1.0pt);\draw [fill=black] (4.5,1.7320508075688776) circle (1.0pt);\draw [fill=black] (3.5,1.7320508075688779) circle (1.0pt);\draw [fill=black] (3.,0.8660254037844395) circle (1.0pt);\draw [fill=black] (4.25,0.43301270189221935) circle (1.0pt);\draw [fill=black] (4.,0.8660254037844388) circle (1.0pt);\draw [fill=black] (3.5,0.8660254037844389) circle (1.0pt);\draw [fill=black] (3.25,0.43301270189221974) circle (1.0pt);\draw [fill=black] (3.25,1.2990381056766587) circle (1.0pt);\draw [fill=black] (4.,0.8660254037844388) circle (1.0pt);\draw [fill=black] (4.,1.7320508075688776) circle (1.0pt);\draw [fill=black] (4.75,1.2990381056766582) circle (1.0pt);\draw [fill=black] (4.75,0.43301270189221935) circle (1.0pt);\draw [fill=black] (5.25,0.4330127018922193) circle (1.0pt);\draw [fill=black] (5.5,0.8660254037844386) circle (1.0pt);\draw [fill=black] (5.5,1.7320508075688752) circle (1.0pt);\draw [fill=black] (5.25,2.1650635094610884) circle (1.0pt);\draw [fill=black] (4.5,2.5980762113533156) circle (1.0pt);\draw [fill=black] (4.,2.5980762113533156) circle (1.0pt);\draw [fill=black] (3.25,2.1650635094611155) circle (1.0pt);\draw [fill=black] (3.,1.7320508075689163) circle (1.0pt);
\end{scope}
\end{tikzpicture}
\end{figure}

As quantidades de pontos em cada figuras são comumente chamado de números poligonais. Assim, \((1,4,9,16, \cdots)\) são números quadrados; \((1,5,12,22, \cdots)\) são números pentagonais; etc.

Nesta atividade, vamos pensar sobre os números triângulares. A imagem a seguir exibe os cinco primeiros:
\begin{figure}[H]
\centering

\begin{tikzpicture}[scale=.75, every node/.style={scale=.75}]
\tikzstyle{circ}=[circle,draw,minimum size=1cm, fill=\currentcolor!80];
\begin{scope}
\node (A) [circ] {};

\end{scope}

\begin{scope}[xshift=1.3cm,node distance=1cm]

\node (A) [circ] {};
\node (B) [circ, right of=A] {};
\node (C) at ($(A)!.5!(B)$) [circ, 
yshift=.86602cm] {};


\end{scope}

\begin{scope}[xshift=3.4cm,node distance=1cm]

\node (A) [circ] {};
\node (B) [circ, right of=A] {};
\node (C) at ($(A)!.5!(B)$) [circ, 
yshift=.86602cm] {};
\node (D) [circ, right of=B] {};
\node (E) [circ, right of=C] {};
\node (F) at ($(C)!.5!(E)$) [circ, yshift=.86602cm] {};


\end{scope}

\begin{scope}[xshift=6.6cm,node distance=1cm]

\node (A) [circ] {};
\node (B) [circ, right of=A] {};
\node (C) at ($(A)!.5!(B)$) [circ, 
yshift=.86602cm] {};
\node (D) [circ, right of=B] {};
\node (E) [circ, right of=C] {};
\node (F) at ($(C)!.5!(E)$) [circ, yshift=.86602cm] {};
\node (G) [circ, right of=D] {};
\node (H) [circ, right of=E] {};
\node (I) [circ, right of=F] {};
\node (J) at ($(F)!.5!(I)$) [circ, yshift=.86602cm] {};


\end{scope}

\begin{scope}[xshift=10.8cm,node distance=1cm]

\node (A) [circ] {};
\node (B) [circ, right of=A] {};
\node (C) at ($(A)!.5!(B)$) [circ, 
yshift=.86602cm] {};
\node (D) [circ, right of=B] {};
\node (E) [circ, right of=C] {};
\node (F) at ($(C)!.5!(E)$) [circ, yshift=.86602cm] {};
\node (G) [circ, right of=D] {};
\node (H) [circ, right of=E] {};
\node (I) [circ, right of=F] {};
\node (J) at ($(F)!.5!(I)$) [circ, yshift=.86602cm] {};
\node (K) [circ, right of=G] {};
\node (L) [circ, right of=H] {};
\node (M) [circ, right of=I] {};
\node (N) [circ, right of=J] {};
\node (O) at ($(N)!.5!(J)$) [circ, yshift=.86602cm] {};


\end{scope}

\end{tikzpicture}
\end{figure}
\begin{enumerate}
\item {} 
Escreva a sequência de números triângulares até o sexto termo.

\item {} 
Os números triangulares formam uma progressão aritmética?

\item {} 
A figura a seguir, destaca as linhas de cada triângulo, uma de cada cor. Escreva o total de bolinhas de \textbf{cada um desses triângulos} como soma das quantidades das suas linhas. Exemplo: \(T_4 = 1 + 2 + 3 + 4\)

\begin{figure}[H]
\centering

\begin{tikzpicture}[scale=.75, every node/.style={scale=.75}]
\tikzstyle{circ}=[circle,draw,minimum size=1cm, fill=\currentcolor!80];
\begin{scope}
\node (A) [circ] {};

\end{scope}

\begin{scope}[xshift=1.3cm,node distance=1cm]

\node (A) [circ] {};
\node (B) [circ, right of=A] {};
\node (C) at ($(A)!.5!(B)$) [circ, 
yshift=.86602cm, fill=session4!80] {};

\end{scope}

\begin{scope}[xshift=3.4cm,node distance=1cm]

\node (A) [circ] {};
\node (B) [circ, right of=A] {};
\node (C) at ($(A)!.5!(B)$) [circ, 
yshift=.86602cm, fill=session4!80] {};
\node (D) [circ, right of=B] {};
\node (E) [circ, right of=C, fill=session4!80] {};
\node (F) at ($(C)!.5!(E)$) [circ, yshift=.86602cm, fill=session2!80] {};

\end{scope}

\begin{scope}[xshift=6.6cm,node distance=1cm]

\node (A) [circ] {};
\node (B) [circ, right of=A] {};
\node (C) at ($(A)!.5!(B)$) [circ, 
yshift=.86602cm, fill=session4!80] {};
\node (D) [circ, right of=B] {};
\node (E) [circ, right of=C, fill=session4!80] {};
\node (F) at ($(C)!.5!(E)$) [circ, yshift=.86602cm, fill=session2!80] {};
\node (G) [circ, right of=D] {};
\node (H) [circ, right of=E, fill=session4!80] {};
\node (I) [circ, right of=F, fill=session2!80] {};
\node (J) at ($(F)!.5!(I)$) [circ, yshift=.86602cm, fill=cor1!80] {};

\end{scope}

\begin{scope}[xshift=10.8cm,node distance=1cm]

\node (A) [circ] {};
\node (B) [circ, right of=A] {};
\node (C) at ($(A)!.5!(B)$) [circ, 
yshift=.86602cm, fill=session4!80] {};
\node (D) [circ, right of=B] {};
\node (E) [circ, right of=C, fill=session4!80] {};
\node (F) at ($(C)!.5!(E)$) [circ, yshift=.86602cm, fill=session2!80] {};
\node (G) [circ, right of=D] {};
\node (H) [circ, right of=E, fill=session4!80] {};
\node (I) [circ, right of=F, fill=session2!80] {};
\node (J) at ($(F)!.5!(I)$) [circ, yshift=.86602cm, fill=cor1!80] {};
\node (K) [circ, right of=G] {};
\node (L) [circ, right of=H, fill=session4!80] {};
\node (M) [circ, right of=I, fill=session2!80] {};
\node (N) [circ, right of=J, fill=cor1!80] {};
\node (O) at ($(N)!.5!(J)$) [circ, yshift=.86602cm,fill=session1!80] {};


\end{scope}

\end{tikzpicture}
\end{figure}
\item Após o item anterior, que relação você percebe entre os números triangulares e o episódio do menino \textit{Gauss}?

\item Com base nessa relação, você seria capaz de determinar o centésimo número triangular? Determine-o.

\item Chamando de \(T_{n}\) o número triangular da posição \(n\), escreva a relação entre \(n\) e \(T_{n}\).
\end{enumerate}

\end{task}

De modo mais geral, a soma dos primeiros termos de qualquer progressão aritmética é expressa por uma função quadrática.

Isso acontece porque o método que usamos para somar números naturais, que formam uma progressão aritmética, continua válido para uma progressão aritmética diferente dessa. Observe.
\begin{align*}\!\begin{aligned}
a_{1}+a_{2}+a_{3}+ \cdots +a_{n-1}+a_{n}\\
a_{n}+a_{n-1}+ \cdots +a_{3}+a_{2}+a_{1}\\
\end{aligned}\end{align*}
Somando um elemento de cada linha e na ordem escrita teremos:
\begin{equation*}
\begin{split}(a_{1}+a_{n})+(a_{2}+a_{n-1})+ \cdots + (a_{n-1}+a_{2})+(a_{n}+a_{1})\end{split}
\end{equation*}
Fazendo uma analogia com a atividade, é fato (verificável de maneira simples) que todas as parcelas dessa soma são iguais, além disso, a quantidade de parcelas é dada pela mesma da quantidade de elementos da progressão. E também, temos que cada par dentro dos parênteses exibe um elemento que, em relação às sequências de onde foram extraídos, varia em \(+r\), enquanto o outro varia \(-r\), onde \(r\) é a razão da progressão aritmética. Assim, dispomos de \(n\) parcelas iguais a, por exemplo, \(a_{1}+a_{n}\). Já podemos concluir o teorema a seguir:

\begin{description}
\item[Teorema 4]

Dada a progressão aritmética \((a_{1},a_{2},a_{3}, \cdots ,a_{n-1},a_{n}, \cdots)\), a soma dos seus \(n\) primeiros termos será indicada por \(S_{n}\) e
\begin{equation*}
\begin{split}S_{n} = \frac{n \cdot (a_{1}+a_{n})}{2}\end{split}
\end{equation*}\end{description}

Contudo, sabe-se que \(a_{n}=a_{1}+(n-1)\cdot r\) e a relação da soma dos primeiros termos da progressão aritmética pode ainda ser apresentada conforme segue:
\begin{align*}
S_{n}&=\frac{[a_1+a_{1}+(n-1)\cdot r] \cdot n}{2}= \frac{[a_1 \cdot n +a_{1}  \cdot n + (n \cdot r-r)\cdot n]}{2}= \frac{2a_{1}n+n^2r-rn}{2}\\
S_{n}&=\frac{r}{2} \cdot n^2 + \frac{(2a_{1}-r)}{2} \cdot n
\end{align*}
que é uma função quadrática dada em sua forma polinomial (com \(c=0\)) e domínio discreto \(\mathbb{N}^*\).

\begin{observation}{}

A expressão que fornece a soma dos \(n\) primeiros termos de uma progressão aritmética, em função de \(n\), é uma \textbf{função quadrática}.
\end{observation}

\cleardoublepage
\def\currentcolor{session1}
\begin{objectives}{O gráfico e a forma canônica}
{
\begin{itemize}
\item Reconhecer que a variação dos valores de a acarretam na concavidade e na existência $(a=0)$ no gráfico de $f$.
\item Reconhecer que a variação dos valores de $p$ acarretam translações horizontais no gráfico de $f$.
\item Reconhecer que a variação dos valores de $q$ acarretam translações verticais no gráfico de $f$.
\item Reconhecer que toda função real $f$ dada por $f(x)=a(x−p)^2+q$ pode ser obtida por translações do gráfico de $f(x)=ax^2$.
\end{itemize}
}{1}{1}
\end{objectives}
\begin{sugestions}{O gráfico e a forma canônica}
{
Prezado colega, após o aluno ser instigado a desenvolver a forma canônica da expressão apresentada na atividade anterior, propomos uma análise mais criteriosa nos coeficientes $a, p$ e $q$, de $y=a(x−p)^2+q$ através das transformações ocorridas na curva. Para isso dispomos da atividade tanto no modelo textual (estático) quanto num modelo interativo disponível na plataforma do geogebra via link em destaque (a seguir no texto). Neste modelo interativo separamos a abordagem em quatro partes:

\begin{itemize}
\item[Parte 1:] Análise dos valores de $a$.
\item[Parte 2:] Análise dos valores de $p$.
\item[Parte 3:] Análise dos valores de $q$.
\item[Parte 4:] Análise dos valores de $a$, $p$ e $q$.
\end{itemize}

Sugerimos ao colega que acesse antes os “links”, não só para testar a funcionalidade deles, mas para se apropriar das vantagens que a plataforma oferece. Caso seja da realidade de seus alunos, sugerimos também o acesso à atividade como tarefa de casa.

Essa atividade é um boa referência para conduzirmos a conclusão de que \textbf{todas as parábolas são semelhantes}.
}{1}{1}
\end{sugestions}
\clearmargin
\marginpar{\vspace{.5em}}
\begin{answer}{O gráfico e a forma canônica}
{
\paragraph{Parte 1}

\begin{enumerate}
\item {} 
Mais fechada.

\item {} 
Mais aberta.

\item {} 
\(a\) é o coeficiente que multiplica o \(x^2\), sendo \(0<a<1\) o valor resultante dessa multiplicação (imagem de \(f\)) é um número menor que \(x^2\), o que acarreta um crescimento (para \(x>0\)) mais lento de \(f\) o que leva a concavidade ser mais aberta. Já no caso \(a>1\) o resultado desse produto (imagem de \(f\)) é um valor maior que \(x^2\), o que acarreta um crescimento (para \(x>0\)) mais acelerado de \(f\), o que leva a concavidade ser mais fechada.

\item {} 
Mais fechada.

\item {} 
Mais aberta.
\end{enumerate}
}{1}
\end{answer}
\clearmargin
\marginpar{\vspace{.5em}}
\begin{answer}{O gráfico e a forma canônica}
{
\begin{enumerate}\setcounter{enumi}{5}
\item {} 
\begin{enumerate}[label=\titem{\roman*)}]
\item A curva na verdade se transforma numa reta, no caso a função real constante \(f\) definida por \(f(x)=0\).

\item Correto.

\item Não há mais curva, e sim a reta \(y=0\).

\item Correto.
\end{enumerate}

\item {} 
\begin{enumerate}[label=\titem{\roman*)}]
\item Quando $a>0a$, da esquerda para direita, a curva é decrescente e ao assumir o seu valor \textbf{mínimo} passa a ser crescente.

\item Quando $a>0a$, da esquerda para direita, a curva é \textbf{decrescente} e ao assumir o seu valor mínimo passa a ser \textbf{crescente}.

\item Quando$a>0a$, da esquerda para direita, a curva é \textbf{crescente} e ao assumir o seu valor máximo passa a ser \textbf{decrescente}.

\item Quando $a>0a$, da esquerda para direita, a curva é crescente e ao assumir o seu valor \textbf{máximo} passa a ser decrescente.
\end{enumerate}
\end{enumerate}
}{1}
\end{answer}
\clearmargin
\marginpar{\vspace{.5em}}

\begin{answer}{O gráfico e a forma canônica}
{
\paragraph{Parte 2}

\begin{enumerate}
\item {} 
Direita.

\item {} 
Esquerda.
\end{enumerate}
}{1}
\end{answer}
\clearmargin
\marginpar{\vspace{.5em}}
\begin{answer}{O gráfico e a forma canônica}
{
\begin{enumerate}\setcounter{enumi}{2}
\item {} 
\begin{enumerate}
\item {} 
; (F); (F) ; (V)
\end{enumerate}
\item {} 
Translação Horizontal.
\end{enumerate}
}{1}
\end{answer}
\clearmargin
\marginpar{\vspace{.5em}}
\begin{answer}{O gráfico e a forma canônica}
{
\paragraph{Parte 3}
\begin{enumerate}
\item {} 
Cima.

\item {} 
Baixo.

\item {} 

\begin{enumerate}
\item {} 
; (F); (F); (V); (F) ; (V)
\end{enumerate}
\item {} 
Translação Vertical.
\end{enumerate}
}{1}
\end{answer}
\clearmargin
\clearmargin
\marginpar{\vspace{.5em}}
\begin{answer}{O gráfico e a forma canônica}
{
\paragraph{Parte 4}
\begin{enumerate}
\item 
\begin{enumerate}[label=\titem{\roman*)}]
\item Horizontal.
\item Vertical.
\end{enumerate}
\end{enumerate}
}{1}
\end{answer}
\clearmargin
\clearmargin
\marginpar{\vspace{.5em}}
\begin{answer}{O gráfico e a forma canônica}
{
\begin{enumerate}\setcounter{enumi}{1}
\item {} 
\(V=(p,q)\)

\item {} 
\(D=\mathbb{R}\) e \(I=[0,+\infty[\)
\end{enumerate}
}{1}
\end{answer}
\clearmargin
\marginpar{\vspace{.5em}}
\begin{answer}{O gráfico e a forma canônica}
{
\begin{enumerate}\setcounter{enumi}{3}
\item {} 
\(D=\mathbb{R}\) e \(I=[-\infty,-4]\)

\item {} 
(F);(F);(F);(V);(V);(F);(F);(F);(F);(V);(F);(F);(V);(F)

\item {} 
(F);(F);(V);(F);(F);(F);(V);(F);(F);(V);(F);(F)
\end{enumerate}
}{1}
\end{answer}


\explore{O Gráfico da Função Quadrática}
\label{\detokenize{AF209-5:sec-funcao-quadratica-parametros-grafico}}\label{\detokenize{AF209-5::doc}}\label{\detokenize{AF209-5:explorando-os-parametros-da-forma-canonica-e-o-grafico-da-funcao-quadratica}}\phantomsection\label{\detokenize{AF209-5:ativ-funcao-quadratica-graf-curva}}
\begin{task}{O gráfico e a forma canônica}



Para melhor explorarmos essa atividade sugerimos a versão online, disponível nos links a seguir:
\begin{itemize}
\item {} 
Parte 1: \href{https://ggbm.at/jdFEcyav}{Forma Canônica e o parâmetro ‘$a$’}

\item {} 
Parte 2: \href{https://ggbm.at/DmKxRtU9}{Forma Canônica e o parâmetro ‘$p$’}

\item {} 
Parte 3: \href{https://ggbm.at/Qcm5QFjH}{Forma Canônica e o parâmetro ‘$q$’}

\item {} 
Parte 4: \href{https://ggbm.at/jVJh78hz}{Forma Canônica}

\end{itemize}

Caso não seja possível, segue a atividade que corresponde à apresentada nos “links”:

Na atividade \hyperref[\detokenize{AF209-2:ativ-funcao-quadratica-investigando-x-a-2}]{\textit{Em busca de padrões em \(f(x)=x^2\)}}, você teve a oportunidade de explorar as propriedades do gráfico da função \(f:\mathbb{R}\to\mathbb{R}\) dada por \(f(x)=x^2\), já na atividade 3, você foi apresentado à um processo que o levou a transformar a relação quadrática dada na forma polinomial: \(f(x)=ax^2 + bx + c\) para forma canônica \(f(x)=a(x-p)^2+q\). O objetivo desta atividade é que você consiga perceber as mudanças ocorridas no gráfico da função \(f\) (dada em sua forma canônica) acarretadas pelas variações dos coeficientes \(a\), \(p\) e \(q\). Esperamos que além de você ter contato com novos conceitos, comprove e consolide os conceitos abordados nas atividades anteriores deste capítulo.

\paragraph{Parte 1}

Dada a função \(f:\mathbb{R}\to\mathbb{R}\), definida na sua forma canônica: \(f(x)=a(x-p)^2+q\), ao assumirmos \(p=q=0\) temos que \(f(x)=ax^2\), onde analisaremos as variações dos valores de \(a>0\), observando a figura a seguir:
\vfill
\begin{figure}[H]
\centering

\begin{tikzpicture}[scale=.5, every node/.style={overlay}]

\draw [thin, help lines, dashed, secundario!40] (-8,-3.5) grid (8,13);
\draw [->] (0,-3.5)--(0,13);
\draw [->] (-8,0)--(8,0);
\node [below] at (8,0) {$x$};
\node [left] at (0,13) {$y$};
\foreach \y in {-3,-2,-1,1,2,...,10,11,12} \node [left, scale=.75] at (0,\y) {\y};
\foreach \x in {-7,-6,...,6,7} \node [below, scale=.75] at (\x+.2,0) {\x};

%f(x)
\draw [color=\currentcolor!80, domain=-7.8:7.8, thick, samples=1000] plot (\x,{ 0.05*((\x)^2)});
\node [below, color=\currentcolor!80!] at (-7,1.5) {$f(x) = 0.05 x^2$};

%g(x)
\draw [color=terciario, domain=-7.8:7.8, thick, samples=1000] plot (\x,{ 0.15*((\x)^2)});
\node [below, color=terciario!30!black] at (7,10) {$g(x) = 0.15 x^2$};

%h(x)
\draw [color=atento!60!black, domain=-4.6:4.6, thick, samples=1000] plot (\x,{ 0.5*((\x)^2)});
\node [below, color=atento!60!black] at (-7,11) {$h(x) = 0.5 x^2$};

%p(x)
\draw [color=destacado,domain=-3.4:3.4, thick, samples=1000] plot (\x,{(\x)^2});
\node [below, color=destacado!60!black] at (5,12.75) {$p(x) = x^2$};

%t(x)
\draw [color=secundario, domain=-1.6:1.6, thick, samples=1000] plot (\x,{5*((\x)^2)});
\node [below, color=secundario] at (2.5,14) {$t(x)=5 x^2$};

%q(x)
\draw [color=yellow!80!black,domain=-2.45:2.45, thick, samples=1000] plot (\x,{2*((\x)^2)});
\node [below, color=yellow!60!black] at (-3.5,13) {$q(x)=2x^2$};
\end{tikzpicture}
\end{figure}

Note que os gráficos apresentados na figura acima apresentam apenas valores de \(a\) maiores que zero, e que a curva em questão é côncava, com base nessa afirmação responda:
\begin{enumerate}
\item {} 
Quando o valor de \(a\) aumenta, a concavidade da curva fica mais aberta ou mais fechada?

\item {} 
Quando o valor de \(a\) se aproxima de zero, a concavidade da curva fica mais aberta ou mais fechada?

\item {} 
Tente explicar com suas palavras uma justificativa para as respostas dadas no item anterior.

Observe as novas figuras a seguir que apresentam novos valores de \(a<0\).


\begin{figure}[H]
\centering

\begin{tikzpicture}[scale=.75]

\draw [thin, help lines, dashed, secundario!40] (-8,3.5) grid (8,-13);
\draw [->] (0,3.5)--(0,-13);
\draw [->] (-8,0)--(8,0);
\node [above] at (8,0) {$x$};
\node [left] at (0,-13) {$y$};
\foreach \y in {3,2,1,-1,-2,...,-10,-11,-12} \node [left, scale=.75] at (0,\y) {\y};
\foreach \x in {-7,-6,...,6,7} \node [above, scale=.75] at (\x+.2,0) {\x};
\draw [color=\currentcolor!80, domain=-7.8:7.8, thick, samples=1000] plot (\x,{ -0.05*((\x)^2)});
\node [above, color=\currentcolor!80!] at (6,-1) {$f(x) = -0.05 x^2$};
\draw [color=terciario, domain=-7.8:7.8, thick, samples=1000] plot (\x,{ -0.15*((\x)^2)});
\node [above, color=terciario!30!black] at (-7,-10) {$g(x) = -0.15 x^2$};
\draw [color=atento!60!black, domain=-4.6:4.6, thick] plot (\x,{ -0.5*((\x)^2)});
\node [above, color=atento!60!black] at (6,-11) {$h(x) = -0.5x^2$};
\draw [color=destacado,domain=-3.4:3.4, thick, samples=1000] plot (\x,{-(\x)^2});
\node [above, color=destacado!60!black] at (-4.8,-12) {$p(x) = -x^2$};
\draw [color=secundario, domain=-1.6:1.6, thick, samples=1000] plot (\x,{-5*((\x)^2)});
\node [above, color=secundario] at (-2.5,-13.5) {$t(x)=5x^2$};
\draw [color=yellow!80!black,domain=-2.45:2.45, thick, samples=1000] plot (\x,{-2*((\x)^2)});
\node [above, color=yellow!60!black] at (3.2,-13) {$q(x)=-2x^2$};
\end{tikzpicture}
\end{figure}


\item {} 
Quando o valor de \(a\) diminui (fica “mais negativo”), a concavidade da curva fica mais aberta ou mais fechada?

\item {} 
Quando o valor de \(a\) se aproxima de zero, a concavidade da curva fica mais aberta ou mais fechada?

A figura a seguir apresenta o gráfico da função \(f\) definida anteriormente para \(a=0\).
\begin{center}\begin{tikzpicture}[scale=1.25]

\draw [help lines,color = secundario!20, step=0.2] (-5.5,-3.5) grid (5.5,3);
\draw [help lines,color = secundario!50] (-5.5,-3.5) grid (5.5,3);
\draw [-,color=destacado] (-5.5,0) -- (5.5,0);
\draw[-] (0,3)--(0,-3.4);
\foreach \x in {-5,...,-1,1,2,...,5} \node [below] at (\x-.1,0) {\x};
\foreach \y in {-3,-2,1,-1,2,3} \node [left] at (0,\y) {\y};
\node [below left] (0,0) {0};
\draw [fill=white] (-5,1) rectangle (-1.2,2.6);
\node [below right] at (-5,2.5) {Fun\c{c}\~ao \textcolor{destacado}{$f(x)=0x^2$}};
\node [below right] at (-3.8,1.7) {$a=0$};
\end{tikzpicture}\end{center}
\item {} 
Com base no gráfico acima, comente cada uma das alternativas a seguir, que indicam o comportamento do gráfico quando \(a=0\).

\begin{enumerate}
\item A curva some, pois não é mais função.

\item Não existe mais curva, o gráfico apresentado é uma reta representada pela função constante \(f:\mathbb{R}\to\mathbb{R}\) dado por \(f(x)=0\)

\item A curva ainda existe mais fica invisível, pois a abertura de sua concavidade tende ao infinito.

A curva se transforma numa reta que está sobreposta ao eixo das abscissas.
\end{enumerate}

\item {} 
Você deve ter notado que quando o valor de \(a>0\) a concavidade da curva aponta para cima, e quando \(a<0\) a concavidade aponta para baixo. Com base neste fato, reescreva as falsas afirmações a seguir, tornando-as verdadeiras:

\begin{enumerate}
\item Quando \(a>0\) a, da esquerda para direita, a curva é decrescente e ao assumir o seu valor máximo passa a ser crescente.

\item Quando \(a>0\) a, da esquerda para direita, a curva é crescente e ao assumir o seu valor mínimo passa a ser decrescente.

\item Quando \(a<0\) a, da esquerda para direita, a curva é decrescente e ao assumir o seu valor máximo passa a ser crescente.

\item Quando \(a<0\) a, da esquerda para direita, a curva é crescente e ao assumir o seu valor mínimo passa a ser decrescente.
\end{enumerate}

\end{enumerate}

\needspace{5em}
\paragraph{Parte 2}

Dada a função \(g:\mathbb{R}\to\mathbb{R}\), definida na sua forma canônica: \(g(x)=a(x-p)^2+q\), tomemos \(a=1\) e \(q=0\) e analisaremos os valores de \(p\) na função \(f(x)=(x-p)^2\) observando a figura a seguir:
\begin{figure}[H]
\centering

\begin{tikzpicture}[scale=.65]
\draw [help lines, thin, dotted, secundario!40, dashed] (-8,-3.5) grid (9,13);
\draw [->] (0,-3.5)--(0,13);
\draw [->] (-8,0)--(9,0);
\node [above,] at (8.5,0) {$x$};
\node [right,] at (0.3,12.6) {$y$};
\foreach \y in {-3,-2,-1,1,2,...,10,11,12} \node [left,] at (0,\y) {\y};
\foreach \x in {-7,-6,...,7,8} \node [below,] at (\x+.2,0) {\x};
\draw [color=\currentcolor!80, domain=-7.5:-0.5, samples=1000, thick] plot (\x,{( 4+(\x))^2});
\node [below, color=\currentcolor!80!,] at (-5.8,-0.8) {$f(x) =(x+4)^2$};
\draw [color=terciario, domain=-5.5:1.5, samples=1000, thick] plot (\x,{( 2+(\x))^2});
\node [below, color=terciario!30!black] at (-2.5,-1.8) {$g(x) = (2+ x)^2$};
\draw [color=atento, domain=-3.5:3.5, samples=1000, thick] plot (\x,{ (\x)^2});
\node [below, color=atento!60!black,] at (1,-0.8) {$h(x) = x^2$};
\draw [color=destacado, domain=-0.5:6.5, samples=1000, thick] plot (\x,{(-3+(\x))^2});
\node [below, color=destacado!60!black,] at (3.8,-1.8) {$p(x) = (x-3)^2$};
\draw [color=secundario, domain=1.5:8.5, samples=1000, thick] plot (\x,{(-5+(\x))^2});
\node [below, color=secundario!50!black,] at (7,-0.8) {$t(x)=(x-5)^2$};
\end{tikzpicture}
\caption{Variações de \(p\).}
\end{figure}
Variações de \(p\).

Em cada um dos itens a seguir destaque as alternativas verdadeiras.
\begin{enumerate}
\item {} 
Quando os valores de \(p\) aumentam a curva se desloca para

({ }{ }{ }) direita.

({ }{ }{ }) cima.

({ }{ }{ }) esquerda.

({ }{ }{ }) baixo.

\item {} 
Quando os valores de \(p\) diminuem a curva se desloca para

({ }{ }{ }) direita.

({ }{ }{ }) cima.

({ }{ }{ }) esquerda.

({ }{ }{ }) baixo.

\item {} 
Você deve ter notado que a curva tangencia o eixo das abscissas em um ponto, que é justamente o ponto em que a curva deixa de ser decrescente e passa a ser crescente. Qual é a relação dos valores de \(p\) com este ponto?

({ }{ }{ }) O ponto de tangência em questão é \((-p,0)\).

({ }{ }{ }) O ponto de tangência em questão é \((0,-p)\).

({ }{ }{ }) O ponto de tangência em questão é \((0,p)\).

({ }{ }{ }) O ponto de tangência em questão é \((p,0)\).

\item {} 
O movimento que a curva faz quando \(p\) varia, é uma

({ }{ }{ }) translação vertical.

({ }{ }{ }) translação horizontal.

({ }{ }{ }) rotação em \(360^{\circ}\).

({ }{ }{ }) rotação em \(180^{\circ}\).

\end{enumerate}

\paragraph{Parte 3}

Dada a função \(g:\mathbb{R}\to\mathbb{R}\), definida na sua forma canônica: \(g(x)=a(x-p)^2+q\), tomemos \(a=1\) e \(p=0\) e analisaremos os valores de \(q\) na função \(f(x)=x^2+q\) observando a figura a seguir:


\begin{figure}[H]
\centering

\begin{tikzpicture}[scale=.7]
[scale=.3]
\draw [ thin, help lines, dashed, secundario!40] (-5.5,-5.5) grid (5.5,9);
\draw [->] (-5.5,0)--(5.5,0) node [above left] {$x$};
\draw [->] (0,-5.5)--(0,9) node [below right] {$y$};
\foreach \y in {-5,...,-1,1,2,...,8} \node [left] at (0,\y) {\y};
\foreach \x in {-5,...,-1,1,2,...,5} \node [below] at (\x,0) {\x};
\node [below left] at (0,0) {0};
\draw [color=secundario, domain=-3.74165:3.74165, thick] plot (\x,{(((\x)^2)-5)});
\node [color=secundario, below] at (0,-5.1) {$x^2-5$};
\draw [color=\currentcolor!80, domain=-3.16227:3.16227, thick] plot (\x,{(((\x)^2)-1)});
\node [color=\currentcolor!80, below] at (0,-1.2) {$x^2-1$};
\draw [color=orange, domain=-3:3, thick] plot (\x,{(((\x)^2))});
\node [above, color=orange,] at (0,0) {$x^2$};
\draw [color=destacado, domain=-2.6457:2.6457, thick] plot (\x,{(((\x)^2)+2)});
\node [below, color=destacado] at (0,1.8) {$x^2+2$};
\draw [color=atento, domain=-2:2, thick] plot (\x,{(((\x)^2)+5)});
\node [below, color=atento] at (0,4.8) {$x^2+5$};
\end{tikzpicture}

\caption{Variação de \(q\)}
\end{figure}

\needspace{10em}
Em cada um dos itens a seguir destaque as alternativas verdadeiras.
\begin{enumerate}
\item {} 
Quando os valores de \(q\) aumentam a curva se desloca para

({ }{ }{ }) direita.

({ }{ }{ }) cima.

({ }{ }{ }) esquerda.

({ }{ }{ }) baixo.

\item {} 
Quando os valores de \(q\) diminuem a curva se desloca para

({ }{ }{ }) direita.

({ }{ }{ }) cima.

({ }{ }{ }) esquerda.

({ }{ }{ }) baixo.

\item {} 
Você deve ter notado que a curva intersecta o eixo das ordenadas em um ponto, que é justamente o ponto em que a curva deixa de ser decrescente e passa a ser crescente. Quais são relações dos valores de \(q\) com esse ponto?

({ }{ }{ }) O ponto de intersecção é \((-q,0)\).

({ }{ }{ }) O ponto de intersecção é \((q,0)\).

({ }{ }{ }) O ponto de intersecção é \((0,-q)\).

({ }{ }{ }) O ponto de intersecção é \((0,q)\).

({ }{ }{ }) Na figura, \(q\) representa o maior valor que essa função atinge.

({ }{ }{ }) Na figura, \(q\) representa o menor valor que essa função atinge.

\item {} 
O movimento que a curva faz quando \(q\) varia, é uma

({ }{ }{ }) translação vertical.

({ }{ }{ }) translação horizontal.

({ }{ }{ }) rotação em \(360^{\circ}\).

({ }{ }{ }) rotação em \(180^{\circ}\).

\end{enumerate}
\needspace{10em}

\paragraph{Parte 4}

Em cada uma das partes anteriores, estudamos as variações gráficas que cada um dos valores de \(a\), \(p\) e \(q\) fazem na curva. Para elucidarmos essas ideias, convidamos a variar esses valores juntos na função \(f:\mathbb{R}\to\mathbb{R}\), definida na sua forma canônica: \(f(x)=a(x-p)^2+q\).
\begin{enumerate}
\item {} 
Observe as figuras a seguir, e note que em todas os valores de \(a\) são sempre iguais a \(1\), já os valores de \(p\) e \(q\) variam.

\begin{figure}[H]
\centering
\capstart

\noindent\includegraphics[width=325bp]{{41}.jpg}
\caption{(\(p=4\) e \(q=-3\))}\label{\detokenize{AF209-5:id9}}\end{figure}

\begin{figure}[H]
\centering
\capstart

\noindent\includegraphics[width=325bp]{{411}.jpg}
\caption{(\(p=3\) e \(q=0\))}\label{\detokenize{AF209-5:id10}}\end{figure}

\begin{figure}[H]
\centering
\capstart

\noindent\includegraphics[width=325bp]{{412}.jpg}
\caption{(\(p=-1\) e \(q=2\))}\label{\detokenize{AF209-5:id11}}\end{figure}

\begin{enumerate}
\item A variação de \(p\) faz com que o gráfico sofra que tipo de translação (vertical ou horizontal?

\item A variação de \(q\) faz com que o gráfico sofra que tipo de translação (vertical ou horizontal?
\end{enumerate}

\item {} 
As figuras a seguir mostram as variações obtidas no gráfico para os valores de \(a = 1\), (\(p =5\) e \(q =5\)); (\(p=-5\) e \(q=5\)); em seguida (\(p=5\) e \(q=-5\)) e por último (\(p=-5\) e \(q=-5\)). Já vimos anteriormente que existe um ponto no gráfico em que a função deixa de ser decrescente e passa a ser crescente, este ponto chamamos de \textbf{vértice} da curva.

\begin{figure}[H]
\centering
\capstart

\noindent\includegraphics[width=325bp]{{42}.jpg}
\caption{(\(p=5\) e \(q=5\))}\label{\detokenize{AF209-5:id12}}\end{figure}

\begin{figure}[H]
\centering
\capstart

\noindent\includegraphics[width=325bp]{{43}.jpg}
\caption{(\(p=-5\) e \(q=5\))}\label{\detokenize{AF209-5:id13}}\end{figure}

\begin{figure}[H]
\centering
\capstart

\noindent\includegraphics[width=325bp]{{44}.jpg}
\caption{(\(p=5\) e \(q=-5\))}\label{\detokenize{AF209-5:id14}}\end{figure}

\begin{figure}[H]
\centering
\capstart

\noindent\includegraphics[width=325bp]{{45}.jpg}
\caption{(\(p=-5\) e \(q=-5\))}\label{\detokenize{AF209-5:id15}}\end{figure}

Exiba as coordenadas do vértice em função de \(p\) e \(q\).

\item {} 
Observe que ao mantermos os valores de $a=1$, $p=0$ e $q=0$, temos a curva $y=x^2$. Considerando uma função \(f\) de Domínio \(D\) e imagem \(I\) dada por $f(x)=y$, utilize a figura a seguir, e em seguida escolha a alternativa na qual os conjuntos \(D\) e \(I\) estão definidos na atividade.


\begin{figure}[H]
\centering
\begin{tikzpicture}[scale=.7, yscale=.7]
\draw [very thin, help lines, dotted, secundario!70] (-5,-2.5) grid (5.5,9);
\draw [->] (0,-2.5)--(0,9);
\draw [->] (-5,0)--(5.5,0);
\node [above] at (5,0) {$x$};
\node [right] at (0.3,8.5) {$y$};
\foreach \y in {,-2,-1,1,2,...,8} \node [left,scale=.75] at (0,\y) {\y};
\foreach \x in {-5,-4,...,4,5} \node [below,scale=.75] at (\x+.2,0) {\x};
\draw [color=\currentcolor!80,domain=-3:3, thick] plot (\x,{(\x)^2});
\end{tikzpicture}

\caption{$(a=1; p=q=0)$}
\end{figure}
({ }{ }{ }) \(D=[-5,5]\) e \(I=[0,5]\)

({ }{ }{ }) \(D=[0,+\infty[\) e \(I=[0,+\infty[\)

({ }{ }{ }) \(D=[0,5]\) e \(I=[-5,5]\)

({ }{ }{ }) \(D=\mathbb{R}\) e \(I=[0,+\infty[\)

({ }{ }{ }) \(D=\mathbb{R}\) e \(I=\mathbb{R}\)

\item {} 
Observe que ao mantermos os valores de \(a=-2\), \(p=3\) e \(q=-4\), temos que \(y=-2(x-3)^2 -4\). Considerando uma função \(f\) de Domínio \(D\) e imagem \(I\) dada por \(f(x)=y\), utilize a figura a seguir, e em seguida escolha a alternativa na qual os conjuntos \(D\) e \(I\) estão definidos na atividade.
\begin{figure}[H]
\centering
\begin{tikzpicture}[scale=.75, yscale=.75]

\draw [very thin, help lines, dotted, secundario!70] (-3.5,-8.5) grid (5.5,1.5);
\draw [->] (0,-8.5)--(0,1.5);
\draw [->] (-3.5,0)--(5.5,0);
\node [above] at (5,0) {$x$};
\node [right] at (0.3,1.5) {$y$};
\foreach \y in {,-8,-7,...,-1,1} \node [left,scale=.75] at (0,\y) {\y};
\foreach \x in {-3,-2,...,4,5} \node [below,scale=.75] at (\x+.2,0) {\x};
\draw [color=\currentcolor!80,domain=--1.47:4.53, thick] plot (\x,{-2*(\x)^2-22+12*(\x)});
\end{tikzpicture}
\caption{$(a=-2, p=3,q=-4)$}

\end{figure}
({ }{ }{ }) \(D=[-4,3]\) e \(I=[-4,3]\)

({ }{ }{ }) \(D=\mathbb{R}\) e \(I=]-\infty,-4]\)

({ }{ }{ }) \(D=[-5,5]\) e \(I=[-5,5]\)

({ }{ }{ }) \(D=[-4,3]\) e \(I=[-4,+\infty[\)

({ }{ }{ }) \(D=\mathbb{R}\) e \(I=\mathbb{R}\)

\item {} 
Em relação à função real \(f\) definida por \(f(x)=a(x-p)^2+q\) , caso \(a\) assuma apenas valores \textbf{positivos}, assinale quais das afirmações seguintes são verdadeiras:

({ }{ }{ }) O valor de \(p\) representa o maior valor que \(f\) pode assumir.

({ }{ }{ }) O valor de \(p\) representa o menor valor que \(f\) pode assumir.

({ }{ }{ }) O valor de \(q\) representa o maior valor que \(f\) pode assumir.

({ }{ }{ }) O valor de \(q\) representa o menor valor que \(f\) pode assumir.

({ }{ }{ }) A função \(f\), não tem valor máximo, mas tem valor mínimo.

({ }{ }{ }) A função \(f\), não tem valor mínimo, mas tem valor máximo.

({ }{ }{ }) A função f, tem valores de máximo e mínimo.

\item {} 
Em relação à função real \(f\) definida por \(f(x)=a(x-p)^2+q\) , caso \(a\) assuma apenas valores \textbf{negativos}, assinale quais das afirmações seguintes são verdadeiras:

({ }{ }{ }) O valor de \(p\) representa o maior valor que \(f\) pode assumir.

({ }{ }{ }) O valor de \(p\) representa o menor valor que \(f\) pode assumir.

({ }{ }{ }) O valor de \(q\) representa o maior valor que \(f\) pode assumir.

({ }{ }{ }) O valor de \(q\) representa o menor valor que \(f\) pode assumir.

({ }{ }{ }) A função \(f\), não tem valor máximo, mas tem valor mínimo.

({ }{ }{ }) A função \(f\), não tem valor mínimo, mas tem valor máximo.

({ }{ }{ }) A função f, tem valores de máximo e mínimo.

\item {} 
Ainda na função \(f\) ao assumirmos os valores de \(a=3\);  \(p=1\) e \(q=-2\), Assinale quais afirmações a seguir são verdadeiras.

({ }{ }{ }) O vértice da curva é \(V=(3,1)\).

({ }{ }{ }) O vértice da curva é \(V=(3,-2)\).

({ }{ }{ }) O vértice da curva é \(V=(1,-2)\).

({ }{ }{ }) O vértice da curva é \(V=(-2,1)\).

({ }{ }{ }) \(-2\), é o maior valor que a função f pode assumir.

({ }{ }{ }) \(3\), é o maior valor que a função f pode assumir.

({ }{ }{ }) \(1\), é o maior valor que a função f pode assumir.

({ }{ }{ }) \(-2\), é o menor valor que a função f pode assumir.

({ }{ }{ }) \(3\), é o menor valor que a função f pode assumir.

({ }{ }{ }) \(1\), é o menor valor que a função f pode assumir.

({ }{ }{ }) A concavidade da curva está voltada para cima, pois \(a>0\).

({ }{ }{ }) A concavidade da curva está voltada para cima, pois \(p>0\).

({ }{ }{ }) A concavidade da curva está voltada para cima, pois \(q<0\).

\end{enumerate}
\end{task}

\clearpage
\def\currentcolor{cor1}
\begin{answer}{Exercícios}
{\exerciselist
\begin{enumerate}
\item 
\begin{enumerate}
\item $f(x)=2x^2-20x+58$
\item $g(x)=-3x^2-12x-19$
\end{enumerate}

\item 
\begin{enumerate}
\item $f(x)=(x-4)^2-10$
\item $f(x)=-(x-4)^2+16$
\item $f(x)=2(x-0)^2+8$
\item $f(x)=2(x+2)^2-8$
\end{enumerate}

\item 
\begin{enumerate}
\item $f(x)=2(x-1)^2+3$
\item $f(x)=-(x-3)^2+1$
\item $f(x)=(x-1)^2-1$
\item $f(x)=-3(x-1)^2$
\end{enumerate}

\end{enumerate}
}{1}
\end{answer}
\exercise


\begin{enumerate}
\item Dadas as funções quadráticas \(f:\mathbb{R}\to\mathbb{R}\) a seguir na forma canônica, passe todas para forma polinomial representando-as graficamente.
\begin{enumerate}
\item {} 
\(f(x)=2(x-5)^2+8\)

\item {} 
\(g(x)=-3(x+2)^2-7\)

\end{enumerate}

\item Dadas as funções quadráticas \(f:\mathbb{R}\to\mathbb{R}\) a seguir na forma polinomial, passe todas para forma canônica representando-as graficamente.
\begin{enumerate}
\item {} 
\(f(x) = x^2-8x+6\)

\item {} 
\(f(x) = -x^2+8x\)

\item {} 
\(f(x) = 2x^2+8\)

\item {} 
\(f(x) = 2x^2+8x\)

\item {} 
\(f(x) = x^2+x+1\)

\end{enumerate}

\item Cada um dos gráficos a seguir representa uma função \(f:\mathbb{R}\to\mathbb{R}\). Exiba a forma canônica em cada caso.
\begin{multicols}{2}
\begin{enumerate}
\item 

\begin{tikzpicture}[scale=.7, yscale=.7, baseline=(current bounding box.north)]

\draw [->] (-3.5,0)--(4.5,0) node [above left] {$x$};
\draw [->] (0,-1.5)--(0,10) node [below right] {$y$};
\draw [dashed, color=secundario] (0,3)--(1,3)--(1,0);
\foreach \y in {3,5} \node [left] at (0,\y) {\y};
\foreach \x in {1} \node [below] at (\x,0) {\x};
\foreach \y in {-1,...,9}  \draw [] (0.1,\y) -- (-0.1,\y);
\foreach \x in {-3,...,4} \draw [] (\x,0.1) -- (\x,-0.1);
\node [below left,] at (0,0) {0};
\draw [color=\currentcolor!80, domain=-0.87072:2.87072, thick] plot (\x,{2*((\x)^2)-4*(\x)+5}) node [below right,] {$f$};
\node [ponto, color=secundario] at (0,5) {};
\node [ponto, color=secundario] at (1,3) {};
\node [below left,] at (1,3) {$V$};
\end{tikzpicture}


\item \begin{tikzpicture}[scale=.7, yscale=.7, baseline=(current bounding box.north)]

\draw [ ->] (-1,0)--(7,0) node [above left,scale=.3] {$x$};
\draw [->] (0,-9)--(0,2.5) node [below right,scale=.3] {$y$};
\draw [dashed,color=secundario] (0,1)--(3,1)--(3,0);
\foreach \y in {-8,1} \node [left,scale=.3] at (0,\y) {\y};
\foreach \x in {3} \node [below,scale=.3] at (\x,0) {\x};
\foreach \y in {-8,...,1}  \draw [scale=.3] (0.1,\y) -- (-0.1,\y);
\foreach \x in {1,...,6} \draw [scale=.2] (\x,0.1) -- (\x,-0.1);
\node [below left,scale=.3] at (0,0) {0};
\draw [color=\currentcolor!80, domain=-0.162277:6.162277, thick] plot (\x,{-((\x)^2)+6*(\x)-8}) node [above right, scale=.3] {$f$};
\node [ponto, color=secundario] at (0,-8) {};
\node [ponto, color=secundario] at (3,1) {};
\node [above,scale=.3] at (3,1) {$V$};
\end{tikzpicture}

\item 

\begin{tikzpicture}[scale=.7, yscale=.7, baseline=(current bounding box.north)]

\draw [->] (-2.5,0)--(5.5,0) node [above left] {$x$};
\draw [->] (0,-3.5)--(0,8) node [below right,] {$y$};
\draw [dashed, color=secundario] (0,-1)--(1,-1)--(1,0);
\foreach \y in {-1} \node [left,] at (0,\y) {\y};
\foreach \x in {1} \node [above,] at (\x,0) {\x};
\foreach \y in {-3,...,7}  \draw [] (0.1,\y) -- (-0.1,\y);
\foreach \x in {-2,...,5} \draw [] (\x,0.1) -- (\x,-0.1);
\node [below left,] at (0,0) {0};
\draw [color=\currentcolor!80, domain=-2:4, thick] plot (\x,{((\x)^2)-2*(\x)});
\node [right] at (1.5,-1) {$f$};
\node [ponto,color=secundario] at (1,-1) {};
\node [ponto,color=secundario] at (0,0) {};
\node [below right,] at (1,-1) {$V$};
\end{tikzpicture}

\item
\begin{tikzpicture}[scale=.7, yscale=.7, baseline=(current bounding box.north)]

\draw [->] (-3,0)--(5,0) node [above left,] {$x$};
\draw [->] (0,-7)--(0,4.5) node [below right,] {$y$};
\foreach \y in {-3} \node [left,] at (0,\y) {\y};
\foreach \x in {1} \node [below,] at (\x,0) {\x};
\foreach \y in {-6,...,4}  \draw (0.1,\y) -- (-0.1,\y);
\foreach \x in {-2,...,4} \draw (\x,0.1) -- (\x,-0.1);
\node [below left,] at (0,0) {0};
\draw [color=\currentcolor!80, domain=-0.52752:2.52752, thick] plot (\x,{-3*((\x)^2)+6*(\x)-3}) node [above right,] {$f$};
\node [ponto, color=secundario] at (0,-3) {};
\node [ponto, color=secundario] at (1,0) {};
\node [above,] at (1,0) {$V$};
\end{tikzpicture}

\end{enumerate}
\end{multicols}
\end{enumerate}

\arrange{O Gráfico da Função Quadrática}
\label{\detokenize{AF209-6:sec-funcao-quadratica-org-ideias-transformacoes}}\label{\detokenize{AF209-6::doc}}\label{\detokenize{AF209-6:organizando-as-ideias-os-parametros-da-forma-canonica-e-o-grafico-da-funcao-quadratica}}
A curva apresentada nas atividades anteriores foi descoberta e utilizada muito antes do surgimento do conceito de função. Os relatos históricos apontam que os gregos já utilizavam curvas obtidas por meio de cortes específicos em cones retos (denominadas \textit{cônicas}), porém foram os textos de Apolônio (262 a.C. — 194 a.C.) que definiram e explicitaram as propriedades destas curvas. Das \textit{cônicas}
definidas por Apolônio, a que estamos estudando, é denominada de \textbf{parábola}.

A abordagem dada à \textbf{parábola} durante muitos séculos foi apenas geométrica, a seguir apresentamos sua definição geométrica:

\begin{description}
\item[Foco]

Dado um ponto \(F\) e uma reta \(d\) que não contém \(F\), chamamos de \textbf{parábola} o conjunto dos pontos \(P\), no plano definido por \(F\) e \(d\), tais que \(P\) equidista de \(F\) e \(d\).  Onde denominamos \(F\) como \textit{foco} e \(d\) como \textit{reta diretiz}.

\begin{figure}[H]
\centering

\begin{tikzpicture}[scale=1.5]

\draw [domain=-3:3, thick] plot (\x,3/4);
\draw (0,-.2) -- (0,5) node [below right] {$l$};0.3
\node [above left] at (3,3/4) {$d$};
\draw [ dashed, color=\currentcolor!80, domain=-2:2, thick] plot (\x,{1+((\x)^2)});
\node [ponto, color=secundario] at (0,1) {};
\node [below right] at (-0.65,1.19) {$V$};
\node [below left] at (-0.1,1.7) {$F$};
\draw [color=secundario!90,fill= secundario!50, fill opacity=0.2]  (0.2,3.2) rectangle (0,3);
\draw [ color=secundario!90,fill= secundario!50, fill opacity=0.2]  (0.2,0.55) rectangle (0,3/4);
\draw [color=secundario!90,fill= secundario!50, fill opacity=0.2]  (1.6,.95) rectangle (1.4,3/4);
\draw [ color=secundario!90,fill= secundario!50, fill opacity=0.2]  (-1.6,.95) rectangle (-1.4,3/4);
\draw[secundario!80,-] (1.4,3/4)--(1.4,3)--(-1.4,3)--(-1.4,3/4);
\draw[secundario!70,-] (1.4,3)--(0,1.25)--(-1.4,3);
\node [ponto, color=secundario] at (0,1.25) {};
\node [ponto, color=secundario] at (1.4,3) {} ;
\node [above right] at (1.4,3) {$P$};
\node [ponto, color=secundario] at (-1.4,3) {};
\node [above right] at (-2.2,3) {$Q$};
\node [ponto, color=secundario] at (0,3) {};
\node [above left] at (0,3) {$M$};
\draw [color=secundario!70] (0.9,2.9)--(0.9,3.1);
\draw [color=secundario!70] (-0.9,2.9)--(-0.9,3.1);
\draw [color=secundario!70] (0.8,2.9)--(0.8,3.1);
\draw [color=secundario!70] (-0.8,2.9)--(-0.8,3.1);
\draw [color=secundario!70] (1.3,1.8)--(1.5,1.8);
\draw [color=secundario!70] (0.6,2.1)--(0.8,2.1);
\draw [color=secundario!70] (-0.6,2.1)--(-0.8,2.1);
\draw [color=secundario!70] (-1.3,1.8)--(-1.5,1.8);
\end{tikzpicture}
\caption{Parábola como Lugar Geométrico}
\end{figure}


Ou seja,
\(P\in\) \textbf{parábola} \(\equiv d(P,F)=d(P,d)\)
\end{description}

Agora vamos mostrar que essa definição atende à função \(f:\mathbb{R}\to\mathbb{R}\) definida por \(f(x)=x^2\).

Na figura a seguir destacamos, além do gráfico da função \(f\), os pontos \(F=(0,\frac{1}{4})\) e a reta \(d:y=-\frac{1}{4}\)

\begin{figure}[H]
\centering

\begin{tikzpicture}[scale=1.35]

\draw [help lines,very thin, secundario!30, step=.5] (-4,-1.2) grid (4,4.5);
\draw [help lines, dotted, very thin, secundario!70, step=.25] (-4,-1.2) grid (4,4.5);
\draw [->] (-4,0) -- (4,0);
\draw [->] (0,-1.2) -- (0,4.5);
\node [above] at (3.9,0.1) {$x$};
\node [right] at (0,4.2) {$y$};
\foreach \y in {-1,-0.5, 0, 0.5, ..., 3.5,4} \node [above left] at (0,\y) {\y};
\foreach \x in {-3.5,-3, -2.5, -2,-1.5, -1, -0.5, 0.5,1,...,3.5,4} node [below left] at (\x,0) {\x};
\draw [color=\currentcolor!80, domain=-2.1:2.1, thick] plot (\x,{(\x)^2});
\node [above] at (2.3,3.6) {$f$};
\node [ponto, color=destacado] at (0,0.25) {};
\node [right] at (0,0.3) {$F$};
\draw [color=atento, domain=-4:4, thick] plot (\x,{-0.25});
\node [below] at (-3.4,-0.25) {$d$};
\node [fill=white, rectangle, align=left, scale=.85] at (-2.9,2.75) {Fun\c c\~ ao $f(x)=x^2$ \\ Ponto $F$= (0,0.25) \\ Reta $d$: y = -0.25};
% \node [below] at (-3.05,3.4) {};
% \node [below] at (-3,3.1) { };
% \node [below] at (-3.05,2.8) {};
\end{tikzpicture}
\caption{Parábola, Foco e diretriz}
\end{figure}


Sabemos que todos ponto pertencentes à \(f\) são do tipo \(P=(x,x^2)\) para que \(f\) satizfaça a definição anterior, temos que para todo \(P\) pertencente à \(f\), a distância de \(P\) ao foco \(F=(0,\frac{1}{4})\) seja a mesma distância de \(P\) à reta diretriz \(d:y=-\frac{1}{4}\), e isto é fato, veja:

\(d(P,F)=\sqrt{(x-0)^2+(x^2-\frac{1}{4})^2}=\sqrt{x^2+(x^2-\frac{1}{4})^2}\)


\begin{figure}[H]
\centering

\begin{tikzpicture}[scale=1.75, every node/.style={scale=1.75}]
\draw [help lines, thin, secundario!30, step=.5] (-2.5,-1.2) grid (2.6,3);
\draw [help lines, dotted, thin, secundario!70, step=.25] (-2.5,-1.2) grid (2.6,3);
\draw [->] (-2.5,0) -- (2.5,0);
\draw [->] (0,-1.2) -- (0,3);
\foreach \y in {-1,-0.5, 0.5,1, ..., 2.5} \node [left,scale=.5] at (0,\y) {\y};
\foreach \x in { -2,-1.5, -1, -0.5, 0.5,1,...,2.5} \node [below,scale=.5] at (\x,-0.01) {\x};
\node [below left,scale=.5] at (0,0) {0};
\draw [color=\currentcolor!80, domain=-1.7:1.7, thick] plot (\x,{(\x)^2});
\draw [color=atento, domain=-2.5:2.5, thick] plot (\x,{-0.25});
\draw [color=secundario, dashed] (1.26,1.53) -- (0,1.53);
\draw [color=secundario, dashed] (1.25,0.25) -- (1.25,0);
\draw [color=secundario] (0,0.25)--(1.25,0.25);
\draw [color=secundario] (1.25,0.25)--(1.25,1.5);
\draw [color=secundario] (1.25,1.5)--(0,0.25);
\node [ponto, color=destacado] at (0,0.25) {};
\node [ponto, color=atento] at (1.25,0.25) {};
\node [ponto, color=black, fill = \currentcolor!80, fill opacity=.8] at (1.26,1.53) {};
\node [left,scale=.5] at (0,0.3) {$F$};
\node [right,scale=.5] at (1.25,0.3) {$B$};
\node [left,scale=.5] at  (1.25,1.6) {$P$};
\node [below,scale=.5] at (1.25,0) {$x$};
\node [above right,scale=.5] at  (0,1.53) {$x^2$};
\node [above right,scale=.5] at  (0.3,0.8) {$PF$};
\node [ above right,scale=.5] at  (0.45,0.01) {$x-0$};
\node [above right,scale=.5] at  (1.3,0.8) {$x^2-$ $\frac{1}{4}$};
\draw [color=secundario] (3,0.25)--(4.25,0.25);
\draw [color=secundario] (4.25,0.25)--(4.25,1.5);
\draw [color=secundario] (4.25,1.5)--(3,0.25);
\node [left,scale=.5] at (3,0.3) {$F$};
\node [right,scale=.5] at (4.25,0.3) {$B$};
\node [left,scale=.5] at  (4.25,1.6) {$P$};
\node [above right,scale=.5] at  (3.3,0.8) {$PF$};
\node [above right,scale=.5] at  (3.6,0) {$x$};
\node [above right,scale=.5] at  (4.3,0.8) {$x^2-$ $\frac{1}{4}$};
\node [ponto, color=destacado] at (3,0.25) {};
\node [ponto, color=atento] at (4.25,0.25) {};
\node [ponto, color=black, fill = \currentcolor!80, fill opacity=.8] at (4.26,1.53) {};
\node [above right,scale=.5] at  (3,-0.3) {$PF^2 = x^2 + (x^2-$ $\frac{1}{4}$)};
\end{tikzpicture}
\caption{Distância de P a F}
\end{figure}


Por outro lado:

$$d(P,d)=x^2+\frac{1}{4}$$
\begin{figure}[H]
\centering

\begin{tikzpicture}[scale=1.75, every node/.style={scale=1.75}]
\draw [help lines, secundario!30, step=.5] (-1.8,-1) grid (2.3,2.8);
\draw [help lines, dotted, secundario!70, step=.25] (-1.8,-1) grid (2.3,2.8);
\draw [->] (-1.8,0) -- (2.3,0);
\draw [->] (0,-1) -- (0,2.8);
\foreach \y in { 0.5,1, ..., 2.5} \node [left,scale=.5] at (0,\y) {\y};
\foreach \x in { ,-1.5, -1, -0.5, 0.5,1,...,2} \node [below,scale=.5] at (\x,-0.01) {\x};
\node [below left,scale=.5] at (0,0) {0};
\draw [color=\currentcolor!80, domain=-1.65:1.65, thick] plot (\x,{(\x)^2});
\draw [color=atento, domain=-1.8:2.3, thick] plot (\x,{-0.25});
\draw[ color=secundario, dashed] (1.26,1.53) -- (0,1.53);
\draw [color=secundario] (1.26,1.53) -- (1.25,-0.25);
\node [ponto, color=destacado] at (0,0.25) {};
\node [ponto, color=black, fill = \currentcolor!80, fill opacity=.8] at (1.26,1.53) {};
\node [left,scale=.5] at (0,0.3) {$F$};
\node[right,scale=.5] at (1.25,0.8) {$PA$};
\node [left,scale=.5] at  (1.25,1.6) {$P$};
\node [above right,scale=.5] at  (1.25,0) {$x$};
\node [above right,scale=.5] at  (0,1.53) {$x^2$};
\node [below,scale=.5] at  (-0.1,-0.25) {$-\frac{1}{4}$};
\node [below left,scale=.5] at  (0,-0.65) {$-0.5$};
\node[above,scale=.5] at (3, 1.5) {$PA=x^2-$($-\frac{1}{4}$)};\node[above,scale=.5] at (3, 1) {$PA=x^2+$($\frac{1}{4}$)};
\node[above,scale=.5] at (3,-0.4) {$d:y=-$($\frac{1}{4}$)};
\end{tikzpicture}
\caption{Distância de $P$ à $d$}
\end{figure}


Como queremos \(d(P,F)=d(P,d)\), temos:

$$\sqrt{x^2+(x^2-\frac{1}{4})^2}=x^2+\frac{1}{4}$$

Elevando ambos os membros ao quadrado, temos:

\begin{align*}
(\sqrt{x^2+(x^2-\frac{1}{4})^2})^2&=(x^2+\frac{1}{4})^2\\
x^2+(x^2-\frac{1}{4})^2&=(x^2+\frac{1}{4})^2
\end{align*}


Desenvolvendo teremos:

\begin{align*}
x^2+(x^4-\frac{1}{2}x^2+\frac{1}{16})=x^4+\frac{1}{2}x^2+\frac{1}{16}\\
x^4+(x^2-\frac{1}{2}x^2)+\frac{1}{16}=x^4+\frac{1}{2}x^2+\frac{1}{16}
\end{align*}

E finalmente:

$$x^4+\frac{1}{2}x^2+\frac{1}{16}=x^4+\frac{1}{2}x^2+\frac{1}{16}$$

Isso nos mostra que a curva descrita no gráfico da função quadrática \(f:\mathbb{R}\to\mathbb{R}\) definida por \(f(x)=x^2\) é realmente uma \textbf{parábola}.

Agora utilizaremos os conceitos abordados na atividade \hyperref[\detokenize{AF209-5:ativ-funcao-quadratica-graf-curva}]{\textit{O gráfico e a forma canônica}}.
\begin{enumerate}
\item {} 
Observamos que a variação de \(a\) na curva \(y=ax^2\) faz com que a concavidade da curva fique mais aberta quando \(a\) se aproxima de zero ou mais fechada quando \(a\) se afasta de zero, e também que o sinal de \(a\) indica se a concavidade aponta para cima (\(a>0\)) ou para baixo (\(a<0\)). É facil demonstrar que o gráfico de toda função real \(f\) dada na forma \(f(x)=ax^2\) é uma \textit{parábola}. Note que o texto anterior, provamos para \(a=1\). Para generalizarmos, basta assumirmos o foco como \(F=(0,\frac{1}{4a})\) e reta diretriz como a reta horizontal \(y=-\frac{1}{4a}\).

\item {} 
Além disso vimos que as variações dos termos \(p\) e \(q\) da forma canônica \(f(x)=a(x-p)^2+q\) provocam as \textit{translações} horizontais e verticais respectivamente. Como as translaçoes não deformam as figuras transladadas, podemos inferir que os gráficos todas as funções reais dadas por \(f(x)=a(x-p)^2+q\) são parábolas. Cujo vértice é dado por \(V=(p,q)\).

\item {} 
Portanto toda função quadrática apresentada na sua forma canônica \(f(x)=a(x-p)^2+q\) e também em sua forma polinomial \(f(x)=ax^2+bx+c\) têm gráficos parabólicos.

\end{enumerate}

\textbf{Observação}

Toda parábola com reta diretriz paralela ao eixo das abscissas será uma função quadrática.
\begin{figure}[H]
\centering

\begin{tikzpicture}[scale=.75, every node/.style={scale=2}]

\draw [help lines, very thin, secundario!30, step=2] (-3,-1.5) grid (7,10);
\draw [help lines, dotted, very thin, secundario!70, step=.25] (-3,-1.5) grid (7,10);
\draw [->] (-3,0) -- (7,0);
\draw [->] (0,-1.5) -- (0,10);
\foreach \y in { -1,1,2,...,9} \node [left,scale=.5] at (0,\y) {\y};
\foreach \x in {-2,-1,1,2,3,...,6} \node [below,scale=.5] at (\x,-0.01) {\x};
\node [below left,scale=.5] at (0,0) {0};
\draw [color=\currentcolor!80, domain=-1.605551275:5.605551275, thick] plot (\x,{0.5*(\x)^2-2*\x+5.5});
\draw [color=atento, domain=-3:7, thick] plot (\x,{3});
\node [ponto, color=destacado] at (2,4) {};
\node [above,scale=.5] at (2,4) {$F$};
\node [above,scale=.5] at (-2.3,2.4) {$d$};
\node [scale=.4, align=left, fill=white, rectangle] at (2.1,1.5) {Foco $F$ = (2,4) \\ Diretriz $d: y = 3$};

\end{tikzpicture}
\caption{São funções de \(x\) em \(y\)}
\end{figure}


\begin{figure}[H]
\centering

\begin{tikzpicture}[scale=.75, every node/.style={scale=2}, yscale=.75]

\draw [help lines, very thin, secundario!30, step=2] (-5,-8.8) grid (7,4.5);
\draw [help lines, dotted, very thin, secundario!70, step=.25] (-5,-8.8) grid (7,4.5);
\draw [->] (-5,0) -- (7,0);
\draw [->] (0,-8.8) -- (0,4.5);
\foreach \y in { -8,-6,...,-2,2,4} \node [left,scale=.5] at (0,\y) {\y};
\foreach \x in {-4,-2,2,4,6} \node [below,scale=.5] at (\x,-0.01) {\x};
\node [below left,scale=.5] at (0,0) {0};
\draw [color=\currentcolor!80, domain =-1.96:5.96, thick] plot (\x,{-0.5*(\x)^2+2*\x-3});
\draw [color=atento,domain=-5:7, thick] plot (\x,{1});
\node [ponto, color=destacado] at (2,0) {};
\node [above,scale=.5] at (2,0) {$F$};
\node [above,scale=.5] at (-4.3,1) {$d$};
\node [above,scale=.4, rectangle, fill=white, align=left] at (2.3,-7.5) {Foco $F$ = (2,0) \\ Diretriz $d: y = 1$};
\end{tikzpicture}
\end{figure}

Note que se esta condição não for aceita, o gráfico apresentado, não será sequer uma função real de \(x\) em \(y\), observe nas figuras a seguir:


\begin{figure}[H]
\centering

\begin{tikzpicture}[scale=.55, every node/.style={scale=1.75}]

\draw [help lines, very thin, secundario!30, step=2] (-3.8,-4.5) grid (11.3,6.5);
\draw [help lines, dotted, secundario!70, step=.25] (-3.8,-4.5) grid (11.3,6.5);
\draw [->] (-3.8,0) -- (11.3,0);
\draw [->] (0,-4) -- (0,7);
\foreach \y in { -4,-3,-2,-1,1,2,3,...,6} \node [left,scale=.5] at (0,\y) {\y};
\foreach \x in {-3,-2,-1,1,2,3,...,11} \node [below,scale=.5] at (\x,-0.01) {\x};
\node [below left,scale=.5] at (0,0) {0};
\draw [color=\currentcolor!80, domain=-3.361547263:3.361547263, rotate around={270:(0,0)}, thick] plot (\x,{(\x)^2});
\draw [color=atento, very thick, domain=-4.5:6.5, thick] plot ({-0.5},\x);
\node [ponto, color=destacado] at (9,3) {};
\node [ponto, color=destacado] at (9,-3) {};
\node [above,scale=.5] at (9,3) {$(9,3)$};
\node [below,scale=.5] at (9,-3) {$(9,-3)$};
\node [below,scale=.5] at (1.3,1) {$c$};

\node [fill=white, rectangle, align=left, scale=.4] at (-2.5,3){C\^onica \\ $cy^2=x$};
\node [below,scale=.4] at (-2,5) {};
\node [below,scale=.4] at (-2,4.6) { };
\end{tikzpicture}
\caption{Não é função de \(x\) em \(y\)}
\end{figure}

Se a figura anterior, representar o gráfico da relação \(\phi:\mathbb{R_x+}\to\mathbb{R_y}\) dada por \(x=y^2\), temos que \(\phi\) não é função, já que a maioria dos pontos do domínio apresentam duas imagens, na figura destacamos apenas as duas imagens de \(x=9\).

Porém, \textbf{mesmo não sendo comum}, se assumirmos a relação \(\phi:\mathbb{R_y}\to\mathbb{R_x+}\) dada por \(x=y^2\), temos que \(\phi\) é função, só que de \(y\) em \(x\).

Já no caso da figura a seguir, o gráfico, não representa uma função de \(x\) em \(y\) nem de \(y\) em \(x\).

\begin{figure}[H]
\centering

\begin{tikzpicture}[scale=.6, every node/.style={scale=1.75}]

\draw [help lines, secundario!30, step=2] (-6,-2.5) grid (10,12);
\draw [help lines, dotted, secundario!70, step=.25] (-6,-2.5) grid (10,12);
\draw [->] (-6,0) -- (10,0);
\draw [->] (0,-2.5) -- (0,12);
\foreach \y in { -2,-1,1,2,...,11} \node [left,scale=.5] at (0,\y) {\y};
\foreach \x in {-5,...,-2,-1,1,2,3,...,9} \node [below,scale=.5] at (\x,-0.01) {\x};
\node [below left,scale=.5] at (0,0) {0};
\draw [color=\currentcolor!80, rotate around={315:(2,2)}, domain=-3.22:6.58, thick] plot (\x,{(2*(\x)^2-8*\x+15.31)/5.66});
\draw [color=atento,domain=-6:4.5,rotate around={0:(2,2)}, thick] plot (\x,{2-\x});
\node [ponto, color=destacado, rotate around={0:(2,2)}] at (2,2) {};
\node [ponto, color=destacado, rotate around={0:(2,2)}] at (3,1) {};
\node [ponto, color=destacado, rotate around={0:(2,2)}] at (3,9) {};
\draw [color=white,fill=white, fill opacity=1]  (-5.4,8) rectangle (-0.8,11);
\node [above,scale=.5] at (2,2) {$F$};
\node [above,scale=.5] at (3,1) {$D$};
\node [above,scale=.5] at (3,9) {$E$};
\node [above,scale=.5] at (-5.25,6) {$d$};
\node [align=center,scale=.4, rectangle, fill=white] at (-3.3,9.5) {C\^onica $c: x^2-2xy+$ \\ $+y^2-4x-4y=-12$\\Ponto D =(3,1)\\E =(3,9)\\F =(2,2)\\Reta $d: x+y=2$};
\end{tikzpicture}
\caption{Não é função de \(x\) em \(y\)}
\end{figure}


Ou seja, para que uma parábola seja o gráfico de uma função quadrática de \(\mathbb{R_x}\to\mathbb{R_y}\) a condição necessária é que sua reta diretriz seja paralela ao eixo das abscissas.

\clearpage
\def\currentcolor{session2}
\begin{objectives}{Fugindo pela parábola}
{
\begin{itemize}
\item Construir estratégia de resolução que dependa da identificação do que não é solução antes da conclusão.

\item {} 
Aplicar a definição geométrica de parábola numa situação prática;
\end{itemize}
}{1}{1}
\end{objectives}
\marginpar{\vspace{-2em}}
\begin{sugestions}{Fugindo pela parábola}
{
Problemas geométricos que envolvem distâncias de um ponto a outro são, em geral, tratados com o recurso das circunferências e suas propriedades. Aqui desejamos ampliar as ferramentas de solução de problemas desse tipo, ajudando o estudante a incorporar como método de solução de problemas de distâncias, a parábola.
\begin{itemize}
\item {} 
Comece a resolver os pontos que parecem óbvios.

\item {} 
Quando o estudante tender a achar que todos são óbvios, sugira que ele utilize régua ou compasso para tentar comprovar a sua intuição.

\item {} 
Para motivar a solução dada pelos autores, sugira outros pontos, próximos da interseção da parábola com a linha contínua na parte superior da imagem.

\item {} 
Questione se existe uma região onde a resposta seria: “Tanto faz”! Estimule os estudantes a exibir ou descrever essa região. Se necessário, fornaça-lhe como opções as definições de circunferência, reta mediatriz e parábola.

\end{itemize}
}{1}{1}
\end{sugestions}
\marginpar{\vspace{-1em}}
\begin{answer}{Fugindo pela parábola}
{
Traçando uma reta no limite da região ‘H’ e usando o ponto ‘Q’, pode-se traçar os pontos do salão que equidistam de ‘H’ ou ‘Q’, que é a parábola. Assim, as melhores chances de fuga se dão para:
\begin{figure}[H]
\centering

\begin{tikzpicture}[every node/.style={scale=2.5}, scale=.75]

\draw [color=black,fill=black, fill opacity=1]  (0,0) rectangle (10,6.02);
\draw[color=secundario, fill=secundario, fill opacity=1] (0,1) rectangle (10,2.4);
\draw[color=secundario, fill=secundario, fill opacity=1] (0,2) rectangle (9.7,3);
\draw[color=secundario, fill=secundario, fill opacity=1] (0,3) rectangle (10,3.8);
\draw[color=secundario, fill=secundario, fill opacity=1] (0.3,3.8) rectangle (10,4.3);
\draw[color=secundario, fill=secundario, fill opacity=1] (0,4.3) rectangle (0.5,4.8);
\draw[color=secundario, fill=secundario, fill opacity=1] (0.5,4.3) rectangle (10,4.8);
\path [fill=secundario, fill opacity=1]  (0,4.8) to  (4,6) -- (4.5,6)--(4.5,5.4)-- (4.8,5.4) ;
\path [fill=secundario, fill opacity=1]  (10,4.8)  to  (6,6) -- (5.5,6) -- (5.5,5.8) --  (5.5,5.4);
\path [fill=secundario, fill opacity=1]  (0,4.8) to  (10,4.8) -- (5.3,5.43) --  (4.8,5.43);
\draw[color=secundario, fill=secundario, fill opacity=1] (4.5,5.4) rectangle (5.5,5.9);
\draw [color=primario, dashed] (0,1) -- (10,1);
\draw [color=primario] (0,4.8) -- (4,6) -- (4.5,6) -- (4.5,5.8);
\draw [color=primario] (10,4.8) -- (6,6) -- (5.5,6) -- (5.5,5.8);
\draw [color=primario]  (5.5,5.6) -- (5.5,5.4) -- (5.3,5.4);
\draw [color=primario]  (4.5,5.6) -- (4.5,5.4) -- (4.8,5.4);
\draw [color=primario]  (0,4.3) -- (0.3,4.3) -- (0.3,4.1);
\draw [color=primario]  (0,3.8) -- (0.3,3.8) -- (0.3,4);
\draw [color=primario]  (10,3) -- (9.7,3) -- (9.7,2.8);
\draw [color=primario]  (10,2.4) -- (9.7,2.4) -- (9.7,2.6);
\draw [color=black, fill opacity=1, fill=black] (5,5.65) circle (0.15);
\node [ponto, color=destacado] at (8,4) {};
\node [ponto, color=destacado] at (7.5,2.5) {};
\node [ponto, color=destacado] at (5,2.8) {};
\node [ponto, color=destacado] at (3,4) {};
\node [ponto, color=destacado] at  (2.5,5)  {};
\node [ponto, color=destacado] at  (2.2,2)  {};
\node [below right, color=white,scale=.4] at (5,5.4) {Q};
\node [above right, color=white,scale=.4] at (0.5,4.3) {S};
\node [above left, color=white,scale=.4] at (10,3) {E};
\node [above left, color=white,scale=.4] at (8,4) {1};
\node [above left, color=white,scale=.4] at (7.5,2.5) {2};
\node [below left, color=white,scale=.4] at (5,2.8) {3};
\node [above left, color=white,scale=.4] at (3,4) {4};
\node [above right, color=white,scale=.4] at (2.5,5) {5};
\node [above left, color=white,scale=.4] at (2.2,2) {6};
\node [right, color=white,scale=.4] at (4,0.5) {Regi\~ao H};              \draw[color=atento, very thick] (0,6) .. controls  (2.2,2.6) and  (8,2.6) .. (10,6);
\draw[color=atento, very thick] (-1,1)--(11,1);
\end{tikzpicture}
\end{figure}


\begin{multicols}{3}
\centering

\(1)\) Região ‘H’.

\(2)\) Região ‘H’.

\(3)\) Região ‘H’.

\(4)\) Região ‘Q’

\(5)\) Região ‘Q’

\(6)\) Região ‘H’
\end{multicols}

}{1}
\end{answer}

\practice{O Gráfico da Função Quadrática}
\label{\detokenize{AF209-6:sec-funcao-quadratica-praticando-parabola-lg}}\label{\detokenize{AF209-6:praticando}}
\phantom{a}
\vspace{-1em}
\vspace{-3\parskip}

\begin{task}{Fugindo pela parábola}
Num jogo eletrônico em que você controla um oficial militar infiltrado. Dentre as fases de treinamento tático há uma que exibe um salão vigiado por câmeras.

\begin{figure}[H]
\centering
\capstart

\noindent\includegraphics[width=170bp]{{MGS_1998_PS_Espreita}.jpg}
\caption{Imagem de divulgação.}\label{\detokenize{AF209-6:id2}}\end{figure}

Como as câmeras fazem movimento de vai e vem, é possível atravessar o salão sem ser detectado, e esse é o objetivo desta fase. A imagem a seguir mostra a vista de cima desta fase.

\begin{figure}[H]
\centering

\begin{tikzpicture}[every node/.style={scale=2.5}, scale=.9]

\draw [color=black,fill=black, fill opacity=1]  (0,0) rectangle (10,6.02);
\draw[color=secundario, fill=secundario, fill opacity=1] (0,1) rectangle (10,2.4);
\draw[color=secundario, fill=secundario, fill opacity=1] (0,2) rectangle (9.7,3);
\draw[color=secundario, fill=secundario, fill opacity=1] (0,3) rectangle (10,3.8);
\draw[color=secundario, fill=secundario, fill opacity=1] (0.3,3.8) rectangle (10,4.3);
\draw[color=secundario, fill=secundario, fill opacity=1] (0,4.3) rectangle (0.5,4.8);
\draw[color=secundario, fill=secundario, fill opacity=1] (0.5,4.3) rectangle (10,4.8);
\path [fill=secundario, fill opacity=1]  (0,4.8) to  (4,6) -- (4.5,6)--(4.5,5.4)-- (4.8,5.4) ;
\path [fill=secundario, fill opacity=1]  (10,4.8)  to  (6,6) -- (5.5,6) -- (5.5,5.8) --  (5.5,5.4);
\path [fill=secundario, fill opacity=1]  (0,4.8) to  (10,4.8) -- (5.3,5.43) --  (4.8,5.43) ;
\draw[color=secundario, fill=secundario, fill opacity=1] (4.5,5.4) rectangle (5.5,5.9);
\draw [color=\currentcolor!80, dashed] (0,1) -- (10,1);
\draw [color=\currentcolor!80] (0,4.8) -- (4,6) -- (4.5,6) -- (4.5,5.8);
\draw [color=\currentcolor!80] (10,4.8) -- (6,6) -- (5.5,6) -- (5.5,5.8);
\draw [color=\currentcolor!80]  (5.5,5.6) -- (5.5,5.4) -- (5.3,5.4);
\draw [color=\currentcolor!80]  (4.5,5.6) -- (4.5,5.4) -- (4.8,5.4);
\draw [color=\currentcolor!80]  (0,4.3) -- (0.3,4.3) -- (0.3,4.1);
\draw [color=\currentcolor!80]  (0,3.8) -- (0.3,3.8) -- (0.3,4);
\draw [color=\currentcolor!80]  (10,3) -- (9.7,3) -- (9.7,2.8);
\draw [color=\currentcolor!80]  (10,2.4) -- (9.7,2.4) -- (9.7,2.6);
\draw [color=black, fill opacity=1, fill=black] (5,5.65) circle (0.15);
\node [ponto, color=destacado] at (8,4) {};
\node [ponto, color=destacado] at (7.5,2.5) {};
\node [ponto, color=destacado] at (5,2.8) {};
\node [ponto, color=destacado] at (3,4) {};
\node [ponto, color=destacado] at  (2.5,5)  {};
\node [ponto, color=destacado] at  (2.2,2)  {};
\node [below right, color=white,scale=.4] at (5,5.4) {Q};
\node [above right, color=white,scale=.4] at (0.5,4.3) {S};
\node [above left, color=white,scale=.4] at (10,3) {E};
\node [above left, color=white,scale=.4] at (8,4) {1};
\node [above left, color=white,scale=.4] at (7.5,2.5) {2};
\node [below left, color=white,scale=.4] at (5,2.8) {3};
\node [above left, color=white,scale=.4] at (3,4) {4};
\node [above right, color=white,scale=.4] at (2.5,5) {5};
\node [above left, color=white,scale=.4] at (2.2,2) {6};
\node [right, color=white,scale=.4] at (4,0.5) {Regi\~ao H};
\end{tikzpicture}
\end{figure}

A região em cinza é uma região que, em algum momento, pode ser enxergado por câmera durante o movimento de vai e vem. A linha verde contínua representa alguma barreira intransponível; já as linhas tracejadas podem ser ultrapassadas pelo personagem para se abrigar das câmeras e terminar a fase. Em ‘E’ o personagem entra no cenário essa passagem se fecha, em ‘S’ ele sai e vence a fase.

Os pontos em vermelho são posições possíveis para o personagem que, percebendo a proximidade do olhar de alguma das câmeras deve correr e se esconder numa região em preto. Sendo assim, para cada posição do personagem, diga para onde ele deve correr: Região horizontal ‘H’ ou Região quadrada ‘Q’.
\end{task}

\begin{research}{}

A definição geométrica da \textbf{parábola} apresentada inicialmente pode ser associada à referência histórica de corte de cone reto, para isso acesse o link do geogebra a seguir e mantenha os valores de \(t\) e \(a\), variando apenas os valores de \(s\).

O cone e as cônicas - excentricidade da parábola (\url{https://ggbm.at/Z38MMkqV})

Para demonstrar que toda parábola é gerada por cortes específicos em cones retos, sugerimos uma leitura das páginas \(13\), \(14\) e \(15\) da dissertação de Monteiro (2014).
\end{research}

\cleardoublepage
\def\currentcolor{session1}
\begin{objectives}{Aumento na Passagem}
{
\begin{itemize}
\item Reconhecer outras possibilidades de escala nos eixos
cartesianos para a representação gráfica de funções.
\item Exercitar a modelagem algébrica problemas.
\item Reconhecer função quadrática e seu gráfico.
\item Inferir domínio e imagem própria da situação problema.
\item Reconhecer as vantagens do uso da forma do vértice para a determinação dos valores de máximo ou mínimo.
\item Inferir sobre outros pontos notáveis na função quadrática: zeros da função.
\end{itemize}
}{1}{1}
\end{objectives}
\marginpar{\vspace{-2.25em}}
\begin{sugestions}{Aumento na Passagem}
{
Prezado colega esta atividade tem como objetivo aplicar o conceito de otimização em função quadrática num contexto econômico, chamando atenção para o aluno de:

\begin{itemize}
\item As vantagens e desvantagens de se trabalhar num plano cartesiano cujos eixos estão em escalas distintas.
\item Guiá-lo para uma modelagem algébrica da situação.
Identificar se a relação encontrada é uma função quadrática e se o gráfico apresentado é de uma parábola.
\item Fazer uma discussão a respeito do domínio e da imagem da função levando em consideração a modelagem da situação.
\item Reforçar a utilização da passagem da forma polinomial para a forma canônica, apontando assim de maneira direta o faturamento máximo e o aumento que irá gerar o faturamento máximo.
\item Apresentar em que pontos a parábola intersecta os eixos coordenados, levando-os a fazer inferências sobre a utilização das coordenadas desses pontos no contexto do problema.
\end{itemize}

Sugerimos que o professor além de fazer a atividade antes de aplicá-la, leia com atenção as respostas das atividades, nela o colega encontrará sugestões que o auxiliarão na condução dessa atividade na sua sala de aula.
}{1}{1}
\end{sugestions}
\begin{answer}{Aumento na Passagem}
{
\begin{enumerate}
\item Novo preço será de \(40+2=42\) reais; A nova quantidade de passageiros será de \(1.200-10 \times 2=1.200-20=1.180\) passageiros; O novo faturamento será de \(42 \times 1180=49.560\) reais. No caso do aumento ser de doze reais teremos na ordem: R\$ $52{,}00$ de novo preço; \(1080\) passageiros; E  R\$ $56.160{,}00$ de faturamento.


\item
\adjustbox{valign=t}
{
\setlength\tabcolsep{2.5pt}
\begin{tabular}{|f|f|>$e{.25\linewidth}<$|>$e{.25\linewidth}<$|}
\hline
$\tcolor{Aumento em reais}$ & $\tcolor{Novo preço}$ & $\tcolor{Nova quantidade de passageiros}$ & $\tcolor{Faturamento em reais}$ \tabularnewline
\hline 
0 & 40 & 1.200 & 48.000 \tabularnewline
\hline
10 & 50 & 1.100 & 55.000 \tabularnewline
\hline
20 & 60 & 1.000 & 60.000 \tabularnewline
\hline
30 & 70 & 900 & 63.000 \tabularnewline
\hline
40 & 80 & 800 & 64.000 \tabularnewline
\hline
50 & 90 & 700 & 63.000 \tabularnewline
\hline
60 & 100 & 600 & 60.000 \tabularnewline
\hline
70 & 110 & 500 & 55.000 \tabularnewline
\hline
80 & 120 & 400 & 48.000 \tabularnewline
\hline
90 & 130 & 300 & 39.000 \tabularnewline
\hline
100 & 140 & 200 & 28.000 \tabularnewline
\hline
110 & 150 & 100 & 15.000 \tabularnewline
\hline 
120 & 160 & 0 & 0 \tabularnewline
\hline
130 & 170 & -100 & -17.000 \tabularnewline
\hline
\end{tabular}
}
\end{enumerate}
}{0}
\end{answer}
\clearmargin
% \marginpar{\vspace{.5em}}
\begin{answer}{Aumento na Passagem}
{
\begin{enumerate}\setcounter{enumi}{2}
\item \adjustbox{valign=t}
{
\resizebox{.95\linewidth}{!}
{
\begin{tikzpicture}[every node/.style={scale=3}, scale=.75]
\draw [help lines, secundario!30, dashed] (0,0) grid (13,9);
\draw [->] (-1,0) -- (13,0);
\draw [->] (0,-1) -- (0,9);
\foreach \y in {1,2,3,...,7} \node [left,scale=.3] at (0,\y) {\y0 000};
\foreach \x in  {1,2,3,...,12} \node [below,scale=.3] at (\x,-0.01) {\x0};
\node [below left,scale=.3] at (0,0) {0};
\node [ponto, color=primario, scale=2] at (0,4.7) {};
\node [ponto, color=primario, scale=2] at (1,5.5) {};
\node [ponto, color=primario, scale=2] at (2,6) {};
\node [ponto, color=primario, scale=2] at (3,6.3) {};
\node [ponto, color=primario, scale=2] at (4,6.5) {};
\node [ponto, color=primario, scale=2] at (5,6.3) {};
\node [ponto, color=primario, scale=2] at (6,6) {};
\node [ponto, color=primario, scale=2] at (7,5.5) {};
\node [ponto, color=primario, scale=2] at (8,4.7) {};
\node [ponto, color=primario, scale=2] at (9,3.8) {};
\node [ponto, color=primario, scale=2] at (10,2.7) {};
\node [ponto, color=primario, scale=2] at (11,1.5) {};
\node [ponto, color=primario, scale=2] at (12,0.05) {};
\node [below, align = center, scale=.3] at (6,-1) {Gr\'afico B};
\node [below left , align = center,scale=.3] at (0,9) {Faturamento \\ em Reais};
\node [below right , align = center,scale=.3] at (13,0) {Aumento \\ em Reais};
\end{tikzpicture}
}
}

\item {} 
O gráfico B, pois nos outros, os valores do eixo das ordenadas não atendiam.

\item {} 
Não. Gráfico A e gráfico C.

\item {} 
\textbf{Escalas distintas}: (\textit{Vantagens}) Podemos visualizar melhor o comportamento do gráfico pois ele passa a ficar visível num espaço menor, além de traça-lo com mais facilidade.

\textbf{Escalas distintas}: (\textit{Desvantagens}) Não podemos analisá-lo geometricamente de maneira satisfatória, as variações entre os eixos são muito discrepantes, e isso pode levar a interpretações equivocadas.

\textbf{Escalas iguais}: (\textit{Vantagens}) Podemos analisá-lo tanto numericamente quanto geometricamente, inferindo com mais precisão.

\textbf{Escalas iguais}: (\textit{Desvantagens}) Precisaríamos de muito espaço e/ou bastante compactação para desenharmos fielmente este gráfico. Note como ficaria:

\begin{figure}[H]
\centering

\resizebox{.7\linewidth}{!}
{
\begin{tikzpicture}[scale=.5, every node/.style={scale=2.5}, yscale=.75]

\draw [help lines, secundario!20, step=.4] (-3,-5) grid (11,11);
\draw [help lines, secundario!50, step=2] (-3,-5) grid (11,11);
\draw [->] (-3,0) -- (11,0);
\draw [ ->] (0,-5) -- (0,11);
\draw [color=atento,domain=-5:11] plot (1.2,\x);
\draw [color=atento,domain=-5:11] plot (-0.4,\x);
\foreach \y in {-4,-2,2,4,6,8,10} \node [left, scale=.3] at (0,\y+0.15) {\y00};
\foreach \x in  {-2,2,4,6,8,10} \node [below, scale=.3] at (\x,-0.01) {\x00};
\node [below left, scale=.3] at (0.5,0) {0};
\node [ponto, color=primario, scale=2] at (1.2,0) {};
\node [ponto, color=primario, scale=2] at (-0.4,0) {};
\node [above right, scale=.3] at (1.2,0) {B};
\node [above left, scale=.3] at (-0.4,0) {A};
\end{tikzpicture}
}
\caption{Escala real}
\end{figure}
\end{enumerate}
}{1}
\end{answer}
\begin{answer}{Aumento da Passagem}
{
\begin{enumerate}\setcounter{enumi}{6}
\item A imagem de \(130\) é negativa, logo se a nova passagem for de \(130\) reais “haveria” um faturamento negativo, o que não é condizente para os dados apresentados no contexto.

\item {} 
Sim, por vários motivos: já vimos que o gráfico de toda função quadrática é uma parábola, e que as função quadráticas são as únicas funções em que as diferenças das imagens, geram uma Progressão aritmética:

\begin{figure}[H]
\centering

% \scalebox{.8}
{
\begin{tikzpicture}[every node/.style={scale=3}, scale=.75]

\draw [-] (0,1) -- (0,13);
% \node [below,scale=.3] at (2,0) {P.A.};
\foreach \y in {1,2,...,13} \draw  (0,\y) -- (-0.7,\y);
\foreach \y in {1,2,...,12} \draw [primario, ->] (-0.25,\y+0.2) -- (-0.25,\y+0.8);
\foreach \y in {1,2,...,12} \draw [primario] (-0.4,\y+0.5) -- (-0.6,\y+0.5);
\node [left,scale=.3] at (-0.75,1) {0};
\node [left,scale=.3] at (-0.75,2) {15 000};
\node [left,scale=.3] at (-0.75,3) {28 000};
\node [left,scale=.3] at (-0.75,4) {39 000};
\node [left,scale=.3] at (-0.75,5) {45 000};
\node [left,scale=.3] at (-0.75,6) {55 000};
\node [left,scale=.3] at (-0.75,7) {60 000};
\node [left,scale=.3] at (-0.75,8) {63 000};
\node [left,scale=.3] at (-0.75,9) {64 000};
\node [left,scale=.3] at (-0.75,10) {63 000};
\node [left,scale=.3] at (-0.75,11) {60 000};
\node [left,scale=.3] at (-0.75,12) {55 000};
\node [left,scale=.3] at (-0.75,13) {48 000};
\node [left,scale=.3] at (1.75,1.5) {-15 000};
\node [left,scale=.3] at (1.75,2.5) {-13 000};
\node [left,scale=.3] at (1.75,3.5) {-11 000};
\node [left,scale=.3] at (1.5,4.5) {-9 000};
\node [left,scale=.3] at (1.5,5.5) {-7 000};
\node [left,scale=.3] at (1.5,6.5) {-5 000};
\node [left,scale=.3] at (1.5,7.5) {-3 000};
\node [left,scale=.3] at (1.5,8.5) {-1 000};
\node [left,scale=.3] at (1.5,9.5) {1 000};
\node [left,scale=.3] at (1.5,10.5) {3 000};
\node [left,scale=.3] at (1.5,11.5) {5 000};
\node [left,scale=.3] at (1.5,12.5) {7 000};
\end{tikzpicture}
}
\caption{Progressão aritmética}
\end{figure}
\vspace{-1em}

\item {} 
\(40+x\).

\item {} 
\(1 200 - 10x\).

\item {} 
\(F(x)=(40+x).(1200-10x)\) ou \(F(x)=-10x^2+800x+48.000\).

\item {} 
Sim. Ou pela justificativa dada no item ‘f’ ou pelo fato da função quadrática ser uma função do polinômio de grau 2, e a função em questão, apresenta \(a=-10\) ; \(b=800\) e \(c=48.000\) coeficientes do polinômio do segundo grau.


\item {} 
\(A\) é o conjunto dos números naturais de \(0\) a \(120\); \(B\) é o conjunto dos números naturais contidos no intervalo:\([0,64.000]\) que são imagens dos elementos do conjunto \(A\).

\item {} 
R\$ \(48.000{,}00\) que representa o faturamento atual, inicial ou seja, o faturamento sem aumento no valor da passagem.


\end{enumerate}
}{1}
\end{answer}
\begin{answer}{Aumento da Passage}
{
\begin{enumerate}\setcounter{enumi}{14}

\item {} 
No ponto \((120,0)\), representa que se o aumento for de R\$ \(120,00\), não haverá faturamento, ou seja, a empresa faturaria zero reais.

\item {} 
O ponto seria \((-40,0)\), ele é desconsiderado pois sua abscissa é negativa, e não cabe na situação utilizar “aumentos negativos”.

\item \(F(x)=-10x^2+800x+48.000\iff F(x)=-10(x^2-80x)+48.000\iff F(x)=-10(x^2-80x+1.600-1.600)+48.000\iff F(x)=-10(x-40)^2+16.000+48.000
\iff F(x)=-10(x-40)^2+64.000\).

\item {} 
R\$ \(40{,}00\).

\item {} 
R\$ \(6400{,}00\). Sim, em ambos.
\end{enumerate}
}{0}
\end{answer}
\begin{answer}{Exercícios}
{\exerciselist
\begin{enumerate}
\item 

\begin{enumerate}\setlength\columnsep{2.5pt}
\begin{multicols}{2}
\item 
\adjustbox{valign=t}
{
\resizebox{.95\linewidth}{!}
{
\begin{tikzpicture}[every node/.style={scale=3}]

\draw [->] (-2.5,0) -- (4.5,0);
\draw [->] (0,-2) -- (0,6);
\foreach \y in {-2,-1, 0,1,...,5} \node [above left,scale=.3] at (0,\y) {\y};
\foreach \x in {-2,-1,1,2,...,4} \node [below left,scale=.3] at (\x,0) {\x};  
\draw [color=primario,  domain=-1.6:3.6] plot (\x,{(\x)^2-2*(\x)});
\node [ponto, color=black, fill = primario, fill opacity=1] at (0,0) {};  
\node [ponto, color=black, fill = primario, fill opacity=1] at (2,0) {};  
\end{tikzpicture}
}
}

\item 
\adjustbox{valign=t}
{
\resizebox{.95\linewidth}{!}
{
\begin{tikzpicture}[every node/.style={scale=3}]

\draw [  ->] (-4.5,0) -- (4.5,0);
\draw [  ->] (0,-1) -- (0,10);
\foreach \y in {-1, 0,1,...,9} \node [above left,scale=.3] at (0,\y) {\y};
\foreach \x in {-4,-3,-2,-1,1,2,...,4} \node [below left,scale=.3] at (\x,0) {\x};  
\draw [color=primario,domain=-3.2:3.2] plot (\x,{-(\x)^2+9});
\node [ponto, color=black, fill = primario, fill opacity=1] at (-3,0) {}; 
\node [ponto, color=black, fill = primario, fill opacity=1] at (3,0) {};  
\node [ponto, color=black, fill = primario, fill opacity=1] at (0,9) {};
\end{tikzpicture}  
}      
}
      
\item 
\adjustbox{valign=t}
{

\resizebox{.95\linewidth}{!}
{
\begin{tikzpicture}[every node/.style={scale=3}]
\draw [->] (-4,0) -- (5,0);
\draw [->] (0,-1) -- (0,8);
\foreach \y in {-1, 0,1,...,7} \node [above left,scale=.3] at (0,\y) {\y};
\foreach \x in {,-3,-2,-1,1,2,...,5} \node [below left,scale=.3] at (\x,0) {\x};  
\draw [color=primario,   domain=-1.8:3.8] plot (\x,{(\x)^2-2*(\x)+1});
\node [ponto, color=black, fill = primario, fill opacity=1] at (1,0) {};  
\node [ponto, color=black, fill = primario, fill opacity=1] at (0,1) {};
\end{tikzpicture}
}
}
      
   
\item 
\adjustbox{valign=t}
{
\resizebox{.95\linewidth}{!}
{
\begin{tikzpicture}[every node/.style={scale=3}]

\draw [->] (-1,0) -- (7.5,0);
\draw [->] (0,-6.2) -- (0,4.5);
\foreach \y in {-6,-5,...,3,4} \node [above left,scale=.3] at (0,\y) {\y};
\foreach \x in {1,2,...,7} \node [below left,scale=.3] at (\x,0) {\x}; 
\draw [color=primario,   domain=-0.2:6.2] plot (\x,{-(\x)^2+6*(\x)-5});
\node [ponto, color=black, fill = primario, fill opacity=1] at (1,0) {};  
\node [ponto, color=black, fill = primario, fill opacity=1] at (5,0) {};  
\node [ponto, color=black, fill = primario, fill opacity=1] at (0,-5) {}; 
\end{tikzpicture}
}
}

\item 
\adjustbox{valign=t}
{
\resizebox{.95\linewidth}{!}
{
\begin{tikzpicture}[every node/.style={scale=3}]


\draw [->] (-4,0) -- (4,0);
\draw [->] (0,-7.2) -- (0,3);
\foreach \y in {-7,-6,...,1,2} \node [above left,scale=.3] at (0,\y) {\y};
\foreach \x in {-3,-2,-1,1,2,...,3} \node [below left,scale=.3] at (\x,0) {\x};
\draw [color=primario,   domain=-1.5:1.5] plot (\x,{-3*(\x)^2});
\node [ponto, color=black, fill = primario, fill opacity=1] at (0,0) {};  
\end{tikzpicture}
}
}


\item 
\adjustbox{valign=t}
{
\resizebox{.95\linewidth}{!}
{
\begin{tikzpicture}[every node/.style={scale=3}]
\draw [->] (-5,0) -- (5,0);
\draw [ ->] (0,-2.2) -- (0,9);
\foreach \y in {-2,-1,...,7,8} \node [above left,scale=.3] at (0,\y) {\y};
\foreach \x in {-4,-3,-2,-1,1,2,...,4} \node [below left,scale=.3] at (\x,0) {\x};
\draw [color=primario,domain=-3.3:2.3] plot (\x,{(\x)^2+(\x)+1});
\node [ponto, color=black, fill = primario, fill opacity=1] at (0,1) {};  
\end{tikzpicture}
}
}
\end{multicols}
\end{enumerate}

\item Gabarito letra E. Em \(f(x)=-x^2+4x+5\), temos que \(C=(0,5)\), \(x_1=-1\) e \(x_2=5\), logo \(A=(-1,0)\) e \(B=(5,0)\).

Como \(x_v=\frac{-b}{2a}=\frac{-4}{2\cdot(-1)}=2\) e \(f(2)=9\), temos que \(V=(2,9)\);

\end{enumerate}
}{1}
\end{answer}
\clearmargin
\begin{answer}{Exercícios}
{\exerciselist
\begin{enumerate}\setcounter{enumi}{2}
\item Gabarito letra E. Como a concavidade da parábola está voltada para baixo, temos que \(a<0\);

Como o \(x_v\) é positivo e \(a<0\), temos que \(x_v=\frac{-b}{2a}\to+=\frac{-?}{-}\) assim \(b>0\).

Como a parábola intersecta o eixo das ordenadas na parte positiva deste eixo, temos que \(c>0\).

\end{enumerate}
}1{}
\end{answer}



\explore{Otimização em Domínio Discreto}
\label{\detokenize{AF209-7:sec-funcao-quadratica-explorando-max-min-can}}\label{\detokenize{AF209-7::doc}}\label{\detokenize{AF209-7:explorando-otimizacao-em-dominio-discreto-e-escalas-graficas}}\phantomsection\label{\detokenize{AF209-7:ativ-funcao-quadratica-aumento-passagem}}
\begin{task}{Aumento na passagem}

Uma empresa de transporte rodoviário, faz o trajeto entre duas cidades brasileiras diariamente, e transporta mensalmente, uma média de \(1200\) passageiros. O custo individual da passagem cobrado pela empresa, é atualmente de R\$$40,00$, porém seus diretores estudam um aumento desse valor. Para isso contratam uma outra empresa para realizar uma pesquisa de mercado, a pesquisa realizada por essa empresa, estima que a cada R\$$1,00$ de aumento no preço da passagem, \(10\) passageiros deixarão de viajar pela transportadora. De posse desta informação, os diretores desejam saber qual é o preço de passagem, em reais, que vai maximizar o faturamento dessa transportadora. Para isso vamos responder os itens a seguir:
\begin{enumerate}
\item Se aumentarmos em R\$$2,00$ a passagem qual será seu novo preço? Qual a nova quantidade de passageiros? Qual será o novo faturamento em reais? E se o aumento fosse de R\$$12,00$?

\item Preencha a tabela a seguir, seguindo o padrão que modela a situação.

\begin{table}[H]
\centering
\setlength\tabcolsep{2.5pt}
\begin{tabu} to \textwidth{|c|l|l|l|}
\hline
\thead
Aumento em reais & Novo preço & Nova quantidade de passageiros & Faturamento em reais \\
\hline 0 & 40 + 1 . 0 = 40 & 1 200 - 10 . 0 = 1 200 & 40 . 1200 = 48 000 \\
\hline
10 & 40 + 1 . 10 = 50 & 1 200 - 10 . 10 = 1 100 & 50 . 1 100 = 55 000 \\
\hline
20 & 40 + 1 . 20 = 60 & 1 200 - 10 . 20 = 1 000 & 60 . 1 000 = 60 000 \\
\hline
30 & & & \\
\hline
40 & & & \\
\hline
50 & & & \\
\hline
60 & & & \\
\hline
70 & & & \\
\hline
80 & & & \\
\hline
90 & & & \\
\hline
100 & & & \\
\hline
110 & & & \\
\hline
130 & & & \\
\hline
\end{tabu}
\end{table}

\item {} 
Escolha um dos planos cartesianos a seguir, para representar os pontos da tabela acima e os represente no plano escolhido.
\begin{figure}[H]
\centering

\begin{tikzpicture}[scale=.5, every node/.style={scale=2.75}, xscale=1.25]

\draw [help lines, secundario!30, step=2] (-1,-1) grid (13,13);
\draw [help lines, dotted, secundario!70, step=.25] (-1,-1) grid (13,13);
\draw [->] (-1,0) -- (13,0);
\draw [->] (0,-1) -- (0,13);
\foreach \y in {1,2,3,...,12} \node [left,scale=.3] at (0,\y) {\y0};
\foreach \x in  {1,2,3,...,12} \node [below,scale=.3] at (\x,-0.01) {\x0};
\node [below left,scale=.3] at (0,0) {0};

\end{tikzpicture}
\caption{Gráfico A}
\end{figure}

\begin{figure}[H]
\centering

\begin{tikzpicture}[scale=.5, every node/.style={scale=2.75}, xscale=1.25]

\draw [help lines, secundario!30, step=2] (-1,-1) grid (13,8);
\draw [help lines, dotted, secundario!70, step=.25] (-1,-1) grid (13,8);
\draw [->] (-1,0) -- (13,0);
\draw [->] (0,-1) -- (0,8);
\foreach \y in {1,2,3,...,7} \node [left,scale=.3] at (0,\y) {\y0 000};        \foreach \x in  {1,2,3,...,12} \node [below,scale=.3] at (\x,-0.01) {\x0};
\node [below left, scale=.3] at (0,0) {0};

\end{tikzpicture}
\caption{Gráfico B}
\end{figure}

\begin{figure}[H]
\centering

\begin{tikzpicture}[scale=.5, every node/.style={scale=2.75}, xscale=1.25]

\draw [help lines, secundario!30, step=2] (-1,-1) grid (14,14);
\draw [help lines, dotted, secundario!70, step=.25] (-1,-1) grid (14,14);
\draw [->] (-1,0) -- (14,0);
\draw [->] (0,-1) -- (0,14);
\foreach \y in {1,2,3,...,13} \node [left,scale=.3] at (0,\y) {\y};
\foreach \x in  {1,2,3,...,13} \node [below,scale=.3] at (\x,-0.01) {\x};
\node [below left,scale=.3] at (0,0) {0};

\end{tikzpicture}
\caption{Gráfico C}
\end{figure}
\item {} 
Qual “gráfico” você escolheu? Justifique sua escolha.

\item {} 
A escala no “gráfico” escolhido é a mesma nos dois eixos? Quais os “gráficos” do item “b” possuem a mesma escala nos dois eixos?

\item {} 
Quais as vantagens e desvantagens em ambos os casos (eixos em escalas distintas e eixos em mesma escala)?

\item {} 
Explique o motivo do valor \(130\) estar na tabela e não estar no gráfico. Justifique levando em consideração o valor de sua imagem dentro do conceito da atividade.

\item {} 
Podemos afirmar que os pontos obtidos, são pontos de uma parábola? Justifique sua resposta.

\item {} 
Ao representarmos por \(x\) o aumento, em reais pretendido , exiba uma expressão algébrica que represente o novo preço da passagem (já com o aumento de \(x\) reais).

\item {} 
Ao representarmos por \(x\) o aumento, em reais pretendido , exiba uma expressão algébrica que represente a nova quantidade mensal de passageiros (já com o aumento de \(x\) reais).

\item {} 
Ao representarmos por \(x\) o aumento, em reais pretendido , exiba uma expressão algébrica que represente o faturamento da empresa em função de \(x\), dado por \(F(x)\).

\item {} 
Se representarmos expressão obtida no item anterior por uma função \(F:A\to B\), onde \(A\) é seu domínio e \(B\) é sua imagem, podemos afirmar que \(F\) é uma função quadrática? Justifique sua resposta

\item {} 
Apresente os conjuntos \(A\) (domínio de \(F\)) e \(B\) (imagem \(F\)) que satisfazem os valores possíveis na situação apresentada.

\item {} 
Em que ponto o gráfico corta o eixo das ordenadas? E o que esse valor representa na situação?

\item {} 
Em que ponto o gráfico corta o eixo das abscissas? O que esse ponto representa na situação?

\item {} 
E se o domínio fosse o \(\mathbb{R}\), qual seria o outro ponto de intersecção com o eixo das abscissas? Por que ele não é considerado na situação?

\item {} 
Utilize o processo de completar quadrados  e apresente a função \(F\) em sua forma canônica.

\item {} 
Enfim, qual é o aumento no preço de passagem, em reais, que vai maximizar o faturamento dessa transportadora?

\item {} 
Qual é o valor desse faturamento máximo? Este valor aparece tabela e no gráfico?

\end{enumerate}
\end{task}


\exercise


\begin{enumerate}

\item Trace os gráficos das funções \(f:\mathbb{R}\to\mathbb{R}\) definidas por:
\begin{enumerate}
\item {} 
\(f(x)=x^2-2x\)

\item {} 
\(f(x)=-x^2+9\)

\item {} 
\(f(x)=x^2-2x+1\)

\item {} 
\(f(x)=-x^2+6x-5\)

\item {} 
\(f(x)=-3x^2\)

\item {} 
\(f(x)=x^2+x+1\)

\end{enumerate}

\item Dada a função real \(f\) dada por \(f(x)=-x^2+4x+5\) , o gráfico da mesma está representado abaixo:
\begin{figure}[H]
\centering

\begin{tikzpicture}[yscale=.5]
\draw [->] (-2,0) -- (6,0);
\draw [->] (0,-1.5) -- (0,9.5);
\node [left] at (0,9) {$f(x)$};
\node [below right] at (5.5,-0.1) {$x$};
\draw [color=\currentcolor!80,domain=-1.2:5.2, thick] plot (\x,{-(\x)^2+4*(\x)+5});
\draw [color=secundario, dashed] (0,9) -- (2,9) -- (2,0);
\node [ponto, color=black, fill = \currentcolor!80, fill opacity=1] at (-1,0) {};
\node [ponto, color=black, fill = \currentcolor!80, fill opacity=1] at (5,0) {};
\node [ponto, color=black, fill = \currentcolor!80, fill opacity=1] at (0,5) {};
\node [ponto, color=black, fill = \currentcolor!80, fill opacity=1] at (2,9) {};
\node [above left] at (-1,0) {A};
\node [above right] at (5,0) {B};
\node [above left] at (0,5) {C};
\node [above] at (2,9) {V};
\end{tikzpicture}
\end{figure}

As coordenadas corretas dos pontos do gráfico são:
\begin{enumerate}
\item {} 
\(C=(0,5)\)  ;  \(A=(-1,0)\)  ; \(B=(5,0)\)  ; \(V=(3,9)\)

\item {} 
\(C=(0,4)\)  ;  \(A=(0,-1)\)  ; \(B=(0,5)\)  ; \(V=(2,9)\)

\item {} 
\(C=(0,5)\)  ;  \(A=(0,-1)\)  ; \(B=(0,5)\)  ; \(V=(9,2)\)

\item {} 
\(C=(0,5)\)  ;  \(A=(-1,0)\)  ; \(B=(4,0)\)  ; \(V=(3,4)\)

\item {} 
\(C=(0,5)\)  ;  \(A=(-1,0)\)  ; \(B=(5,0)\)  ; \(V=(2,9)\)

\end{enumerate}
\clearpage
\item Seja a função real \(g\), definida por \(g(x)=ax^2+bx+c\) representada no gráfico a seguir:
\begin{figure}[H]
\centering

\begin{tikzpicture}[yscale=.5]
\draw [->] (-2,0) -- (6,0);
\draw [->] (0,-1.5) -- (0,9.5);
\node [left] at (0,9) {$g(x)$};
\node [below right,scale=.3] at (5.5,0) {$x$};
\draw [color=\currentcolor!80,domain=-1.2:5.2, thick] plot (\x,{-(\x)^2+4*(\x)+5});
\draw [color = secundario, dashed] (2,9) -- (2,0);
\node [ponto, color=black, fill = \currentcolor!80, fill opacity=1] at (2,0) {};
\end{tikzpicture}
\end{figure}

Pode-se afirmar que:
\begin{enumerate}
\item {} 
\(a>0\); \(b>0\); \(c<0\)

\item {} 
\(a>0\); \(b<0\); \(c>0\)

\item {} 
\(a<0\); \(b<0\); \(c<0\)

\item {} 
\(a<0\); \(b>0\); \(c<0\)

\item {} 
\(a<0\); \(b>0\); \(c>0\)

\end{enumerate}
\end{enumerate}

\arrange{Interseção com os Eixos Coordenados}
\label{\detokenize{AF209-8::doc}}\label{\detokenize{AF209-8:sec-funcao-quadratica-org-ideias-intersecoes-com-eixos}}\label{\detokenize{AF209-8:organizando-as-ideias-intersecao-com-os-eixos-coordenados}}
Em atividades anteriores, observamos as mudanças ocorridas no gráfico de uma funçao quadrática em sua forma canônica. Ao observarmos essa função definida em sua forma polinomial, também conseguimos perceber a influência que cada um de seus coeficientes (\(a\), \(b\) e  \(c\)) tem na posição do gráfico da função.

Dada a função real \(f\) definida por \(f(x)=ax^2+bx+c\), podemos descobrir em quais pontos \(f\) intersecta os eixos coordenados:
\begin{enumerate}
\item {} 
\textbf{O eixo das ordenadas}

Todo ponto pertencente ao eixo das ordenadas possui coordenada \(x=0\), logo \(P=(0,f(0))\) pertence à parábola e ao eixo dadas ordenadas com \(P=(0,c)\), pois:
\begin{equation*}
\begin{split}f(0)=a\cdot(0)^2+b\cdot0 + c = 0 + 0 + c = c\end{split}
\end{equation*}
\item {} 
\textbf{O eixo das abscissas}

Todo ponto pertencente ao eixo das abscissas possui coordenada \(f(x)=0\), logo \(P=(x,0)\) pertence à parábola e ao eixo das abscissas. E quando substituímos \(f(x)\) por \(0\) , temos uma equação do segundo grau, ela pode ser resolvida por diversas maneiras, a mais comum utilizando a fórmula quadrática, conhecida no Brasil erroneamente por fórmula de Báskara:
\begin{equation*}
\begin{split}x=\frac{-b\pm\sqrt{\Delta}}{2a} \text{,}\end{split}
\end{equation*}
onde \(\Delta=b^2-4ac\).

Ao resolvermos a equação quadrática é possível que encontremos os \textbf{zeros da função}, termo utilizado para os valores de \(x\) que fazem \(f(x)\) ser nula, \(f(x)=0\). Neste caso obteremos o(s) ponto(s) em que o gráfico da função intersecta o eixo das abscissas.

Analisemos o sinal de \(\Delta\):

\textbf{Caso 1} Para \(\Delta>0\) , teremos na fórmula quadrática, uma raiz quadrada de um número positivo, sendo seu resultado um número real positivo qualquer (inteiro, racional não inteiro ou irracional), ao somarmos esse valor com \(-b\) e dividirmos o resultado por \(2a\),  encontraremos um dos zeros da função \(f\) que chamaremos de \(x_1\); ao subtrairmos esse valor de \(-b\) e dividirmos o resultado por \(2a\), teremos o outro zero da função \(f\) que chamaremos de \(x_2\), sendo: $$x_1= \frac{-b+\sqrt{\Delta}}{2a} \text{ e } x_2=\frac{-b-\sqrt{\Delta}}{2a}$$

Portanto a parábola (gráfico de \(f\)) intersectará o eixo das abscissas em dois pontos: \(P_1=(x_1,0)\)  e \(P_2=(x_2,0)\)
\begin{figure}[H]
\centering
\begin{multicols}{2}
\begin{tikzpicture}[yscale=.5, scale=.9]

\draw [ thick, ->] (-1,0) -- (6,0);
\draw [ thick, ->] (0,-2) -- (0,8);
\draw [color=\currentcolor!80,  thick, domain=-0.36:5.6] plot (\x,{(\x)^2-5.25*(\x)+5.5});
\node [ponto, color=black, fill = \currentcolor!80, fill opacity=1] at (3.8,0) {};
       \node [ponto, color=black, fill = \currentcolor!80, fill opacity=1] at (0,5.5) {};
\node [ponto, color=black, fill = \currentcolor!80, fill opacity=1] at (1.4,0) {};
\node [left] at (-0.5,7) {$f$};
       \node [above right] at (3.9,0) {($x_2$,0)};
       \node [above left] at (1.4,0) {($x_1$,0)};
       \node [right] at (0,5.5) {(0,c)};
\node [above left] at (3.3,-2.25) {$a>0$};
\node [above left] at (-0.64,0) {\phantom{($x_1$,0)}};
\end{tikzpicture}

\begin{tikzpicture}[yscale=.5, scale=.9]
  \draw [ thick, ->] (-1,0) -- (6,0);
       \draw [ thick, ->] (0,-2) -- (0,7.5);
\draw [color=\currentcolor!80,  thick, domain=-1:5] plot (\x,{-(\x)^2+4*(\x)+3});
\node [ponto, color=black, fill = \currentcolor!80, fill opacity=1] at (4.6,0) {};
\node [ponto, color=black, fill = \currentcolor!80, fill opacity=1] at (0,3) {};
\node [ponto, color=black, fill = \currentcolor!80, fill opacity=1] at (-0.64,0) {};
\node [right] at (-1.5,-0.5) {$f$};
       \node [above right] at (4.6,0) {($x_2$,0)};
       \node [above left] at (-0.64,0) {($x_1$,0)};
       \node [right] at (0,3) {(0,c)};
\node [above left] at (2.5,-1) {$a<0$};
\end{tikzpicture}
\end{multicols}
\caption{(\(\Delta>0\))}
\end{figure}


\textbf{Caso 2} Para \(\Delta=0\) , teremos na fórmula quadrática, uma raiz quadrada de zero, que resulta em zero, e somar ou subtrair \(0\) de \(-b\) acharemos sempre \(-b\), resultando assim num único zero de \(f\), no caso: \(\displaystyle x= \frac{-b\pm\sqrt{0}}{2a}=\frac{-b\pm0}{2a}=\frac{-b}{2a}=p=x_v\). Note que esta expressão é a mesma apresentada anteriormente para a coordenada \(x\) do vértice da parábola.

Portanto a parábola (gráfico de \(f\)) “tocará” no eixo das abscissas em apenas um ponto: \(\displaystyle V=(\frac{-b}{2a},0)\). Ou melhor, o eixo das abscissas será tangente à parábola no ponto \(V\).
\begin{figure}[H]
\centering
\begin{multicols}{2}
\begin{tikzpicture}[yscale=.5, scale=.9]
\draw [ thick, ->] (-1,0) -- (6,0);
       \draw [ thick, ->] (0,-2) -- (0,8);
                     \draw [color=\currentcolor!80, thick, domain=-0.36:5.6] plot (\x,{(\x)^2-5.25*(\x)+5.5});
              \node [ponto, color=black, fill = \currentcolor!80, fill opacity=1] at (0,5.5) {};
              \node [ponto, color=black, fill = \currentcolor!80, fill opacity=1] at (2.62,-1.39) {};
       \node [left, ] at (-0.5,7) {$f$};
       \node [right, ] at (0,5.5) {(0,c)};
       \node [above, ] at (2.62,-3.55) {$a>0$};
       \node [above, align=center, ] at (2.62,-2.85) {V=($\frac{b}{2a}$, 0)};
\end{tikzpicture}

\begin{tikzpicture}[yscale=.5, scale=.9]
    \draw [ thick, ->] (-1,0) -- (6,0);
       \draw [ thick, ->] (0,-8) -- (0,2);
       \draw [color=\currentcolor!80,  thick, domain=-0.8:4.8] plot (\x,{-(\x)^2+4*(\x)-4});
       \node [ponto, color=black, fill = \currentcolor!80, fill opacity=1] at (0,-4) {};
       \node [ponto, color=black, fill = \currentcolor!80, fill opacity=1] at (2,0) {};
       \node [left, ] at (-0.5,-6) {$f$};
  \node [left, ] at (0,-4) {(0,c)};
       \node [above, ] at (2,-8) {$a<0$};
  \node [above, align=center, ] at (2,0.5) {V=($-\frac{b}{2a}$, 0)};
\end{tikzpicture}
\end{multicols}
\caption{(\(\Delta=0\))}
\end{figure}


\textbf{Caso 3} Para \(\Delta<0\) , teremos na fórmula quadrática, uma raiz quadrada de um número real negativo, que resulta no fato de não existir um valor de \(x\) real que atenda essa equação quadrática. Sendo assim o gráfico nem toca nem intersecta o eixo das abscissas, logo a função \(f\) não possuirá zeros.

Portanto a parábola (gráfico de \(f\)) ficará posicionada totalmente acima (\(a>0\)) ou abaixo (\(a<0\)) do eixo das abscissas.
\begin{figure}[H]
\centering
\begin{multicols}{2}
\begin{tikzpicture}[yscale=.5, every node/.style={scale=2.5}, scale=.9]

  \draw [ thick, ->] (-6,0) -- (1,0);
       \draw [ thick, ->] (0,-2) -- (0,8);
       \draw [color=\currentcolor!80,  thick, domain=-5.3:0.07] plot (\x,{(\x)^2+5.25*(\x)+7.5});
       \node [ponto, color=black, fill = \currentcolor!80, fill opacity=1] at (0,5.5) {};
       \node [ponto, color=black, fill = \currentcolor!80, fill opacity=1] at (-2.625,0.609375) {};
       \node [left, scale=0.4] at (-5.5,7) {$f$};
       \node [right, scale=0.4] at (0,5.5) {(0,c)};
       \node [above, scale=0.4] at (-2.62,-3) {$a>0$};
  \node [above, align=center, scale=0.4] at (-2.625,0.609375) {V};
\end{tikzpicture}

\begin{tikzpicture}[yscale=.5, every node/.style={scale=2.5}, scale=.9]

  \draw [ thick, ->] (-1.5,0) -- (5.5,0);
       \draw [ thick, ->] (0,-8.5) -- (0,1.5);
              \draw [color=\currentcolor!80,  thick, domain=-0.8:4.8] plot (\x,{-(\x)^2+4*(\x)-4.5});
\node [ponto, color=black, fill = \currentcolor!80, fill opacity=1] at (0,-4.4) {};
  \node [ponto, color=black, fill = \currentcolor!80, fill opacity=1] at (2,-0.5) {};
  \node [left, scale=0.4] at (-0.5,-6) {$f$};
  \node [left, scale=0.4] at (0,-4.4) {(0,c)};
       \node [above, scale=0.4] at (2,-8) {$a<0$};
  \node [below, align=center, scale=0.4] at (2,-0.5) {V};
\end{tikzpicture}
\end{multicols}
\caption{(\(\Delta<0\))}
\end{figure}


\end{enumerate}

Vale ressaltar que uma função quadrática \(f:\mathbb{R}\to\mathbb{R}\) definida em sua forma polinomial por: \(f(x)=ax^2+bx+c\), além de poder ser representada em sua forma canônica: \(f(x)=a(x-p)^2+q\), também pode ser escrita em sua forma fatorada: \(f(x)=a(x-x_1)(x-x_2)\) , onde \(x_1\) e \(x_2\) são os zeros de \(f\).

Exemplo \(1\) : Seja a função real \(f\) definida por \(f(x) = x^2 - 6x + 8\) representada graficamente por:
\begin{figure}[H]
\centering

\begin{tikzpicture}[yscale=.4, every node/.style={scale=2.5}]
       \draw [ thick, ->] (-1.5,0) -- (7.5,0);
     \draw [ thick, ->] (0,-1.5) -- (0,10.5);
     \node [below left, scale=.4] at (0,0) {0};
       \foreach \y in {-1,1,2,...,9,10} \node [left, scale=0.4] at (0,\y) {\y};
       \foreach \x in {-1,1,2,...,7} \node [below , scale=0.4] at (\x,0) {\x};
       \draw [color=\currentcolor!80,  thick, domain=-0.36:6.36] plot (\x,{(\x)^2-6*(\x)+8});
       \node [ponto, color=black, fill = \currentcolor!80, fill opacity=1] at (2,0) {};
     \node [ponto, color=black, fill = \currentcolor!80, fill opacity=1] at (4,0) {};
     \node [ponto, color=black, fill = \currentcolor!80, fill opacity=1] at (0,8) {};
     \node [left, scale=0.4] at (6,9) {$f$};
\end{tikzpicture}
\caption{\(f(x) = x^2 - 6x + 8\)}
\end{figure}


Note que suas raízes são \(x_1=2\) e \(x_2=4\) (podemos descobrir esses valoress utilizando a fórmula quadrática), note que a forma fatorada de \(f\) será:
\begin{equation*}
\begin{split}f(x)=(x-2)(x-4)\end{split}
\end{equation*}
Para retornarmos para a forma polinomial basta efetuarmos o produto indicado.

Exemplo 2: Seja a função real \(f\) definida por \(f(x) = -2x^2 +4x + 6\) representada graficamente por:
\begin{figure}[H]
\centering

\begin{tikzpicture}[yscale=.5, every node/.style={scale=2.5}]

     \draw [ thick, ->] (-3.5,0) -- (4.5,0);
       \draw [ thick, ->] (0,-2.5) -- (0,9.5);
       \node [below left,scale=.4] {0};
              \foreach \y in {-2,-1,1,2,...,8,9} \node [left, scale=0.4] at (0,\y) {\y};
     \foreach \x in {-2,-1,1,2,...,4} \node [below, scale=0.4] at (\x,0) {\x};
             \draw [color=\currentcolor!80,  thick, domain=-1.2:3.2] plot (\x,{-2*(\x)^2+4*(\x)+6});
       \node [ponto, color=black, fill = \currentcolor!80, fill opacity=1] at (-1,0) {};
       \node [ponto, color=black, fill = \currentcolor!80, fill opacity=1] at (3,0) {};
     \node [ponto, color=black, fill = \currentcolor!80, fill opacity=1] at (0,6) {};
     \node [left, scale=0.4] at (3,6) {$f$};
\end{tikzpicture}
\caption{\(f(x) = -2x^2 +4x + 6\)}
\end{figure}

Note que suas raízes são \(x_1=3\) e \(x_2=-1\) (podemos descobrir esses valores utilizando a fórmula quadrática), note que a forma fatorada de \(f\) será:
\begin{equation*}
\begin{split}f(x)=a(x-3)(x+1)\end{split}
\end{equation*}
Substituindo o ponto \((0,6)\) temos:
\begin{align*}
6&=a(0-3)(0+1)\\
6&=-3a\\
a&=-2\\
\end{align*}
Portanto, \(f\) na sua forma fatorada será dada por: \(f(x)=-2(x-3)(x+1)\)

Para retornarmos para a forma polinomial basta efetuarmos o produto indicado.

\cleardoublepage
\def\currentcolor{session1}
\begin{objectives}{Altura do arco da praça da Apoteose}
{
\begin{itemize}
\item Relacionar, a partir de dados gráficos, qual a forma da função quadrática que melhor descreve a situação.

\item {} 
Associar situações concretas à forma da parábola e buscar soluções a partir da aplicação das ferramentas da função quadrática.

\item {} 
Inferir sobre a utilidade da função quadrática no cotidiano.

\item {} 
Distinguir em problemas concretos o papel de abscissa e ordenada para a representação gráfica da parábola.
\end{itemize}
}{1}{1}
\end{objectives}
\begin{sugestions}{Altura do arco da praça da Apoteose}
{
Como um dos objetivos principais deste conjunto de atividades é trazer alguma sugestão de aplicação ou contextualização do tema, todos os enunciados apresentam um texto de motivação ou ambientação.

Além disso, as atividades tem um apelo mais visual, no sentido de conectar a função quadrática à sua representação gráfica, a parábola. Sendo assim, em todos os casos apresentados, o estudante deve fazer a escolha do sistema de coordenadas cartesiana q      ue melhor se adequa à situação apresentada e, a partir daí usar os conteúdos estudados para responder as questões sobre a situação apresentada.

Orientamos o professor a calcular o tempo das atividades levando em conta a possibilidade de um aluno não seguir o caminho que as respostas das atividades sugerem para que eles entendam as dificuldades que essas escolhas trazem. Isso será, certamente, tão importante quanto a resposta prevista em si.

Para cada sugestão gráfica apresentada para desenvolver o problema, deixe que os alunos argumente o porquê de suas escolhas e peça para aqueles que escolheram a forma que sugerimos nas respostas argumente sobre sua escolha.

Mesmo esse grupo de atividades sendo único, adicionamos à última atividade um texto sobre jogos eletrônicos e seus uso em sala de aula. Posto aqui, as orientações ficariam maiores do que o necessário e no tempo errado.
}{1}{1}
\end{sugestions}
\clearmargin
\marginpar{\vspace{.5em}}
\begin{answer}{Altura do arco da praça da Apoteose}
{
\begin{enumerate}
\item {} 
Apenas o comprimento da base, de \(50\) m.

\item {} 
Sim, seria a imagem do vértice.

\item {} 
Figura 2, pois a base do “arco” foi rascunhado sobre o eixo \(x\) e a altura procurada está sobre o eixo \(y\).

\item {} 
\(x\) pontos na base do arco e \(y\) medidas referentes às alturas de cada ponto da base do arco.

\item {} 
\((-25,0)\) e \((25,0)\).

\item {} 
\(f(x)=a(x-x_1)(x-x_2)\)

\item {} 
Apenas um, o \(a\).

\item {} 
\(a=-0\text{,}04\), portanto \(f(x)=-0\text{,}04(x+25)(x-25)\).

\item {} 
A altura aproximada do arco acontece para \(x=0\). Assim, \(f(0)=-0\text{,}04(0+25)(0-25)=-0\text{,}04 \dot (-625)=25\) m.

\item {} 
Sim.

\end{enumerate}
}{1}
\end{answer}
\clearmargin
\begin{objectives}{Mãos à obra!}
{
\begin{itemize}
\item Relacionar, a partir de dados gráficos, qual a forma da função quadrática que melhor descreve a situação.

\item {} 
Associar situações concretas à forma da parábola e buscar soluções a partir da aplicação das ferramentas da função quadrática.

\item {} 
Inferir sobre a utilidade da função quadrática no cotidiano.

\item {} 
Distinguir em problemas concretos o papel de abscissa e ordenada para a representação gráfica da parábola.
\end{itemize}
}{1}{1}
\end{objectives}
\clearmargin
\begin{answer}{Mãos à obra!}
{
\begin{enumerate}
\item {} 
Sim.

\item {} 
Figura 2.

\item {} 
Em \(x\) temos medidas que se referem a base dos túneis e em \(y\) temos para cada ponto das bases, as alturas relativas na curva.

\item {} 
\((\ -\dfrac{2\text{,}6+0\text{,}4}{2};4\text{,}3)\ = (-1\text{,}5;4\text{,}3)\), \((0\text{,}5)\) e \((\ \dfrac{2\text{,}6+0\text{,}4}{2};4\text{,}3)\ = (1\text{,}5;4\text{,}3)\).

\item {} 
\(f(x)=a(x-p)^2+q\).

\item {} 
Somente um, o valor de \(a\).

\item {} \begin{equation*}
\begin{split}f(1\text{,}5)= a \cdot (1\text{,}5-0)^2+5=4\text{,}3 
      & \Rightarrow 2\text{,}25 \cdot a = 4\text{,}3-5 \\
      & \Rightarrow a = \frac{-0\text{,}7}{2\text{,}25} \\
      & \Rightarrow a = - \frac{14}{45}. \\\end{split}
\end{equation*}
\item {} \begin{equation*}
\begin{split}- \frac{14}{45} x^2 + 5 = 0 & \Rightarrow \frac{14}{45} x^2 = 5 \\
& \Rightarrow x^2 = \frac{45 \cdot 5}{14} \Rightarrow x= \pm \sqrt{\frac{225}{14}} \\
& \Rightarrow x = \pm \frac{\sqrt{225}}{\sqrt{14}} \Rightarrow x = \pm \frac{15}{3\text{,}75} \\
& \Rightarrow x = \pm 4. \\\end{split}
\end{equation*}
Portanto, as coordenadas das extremidades das bases são \((-4,0)\) e \((4,0)\).

\item {} 
As larguras das bases dos túneis deverão ser iguais a \(2 \cdot 4 = 8\) m.

\end{enumerate}

}{1}
\end{answer}
\clearmargin

\begin{objectives}{A jogada vencedora}
{
\begin{itemize}
\item Relacionar, a partir de dados gráficos, qual a forma da função quadrática que melhor descreve a situação.

\item {} 
Associar situações concretas à forma da parábola e buscar soluções a partir da aplicação das ferramentas da função quadrática.

\item {} 
Inferir sobre a utilidade da função quadrática no cotidiano.

\item {} 
Distinguir em problemas concretos o papel de abscissa e ordenada para a representação gráfica da parábola.
\end{itemize}
}{1}{1}
\end{objectives}
\begin{sugestions}{A jogada vencedora}
{
Os jogos eletrônicos constituem ótimos laboratórios de aprendizagem por simular situações que podem ir assumindo toda a complexidade da realidade aos poucos, uma “variável” por vez. De acordo com {[}WANG{]}, jogos de computador podem criar ambientes e mundos que de outra forma seriam inacessíveis aos estudantes.

Existe disponível na internet diversos projetos que envolvem o uso do jogo Angry Birds para o estudo das parábolas e lançamentos oblíquos. Por exemplo, \href{https://algebra2coach.com/transforming-parabolas-angry-birds-project/}{Transforming Parabolas \textendash{} The Angry Birds Project} e \href{https://www.tes.com/teaching-resource/angry-bird-parabolas-graphing-quadratic-equations-6165424}{Transforming Parabolas \textendash{} The Angry Birds Project}.
}{1}{1}
\end{sugestions}
\begin{answer}{A jogada vencedora}
{
\begin{enumerate}
\item {} 
\((0,0)\), \((5,3)\) e \((7,1)\).

\item {} 
\(x\) será o deslocamento horizontal do pássaro após o lançamento e \(y\) será a altura do pássaro em relação ao eixo \(x\) durante o arremesso.
\end{enumerate}
}{1}
\end{answer}
\clearmargin
\marginpar{\vspace{.5em}}
\begin{answer}{A jogada vencedora}
{
\begin{enumerate}
\item {} 
\(f(x)=ax^2+bx+c\).

\item {} 
\(f(0)=a \cdot 0^2+b \cdot 0+c=0 \Rightarrow c=0\).

\item {} \begin{align*}\!\begin{aligned}
f(5)& =a \cdot 5^2+b \cdot 5+0=3 \Rightarrow 25a+5b=3 \\\\
f(7)& =a \cdot 7^2+b \cdot 7+0=1 \Rightarrow 49a+7b=1 \\\\
\end{aligned}\end{align*}
\item {} \begin{align*}\!\begin{aligned}
49 \cdot 25a+ 49 \cdot 5b= 49 \cdot 3 & \Rightarrow 1225a+245b=147 \\\\
25 \cdot 49a+ 25 \cdot 7b = 25 \cdot 1 & \Rightarrow 1225a+175b=25 \\\\
(245-175) \cdot b = 147-25 & \Rightarrow b= \frac{122}{70} \Rightarrow b= \frac{61}{35} \\\\
\end{aligned}\end{align*}
\item {} \begin{align*}\!\begin{aligned}
7 \cdot 25a+ 7 \cdot 5b= 7 \cdot 3 & \Rightarrow 175a+35b=21 \\\\
5 \cdot 49a+ 5 \cdot 7b = 5 \cdot 1 & \Rightarrow 245a+35b=5 \\\\
(245-175) \cdot a = 5-21 & \Rightarrow a= - \frac{16}{70} \Rightarrow a=- \frac{8}{35} \\\\
\end{aligned}\end{align*}
\item {} 
\(f(x)= - \frac{8}{35}x^2+ \frac{61}{35}x\).
\end{enumerate}
}{1}
\end{answer}

\explore{Determinando a Função Quadrática}
\label{\detokenize{AF209-9:explorando-determinando-a-funcao-quadratica-atraves-do-grafico}}\label{\detokenize{AF209-9::doc}}\label{\detokenize{AF209-9:sec-funcao-quadratica-obtendo-lei-do-grafico}}\phantomsection\label{\detokenize{AF209-9:ativ-funcao-quadratica-altura-do-arco}}
\begin{task}{Altura do arco da praça da Apoteose}

A passarela Professor Darcy Ribeiro \((\star 1922, \dagger 1997)\), mais conhecida como Sambódromo, fica na cidade do Rio de Janeiro e foi construida em 1984. Com projeto arquitetônico de Oscar Niemeyer \((\star 1907, \dagger 2012)\), ela foi concebida para ser o local fixo de uma das maiores festas populares do Brasil, o Carnaval. Ao final da passarela, encontra-se a praça da apoteose, com o museu do samba e um enorme arco cujo formato lembra o de uma parábola.

\begin{figure}[H]
\centering
\capstart

\noindent\includegraphics[width=275bp]{{Apoteose_do_Tiro_com_Arco}.jpg}
\caption{Foto de \href{https://commons.wikimedia.org/wiki/File:Apoteose\_do\_Tiro\_com\_Arco.jpg}{Jorge Mello} CC-BY-SA}\label{\detokenize{AF209-9:id3}}\end{figure}

Em \(2011\), pela primeira vez desde a construção, a prefeitura providenciou a limpeza do arco.

\begin{figure}[H]
\centering
\capstart

\noindent\includegraphics[width=150bp]{{02_15_gvg_rio_lavagem10}.jpg}
\caption{\href{https://extra.globo.com/noticias/rio/banho-nos-arcos-do-sambodromo-1077277.html}{Banho nos arcos do Sambódromo}}\label{\detokenize{AF209-9:id4}}\end{figure}

A empresa que foi contratada para fazer essa limpeza, precisou ter uma estimativa da altura do arco, com a finalidade de saber se seu equipamento seria suficiente para a tarefa, já que a altura máxima que o equipamento suportaria, seria de \(40\) m de altura. Uma busca rápida na internet não forneceu o resultado esperado, apenas que o comprimento da base é de \(50\) m. Sendo assim, a estimativa teve que ser feita através de cálculos. Admitindo por aproximação que o arco seja parabólico, faça o que se pede:
\begin{enumerate}
\item {} 
Quantas informações concretas são fornecidas para esta parábola?

\item {} 
Caso você soubesse a função que descreve essa parábola, você seria capaz de  determinar a altura aproximada do arco?

\item {} 
Dentre as opções a seguir marque a que faz o rascunho do arco no plano cartesiano.


\begin{multicols}{3}
\begin{center}\begin{tikzpicture}[scale=.5]

  \draw [help lines, secundario!30, step=.5] (-2,-1.5) grid (6,5);
       \draw [, ->] (-2,0) -- (6,0) node [below left, scale=0.5] {$x$};
       \draw [, ->] (0,-1.5) -- (0,5) node [below left, scale=0.3] {$y$};
  \draw [color=\currentcolor!80,  , domain=-0.35:5.35] plot (\x,{-0.8*(\x)^2+4*(\x)});
  \node [above, align=center] at (2.5,5) {Figura 1};
\end{tikzpicture}\end{center}\begin{center}\begin{tikzpicture}
[scale=.5]
  \draw [help lines, secundario!30, step=.5] (-4,-1.5) grid (4,5);
  \draw [, ->] (-4,0) -- (4,0) node [below left, scale=0.5] {$x$};
       \draw [, ->] (0,-1.5) -- (0,5) node [below left, scale=0.3] {$y$};
             \draw [color=\currentcolor!80,, domain=-3.15:3.15] plot (\x,{-0.4*(\x)^2+4});
       \node [above, align=center] at (0,5) {Figura 2};
\end{tikzpicture}\end{center}\begin{center}\begin{tikzpicture}
[scale=.5]
       \draw [help lines, secundario!30, step=.5] (-4,-5) grid (4,1.5);
       \draw [, ->] (-4,0) -- (4,0) node [below left, scale=0.5] {$x$};
       \draw [, ->] (0,-5) -- (0,1.5) node [below left, scale=0.3] {$y$};
       \draw [color=\currentcolor!80, , domain=-3.5:3.5] plot (\x,{-0.4*(\x)^2});
  \node [above, align=center] at (0,1.5) {Figura 3};
\end{tikzpicture}\end{center}

\end{multicols}


\item {} 
Para essa escolha, qual o significado dos valores de \(x\) e de \(y\)?

\item {} 
Que pontos do plano cartesiano são conhecidos, se juntarmos a escolha gráfica com os dados fornecidos sobre o arco?

\item {} 
Com base em sua escolha do rascunho gráfico mais adequado e considerando os pontos conhecidos da parábola, qual forma da função quadrática resulta em maior quantidade de informações conhecidas?

\(\Box \; f(x)=ax^2+bx+c\)

\(\Box \; f(x)=a(x-p)^2+q\)

\(\Box \; f(x)=a(x-x_1)(x-x_2)\)

\item {} 
Quantos dados estão faltando para que seja conhecida a função que descreve esta parábola?

\item {} 
Com o auxílio da calculadora gráfica em: Estimando a parábola(\url{https://ggbm.at/VFR6nWHM}) obtenha a informação que falta para obter a função que descreve a parábola.

\item {} 
Qual a altura estimada para a altura do arco?

\item {} 
A empresa contratada para a limpeza do arco teve capacidade de concluir o serviço com o equipamento que possuia?

\end{enumerate}
\end{task}


\phantomsection\label{\detokenize{AF209-9:ativ-funcao-quadratica-largura-tunel}}
\begin{task}{Mãos à obra!}

A prefeitura de uma cidade, com o fim de melhorar as atividades comerciais locais, fez um levantamento com produtores, fornecedores e compradores. Ficou claro que a redução no percurso até a cidade beneficiaria a todos. Por esse motivo, a prefeitura encomendou a contrução de uma nova estrada, que exigiria dois túneis em certo trecho, um para cada sentido da estrada. O formato das entradas ou das saídas dos túneis, a pedido da prefeitura, deverão ser arcos parabólicos.

\begin{figure}[H]
\centering

\noindent\includegraphics[width=300bp]{{5_1}.jpg}
\end{figure}

Limitações geológicas impedem que as alturas dos túneis sejam maiores do que \(5\) m e cada túnel deve permitir a passagem de caminhões comerciais, que tem \(4\text{,}3\) m de altura e \(2\text{,}6\) m de largura. Além disso, para que os caminhões não arrastem pelas paredes dos túneis, uma largura extra de \(0\text{,}4\) m deverá ser considerada conforme o rascunho a seguir.
\begin{center}\begin{tikzpicture}[every node/.style={scale=4}]

\draw [dashed, color=secundario] (2.2,-2) -- (-2.2,-2);
       \draw [color=atento,  thick] (3.6,-4.9) -- (3.8,-4.9) --  (3.7,-4.9) -- (3.7,-1.98) -- (3.6, -1.98) -- (3.8,-1.98);
       \draw [color=atento,  thick] (5.6,-4.9) -- (5.8,-4.9) --  (5.7,-4.9) -- (5.7,0) -- (5.6, 0) -- (5.8,0);
       \node [above, align=center, scale=0.25] at (0, -2) {(2,6+0,4)$m$};
       \node [above, align=center, scale=0.25] at (4.3,-3.5) {4,3 $m$};
       \node [above, align=center, scale=0.25] at (6.2,-2.5) {5 $m$};
       \draw [ thick, domain=-3.5:3.5] plot (\x,{-0.4*(\x)^2});
\end{tikzpicture}\end{center}
Por fim, o projeto dos túneis deve satisfazer as condições mínimas apresentadas por questões econômicas. Sendo assim, a empresa deve calcular a largura das bases das entradas ou saídas dos túneis. {[}Para simplificar o texto, as medidas das entradas ou saídas dos túneis serão tratatas apenas por \textit{medidas dos túneis}.{]}
\begin{enumerate}
\item {} 
Caso você conhecesse a função que descreve essa parábola, você seria capaz de calcular a largura da base dos túneis?

\item {} 
Dentre as opções a seguir marque a que faz o rascunho de um dos túneis no plano cartesiano.


\begin{multicols}{3}
\begin{center}\begin{tikzpicture}
[scale=0.5]
  \draw [help lines, secundario!30, step=.5] (-2,-1.5) grid (6,5);
       \draw [, ->] (-2,0) -- (6,0) node [below left, scale=0.3] {$x$};
       \draw [, ->] (0,-1.5) -- (0,5) node [below left, scale=0.3] {$y$};
  \draw [color=\currentcolor!80,  , domain=-0.35:5.35] plot (\x,{-0.8*(\x)^2+4*(\x)});
  \node [above, align=center] at (2.5,5) {Figura 1};
\end{tikzpicture}\end{center}\begin{center}\begin{tikzpicture}
[scale=0.5]
\draw [help lines, secundario!30, step=.5] (-4,-1.5) grid (4,5);
\draw [, ->] (-4,0) -- (4,0) node [below left, scale=0.3] {$x$};
       \draw [, ->] (0,-1.5) -- (0,5) node [below left, scale=0.3] {$y$};
             \draw [color=\currentcolor!80,, domain=-3.15:3.15] plot (\x,{-0.4*(\x)^2+4});
       \node [above, align=center] at (0,5) {Figura 2};
\end{tikzpicture}\end{center}\begin{center}\begin{tikzpicture}
[scale=0.5]
       \draw [help lines, secundario!30, step=.5] (-4,-5) grid (4,1.5);
       \draw [, ->] (-4,0) -- (4,0) node [below left, scale=0.3] {$x$};
       \draw [, ->] (0,-5) -- (0,1.5) node [below left, scale=0.3] {$y$};
       \draw [color=\currentcolor!80, , domain=-3.5:3.5] plot (\x,{-0.4*(\x)^2});
  \node [above, align=center] at (0,1.5) {Figura 3};
\end{tikzpicture}\end{center}
\end{multicols}
\item {} 
Para essa escolha, qual o significado dos valores de \(x\) e de \(y\)?

\item {} 
Que pontos do plano cartesiano são conhecidos, se juntarmos a escolha gráfica com os dados fornecidos as medidas dos túneis?

\item {} 
Com base em sua escolha do rascunho gráfico mais adequado e considerando os pontos conhecidos da parábola, qual forma da função quadrática resulta em maior quantidade de informações conhecidas?

\(\Box \; f(x)=ax^2+bx+c\)

\(\Box \; f(x)=a(x-p)^2+q\)

\(\Box \; f(x)=a(x-x_1)(x-x_2)\)

\item {} 
Quantos dados estão faltando para que seja conhecida a função que descreve esta parábola?

\item {} 
Com alguma coordenada ainda não utilizada desta curva, determine a informação que falta para conhecer a função que descreve esta parábola.

\item {} 
Determine, segundo esse plano cartesiano, as coordenadas das extremidades das bases desses túneis {[}Se julgar útil, use apenas a aproximação \(\sqrt{14}=3,75\){]}.

\item {} 
Com tudo que foi feito, qual a largura das bases desses túneis?

\end{enumerate}
\end{task}


\phantomsection\label{\detokenize{AF209-9:ativ-funcao-quadratica-angry-birds}}
\clearpage
\begin{task}{A jogada vencedora}

Vamos trabalhar aqui com um famoso jogo que simula lançamento de objetos. No caso, são “pássaros” caricaturados em formato de personagens de cinema que tem que impedir o plano dos “porcos verdes” de roubarem seus ovos e trazer destruição ao universo. A “variável” resistência do ar, por exemplo, não está incluída em boa parte das fases deste jogo.

Digamos que o programador de uma das fases decida, dentre todos os possíveis lançamentos, um que forneça a maior quantidade de pontos possível para a fase. Entendendo a tela como um plano cartesiano, o programador deve escolher a parábola que representará a “Jogada Vencedora”. A figura a seguir ilustra a situação.

\begin{figure}[H]
\centering
\capstart

\noindent\includegraphics[width=300bp]{{AB_Plano_Cartesiano}.png}
\caption{Imagem de divulgação.}\label{\detokenize{AF209-9:id6}}\end{figure}

Com a finalidade de inserir na programação a função que descreve a “Jogada Vencedora” o programador usou três coordenadas como referência: o pássaro e os dois “sóis”, cujas coordenadas estão destacadas a seguir.

\begin{figure}[H]
\centering

\noindent\includegraphics[width=300bp]{{AB_Coordenadas}.png}
\end{figure}
\begin{enumerate}
\item {} 
Quais são as coordenadas indicadas no gráfico pelo programador?

\item {} 
Quais os significados dos valores de \(x\) e de \(y\) neste contexto?

\item {} 
Das formas da função quadrática apresentadas a seguir, qual delas parece mais adequada diante das informações fornecidas?

\(\Box \; f(x)=ax^2+bx+c\)

\(\Box \; f(x)=a(x-p)^2+q\)

\(\Box \; f(x)=a(x-x_1)(x-x_2)\)

\item {} 
Substituido a origem na forma escolhida do item anterior, qual a conclusão?

\item {} 
Faça o mesmo para as outras duas coordenadas, mas considere também o que você concluiu no item anterior, e obtenha duas equações diferentes com variáveis \(a\) e \(b\).

\item {} 
Nas equações apresentadas no item anterior, uma tem o \(49\) e a outra tem o \(25\). Na que tem o \(49\), multiplique toda ela por \(25\) e, na outra, a que tem o \(25\), multiplique toda ela por \(49\). Feito isso, subtrai, membro a membro, as duas equações resultantes. Qual a conclusão?

\item {} 
Mais uma vez vamos pegar as equações do item ‘e’. Repare que uma tem um coeficiente \(7\) e a outra tem um coeficiente \(5\). Multiplique a que tem o \(7\) por \(5\) e a que tem o \(5\), por \(7\). Depois subtrai, membro a membro, as equações assim obtidas. Qual a conclusão?

\item {} 
Qual a função que o programador vai inserir como a “Jogada Vencedora”?

\end{enumerate}
\end{task}


\arrange{Vantagens de Cada Forma}
\label{\detokenize{AF209-10:organizando-as-ideias-vantagens-de-cada-forma}}\label{\detokenize{AF209-10::doc}}\label{\detokenize{AF209-10:sec-funcao-quadratica-org-ideias-muv-graf-para-lei}}
Nesta seção vimos como obter a lei de formação de algumas funções quadráticas através de alguns dados iniciais. A forma escolhida para a lei de formação depende de cada informação dada, mas isso pode ser generalizado conforme passamos a apresentar.

\textbf{Uso da forma} \(f(x)=a(x-p)^2+q\)

Essa forma necessita de apenas duas coordenadas da parábola: o vértice que fornece \(p\) e \(q\), e uma outra coordenada qualquer para que seja montada a equação que fornece o \(a\).

Supondo que essa outra coordenada é \((\kappa,\lambda)\) teremos:
\begin{equation*}
\begin{split}a \cdot ( \kappa -p)^2+q= \lambda & \Rightarrow a \cdot ( \kappa -p)^2 = \lambda -q \\
                    & \Rightarrow a = \frac{\lambda -q}{( \kappa -p)^2}\\\end{split}
\end{equation*}
Assim, fica concluida a tarefa de determinar a lei de formação procurada.

\textbf{Uso da forma} \(f(x)=a(x-x_1)(x-x_2)\)

Aqui há a necessidade de três coordenadas, sendo os dois \textbf{zeros da função}, ou seja, ambos os valores devem existir e precisam ser diferentes; a terceira coordenada pode ser outra qualquer, incluindo o vértice. Contudo, dispondo do vértice a técnica anterior resulta num caminho mais rápido.

Aqui, mais uma vez, o \(a\) fica sendo o valor desconhecido a ser determinado com a terceira coordenada citada, vamos supor que ela seja \((\kappa,\lambda)\). Assim,
\begin{equation*}
\begin{split}a\cdot (\kappa -x_1) \cdot (\kappa -x_2) = \lambda \Rightarrow a= \frac{\lambda}{(\kappa-x_1) \cdot (\kappa-x_2)}\end{split}
\end{equation*}
\textbf{Uso da forma} \(f(x)=ax^2+bx+c\)

Se as coordenadas fornecidas não apresentarem nenhuma das particularidades relatas nos casos anteriores, o caminho será o apresentado aqui. A utilização dos símbolos para representar os valores conhecidos ficariam pouco atrativo e não apresentariam uma simplificação digna de nota. Por esse motivo vamos seguir com um exemplo.

\def\currentcolor{cor1}
\begin{answer}{Exemplo}
{
\begin{enumerate}
\item Fazendo $t=0$ temos $h(0)=1$, portanto a bola foi laçada de uma altura de $1$ m.
\item Para $h(t)=0$, temos $-t^2+7t+1=0\implies t=7\pm53\sqrt{2}$, que tem resultado positivo aproximado de $7{,}14$ segundos. Esse foi o tempo total da bola no ar.
\item A bola atinge sua altura máxima em $-72\cdot(-1)=3{,}5$ segundos. Logo, sua altura máxima foi $h(3{,}5)=-(3{,}5)2+7\cdot(3{,}5)+1=−534\cdot(-1)=13{,}25$ metros.
\end{enumerate}
}{1}
\end{answer}
\def\currentcolor{session4}
\begin{example}{}

Uma equipe técnica está analisando máquinas que arremessam bolas de tênis para decidir qual a mais adequada às necessidades de treinamento dos atletas dessa equipe.

Com instrumento adequado, sem vento, eles mediram a altura da bola em alguns instantes de tempo, conforme tabela.

\begin{table}[H]
\centering
\begin{tabu} to \textwidth{|c|c|c|c|}
\hline
\cellcolor{\currentcolor!80}\textcolor{white}{\textbf Tempo em segundos} & \(1\) & \(2\) & \(3\) \\
\hline
\cellcolor{\currentcolor!80}\textcolor{white}{\textbf Altura em metros} & \(7\) & \(11\) & \(13\) \\
\hline
\end{tabu}
\end{table}


Sem vento, podemos considerar a trajetória da bola ao longo do tempo como uma parábola. Logo, as três coordenadas apresentadas na tabela permitem a determinação da função quadrática que dá a altura da bola em função do tempo.

Não sabemos se um desses pontos é o vértice e, com certeza, nenhum deles é zero da função. Teremos que utilizar a forma \(f(x)=ax^2+bx+c\). Com os dados da tabela, obtemos três equações para \(a\), \(b\) e \(c\):
\begin{equation*}
\begin{split}a \cdot (1)^2+b \cdot (1)+c=7 \Rightarrow a+b+c=6 \;\;\;\;\;\; & [1] \\
a \cdot (2)^2+b \cdot (2)+c=11 \Rightarrow 4a+2b+c=11 \;\;\;\;\;\; & [2] \\
a \cdot (3)^2+b \cdot (3)+c=13 \Rightarrow 9a+3b+c=13 \;\;\;\;\;\; & [3] \\\end{split}
\end{equation*}
Subtraindo \([2]-[1]\) e \([3]-[2]\) obtemos outras duas equações, mas desta só para \(a\) e \(b\):
\begin{equation*}
\begin{split}3a+b=10-6 \Rightarrow 3a+b=4 \;\;\;\;\;\; & [4] \\
5a+b=12-10 \Rightarrow 5a+b=2 \;\;\;\;\;\; & [5] \\\end{split}
\end{equation*}
Subtraindo \([5]-[4]\) ou percebendo que \([5]\) é igual \(2a+(3a+b)=2\) chega-se na mesma conclusão, \(2a=-2 \Rightarrow a=-1\).

Substituindo esse resultado em \([4]\), temos que \(3 \cdot (-1)+b=4 \Rightarrow b=7\).

Por fim, ao usar estes valores em \([1]\) encontramos \((-1)+(7)+c=7 \Rightarrow c=1\).

Assim, a função quadrática que dá a altura \(h\) da bola em função do tempo \(t\) para o arremesso analisado é \(h(t)=-t^2+7t+1\).

Com essa função podemos determinar de que altura a bola foi lançada, quanto tempo ela ficou no ar e até altura máxima que atingiu. Tente!
\end{example}


\know{A Propriedade Refletora da Parábola}
\label{\detokenize{AF209-11:sub-funcao-quadratica-prop-refletora}}\label{\detokenize{AF209-11:a-propriedade-refletora-da-parabola}}
A \textit{Parábola} possui uma propriedade que é bastante utilizada na composição de alguns objetos do nosso cotidiano. Na realidade, a propriedade acaba sendo utilizada em uma superfície tridimensional obtida com base na parábola. Para a construção dessa superfície, rotacio-se uma semiparábola em torno do seu eixo de simetria, formando o chamado \textit{parabolóide de revolução}. Contudo, para simplificar a linguagem, vamos tratar esse parabolóide por \textbf{superfície parabólica}. Isto pode ser visto no vídeo \textit{Parabolóide de Revolução} ({\url{https://youtu.be/4Jk4T9oubDM}).

\begin{observationtitle}{A Propriedade Refletora da Parábola}

Se uma fonte de luz estiver fixada no foco da parábola que gerou uma superfície parabólica, todo raio que insidir sobre essa superfície será refletido  paralelamente ao eixo de rotação (eixo de simetria).

No sentido oposto, se os raios, ou mesmo ondas, emitidos de uma fonte externa, mas que cheguem na superfície parabólica pela direção de retas paralelas ao eixo de rotação dessa superfície, estes raios ou ondas, serão refletidos para o foco.
\end{observationtitle}

\paragraph{Aplicações}

Algumas das aplicações que relataremos aqui pode ser vista em vídeo através do programa \href{https://youtu.be/X59mM76CL\_g}{Isto é Matemática T03 E02}.
\begin{itemize}
\item {} 
\textbf{O Farol Automotivo}

\end{itemize}

Alguns faróis automotivos utilizam espelhos parabólicos para que os raios luminosos iluminem especificamente uma determinada região (no caso a rodovia), sem desperdiçar a emissão para regiões que não são de interesse luminoso do condutor do automóvel. Além disso, feixes não direcionados de luz, numa estrada escura, ofuscaria a visão do condutor. Portanto, além de um aproveitamento otimizado dos feixes luminosos, evita-se que eles prejudiquem a direção.

\begin{figure}[H]
\centering
\capstart

\noindent\includegraphics[width=300bp]{{lanterna}.jpg}
\caption{Farol automotivo}\label{\detokenize{AF209-11:id6}}\end{figure}

Para essa reflexão induzida, a lâmpada emisora dos raios luminosos é acoplada de tal maneira que a emissão de luz ocorra no foco da superfície do espelho parabólico, fazendo com que os raios saiam todos na mesma direção, sempre paralelos ao eixo de simetria que contém o foco.
\begin{itemize}
\item {} 
\textbf{A Antena Parabólica}

\end{itemize}

Em regiões afastadas dos grandes centros, sinais, emitidos por ondas, de radio, televisão, internet, etc., chegam com baixa intensidade, é necessário que uma antena receptora tenha a capacidade de reunir uma quantidade significativa dessas ondas, concentrando-as num único ponto, fazendo assim o sinal ficar forte o suficiente para ser processado. Mais uma vez a propriedade é usada, criando uma antena no formato de uma superície parabólica refletora, colocando o receptor de sinal garantimos que todos as ondas que incidirem paralelamente ao eixo de simetria do parabolóide, se concentrem no foco (receptor de sinal).

\begin{figure}[H]
\centering
\capstart

\noindent\includegraphics[width=300bp]{{antena_parabolica_1}.jpg}
\caption{Antena Parabólica}\label{\detokenize{AF209-11:id7}}\end{figure}

Vale ainda observar que apesar das ondas emitidas não serem paralelas, o artefato funciona pois como foi dito a antena é útil quando instalada longe da antena emissora da onda, assim como os raios solares na superfície da terra, a distância faz com que boa parte dessas ondas acabem chegando paralelas o eixo de simetria da antena parabólica.
\begin{itemize}
\item {} 
\textbf{O Forno de Odeillo}

\end{itemize}

“Na foto a seguir vemos o \href{http://osfundamentosdafisica.blogspot.com.br/2010/06/forno-solar.html}{forno solar} de Odeillo, cuja potência é de 1MW e está instalado no sul da França, na região dos Pirineus. O espelho parabólico é constituído por \(9\,500\) pequenos espelhos planos. A temperatura atingida chega até \(3\,800\) ºC.”

\begin{figure}[H]
\centering
\capstart

\noindent\includegraphics[width=225bp]{{forno}.jpg}
\caption{\href{https://pixabay.com/pt/forno-solar-odello-odeillo-fran\%C3\%A7a-921116/}{O Forno de Odeillo} - Acesso em 21/02/2018}\label{\detokenize{AF209-11:id8}}\end{figure}

O forno na \fref{\detokenize{AF209-11:id8}} consegue derreter aço numa fração de minuto, ele utiliza os mesmo conceito da antena parabólica, pois como estamos bastante afastados do sol, os raios solares chegam praticamente paralelos à superfície terrestre, com isso eles refletem no espelho e se concentram no interior da “Câmara de concreto”.

Além das aplicações apresentadas, essa propriedade támbem é utilizada em alguns grandes telescópios, aparelhos radioterápicos e/ou ultrasônicos de uso da medicina, todos utilizando espelhos parabólicos.

\textbf{Entendendo o porquê da propriedade}

Para entender a propriedade, partimos de conceitos já conhecidos da física, como o princípio que \textit{“todo raio que incide sobre uma superfície refletora, o ângulo de incidência é igual ao ângulo de reflexão”}. Observe a figura que exemplifica o princípio: onde \(\alpha\) representa o ângulo de incidência e \(\beta\) o ângulo de reflexão.

\begin{figure}[H]
\centering

\begin{tikzpicture}[rotate=-20,scale=1.25]

\draw [ , color=terciario, fill=terciario!50] (8,4) -- (8.5,4.5) arc (45:80:0.7) -- cycle;
\draw [ , color=terciario, fill=terciario!50] (8,4) -- (7.5,3.5) arc (225:190:0.7) -- cycle;
\draw [color=destacado, , rotate around={10:(8,4)}] (8,4) -- (3,4) node [above right] {Raio refletindo};
\draw [color=destacado, , rotate around={-100:(8,4)}] (8,4) -- (3,4) node [ left] {Raio incidindo};
\draw [color=\currentcolor!80,  thick, domain=4:11] plot (\x,{((\x)^2)/8 -\x +4}) node [above left] {$f$};
\draw [color=atento,  thick, domain=4:11] plot (\x,{(\x-4)}) node [below] {$t$};
\node [ponto] at (8,4) {} node at (8,4) [below] {$P$};
\node [] at (8.75,5.35) {$\alpha$=35\textsuperscript{o}};
\node [] at (6.8,3.5) {$\beta$=35\textsuperscript{o}};
\node [] at (8.5,2.5) {$\alpha=\beta$};
\end{tikzpicture}
\caption{Curva \(f\), reta tangente \(t\) e raios}
\end{figure}


Imagine uma superfície parabólica refletora, esta superfície pode ser substituída pela curva \textit{parábola} (representada na figura acima pela curva \(f\)) que é a interseção dessa superfície com o plano que contém os raios (incidente e refletido) e o eixo da parábola (eixo de rotação da superfície parabólica).
É também fato que, o ângulo entre uma reta \(r\) e uma curva \(\lambda\), é por definição, o ângulo que \(r\) faz com a reta tangente à curva \(\lambda\), tangente esta que é traçada à partir do ponto em que \(r\) intersecta a curva \(\lambda\).

De posse dessas afirmativas, podemos dizer que a parábola divide o plano em duas regiões: a região interior à concavidade, que chamaremos de \textit{região focal}, por conter o foco, e a região exterior à concavidade, chamaremos de \textit{região não-focal}. Com isso, admita dois pontos \(P_1\) e  \(P_2\) contidos na reta \(r\) que contém \(P\) e é paralela à diretriz \(d\). Observe na figura a seguir onde,

\begin{figure}[H]
\centering

\begin{tikzpicture}[scale=.6, every node/.style={scale=3.3333}]

\draw [ ] (13,0) -- (0,0) node [above right, scale=0.3] {$d$};
\draw [color= secundario, densely dashed, very thin] (13,6.50) -- (0,6.50) node [above right, scale=0.3] {$r$};
\node [ponto, color=secundario] at (4,4) {};
\draw [color=\currentcolor!80,  , domain=0:12] plot (\x,{((\x)^2)/8 -\x +4});
\draw [thin] (10,0) -- (10,6.5);
\draw [ , color=atento, domain=5:12.33333] plot (\x,{1.5*\x-8.5});
\draw [, color=destacado, thin] (4,4) -- (12,6.5);
\draw [, color=destacado, thin] (4,4) -- (8,6.5);
\draw [thin] (4,4) -- (10,6.5);
\draw [thin] (8,0) -- (8,6.5);
\draw [thin] (12,0) -- (12, 6.5);
\node [left, scale=0.3] at (4,4) {$F$};
\node [above, scale=0.3] at (10,6.5) {$P$};
\node [below, scale=0.3] at (10,0) {$P'$};
\node [ponto, color=secundario] at (10,0) {};
\node [ponto, color=secundario] at (10,6.5) {};
\node [ponto, color=secundario] at (8,6.5) {};
\node [ponto, color=secundario] at (8,0) {};
\node [ponto, color=secundario] at (12,0) {};
\node [ponto, color=secundario] at (12,6.5) {};
\node [above, scale=0.3] at (8,6.5) {$P_{1}$} ;
\node [above, scale=0.3] at (12,6.5) {$P_{2}$};
\node [below, scale=0.3] at (8,0) {$P'_{2}$};
\node [below, scale=0.3] at (12,0) {$P'_{2}$};
\end{tikzpicture}
\caption{\(d(P_1,F)<d(P_1,d)\) e \(d(P_2,F)>d(P_2,d)\)}
\end{figure}

o ponto \(P_1\) está na região focal e o ponto \(P_2\) na região não-focal. Com isso, é fácil perceber que a distância de \(P_1\) ao foco é menor que a distância de \(P_1\) à reta diretriz \(d\), já o ponto \(P_2\), tem distância até o foco maior que a sua distância à reta diretriz \(d\).

Com isso podemos concluir a:

\textbf{Propriedade 2.1} Um ponto \(P_1\) está na \textit{região focal} de uma parábola, se e somente se, a sua distância ao foco for \textbf{menor} que a sua distância à reta diretriz.

\textbf{Propriedade 2.3} Um ponto \(P_2\) está na \textit{região não-focal} de uma parábola, se e somente se, a sua distância ao foco for \textbf{maior} que a sua distância à reta diretriz.

Dado um ponto \(P\) da parábola de foco F e diretriz \(d\), tracemos o triângulo \(PFP'\) (onde \(P'\) é a projeção ortogonal de \(P\) na reta \(d\)) e a reta \(t\) como sendo a reta bissetriz do ângulo \(F\widehat{P}P'=\alpha\), vamos mostrar que \(t\) é tangente à parábola.

\begin{figure}[H]
\centering

\begin{tikzpicture}[scale=0.6, every node/.style={scale=3.3333}]

\draw [ , color=terciario, fill=terciario!50] (10,6.5) -- (10,5.5) arc (270:236.3099324:1) -- cycle;
\draw [ , color=terciario, fill=terciario!50] (10,6.5) -- (9.076923076,6.115384618) arc (202.6198648:236.3099324:1) -- cycle;
\draw [ ] (4,4) -- (10,6.5) -- (10,0) -- cycle;
\draw [ ] (13,0) -- (0,0);
\node [ponto, color=secundario] at (4,4) {};
\draw [color=\currentcolor!80,  thick, domain=0:12] plot (\x,{((\x)^2)/8 -\x +4});
\draw [ ] (10,0) -- (10,6.5);
\draw [ , color=atento, domain=5:12.33333] plot (\x,{1.5*\x-8.5}) node [below right, scale=0.3] {$t$};
\node [left, scale=0.3] at (4,4) {$F$};
\node [right, scale=0.3] at (10,6.5) {$P$};
\node [above right, scale=0.3] at (10,0) {$P'$};
\node [below, scale=0.3] at (7,2) {$D$};
\node [ponto, color=secundario] at (7,2) {};
\node [color=terciario!20!black, scale=0.3] at (9,5.75) {$\alpha$};
\node [color=terciario!20!black, scale=0.3] at (9.6,5.3) {$\alpha$};
\node [ponto, color=secundario] at (10,0) {};
\node [ponto, color=secundario] at (10,6.5) {};
\draw [ ] (9.8,3.25) -- (10.2,3.25);
\draw [rotate around={112.61986:(7,5.25)},  ] (6.8,5.25) -- (7.2,5.25);
\end{tikzpicture}
\caption{Reta \(t\) bissetriz de \(F\widehat{P}P'\)}
\end{figure}

Sendo \(D\) o ponto de intersecção da reta \(t\) com o lado \(FP'\), temos que a ceviana \(PD\) não apenas é bissetriz interna do triângulo, mas também mediana e altura, já que o triângulo \(FPP'\) é isósceles devido à definição de parábola \((PF=PP')\), logo podemos concluir que \(t\) é mediatriz do segmento \(FP'\).

Marquemos sobre \(t\) um ponto \(Q\) distinto de \(P\) onde sua projeção ortogonal sobre \(d\) seja \(Q'\), como mostra a figura a seguir:
\begin{figure}[H]
\centering

\begin{tikzpicture}[scale=0.6, every node/.style={scale=3.3333}]

\draw [, color=terciario, fill=terciario!50] (10,6.5) -- (10,5.5) arc (270:236.3099324:1) -- cycle;
\draw [, color=terciario, fill=terciario!50] (10,6.5) -- (9.076923076,6.115384618) arc (202.6198648:236.3099324:1) -- cycle;
\draw [] (4,4) -- (10,6.5) -- (10,0) -- cycle;
\draw [] (13,0) -- (0,0);
\node [ponto, color=secundario] at (4,4) {};
\draw [color=\currentcolor!80,  thick, domain=0:12] plot (\x,{((\x)^2)/8 -\x +4});
\draw [] (10,0) -- (10,6.5);
\draw [, color=atento, domain=5:12.33333] plot (\x,{1.5*\x-8.5}) node [below right, scale=0.3] {$t$};
\node [left, scale=0.3] at (4,4) {$F$};
\node [right, scale=0.3] at (10,6.5) {$P$};
\node [above right, scale=0.3] at (10,0) {$P'$};
\node [below , scale=0.3] at (7,2) {$D$};
\node [ponto, color=secundario] at (7,2) {};
\node [color=terciario!20!black, scale=0.3] at (9,5.75) {$\alpha$};
\node [color=terciario!20!black, scale=0.3] at (9.6,5.3) {$\alpha$};
\node [ponto, color=secundario] at (10,0) {};
\node [ponto, color=secundario] at (10,6.5) {};
\draw [] (9.8,3.25) -- (10.2,3.25);
\draw [rotate around={112.61986:(7,5.25)}, ] (6.8,5.25) -- (7.2,5.25);
\node [ponto, color=secundario] at (7.5,2.75) {};
\node [ponto, color=secundario] at (7.5,0) {};
\draw [dashed, color=secundario, thin] (4,4) -- (7.5,2.75) -- (10,0);
\draw [dashed, color=secundario, thin] (7.5,2.75) -- (7.5,0);
\node [below, scale=0.3] at (7.5,0) {$Q'$};
\node [right, scale=0.3] at (7.5,2.75) {$Q$};
\end{tikzpicture}
\caption{\((Q \in t)\)}
\end{figure}

Como \(Q\) está sobre a metriatriz \(t\), temos que:

\(FQ = P'Q > QQ'\)

(pois \(QQ'\) é cateto e \(P'Q\) é hipotenusa do triângulo \(QQ'P'\)).

Logo, pela \textit{propriedade 2.2} pode-se afirmar que o ponto \(Q\) está na \textit{região não-focal} da parábola, assim como qualquer outro ponto da reta \(t\), exceto \(P\) que é ponto da parábola. Com isso comcluímos que que a reta \(t\) é tangente à parábola no ponto \(P\).

Agora observe a figura a seguir fecharmos a conclusão à respeito da \textbf{Propriedade Refletora da Parábola}:
\begin{figure}[H]
\centering

\begin{tikzpicture}[scale=0.6, every node/.style={scale=3.3333}]

\draw [, color=terciario, fill=terciario!50] (10,6.5) -- (10,7.5) arc (90:56.3099324:1) -- cycle;
\draw [, color=terciario, fill=terciario!50] (10,6.5) -- (10,5.5) arc (270:236.3099324:1) -- cycle;
\draw [, color=terciario, fill=terciario!50] (10,6.5) -- (9.076923076,6.115384618) arc    (202.6198648:236.3099324:1) -- cycle;
\draw[, densely dashed, color=secundario] (4,-1) -- (4,10);
\draw [ , color = black, fill=secundario!50, opacity=0.5] (3.7,0) rectangle (4,0.3);
\draw [ , color = black] (3.7,0) rectangle (4,0.3);
\draw [, fill=secundario!50, opacity=0.5] (10,0) rectangle (10.3,-0.3);
\draw [] (10,0) rectangle (10.3,-0.3);
\draw [] (13,0) -- (0,0) node [above right, scale=0.3] {$d$};
\node [ponto, color=secundario] at (4,4) {};
\draw [color=\currentcolor!80,  thick, domain=0:12] plot (\x,{((\x)^2)/8 -\x +4});
\draw [] (10,-1) -- (10,6.5);
\draw [, color=destacado] (4,4)--(10,6.5)--(10,10);
\draw [, color=atento, domain=5:12.33333] plot (\x,{1.5*\x-8.5}) node [below right, scale=0.3] {$t$};
\node [left, scale=0.3] at (4,4) {$F$};
\node [right, scale=0.3] at (10,6.5) {$P$};
\node [above left, scale=0.3] at (10,0) {$P'$};
\node [color=terciario!20!black, scale=0.3] at (9,5.75) {$\alpha$};
\node [color=terciario!20!black, scale=0.3] at (10.4,7.7) {$\alpha$};
\node [color=terciario!20!black, scale=0.3] at (9.6,5.3) {$\alpha$};
\node [ponto, color=secundario] at (10,0) {};
\node [ponto, color=secundario] at (10,6.5) {};
\node [left, align=center, scale=0.3] at (4,9) {Eixo de \\ simetria};
\end{tikzpicture}
\caption{Propriedade Refletora da Parábola}
\end{figure}

\needspace{7.5em}
Na figura anterior, é fato que os ângulos representados por \(\alpha\) são todos iguais, devido aos fatos que:
\begin{enumerate}
\item {} 
\(t\) é bissetriz do ângulo \(FPP'\)

\item {} 
os ângulos entre as retas \(PP'\) e \(t\) são opostos pelo vértice.

\end{enumerate}

Portanto, todas as ondas emitidas de F, ao tocarem a superfície parabólica refletora partem paralelas ao eixo de simetria e analogamente, todas as ondas que chegam paralelas ao eixo de simetria, ao tocarem na superfície parabólica refletora, partem em direção ao foco.


\subsection{Será que é parábola?}
\label{\detokenize{AF209-11:sera-que-e-parabola}}\label{\detokenize{AF209-11:sub-funcao-quadratica-voce-sabia-catenaria}}
\textbf{A Catenária}

Um famoso problema da história do cálculo é a descoberta da relação que fornece as coordenadas de um fio suspenso no ar por dois pontos de apoio como, por exemplo, os fios de alta tensão de postes públicos de energia.

\begin{figure}[H]
\centering
\capstart

\noindent\includegraphics[width=200bp]{{Aalborg_power_lines}.jpg}
\caption{Foto de \href{https://commons.wikimedia.org/wiki/File:Aalborg\_power\_lines.jpg}{Heb} CC BY-SA.}\label{\detokenize{AF209-11:id9}}\end{figure}

O conhecimento dessa relação permite, por exemplo, calcular o seu comprimento para fins de planejamento e economia na execução de um projeto.

O conhecimento adquirido sobre as parábolas e a relação que ela tem com a queda dos corpos, nos conduz à certeza de que a forma desses fios suspensos por dois pontos de apoio é também uma parábola. Foi em \(1690\) que esse problema foi oficialmente lançado para a comunidade científica da época por Jakob Bernoulli \((\star 1654, \dagger 1705)\), através do \textit{Acta eruditorum}, jornal fundado por Leibniz \((\star 1646, \dagger 1716)\). Porém, antes disso, famosos como Leonardo da Vinci \((\star 1452, \dagger 1519 )\) e Galileu Galilei \((\star 1564,\dagger 1642)\) tentaram resolver esse problema, obtendo a conclusão de que tratava-se de uma parábola.

Após a divulgação do problema, três estudiosos se destacaram nesta que é considerada uma das soluções mais difíceis da história do cálculo: O irmão mais novo de Jakob, Johann Bernoulli \((\star 1667, \dagger 1748)\), Leibniz e Huygens (star 1629, dagger 1695). Considerando aspectos da mecânica eles concluiram que a curva em questão não era uma parábola! Foi Leibniz quem deu-lhe o nome de \textbf{catenária}, que do latim, vem de \textit{catena} que significa \textit{cadeia}.

Hoje, com os recuros computacionais gerados a partir desses e de outros fatos históricos, podemos verificar experimentalmente que,  de fato, o problema do fio suspenso por dois pontos de apoio não se resolve com uma parábola. A figura \textit{Parábola ou Catenária?} exibe um cordão comum suspenso por dois pontos, qual a curva que se sobrepõe perfeitamente no cordão? Esta animação pode ser manipulada em \url{https://ggbm.at/wGMsrZb3}


\textit{O telhado com forma de catenária é importante tanto esteticamente como funcionalmente, dá estabilidade, flexibilidade e firmeza a estrutura, sua forma tem qualidade acústica dispersando os ruídos rapidamente, algo de grande valor para um aeroporto, também sua forma evita alguns afeitos dos ventos.} (Saarinem apud Torres, 2004)

\begin{figure}[H]
\centering
\capstart

\noindent\includegraphics[width=200bp]{{Dulles_International_Airport}.png}
\caption{Dulles International Airport. Disponível em {\href{http://imarrero.webs.ull.es/sctm04/modulo1/10/ribanez.pdf}{Torres (2014)}}}\end{figure}

\begin{figure}[H]
\centering
\capstart

\noindent\includegraphics[width=200bp]{{Catenaria_Casa_Mila}.png}
\caption{Arco Catenário da casa de Milá. A catenária invertida é uma estrutura que se auto sustenta. Disponível em {\href{http://imarrero.webs.ull.es/sctm04/modulo1/10/ribanez.pdf}{Torres (2014)}}}\end{figure}

\begin{figure}[H]
\centering
\capstart

\noindent\includegraphics[width=200bp]{{Ponte-Bisantis-Catanzaro}.jpg}
\caption{Ponte Bisantis, Itália. Conhecida pelo nome do engenhero que a projetou: Viaduto Morandis. Disponível em \href{https://3.bp.blogspot.com/-EPxjzeb0KMs/UPUgsG5jn5I/AAAAAAAAtmY/VBjNknF0oEk/s400/Ponte-Bisantis-Catanzaro.jpg}{Morandis} .}\label{\detokenize{AF209-11:id13}}\end{figure}

Acesse Ponte Morandis (\url{https://www.geogebra.org/m/qezn7h4M}) para ver como a catenária descreve o arco desta ponte, mas a parábola não.

Segundo Talavera (2018), a catenária tem equação \(\displaystyle y=\frac{e^{ax}+ e^{-ax}}{2a}\), sendo \(e\) um número irracional tal que \(\exp \approx 2,71\), e \(a\) é uma constante não nula; Além disso, ela pode ser representada na forma do cosseno hiperbólico, \(y=a \cdot \cosh (\ \frac{x}{a} )\ +b-a\). A estimativa de erro da catenária em relação a parábola, ao termarmos fazê-las coincidir é da ordem de \(\frac{1}{16}\).

\textbf{A função real definida por} \(f(x)=x^{2^2}\) \textbf{, ou melhor,} \(f(x)=x^4\)
\begin{itemize}
\item {} 
\textbf{Análise Algébrica}

\end{itemize}

Queremos determinar se os pontos da curva \(h\) definida por \(h(x)=x^4\), de domínio real, é uma parábola.

Uma forma de construir tal determinação, é avaliando a coincidência de pontos entre essa curva e a parábola. Em outras palavras, queremos verificar as interseções entre essas curvas, uma descrita pela função \(f(x) = x^2\) e a outra descrita por \(h(x)=x^4\). Caso esse processo revele infinitos pontos em comum dessas curvas, teremos que tentar outro método, visto que só poderámos concluir que o gráfico de \(h\) é uma parábola se todos os pontos coincidirem. Por outro lado, se a quantidade de pontos em comum for finita, teremos a garantia de que \(h\) não descreve uma parábola. Explicada a metodologia, vamos para a prática:
\begin{equation*}
\begin{split}x^4 & = x^2 \\
x^4-x^2 & = 0 \\
x^2 \cdot (x^2-1) & = 0 \\
x^2 = 0 & \text{ ou } x^2-1=0 \\
x = 0 & \text{ ou } x^2=1 \\
x = 0 & \text{ ou } x = \pm 1 \\\end{split}
\end{equation*}
Ou seja, as funções de domínios reais e dadas por \(f(x)=x^2\) e \(h(x)=x^4\), só possuem três pontos em comum: \((0,0)\), \((-1,1)\) e \((1,1)\).

Resta ainda a dúvida, para o caso em que altera-se a função \(f\) pelo fator \(a>0\), conforme trabalhado em \hyperref[\detokenize{AF209-5:ativ-funcao-quadratica-graf-curva}]{O gráfico e a forma canônica}, \textbf{parte 1}. Será que algum valor de \(a>0\) faria com que o gráfico de \(h\) coincidisse com o de \(f\), revelando que o gráfico de \(h\) é uma parábola?… Vamos buscar os pontos em comum para essas funções:
\begin{equation*}
\begin{split}x^4 & = ax^2 \\
x^4-ax^2 & = 0 \\
x^2 \cdot (x^2-a) & = 0 \\
x^2 = 0 & \text{ ou }  x^2-a=0 \\
x = 0 & \text{ ou }  x^2=a \\
x = 0 & \text{ ou }  x = \pm \sqrt{a}\; \text{, já que }  a>0 \\\end{split}
\end{equation*}
E ainda assim, a quantidade de interseções está restrita a três pontos: \((0,0)\), \((-\sqrt{a},a^2)\) e \((\sqrt{a},a^2)\).

Com isso nossa conclusão é clara: Não há como obter uma função quadrática do tipo \(g(x)=ax^2\) que represente o gráfico de \(h(x)=x^4\), ou seja, \(h(x)=x^4\) \textbf{não é uma parábola!}
\begin{itemize}
\item {} 
\textbf{Análise Gráfica}

\end{itemize}

Seja \(f:\mathbb{R}\to\mathbb{R}\) uma função definida por \(f(x)=x^4\).

Ao preenchermos a tabela a seguir com as imagens dessa função podemos notar algumas características de funções quadráticas:


\begin{table}[H]
\centering
\begin{tabu} to \textwidth{|c|c|}
\hline
\thead
$\bm{x}$ & $\bm{f(x)}$ \\
\hline
\(-2\) & \(16\) \\
\hline
\(-3/2\) & \(81/16\) \\
\hline
\(-1\) & \(1\) \\ 
\hline
\(-1/2\) & \(1/16\) \\
\hline
\(0\) & \(0\) \\ 
\hline
\(1/2\) & \(1/16\) \\
\hline
\(1\) & \(1\) \\
\hline
\(3/2\) & \(81/16\) \\
\hline
\(2\) & \(16\) \\
\hline
\end{tabu}
\end{table}

Note que existe uma simetria em relação ao eixo das ordenadas, ou seja, temos que \(f(-x)=f(x)\). Além disso, \(f(x)\geq0\). Pode-se verificar essas propriedades no gráfico da função respresentado na figura a seguir:


\begin{figure}[H]
\centering

\begin{tikzpicture}[every node/.style={scale=3}, yscale=.3, xscale=.5,scale=.75]
\draw [, ->] (-7,0) -- (7,0);
\draw [, ->] (0,-1) -- (0,23);
\foreach \x in {-6,-4,-2,2,4,6} \node [below,scale=.3] at (\x,0) {\x};
\foreach \x in {2,4,...,22} \node [left,scale=.3] at (0,\x) {\x};
\foreach \x in {-6,-4,-2,2,4,6} \draw (\x,0.1) -- (\x,-0.1);
\foreach \x in {2,4,...,22} \draw (0.1,\x) -- (-0.1,\x);
\draw [ thick, color=\currentcolor!80, domain=-2.189938703:2.189938703] plot (\x,{(\x)^4}) node [below right,scale=.3] {$f$};
\node [ponto,scale=.1] at (0,0) {0};
\end{tikzpicture}
\caption{(\(f(x)=x^4\))}
\end{figure}

Porém, ao atender algumas propriedades específicas do gráfico de “\(y=x^2\)”, não a caracteriza como sendo uma parábola. Para respondermos a essa pergunta, usaremos a seguinte estratégia:

Vamos supor que o gráfico de \(f:\mathbb{R}\to\mathbb{R}\) definida por  \(f(x)=x^4\) seja uma \textbf{parábola}, ou seja, existe um ponto \(F=(0,p)\) e uma reta \(d:y=-p\) tal que: \(PF=Pd\), onde \(P=(x,x^4) \in f\), logo:
\begin{align*}
\sqrt{(x^4-p)^2+(x-0)^2}&=x^4-(-p)\\
(x^4-p)^2+x^2&=(x^4+p)^2\\
x^8-2px^4+p^2+x^2&=x^8+2px^4+p^2\\
\^2&=4px^4\\
1&=4px^2\\
p&=\frac{1}{4x^2}\\
\end{align*}
logo $d$ é definida por \(\displaystyle d:y=-\frac{1}{4x^2}\) o que não é uma reta e sim uma curva.

Como consequência, por não existir a reta diretriz não temos uma parábola.

Portanto o gráfico de $f$ \textbf{não é uma parábola}.

\begin{observation}{}

Para comprovarmos que \(f\) não é uma função quadrática, podemos utilizar o fato que \(f(x)=x^4\) não atende uma das propriedade das funções quadráticas. Por exmplo: Note que se escolhermos um subconjunto do domínio de \(f\) onde seus elementos estejam em Progressão Aritmética (P.A.), as diferenças entre as imagens desses elementos não forma uma P.A., portanto \(f\) não é uma função quadrática.

\begin{table}[H]
\centering
\begin{tabu} to \textwidth{|c|c|c|}
\hline
\thead
\(\bm{x}\) & \(\bm{f(x)=x^4}\) & Diferenças \\
\hline
\(-4\) & \(256\) & \(81-256=-175\) \\
\hline
\(-3\) & \(81\) & \(16-81=-65\) \\
\hline
\(-2\) & \(16\) & \(1-16=-15\) \\
\hline
\(-1\) & \(1\) & \(0-1=-1\) \\
\hline
\(0\) & \(0\) & \(1-0=1\) \\
\hline
\(1\) & \(1\) & \(16-1=15\) \\
\hline
\(2\) & \(16\) & \(81-16=65\) \\
\hline
\(3\) & \(81\) & \(256-81=175\) \\
\hline
\(4\) & \(256\) & \(\cdots\) \\
\hline
\end{tabu}
\end{table}

\end{observation}

\textbf{E se for} \(g(x) = \sqrt{x^2+1}\) \textbf{, é parábola?}
\begin{itemize}
\item {} 
\textbf{Análise Algébrica}

\end{itemize}

Da mesma forma que em \(f(x)=x^4\), a busca por pontos em comum tem um conjunto solução bem limitado. Analisando o ponto mínimo, concluí-se que o mínimo de \(\sqrt{x^2+1}\) é obtido pela raiz quadrada do mínimo de \(x^2+1\). Assim, supondo que \(g(x) = \sqrt{x^2+1}\) seja uma parábola, seu vértice será \(V(0,1)\). A função quadrática que tem esse ponto como vértice e \(1\) como valor mínimo, é \(f(x)=ax^2+1\), com \(a>0\). Vamos investigar os pontos em comum dessas curvas além de \((0,1)\):
\begin{equation*}
\begin{split}\sqrt{x^2+1} = ax^2+1 \Leftrightarrow (\sqrt{x^2+1})^2 = (ax^2+1)^2\end{split}
\end{equation*}
Como \(x^2+1>0\), para todo \(x \in \mathbb{R}\), temos:
\begin{equation*}
\begin{split}x^2+1 & =a^2x^4+2ax^2+1 \\
a^2x^4+(2a-1)x^2 & =0 \\
x^2(a^2x^2+2a-1) & =0 \\
x^2=0 \text{ ou } a^2x^2+2a-1 & = 0 \\
x=0 \text{ ou } x^2 & = \frac{1-2a}{a^2} \\\end{split}
\end{equation*}
\(x=0\) já era uma solução conhecida, mas \(x^2 = \frac{1-2a}{a^2}\) traz novidades. Perceba, que para que essa expressão exista, \(1-2a\) precisa ser maior ou igual a zero, ou seja, \(2a<1 \Leftrightarrow a< \frac{1}{2}\). Neste caso, obtemos mais dois valores para \(x\), que são \(\displaystyle x=\frac{\sqrt{1-2a}}{a}\) e \(\displaystyle x=- \frac{\sqrt{1-2a}}{a}\), pois \(0<a<\frac{1}{2}\).

Mais uma vez, a busca por pontos em comum gera, no máximo, três pontos em comum, para \(\displaystyle x \in \{- \frac{\sqrt{1-2a}}{a},0,\frac{\sqrt{1-2a}}{a} \}\), quando \(\displaystyle0<a<\frac{1}{2}\). Para \(\displaystyle a \geq \frac{1}{2}\) o número de valores possíveis para \(x\) cai para apenas um. Sendo assim, \(g(x) = \sqrt{x^2+1}\) \textbf{não é parábola}.
\begin{itemize}
\item {} 
\textbf{Análise Gráfica}

\end{itemize}

Seja \(g:\mathbb{R}\to\mathbb{R}\) uma função definida por \(g(x)=\sqrt{x^2+1}\).

Observe seu gráfico:
\begin{figure}[H]
\centering

\begin{tikzpicture}[every node/.style={scale=3.3333}, scale=.75]

\draw [color=secundario!30, densely dashed, very thin] (-6.5,-1) grid (6.5,6);
\draw [, ->] (-6.5,0) -- (6.5,0);
\draw [, ->] (0,-1) -- (0,6);
\draw [ thick, color =\currentcolor!80, domain=-6:6] plot (\x,{sqrt((\x)^2+1)});
\foreach \x in {-6,...,-1,1,2,...,6} \node [below, scale=0.3] at (\x,0) {\x};
\foreach \x in {1,...,5} \node [left, scale=0.3] at (0,\x) {\x};
\node [below left, scale=0.3] at (0,0) {0};
\node [below, scale=0.3] at (0,-1) {$f(x)=\sqrt{x^2+1}$};
\end{tikzpicture}
\caption{É uma parábola?}
\end{figure}

Podemos afirmar que o gráfico da função \(g\) é uma parábola?

A resposta é não, deixaremos como exercício para o leitor repetir o processo utilizado na parte 1, porém é fácil mostrar que \(g\) não é função quadrática, basta mostrarmos que \(g\) não atende a propriedade das funções quadráticas euniciada ao final da parte 1. Observe:
\clearpage
\def\currentcolor{cor1}
\begin{answer}{Exercícios}
{\exerciselist
\begin{enumerate}
\item 
\begin{enumerate}
\item {} 
\(x\): cabeceira  e  \(y\): lateral

Temos que \(2x + 2y = 4 \to y = 2 - x\)

Gasto é dado por \(10xy + 25 \cdot 2x + 30 \cdot 2y = 10x(2 - x) +50x +60(2 - x)\)

Gasto = \(120 +10x - 10x^2\)

\item {} 
O gasto é máximo para \(x=\frac{-10}{2x-10}=\frac{1}{2}\) m
\end{enumerate}
\end{enumerate}
}{1}
\end{answer}
\def\currentcolor{session3}

\begin{table}[H]
\centering
\setlength\tabulinesep{1mm}
\begin{tabu} to \textwidth{|c|c|c|}
\hline
\thead
\(\bm{x}\) & \(\bm{f(x)=\sqrt{x^2+1}}\) & Diferenças \\
\hline
\(-4\) & \(\sqrt{17}\) & \(\sqrt{10}-\sqrt{17} \approx -0,961\) \\
\hline
\(-3\) & \(\sqrt{10}\) & \(\sqrt{5}-\sqrt{10} \approx -0,926\) \\
\hline
\(-2\) & \(\sqrt{5}\) & \(\sqrt{2}-\sqrt{5} \approx -0,822\) \\
\hline
\(-1\) & \(\sqrt{2}\) & \(1-\sqrt{2} \approx -0,414\) \\
\hline
\(0\) & \(1\) & \(\sqrt{2}-1 \approx 0,414\) \\
\hline
\(1\) & \(\sqrt{2}\) & \(\sqrt{5}-\sqrt{2} \approx 0,822\) \\
\hline
\(2\) & \(\sqrt{5}\) & \(\sqrt{10}-\sqrt{5} \approx 0,926\) \\
\hline
\(3\) & \(\sqrt{10}\) & \(\sqrt{17}-\sqrt{10} \approx 0,961\) \\
\hline
\(4\) & \(\sqrt{17}\) & \(\cdots\) \\
\hline
\end{tabu}
\end{table}


Note que as diferenças não estão em progressão aritmética, o que a descaracteriza como função quadrática.


\exercise
\clearmargin
\clearmargin
\begin{answer}{Exercícios}
{\exerciselist
\begin{enumerate}\setcounter{enumi}{1}
\item \adjustbox{valign=t}
{
\begin{tikzpicture}[scale=.7]
\draw [, ->] (-3,0)--(5,0) ;
\draw [, ->] (0,-3)--(0,5) ;
\draw [, domain=-1.25:5] plot (\x,{(1/2.5)*(\x*\x)-1.5*\x+1}) node [above, color=destacado, scale=0.3] {$m>0$};
\draw [, domain=-2.5:4] plot (\x, {\x+1}) node [above, color=destacado, scale=0.3] {$m>0$};
\node [ponto, color=destacado, scale=0.3] at (0,1) {};
\node [left, color=destacado, scale=0.3] at (0,1) {$p$};
\end{tikzpicture}
}
\item 

\adjustbox{valign=t}
{
\begin{tikzpicture}[yscale=.2, scale=.5, every node/.style={scale=3}]
draw [thin, dashed, densely dashed, color=secundario] (0,80) -- (3,80) -- (3,0);
\draw [thin, dashed, densely dashed, color=secundario] (-6,0) -- (-6,-1) -- (3,-1) -- (3,0);
\draw [thin, dashed, densely dashed, color=secundario] (-8,0) -- (-8,80);
\draw [ , ->] (-9,0) -- (7,0) node [above left, scale=0.5] {$x$};
\draw [ , ->] (0,-5) -- (0,90) node [below right, scale=0.5] {$y$};
\draw [ thick, domain=-8:3, color=primario] plot (\x,{(\x)^2+12*\x+35});
\node [left,scale=0.35] at (0,80) {80};
\node [left,scale=0.35] at (0,35) {35};
\node [below right, scale=0.35] at (0,-1) {-1};
\node [above, scale=0.35] at (-6,0) {-6};
\node [below,scale=0.35] at (-8,0) {-8};
\node [below,scale=0.35] at (3,0) {3};
\node [ponto,scale=0.8] at (-8,3) {};
\node [ponto,scale=0.8] at (3,80) {};
\node [ponto,scale=0.8] at (0,35) {};
\node [ponto,scale=0.8] at (-6,-1) {};
\draw [ ] (4,80) -- (4,-1);
\draw [ ] (3.7,80) -- (4.3,80);
\draw [ ] (3.7,-1) -- (4.3,-1);
\node [right,  scale=0.35] at (4,39.5) {$80-(-1) = 81$};
\end{tikzpicture}
}

\item Gabarito: \textit{b)}. Seja \(c\) a velocidade constante da correnteza, \(2+c\) velocidade de subida e \(8-c\) velocidade de descida.

\(t(subida) + t(descida) = 10 min\)

\(\dfrac{d}{2+c}+\dfrac{d}{8-c}=600seg\)

\(d(c)=-60c^2+360c+960\)

\(x_v=3\) e \(f(3)= 1500\)
\end{enumerate}
}{1}
\end{answer}
\clearmargin
\begin{answer}{Exercícios}
{\exerciselist
\begin{enumerate}\setcounter{enumi}{4}
\item 
\adjustbox{valign=t}
{
\begin{tikzpicture}[scale=.75]
\draw [ thick] (0,0) rectangle (5,5);
\draw [ thick, fill=white] (-0.25,0) rectangle (0.25,3);
\node [below] at (2.5,0) {$y$};
\node [above] at (2.5,5) {$y$};
\node [left, scale=0.7] at (-0.25,1.5) {6};
\node [left, scale=0.7] at (0,4) {$x$};
\draw [thick] (5.5,0) -- (5.5,5);
\draw [thick] (5.4,0) --(5.6,0);
\draw [thick] (5.4,5) -- (5.6,5);
\node [above, rotate=270] at (5.5,2.5) {$x+6$};
\end{tikzpicture}
}

O perímetro do cercado é dado por: \(6+x+y+x+6+y\) .

Como o muro de 6m será aproveitado, tem-se que \(34=x+y+x+6+y\), ou seja \(y=14–x\).

A área do cercado é dada por \(A= (x + 6)y = (x + 6)(14 – x) = -x^2 + 8x + 84\), \(0 \leq x <14\) que pode ser representada graficamente  por um arco de parábola, com concavidade voltada para baixo e vértice no ponto de abscissa \(x_v=4\), que fornece o maior valor para a área. Portanto, o valor de \(y\) no cercado é \(y = 14 – x = 14 – 4 = 10\).

Logo, o cercado de maior área será o quadrado de lado igual a \(10m\).


\item Gabarito \textit{a)}. Note que \((0,0)\) e \((10,10)\) pertencem à reta \(y=x\) porém o ponto \((5,6)\) não pertence à ela, o que nos faz concluir que trata-se de uma função quadrática que passa pela origem, logo é da forma: \(y=ax^2+bx\), substituindo os pontos \((10,10)\) e \((5,6)\) encontramos \(a=-\dfrac{1}{25}\) e \(b=\dfrac{7}{5}\).


\item 
\begin{enumerate}[wide]
\item Um caminho é reconhecer que o domínio de está restrito a \(D \in [0,6]\) indicando um total de seis dias de infecção e, portanto, tempo em que a temperatura excede \(36^{\circ}C\), devido à \(a=-\frac{4}{9} <0\). Outro caminho é definir para o domínio da função os dias em que a temperatura é \(36^{\circ}C\), pois fora disso ele será maior, indicando o estado febril. Assim, \small
\begin{equation*}
-\frac{4}{9} \cdot D^2 + \frac{8}{3} \cdot D + 36 = 36 \iff -\frac{4}{9} \cdot D^2 + \frac{8}{3} \cdot D = 0 \iff -\frac{4}{3} \cdot D \left( \frac{D}{3}-2 \right) .
\end{equation*}\normalsize
Portanto, \(D=0\) ou \(D=6\), e nesse intervalo há febre.


\item \(p=\dfrac{0+6}{2}=3\). Logo, \(T(3)=-\dfrac{4}{9} \cdot (3)^2 + \dfrac{8}{3} \cdot (3) + 36 = 40 \, ^{o}C\).
\end{enumerate}
\end{enumerate}
}{1}
\end{answer}
\clearmargin
\begin{answer}{Exercícios}
{\exerciselist
\begin{enumerate}\setcounter{enumi}{8}
\item A área sombreada \(A\) em função de \(x\) é resultado da diferença entre a área do retângulo \(4 \times 8\) e os dois triângulos retângulos em branco. Assim, \(A(x) = 32 - \frac{8 \cdot x}{2} - \frac{(8-2x)(4-x)}{2} = 16+4x-x^2\). De onde vem que \(p=-\frac{4}{2 \cdot (-1)} = 2\), portanto \(A(2)=16+4 \cdot (2) - (2)^2 = 20\). Letra \textit{c)}.


\item Pela simetria do gráfico da parábola, os zeros da função são \(10\) e \(-6\). Daí, a função que tem como gráfico essa parábola é \(f(x)=a(x-10)(x+6)\). Como o ponto \((0,15)\) é ponto dessa parábola, temos ainda \(f(0)=a(0-10)(0+6)=15 \Leftrightarrow a=-\frac{15}{60}=- \frac{1}{4}\). Portanto a altura máxima atingida nesse arremesso foi \(f(2)=- \frac{1}{4} \cdot (2-10)(2+6) = - \frac{1}{4} \cdot -64 = 16\) m.
\end{enumerate}
}{1}
\end{answer}
\clearmargin
\begin{answer}{Exercícios}
{\exerciselist
\begin{enumerate}\setcounter{enumi}{9}
\item A função que fornece o custo total \(y\) em função das \(x\) unidades produzidas é uma função afim com coordenadas \((0,1500)\) e \((10,2100)\). Assim, temos \(y= \frac{2100-1500}{10-0} \cdot x + 1500\). Já a arrecadação \(A\) em função das \(x\) unidades agora vendidas, será \(A(x)=(220-x) \cdot x\) e o lucro \(L(x)=A(x)-y=220x-x^2-(60x+1500)\), portanto \(L(x) = -x^2 +160x -1500\) e a quantidade \(x\) que deve ser produzida e vendida para se ter o maior lucro possível será \(p=- \frac{160}{2 \cdot (-1)} = 80\) unidades.



\item Primeiro iremos encontrar os valores de \(t\) para os quais \(h(t)=14\) , com isso teremos: \(14 = 10 +5t - t^2\) logo: \(t^2-5t+4=0\) resolvendo encontramos: \(t_1=1\) e \(t_2=4\)

\begin{tikzpicture}[yscale=.5,scale=.75, every node/.style={scale=2.5}]
\draw [dashed] (1,14) -- (1,0);
\draw [dashed] (0,14) -- (4,14) -- (4,0);
\draw [ thick, color=primario, domain=0:6.531128874] plot (\x,{-(\x)^2+5*(\x)+10});
\draw [ thick, ->] (-1,0) -- (8,0) node [above left, scale=0.4] {$t$};
\draw [ thick, ->] (0,-1) -- (0,17) node [below right, scale=0.4] {$h(t)$};
\foreach \x in {1,2,...,7} \node [below, scale=0.4] at (\x,0) {\x};
\foreach \x in {2,4,...,16} \node [left, scale=0.4] at (0,\x) {\x};
\node [ponto] at (1,14) {};
\node [ponto] at (4,14) {};
\draw [ ] (1,-1.7) -- (4,-1.7);
\draw [ ] (1,-1.6) -- (1,-1.8);
\draw [ ] (4,-1.6) -- (4,-1.8);
\node [below, scale=0.4] at (2.5, -1.7) {$3s$};
\end{tikzpicture}

Letra \textit{a)}

\item \adjustbox{valign=t}
{
\begin{tikzpicture}[scale=0.75, every node/.style={scale=2}]
\draw [ thick, ->] (-1,0) -- (7,0) node [above left, scale=0.5] {$x$};
\draw [ thick, ->] (0,-1) -- (0,10) node [below right, scale=0.5] {$y$};
\draw [dashed, color=secundario] (0,8) -- (2,8) -- (2,0);
\draw [ thick, color=primario, domain=0:2] plot (\x,{4*\x});
\draw [ thick, color=destacado, domain= 2:6] plot (\x,{-(\x)^2+6*\x});
\node [ponto] at (0,0) {};
\node [ponto] at (2,8) {};
\node [ponto] at (6,0) {};
\node [below left, scale=0.5] at (0,0) {0};
\node [below, scale=0.5] at (2,0) {2};
\node [below, scale=0.5] at (6,0) {6};
\node [left, scale=0.5] at (0,8) {8};
\end{tikzpicture}
}
\end{enumerate}
}{1}
\end{answer}
\begin{answer}{Exercícios}
{\exerciselist
\begin{enumerate}\setcounter{enumi}{12}
\item Temos que a parábola tem equação \(y=-2x^2+8\), logo a base e a altura do retângulo são dadas respectivamente por: \(2x\) e \(-2x^2+8\), como o perímetro é dado por: \(2p=2(b+h)\), temos que \(2p(x)=-4x^2+4x+16\) cujo \(x_V=\frac{1}{2}=0,5\). Letra \textit{b)}.

\item Temos que: \(T(t) = 75t+20\) substituindo temos: \(48 = 75t+20\) logo \(t = 20\) min. Por outro lado temos quando for  retirada do forno  a uma temperatura  de 200ºC,  teremos:
\(T(t) = \frac{2}{125}t^2− \frac{16}{5}t+320\) substituindo temos: \(200 = \frac{2}{125}t^2− \frac{16}{5}t+320\) daí, \(t^2 – 200t + 7 500 = 0\)
Assim, \(t = 150\) minutos. Portanto, o tempo de permanência dessa peça no forno é de \(150 – 20 = 130\) minutos. Letra \textit{d)}.
\end{enumerate}
}{1}
\end{answer}
\clearmargin
\begin{answer}{Exercícios}
{\exerciselist
\begin{enumerate}\setcounter{enumi}{14}
\item Seja \(x\) e \(y\) representados na figura a seguir:

\begin{figure}[H]
\centering

\begin{tikzpicture}[scale=2, every node/.style={scale=2}]
\draw [fill=primario!70, color=primario!70] (0,0) -- (2.5,0) -- (2.5,1.5) -- (0,1.5) -- cycle;
\draw [fill=atento, color=secundario!50] (0,0) -- (1.666666,1.5) -- (2.5,1.5) -- (0.833333,0) -- cycle;
\draw [densely dashed, color=secundario] (0,0) -- (1.6666,1.5);
\draw [densely dashed, color=secundario] (0.83333,0) -- (2.5,1.5);
\node [above, scale=0.5] at (0,1.5) {D};
\node [below, scale=0.5] at (0,0) {A};
\node [below, scale=0.5] at (0.833333,0) {Q};
\node [below, scale=0.5] at (2.5,0) {B};
\node [above, scale=0.5] at (1.66666,1.5) {P};
\node [above, scale=0.5] at (2.5,1.5) {C};
\node [right, scale=0.5] at (2.5,0.75) {$x$};
\node [above, scale=0.5] at (2.0833333,1.5) {$y$};
\end{tikzpicture}
\end{figure}

Temos que \(2y+4x=800\) logo \(y=400-2x\) , daí temos a área em função de \(x\), dada por \(A(x)=y\cdot x=(400-2x)\cdot x=-2x^2+400\) portanto a área máxima é dada por \(A=\dfrac{-\Delta}{4a}=20000m^2\).


\item Seja \(k\) a constante de proporcionalidade de \(d=kv^2\), temos que:

\begin{align*}
32&=k \cdot (50 000)^2 \to k=\frac{32}{(50 000)^2}\\
d&=\frac{32}{(50 000)^2} \cdot (100 000)^2\\
d&=32 \cdot 4 = 128\text{ m}.
\end{align*}
\end{enumerate}
}{1}
\end{answer}
\clearmargin
\begin{answer}{Exercícios}
{\exerciselist
\begin{enumerate}\setcounter{enumi}{16}
\item a função \(f\) o valor de \(x_V=\frac{-b}{2a}=\frac{-(-6)}{2\cdot\frac{3}{2}}=\frac{6}{3}=2\).

Daí, temos que \(V=(2,0)\) substituindo essas coordenadas em \(f\) termos:

\(0=\frac{3}{2}\cdot2^2-6\cdot2+C\) o que resulta em \(C=6\). Letra \textit{e)}.


\item A função \(f\) representada pelo gráfico é dada por: \(f(x)=-\frac{1}{12}x^2+\frac{13}{12}x\) , onde seus zeros são: \(0\) e \(13\), logo o ponto procurado é \((13,0)\). Letra \textit{c)}.

\end{enumerate}
}{1}
\end{answer}
\clearmargin
\begin{answer}{Exercícios}
{\exerciselist
\begin{enumerate}\setcounter{enumi}{18}
\item Sendo \(f(x)=a(x+10(x-30)\) fazendo \(f(10)=200\) temos \(a=-\frac{1}{2}\) logo, \(f(0)=150\), letra \textit{d)}.

\item Sendo \(f(x)=ax^2+bx+c\) substituindo temos as equações:
\(a+b+c=3=4a+2b+c=5\)  e  \(9a+3b+c=1\), resolvendo temos:

\(a=-3\) , \(b=11\) e \(c=-5\) , logo \(f(x)=-3x^2+11x-5\), portanto \(f(2,5)=3,75\). Letra D.
\end{enumerate}
}{1}
\end{answer}
\clearmargin
\begin{answer}{Exercícios}
{\exerciselist
\begin{enumerate}[wide]\setcounter{enumi}{20}
\item Inserindo eixos cartesianos conforme a figura a seguir, teremos:

\begin{figure}[H]
\centering

\begin{tikzpicture}[every node/.style={scale=2.5}]
\draw [very thick, fill=secundario!70] (-0.5,-1.5) rectangle (0,2); 
\draw [very thick, fill=secundario!70] (6,-1.5) rectangle (6.5,2); 
\draw [very thick] (0,0) -- (6,0);
\draw [very thick](1,0) -- (1,0.905);
\draw [very thick](5,0) -- (5,0.905);
\draw [very thick] (2,0) -- (2,0.2);
\draw [very thick] (4,0) -- (4,0.2);
\draw (-0.8,0) -- (-0.8,2);
\node [left,scale=0.4] at (-0.8,1) {20m};
\draw [ , ->, color=red] (-0.6,0)--(7,0) node [above right, scale=0.5] {$x$};
\draw [ , ->, color=red] (0,-2.5)--(0,2.5) node [above right, scale=0.5] {$y$};
\draw [ , dashed, color=red] (0,2) -- (6,2);
\foreach\x in {1,...,5} \node [below,scale=0.4] at (\x,0) {\x0};
\node   [scale=0.4] [below right] at (0,0) {0};
\node [below left, scale=0.4] at (6,0) {60};
\draw [very thick, domain=0:6] plot (\x,{(1/4.5)*((\x)^2)-4/3*(\x)+2});
\node [ponto, color=red] at (0,2) {};
\node [ponto, color=red] at (1,0.905) {};
\node [ponto, color=red] at (2,0.21) {};
\node [ponto, color=red] at (3,0) {};
\node [ponto, color=red] at (4,0.21) {};
\node [ponto, color=red] at (5,0.905) {};
\node [ponto, color=red] at (6,2) {};
\end{tikzpicture}
\end{figure}

Os pontos fornecidos da função que representa o cabo em forma de arco são $(30,0)$, o vértice; $(0,20)$ e $(60,20)$. Usando o vértice fica $f(x)=a(x−30)2+0$. Pelo o ponto $(0,20)$, temos $f(0)=a(0-30)2=20\iff900a=20\iff a=145$. Como os pontos onde $x=10$ e $x=20$ são respectivamente simétricos de $x=50$ e $x=40$ vamos determinar as alturas dos apoios verticais somente dos dois primeiros $x$ citados e, por simetria, concluir os outros.

Para $x=10$ ou $x=50$, $f(10)=\dfrac{1}{45}(10-30)^2=\dfrac{400}{45}=\dfrac{80}{9}=8\cdot\dfrac{10}{9}=8$ m

Para $x=20$ ou $x=40$, $f(20)=\dfrac{1}{45}(20-30)^2=\dfrac{100}{45}=\dfrac{20}{9}=2\cdot\dfrac{10}{9}=2$ m

Para os quatro apois teremos $8+2+2+8=20$ m de um lado e $20$ m do outro, totalizando $40$ m.

Assim, o valor gasto com os apoios verticais será de $40\cdot500=20.000$ reais.

Obs.: Também é possível resolver o problema escolhendo o eixo x no topo da ponte. Assim, teríamos $V(30,−20)$, $(0,0)$ e $(60,0)$ como pontos conhecidos. Usando a forma $f(x)=a(x-x_1)(x-x_2)$ concluí-se ainda que os gastos com apoios verticais será R\$ $20.000{,}00$.


\item 
\begin{enumerate}[wide]
\item $15\cdot80=1.200$ reais
\item $10\cdot(80+5\cdot10)=1.300$ reais
\item Se o preço da pizza for $15-x$, a pizzaria arrecada $(15-x)(80+10x)$ em um dia, ou seja, sendo $A$ a arrecadação diária em função de $x$, teremos $A(x)=1200+70x−10x^2$. O $x$ do vértice será
\begin{equation*}
p=\frac{-70}{2\cdot(-10)}=1.200+70\cdot3{,}5-10\cdot(3{,}5)^2=1.322{,}5
\end{equation*}
Assim, o preço da pizza deve cair para $15-3{,}5=11{,}5$, ou seja, R\$ $11{,}50$

\item $A(3{,}5)=1.200+70\cdot3{,}5-10\cdot(3{,}5)^2=1.322{,}5$ reais por dia.
\end{enumerate}
\end{enumerate}
}{1}
\end{answer}

\begin{enumerate}
\label{\detokenize{AF209-E:sec-funcao-quadratica-exercicios}}\label{\detokenize{AF209-E:exercicios}}\label{\detokenize{AF209-E::doc}}
\item \textbf{(UFRJ)} Um fabricante está lançando a série de mesas  “Super 4”. Os tampos das mesas dessa série são retangulares e têm \(4\) metros de perímetro. A fórmica usada para revestir o tampo custa R\$ $10\text{,}00$ por metro quadrado. Cada metro de ripa usada para revestir as cabeceiras custa R\$ $25\text{,}00$ e as ripas para as outras duas laterais custam R\$ $30\text{,}00$ por metro.
\begin{center}\begin{tikzpicture}
Exercício 1
\begin{scope} [scale=0.35, every node/.style={scale=2.5}]
\draw [ thick] (0,0) -- (10,0) -- (17.320508,5.773502) -- (7.320509,5.773502) -- (0,0);
\draw [ thick] (0,-0.5) -- (10,-0.5) -- (17.320508,5.273502);
\draw [ thick] (0,0) -- (0,-0.5);
\draw [ thick] (10,0) -- (10,-0.5);
\draw [ thick] (17.320508,5.773502) --  (17.320508,5.273502);
\draw [->,  thick,] (13,8) -- (9.5,4.5) ;
\node [above, scale=0.4] at (13,8) {R\$ 10,00/$m^2$};
\draw [->,  thick] (16,-1) -- (13.14,2);
\node [below, scale=0.4] at (16,-1) {R\$ 25,00/m};
\draw [->,  thick] (3.5,-2.5) -- (6,-0.6);
\node [below, scale=0.4] at (3.5,-2.5) {R\$ 30,00/m};
\end{scope}
\end{tikzpicture}\end{center}\begin{enumerate}
\item {} 
Determine o gasto do fabricante para revestir uma mesa dessa série com cabeceira de medida x.

\item {} 
Determine as dimensões da mesa da série “Super 4” para a qual o gasto com o revestimento é o maior possível.

\end{enumerate}

\needspace{10em}
\item \textbf{(UFF)} Considerem  \(m\) , \(n\)  e  \(p\)  números reais e as funções reais  \(f\)  e  \(g\)  de variável real, definidas por \(f(x)= mx^2+nx+p\)   e   \(g(x) = mx + p\) .  A alternativa que melhor representa os gráficos de  \(f\)  e  \(g\) é:

\begin{multicols}{3}
\begin{enumerate}
\item
\begin{tikzpicture}
[scale=0.35, baseline=(current bounding box.north)]
       \draw [ ->] (-3,0)--(5,0) ;
       \draw [ ->] (0,-3)--(0,5) ;
\draw [ domain=-1.5:4] plot (\x,{-(1/2.5)*(\x*\x)+(1)*\x+3});
\draw [ domain=-3:4.5] plot (\x, {1/3*\x+3});
\end{tikzpicture}
\item
\begin{tikzpicture}
[scale=0.35, baseline=(current bounding box.north)]
\draw [ ->] (-3,0)--(5,0) ;
       \draw [ ->] (0,-3)--(0,5) ;
  \draw [ domain=-1.5:4] plot (\x,{-(1/2.5)*(\x*\x)+(1)*\x+3});
  \draw [ domain=-1:4.5] plot (\x, {\x-1});
\end{tikzpicture}
\item
\begin{tikzpicture}
[scale=0.35, baseline=(current bounding box.north)]
\draw [ ->] (-3,0)--(5,0) ;
       \draw [ ->] (0,-3)--(0,5) ;
  \draw [ domain=-1.25:5] plot (\x,{(1/2.5)*(\x*\x)-1.5*\x+1});
       \draw [ domain=-2.5:5] plot (\x, {-1/2*\x+1});
\end{tikzpicture}
\item
\begin{tikzpicture}
[scale=0.35, baseline=(current bounding box.north)]
\draw [ ->] (-3,0)--(5,0) ;
       \draw [ ->] (0,-3)--(0,5) ;
  \draw [ domain=-2.5:5] plot (\x,{(1/2)*(\x*\x)-1.25*\x-2});
       \draw [ domain=-2:5] plot (\x, {-0.83137146*\x+3});
\end{tikzpicture}
\item
\begin{tikzpicture}
[scale=0.35, baseline=(current bounding box.north)]
\draw [->] (-3,0)--(5,0) ;
       \draw [->] (0,-3)--(0,5) ;
       \draw [domain=-1.25:5] plot (\x,{(1/2.5)*(\x*\x)-1.5*\x+1});
  \draw [domain=-2.5:4] plot (\x, {\x+1});
\end{tikzpicture}
\end{enumerate}
\end{multicols}

\item \textbf{(PUC-RJ)} Considere a função \(f:[-8,3]\to\mathbb{R}\), definida por \(f(x)=x^2+12x+35\) . Então a imagem de \(f\) é um intervalo de comprimento:

\begin{enumerate}
\item 75
\item 78
\item 81
\item 83
\item 90
\end{enumerate}

\item \textbf{(UERJ)} Um barco percorre seu trajeto de descida de um rio, a favor da correnteza, com a velocidade de \(2\) m/s em relação à água. Na subida, contra a correnteza, retornando ao ponto de partida, sua velocidade é de \(8\) m/s, também em relação à água.

Considere que:
\begin{itemize}
\item {} 
o barco navegue sempre em linha reta e na direção da correnteza;

\item {} 
a velocidade da correnteza seja sempre constante;

\item {} 
a soma dos tempos de descida e de subida do barco seja igual a \(10\) min.

\end{itemize}

Assim, a maior distância, em metros, que o barco pode percorrer, neste intervalo de tempo, é igual a:

\begin{enumerate}
\item 1.250
\item 1.500
\item 1.750
\item 2.000 
\end{enumerate}

\needspace{10em}
\item \textbf{(UFF)} Um muro, com \(6\) metros de comprimento, será aproveitado como parte de um dos lados do cercado retangular que certo criador precisa construir. Para completar o contorno desse cercado o criador usará \(34\) metros de cerca.

Determine as dimensões do cercado retangular de maior área possível que o criador poderá construir.

\item \textbf{(ENEM-2014)} Um professor, depois de corrigir as provas de sua turma, percebeu que várias questões estavam muito difíceis. Para compensar, decidiu utilizar uma função polinomial \(f\), de grau menor que \(3\), para alterar as notas \(x\) da prova para notas \(y = f(x)\), da seguinte maneira:
\begin{itemize}
\item {} 
A nota zero permanece zero.

\item {} 
A nota \(10\) permanece \(10\).

\item {} 
A nota \(5\) passa a ser \(6\).

\end{itemize}

A expressão da função \(y = f(x)\) a ser utilizada pelo professor é

\begin{enumerate}
\item $\displaystyle y=-\frac{1}{25}x^2+\frac{7}{5}x$
\item $\displaystyle y=-\frac{1}{10}x^2+2x$
\item $\displaystyle y=\frac{1}{24}x^2+\frac{7}{12}x$
\item $\displaystyle y=\frac{4}{5}x^2+2x$
\item $y=$
\end{enumerate}

\item Um médico acompanha o estado febril de pacientes acometidos de uma determinada infecção. Ele conseguiu identificar dentre os casos analisados que, em geral, uma pessoa tem sua temperatura corporal \(T\), em função dos dias de infecção \(D\), dada por \(T(D)=-\frac{4}{9} \cdot D^2 + \frac{8}{3} \cdot D +36\), com \(0 \leq D \leq 6\). Considere que a temperatura saudável de uma pessoa seja de \(36\,^oC\). Com base nessas informações, responda as questões que seguem:

\begin{figure}[H]
\centering
\capstart

\noindent\includegraphics[width=200bp]{{Doctor_consults_with_patient_(5)}.jpg}
\caption{Foto do \href{https://commons.wikimedia.org/wiki/File:Doctor\_consults\_with\_patient\_(5).jpg}{National Cancer Institute}}\label{\detokenize{AF209-E:id1}}\end{figure}
\begin{enumerate}
\item {} 
Quanto tempo dura o estado febril de um paciente infectado e nas condições analisadas?

\item {} 
Qual a temperatura máxima que uma pessoa com esta infecção atinge em sua febre?

\end{enumerate}

\clearpage
\item Na figura retangular, fazendo-se o valor de \(x\) variar de \(0\) a \(4\), a área da região sombreada também varia. O valor máximo que essa área poderá ter é:

\begin{figure}[H]
\centering

\begin{tikzpicture}[scale=.75]

\draw [ultra thick, fill=black] (0,0) rectangle (8,4);
\draw [fill=white] (2,4) -- (8,4) -- (8,1) -- cycle;
\draw [fill=white] (0,0) -- (8,0) -- (8,1) -- cycle;
\node [below] at (4,0) {8};
\node [above] at (1,4) {$2x$};
\node [left] at (0,2) {4};
\node [right] at (8,0.5) {$x$};
\end{tikzpicture}
\end{figure}
\begin{enumerate}
\item 30
\item 24
\item 20
\item 18
\item 16
\end{enumerate}

\item A representação gráfica a seguir, representa um objeto arremessado de um prédio e que segue uma trajetória parabólica.

\begin{figure}[H]
\centering
\capstart

\noindent\includegraphics[width=200bp]{The_Headquaters_of_the_Institute_of_Applied_Computer_Science.jpg}
\caption{\href{https://commons.wikimedia.org/wiki/File:The\_Headquaters\_of\_the\_Institute\_of\_Applied\_Computer\_Science.jpg}{The Headquaters of the Institute of Applied Computer Science} - Polônia, CC BY-SA.}\label{\detokenize{AF209-E:id2}}\end{figure}

Sabendo que as medidas estão em metros, determine a altura máxima atingida por esse objeto, uma vez que essa altura foi alcançada a \(2\) metros do prédio.

\needspace{10em}
\item Uma fábrica tem o custo de sua produção descrito no gráfico a seguir.
\begin{figure}[H]
\centering

\begin{tikzpicture}[yscale=.75, every node/.style={scale=2}]

       \draw [->,  thick] (-0.5,0) -- (4,0) node [above left, scale=0.5] {$x$};
       \draw [ ->,  thick] (0,-0.5) -- (0,6) node [below right, scale=0.5] {$y$};
     \draw [ dashed, color=secundario] (0,4.2) -- (1,4.2) -- (1,0);
       \foreach \y in {3,4.2}  \draw  (0.1,\y) -- (-0.1,\y);
       \foreach \x in {1} \draw  (\x,0.1) -- (\x,-0.1);
     \draw [ thick, color=\currentcolor!80,domain=0:2.47] plot (\x,{1.2*\x+3});
       \node [left, scale=0.5] at (0,3) {1500};
       \node [left, scale=0.5] at (0,4.2) {2100};
       \node [below, scale=0.5] at (1,0) {10};
\end{tikzpicture}
\end{figure}

\(x\) representa a quantidade de unidades produzidas e \(y\) o custo total, em reais, para produzir essas quantidades.
Considere que o preço de venda das \(x\) unidades produzidas seja \(220 – x\); Lembre-se que o lucro é a diferença entre o que se arrecada e o gasto que se tem. Nessas condições, qual deve ser a quantidade \(x\) produzida para se obter o lucro máximo?

\item (\textbf{UERJ-2005}) Numa operação de salvamento marítimo, foi lançado um foguete sinalizador que permaneceu aceso durante toda sua trajetória. Considere que a altura \(h\), em metros, alcançada por este foguete, em relação ao nível do mar, é descrita por \(h = 10 + 5t - t^2\), em que \(t\) é o tempo, em segundos, após seu lançamento. A luz emitida pelo foguete é útil apenas a partir de \(14\) m acima do nível do mar. O intervalo de tempo, em segundos, no qual o foguete emite luz útil é igual a:
\begin{enumerate}
\item 3
\item 4
\item 5
\item 6
\end{enumerate}

\item (\textbf{UFRJ}) Considere a função \(y = f(x)\) definida por:
\begin{quote}

\(y = f(x) = \left\{ \begin{array}{rlll} 4x, & \text{se} & 0 \leq x \leq 2 \\ -x^2+6x, & \text{se} & 2 \leq x \leq 6 \\ \end{array} \right.\)
\end{quote}
\begin{enumerate}
\item {} 
Esboce o gráfico de \(y = f(x)\) no intervalo de \([0,6]\);

\item {} 
Para que valores de \(x\) temos \(f(x) = 5\) ?

\end{enumerate}

\clearpage
\item (\textbf{AFA}) O retângulo, com base no eixo das abcissas, está inscrito numa parábola, conforme figura abaixo. O valor de  \(x\)  que faz esse retângulo ter perímetro máximo é
\begin{figure}[H]
\centering

\begin{tikzpicture}
\begin{scope}[every node/.style={scale=10/6}, yscale=.4]
\draw [, fill=terciario!50] (-1.25,0) -- (-1.25,4.875) -- (1.25,4.875) -- (1.25,0) -- cycle;
\draw [ thick, ->] (-3,0) -- (3,0) node [above left, scale=0.6] {$x$};
\draw [ thick, ->] (0,-2) -- (0,10) node [below right, scale=0.6] {$y$};
\draw [ thick, color=\currentcolor!80, domain=-2.1:2.1] plot (\x,{-2*(\x)^2+8});
\node [ponto] at (-2,0) {};
\node [ponto] at (2,0) {};
\node [ponto] at (0,8) {};
\node [above left, scale=0.6] at (0,8) {8};
\node [above left, scale=0.6] at (-2,0) {-2};
\node [above right, scale=0.6] at (2,0) {2};
\node [ponto] at (1.25,4.875) {};
\node [ponto] at (-1.25,4.875) {};
\node [below, scale=0.6] at (1.25,0) {$x$};
\node [below, scale=0.6] at (-1.25,0) {-$x$};
\end{scope}
\end{tikzpicture}
\end{figure}
\begin{enumerate}
\item 1
\item 0,5
\item 0,25
\item 1,25
\end{enumerate}

\item (\textbf{ENEM-2010}) Nos processos industriais, como na indústria de cerâmica, é necessário o uso de fornos capazes de produzir elevadas temperaturas e, em muitas situações, o tempo de elevação dessa temperatura deve ser controlado, para garantir a qualidade do produto final e a economia do processo.
Em uma indústria de cerâmica, o forno é programado para elevar a temperatura ao longo do tempo de acordo
com a função:

\begin{equation*}
T(t) = \left\{ \begin{array}{rlll}\displaystyle \frac{7}{5}t+20, & \text{para} & 0 \leq t < 100 \\\displaystyle \frac{2}{125}t^2- \frac{16}{5}t +320, & \text{para} & t \geq 100 \\ \end{array} \right.
\end{equation*}

em que \(T\) é o valor da temperatura atingida pelo forno, em graus Celsius, e \(t\) é o tempo, em minutos, decorrido desde o instante em que o forno é ligado.
Uma peça deve ser colocada nesse forno quando a temperatura for \(48 \,^{o}C\) e retirada quando a temperatura for \(200 \,^{o}C\).

O tempo de permanência dessa peça no forno é, em
minutos, igual a:
\begin{enumerate}
\item 100
\item 108
\item 128
\item 130
\item 150
\end{enumerate}
\clearpage

\item (\textbf{UERJ - 2010}) Um terreno retangular tem \(800\) m de perímetro e será dividido pelos segmentos \(\overline{PA}\) e \(\overline{CQ}\) em três partes, como mostra a figura.

\begin{figure}[H]
\centering

\begin{tikzpicture}
\begin{scope} [scale=2, every node/.style={scale=2}]
\draw [fill=\currentcolor!80, color=\currentcolor!80] (0,0) -- (2.5,0) -- (2.5,1.5) -- (0,1.5) -- cycle;
\draw [fill=atento, color=secundario!50] (0,0) -- (1.666666,1.5) -- (2.5,1.5) -- (0.833333,0) -- cycle;
\draw [densely dashed, color=secundario] (0,0) -- (1.6666,1.5);
\draw [densely dashed, color=secundario] (0.83333,0) -- (2.5,1.5);
\node [above, scale=0.5] at (0,1.5) {D};
\node [below, scale=0.5] at (0,0) {A};
\node [below, scale=0.5] at (0.833333,0) {Q};
\node [below, scale=0.5] at (2.5,0) {B};
\node [above, scale=0.5] at (1.66666,1.5) {P};
\node [above, scale=0.5] at (2.5,1.5) {C};
\end{scope}
\end{tikzpicture}
\end{figure}

Admita que os segmentos de reta \(\overline{PA}\) e \(\overline{CQ}\) estão contidos nas bissetrizes de dois ângulos retos do terreno e que a área do paralelogramo \(PAQC\) tem medida \(S\).
Determine o maior valor, em \(m^2\) , que \(S\) pode assumir.

\item (\textbf{UERJ - 2012}) Distância de frenagem é aquela percorrida por um carro do instante em que seu freio é acionado até o momento em que ele para. Essa distância é diretamente proporcional ao quadrado da velocidade que o carro está desenvolvendo no instante em que o freio é acionado.

\begin{figure}[H]
\centering

\begin{tikzpicture}
[scale=0.75, every node/.style={scale=2.5}]
\draw [ thick, color=\currentcolor!80, domain=0:7.3] plot (\x,{0.128*(\x)^2});
\draw [ thick, ->] (-0.5,0) -- (7.5,0) node [below right, scale=0.4] {$v$(km/h)};
\draw [ thick, ->] (0,-0.5) -- (0,7) node [below left, scale=0.4] {$d$(m)};
\draw [dashed] (0,3.2) -- (5,3.2) -- (5,0);
\node[ponto] at (5,3.2){};
\node [below, scale=0.4] at (5,0) {50};
\node [below left, scale=0.4] at (0,0) {0};
\node [left, scale=0.4] at (0,3.2) {32};
\end{tikzpicture}
\end{figure}

O gráfico abaixo indica a distância de frenagem \(d\), em metros, percorrida por um carro, em função de sua velocidade \(v\), em quilômetros por hora.

Admita que o freio desse carro seja acionado quando ele alcançar a velocidade de \(100\) km/h.

Calcule sua distância de frenagem, em metros.


\clearpage
\item (\textbf{ENEM - 2013}) A parte interior de uma taça foi gerada pela rotação de uma parábola em torno de um eixo \(z\), conforme mostra a figura.

\begin{figure}[H]
\centering

\begin{tikzpicture}
[yscale=0.333333, every node/.style={scale=3.3333}, scale=.75]
       \draw [ , ->] (-1,0) -- (6,0) node [below left, scale=0.3] {$x$ (cm)};
       \draw [ , ->,] (0,-8) -- (0,12) node [below left,scale=0.3] {$y$ (cm)};
       \draw  [domain=0:4, fill=destacado!70!black] plot (\x,{3/2*(\x)^2-6*\x+6});
       \draw [thin, fill=destacado!70!black] (2,6) ellipse (2cm and 1cm);
       \draw [ ] (2,9.375) ellipse (2.5cm and 1.5cm);
     \draw [ , domain=-0.5:4.5] plot (\x,{3/2*(\x)^2-6*\x+6});
       \draw [ ] (2,-6) ellipse (1cm and 0.8cm);
       \draw [,fill=white] (2,-6) ellipse (0.2 cm and 0.2cm);
       \draw [ , fill=white] (1.8,0) rectangle (2.2,-6);
       \draw [white,  ] (1.82,-6) -- (2.18,-6);
       \draw [dashed, ->, color=secundario] (2,6) -- (2,12) node [right, color=black, scale=0.3] {$z$ Eixo de rotacao};
       \node [ponto] at (0,6) {} node at (0,6) [left, scale=0.3] {$C$} node [ponto] at (2,0) {} node [below, scale=0.3] at (2,0) {$V$};
\end{tikzpicture}
\end{figure}

A função real que expressa a parábola, no plano cartesiano da figura, é dada pela lei \(\displaystyle f(x)=\frac{3}{2}x^2-6x+C\), onde \(C\) é a medida da altura do líquido contido na taça, em centímetros. Sabe-se que o ponto \(V\), na figura, representa o vértice da parábola, localizado sobre o eixo \(x\).
Nessas condições, a altura do líquido contido na taça, em centímetros, é
\begin{enumerate}
\item 1
\item 2
\item 4
\item 5
\item 6
\end{enumerate}

\needspace{10em}
\item (\textbf{FGV - 2014}) A figura a seguir mostra uma parte do gráfico da função quadrática que simula a trajetória de uma bala de canhão. Com os eixos e escala adequados, o canhão estava no solo, no ponto \((0,0)\) e a bala passou, em seguida, pelos pontos \((1,1)\) e \((4,3)\).
\begin{center}\begin{tikzpicture}[scale=.9,every node/.style={scale=2}]
\draw [dashed,, color=secundario] (0,1) -- (1,1) -- (1,0);
\draw [dashed, , color=secundario] (0,3) -- (4,3) -- (4,0);
\draw [ thick, color=\currentcolor!80, domain=0:4.2] plot (\x,{(-1/12)*(\x)^2+(13/12)*\x});
\draw [ thick, ->] (-0.5,0) -- (4.5,0) node [below, scale=0.6] {$x$};
\draw [ thick, ->] (0,-0.5) -- (0,3.5) node [left, scale=0.6] {$y$};
\foreach \x in {1,...,4} \node [below, scale=0.5] at (\x,0) {\x};
\foreach \y in {1,...,3} \node [left, scale=0.5] at (0,\y) {\y};
\foreach \x in {1,...,4} \draw [] (\x,0.05) -- (\x,-0.05);
\foreach \y in {1,...,3} \draw [] (0.05,\y) -- (-0.05,\y);
\draw [dashed] (0,1) -- (1,1) -- (1,0);
\draw [dashed] (0,3) -- (4,3) -- (4,0);
\node [ponto,color=secundario] at (1,1) {};
\node [ponto, color=secundario] at (4,3) {};
\node [below left, scale=0.5] at (0,0) {0};
\end{tikzpicture}\end{center}
A bala atingirá o solo no ponto
\begin{enumerate}
\item (11,0)
\item (14,0)
\item (13,0)
\item (12,0)
\item (15,0)
\end{enumerate}

\item (\textbf{FUVEST}) A trajetória de um projétil, lançado da beira de um penhasco sobre um terreno plano e horizontal, é parte de uma parábola com eixo de simetria vertical, como ilustrado na figura. O ponto \(P\) sobre o terreno, pé da perpendicular traçada a partir do ponto ocupado pelo projétil, percorre \(30m\) desde o instante do lançamento até o instante em que o projétil atinge o solo. A altura máxima do projétil, de \(200m\) acima do terreno, é atingida no instante
em que a distância percorrida por \(P\), a partir do instante do lançamento, é de \(10m\). Quantos metros acima do terreno estava o projétil quando foi lançado?

\begin{figure}[H]
\centering
\capstart

\noindent\includegraphics[width=150bp]{{Vertical_granite_cliff_at_sunset}.jpg}
\caption{Foto de \href{https://commons.wikimedia.org/wiki/File:Vertical\_granite\_cliff\_at\_sunset.jpg}{W. Carter} CC-BY.}\label{\detokenize{AF209-E:id3}}\end{figure}
\begin{enumerate}
\item 60
\item 90
\item 120
\item 150
\item 180
\end{enumerate}

\item (\textbf{ITA}) Os dados experimentais da tabela a seguir correspondem às concentrações de uma substância química medida em intervalos de \(1\) segundo.

\begin{table}[H]
\centering
\begin{tabu} to \textwidth{|c|c|c}
\hline
\thead
Tempo (s) & Concentração (moles) \\
\hline
\(1\) & \(3\text{,}00\) \\
\hline
\(2\) & \(5\text{,}00\) \\
\hline
\(3\) & \(1\text{,}00\) \\
\hline
\end{tabu}
\end{table}

Assumindo que a linha que passa pelos três pontos experimentais é uma parábola, tem-se que a concentração (em moles) após \(2\text{,}5\) segundos é:
\begin{enumerate}
\item 3,60
\item 3,65
\item 3,70
\item 3,75
\item 3,80
\end{enumerate}

\item Uma ponte será sustentada por dois cabos principais,  cujo formato consideraremos o de um arco parabólico. A ponte terá \(60\) m de comprimento e, a cada \(10\) m, haverá um apoio vertical, ligando a ponte com o cabo principal, estabilizando a estrutura. A figura abaixo exibe o esquema de um dos lados dessa ponte.
\begin{center}\begin{tikzpicture}
[every node/.style={scale=2.5}]
       \draw [very thick, fill=secundario!70] (-0.5,-1.5) rectangle (0,2);
       \draw [very thick, fill=secundario!70] (6,-1.5) rectangle (6.5,2);
       \draw [very thick] (0,0) -- (6,0);
       \draw [very thick](1,0) -- (1,0.905);
       \draw [very thick](5,0) -- (5,0.905);
       \draw [very thick] (2,0) -- (2,0.2);
     \draw [very thick] (4,0) -- (4,0.2);
       \draw (-0.8,0) -- (-0.8,2);
       \node [left,scale=0.4] at (-0.8,1) {20m};
       \node [left,scale=0.4] at (6.8,1) {\phantom{20m}};
       \foreach\x in {1,...,5} \node [below,scale=0.4] at (\x,0) {\x0};
       \node  [scale=0.4] [below right] at (0,0) {0};
     \node [below left, scale=0.4] at (6,0) {60};
       \draw [very thick, domain=0:6] plot (\x,{(1/4.5)*((\x)^2)-4/3*(\x)+2});
\end{tikzpicture}\end{center}
O valor do metro do apoio vertical é R\$ \(500\text{,}00\). Nessas condições, calcule o gasto com os apoios verticais para a construção dessa ponte. (Use a aproximação \(\frac{10}{9} = 1\)).

\item Uma pizzaria só vende pizza de tamanho individual. Ela cobra R\$ \(15\text{,}00\) por cada pizza e considera como um padrão a venda de \(80\) pizzas por dia.

\begin{figure}[H]
\centering
\capstart

\noindent\includegraphics[width=200bp]{Pizza_(17425076966).jpg}
\caption{Foto do \href{https://commons.wikimedia.org/wiki/File:Pizza\_(17425076966).jpg}{Nicola} CC BY.}\label{\detokenize{AF209-E:id4}}\end{figure}
\needspace{5em}
Um estudo foi contratato e realizado na vizinhaça dessa pizzaria, em lojas, escolas, escritórios e pontos de ônibus. A conclusão revelou que a cada real reduzido no preço da pizza, aumentaria em 10 a quantidade padrão de venda de pizzas por dia. Nessas condições, responda:
\begin{enumerate}
\item {} 
Quanto arrecada em um dia essa pizzaria, cobrando R\$ \(15\text{,}00\) por pizza?

\item {} 
Quanto arrecada em um dia essa pizzaria, cobrando R\$ \(10\text{,}00\) por pizza?

\item {} 
Qual é o valor ideal para o preço da pizza deste estabelecimento, de modo a tornar máxima a arrecadação?

\item {} 
Com o valor ideal, qual o ganho diário esperado?

\end{enumerate}
\end{enumerate}

\ifnum\aluno=1
\clearpage
\else
\notasfinais
\fi

\bibliographystyle{apalike-pt}
\bibliography{../Bibliografia/funcao-quadratica_bibliografia.bib}

\nocite{*}
 
\renewcommand\chapterillustration{./abertura-estatistica1}%Photo by Hoach Le Dinh on Unsplash, https://unsplash.com/photos/c8TWWQ5ZnUw?utm_source=unsplash&utm_medium=referral&utm_content=creditCopyText 
\renewcommand\chapterwhat{Especificidade do pensamento estatístico a partir de problemas. Conceitos: população e amostra, parâmetro e estimador. Variáveis estatísticas e suas classificações. Organização dos dados em tabelas de frequências. Representações gráficas adequadas para os diferentes tipos de variáveis. Noções básicas de amostragem.}
\renewcommand\chapterbecause{A Estatística está presente no mundo contemporâneo e chega aos cidadãos em todos os
meios de comunicação. Diariamente somos confrontados com informações estatísticas
sobre temas como Economia, Educação, Esportes, Saúde, Meio-Ambiente, entre outros.
Tais informações orientam decisões em nossas vidas pessoais e permitem-nos exercer
nossas responsabilidades como cidadãos. Um conhecimento básico de Estatística é
fundamental na formação do cidadão para que este possa, de forma competente, apreciar
e criticar argumentos baseados em dados.} 
\chapter{A natureza da Estatística}
\label{\detokenize{PE103:a-natureza-da-estatistica}}\label{\detokenize{PE103::doc}}

\begin{sphinxadmonition}{note}{Para o professor}

A Estatística está presente no mundo contemporâneo e chega aos cidadãos em todos os meios de comunicação e é a ferramenta por excelência no tratamento de modelagem de fenômenos aleatórios (não-determinísticos). Diariamente somos confrontados com informações estatísticas sobre temas que variam de Economia à Educação, de filmes a esportes, de comida à medicina, e de pesquisas de opinião a comportamento social. Tais informações orientam decisões em nossas vidas pessoais e permitem-nos exercer nossas responsabilidades como cidadãos. (Franklin, C. A., 2007, GAISE).

A produção de conhecimento - sempre em constante evolução e reavaliação - nas mais variadas áreas muitas vezes requer o conhecimento estatístico.

A relevância do raciocíno estatístico e do conhecimento para o efetivo funcionamento na sociedade da informação levou à introdução do termo \sphinxstylestrong{Letramento Estatístico}: ``A capacidade de compreender e avaliar criticamente resultados estatísticos que permeiam a vida diária,  acompanhada da capacidade de apreciar como o pensamento estatístico pode contribuir em decisões públicas e privadas, profissionais e pessoais.'' (Batanero, Borovcnik, 2016)

De acordo com De Veaux et al. (2008), o desafio para o estudante (e o professor) de Estatística introdutória é que, como na literatura e na arte, navegar por e dar sentido a exige não somente um conjunto de regras e axiomas, mas experiência de vida e ``senso comum''. São várias habilidades a serem trabalhadas e a maior parte delas exige capacidade de avaliação crítica em adição à manipulação matemática. A capacidade de avaliação crítica é adquirida com exemplos e experiência e isso demanda mais tempo.

\begin{figure}[H]
\centering

\noindent\sphinxincludegraphics[width=300bp]{{menina-globo}.png}
\end{figure}

\begin{sphinxadmonition}{note}{Objetivo geral}

Motivar o pensamento estatístico a partir de suas ideias fundamentais, a saber, população e amostra, parâmetro e estimador, distribuições empíricas de dados.
\end{sphinxadmonition}

\begin{sphinxadmonition}{note}{EM11MT03}

Realizar pesquisas, considerando: o planejamento, a discussão (se será censitária ou por amostra), a seleção de amostras, a elaboração e aplicação de instrumentos de coleta, a organização e representação dos dados (incluindo agrupamentos de dados em classe), a construção de gráficos apropriados (incluindo o histograma), a interpretação e a análise crítica apresentadas em relatórios descritivos.

\begin{sphinxadmonition}{note}{Pré-requisitos}

(EF08MT08)
Identificar, em gráficos de barras, colunas ou setores, divulgados pela mídia, as variáveis e seus valores, os resultados e os elementos constitutivos do gráfico (título, eixos, legenda e fonte), interpretando-os para analisar a adequação do gráfico ao tema e aos dados e para propor outras formas de comunicação dos resultados da pesquisa, tais como texto escrito ou outro tipo de gráfico.

(EF08MA23)
Selecionar razões, de diferentes naturezas (física, ética ou econômica), que justificam a realização de pesquisas amostrais e não censitárias, e reconhecer que a seleção da amostra pode ser feita de diferentes maneiras (amostra casual simples, sistemática e estratificada).

(EF09MT09)
Escolher e construir o gráfico mais adequado (colunas, setores, linhas e histogramas) para apresentar um determinado conjunto de dados de uma pesquisa, destacando aspectos como as medidas de tendência central para compor um relatório descritivo dos resultados.

(EF09MT10)
Planejar uma pesquisa amostral envolvendo tema da realidade social, definir a técnica de amostragem e a amostra, coletar, organizar e interpretar os dados, para comunicar os resultados por meio de relatório contendo texto escrito, avaliação de medidas de tendência central e da amplitude, tabelas e gráficos adequados construídos com o apoio de planilhas eletrônicas.

(EF09MA05)
Resolver e elaborar problemas que envolvam porcentagens, com a ideia de aplicação de percentuais sucessivos e a determinação das taxas percentuais, preferencialmente com o uso de tecnologias digitais, no contexto da educação financeira.

\sphinxstylestrong{Observação:} Como a BNCC ainda não entrou em vigor, os pré-requisitos acima, não necessariamente foram contemplados no Ensino Fundamental. Por essa razão, muitos deles serão abordados nesse capítulo e no capítulo de ``Medidas de Posição e Dispersão'' que dá sequência a esse capítulo.

Ao longo do capítulo utilizamos o termo \sphinxstyleemphasis{progressão aritmética}. O conhecimento deste tópico não é um pré-requisito. A conexão entre conceitos da Matemática é favorável para a visão do estudante sobre a Matemática como um todo.
\end{sphinxadmonition}

Este capítulo aborda os conteúdos de organização e representação dos dados (incluindo agrupamentos de dados em classes), a construção de gráficos apropriados (incluindo o histograma), a interpretação e a análise crítica apresentadas em relatórios descritivos destacados na habilidade. Os conteúdos:
\begin{enumerate}
\item {} 
realização de pesquisas considerando o planejamento, a discussão (se será censitária ou por amostra),

\item {} 
seleção de amostras,

\item {} 
elaboração e aplicação de instrumentos de coleta

\end{enumerate}

serão trabalhados de forma transversal ao  longo dos capítulos que tratam de Estatística, revisitando pré-requisitos previstos pela BNCC para o Ensino Fundamental.

As atividades propostas envolvem o uso da Estatística em diferentes situações, motivando o pensamento estatístico a partir de suas ideias fundamentais, a saber, população e amostra, parâmetro e estimador, distribuição e caracterizações da distribuição (posição e dispersão). Essas atividades não têm como objetivo o cálculo das medidas, mas a sua compreensão estrutural. Também serão trabalhados alguns distratores nessas atividades tais como:
\begin{enumerate}
\item {} 
explorar a diferença entre um gráfico de barras e um histograma;

\item {} 
destacar que a informação importante no gráfico de barras, adequado para variáveis qualitativas ou quantitativas discretas que assumem um conjunto moderado de valores, é a frequência na qual cada resposta ocorre,

\item {} 
destacar ainda que, para efeito de comparações múltiplas, a frequência deve ser relativa ou porcentagem, dado que diferentes conjuntos podem ter tamanhos diferentes.

\end{enumerate}

Neste capítulo serão apresentadas algumas atividades envolvendo a realização de pesquisas e coleta de dados e, no final do capítulo, será sugerida a realização de um projeto que deverá ser realizado ao longo de pelo menos três meses paralalelamente às aulas. O projeto envolverá a formulação de um problema a ser investigado (de preferência envolvendo outra disciplina), a definição da população, a construção de um questionário, a coleta de dados (amostra ou censo), a análise dos resultados obtidos construindo gráficos e calculando medidas-resumo e a confecção de relatório final. Na conclusão do projeto, o capítulo ``Medidas de Posição e Dispersão'' já terá sido trabalhado. Recomenda-se que essa atividade seja preferencialmente trabalhada no primeiro ou segundo ano do Ensino Médio, pois no último ano há maior limitação de tempo em razão dos vários exames a serem realizados pelos estudantes.

Neste capítulo incluem-se:
\begin{enumerate}
\item {} 
apresentação do diagrama de pontos introduzindo o conceito de distribuição empírica tanto em seu aspecto morfológico quanto variacional logo na primeira atividade revelando com isso a essência da Estatística;

\item {} 
reflexão sobre possíveis equivalências, do ponto de vista estatístico, de medidas-resumo com a finalidade de tomada de decisão sob incerteza;

\item {} 
utilização de uma base de dados reais de uma pesquisa já realizada;

\item {} 
discussão sobre a adequação entre tipo de variável e tipo de gráfico;

\item {} 
uso de tecnologia para a construção de gráficos;

\item {} 
conceituação de parâmetro e estimador, elementos cruciais na Estatística.

\item {} 
abordagem da estatística e seus problemas, privilegiando o pensamento estatístico para interpretação dos resultados, ao invés de um puro tratamento matemático dos cálculos que levam aos resultados.

\end{enumerate}

De acordo com Batanero e Borovnick (2016), mesmo que os métodos de análise de dados nessa fase do ensino envolvam somente calcular e interpretar porcentagens  ou medidas estatísticas simples, bem como interpretar vários tipos de gráficos, os autores sugerem que os estudantes apresentam problemas na compreensão dos conceitos e na relação desses conceitos para o contexto de modo a ter algum significado. Uma razão para essas dificuldades é que o ensino, em geral, foca sobre a aplicação de métodos em detrimento à interpretação de resultados em um dado contexto, buscando, assim, de forma equivocada, dar a estes um caráter determinístico.

Os distratores apresentados a seguir refletem a experiência dos envolvidos com o desenvolvimento desse capítulo.
\begin{enumerate}
\item {} 
Confundir o valor da variável com o da frequência.

\item {} 
Em caso de variável quantitativa discreta, considerar apenas os valores da variável apresentados na tabela ignorando as frequências.

\item {} 
Confundir gráfico de barras com o histograma.

\item {} 
Dificuldade de interpretar um resultado obtido via procedimento de inferência estatística.

\end{enumerate}

Apesar de variáveis e variação também aparecerem em muitas áreas da Matemática, a Matemática lida com variação funcional (determinística) enquanto que a Estatística lida com variação aleatória. Portanto, um objetivo da Educação Estatística é capacitar os estudantes a raciocinar sobre dados em contextos sob condições de incerteza, e distinguir entre raciocínio estatístico e raciocínio matemático. Além disso, a Estatística fornece métodos para identificar, quantificar, explicar, controlar e reduzir variação.

Para evitar o uso de vários termos com o mesmo significado: variação, variabilidade e dispersão, optamos por usar a palavra dispersão no livro.

Como estratégia pedagógica propomos usar um processo reflexivo baseado no pensamento estatístico.
\begin{enumerate}
\item {} 
Cálculos não serão valorizados, o mais importante neste capítulo é a compreensão dos conceitos.

\item {} 
As atividades deverão estar sempre bem caraterizadas a um problema a ser resolvido em um contexto específico.

\item {} 
O uso de recursos tecnológicos para a realização de cálculos e para a construção de gráficos é recomendado. Como recurso tecnológico, fez-se a opção pelo Geogebra e não pelo R, ainda que o R seja mais adequado para a análise estatística de dados. O Geogebra atende satisfatoriamente as demandas da abordagem e das atividades propostas e é o recurso digital que ampara o texto em outros eixos temáticos. Além disso, de maneira geral, os professores têm maior familiaridade com o Geogebra do que com o R, pois a maioria dos professores será de Licenciados em Matemática e não em Estatística.

\end{enumerate}

O capítulo está estruturado em três seções principais.

\sphinxstylestrong{Explorando 1} Proposição de atividades que ensejam uma reflexão sobre o papel central da variabilidade na Estatística como ferramenta fundamental no tratamento da incerteza. Na sequência,  apresentamos os conceitos básicos trabalhados nas atividades com discussão e algumas atividades complementares.

Na primeira atividade será trabalhada a noção de distribuição empírica, conceito chave para a construção de modelos de probabilidade. Em Estatística e Probabilidade, distribuição é uma coleção de propriedades de um conjunto de dados como um todo, não de um particular valor do conjunto. Uma distribuição consiste de todos os valores diferentes nos dados incluindo as frequências (ou probabilidades) associadas com cada valor. Variação e distribuição estão relacionadas a outras noções estatísticas fundamentais tais como ``centro'' ou ``posição'' (modeladas pela média, mediana, ou moda), dispersão (modeladas pelo desvio-padrão, ou variância, etc) e forma (por exemplo, bi-modal, uniforme, simétrica, assimétrica à direita, etc). Medidas de ``centro'' ou ``posição'' resumem a informação sobre uma distribuição, enquanto medidas de dispersão resumem a variabilidade no conjunto de dados. Cada valor de uma variável mostra algum desvio do ``centro''. Tais medidas serão trabalhadas no capítulo que dá sequência ao Capítulo ``A Natureza da Estatística'' (``Medidas de Posição e Dispersão''), mas elas já ocorrem nas atividades propostas nesse capítulo, pois média, mediana e moda são trabalhadas no Ensino Fundamental.

\sphinxstylestrong{Explorando -2 .} Proposição de atividades que envolvem analisar variáveis quantitativas contínuas: uma cujo objetivo é estudar a distribuição de frequências dos valores observados e, a outra, cujo objetivo é estudar seu comportamento ao longo do tempo. Na sequência, destacamos algumas propriedades do histograma.

\sphinxstylestrong{Aprofundando o assunto}
\begin{enumerate}
\item {} 
Tipos de seleção de amostras serão apresentados com um exemplo, lembrando que, na BNCC do Ensino Fundamental, está previsto trabalhar no oitavo ano com amostras probabilísticas aleatória simples, sistemática e estratificada. Após a descrição de alguns tipos de seleção de amostra, um exemplo é explorado.

\item {} 
Projeto a ser realizado ao longo de pelo menos três meses paralalelamente às aulas. O projeto envolverá a formulação de um problema a ser investigado (de preferência envolvendo outra disciplina), a definição da população, a construção de um questionário, a coleta de dados (amostra ou censo), a análise dos resultados obtidos construindo gráficos e calculando medidas-resumo e a confecção de relatório final. Na conclusão do projeto, o capítulo ``Medidas de Posição e Dispersão'' já terá sido trabalhado. Serão recomendados para o professor vários temas, caso os grupos ou a turma demandem. As etapas sugeridas para o desenvolvimento do projeto estão destacadas no documento da ABE (2015).

\end{enumerate}

Ao final do capítulo são sugeridos vídeos e projetos aplicados envolvendo  Estatística, várias páginas para pesquisar dados reais e exercícios incluindo questões do ENEM e Vestibulares, abordando os conteúdos desse capítulo. Nos exercícios serão tratados os distratores.

ABE (2015) ABE: Reflexões a respeito dos conteúdos de probabilidade e estatística na escola no Brasil - uma proposta. Disponível em: \textless{}\sphinxurl{https://goo.gl/OBtwpv}\textgreater{}. Acesso em: 18 ago. 2017.

Batanero, C., Burrill, G., \& Reading, C. (Eds.). (2011). Teaching statistics in school mathematics-challenges for teaching and teacher education: A joint ICMI/IASE study: the 18th ICMI study (Vol. 14). Springer Science \& Business Media.

Batanero, C., \& Borovcnik, M. (2016). Statistics and probability in high school. Springer.

Bussab, W. O. \& Morettin, P. A. (2017). Estatística Básica.  Saraiva. Nona edição.

Cordani, Lisbeth K. ``Estatística para todos.'' (2002). \textless{}\sphinxurl{http://www.estatistica.ccet.ufrn.br/cdee/wp-content/themes/cdee/arquivos/projeto02/oficina\_site\_educacao.pdf}\textgreater{} Acesso em: 22 set. 2017.

De Veaux, R. D., College, W., Velleman, P. F. (2008), Math is Music; statistics is literature (or why are there no six-year-old novelists?). Amstat news. pp 54-57.

IBGE (2017) \textless{}\sphinxurl{https://vamoscontar.ibge.gov.br/}\textgreater{} Acesso em: 29 ago. 2017.

Franklin, C. A. (2007). Guidelines for assessment and instruction in statistics education (GAISE) report: A pre-K\textendash{}12 curriculum framework. American Statistical Association.

Pfenning, N. (2011). Elementary Statistics: looking at the big picture. Cengage Learning.

Rossman, Allan J., and Beth L. Chance. (1998).  Workshop Statistics:: Discovery With Data and Minitab. Springer Science \& Business Media.
\end{sphinxadmonition}
\end{sphinxadmonition}


\explore{ a natureza da Estatística}
\label{\detokenize{PE103-0:explorando-compreendendo-a-natureza-da-estatistica}}\label{\detokenize{PE103-0::doc}}\label{\detokenize{PE103-0:cap-a-natureza-da-estatistica}}
Vivemos cercados de incertezas. A todo momento somos bombardeados por informações sobre pequisas científicas comprovando (estatisticamente) que tal substância causa uma patologia, ou sobre pesquisas de opinião, índices de pobreza, características sobre o envelhecimento da população, e outros temas de natureza incerta. Num mundo assim, é importante ter espírito crítico para informações sujeitas à incerteza a fim de poder interpretá-las e, quando necessário, poder escolher, entre diferentes opções, aquela que parece melhor diante da incerteza.  Nesse sentido, a Estatística é uma disciplina fundamental para todos os estudantes e, certamente, com grande responsabilidade para a formação crítica do cidadão, pois ela é usada nas mais variadas áreas do conhecimento tais como: Medicina, Economia, Política, Direito, Psicologia, Engenharia, Educação, entre outras.

Mas afinal o que é Estatística?
\begin{description}
\item[{Estatística\index{Estatística|textbf}}] \leavevmode\phantomsection\label{\detokenize{PE103-0:term-estatistica}}
Arte e ciência de coletar, analisar, apresentar e interpretar dados, para que se tomem decisões sob incerteza.
\end{description}

\begin{sphinxadmonition}{note}{Atividade}{Escolha do melhor fornecedor - Tomada de decisão}
\phantomsection\label{\detokenize{PE103-0:ativ-1-escolha-do-melhor-fornecedor}}

\sphinxstyleemphasis{Controle de Qualidade na Produção de Parafusos (Inspirada em ROSSMAN and CHANCE, 1998).}


Uma indústria precisa comprar parafusos de diâmetro 15 mm cuja variação aceitável é 15,0 mm ``mais ou menos'' 0,2 mm. Há quatro empresas, A, B, C e D, fornecedoras desses parafusos, que são vendidos em caixas com 60 unidades. Para decidir de qual fornecedor passará a comprar os parafusos, a empresa resolveu comprar e analisar uma caixa de cada um dos fornecedores. Os diâmetros das peças foram medidos com instrumento de alta precisão e os valores obtidos estão representados nos gráficos a seguir, em que cada círculo representa um parafuso posicionado sobre a abscissa correspondente à medida do seu diâmetro, medido em precisão de 0,02 mm.
\phantomsection\label{\detokenize{PE103-0:fig-parafusos}}


\begin{center}
\begin{tikzpicture}[x = 200, y=5, scale=1.2]

   \draw [help lines, lightgray, xstep=0.02,   ystep=1,xshift=-0.6] (14.383,0) grid (15.625,15) ;
   \draw [eixos] (14.37,0) -- (15.65,0);
   \foreach \x in {0,...,12}{
   \newcommand \y {\pgfmathparse{14.4+0.1* \x}\pgfmathprintnumber{\pgfmathresult}}
      \coordinate (A\x) at ($(14.4,0)+(0.1*\x,0)$);
      \draw ($(A\x)+(0,2pt)$) -- ($(A\x)-(0,2pt)$);
      \node [below] at ($(A\x)-(0,0.5ex)$) {\small \y} ;
   }
   \node[left] at (15.6,16) {Fornecedor A};
   \foreach \x/\y in {14.42/1,14.44/8,14.46/9,14.48/10,14.50/13,14.52/7,14.54/8,14.56/3,14.58/1}{
      \foreach \i in {1,...,\y}{
         \filldraw[color=primario] (\x,\i) circle (1.5pt);
      }}
\end{tikzpicture}
   
\begin{tikzpicture}
\begin{scope}[x = 200, y=5, scale=1.2]

   \draw [help lines, lightgray, xstep=0.02,   ystep=1,xshift=-0.6] (14.383,0) grid (15.625,15) ;
   \draw [eixos] (14.37,0) -- (15.65,0);
   \foreach \x in {0,...,12}{
   \newcommand \y {\pgfmathparse{14.4+0.1*  \x}\pgfmathprintnumber{\pgfmathresult}}
      \coordinate (A\x) at ($(14.4,0)+(0.1*\x,0)$);
      \draw ($(A\x)+(0,2pt)$) -- ($(A\x)-(0,2pt)$);
      \node [below] at ($(A\x)-(0,0.5ex)$) {\small \y} ;
   }
   \node[left] at (15.6,16) {Fornecedor B};
   \foreach \x/\y in {14.6/1,14.82/1,14.84/1,14.86/1,14.88/3,14.9/3,14.92/3,14.94/3,14.96/2,14.98/6,15/10,15.02/4,15.04/5,15.06/3,15.08/2,15.1/6,15.12/2,15.18/3,15.24/1}{
      \foreach \i in {1,...,\y}{
         \filldraw[color=primario] (\x,\i) circle (1.5pt);
      }
   }
   \end{scope}
\end{tikzpicture}
   
\begin{tikzpicture}
\begin{scope}[x = 200, y=5, scale=1.2]

   \draw [help lines, lightgray, xstep=0.02,   ystep=1,xshift=-0.6] (14.383,0) grid (15.625,15) ;
   \draw [eixos] (14.37,0) -- (15.65,0);
   \foreach \x in {0,...,12}{
   \newcommand \y {\pgfmathparse{14.4+0.1*  \x}\pgfmathprintnumber{\pgfmathresult}}
      \coordinate (A\x) at ($(14.4,0)+(0.1*\x,0)$);
      \draw ($(A\x)+(0,2pt)$) -- ($(A\x)-(0,2pt)$);
      \node [below] at ($(A\x)-(0,0.5ex)$) {\small \y} ;
   }
   \node[left] at (15.6,16) {Fornecedor C};
   \foreach \x/\y in {14.48/1,14.52/1,14.54/1,14.62/2,14.66 /2,14.7/2,14.72/1,14.78/2,14.8/2,14.84/2,14.88/2,14.9 /2,14.92/4,14.98/3,15/5,15.02/4,15.04/1,15.08/3,15.12 /3,15.16/4,15.18/1,15.2/2,15.22/1,15.3/1,15.32/1,15.38 /1,15.44/2,15.46/1,15.48/2,15.6/1}{
      \foreach \i in {1,...,\y}{
         \filldraw[color=primario] (\x,\i) circle (1.5pt);
      }
   }
   \end{scope}
\end{tikzpicture}
   
\begin{tikzpicture}
\begin{scope}[x = 200, y=5, scale=1.2]

   \draw [help lines, lightgray, xstep=0.02,    ystep=1,xshift=-0.6] (14.383,0) grid (15.625,15) ;
   \draw [eixos] (14.37,0) -- (15.65,0);
   \foreach \x in {0,...,12}{
   \newcommand \y {\pgfmathparse{14.4+0.1*  \x}\pgfmathprintnumber{\pgfmathresult}}
      \coordinate (A\x) at ($(14.4,0)+(0.1*\x,0)$);
      \draw ($(A\x)+(0,2pt)$) -- ($(A\x)-(0,2pt)$);
      \node [below] at ($(A\x)-(0,0.5ex)$) {\small \y} ;
   }
   \node[left] at (15.6,16) {Fornecedor D};
   \foreach \x/\y in {14.46/1,14.48/2,14.54/1,14.58/1,14.62/3,14.64/5,14.68/6,14.7/4,14.72/2,14.74/9,14.76/1,14.78/3,14.8/2,14.82/2,14.88/3,14.9/2,14.92/2,14.94/4,14.96/2,15/1,15.02/1,15.08/1,15.12/1}{
      \foreach \i in {1,...,\y}{
         \filldraw[color=primario] (\x,\i) circle (1.5pt);
      }
   }
\end{scope}
\end{tikzpicture}

\end{center}


\begin{enumerate}
\item {} 
Que informações foram usadas para a construção desses gráficos?

\item {} 
Quantos parafusos da caixa do fornecedor A atendem a especificação do comprador?

\item {} 
Para cada fornecedor, identifique a medida do diâmetro de maior \index{frequência}frequência.

\item {} 
Considerando cada um dos fornecedores, identifique o menor e o maior diâmetros observados.

\item {} 
Com base na sua resposta anterior, identifique os fornecedores cujos diâmetros dos parafusos observados variaram nos intervalos de menor \index{amplitude}amplitude e de maior amplitude.


\begin{description}
\item[{Amplitude\index{Amplitude|textbf}}] \leavevmode\phantomsection\label{\detokenize{PE103-0:term-amplitude}}
Em Estatística, a amplitude é definida como a diferença entre o maior e o menor valores observados.

\end{description}

\item De qual fornecedor você classifica o comportamento dos diâmetros dos parafusos como o de maior    \index{dispersão}dispersão? E o de menor dispersão?
\begin{description}
\item[{Dispersão\index{Dispersão|textbf}}] \leavevmode\phantomsection\label{\detokenize{PE103-0:term-dispersao}}
Segundo o dicionário Aurélio, dispersão significa (1) ato ou efeito de dispersar; (2) separação (de pessoas ou coisas) para diferentes partes.  Em Estatística, existem diferentes medidas de dispersão, dentre as quais, a amplitude.

\end{description}

\item Com base nesses dados, a(s) caixa(s) de qual(is)  fornecedor(es) apresenta(m) pelo menos um parafuso dentro das especificações do comprador?

\item Supondo que, para cada fornecedor, os comportamentos dos diâmetros dos parafusos sejam similares para as outras caixas, que fornecedor, com base nas especificações do comprador, você recomendaria ao comprador? Por quê?

\item Todos os parafusos da caixa do fornecedor escolhido no item anterior seriam aproveitados?
\end{enumerate}

\end{sphinxadmonition}



\begin{sphinxadmonition}{note}{Resposta}

\begin{enumerate}
\item {} 
Apenas as medidas dos diâmetros dos parafusos.

\item {} 
Nenhum, pois todos apresentam diâmetro inferior ao mínimo aceitável 14,8 mm.

\item {} 
Fornecedor A: 14,5 mm; fornecedor B: 15,0 mm; fornecedor C: 15,0 mm e fornecedor D: 14,74 mm.

\item {} 

\begin{savenotes}\sphinxattablestart
\centering
\begin{tabulary}{\linewidth}[t]{|T|T|T|}
\hline
\sphinxstylethead{\sphinxstyletheadfamily 
Fornecedor
\unskip}\relax &\sphinxstylethead{\sphinxstyletheadfamily 
Valor Mínimo
\unskip}\relax &\sphinxstylethead{\sphinxstyletheadfamily 
Valor Máximo
\unskip}\relax \\
\hline
A
&
14,42
&
14,58
\\
\hline
B
&
14,60
&
15,24
\\
\hline
C
&
14,58
&
15,60
\\
\hline
D
&
14,56
&
15,18
\\
\hline
\end{tabulary}
\par
\sphinxattableend\end{savenotes}

\item {} 
Menor amplitude: fornecedor A e maior amplitude: fornecedor C

\item {} 
Em relação à amplitude, menor dispersão: fornecedor A e maior dispersão: fornecedor C.

\item {} 
Fornecedores B, C e D.

\item {} 
Fornecedor B, pois é o que tem maior número de parafusos dentro das especificações.

\item {} 
Não, dois seriam descartados.

\end{enumerate}
\end{sphinxadmonition}


\begin{sphinxadmonition}{note}{Para refletir}{}

\begin{itemize}
\item Comente a estratégia usada para a obtenção dos dados dos fornecedores: as medidas obtidas refletem o comportamento das medidas de todos os parafusos produzidos pelo fornecedor? Seria razoável medir todos os parafusos fabricados por um fornecedor?

\item Que procedimento você usaria para confirmar a sua escolha inicial?

\item Em Controle de Qualidade, área de aplicação da Estatística na Indústria, é muito comum realizar comparações de diferentes produtos para fazer uma escolha ou verificar se os mesmos atendem às especificações apresentadas. Proponha um problema desse tipo com algum produto e indique a estratégia a ser usada e que medidas deveriam ser observadas.

\end{itemize}
\end{sphinxadmonition}


\phantomsection\label{\detokenize{PE103-0:ativ-2-comparacao-de-medicamentos}}
\begin{sphinxadmonition}{note}{Atividade}{ Comparação de medicamentos}
\begin{sphinxadmonition}{note}{Para o professor}
\sphinxstylestrong{Objetivos específicos}
\begin{itemize}
\item {} 
Construir diagrama de pontos

\item {} 
Analisar distribuições empíricas, ou seja, construídas a partir de dados experimentais, usando diagrama de pontos para comparar médias; mais especificamente, para comparar médias populacionais, verificando que nem sempre é possível concluir que estas são iguais quando as médias amostrais são diferentes.

\end{itemize}

\sphinxstylestrong{Observações e sugestões}

O objetivo principal dessa atividade é mostrar situações distintas nas quais ao comparar duas médias diferentes (resultantes de amostras), não é possível afirmar que na população, os parâmetros correspondentes sejam diferentes. Por exemplo, situações nas quais apesar de as médias amostrais serem diferentes, não podemos rejeitar a hipótese de que as médias populacionais são iguais, devido à dispersão resultante da amostra.

As respostas possíveis a serem relatadas no campo \sphinxstyleemphasis{para pesquisar} devem estar contidas nos campos sobre observações referentes a reações adversas, interações medicamentosas, etc. Em geral, as bulas sempre relatam situações que envolvem a observação de dados nesses casos e, algumas, apresentam a frequência na qual essas interações ou reações ocorrem. No entanto, pode ocorrer que uma particular bula não contenha informações do tipo solicitado.
\end{sphinxadmonition}


Deseja-se comparar três medicamentos, X, Y e Z, no tratamento da dor de cabeça. Para isso 60 pacientes com perfis similares foram separados aleatoriamente em três grupos de 20 cada. Para cada grupo,  será ministrado um dos medicamentos e observado o tempo de cura da dor de cabeça (em minutos). No quadro a seguir estão dispostos os dados obtidos.
\phantomsection\label{\detokenize{PE103-0:tabela-medicamentos}}

\begin{savenotes}\sphinxattablestart
\centering
\begin{tabulary}{\linewidth}[t]{|T|T|T|T|T|T|T|T|T|T|T|T|T|T|T|T|T|T|T|T|T|T|}
\hline
\sphinxstylethead{\sphinxstyletheadfamily 
medicamento
\unskip}\relax &\sphinxstartmulticolumn{20}%
\begin{varwidth}[t]{\sphinxcolwidth{20}{22}}
\sphinxstylethead{\sphinxstyletheadfamily tempo em minutos
\unskip}\relax \par
\vskip-\baselineskip\vbox{\hbox{\strut}}\end{varwidth}%
\sphinxstopmulticolumn
&\sphinxstylethead{\sphinxstyletheadfamily 
soma
\unskip}\relax \\
\hline
X
&
7
&
8
&
8
&
9
&
9
&
9
&
9
&
10
&
10
&
10
&
10
&
10
&
10
&
11
&
11
&
11
&
11
&
12
&
12
&
13
&
200
\\
\hline
Y
&
7
&
8
&
9
&
9
&
10
&
10
&
11
&
11
&
11
&
12
&
12
&
12
&
13
&
13
&
14
&
14
&
15
&
15
&
16
&
18
&
240
\\
\hline
Z
&
11
&
11
&
11
&
11
&
11
&
12
&
12
&
12
&
12
&
12
&
12
&
12
&
12
&
12
&
12
&
13
&
13
&
13
&
13
&
13
&
240
\\
\hline
\end{tabulary}
\par
\sphinxattableend\end{savenotes}
\begin{enumerate}
\item {} 
Organize as informações apresentadas no quadro acima em diagramas de pontos. Utilize uma folha de papel quadriculada, usando a mesma escala.

\item {} 
A partir dos diagramas construídos, identifique o grupo que apresentou maior dispersão dos tempos de cura com base na amplitude.

\item {} 
Determine os tempos médios de cura da dor de cabeça para cada substância.

\item {} 
A partir dos diagramas construídos e das médias calculadas, responda:
\begin{enumerate}
\item Entre X e Y, qual medicamento você escolheria? Por quê?
\item Entre X e Z, qual medicamento você escolheria? Por quê?
\item Entre Y e Z, qual medicamento você escolheria? Por quê?
\item A partir dos dados disponíveis, é possível garantir que algum medicamento é melhor que os outros? Por quê?
\end{enumerate}
\end{enumerate}
\end{sphinxadmonition}

\begin{sphinxadmonition}{note}{Para pesquisar}

Em casa, procure algum remédio e leia a sua bula. Em seguida, identifique informações que você considera como resultantes de estudos que envolvam Estatística e anote-as em seu caderno.
\end{sphinxadmonition}


\phantomsection\label{\detokenize{PE103-0:ativ-3-pesquisa-ibge-pnad}}
\begin{sphinxadmonition}{note}{Atividade}{ Pesquisa sobre a prática de esportes e atividade física}

\sphinxstyleemphasis{Fonte: IBGE, Suplemento da PNAD/2015}

\begin{sphinxadmonition}{note}{Para o professor}

\sphinxstylestrong{Objetivos específicos}
\begin{itemize}
\item {} 
Apresentar os conceitos de população e amostra.

\item {} 
Comparar os diferentes tipos de variáveis analisados em uma  pesquisa para adiante identificar variáveis qualitativas e quantitativas.

\end{itemize}

\sphinxstylestrong{Observações e sugestões}
\begin{itemize}
\item {} 
No item (a), espera-se que sejam indicadas algumas entre as seguintes variáveis: idade, sexo,  educação, trabalho, rendimento, se pratica ou não atividade física, modalidade da atividade para quem pratica, motivação para a prática de atividade física, local da prática, frequência da prática, duração da atividade, participação em competições, etc.

\item {} 
No item (b) deve-se informar as variáveis que assumem atributos (respostas não-numéricas) tais como sexo, prática de atividade física (sim ou não), modalidade da atividade física praticada, etc.

\item {} 
No item (c) deve-se informar as variáveis que assumem valores numéricos tais como idade, rendimento, duração da atividade física, etc.

\end{itemize}
\end{sphinxadmonition}

A Pesquisa Nacional por \index{Amostra}Amostra de Domicílios (PNAD), realizada pelo \sphinxhref{https://www.ibge.gov.br/estatisticas-novoportal/sociais/populacao/9127-pesquisa-nacional-por-amostra-de-domicilios.html}{IBGE}, obtém informações anuais sobre características demográficas e socioeconômicas da população, como sexo, idade, educação, trabalho e rendimento, e características dos domicílios. Com periodicidade variável, a PNAD obtém informações sobre migração, fecundidade, entre outras, tendo os domicílios como unidade de coleta da informação. Temas específicos abrangendo aspectos demográficos, sociais e econômicos também são investigados.

Um aspecto fundamental da Estatística praticado nessa pesquisa é a forma na qual a \index{amostra}amostra, subconjunto da \index{população}população, é selecionada. Essa seleção é cuidadosamente planejada de modo que seja adequado estender os resultados obtidos na amostra para a população.

Para que os resultados de uma amostra possam ser estendidos para a população, é necessário planejar com cuidado como a amostra será selecionada, pois o critério de seleção da amostra depende da estrutura da população. Por exemplo, para saber se o feijão cozinhando na panela está bem temperado, basta provar uma pequena colherada. Por quê?  Partimos do pressuposto de que todos os ingredientes foram bem misturados e, assim, a mistura é homogênea.

Quando dispomos de dados provenientes de um subconjunto da população sempre podemos descrever os dados nos restringindo apenas ao subconjunto. Se quisermos estender nossas conclusões para a população, será necessário o uso de outras tecnologias que permitam calcular as incertezas associadas a essas extensões.

Na PNAD 2015 foi realizada a investigação de um tema específico chamado ``Suplemento de Práticas de Esporte e Atividade Física'' no qual foram investigadas as pessoas moradoras de 15 anos ou mais de idade, \sphinxstylestrong{em seu tempo livre}, no período de referência de 365 dias, com o objetivo de quantificar aquelas que praticaram algum esporte ou atividade física no período considerado bem como a sua percepção quanto a isso. As informações levantadas nessa pesquisa foram obtidas por meio de um questionário no qual se perguntou:
\begin{itemize}
\item {} 
Se a pessoa moradora havia praticado esporte, e em caso afirmativo, a respectiva modalidade.

\item {} 
Independente da resposta anterior, também se perguntou se a pessoa praticava alguma atividade física que não considerava como esporte, informando, em caso positivo, também a modalidade.

\item {} 
Outras informações levantadas nessa pesquisa foram: motivação para a prática da atividade física, local onde é praticada a atividade, frequência na qual a atividade é praticada, duração da atividade; e a participação em competições.

\item {} 
Também foram levantadas informações sobre as pessoas que responderam que não praticavam atividade física. Perguntou-se o motivo de não o fazerem e se haviam praticado anteriormente, caso em que se perguntou a modalidade praticada, a idade em que parou de praticar e a causa da interrupção.

\item {} 
Além dessas informações, a pesquisa investigou também a avaliação da população sobre a opção de o poder público investir no desenvolvimento de atividades físicas e esportivas ou em outra área (saúde, educação, etc.) na vizinhança de seu domicílio.

\end{itemize}
\begin{enumerate}
\item {} 
Liste pelo menos oito \index{variáveis}variáveis investigadas na PNAD e no ``Suplemento de Práticas de Esporte e Atividade Física'' da PNAD 2015, baseando-se no texto apresentado.

\item {} 
Das variáveis citadas no item anterior, quais delas apresentam respostas não numéricas?

\item {} 
Das variáveis citadas no item a), quais delas apresentam respostas numéricas?

\end{enumerate}

Cada uma das unidades investigadas em um estudo estatístico é denominada um \index{elemento}elemento.  Assim, cada parafuso investigado é um elemento na atividade ``Escolha do fornecedor''; cada paciente observado é um elemento na atividade ``Comparação de medicamentos''; e cada domicílio e seus residentes são elementos na atividade da PNAD.

Cada característica observada de um elemento é uma \index{variável}variável estatística. Assim, a medida do diâmetro do parafuso é uma variável na atividade ``Escolha do fornecedor'', o tempo de cura da dor de cabeça é uma variável na atividade ``Comparação de medicamentos'' e, na atividade da PNAD, estão presentes várias variáveis estatísticas de interesse do domicílio e de seus residentes tais como local, número de cômodos, número de residentes; sexo, idade e rendimento dos residentes, etc.
\end{sphinxadmonition}

\begin{sphinxadmonition}{note}{Resposta}
\begin{enumerate}
\item {} 
Sexo. Idade. Educação. Trabalho. Rendimento. Prática de Atividade Física(AF). Modalidade da AF para quem pratica. Motivação para a AF. Local da Prática da AF. Duração da Prática da AF, etc.

\item {} 
Sexo. Educação. Trabalho. Prática de AF. Modalidade de AF. Motivação da Prática de AF. Local da Prática da AF.

\item {} 
Idade. Rendimento. Duração da Prática de AF.

\end{enumerate}
\end{sphinxadmonition}
\phantomsection\label{\detokenize{PE103-0:ativ4-analise-de-infograficos}}
\begin{sphinxadmonition}{note}{Atividade}{ Análise de infográficos}

\begin{sphinxadmonition}{note}{Para o professor}

\sphinxstylestrong{Objetivos específicos}
\begin{itemize}
\item {} 
Análise de infográficos. Mais especificamente, analisar infográficos construídos pelo IBGE com os resultados da pesquisa PNAD/2015 referente ao suplemento especial de Prática de Atividades Físicas.

\item {} 
Explorar possíveis associações sobre a prática de atividades físicas com outras variáveis envolvidas na pesquisa, tais como sexo, nível de instrução e rendimento.

\end{itemize}

\sphinxstylestrong{Observações e sugestões}

\sphinxstyleemphasis{Infográfico 1}

O item (b) pretende estimular a reflexão sobre o papel da inferência estatística. De fato, foi observada uma amostra de domicílios de algumas cidades brasileiras, mas como a amostra foi cuidadosamente planejada e a estrutura da população brasileira é conhecida, foi possível dar um passo maior e calcular uma estimativa da proporção das pessoas de 15 anos ou mais que praticam atividades físicas no Brasil. A porcentagem 37,9\%, realização numérica de um estimador, representa uma estimativa da proporção das pessoas de 15 anos ou mais que praticaram atividades físicas no Brasil (2015) (parâmetro). Observe que não foi realizado um censo para obter essa informação. Portanto, associada a essa estimativa existe uma margem de erro (valor correspondente à oscilação em torno da estimativa pontual) e um nível de confiança. Por exemplo, se o nível de confiança for 95\% isso implica que para cada 100 amostras de mesmo tamanho, em 95\% delas o parâmetro se situa no intervalo considerando a margem de erro. Claro que a margem de erro deve ser pequena e o nível de confiança alto na PNAD. Esses conceitos, margem de erro e nível de confiança, têm sido bem divulgados nas pesquisas eleitorais para o público em geral. Se for um ano de eleição, peça aos alunos para trazer resultados de pesquisas eleitorais incluindo a margem de erro e o nível de confiança.
Cabe também destacar que todas as proporções apresentadas na pesquisa são estimativas que devem ter pequena margem de erro com nível de confiança alto. Assim, pequenas diferenças nessas proporções devem ser olhadas com cuidado, não sendo possível afirmar que elas são diferentes.

O item (c) visa levar a uma reflexão sobre hábitos saudáveis. Por que achamos que a prática de atividades físicas é importante para a saúde de uma pessoa? Como essa conclusão foi obtida?

Os itens (d) e (e) têm como objetivo estudar possíveis associações entre duas variáveis qualitativas, a saber, sexo e prática de atividade física (d) e faixa etária e prática de atividade física (e). Observe que embora a idade seja uma variável quantitativa, quando ela é representada por faixas etárias ela se torna qualitativa.

É importante destacar, na análise desses gráficos, que o que se fez foi separar o conjunto de dados em subconjuntos como por exemplo, sexo feminino e sexo masculino e depois, observou-se a resposta sobre a prática de atividade física em cada subgrupo. Para efeito de comparação de grupos distintos, é importante trabalhar com a frequência relativa (ou porcentagem), pois os grupos podem ser de tamanhos diferentes e se os gráficos forem construídos com as frequências absolutas não será possível visualisar as relações entre as variáveis analisadas.

\sphinxstyleemphasis{Infográfico 2}

Os itens (a) e (b) têm como objetivo estudar possíveis associações entre duas variáveis qualitativas, a saber, grau de instrução e prática de atividade física (a) e rendimento per capita e prática de atividade física (b). Observe que, embora rendimento seja uma variável quantitativa, quando ele é representado por intervalos de rendimento, se torna variável qualitativa. Novamente aqui é importante destacar, na discussão, que o conjunto inteiro foi subdividido em subconjuntos ditados pelas categorias, grau de instrução ou faixas de rendimento, e que para cada subconjunto calculou-se a porcentagem de pessoas que praticam atividade física. Usar frequências absolutas não seria útil para comparar os diferentes grupos quando eles têm tamanhos diferentes.

\sphinxstyleemphasis{Infográfico 3}

Na análise do infográfico 3, cabe destacar que trata-se de um gráfico de barras típico representando a distribuição de frequências de uma variável qualitativa. É importante levar os alunos a perceber que para a variável modalidade, considerando o conjunto de todas as pessoas que responderam essa questão, calculou-se as porcentagens para cada tipo de atividade indicada. Discuta sobre a categoria \sphinxstyleemphasis{outras atividades} indicando que foram respostas com frequência muito pequena e, de fato, não faria sentindo ir listando uma a uma essas modalidades. Em geral, nesses casos, o que se faz é agregar as respostas com frequência muito pequena na categoria outras. Sugira ao aluno pesquisar no link dessa pesquisa para verificar se, no instrumento de coleta de dados, essa questão era aberta (resposta livre) ou fechada (com opções a serem assinaladas).

Na análise desse gráfico, deve-se destacar que a altura das barras correspondem às porcentagens (frequências relativas) na qual ocorreram e que a soma dessas porcentagens será 100\%.  Também cabe comentar que as barras devem ter larguras iguais, mas não existe nenhum lugar geométrico definido ao longo do eixo horizontal para as respostas da variável modalidade de prática neste gráfico, ou seja, podemos mudar a posição das diferentes modalidades. As barras, separadas, são equidistantes e foram organizadas por ordem de decrescente de frequência. Como só há um eixo numérico (frequência), comente que as barras podem ser tanto verticais, como horizontais e essa orientação determinará a orientação do eixo que representa as frequências no gráfico.

\sphinxstyleemphasis{Infográfico 4}

Na análise do infográfico 4, é importante destacar que foram usados dois tipos de gráficos diferentes  para representar variáveis qualitativas, mas ambos usam a mesma ideia, a saber, uma região é subdividida de maneira harmônica em sub-regiões (o círculo em setores circulares e o retângulo em retângulos menores de mesma largura contidos nele) cujas áreas em relação à área da região correspondem exatamente à frequência relativa (ou porcentagem) da categoria de resposta que a sub-região representa. Por exemplo, a área do setor em vermelho dividida pela área do círculo é 0,147 (ou 14,7\% da área do círculo). A área do retângulo verde dividida pela área do retângulo inteiro é 0,578 (ou 57,8\% da área do retângulo inteiro).  São duas formas de olhar como cada categoria de resposta aparece em relação ao todo.
\end{sphinxadmonition}

A seguir apresentaremos quatro \index{infográficos}infográficos, produzidos pelo IBGE (\sphinxhref{https://vamoscontar.ibge.gov.br/atividades/ensino-medio/9801-pesquisando-a-pratica-de-esportes-e-atividades-fisicas-no-brasil.html}{vamoscontar.ibge.gov.br}) usando os dados do Suplemento Prática de Esporte e Atividade Física da PNAD 2015.

Um \index{infográfico}infográfico é uma apresentação de informações integradas em textos sintéticos com dados numéricos e elementos gráficos e visuais tais como fotografias, desenhos, diagramas estatísticos, gráficos, etc.

\begin{figure}[H]
\centering
\capstart

\noindent\includegraphics[width=300bp]{PNAD_2015_Esportes_01quem2.png}
\caption{PNAD - Infográfico 1}\label{\detokenize{PE103-0:fig-infografico-pnad-1}}\label{\detokenize{PE103-0:id1}}\end{figure}
\begin{enumerate}
\item {} 
Segundo a pesquisa, qual a porcentagem de pessoas de 15 anos ou mais que praticaram algum esporte ou atividade física no período de um ano?

\item {} 
O título genérico deste infográfico, a saber, ``Quem mais pratica esportes e atividades físicas? - Percentual de pessoas de 15 anos ou mais que praticaram algum esporte ou atividade física-Brasil (2015)'', diz respeito à população brasileira de 15 anos ou mais ou à amostra coletada?

\item {} 
Com base nas recomendações médicas sobre a prática de atividades físicas para se ter boa saúde, como você avalia o resultado obtido na pesquisa para a população brasileira de 15 anos ou mais?

\item {} 
Considerando homens e mulheres separadamente, percebe-se alguma diferença com relação à prática de atividades físicas? Em caso afirmativo, descreva a(s) diferença(s) observada(s).

\item {} 
Considerando as faixas etárias discriminadas no infográfico, percebe-se alguma diferença com relação à prática de atividades físicas? Em caso afirmativo, descreva a(s) diferença(s) observada(s).

\end{enumerate}

\begin{figure}[H]
\centering
\capstart

\noindent\sphinxincludegraphics[width=300bp]{{PNAD_2015_Esportes_03instrrend2}.png}
\caption{PNAD - Infográfico 2}\label{\detokenize{PE103-0:fig-infografico-pnad-2}}\label{\detokenize{PE103-0:id2}}\end{figure}
\begin{enumerate}
\item {} 
Considerando os diferentes graus de instrução, percebe-se alguma diferença com relação à prática de atividades físicas? Em caso afirmativo, descreva a(s) diferença(s) observada(s).

\item {} 
Considerando as faixas de rendimento mensal per capita do domicílio, percebe-se alguma diferença com relação à prática de atividades físicas? Em caso afirmativo, descreva a(s) diferença(s) observada(s).

\end{enumerate}

\begin{figure}[H]
\centering
\capstart

\noindent\sphinxincludegraphics[width=300bp]{{PNAD_2015_Esportes_04principais}.png}
\caption{PNAD - Infográfico 3}\label{\detokenize{PE103-0:fig-infografico-pnad-3}}\label{\detokenize{PE103-0:id3}}\end{figure}
\begin{enumerate}
\item {} 
Qual foi a variável estudada no gráfico acima?

\item {} 
A variável estudada tem respostas de que tipo: numéricas ou não-numéricas?

\item {} 
Qual foi a resposta que apresentou a maior frequência?

\item {} 
O que você acha que representa a resposta ``Outros Esportes''?

\end{enumerate}

\begin{figure}[H]
\centering
\capstart

\noindent\sphinxincludegraphics[width=300bp]{{PNAD_2015_Esportes_05investimento}.png}
\caption{PNAD - Infográfico 4}\label{\detokenize{PE103-0:fig-infografico-pnad-4}}\label{\detokenize{PE103-0:id4}}\end{figure}
\begin{enumerate}
\item {} 
Qual a porcentagem de pessoas de 15 anos ou mais que concorda com que o poder público deva investir em atividades físicas ou desportivas?

\item {} 
Qual a opinião das pessoas de 15 anos ou mais que concordam que o poder público deve investir em atividades físicas ou esportivas com relação à prioridade de investimentos?

\item {} 
Entre as pessoas de 15 anos ou mais que não concordam que o poder público deve investir em atividades físicas ou esportivas, que área elas entendem como prioritária?

\item {} 
Podemos afirmar que 57,8\% das pessoas de 15 anos ou mais defendem que o poder público deve investir em Saúde?''

\end{enumerate}
\end{sphinxadmonition}

\begin{sphinxadmonition}{note}{Resposta}

\sphinxstylestrong{Infográfico 1}
\begin{enumerate}
\item {} 
37,9\%

\item {} 
População brasileira de 15 anos ou mais.

\item {} 
Não parece satisfatório. Vários estudos têm demonstrado que a prática de atividades físicas é fundamental para se ter boa saúde.

\item {} 
Sim. Entre os homens brasileiros de 15 anos ou mais, pouco mais de 40\% praticam atividade física; enquanto esse percentual para mulheres brasileiras de 15 anos ou mais é pouco menor do que 35\%.

\item {} 
Sim. Percebe-se uma diminuição dos percentuais de pessoas que praticam atividade física, conforme a idade aumenta. Na faixa de 15 a 17 anos temos mais de 50\%, na faixa de 18 a 24 anos temos um pouco menos do que 50\%, na faixa de 25 a 39 anos temos pouco mais de 40\%, na faixa de 40 a 59 anos temos mais de 30\% e na faixa 60 anos ou mais temos menos de 30\%.

\end{enumerate}

\sphinxstylestrong{Infográfico 2}
\begin{enumerate}
\item {} 
Sim, a porcentagem de pessoas de 15 anos ou mais que pratica atividade física cresce conforme o grau de instrução é maior.

\item {} 
Sim, a porcentagem de pessoas de 15 anos ou mais que pratica atividade física cresce conforme a faixa de rendimento per capita é maior.

\end{enumerate}

\sphinxstylestrong{Infográfico 3}
\begin{enumerate}
\item {} 
Modalidade de atividade física praticada.

\item {} 
Não-numéricas: futebol, natação, etc.

\item {} 
Futebol

\item {} 
Como as últimas modalidades discriminadas no gráfico apresentaram porcentagens muito pequenas (``ciclismo'', ``ginástica rítmica e artística'', ``lutas e artes marciais'', ``voleibol, basquetebol e handebol''), cerca de 2\%, a categoria outros esportes reuniu modalidades que ocorreram com porcentagens muito pequenas, não cabendo representá-las separadamente no gráfico. Observe que a última modalidade, antes de ``outros esportes'' já está reunida em mais de uma modalidade, a saber, ``voleibol, basquetebol e handebol''.

\end{enumerate}

\sphinxstylestrong{Infográfico 4}
\begin{enumerate}
\item {} 
73,3\%

\item {} 
Entre as pessoas que acham que se deva priorizar investimentos em atividades físicas, 91,1\% acha que o investimento deve ser para atividades físicas para as pessoas em geral, 8\% acha que deve ser para a formação de atletas e, o restante (0,9\%) respondeu outro tipo de prioridade.

\item {} 
Entre as pessoas que não concordam que o poder público deve investir em atividades físicas, 57,8\% acham que a prioridade deve ser Saúde, 21,3\% acham que a prioridade deve ser Segurança, 16,5\%, acham que a prioridade deve ser Educação e, o restante (4,4\%) respondeu outros tipos de prioridade.

\item {} 
Não, de fato, são 57,8\% de 14,7\% o que dá cerca de 8,5\% das pessoas de 15 anos ou mais.

\end{enumerate}
\end{sphinxadmonition}


\arrange{ }
\label{\detokenize{PE103-1:organizando-as-ideias}}\label{\detokenize{PE103-1::doc}}
Nas atividades anteriores foram trabalhados vários conceitos importantes da Estatística. Alguns desses conceitos serão apresentados a seguir.
\phantomsection\label{\detokenize{PE103-1:sub-conceitos-basicos}}
\sphinxstylestrong{Conceitos Básicos}

Em geral, a palavra população representa um conjunto de habitantes de um determinado lugar. No entanto, em Estatística, \index{população}população tem um sentido mais amplo e pode ser definida como o conjunto de todos os elementos com pelo menos uma característica em comum. Observe que é exatamente essa característica em comum que vai definir o universo (população) de uma pesquisa.

Assim, em Estatística, a população não precisa ser um conjunto de pessoas, pode ser o conjunto de parafusos fabricados por uma indústria, o conjunto de animais de certa espécie que vivem em uma região, todos os estudantes universitários de um país, etc.
\begin{description}
\item[{Amostra\index{Amostra|textbf}}] \leavevmode\phantomsection\label{\detokenize{PE103-1:term-amostra}}
é um subconjunto não-vazio da população.

\end{description}

Cada uma das unidades investigadas em um estudo estatístico é denominada um \index{elemento}elemento.  Assim, cada parafuso investigado é um elemento na atividade ``Escolha do fornecedor''; cada paciente observado é um elemento na atividade ``Comparação de medicamentos''; e cada domicílio e seus residentes são elementos na atividade da PNAD.

Cada característica observada de um elemento é uma \index{variável}variável estatística. Assim, a medida do diâmetro do parafuso é uma variável na atividade ``Escolha do fornecedor'', o tempo de cura da dor de cabeça é uma variável na atividade ``Comparação de medicamentos'' e, na atividade da PNAD, estão presentes várias variáveis estatísticas de interesse do domicílio e de seus residentes tais como local, número de cômodos, número de residentes; sexo, idade e rendimento dos residentes, etc.

Suponha que deseja-se investigar a opinião dos estudantes de um colégio quanto à modificação da lista de produtos vendidos na cantina para outros mais saudáveis, trocando refrigerantes por sucos naturais entre outros. Para isso, a direção da escola irá entrevistar cinco alunos sorteados de cada uma de suas 40 turmas. Nesse exemplo, a população corresponde a todos os estudantes deste colégio e, a amostra, aos 200 estudantes que foram entrevistados. Cada estudante entrevistado é um elemento e, a variável de interesse  é a opinião do estudante: ``a favor'' ou ``contra'' à mudança. Num estudo desse tipo, costuma-se registrar também outras variáveis como sexo, idade, ano de ensino, turno, etc.
\begin{description}
\item[{Parâmetro\index{Parâmetro|textbf}}] \leavevmode\phantomsection\label{\detokenize{PE103-1:term-parametro}}
característica numérica da população.

\end{description}
\begin{description}
\item[{Estimador\index{Estimador|textbf}}] \leavevmode\phantomsection\label{\detokenize{PE103-1:term-estimador}}
função que produz estimativas de parâmetros usando os dados da amostra.

\end{description}

Voltando ao exemplo anterior, sobre a modificação da lista de produtos da cantina, temos que o parâmetro corresponde à proporção dos estudantes desse colégio que são favoráveis à mudança (na maioria das vezes não acessível, a menos que se realize um censo). O estimador desse parâmetro corresponderá à proporção de estudantes favoráveis à mudança na amostra, que resultará numa estimativa do parâmetro.

As etapas da análise estatística podem ser divididas em duas estruturas básicas: \index{Estatística Descritiva}Estatística Descritiva e \index{Estatística Inferencial}Estatística Inferencial. A primeira corresponde a uma exploração das informações que podem ser retiradas dos dados amostrais de modo a reconhecer estruturas que possibilitem futuramente inferir sobre parâmetros de interesse. A segunda consiste em estabelecer modelos probabilísticos para que se possa fazer afirmações sobre a população com algum nível de confiança. Vide a caixa ``Para refletir'' a seguir.

Em resumo, a Estatística Descritiva é uma espécie de arqueologia dos dados observados e, a Estatística Inferencial, a indução das informações obtidas da amostra para características da população não observada em sua totalidade.

A PNAD faz uso da inferência estatística, pois ela investiga uma amostra de domicílios em algumas cidades brasileiras, mas propõe estimativas para as características da população brasileira.

Quando se realiza um \index{censo}censo - levantamento de dados de toda a população -, não existe a necessidade de fazer uma inferência estatística. No entanto, muitas vezes a realização de um censo é inviável, por várias razões como custo muito alto, tempo muito longo, entre outras.

\begin{sphinxadmonition}{note}{Para refletir}

Proposições são elementos importantes na construção de toda a ciência. No que se refere à natureza da Estatística, em contraponto à natureza da Matemática, podemos destacar dois tipos de proposições.

Uma proposição é dita matemática se é possível classificá-la em \sphinxstyleemphasis{verdadeira} ou \sphinxstyleemphasis{falsa}, ainda que essa afirmação seja uma conjectura não provada. Assim, a proposição

\sphinxstyleemphasis{``O quadrado de um número par é par.''}

é uma proposição matemática, pois sabemos que ela é verdadeira. Da mesma forma, a proposição

\sphinxstyleemphasis{``O triângulo de lados 6, 4 e 3 é um triângulo retângulo.''}

é uma proposição matemática, pois sabemos que é falsa.

Por outro lado, uma proposição estatística é uma afirmação sobre a qual nunca teremos condição de afirmar se é \sphinxstyleemphasis{verdadeira} ou \sphinxstyleemphasis{falsa}, mas apenas aferir um nível de confiança para ela. A proposição
\begin{itemize}
\item {} 
``Uma moeda, que ao ser lançada 10 vezes, resulta em 10 coroas, não é uma moeda equilibrada.''*

\end{itemize}

é uma proposição estatística, pois existe a possibilidade de em 10 lançamentos de uma moeda equilibrada obtermos 10 coroas, embora isso seja pouco provável de ocorrer.

\sphinxstylestrong{Observação:} Uma moeda é dita ser equilibrada se as probabilidades de se obter cara e coroa são iguais. Caso contrário, a moeda é dita ser não-equilibrada.

Se lançarmos 100 vezes essa mesma moeda e obtivermos 8 caras, teremos mais evidências para aceitar a proposição de que não seja equilibrada, mas ainda assim não poderemos afirmar que a proposição seja verdadeira. Proposições desse tipo que envolvem um nível de confiança sobre sua veracidade são propsições de natureza estatística.
\end{sphinxadmonition}
\phantomsection\label{\detokenize{PE103-1:sub-classificacao-de-variaveis}}
\sphinxstylestrong{Classificação de variáveis}

A classificação das variáveis estudadas é importante, pois as técnicas e procedimentos estatísticos de análise de dados dependem do tipo de variável investigado. Nesse sentido é importante reconhecer a natureza de cada variável investigada para posterior tratamento da informação obtida. Por exemplo, se estamos estudando a modalidade de atividades físicas praticadas pelos brasileiros de 15 anos ou mais, não faz sentido calcular média, pois  ela não assume valores numéricos.

Existem dois tipos principais de variáveis (qualitativas e quantitativas), que se subdividem, por sua vez, em duas categorias, conforme a figura 2.2.
\phantomsection\label{\detokenize{PE103-1:fig-classificacao-de-variaveis}}\begin{figure}[htp]\centering\begin{tikzpicture}
\tikzstyle{vecArrow} = [thick, decoration={markings , mark=at position
1 with {\arrow[semithick, fill=white]{triangle 60}}},
double distance=1.4pt, shorten >= 5.5pt,
preaction = {decorate,line width=1.4pt},
postaction = {draw,line width=1.4pt, white,shorten >= 4.5pt}]
\begin{scope}[x=15, y=10]
   \draw[rounded corners = 5pt, fill=primario] (-9,0) rectangle (-5,2);
   \node[color=white] at (-7,1) {nominal};
   \draw[rounded corners = 5pt, fill=primario] (-4.5,0) rectangle (-0.5,2);
   \node[color=white] at (-2.5,1) {ordinal};
   \draw[rounded corners = 5pt, fill=primario] (0.5,0) rectangle (4.5,2);
   \node[color=white] at (2.5,1) {discreta};
   \draw[rounded corners = 5pt, fill=primario] (5,0) rectangle (9,2);
   \node[color=white] at (7,1) {contí­nua};
   \draw[vecArrow] (-4.75,3.25) -- (-6,1.5);
   \draw[vecArrow] (-4.75,3.25) -- (-3.5,1.5);
   \draw[vecArrow] (4.75,3.25) -- (6,1.5);
   \draw[vecArrow] (4.75,3.25) -- (3.5,1.5);
   \draw[rounded corners = 5pt, fill=primario] (-9,3) rectangle (-0.5,5);
   \node[color=white] at (-4.75,4) {qualitativa};
   \draw[rounded corners = 5pt, fill=primario] (0.5,3) rectangle (9,5);
   \node[color=white] at (4.75,4) {quantitativa};
   \draw[vecArrow] (0,6.25) -- (-1.5,4.5);
   \draw[vecArrow] (0,6.25) -- (1.5,4.5);
   \draw[rounded corners = 5pt, fill=primario] (-9,6) rectangle (9,8);
   \node[color=white] at (0,7) {Variável};
\end{scope}
\end{tikzpicture}\caption{Classificação dos tipos de variáveis}\end{figure}\begin{description}
\item[{Variável qualitativa\index{Variável qualitativa|textbf}}] \leavevmode\phantomsection\label{\detokenize{PE103-1:term-variavel-qualitativa}}
Uma variável estatística é qualitativa se as possíveis respostas para ela são atributos não-numéricos. A maior parte das variáveis identificadas no ``Suplemento de Práticas de Esporte e Atividade Física'' da PNAD/2015, representa variáveis qualitativas.

\end{description}

Uma variável qualitativa é classificada em nominal ou ordinal.
\begin{description}
\item[{Variável qualitativa nominal\index{Variável qualitativa nominal|textbf}}] \leavevmode\phantomsection\label{\detokenize{PE103-1:term-variavel-qualitativa-nominal}}
Uma variável qualitativa é nominal quando não existe nenhuma ordenação natural das respostas associadas à variável. Exemplos de variáveis nominais: bairro de residência, tipo sanguíneo, modalidade de atividade física que pratica, etc.

\end{description}
\begin{description}
\item[{Variável qualitativa ordinal\index{Variável qualitativa ordinal|textbf}}] \leavevmode\phantomsection\label{\detokenize{PE103-1:term-variavel-qualitativa-ordinal}}
A variável qualitativa é ordinal quando é possível estabelecer uma relação de ordem entre as respostas associadas a ela. Por exemplo, nível de instrução da mãe com as respostas possíveis: Ensino Fundamental completo, Ensino Médio completo, Ensino Superior incompleto e Ensino Superior completo. Podemos perceber que quem tem Ensino Médio completo tem maior nível de instrução de quem tem Ensino Fundamental completo.

\end{description}
\begin{description}
\item[{Variável quantitativa\index{Variável quantitativa|textbf}}] \leavevmode\phantomsection\label{\detokenize{PE103-1:term-variavel-quantitativa}}
Uma variável é quantitativa se as respostas para ela são numéricas. Exemplos de variáveis quantitativas são idade, peso, altura, temperatura, número de irmãos, número de horas semanais dedicadas à prática de atividade física.

\end{description}

Uma variável quantitativa é classificada em discreta ou contínua.
\begin{description}
\item[{Variável quantitativa discreta\index{Variável quantitativa discreta|textbf}}] \leavevmode\phantomsection\label{\detokenize{PE103-1:term-variavel-quantitativa-discreta}}
As variáveis discretas resultam de uma contagem ou são variáveis cuja quantidade de valores possíveis é finita. Por exemplo, o número de atendimentos em um Pronto-Socorro nos finais de semana, o número de erros de impressão na página de um livro, número de irmãos, etc.

\end{description}
\begin{description}
\item[{Variável quantitativa contínua\index{Variável quantitativa contínua|textbf}}] \leavevmode\phantomsection\label{\detokenize{PE103-1:term-variavel-quantitativa-continua}}
As variáveis quantitativas contínuas em geral resultam de uma medição. Por exemplo, altura, peso, temperatura, etc.

\end{description}

\begin{sphinxadmonition}{note}{Observação}

Na análise dos infográficos vimos que uma variável quantitativa pode ser tratada como qualitativa, por exemplo, a idade trabalhada em faixas etárias torna-se uma variável qualitativa ordinal. No entanto, se consideramos a idade em anos completos temos uma variável quantitativa. Por outro lado, também podemos transformar uma variável qualitativa em quantitativa. Considere a variável ``prática de atividades físicas'' que tem como respostas ``Sim'' ou ``Não''. Esse tipo de variável com apenas duas respostas é chamado \index{variável binária}variável binária e tem uma representação numérica natural. Podemos atribuir o número 1 para a resposta ``Sim'' e o número 0 para a resposta ``Não''. Essa estratégia permite somar todas as respostas. Observe que a soma representará o número de pessoas na amostra que praticam atividade física e a ``média'' representará a proporção de pessoas na amostra que praticam atividade física.
\end{sphinxadmonition}

\sphinxstylestrong{Gráficos para Variáveis Qualitativas}

Nas análises dos infográficos, trabalhamos com alguns tipos de gráficos para representar a distribuição de frequências de variáveis qualitativas. No {\hyperref[\detokenize{PE103-0:fig-infografico-pnad-3}]{\sphinxcrossref{\DUrole{std,std-ref}{infográfico 3}}}}, tem-se um \index{gráfico de barras}gráfico de barras. Nesse gráfico, cada barra, de mesma largura, representa uma resposta e seu comprimento corresponde à \index{frequência}frequência na qual a resposta ocorre. Observe também que, nesse gráfico, se estivermos trabalhando com as porcentagens de cada resposta, a soma das porcentagens deve ser 100\%.

Em geral, se a variável for ordinal dispomos as respostas em ordem crescente. Se a variável é nominal, podemos dispor as respostas em ordem decrescente de frequência.

\begin{sphinxadmonition}{note}{Frequência absoluta e frequência relativa}

Numa turma de um colégio foram observados os tipos sanguíneos de seus 40 alunos. Verificou-se que 18 alunos têm sangue tipo ``O'', 12, tipo ``A'', 6, tipo ``AB'' e 4, tipo ``B''. Nesse exemplo, temos que as \index{frequência absoluta}frequências absolutas para os tipos sanguíneos ``O'', ``A'', ``AB'' e ``B'' foram, respectivamente, 18, 12, 6 e 4. Em geral, quando queremos comparar grupos diferentes, usamos a \index{frequência relativa}frequência relativa em vez da frequência absoluta. A frequência relativa é dada pela razão entre a frequência absoluta e o número total de observações. Nesse exemplo, temos que as frequências relativas para os tipos sanguíneos ``O'', ``A'', ``AB'' e ``B'' foram, respectivamente, 0,45; 0,30; 0,15 e 0,10. Observe que em termos percentuais as frequências relativas observadas equivalem a, respectivamente, 45\%, 30\%, 15\% e 10\%.
É comum resumir esse tipo de informação, usando uma tabela, informando as respostas da variável e suas frequências. Nesse exemplo a variável é tipo sanguíneo e sua classificação é qualitativa nominal, pois assume respostas não numéricas ``A'', ``B'', ``AB'' e ``O'', sem uma ordenação natural. Em geral dispomos os valores dessa variável em ordem decrescente de frequência.


\begin{savenotes}\sphinxattablestart
\centering
\begin{tabulary}{\linewidth}[t]{|T|T|T|T|}
\hline
\sphinxstylethead{\sphinxstyletheadfamily 
tipo
sanguíneo
\unskip}\relax &\sphinxstylethead{\sphinxstyletheadfamily 
frequência
absoluta
\unskip}\relax &\sphinxstylethead{\sphinxstyletheadfamily 
frequência
relativa
\unskip}\relax &\sphinxstylethead{\sphinxstyletheadfamily 
porcentagem
(\%)
\unskip}\relax \\
\hline
O
&
18
&
0,45
&
45
\\
\hline
A
&
12
&
0,30
&
30
\\
\hline
AB
&
6
&
0,15
&
15
\\
\hline
B
&
4
&
0,10
&
10
\\
\hline
total
&
40
&
1,00
&
100
\\
\hline
\end{tabulary}
\par
\sphinxattableend\end{savenotes}
\end{sphinxadmonition}

Os gráficos apresentados nos {\hyperref[\detokenize{PE103-0:fig-infografico-pnad-1}]{\sphinxcrossref{\DUrole{std,std-ref}{infográfico 1}}}} e {\hyperref[\detokenize{PE103-0:fig-infografico-pnad-2}]{\sphinxcrossref{\DUrole{std,std-ref}{infográfico 2}}}} são gráficos de barras?

Esses gráficos usam barras para representar as frequências em subgrupos diferentes do conjunto observado. Mas eles não se encaixam na apresentação anterior. Verifique que se somarmos as porcentagens elas não resultarão em 100\%. De fato, são \index{gráficos de barras múltiplas}gráficos de barras múltiplas, úteis para comparar diferentes distribuições de frequências. Observe que, em cada um desses gráficos, a variável sob investigação é se a pessoa pratica ou não atividade física. No entanto, em vez de apresentar as porcentagens das respostas \sphinxstyleemphasis{Sim} e \sphinxstyleemphasis{Não} no universo de homens e no universo de mulheres, como a variável é binária, só foram apresentadas as porcentagens de \sphinxstyleemphasis{Sim} em cada subgrupo, pois nesse caso, as correspondentes porcentagens de \sphinxstyleemphasis{Não} são dadas pelo complementar em cada subgrupo considerado.

\begin{figure}[H]
\centering
\capstart

\noindent\sphinxincludegraphics[width=400pt]{{barrasmultiplas_sexo}.png}
\caption{Detalhe legendado do \DUrole{xref,std,std-ref}{infográfico 1}}\label{\detokenize{PE103-1:fig-infografico-1-detalhe}}\label{\detokenize{PE103-1:id2}}\end{figure}

O mesmo ocorre quando analisamos os gráficos para faixa etária, grau de instrução e rendimento. Todos são gráficos de barras múltiplas que nos apoiaram em nossas análises sobre a associação entre a prática de atividades físicas e a outra variável (sexo, faixa etária, grau de instrução, rendimento).

No {\hyperref[\detokenize{PE103-0:fig-infografico-pnad-4}]{\sphinxcrossref{\DUrole{std,std-ref}{infográfico 4}}}}, temos um \index{gráfico de setores}gráfico de setores e dois \index{gráficos de retângulos}gráficos de retângulos. A ideia por trás desses gráficos é subdividir de maneira proporcional a figura maior em partes cujas áreas em relação à figura maior correspondam à frequência de cada resposta. Por exemplo, no gráfico de setores, subdividimos o círculo em setores de tal modo que a razão da área de cada setor em relação a área do círculo corresponde à frequência (ou porcentagem) da resposta que ele representa. Portanto a soma das frequências, quando apresentadas em porcentagens, tem que ser igual a 100$\%$.

No gráficos de retângulos essa mesma ideia é usada: o retângulo maior é subdividido em retângulos cujas áreas relativas correspondem às porcentagens das respostas que eles representam. Esses gráficos foram construídos para representar as respostas à pergunta ``Em quais áreas em que deve ocorrer investimento público?'' para quem respondeu \sphinxstyleemphasis{Não} à pergunta ``O poder público deve investir em atividades físcas ou desportivas?'' e também para representar as respostas à pergunta ``Qual deve ser a prioridade nos investimentos?'' para quem respondeu ``Sim'' à pergunta ``O poder público deve investir em atividades físicas ou desportivas?''.

\begin{sphinxadmonition}{note}{Observação}

Quando estamos trabalhando com variáveis qualitativas usamos a escala da frequência (absoluta, relativa, porcentagem)  na construção de gráficos para representar a distribuição de frequências das respostas dadas à variável sob investigação. As representações gráficas mais comuns são gráficos de barras e gráficos de setores. Para comparações da mesma variável em grupos diferentes é comum usar o gráfico de barras múltiplas com frequências relativas ou porcentagens.
\end{sphinxadmonition}

\begin{sphinxadmonition}{note}{Para o professor}

Como escolher entre o gráfico de setores ou o gráfico de barras para representar a distribuição de frequências de uma variável qualitativa? Se o número de respostas diferentes é grande, maior que 4, ou se  as diferenças nas frequências das respostas são pequenas, por exemplo uma tem porcentagem 22\% e a outra tem porcentagem 25\%, o gráfico de setores não será adequado, pois pequenas diferenças de ângulos  não são perceptíveis, enquanto que no gráfico de barras é fácil perceber pequenas diferenças. Se deseja-se fazer comparações múltiplas o gráfico de setores não é adequado. Observe que todos infográficos da atividade para comparar diferentes grupos quanto à prática de atividades físicas são gráficos de barras múltiplas. Finalmente, e não menos importante, sempre lembrar que em comparações múltiplas é fundamental relativizar a frequência absoluta usando frequências relativas ou porcentagens quando os grupos investigados têm tamanhos diferentes, pois a diferença em tamanhos pode mascarar possíveis similaridades. Por exemplo, suponha o exemplo com os dados de tipos sanguíneos dos 40 alunos de uma turma. Agora desejamos comparar as respostas obtidas com um conjunto de 120 observações para as quais 30 são tipo ``A''; 12, tipo ``AB''; 18, tipo ``B'' e 60, tipo ``O''. Os gráficos de barras na mesma escala, usando a frequência absoluta parecem bem diferentes, como mostra a figura a seguir.

\begin{figure}[H]
\centering
\capstart

\noindent\sphinxincludegraphics[width=200bp]{{exemplo_escala_absoluta}.png}
\caption{Gráficos de barras da distribuição na escla da frequência absoluta}\label{\detokenize{PE103-1:fig-coloque-aqui-o-nome}}\label{\detokenize{PE103-1:id3}}\end{figure}

Porém, os gráficos construídos, usando a escala da porcentagem, não parecem tão diferentes, como mostra a figura a seguir.

\begin{figure}[H]
\centering
\capstart

\noindent\sphinxincludegraphics[width=200bp]{{exemplo_escala_porcentagem}.png}
\caption{Gráficos de barras na escala da porcentagem}\label{\detokenize{PE103-1:id1}}\label{\detokenize{PE103-1:id4}}\end{figure}

Comparando os dois, percebem-se  apenas pequenas diferenças quanto às porcentagens dos sangues tipo ``AB'' e tipo ``B'', comparando os dois gráficos.
\end{sphinxadmonition}


\practice{ }
\label{\detokenize{PE103-2::doc}}\label{\detokenize{PE103-2:praticando}}\phantomsection\label{\detokenize{PE103-2:ativ-1-pratica-atividade-fisica-na-turma}}
\begin{sphinxadmonition}{note}{Atividade}{ Prática de atividade física na turma}

\begin{sphinxadmonition}{note}{Para o professor}

\sphinxstylestrong{Objetivos específicos} Conduzir uma coleta de dados sobre a turma envolvendo as informações do suplemento ``Prática de Esporte e Atividade Física'' para comparar os resultados dessa ``amostra'' com os da PNAD/2015.

\sphinxstylestrong{Observações e sugestões}
\begin{itemize}
\item {} 
Preparar uma tabela a ser preenchida pela turma com as informações: sexo, idade, prática ou não de atividade física em seu tempo livre, e a modalidade, de maneira a viabilizar a comparação dos dados obtidos com os resultados da PNAD/2015. A tabela poderá conter outras variáveis se forem julgadas de interesse pela turma como por exemplo, local da prática, duração da prática entre outras. Mas, para efeito de comparação com os infográficos, sexo e idade serão as variáveis necessárias nesse levantamento. Comente com os alunos que essa será uma amostra de conveniência, pois o interesse é estudar o perfil da turma quanto à prática de atividades físicas e por isso, as respostas da turma podem não ser similares às da pesquisa.

\item {} 
Com base nas respostas obtidas, resumir a informação em tabelas de frequências, contar quantas respostas foram sim, calcular a porcentagem da turma que pratica atividade física e comparar com o resultado geral das pessoas de 15 anos ou mais, o percentual correspondente a essa faixa etária e o percentual correspondente a esse grau de instrução. Construir uma tabela de frequências com as modalidades esportivas incluindo as categorias apresentadas no infográfico do IBGE. Construir gráficos para representar as distribuições de frequências das variáveis investigadas nessa pesquisa. Construir gráficos de barras múltiplas, isto é, gráficos de barras separados por grupos diferentes, como por exemplo, sexo.

\end{itemize}
\end{sphinxadmonition}

Deseja-se comparar os hábitos de atividade física em tempo livre dos alunos da turma com os dados obtidos da PNAD/2015. Para isso preencha o formulário de dados fornecido pelo professor. Construa tabelas e gráficos resumindo a informação obtida.
\end{sphinxadmonition}
\phantomsection\label{\detokenize{PE103-2:ativ-classificacao-de-variaveis}}
\begin{sphinxadmonition}{note}{Atividade}{ Classificação de variáveis}

\begin{sphinxadmonition}{note}{Para o professor}

\sphinxstylestrong{Objetivos específicos}
\begin{itemize}
\item {} 
Diferenciar variável qualitativa e variável quantitativa.

\item {} 
Identificar variáveis qualitativas binárias.

\end{itemize}
\end{sphinxadmonition}

Suponha que cada uma das variáveis a seguir foi observada para todos os alunos de sua turma. Indique se cada uma delas é uma variável qualitativa ou quantitativa. Se for uma variável qualitativa, indique se ela é binária (apenas duas respostas possíveis) ou não.
\begin{enumerate}
\item {} 
altura (em metros)

\item {} 
peso (em quilos)

\item {} 
razão do peso sobre o quadrado da medida da altura

\item {} 
tempo de sono na noite anterior

\item {} 
se foi dormir na noite anterior antes ou depois da meia-noite

\item {} 
mês de nascimento

\item {} 
número de irmãos

\item {} 
nota obtida na última avaliação de Matemática

\item {} 
se tirou nota maior ou igual a 6,0 ou menor do que 6,0 na última avaliação de Matemática

\item {} 
distância da casa à escola

\item {} 
se o indivíduo possui cartão de crédito ou não

\item {} 
modo de locomoção para a escola

\end{enumerate}
\end{sphinxadmonition}

\begin{sphinxadmonition}{note}{Resposta}
\begin{enumerate}
\item {} 
quantitativa, (b) quantitativa, (c) quantitativa (d) quantitativa, (e) qualitativa binária, (f) qualitativa (g) quantitativa (h) quantitativa (i) qualitativa binária (j) quantitativa (k) qualitativa binária (l) qualitativa.

\end{enumerate}
\end{sphinxadmonition}
\phantomsection\label{\detokenize{PE103-2:ativ-3-construcao-de-grafico-variavel-qualitativa}}
\begin{sphinxadmonition}{note}{Atividade}{ Construção de gráficos para variáveis qualitativas}

\begin{sphinxadmonition}{note}{Para o professor}

\sphinxstylestrong{Objetivos específicos} Construir gráficos de distribuições de frequências para variáveis qualitativas.

\sphinxstylestrong{Observações e sugestões} Embora os gráficos solicitados nesta atividade sejam simples, recomenda-se sugerir aos alunos usar algum recurso tecnológico para a construção dos mesmos, tais como, uma planilha ou o GeoGebra.
\end{sphinxadmonition}

Considerando o {\hyperref[\detokenize{PE103-0:fig-infografico-pnad-4}]{\sphinxcrossref{\DUrole{std,std-ref}{infográfico 4}}}}, transforme o gráfico de setores em gráfico de retângulos e os gráficos de retângulos em gráficos de setores.
\end{sphinxadmonition}

\begin{sphinxadmonition}{note}{Resposta}
\phantomsection\label{\detokenize{PE103-2:fig-trocando-setor-e-barra}}\begin{center}\begin{tikzpicture}
Infográfico 4 transformado
\begin{scope}[x=20,y=20]
  \node at (0,11) {O poder público deve investir em atividades fí­sicas?};
\draw [fill=blue] (-1,10) rectangle (1,2.67);
\draw [fill=red] (-1,2.67) rectangle (1,1.2);
\draw [fill=gray] (-1,1.2) rectangle (1,0);
\node [color=white] at (0,6.335) {Sim};
\node [color=white] at (0,1.935) {Não};
\draw [thick,->] (-1.5,9.5) -- (-7,9.5) -- (-7,8.5);
\draw [thick,->] (1.5,1.935) -- (7,1.935) -- (7,3);
\draw[fill=green] (7,7) -- (7,10) arc (90:-118.08:3);
\draw [fill=orange] (7,7) -- + (-118.08:3) arc (-118.08:-194.76:3);
\draw [fill=blue] (7,7) -- + (-194.76:3) arc (-194.76:-254.16:3);
\draw [fill=gray] (7,7) -- + (-254.16:3) arc (-254.16:-270:3);
\draw (7,7) -- (7,10);
\draw[fill=pink] (-7,4.5) -- (-7,7.5) arc (90:-237.96:3);
\draw [fill=yellow] (-7,4.5) -- + (-237.96:3) arc (-237.96:-266.76:3);
\draw [fill=gray] (-7,4.5) -- + (-266.76:3) arc (-266.76:-270:3);
\draw (-7,4.5) -- (-7,7.5);
\draw [fill=pink, line width=0] (-10,-1) rectangle (-9,-2);
\node [right] at (-9,-1.5) {Atividades para};
\node [right] at (-9,-2.5) {pessoas em geral};
\draw [fill=yellow, line width=0] (-10,-3.5) rectangle (-9,-4.5);
\node [right] at (-9,-4) {Formação de atletas};
\draw [fill=gray, line width=0] (-10,-5) rectangle (-9,-6);
\node [right] at (-9,-5.5) {Otra};
\draw [fill=green, line width=0] (4,-1) rectangle (5,-2);
\node [right] at (5,-1.5) {Saúde};
\draw [fill=orange, line width=0] (4,-2.5) rectangle (5,-3.5);
\node [right] at (5,-3) {Segurança};
\draw [fill=blue, line width=0] (4,-4) rectangle (5,-5);
\node [right] at (5,-4.5) {Educação};
\draw [fill=gray, line width=0] (4,-5.5) rectangle (5,-6.5);
\node [right] at (5,-6) {Otra};
\end{scope}
\end{tikzpicture}\end{center}\end{sphinxadmonition}
\phantomsection\label{\detokenize{PE103-2:ativ-4-analise-de-grafico}}
\begin{sphinxadmonition}{note}{Atividade}{ Análise de gráfico}

\begin{sphinxadmonition}{note}{Para o professor}

\sphinxstylestrong{Objetivos especícificos} Mostrar que podem existir diversas formas de usar barras para representar algum tipo de dado, mas que nem todos os gráficos que usam barras são gráficos de barras no sentido da representação de uma distribuição de frequências.

\sphinxstylestrong{Observações e sugestões} O gráfico desse exemplo é ``um gráfico de barras'', mas as barras representam o valor da inflação da alimentação acumulado nos últimos 12 meses em função do tempo: de agosto de 2016 até agosto de 2017. Na seção ``Explorando 2'', veremos que, para esse tipo de informação - valores de uma variável quantitativa ao longo do tempo -, é mais comum usar um gráfico de linhas unindo por segmentos os pontos consecutivos dados (tempo,valor da variável).
\end{sphinxadmonition}

Observe o gráfico a seguir publicado em um jornal.
\begin{enumerate}
\item {} 
Como você classificaria esse gráfico?

\item {} 
Qual é a informação representada pelo comprimento da barra nesse gráfico?

\item {} 
Que tipo(s) de variável(is) ele está representando?

\item {} 
Construa um gráfico diferente para representar a mesma informação, marcando num plano Cartesiano os pontos (x,y) em que x corresponde ao tempo e y corresponde à inflação acumulada no domicílio, unindo os pontos consecutivos por segmentos. É possível perceber a partir desse gráfico algum tipo de comportamento no período observado?

\end{enumerate}
\phantomsection\label{\detokenize{PE103-2:linhaversusbarra}}\begin{figure}[htp]\centering\begin{tikzpicture}
\begin{scope}[x=20, y = 10]
\draw (0,0) -- (28,0);
\foreach \ye in {-5,5,10,15}{
\draw [help lines, lightgray] (0,\ye) -- (28,\ye);
}
\foreach \x/ \y/\z in {1/16.79/AGO,2/16.11/SET,3/14.85/OUT,4 /11.57/NOV,5/9.36/DEZ,6/6.47/JAN,7/4.34/FEV,8/3/MAR,9 /2.54/ABR,10/1.08/MAI,11/-0.56/JUN,12/-3.07/JUL,13/-5.19 /AGO}{
   \draw [fill=primario] (2*\x-0.5,0) rectangle  (2*\x+0.5,\y);
   \node at (2*\x,-8) {\z};
}
\node at (2,-9) {2016};
\node at (12,-9) {2017};
\node [right] at (0,21.5) {INFLAÇÃO DA ALIMENTAÇÃOO NO   DOMICÍLIO};
\node [right] at (0,20) {(acumulado em 12 meses, em   $\%$)};\\
\end{scope}
\end{tikzpicture}\caption{Inflação da alimentação acumulada nos últimos 12 meses (Fonte: IBGE)}\end{figure}\end{sphinxadmonition}

\begin{sphinxadmonition}{note}{Resposta}
\begin{enumerate}
\item {} 
É um gráfico que usa barras, mas nesse gráfico o comprimento das barras não é frequência.

\item {} 
Valor da inflação da alimentação acumulada nos últimos 12 meses. Esses valores são apresentados em função do período de tempo: agosto de 2016 até agosto de 2017.

\item {} 
O valor da inflação da alimentação acumulada nos últimos 12 meses é uma variável quantitativa, o período de tempo representado em mês/ano é uma variável qualitativa ordinal.

\item {} 
(gráfico) Como evoluiu a inflação da alimentação acumulada em 12 meses no período investigado, a saber, agosto de 2017 até agosto de 2017. Mais precisamente, percebe-se que a inflação da alimentação acumulada em 12 meses apresentou no período analisado uma forte tendência de queda.

\end{enumerate}
\phantomsection\label{\detokenize{PE103-2:fig-grafico-de-linha-da-inflacao-alimentacao}}\begin{figure}[htp]\centering\begin{tikzpicture}
\begin{scope}[x=20, y=10]
\draw (0,0) -- (28,0);
\foreach \ye in {-5,5,10,15}{
   \draw [help lines, lightgray] (0,\ye) -- (28,\ye);
}
\foreach \x/\y/\z in {1/16.79/AGO,2/16.11/SET,3/14.85/OUT,4/11.57/NOV,5/9.36/DEZ,6/6.47/JAN,7/4.34/FEV,8/3/MAR,9/2.54/ABR,10/1.08/MAI,11/-0.56/JUN,12/-3.07/JUL,13/-5.19/AGO}{
\draw [help lines, lightgray] (2*\x,-7) -- (2*\x,18);
\node at (2*\x,-8) {\z};
}
\draw [color=primario, thick] (2,16.79) node[ponto] {}
\foreach \x/\y/\z in {1/16.79/AGO,2/16.11/SET,3/14.85/OUT,4/11.57/NOV,5/9.36/DEZ,6/6.47/JAN,7/4.34/FEV,8/3/MAR,9/2.54/ABR,10/1.08/MAI,11/-0.56/JUN,12/-3.07/JUL,13/-5.19/AGO}{
   -- (2*\x,\y) node[ponto] {}
};
\node at (2,-9) {2016};
\node at (12,-9) {2017};
\node [right] at (0,21.5) {INFLAÇÃO DA ALIMENTAÇÃOO NO DOMICÍLIO};
\node [right] at (0,20) {(acumulado em 12 meses, em $\%$)};
\end{scope}
\end{tikzpicture}\caption{Gráfico de linha da inflação da alimentação acumulada nos últimos 12 meses}\end{figure}\end{sphinxadmonition}


\explore{estruturas de variáveis quantitativas}
\label{\detokenize{PE103-3:explorando-visualizando-estruturas-de-variaveis-quantitativas}}\label{\detokenize{PE103-3::doc}}\phantomsection\label{\detokenize{PE103-3:ativ-construcao-histograma}}
\begin{sphinxadmonition}{note}{Atividade}{ Construção do histograma}

\begin{sphinxadmonition}{note}{Para o professor}

\sphinxstylestrong{Objetivos específicos}

Identificar, na contrução de um gráfico que represente a distribuição de frequências, a necessidade de agrupar em intervalos de classe os valores observados de uma variável quantitativa contínua.

\sphinxstylestrong{Observações e sugestões}

A construção do histograma será dirigida nessa atividade, mas  recomenda-se fortemente o uso de recursos tecnológicos, como  o GeoGebra, para esse tipo de construção.

Incluir link do arquivo desses dados em formato GeoGebra. Em alguns aplicativos, e é o caso do GeoGebra, é necessário substituir a vírgula como separador decimal do padrão brasileiro por ponto.
\end{sphinxadmonition}

\begin{figure}[H]
\centering
\capstart

\noindent\sphinxincludegraphics[width=300bp]{{USA.NM.VeryLargeArray.03}.jpg}
\caption{Arranjo de radiotelescópios - Very Large Array (VLA), New Mexico, EUA. \sphinxhref{https://commons.wikimedia.org/wiki/File:USA.NM.VeryLargeArray.03.jpg}{Foto: Hajor CC-by-sa}}\label{\detokenize{PE103-3:fig-radiotelescopios}}\label{\detokenize{PE103-3:id1}}\end{figure}

Um arranjo de oito radiotelescópios (A, B, C, D, E, F, G e H) como  ilustrado na {\hyperref[\detokenize{PE103-3:fig-radiotelescopios}]{\sphinxcrossref{\DUrole{std,std-ref}{Figura}}}} detectou sinais cujos oito registros de tempo para cada radiotelescópio se encontram na tabela a seguir.


\begin{savenotes}\sphinxattablestart
\centering
\begin{tabulary}{\linewidth}[t]{|T|T|T|T|T|T|T|T|}
\hline
\sphinxstylethead{\sphinxstyletheadfamily 
A
\unskip}\relax &\sphinxstylethead{\sphinxstyletheadfamily 
B
\unskip}\relax &\sphinxstylethead{\sphinxstyletheadfamily 
C
\unskip}\relax &\sphinxstylethead{\sphinxstyletheadfamily 
D
\unskip}\relax &\sphinxstylethead{\sphinxstyletheadfamily 
E
\unskip}\relax &\sphinxstylethead{\sphinxstyletheadfamily 
F
\unskip}\relax &\sphinxstylethead{\sphinxstyletheadfamily 
G
\unskip}\relax &\sphinxstylethead{\sphinxstyletheadfamily 
H
\unskip}\relax \\
\hline
3,03
&
4,37
&
5,04
&
5,73
&
4,03
&
5,37
&
6,04
&
6,74
\\
\hline
3,38
&
4,46
&
5,11
&
5,84
&
4,38
&
5,46
&
6,11
&
6,84
\\
\hline
3,60
&
4,55
&
5,19
&
5,95
&
4,60
&
5,55
&
6,19
&
6,96
\\
\hline
3,78
&
4,63
&
5,29
&
6,08
&
4,78
&
5,64
&
6,29
&
7,08
\\
\hline
3,92
&
4,71
&
5,36
&
6,23
&
4,92
&
5,72
&
6,36
&
7,23
\\
\hline
4,04
&
4,79
&
5,45
&
6,41
&
5,04
&
5,79
&
6,45
&
7,40
\\
\hline
4,16
&
4,87
&
5,54
&
6,62
&
5,16
&
5,87
&
6,54
&
7,63
\\
\hline
4,27
&
4,95
&
5,64
&
6,97
&
5,26
&
5,95
&
6,64
&
7,97
\\
\hline
\end{tabulary}
\par
\sphinxattableend\end{savenotes}
\end{sphinxadmonition}

\begin{sphinxadmonition}{note}{Para refletir}
\begin{itemize}
\item {} 
Como construir uma tabela de frequências desses dados uma vez que os registros de tempo são todos distintos?

\item {} 
Como você faria para visualizar o comportamento de uma variável com estas características?

\end{itemize}
\end{sphinxadmonition}

A natureza quantitativa de uma variável contínua pode muitas vezes levar a resultados que praticamente não se repetem. Eles podem ser todos diferentes, como é observado no exemplo. Com o objetivo de identificar alguma estrutura no comportamento deste tipo de variável é necessário agrupar os valores em intervalos de classe, o que permite analisar a sua distribuição de frequências.

\(a)\) Complete a tabela a seguir que utiliza de intervalos de amplitude 0,5 começando em 3,0. Observe que cada intervalo na tabela é fechado à esquerda e aberto à direita, isto quer dizer que, o limite inferior está incluso e o limite superior não está incluso.


\begin{savenotes}\sphinxattablestart
\centering
\begin{tabulary}{\linewidth}[t]{|T|T|}
\hline
\sphinxstylethead{\sphinxstyletheadfamily 
Intervalo de classe
\unskip}\relax &\sphinxstylethead{\sphinxstyletheadfamily 
Número de observações
\unskip}\relax \\
\hline
{[} 3,0 ; 3,5 {[}
&\\
\hline
{[} 3,5 ; 4,0 {[}
&\\
\hline
{[} 4,0 ; 4,5 {[}
&\\
\hline
{[} 4,5 ; 5,0 {[}
&\\
\hline
{[} 5,0 ; 5,5 {[}
&\\
\hline
{[} 5,5 ; 6,0 {[}
&\\
\hline
{[} 6,0 ; 6,5 {[}
&\\
\hline
{[} 6,5 ; 7,0 {[}
&\\
\hline
{[} 7,0 ; 7,5 {[}
&\\
\hline
{[} 7,5 ; 8,0 {[}
&\\
\hline
\end{tabulary}
\par
\sphinxattableend\end{savenotes}

Para visualizar o comportamento desses dados, iremos construir um gráfico chamado \index{histograma}histograma, composto por retângulos adjacentes cujas alturas representam a frequência de observações que ocorrem no intervalo correspondente. A base de cada retângulo corresponde aos limites do intervalo definido no agrupamento dos dados.

\(b)\) Complete a figura a seguir com os demais retângulos do {\hyperref[\detokenize{PE103-3:fig-histograma}]{\sphinxcrossref{\DUrole{std,std-ref}{histograma}}}}.
\begin{quote}

\begin{figure}[H]
\centering
\capstart

\noindent\sphinxincludegraphics[width=300bp]{{Histograma}.png}
\caption{Histograma dos dados coletados pela grade de radiotelescópios}\label{\detokenize{PE103-3:fig-histograma}}\label{\detokenize{PE103-3:id2}}\end{figure}
\end{quote}

\(c)\) Calcule a média dos dados da tabela e localize-a no gráfico, sabendo que a soma dos 64 registros de tempo é 351,95. O que você pode observar quanto à localização da média no histograma construído?

\begin{sphinxadmonition}{note}{Resposta}

\(a)\)


\begin{savenotes}\sphinxattablestart
\centering
\begin{tabulary}{\linewidth}[t]{|T|T|}
\hline
\sphinxstylethead{\sphinxstyletheadfamily 
Intervalo de classe
\unskip}\relax &\sphinxstylethead{\sphinxstyletheadfamily 
Número de observações
\unskip}\relax \\
\hline
{[} 3,0 ; 3,5 {[}
&
2
\\
\hline
{[} 3,5 ; 4,0 {[}
&
3
\\
\hline
{[} 4,0 ; 4,5 {[}
&
7
\\
\hline
{[} 4,5 ; 5,0 {[}
&
9
\\
\hline
{[} 5,0 ; 5,5 {[}
&
11
\\
\hline
{[} 5,5 ; 6,0 {[}
&
11
\\
\hline
{[} 6,0 ; 6,5 {[}
&
9
\\
\hline
{[} 6,5 ; 7,0 {[}
&
7
\\
\hline
{[} 7,0 ; 7,5 {[}
&
3
\\
\hline
{[} 7,5 ; 8,0 {[}
&
2
\\
\hline
\end{tabulary}
\par
\sphinxattableend\end{savenotes}

\(b)\) Figura 4.3 e \(c)\) O valor da média é aproximadamente 5,5. O histograma apresenta uma distribuição de frequências simétrica e a posição da média está no eixo de simetria do histograma.
\end{sphinxadmonition}

\begin{figure}[H]
\centering
\capstart

\noindent\sphinxincludegraphics[width=300bp]{{Histograma-resposta_1}.png}
\caption{Histograma dos registros de tempo}\label{\detokenize{PE103-3:fig-histograma-resposta}}\label{\detokenize{PE103-3:id3}}\end{figure}
\phantomsection\label{\detokenize{PE103-3:ativ-medicao-temperatura-serie-temporal}}
\begin{sphinxadmonition}{note}{Atividade}{ Medição da temperatura ao longo do tempo}

\sphinxstylestrong{Objetivos específicos}
\begin{itemize}
\item {} 
Definir série temporal a partir de um conjunto de observações sobre uma variável quantitativa contínua variando no tempo.

\item {} 
Trabalhar com  gráficos de linha para ilustrar a evolução dos valores da variável ao longo do tempo.

\end{itemize}

\sphinxstylestrong{Observações e sugestões}

Para a construção do gráfico de linha será fornecida uma malha quadriculada para o preenchimento dos pontos, recomenda-se também uso de planilhas de cálculo para essa construção. Veja nesse \sphinxhref{https://docs.google.com/spreadsheets/d/1B2bbuggIVjdfG6NivzDYmsmzovkt0FuFUUhSVFaAyDc/edit\#gid=1474980695}{link}, uma sugestão para realizar esta atividade.

Respostas possíveis na reflexão proposta são: índices de inflação, preços de diversos bens,  índices da bolsa de valores, a população total em um território, a incidência de alguma enfermidade, a quantidade de vendas de um produto. É importante usar exemplos de dados que tenham aparecido recentemente na mídia ou que tenham relevância local.

Na discussão sobre sazonalidade, pedir aos alunos para trazer notícias de jornais ou revistas que contenham séries temporais. Mostrar que existem várias medições que são comparadas com as do ano anterior, por exemplo, inflação, crescimento do PIB, taxas de desemprego por trimestre, entre outras.

Você deve ter notado que a previsão do tempo é feita sempre a partir de dois números, isto ocorre porque a temperatura varia de forma contínua ao longo do dia e o que está sendo previsto são as temperaturas máxima e mínima. Por exemplo: 28° / 19°, significa que a previsão da temperatura máxima durante o dia será aproximadamente de 28°C e, a mínima, 19°C.

Diversas variáveis meteorológicas (no sentido estatístico) são registradas nas estações meteorológicas: temperatura, precipitação (quantidade de chuva), umidade do ar, entre outras.

No Brasil, as estações estão a cargo do {\color{red}\bfseries{}{}`Instituto Nacional de Meteorologia (INMET)\textless{}http://www.inmet.gov.br/\textgreater{}{}`\_} e as informações são armazenadas em bases de dados. Para poder tratar essas informações, frequentemente elas são resumidas por períodos de tempo de diferentes magnitudes: dias, semanas, meses ou anos.

Dados coletados ao longo do tempo (como a informação meteorológica) são conhecidos como séries de dados temporais ou, apenas, \index{séries temporais}séries temporais, já que correspondem a variáveis que mudam continuamente ao longo do tempo e a informação só é útil se sabemos o momento em que foram realizadas as medições.
\end{sphinxadmonition}

\begin{sphinxadmonition}{note}{Para refletir}

Forneça outros exemplos de séries temporais nas áreas de saúde, economia, finanças, educação, etc.
\end{sphinxadmonition}

A tabela a seguir fornece a média das temperaturas máximas para cada mês nos anos de 1991 a 2000 da cidade de Porto Alegre em graus centígrados (Fonte: \sphinxhref{http://www.inmet.gov.br/portal/index.php?r=bdmep/bdmep}{Banco de Dados Meteorológicos para Ensino e Pesquisa, BDMEP - INMET})


\begin{savenotes}\sphinxattablestart
\centering
\begin{tabulary}{\linewidth}[t]{|T|T|T|T|T|T|T|T|T|T|T|}
\hline
\sphinxstartmulticolumn{11}%
\begin{varwidth}[t]{\sphinxcolwidth{11}{11}}
\sphinxstylethead{\sphinxstyletheadfamily Temperatura Máxima Média mensal nos anos 1991-2000 na cidade de Porto Alegre
\unskip}\relax \par
\vskip-\baselineskip\vbox{\hbox{\strut}}\end{varwidth}%
\sphinxstopmulticolumn
\\
\hline\sphinxstylethead{\sphinxstyletheadfamily 
Mes
\unskip}\relax &\sphinxstylethead{\sphinxstyletheadfamily 
1991
\unskip}\relax &\sphinxstylethead{\sphinxstyletheadfamily 
1992
\unskip}\relax &\sphinxstylethead{\sphinxstyletheadfamily 
1993
\unskip}\relax &\sphinxstylethead{\sphinxstyletheadfamily 
1994
\unskip}\relax &\sphinxstylethead{\sphinxstyletheadfamily 
1995
\unskip}\relax &\sphinxstylethead{\sphinxstyletheadfamily 
1996
\unskip}\relax &\sphinxstylethead{\sphinxstyletheadfamily 
1997
\unskip}\relax &\sphinxstylethead{\sphinxstyletheadfamily 
1998
\unskip}\relax &\sphinxstylethead{\sphinxstyletheadfamily 
1999
\unskip}\relax &\sphinxstylethead{\sphinxstyletheadfamily 
2000
\unskip}\relax \\
\hline
1
&
30,23
&
30,43
&
31,34
&
30,33
&
30,74
&
29,89
&
32,09
&
29,13
&
30,65
&
30,63
\\
\hline
2
&
31,03
&
31,48
&
29,28
&
28,85
&
29,46
&
29,78
&
29,62
&
28,26
&
29,56
&
29,93
\\
\hline
3
&
30,55
&
30,05
&
28,22
&
28,05
&
29,12
&
28,67
&
28,63
&
27,20
&
31,64
&
27,85
\\
\hline
4
&
26,15
&
25,52
&
27,66
&
25,51
&
26,22
&
27,03
&
26,56
&
24,03
&
24,00
&
26,32
\\
\hline
5
&
25,31
&
21,44
&
23,29
&
24,33
&
21,95
&
22,94
&
22,95
&
22,00
&
21,51
&
21,78
\\
\hline
6
&
20,32
&
22,68
&
19,13
&
20,09
&
20,45
&
17,76
&
19,42
&
19,60
&
18,87
&
21,50
\\
\hline
7
&
19,75
&
16,91
&
17,97
&
20,41
&
21,60
&
16,99
&
20,67
&
20,47
&
18,78
&
17,59
\\
\hline
8
&
21,81
&
20,50
&
21,90
&
21,28
&
21,55
&
22,59
&
23,06
&
19,77
&
21,94
&
20,85
\\
\hline
9
&
23,99
&
22,14
&
20,83
&
25,21
&
22,62
&
21,40
&
22,32
&
21,22
&
22,65
&
22,25
\\
\hline
10
&
26,17
&
26,16
&
26,40
&
24,60
&
24,17
&
25,34
&
23,27
&
25,19
&
23,07
&
24,02
\\
\hline
11
&
26,93
&
27,16
&
28,07
&
26,53
&
28,93
&
28,40
&
26,51
&
28,24
&
26,36
&
26,87
\\
\hline
12
&
30,60
&
29,95
&
29,73
&
32,05
&
30,44
&
29,87
&
30,28
&
28,91
&
29,08
&
29,51
\\
\hline
\end{tabulary}
\par
\sphinxattableend\end{savenotes}
\begin{enumerate}
\item {} 
Escolha dois anos diferentes e localize os pontos da tabela na grade quadriculada usando o mês como abscissa (x) e a temperatura como ordenada (y). Utilize cores diferentes para a série de cada ano.

\item {} 
Una os pontos correspondentes ao mesmo ano (mesma série) de meses consecutivos com um segmento e observe o resultado. Você percebe algum comportamento similar para a  temperatura em anos diferentes?

\item {} 
Compare seu gráfico com o de colegas que escolheram outros anos (ou acrescente séries de outros anos ao seu gráfico). O que você percebe com relação à temperatura nos meses iniciais, intermediários e finais do ano?  A que se deve esse comportamento da temperatura?

\end{enumerate}

\(a)\) e \(b)\) Percebe-se temperaturas mais altas nos meses iniciais e finais do ano e, mais baixas, no meio do ano.
\begin{quote}

\begin{figure}[H]
\centering

\noindent\sphinxincludegraphics[width=300bp]{{linhas-temperatura}.png}
\end{figure}
\end{quote}

\(c)\) Idem ao item b). Isso ocorre devido às estações do ano. No hemisfério sul temos temperaturas mais altas nos meses finais e iniciais do ano e temperaturas mais baixas no meio do ano.

Os gráficos que você acabou de construir são chamados \index{gráficos de linha}gráficos de linha. Esse tipo de gráfico é muito utilizado para variáveis quantitativas contínuas que dependem de uma outra variável quantitativa, neste caso o tempo. Quando a variável quantitativa é observada ao longo do tempo, o conjunto de dados resultante é chamado uma série temporal.

\begin{sphinxadmonition}{note}{Observação}

Como você já deve ter observado, a temperatura em Porto Alegre é mais baixa nos meses correspondentes ao inverno e mais alta na primavera e no verão, o que se repete cada ano. Este fenômeno, que se observa nos ciclos do gráfico, é chamado de \index{sazonalidade}sazonalidade. A origem deste conceito é exatamente o da sazonalidade que observamos na natureza com as estações ao longo do ano.
\end{sphinxadmonition}
\begin{description}
\item[{Sazonalidade\index{Sazonalidade|textbf}}] \leavevmode\phantomsection\label{\detokenize{PE103-3:term-sazonalidade}}
Variações periódicas que se observam em séries temporais e que devem sua presença a um fenômeno implícito que incide de forma direta nas medições da variável observada.

\end{description}

Considere novamente os dados de temperatura da atividade anterior. Se representarmos todos os dados da tabela num único gráfico com a escala temporal das abscissas ao longo dos dez anos, obtemos o seguinte gráfico:

\begin{figure}[H]
\centering
\capstart

\noindent\sphinxincludegraphics[width=400\sphinxpxdimen]{{linhas-sazonalidade}.png}
\caption{Efeito da sazonalidade no gŕafico de linhas da temperatura máxima média}\label{\detokenize{PE103-3:fig-linhas-sazonalidade}}\label{\detokenize{PE103-3:id4}}\end{figure}


\arrange{ }
\label{\detokenize{PE103-4:cap-organizando-as-ideias2}}\label{\detokenize{PE103-4:organizando-as-ideias}}\label{\detokenize{PE103-4::doc}}
Dois tipos de gráficos para representar variáveis quantitativas contínuas foram apresentados: o histograma e o gráfico de linha.
\begin{description}
\item[{Histograma\index{Histograma|textbf}}] \leavevmode\phantomsection\label{\detokenize{PE103-4:term-histograma}}
O histograma é uma representação gráfica da distribuição de frequências de uma variável quantitativa contínua agrupada em intervalos usando retângulos adjacentes. Cada retângulo no histograma corresponde a um intervalo considerado e a razão da área desse retângulo em relação à área total do histograma deve ser igual à frequência relativa de casos desse intervalo.

\end{description}
\begin{description}
\item[{Gráfico de linha\index{Gráfico de linha|textbf}}] \leavevmode\phantomsection\label{\detokenize{PE103-4:term-grafico-de-linha}}
O gráfico de linha é uma representação útil quando os dados são uma série temporal, ou seja, os dados são coletados ao longo do tempo. Esse gráfico é construído marcando-se no plano Cartesiano os pontos \((x,y)\) em que abscissa \(x\) representa o tempo e, a ordenada \(y\), a variável quantitativa. Os pontos consecutivos são unidos por segmentos.

\end{description}

\begin{sphinxadmonition}{note}{Quantos intervalos de classe considerar no agrupamento dos dados?}

Quando existe a necessidade de agrupar os dados em intervalos, uma questão que se coloca é: quantos intervalos usar para que se possa reconhecer estruturas de frequências nesse conjunto? Não existe uma única resposta para essa questão. No entanto, devemos evitar tanto usar um número reduzido de intervalos, quanto usar um número grande de intervalos. Por exemplo, se usarmos um único intervalo, o histograma seria representado por um único retângulo que nada informaria sobre o comportamento dos dados, conforme o gráfico a seguir.

\begin{figure}[H]
\centering
\capstart

\noindent\sphinxincludegraphics[width=200bp]{{histograma_2intervalos}.png}
\caption{Histograma dos resgistros de tempo considerando apenas dois intervalos}\label{\detokenize{PE103-4:id4}}\end{figure}

Por outro lado, se o número de intervalos for igual ou superior ao número de observações, o histograma potencialmente teria apenas classes com uma única observação e o objetivo de visualizar estruturas dos dados em análise se perderia. A figura a seguir contruída a partir de 100 intervalos não revela a estrutura dos dados de registro de tempo, uma vez que cada classe contém no máximo duas observações.

\begin{figure}[H]
\centering
\capstart

\noindent\sphinxincludegraphics[width=300bp]{{histograma_100intervalos}.png}
\caption{Histograma dos resgistros de tempo considerando cem intervalos}\label{\detokenize{PE103-4:id5}}\end{figure}

Embora não exista uma resposta única sobre quantos intervalos considerar, alguns autores sugerem usar o número inteiro mais próximo da raiz quadrada do número de observações, outros sugerem usar de 5 a 15 intervalos de amplitudes iguais. No GeoGebra, por exemplo, a função que constrói histogramas permite trabalhar com 3 a 20 intervalos. A figura a seguir apresenta um histograma construído com \(\sqrt{64}=8\) intervalos.

\begin{figure}[H]
\centering
\capstart

\noindent\sphinxincludegraphics[width=300bp]{{histograma_8intervalos}.png}
\caption{Histograma dos resgistros de tempo considerando oito intervalos}\label{\detokenize{PE103-4:id6}}\end{figure}
\end{sphinxadmonition}

Até aqui, consideramos intervalos de mesma amplitude e usamos, como a altura dos retângulos, a frequência absoluta ou relativa das observações no intervalo. Suponha a seguinte distribuição de frequências de um conjunto de 50 observações.


\begin{savenotes}\sphinxattablestart
\centering
\begin{tabulary}{\linewidth}[t]{|T|T|T|}
\hline
\sphinxstylethead{\sphinxstyletheadfamily 
Intervalo de classe
\unskip}\relax &\sphinxstylethead{\sphinxstyletheadfamily 
frequência absoluta
\unskip}\relax &\sphinxstylethead{\sphinxstyletheadfamily 
frequência relativa
\unskip}\relax \\
\hline
{[} 1 ; 3 {[}
&
4
&
0,08
\\
\hline
{[} 3 ; 5 {[}
&
12
&
0,24
\\
\hline
{[} 5 ; 7 {[}
&
20
&
0,40
\\
\hline
{[} 7 ; 9 {[}
&
8
&
0,16
\\
\hline
{[} 9; 11 {[}
&
6
&
0,12
\\
\hline
\end{tabulary}
\par
\sphinxattableend\end{savenotes}

Observe que nessa tabela todos os intervalos têm amplitude 2. Veja o histograma construído para esses dados a seguir.

\begin{figure}[H]
\centering
\capstart

\noindent\sphinxincludegraphics[width=400bp]{{exemplo_histograma_areas_1}.png}
\caption{Histograma na escala da frequência absoluta}\label{\detokenize{PE103-4:id7}}\end{figure}

Verifique que a razão da área de cada retângulo em relação à área total é igual à frequência relativa do intervalo correspondente.

Porém, quando os intervalos apresentam amplitudes desiguais, usar a frequência não será mais apropriado.

Suponha agora a seguinte distribuição de frequências de um conjunto de 50 observações.


\begin{savenotes}\sphinxattablestart
\centering
\begin{tabulary}{\linewidth}[t]{|T|T|T|T|}
\hline
\sphinxstylethead{\sphinxstyletheadfamily 
Intervalo de classe
\unskip}\relax &\sphinxstylethead{\sphinxstyletheadfamily 
frequência absoluta
\unskip}\relax &\sphinxstylethead{\sphinxstyletheadfamily 
frequência relativa
\unskip}\relax &\sphinxstylethead{\sphinxstyletheadfamily 
amplitude do intervalo
\unskip}\relax \\
\hline
{[} 1 ; 3 {[}
&
4
&
0,08
&
2
\\
\hline
{[} 3 ; 5 {[}
&
8
&
0,16
&
2
\\
\hline
{[} 5 ; 8 {[}
&
18
&
0,36
&
3
\\
\hline
{[} 8 ; 12 {[}
&
12
&
0,24
&
4
\\
\hline
{[}12; 16 {[}
&
8
&
0,16
&
4
\\
\hline
\end{tabulary}
\par
\sphinxattableend\end{savenotes}

Nesse caso devemos usar a densidade de frequência absoluta ou relativa obtida pela razão entre frequência e amplitude do intervalo.

\(\textsf{densidade de frequência absoluta}=\frac{\textsf{frequência absoluta do intervalo}}{\textsf{amplitude do intervalo}}\)

\(\textsf{densidade de frequência relativa}=\frac{\textsf{frequência relativa do intervalo}}{\textsf{amplitude do intervalo}}\)

Veja a seguir uma construção equivocada do histograma desses dados, usando a frequência absoluta.

\begin{figure}[H]
\centering
\capstart

\noindent\sphinxincludegraphics[width=400bp]{{histograma_incorreto_2}.png}
\caption{Histograma incorreto}\label{\detokenize{PE103-4:fig-coloque-aqui-o-nome}}\label{\detokenize{PE103-4:id8}}\end{figure}

Observe que a razão da área do primeiro retângulo em relação à área total é dada por \(\displaystyle{\frac{8}{158}}\approx  0,051\), porém a frequência relativa do primeiro intervalo é 0,08! A razão da área do último retângulo é \(\displaystyle{\frac{4\cdot 8}{158}}\approx 0,20\), porém a frequência relativa desse intervalo é 0,16! Ou seja, esse histograma não representa corretamente a distribuição de frequências desses dados. Na tabela a seguir, foram calculadas as densidades de frequência absoluta.


\begin{savenotes}\sphinxattablestart
\centering
\begin{tabulary}{\linewidth}[t]{|T|T|T|T|}
\hline
\sphinxstylethead{\sphinxstyletheadfamily 
Intervalo de classe
\unskip}\relax &\sphinxstylethead{\sphinxstyletheadfamily 
frequência absoluta
\unskip}\relax &\sphinxstylethead{\sphinxstyletheadfamily 
amplitude do intervalo
\unskip}\relax &\sphinxstylethead{\sphinxstyletheadfamily 
dens. freq. absoluta
\unskip}\relax \\
\hline
{[} 1 ; 3 {[}
&
4
&
2
&
2
\\
\hline
{[} 3 ; 5 {[}
&
8
&
2
&
4
\\
\hline
{[} 5 ; 8 {[}
&
18
&
3
&
6
\\
\hline
{[} 8 ; 12 {[}
&
12
&
4
&
3
\\
\hline
{[}12; 16 {[}
&
8
&
4
&
2
\\
\hline
\end{tabulary}
\par
\sphinxattableend\end{savenotes}

Veja a seguir a construção do histograma na escala da densidade de frequência absoluta e observe que agora ele representa corretamente a distribuição de frequências.
\phantomsection\label{\detokenize{PE103-4:id1}}\begin{quote}

\begin{figure}[H]
\centering
\capstart

\noindent\sphinxincludegraphics[width=400bp]{{histograma_correto}.png}
\caption{Histograma correto}\label{\detokenize{PE103-4:id9}}\end{figure}
\end{quote}

Comparando as figuras 6.9 e 6.10, podemos perceber que a primeira distorce a estrutura da distribuição de frequências, atribuindo pesos maiores aos intervalos de maior amplitude e, menores, aos intervalos de menor amplitude.

Em que situações há a necessidade de considerarmos intervalos de amplitudes desiguais?

Normalmente, na primeira construção dos intervalos consideramos sempre intervalos de amplitudes iguais. Mas pode acontecer, nesse agrupamento, intervalos vazios ou intervalos com um número muito grande de observações. Quando essas situações ocorrem recomenda-se juntar dois intervalos consecutivos no primeiro caso ou subdividir o intervalo no segundo caso.

\begin{sphinxadmonition}{note}{Gráfico de Barras versus Histograma}

O gráfico de barras não é um histograma, apesar de suas representações serem parecidas.  Os gráficos de barras são úteis para descrever a distribuição de frequências de uma variável qualitativa. Nesse gráfico só há um eixo com escala que corresponde aos valores das frequências das categorias (respostas) da variável. As barras podem ser tanto verticais como horizontais e são apresentadas de forma igualmente espaçada. Cada barra representa uma resposta da variável qualitativa e a altura da barra corresponde à frequência daquela resposta. Observe que o posicionamento das barras é livre, conforme as figuras a seguir.

\begin{figure}[H]
\centering
\capstart

\noindent\sphinxincludegraphics[width=300bp]{{g_barras_tipo_s_1}.png}
\caption{Gráfico de barras: duas formas de apresentação}\label{\detokenize{PE103-4:id2}}\label{\detokenize{PE103-4:id10}}\end{figure}

O mais comum é dispor as respostas em ordem decrescente de frequência. Esse tipo de gráfico também pode ser usado para representar uma variável quantitativa discreta, sendo que nesse caso, as posições das barras correspondem aos valores assumidos pela variável. Pela natureza discreta da variável, as barras não são adjacentes e, pela natureza quantitativa da variável, o posicionamento das barras não é livre.

Os histogramas são úteis para representar a distribuição de frequências de uma variável quantitativa contínua cujos valores foram agrupados em intervalos. No histograma, o eixo das abscissas (horizontal) representa a escala da variável contínua e, o eixo das ordenadas (vertical) representa a escala da frequência ou densidade de frequência que é definida como a razão entre a frequência e a amplitude do intervalo.

\begin{figure}[H]
\centering
\capstart

\noindent\sphinxincludegraphics[width=300bp]{{histograma_5intervalos}.png}
\caption{Histograma dos registros, considerando 5 intervalos}\label{\detokenize{PE103-4:id3}}\label{\detokenize{PE103-4:id11}}\end{figure}

Não podemos variar livremente a posição dos intervalos nesse gráfico (figura 5.8). Ele revela uma estrutura importante desses dados, a saber, os registros de tempo ocorrem com maior frequência nos intervalos intermediários (de 4 a 6) e com frequência bem menor nos intervalos extremos (de 3 a 4 e de 7 a 8).
\end{sphinxadmonition}


\practice{ }
\label{\detokenize{PE103-5::doc}}\label{\detokenize{PE103-5:praticando}}\label{\detokenize{PE103-5:cap-praticando2}}\phantomsection\label{\detokenize{PE103-5:ativ-variacoes-do-histograma}}
\begin{sphinxadmonition}{note}{Atividade}{ Construção de histogramas}

\begin{sphinxadmonition}{note}{Para o professor}

\sphinxstylestrong{Objetivo específico}
\begin{itemize}
\item {} 
Avaliar a forma do histograma a partir da variação do número de intervalos considerados
\begin{quote}

\sphinxstylestrong{Observações e sugestões}
\end{quote}

\end{itemize}

Essa atividade deve ser realizada com algum recurso tecnológico. Um exemplo de como realizá-la usando o Geogebra pode ser acessado nesse \sphinxhref{https://www.geogebra.org/m/HmTzSJKM}{link}.
\phantomsection\label{\detokenize{PE103-5:fig-coloque-aqui-o-nome}}
\begin{figure}[H]
\centering

\noindent\sphinxincludegraphics[width=300bp]{{histogramas_geo}.png}
\label{\detokenize{PE103-5:fig-coloque-aqui-o-nome}}\end{figure}

Sugestão de realização da atividade.

Arrastando o cursor na linha que representa classes (intervalos) é possível variar de três a 20 intervalos.
\end{sphinxadmonition}

Refaça o histograma dos dados de registros de tempo variando o número de intervalos de classe. Compare a forma dos histogramas obtidos com a forma do histograma construído na atividade 4.1.
\end{sphinxadmonition}

\begin{sphinxadmonition}{note}{Resposta}

Existem várias possibilidades e algumas delas estão apresentadas aqui. Na comparação é importante perceber que esses dados revelam uma estrutura simétrica, ocorrendo com frequências altas entre 4 e 6, e, occorrendo com frequências bem menores nos intervalos extremos inferior e superior.

\begin{figure}[H]
\centering
\capstart

\noindent\sphinxincludegraphics[width=150bp]{{hist6c}.png}
\caption{Histograma com 6 intervalos}\label{\detokenize{PE103-5:id1}}\label{\detokenize{PE103-5:id7}}\end{figure}

\begin{figure}[H]
\centering
\capstart

\noindent\sphinxincludegraphics[width=150bp]{{hist9c}.png}
\caption{Histograma com 9 intervalos}\label{\detokenize{PE103-5:id2}}\label{\detokenize{PE103-5:id8}}\end{figure}

\begin{figure}[H]
\centering
\capstart

\noindent\sphinxincludegraphics[width=150bp]{{hist12c}.png}
\caption{Histograma com 12 intervalos}\label{\detokenize{PE103-5:id3}}\label{\detokenize{PE103-5:id9}}\end{figure}

\begin{figure}[H]
\centering
\capstart

\noindent\sphinxincludegraphics[width=150bp]{{hist15c}.png}
\caption{Histograma com 15 intervalos}\label{\detokenize{PE103-5:id4}}\label{\detokenize{PE103-5:id10}}\end{figure}
\end{sphinxadmonition}
\phantomsection\label{\detokenize{PE103-5:ativ-titulo-da-histogramas-intervalos-desiguais}}
\begin{sphinxadmonition}{note}{Atividade}{histogramas com intervalos de amplitudes desiguais}

\begin{sphinxadmonition}{note}{Para o professor}

\sphinxstylestrong{Objetivos específicos}
\begin{itemize}
\item {} 
Construir histogramas nos casos em que os intervalos apresentam amplitudes desiguais.

\item {} 
Definir densidade de frequência absoluta e relativa.

\end{itemize}

\sphinxstylestrong{Observações e sugestões}

Nessa atividade o histograma deve ser construído, usando a escala de densidade de frequência (absoluta ou relativa).
\end{sphinxadmonition}

Suponha a seguinte distribuição de frequências de salários medidos em salários mínimos para 200 funcionários de uma empresa.


\begin{savenotes}\sphinxattablestart
\centering
\begin{tabulary}{\linewidth}[t]{|T|T|T|}
\hline
\sphinxstylethead{\sphinxstyletheadfamily 
Intervalo de classe
\unskip}\relax &\sphinxstylethead{\sphinxstyletheadfamily 
frequência absoluta
\unskip}\relax &\sphinxstylethead{\sphinxstyletheadfamily 
frequência relativa
\unskip}\relax \\
\hline
{[} 2,0 ; 3,0 {[}
&
12
&
0,06
\\
\hline
{[} 3,0 ; 5,0 {[}
&
40
&
0,20
\\
\hline
{[} 5,0 ; 7,0 {[}
&
80
&
0,40
\\
\hline
{[} 7,0 ; 10,0 {[}
&
48
&
0,24
\\
\hline
{[} 10,0 ; 15,0 {[}
&
20
&
0,10
\\
\hline
\end{tabulary}
\par
\sphinxattableend\end{savenotes}
\begin{enumerate}
\item {} 
Determine as amplitudes de cada intervalo considerado na tabela.

\item {} 
Construa um histograma adequado para esses dados.

\end{enumerate}
\end{sphinxadmonition}

\begin{sphinxadmonition}{note}{Resposta}


\begin{savenotes}\sphinxattablestart
\centering
\begin{tabulary}{\linewidth}[t]{|T|T|T|T|}
\hline
\sphinxstylethead{\sphinxstyletheadfamily 
Intervalo de classe
\unskip}\relax &\sphinxstylethead{\sphinxstyletheadfamily 
freq. absoluta
\unskip}\relax &\sphinxstylethead{\sphinxstyletheadfamily 
amplitude
\unskip}\relax &\sphinxstylethead{\sphinxstyletheadfamily 
dens. de freq. absoluta
\unskip}\relax \\
\hline
{[} 2,0 ; 3,0 )
&
12
&
1
&
12
\\
\hline
{[} 3,0 ; 5,0 )
&
40
&
2
&
20
\\
\hline
{[} 5,0 ; 7,0 )
&
80
&
2
&
40
\\
\hline
{[} 7,0 ; 10,0 )
&
48
&
3
&
16
\\
\hline
{[} 10,0 ; 15,0 )
&
20
&
5
&
4
\\
\hline
\end{tabulary}
\par
\sphinxattableend\end{savenotes}
\end{sphinxadmonition}

\begin{figure}[H]
\centering
\capstart

\noindent\sphinxincludegraphics[width=300bp]{{histogramaerrado}.png}
\caption{Histograma errado}\label{\detokenize{PE103-5:id5}}\label{\detokenize{PE103-5:id11}}\end{figure}

\begin{figure}[H]
\centering
\capstart

\noindent\sphinxincludegraphics[width=300bp]{{histogramacorreto_1}.png}
\caption{Histograma correto}\label{\detokenize{PE103-5:id6}}\label{\detokenize{PE103-5:id12}}\end{figure}


\know{A seleção de amostras}
\label{\detokenize{PE103-A::doc}}\label{\detokenize{PE103-A:para-saber-mais}}


\label{\detokenize{PE103-A:a-questao-da-selecao-de-amostras}}\label{\detokenize{PE103-A:sec-coloque-aqui-o-nome}}
Quando queremos estender nossas observações provenientes de uma amostra para a população é importante ter cuidado na sua seleção, pois ela deve ser representativa da população. Embora não seja nosso objetivo aqui descrever métodos variados de seleção de amostras, cabe destacar que existem dois tipos principais de seleção: os probabilísticos e os não probabilísticos.

O primeiro tipo é fundamental para que seja possível avaliar a incerteza das conclusões devido à amostragem tais como margem erro e nível de confiança.  Nesse tipo de seleção de amostra, conhecemos a probabilidade de seleção dos elementos da população na amostra. Entre os métodos probabilísticos mais comuns destacam-se
\begin{enumerate}
\item {} 
\index{amostragem aleatória simples}amostragem aleatória simples: todas as amostras de igual tamanho têm probabilidades iguais de serem selecionadas.

\item {} 
\index{amostragem estratificada}amostragem estratificada: a população é dividida em grupos de elementos homogêneos (similares nas características a serem investigadas) e os grupos são heterogêneos entre si. A amostra é composta por amostras aleatórias simples de cada grupo, em geral, proporcionalmente aos tamanhos dos grupos.

\item {} 
\index{amostragem por coglomerados}amostragem por conglomerados: a população é subdividida em conglomerados (subpopulações). Uma amostra aleatória simples de conglomerados é obtida e, em seguida, todos os elementos dos conglomerados escolhidos são observados.

\item {} 
\index{amostragem sitemática}amostragem sistemática: toda a população deve estar catalogada numa lista, por exemplo, lista dos alunos matriculados numa escola em ordem alfabética. Suponha que a lista contenha 1000 alunos e que se deseja obter uma amostra de tamanho 50. Para isso divide-se 1000 por 50 obtendo-se 20 blocos de 50 alunos. Sorteia-se ao acaso um número de 1 a 20, por exemplo, o número 9. Seleciona-se o aluno de número 9 e, depois, os próximos elementos são selecionados de 20 em 20 como uma Progressão Aritmética de razão 20 e primeiro termo 9.

\end{enumerate}

Os casos mais comuns de métodos não probabilísticos são \index{amostragem por conveniência}amostragem por conveniência e \index{amostragem por julgamento}amostragem por julgamento. A amostragem por conveniência carateriza-se por não ter um plano particular de amostragem. O objetivo nesse caso não seria generalizar conclusões e sim descrever as características principais do grupo de estudo.  Nas amostras por julgamento, os elementos da amostra são escolhidos por um especialista no assunto sob investigação. A grande desvantagem dos métodos não probabilísticos é a impossibilidade de avaliar incertezas devido à amostragem.

\begin{sphinxadmonition}{note}{Exemplo}{Horário de entrada}

A direção de uma escola de Ensino Médio deseja realizar uma pesquisa para conhecer a opinião de seus 520 alunos sobre a antecipação em 30 minutos dos horários de seus turnos. Para tanto, devem decidir entre as seguintes estratégias de seleção de amostra.
\begin{enumerate}
\item {} 
40 alunos considerando os primeiros a chegar na  escola na segunda-feira.  \sphinxstyleemphasis{Temos, nesse caso, uma amostra de conveniência, pois a probabilidade de seleção dos alunos não é determinada no plano de amostragem: selecionar os 40 primeiros. Observe também que, esse esquema de seleção não parece razoável para essa pesquisa, pois é possível resultar numa resposta viesada, isto é, tendendo a favorecer à mudança de horário por considerar apenas os primeiros a chegar, não representando necessariamente a opinião da maioria dos 520 alunos da escola.}

\item {} 
40 alunos escolhidos a partir do cadastro de 520 alunos matriculados da seguinte forma: como \(520/40=13\), sorteia-se ao acaso um número entre 1 e 13, por exemplo 8; e, depois, seleciona-se do cadastro os alunos nas posições 8, 21, 34, 47,  e, assim sucessivamente de 13 em 13, até o aluno de posição 515 no cadastro de alunos, totalizando 40 observações. \sphinxstyleemphasis{Trata-se de uma amostra sistemática cuja probabilidade de seleção é conhecida, a saber,} \(\frac{1}{13}\). \sphinxstyleemphasis{Nesse caso, espera-se que a amostra seja representativa do conjunto de estudantes do colégio.}

\item {} 
40 alunos sendo 16 do primeiro ano, 14 do segundo ano e 10 do terceiro ano, escolhidos ao acaso, tendo em mente que na escola 40\% dos alunos são de primeiro ano, 35\% dos alunos são de segundo ano e 25\% dos alunos são de terceiro ano. \sphinxstyleemphasis{Trata-se de uma amostra estratificada proprocionalmente ao tamanho dos estratos que corrrespondem aos anos do Ensino Médio. As probabilidades de seleção de amostra são conhecidas.  Nesse caso, também espera-se que a amostra seja representativa do conjunto de estudantes do colégio.}

\item {} 
40 alunos de uma turma do segundo ano na qual está um filho do diretor. \sphinxstyleemphasis{Trata-se de uma amostra de conveniência e que pode resultar num resultado duplamente viesado, tanto pelo fato de que a turma escolhida não representa necessariamente a maioria dos estudantes da escolat, quanto pela presença do filho do diretor nessa turma que pode influenciar o resultado.}

\end{enumerate}
\end{sphinxadmonition}


\section{Projeto Aplicado}
\label{\detokenize{PE103-A:projeto}}\label{\detokenize{PE103-A:id1}}
\begin{sphinxadmonition}{note}{Para o professor}

\sphinxstylestrong{Objetivos específicos}

Realizar uma pesquisa envolvendo: a definição do tema e da população a ser investigada, a construção de um questionário para a coleta de dados, o planejamento e a seleção da amostra.
Após a coleta de dados, os alunos deverão empregar as ferramentas estudadas para resumir a informação obtida por meio de gráficos e cálculo de medidas apropriadas ao estudo. Ao final, cada grupo deverá elaborar um relatório sobre o tema investigado, incluindo os resultados obtidos e suas conclusões.

\sphinxstylestrong{Observações e sugestões}

Essa atividade terá duração de pelo menos três meses para que ela possa ser desenvolvida de forma completa e deverá ser realizada preferencialmente nos dois primeiros anos do Ensino Médio. As seguintes etapas deverão ser realizadas. Recomenda-se definir o universo da pesquisa como o ambiente escolar para viabilizar a coleta de dados.
\begin{enumerate}
\item {} 
Definir o tema a ser investigado.  Os temas, preferencialmente interdisciplinares, deverão ser submetidos ao professor para avaliar a viabilidade e pertinência da pesquisa. Uma sugestão pode ser trabalhar com algum assunto que, dentro de três meses será relevante, por exemplo, em três meses teremos o dia mundial sem tabaco, o dia mundial da Diabetes, o dia mundial sem carro, o dia do meio ambiente, dia nacional da acessibilidade, etc.

\item {} 
Definir a população e os elementos (unidades de observação) (pessoa, família, domicílio, cidade, escolas, turmas,…).

\item {} 
Definir como a amostra será escolhida e quantos elementos serão considerados na amostra.

\item {} 
Definir que variáveis serão observadas para cada elemento; em cada caso pode ser importante coletar informações de outras variáveis que podem estar relacionadas à questão de interesse para verificar se essas variáveis de alguma forma estão relacionadas. Por exemplo, na pesquisa PNAD/2015 identificamos que faixa etária, renda, escolaridade e sexo de alguma forma influenciam na proporção de pessoas que praticam atividades físicas. Se na coleta só observarmos se a pessoa pratica ou não a atividade física não será possível estudar essas relações.

\item {} 
Construir um questionário para a coleta de informações (para facilitar as análises é importante recomendar que os questionários tenham perguntas de respostas fechadas, incluindo, quando for o caso, a categoria ``outras'' ou ``sem opinião''.

\item {} 
Organizar os dados em tabelas e gráficos apropriados ao tipo de variável e calcular medidas resumo quando for o caso, que serão tratadas no capítulo medidas de posição e dispersão.

\item {} 
Elaborar um relatório de pesquisa incluindo todas as etapas do projeto, resultados, análises e conclusões.

\end{enumerate}
\end{sphinxadmonition}

Faça uma investigação sobre algum tema de interesse. Essa atividade deve ser realizada em grupos.  Após a aprovação do tema pelo professor, as seguintes etapas deverão ser realizadas:
\begin{enumerate}
\item {} 
Elaborar um cronograma considerando um prazo de três meses para concluir o projeto.

\item {} 
Definir a população e os elementos (unidades de observação) (pessoa, família, domicílio, cidade, escolas, turmas, estudadantes, etc.).

\item {} 
Definir como a amostra será escolhida e quantos elementos serão considerados na amostra.

\item {} 
Definir que variáveis serão observadas para cada elemento; em cada caso pode ser importante coletar informações de outras variáveis que podem estar relacionadas à questão de interesse para verificar se essas variáveis de alguma forma estão relacionadas.

\item {} 
Construir um questionário para a coleta de informações.

\item {} 
Coletar as informações.

\item {} 
Construir uma planilha com os dados obtidos.

\item {} 
Organizar os dados em tabelas e gráficos apropriados ao tipo de variável, resumindo a informação obtida.

\item {} 
Elaborar um relatório de pesquisa, incluindo todas as etapas do projeto, resultados, análises e conclusões.

\end{enumerate}


\section{Material suplementar}
\label{\detokenize{PE103-A:material-suplementar}}\label{\detokenize{PE103-A:cap-materialsuplementar-referencias}}
Sugestão de vídeos sobre o que é a Estatística, para que serve a Estatística e exemplos de aplicação da Estatística.
\begin{itemize}
\item {} 
O Prazer da Estatística - \sphinxurl{https://www.youtube.com/watch?v=nB5l9OW2eyo}

\item {} 
O que é Estatística? - \sphinxurl{https://www.youtube.com/watch?v=-Wm9cxiXUe0}

\item {} 
Ação, Reação, Correlação - \sphinxurl{http://m3.ime.unicamp.br/recursos/1043}

\end{itemize}

Sugestão de páginas para trabalhar com dados reais.
\begin{itemize}
\item {} 
Página do Programa de Desenvolvimento das Nações Unidas \textendash{} hdr.undp.org/en/data

\item {} 
Organização Mundial de Saúde \textendash{} www.who.int/

\item {} 
Instituto Brasileiro de Geografia e Estatística (IBGE) \textendash{} \sphinxurl{https://www.ibge.gov.br/}
\begin{itemize}
\item {} 
Estimativas de população dos municípios brasileiros - \sphinxurl{https://www.ibge.gov.br/estatisticas-novoportal/sociais/populacao/}

\item {} 
Atividades para o Ensino Médio - \sphinxurl{https://vamoscontar.ibge.gov.br/atividades/ensino-medio.html}

\end{itemize}

\item {} 
Instituto Nacional de Estudos e Pesquisas educacionai Anísio Teixeira (INEP)-  \sphinxurl{http://portal.inep.gov.br/inep-data}
\begin{itemize}
\item {} 
Censo Escolar INEP (último censo 2014) - \sphinxurl{http://inepdata.inep.gov.br/analytics/saw.dll?Portal\&PortalPath=\%2Fshared\%2FGeral\%2F\_portal\%2FDissemina\%C3\%A7\%C3\%A3o\%20dos\%20Censos}

\end{itemize}

\item {} 
Frota de veículos por município do RJ: \sphinxurl{http://www.detran.rj.gov.br/\_estatisticas.veiculos/index.asp}

\item {} 
Instituto de Pesquisa Econômica Aplicada (IPEA) \textendash{} \sphinxurl{http://www.ipea.gov.br/portal/}
\begin{itemize}
\item {} 
IPEA Data - \sphinxurl{http://www.ipeadata.gov.br/Default.aspx}

\end{itemize}

\item {} 
Ministério da Saúde \textendash{} Datasus - \sphinxurl{http://datasus.saude.gov.br/transferencia-download-de-arquivos/arquivos-de-dados}

\end{itemize}


\exercise
\label{\detokenize{PE103-E:exercicios}}\label{\detokenize{PE103-E::doc}}
\begin{enumerate}
\item Estabeleça se as seguintes conclusões a respeito de dados estatísticos provêm do uso da Estatística Descritiva (D) ou Estatística Inferencial (I), justificando as respostas.
\begin{enumerate}
\item {} 
No estado do Rio de Janeiro, a média de gasto semanal de consumo de gasolina numa amostra de 500 proprietários de carros foi de R\$ 200,00. O governo do Rio de Janeiro afirma que a média semanal de gasto em gasolina no estado é R\$ 200,00.

\item {} 
Uma amostra de 250 residentes de uma cidade indicou que 45 destes são funcionários públicos. Assim 18\% desses 250 residentes trabalham para o governo.

\item {} 
A média de idade de trabalhadores formais obtida de uma amostra de 380 habitantes de Nova Iguaçu foi de 34 anos.

\item {} 
Numa pesquisa feita com 2000 habitantes da Grande São Paulo, 768 disseram fazer uso regular de sacola ecológica pessoal em suas compras de supermercado.  A prefeitura conclui que mais de um terço dos habitantes da Grande São Paulo já aderiu à sacola ecológica.

\end{enumerate}

\item Classifique as proposições abaixo como sendo de natureza Matemática (M) ou de natureza Estatística (E), justificando as respostas.
\begin{enumerate}
\item {} 
Não existe número real \(x\) que satisfaça \(2x^2 + 3x + 2 = 0\).

\item {} 
Um dado que lançado 180 vezes não tenha gerado a face 6 é um dado viciado (não equilibrado).

\item {} 
Se de 100 crianças vacinadas pela BCG contra a tuberculose apenas duas contraíram a doença, então a vacina tem eficácia de 98\%.

\item {} 
Todo número par maior que 2 pode ser representado pela soma de dois números primos.

\item {} 
A precipitação de chuva amanhã será de aproximadamente 36 mm.

\item {} 
Se um feixe de paralelas está cortado por duas transversais então os segmentos determinados sobre uma transversal são respectivamente proporcionais aos segmentos determinados na outra.

\end{enumerate}

\item Suponha que pesquisadores desejam investigar como o peso pode afetar a pressão sanguínea. Classifique os tipos de variáveis em cada uma das situações a seguir.
\begin{enumerate}
\item {} 
O peso e a pressão sanguínea dos indivíduos são registrados.

\item {} 
Os indivíduos são classificados como abaixo do peso, normais ou acima do peso e suas pressões sanguíneas são registradas.

\item {} 
Os indivíduos são classificados como tendo pressão alta, normal ou baixa e seus pesos são registrados.

\item {} 
Os indivíduos são classificados como abaixo do peso, normais ou acima do peso e como tendo pressão alta, normal ou baixa.

\end{enumerate}

\item Numa pesquisa realizada foi descoberto que 63\% dos americanos adultos pesquisados não querem viver até os 100 anos. Em média, as pessoas pesquisadas disseram desejar viver até a idade de 91 anos.
\begin{enumerate}
\item {} 
Devemos considerar os americanos adultos pesquisados como a amostra ou como a população? Por quê?

\item {} 
Existe uma variável qualitativa de interesse nessa pesquisa; qual poderia ter sido a questão formulada na pesquisa para obter as informações dessa variável? Justifique a sua resposta.

\item {} 
Existe uma variável quantitativa de interesse nessa pesquisa; qual poderia ter sido a questão formulada na pesquisa para obter as informações dessa variável? Justifique a sua resposta.

\end{enumerate}

\item ``O telefone celular, antes tão mal visto no ambiente escolar, vai ocupando cada vez mais espaço na sala de aula: em 2016, 52\% das escolas utilizavam o aparelho em atividades com os alunos. É o que aponta a pesquisa TIC Educação 2016, do Centro de Estudos sobre as Tecnologias da Informação e da Comunicação (Cetic).
Foram coletados dados de 1.106 escolas, em turmas de 5º e 9º ano do ensino fundamental e do 2º ano do ensino médio. Participaram das entrevistas 935 diretores, 922 coordenadores pedagógicos, 1.854 professores de diversas disciplinas e 11.069 estudantes. A pesquisa aconteceu entre agosto e dezembro de 2016.'' (\sphinxhref{https://g1.globo.com/educacao/notici/52-das-instituicoes-de-educacao-basica-usam-celular-em-atividades-escolares-aponta-estudo-da-cetic.gtml}{Leia a reportagem} publicada no G1.com.br).

O gráfico a seguir ilustra um dos resultados dessa pesquisa.

\begin{figure}[H]
\centering
\capstart

\noindent\sphinxincludegraphics[width=300bp]{{internet_TIC}.png}
\caption{Gráfico do exercício 5}\label{\detokenize{PE103-E:fig-internet-tic}}\label{\detokenize{PE103-E:id16}}\end{figure}
\begin{enumerate}
\item {} 
Que variável foi analisada nesse gráfico e qual a sua classificação?

\item {} 
Que tipo de gráfico foi usado para representar as respostas dessa variável?

\item {} 
Você acha que o gráfico de setores seria adequado para representar essa distribuição de frequências? Por que?

\end{enumerate}

\item ``Poluição ambiental provoca uma em cada seis mortes no mundo'': reportagem publicada pelo jornal O Globo em 20 de outubro de 2017. Analise os gráficos publicados.
(\sphinxhref{https://oglobo.globo.com/sociedade/sustentabilidade/poluicao-matou-9-milhoes-de-pessoas-no-mundo-em-2015-21969023}{Leia a reportagem}).

\begin{figure}[H]
\centering
\capstart

\noindent\sphinxincludegraphics[width=300bp]{{ar_carregado}.png}
\caption{Gráfico do exercício 6}\label{\detokenize{PE103-E:fig-ar-carregado}}\label{\detokenize{PE103-E:id17}}\end{figure}
\begin{enumerate}
\item {} 
Que variável foi observada no primeiro gráfico? Que escala foi usada para o comprimento das barras nesse gráfico? Em que posição está o Brasil?

\item {} 
Que variável foi observada no segundo gráfico? A que se deve a mudança radical de posição do Brasil nesse ranking?

\item {} 
Por que é importante conhecer também o número absoluto de mortes atribuíveis à poluição e não olhar apenas para a proporção de mortes atribuíveis à poluição?

\item {} 
Faça uma pesquisa para obter informações sobre as principais causas de óbito no Brasil.

\end{enumerate}

\item (UFPR 2017-adaptado)  O Centro de Estudos, Resposta e Tratamento de Incidentes de Segurança no Brasil (CERT.br) é responsável por tratar incidentes de segurança em computadores e redes conectadas à Internet no Brasil. A tabela abaixo apresenta o número de mensagens não solicitadas (spams) notificadas ao CERT.br no ano de 2015, por trimestre.


\begin{savenotes}\sphinxattablestart
\centering
\begin{tabulary}{\linewidth}[t]{|T|T|}
\hline

Trimestre
&
Notificações
\\
\hline
4T
&
135.335
\\
\hline
3T
&
171.523
\\
\hline
2T
&
154.866
\\
\hline
1T
&
249.743
\\
\hline
total
&
711.467
\\
\hline
\end{tabulary}
\par
\sphinxattableend\end{savenotes}

Construa um gráfico para representar a distribuição do número de notificações por trimestre.

\item (UFRGS 2016 - adaptado)  O gráfico a seguir representa a população economicamente ativa de homens e mulheres no Brasil de 2003 a 2015.

\begin{figure}[H]
\centering
\capstart

\noindent\sphinxincludegraphics[width=300bp]{{exercicio8_enunciado}.png}
\caption{Gráfico do exercício 8}\label{\detokenize{PE103-E:fig-coloque-aqui-o-nome}}\label{\detokenize{PE103-E:id18}}\end{figure}

Classifique cada uma das afirmações a seguir em verdadeira ou falsa.
\begin{enumerate}
\item {} 
No ano de 2009, a população economicamente ativa de mulheres era cerca de 50\% da população economicamente ativa de homens.

\item {} 
De 2003 a 2015, em termos percentuais, a população economicamente ativa de homens cresceu mais do que a de mulheres.

\item {} 
Em relação a 2005, a população economicamente ativa de mulheres em 2011 cresceu cerca de 5\%.

\item {} 
De 2003 a 2015, em termos percentuais, a população economicamente ativa de mulheres cresceu mais do que a de homens.

\item {} 
Em relação a 2007, a população economicamente ativa de homens em 2015 cresceu cerca de 3\%.

\end{enumerate}

\item (ENEM 2ª aplicação 2016)  A diretoria de uma empresa de alimentos resolve apresentar para seus acionistas uma proposta de novo produto. Nessa reunião, foram apresentadas as notas médias dadas por um grupo de consumidores que experimentaram o novo produto e dois produtos similares concorrentes (A e B).

\begin{figure}[H]
\centering
\capstart

\noindent\sphinxincludegraphics[width=300bp]{{exercicio9_enunciado_1}.png}
\caption{Gráfico do exercício 9}\label{\detokenize{PE103-E:id2}}\label{\detokenize{PE103-E:id19}}\end{figure}

A característica que dá a maior vantagem relativa ao produto proposto e que pode ser usada, pela diretoria, para incentivar a sua produção é a
\begin{enumerate}
\item {} 
textura.

\item {} 
cor.

\item {} 
tamanho.

\item {} 
sabor.

\item {} 
odor.

\end{enumerate}

\item (UFRGS 2016 - adaptado)  Observe o gráfico a seguir.

\begin{figure}[H]
\centering
\capstart

\noindent\sphinxincludegraphics[width=300bp]{{exercicio10_enunciado}.png}
\caption{Gráfico do exercício 10}\label{\detokenize{PE103-E:id3}}\label{\detokenize{PE103-E:id20}}\end{figure}

Nele está retratado o número de transplantes realizados no Rio Grande do Sul, até julho de 2015, e a quantidade de pessoas que aguardam na fila por um transplante no Estado, no mês de julho de 2015.

Com base no gráfico apresentado, classifique cada afirmação a seguir em verdadeira ou falsa.
\begin{enumerate}
\item {} 
Mais da metade dos transplantes realizados até julho de 2015 foram transplantes de córnea.

\item {} 
O percentual de pessoas que aguardavam transplante de pulmão em julho de 2015 correspondeu a 70\% do total de pessoas na fila de espera por transplantes.

\item {} 
O transplante de fígado é o que apresentou maior diferença percentual entre o número de transplantes realizados e o número de pessoas que aguardavam transplante.

\item {} 
O número de transplantes de fígado realizados até julho de 2015 foi maior do que o número de transplantes de pulmão realizados no mesmo período.

\item {} 
O transplante de córneas é o que tem a menor quantidade de pessoas aguardando transplante.

\end{enumerate}

\item (UFPA 2016 - adaptado)  O gráfico abaixo, retirado do Boletim Epidemiológico 16 de 2016 do Ministério da Saúde, registra os casos de dengue por semana, no Brasil, nos anos de 2014, 2015 e início de 2016.

\begin{figure}[H]
\centering
\capstart

\noindent\sphinxincludegraphics[width=300bp]{{exercicio11_enunciado}.png}
\caption{Gráfico do exercício 11}\label{\detokenize{PE103-E:id4}}\label{\detokenize{PE103-E:id21}}\end{figure}

Com base no gráfico apresentado,
\begin{enumerate}
\item {} 
o número de casos de dengue tem comportamento crescente próximo da vigésima segunda semana?

\item {} 
os dados das 7 primeiras semanas de 2016 indicam uma diminuição do número de casos em relação a 2014 e  2015?

\item {} 
no ano de 2015 houve mais de um milhão de casos?

\item {} 
o maior número de casos ocorre em cada ano na décima quarta semana?

\item {} 
em torno de que semana do ano 2016 é esperado o maior número de casos de dengue? Por que?

\end{enumerate}

\item (ENEM 2016)  O cultivo de uma flor rara só é viável se do mês do plantio para o mês subsequente o clima da região possuir as seguintes peculiaridades:
\begin{itemize}
\item {} 
a variação do nível de chuvas (pluviosidade), nesses meses, não for superior a  50 mm

\item {} 
a temperatura mínima, nesses meses, for superior a  15°C;

\item {} 
ocorrer, nesse período, um leve aumento não superior a  5 °C na temperatura máxima.

\end{itemize}

Um floricultor, pretendendo investir no plantio dessa flor em sua região, fez uma consulta a um meteorologista que lhe apresentou o gráfico com as condições previstas para os   meses seguintes nessa região.

\begin{figure}[H]
\centering
\capstart

\noindent\sphinxincludegraphics[width=300bp]{{exercicio12_enunciado}.png}
\caption{Gráfico do exercício 12}\label{\detokenize{PE103-E:id5}}\label{\detokenize{PE103-E:id22}}\end{figure}

Com base nas informações do gráfico, o floricultor verificou que poderia plantar essa flor rara.

O mês escolhido para o plantio foi
\begin{enumerate}
\item {} 
janeiro.

\item {} 
fevereiro.

\item {} 
agosto.

\item {} 
novembro.

\item {} 
dezembro.

\end{enumerate}

\item (ENEM 2015)  O polímero de PET (Politereftalato de Etileno) é um dos plásticos mais reciclados em todo o mundo devido à sua extensa gama de aplicações, entre elas, fibras têxteis, tapetes, embalagens, filmes e cordas. Os gráficos mostram o destino do PET reciclado no Brasil, sendo que, no ano de 2010, o total de PET reciclado foi de 282 kton (quilotoneladas).

\begin{figure}[H]
\centering
\capstart

\noindent\sphinxincludegraphics[width=300bp]{{exercicio13_enunciado}.png}
\caption{Gráfico do exercício 13}\label{\detokenize{PE103-E:id6}}\label{\detokenize{PE103-E:id23}}\end{figure}

De acordo com os gráficos, a quantidade de embalagens PET recicladas destinadas a produção de tecidos e malhas, em kton é mais aproximada de
\begin{enumerate}
\item {} 
16,0

\item {} 
22,9

\item {} 
32,0

\item {} 
84,6

\item {} 
106,6

\end{enumerate}

\item (UFRGS 2015 - adaptado)  O gráfico abaixo apresenta a evolução da emissão de Dióxido de carbono ao longo dos anos.

\begin{figure}[H]
\centering
\capstart

\noindent\sphinxincludegraphics[width=300bp]{{exercicio14_enunciado}.png}
\caption{Gráfico do exercício 14}\label{\detokenize{PE103-E:id7}}\label{\detokenize{PE103-E:id24}}\end{figure}

Com base no gráfico apresentado, classifique cada afirmação a seguir em verdadeira ou falsa.
\begin{enumerate}
\item {} 
Ao longo do período, a emissão de dióxido de carbono apresentou taxa de variação constante.

\item {} 
Em relação aos anos 80, os anos 90 apresentaram emissão de dióxido de carbono 30\% maior.

\item {} 
O ano de 2009 apresentou menor valor de emissão de dióxido de carbono da primeira década do século XXI.

\item {} 
De 2000 a 2013, houve crescimento percentual de 11,7\%  na emissão de dióxido de carbono.

\item {} 
Em relação a 2000, o ano de 2013 apresentou emissão de dióxido de carbono aproximadamente 50\%  maior.

\end{enumerate}

\item (ENEM 2013)  Uma falsa relação

O cruzamento da quantidade de horas estudadas com o desempenho no Programa Internacional de Avaliação de Estudantes (Pisa) mostra que mais tempo na escola não é garantia de nota acima da média.

\begin{figure}[H]
\centering
\capstart

\noindent\sphinxincludegraphics[width=400bp]{{exercicio15_enunciado_1}.png}
\caption{Gráfico do exercício 15}\label{\detokenize{PE103-E:id8}}\label{\detokenize{PE103-E:id25}}\end{figure}

Dos países com notas abaixo da média nesse exame, aquele que apresenta maior quantidade de horas de estudo é
\begin{enumerate}
\item {} 
Finlândia.

\item {} 
Holanda.

\item {} 
Israel.

\item {} 
México.

\item {} 
Rússia.

\end{enumerate}

\item (UF-AM) O gráfico a seguir mostra quanto tempo um estudante gasta com suas atividades durante o dia.

\begin{figure}[H]
\centering
\capstart

\noindent\sphinxincludegraphics[width=300bp]{{exercicio16_enunciado_1}.png}
\caption{Gráfico do exercíco 16}\label{\detokenize{PE103-E:id9}}\label{\detokenize{PE103-E:id26}}\end{figure}

A quantidade de horas gastas pelo estudante com otras atividades em um dia é de:
\begin{enumerate}
\item {} 
2,25 h

\item {} 
3,02 h

\item {} 
3,57 h

\item {} 
5,04 h

\item {} 
6,70 h

\end{enumerate}

\item (UERJ-adaptada)  Após serem medidas as alturas dos alunos de uma turma, elaborou-se o seguinte histograma:

\begin{figure}[H]
\centering
\capstart

\noindent\sphinxincludegraphics[width=300bp]{{exercicio17_enunciado_1}.png}
\caption{Histograma referente ao exercício 17}\label{\detokenize{PE103-E:id10}}\label{\detokenize{PE103-E:id27}}\end{figure}

Em um histograma, se uma reta vertical de equação \(x=x_0\) divide o histograma em duas partes de mesma área, então o valor de \(x_0\) corresponde à \index{mediana}mediana da distribuição representada no histograma. Calcule a mediana das alturas dos alunos com base no histograma apresentado.

\item Numa pesquisa sobre a preferência dos jovens por sucos, obteve-se, entre os tipos principais A, B e C, o seguinte resultado.

\begin{figure}[H]
\centering
\capstart

\noindent\sphinxincludegraphics[width=250bp]{{exercicio19_enunciado_1}.png}
\caption{O número entre parênteses corresponde ao número de respostas para cada tipo de suco entre os jovens selecionados na pesquisa}\label{\detokenize{PE103-E:id11}}\label{\detokenize{PE103-E:id28}}\end{figure}
\begin{enumerate}
\item {} 
Olhando o gráfico é razoável dizer que a preferência pelo tipo A é maior que o dobro das preferências somadas pelo tipo C e outros tipos? Por que?

\item {} 
Refaça o gráfico de barras.

\end{enumerate}

\item (ENEM) Para convencer a população local da ineficiência da Companhia Telefônica Vilatel na expansão de oferta de linhas, um político publicou no jornal local o gráfico I, representado a seguir. A Companhia Vilatel respondeu dias depois publicando o gráfico II, com o o qual pretende justificar um grande aumento na oferta de linhas. O fato é que, no período considerado, foram instaladas, efetivamente, 200 linhas telefônicas novas.

\begin{figure}[H]
\centering
\capstart

\noindent\sphinxincludegraphics[width=300bp]{{exercicio20_enunciado_1}.png}
\caption{Gráficos do exercício 20}\label{\detokenize{PE103-E:id12}}\label{\detokenize{PE103-E:id29}}\end{figure}

Analisando os gráficos, pode-se concluir que:
\begin{enumerate}
\item {} 
o gráfico II apresenta um crescimento real maior do que o gráfico I.

\item {} 
o gráfico I apresenta um crescimento real, sendo o gráfico II incorreto.

\item {} 
o gráfico II apresenta o crescimento real, sendo o gráfico I incorreto.

\item {} 
a aparente diferença de crescimento nos dois gráficos decorre da escolha de escalas diferentes.

\item {} 
os dois gráficos são incomparáveis, pois usam escalas diferentes.

\end{enumerate}

\item As fichas dos 800 funcionários de uma empresa estão catalogadas por ordem alfabética no setor de recursos humanos. Como você faria para obter uma amostra sistemática de tamanho 50 dos funcionários dessa empresa?
\end{enumerate}

\begin{sphinxadmonition}{note}{Resposta}

\(1.\) a) I b) D c) D d) I

\(2.\) a) M b) E c) E d) M e) E f) M

\(3.\) a) peso e pressão são tratados como variáveis quantitativas contínuas b) o peso é tratado como variável qualitativa ordinal e a pressão é tratada como variável quantitativa contínua c) o peso é tratado como variável quantitativa contínua e a pressão como variável qualitativa ordinal d) ambos são tratados como variáveis qualitativas ordinais.

\(4.\) a) amostra b) ``Você deseja viver até os 100 anos?'' c) ``Até que idade você gostaria de viver?''

\(5.\) a) principal equipamento usado por aluno para acessar a internet. variável qualitativa nominal. b) Gráfico de barras. c) De fato, vimos que o gráfico de setores é um gráfico adequado para representar as frequências de respostas de variáveis qualitativas, mas nesse caso, há frequências muito pequenas($1\%$, $2\%$, $5\%$, $6\%$) e essas pequenas diferenças levarão a setores pouco distinguíveis entre si.

\(6.\) a) países com maior número absoluto de mortes atribuíveis à poluição em 2015, que é uma variável qualitativa nominal e foi organizada no gráfico em ordem decrescente de frequência. frequência absoluta de casos. décima primeira. b) a porcentagem de mortes  atribuíveis à poluição em relação ao total de óbitos em 2015. De fato, cada óbito foi classificado em ``atribuível à poluição'' ou não (variável qualitativa) e em cada país calculou-se a porcentagem de óbitos atribuíveis à poluição. Trata-se de um gráfico de barras múltiplas, para comparar os diversos países em relação a essa porcentagem. A mudança radical de posição no Brasil se deve ao fato de que em relação ao total de óbitos, os atribuíveis à poluição correspondem a apenas $7{,}49\%$, não sendo esse o caso mais comum. (Pesquise na internet sobre a distribuição de óbito por causa no Brasil) c) O número absoluto é importante, por exemplo, para que seja possível fazer planejamento de alocação de recursos na saúde.

\(7.\) Existem diversas possibilidades. A figura apresentada agui é um gráfico de barras em porcentagem com as barras na orientação horizontal.
\phantomsection\label{\detokenize{PE103-E:id13}}\begin{quote}

\begin{figure}[H]
\centering
\capstart

\noindent\sphinxincludegraphics[width=200bp]{{exercicio6_resposta}.png}
\caption{Distribuição percentual do número de notificações por trimestre}\label{\detokenize{PE103-E:id30}}\end{figure}
\end{quote}
\begin{description}
\item[{\(8.\)}] \leavevmode\begin{enumerate}
\item {} 
Falsa. As mulheres economicamente ativas eram cerca de 44 milhões e, os homens, cerca de 56 milhões, o que leva a concluir a população economicamente ativa de mulheres era cerca de 79\% da população economicamente ativa de homens.

\item {} 
Falsa. Para homens cresceu de cerca de 52 milhões para cerca de 58 milhões, o que dá um crescimento percentual relativo a 2003 de cerca de 12\%. Para mulheres cresceu de cerca de 37,5 milhões para cerca de 47,5 milhões, o que dá um crescimento percentual relativo a 2003 de cerca de 27\%.

\item {} 
Falsa. Em 2005 eram cerca de 40 milhões e, em 2011, cerca de 45 milhões, o que dá um crescimento percentual relativo a 2005 de cerca de 12,5\%.

\item {} 
Verdadeira. Ver justificativa do item b.

\item {} 
Falsa. Em 2007 eram cerca de 54 milhões e em 2015 cerca de 58 milhões, o que dá um crescimento percentual relativo a 2007 de cerca de 7\%.

\end{enumerate}
\begin{description}
\item[{\(9.\) d}] \leavevmode
A maior vantagem relativa corresponde à maior diferença entre a nota do produto proposto e as notas dos produtos A e B de tal sorte que a nota do produto proposto seja maior do que as notas alcançadas por A  e B.  Desse modo, é fácil ver que a característica a ser escolhida é o sabor.

\(10.\) Considerando a tabela dos percentuais (valores relativos), a única afirmação correta é a da letra (a).
.. table:: Porcentagens dos números de transplantes até julho 2015 e das pessoas em fila de espera em julho de 2015 por órgão


\begin{savenotes}\sphinxattablestart
\centering
\begin{tabulary}{\linewidth}[t]{|T|T|T|}
\hline
\sphinxstylethead{\sphinxstyletheadfamily 
Órgão
\unskip}\relax &\sphinxstylethead{\sphinxstyletheadfamily 
transplantes
\unskip}\relax &\sphinxstylethead{\sphinxstyletheadfamily 
fila de espera
\unskip}\relax \\
\hline
rim
&
33
&
75
\\
\hline
fígado
&
9
&
15
\\
\hline
pulmão
&
3
&
6
\\
\hline
coração
&
1
&
1
\\
\hline
rim/pâncreas
&
1
&
1
\\
\hline
córnea
&
53
&
2
\\
\hline
total
&
100
&
100
\\
\hline
\end{tabulary}
\par
\sphinxattableend\end{savenotes}

\(11.\) {[}A{]} Não. Tanto em 2014 como em 2015 o comportamento é decrescente. {[}B{]} Não. O gráfico de 2016 está acima dos gráficos de 2014 e 2015 nas sete primeiras semanas. {[}C{]} Sim. Basta observar que entre as semanas 9 e 18 o número de casos foi maior do que ou igual a 80.000. {[}D{]} Nâo. Não há informações sobre o número de casos na décima quarta semana em 2016. {[}E{]} Entre a décima terceira e a décima sétima semana, pois nos anos anteriores, 2014 e 2015, foi entre essas semanas que ocorreu o maior número de casos.

\(12.\) {[}A{]} O único mês que satisfaz todas as condições é janeiro. Com efeito,

\end{description}

\end{description}
\begin{enumerate}
\item {} 
de fevereiro para março e de novembro para dezembro houve redução na temperatura máxima;

\item {} 
a variação da pluviosidade de agosto para setembro e de dezembro para janeiro foi maior do que  50 mm.

\end{enumerate}

\(13.\) {[}C{]} Sendo de 37,8\%  a porcentagem do total de PET reciclado para uso final têxtil, e de 30\%  dessa quantidade para tecidos e malhas, segue que a resposta é dada por \(\frac{37,8}{100}\cdot \frac{30}{100}\cdot 282 \approx 32,0 \textsf{ kton}\)

\(14.\) {[}A{]} Falsa, houve períodos de crescimento e períodos de decaimento, portanto, com taxas de variação diferentes. {[}B{]} Falsa, pois 22,3 \textendash{} 19,3 não representam 30\% de 19,3. {[}C{]} Falsa, pois em 2005 e 2006 as emissões foram inferiores à emissão em 2008.  {[}D{]} Falsa, pois 36,3 \textendash{} 24,6 = 11,7, aproximadamente 50\%. {[}E{]} Verdadeira, pois 36,3 \textendash{} 24,6 = 11,7, aproximadamente 50\% de 24,6.

\(15.\) {[}C{]} Os países com notas abaixo da média de 24,6 são: Rússia, Portugal, México, Itália e Israel. Dentre esses países, o que apresenta maior quantidade de horas de estudo é Israel.

\(16.\) A porcentagem de horas do dia com outras atividades é dada por \(100-(25+10+30+14)=21%\). 21\% de 24 h é dado por \(\frac{21}{100}\cdot 24 =5,04\) h. A opção correta é a {[}D{]}.

\(17.\) Observe que o histograma apresentado é composto por quatro retângulos cujas bases medem 0,1. Assim a área total do histograma é dada por \(0,1\cdot (3+9+6+2)=2,0\) , ou seja, soma das áreas dos 4 retângulos que compõem o histograma. Assim, a metade da área corresponde ao valor 1,0.

Considerando os dois primeiros retângulos, a área é \(0,1\cdot 12=1,2\) que supera a metade da área total. Isso significa que a mediana será um valor que está entre 1,70 e 1,80. Considerando o primeiro retângulo, falta para completar 1 um sub-retângulo do segundo, com área igual a 0,7, ou seja, \((x_0-1,7)\cdot 9=0,7\) tal que \(x_0=1,7+ \frac{0,7}{9}\approx 1,78\) m.
\phantomsection\label{\detokenize{PE103-E:id14}}\begin{quote}

\begin{figure}[H]
\centering
\capstart

\noindent\sphinxincludegraphics[width=300bp]{{exercicio17_resposta}.png}
\caption{Esquema da resposta do exercício 17}\label{\detokenize{PE103-E:id31}}\end{figure}
\end{quote}

\(18.\) De fato, pelo comprimento das barras, a barra correspondente ao tipo A tem comprimento maior do que o dobro dos comprimentos somados das barras correspondentes ao tipo C e outros. O gráfico não está correto, pois não respeita a escala de frequências: usa o mesmo tamanho para representar 10 unidades de 0 a 60, para representar apenas uma unidade entre 71 e 75.

Gráfico adequado:
\phantomsection\label{\detokenize{PE103-E:id15}}\begin{quote}

\begin{figure}[H]
\centering
\capstart

\noindent\sphinxincludegraphics[width=300bp]{{exercicio19_resposta}.png}
\caption{Resposta do exercício 18}\label{\detokenize{PE103-E:id32}}\end{figure}
\end{quote}

\(19.\) A opção correta é a letra {[}D{]}: a inclinação maior no gráfico II comparada ao gráfico I deve-se a escolha de escalas distintas. No gráfico I a amplitude do intervalo no eixo vertical para 50 unidades é um pouco menor do que a amplitude correspondente utilizada no gráfico II. As outras opções estão incorretas.

\(20.\) Considere a organização dos 800 funcionários em ordem alfabética: a cada funcionário corresponde uma posição de 1 a 800. Temos que \(\frac{800}{50}=16\). Sorteie ao acaso um número entre 1 e 16, por exemplo 5. Agora considere os 50 primeiros termos de uma Progressão Aritmética de primeiro termo 5 e razão 16: 5, 21, 37, …, 789. Selecione então os 50 funcionários correspondentes a essas posições na listagem em ordem alfabética.
\end{sphinxadmonition}



% 

\renewcommand\chapterillustration{./abertura-estatistica1}\ %Photo by Annie Spratt on Unsplash, https://unsplash.com/photos/MwxZTqG6cUw
\renewcommand\chapterwhat{Medidas de posição: média, mediana, moda e quartis. Medidas de dispersão:
desvio médio, variância, desvio padrão, amplitude amostral, distância entre quartis e
coeficiente de variação. Construção do desenho-esquemático (boxplot).}\ 
\renewcommand\chapterbecause{As medidas resumo (posição e dispersão) correspondem a uma síntese do conjunto de
dados observados e ao passo preliminar para fazer uma inferência estatística, ou seja, a
partir das informações obtidas na amostra, expandir nossas conclusões para a população.
Como as distribuições podem apresentar formas variadas é importante conhecer
diferentes tipos de medidas resumo, tanto de posição como de dispersão, para usar
medidas apropriadas em cada caso.}\ 
\chapter{Medidas de posição e dispersão}
\label{\detokenize{PE104:medidas-de-posicao-e-dispersao}}\label{\detokenize{PE104::doc}}

\explore{medidas de posição}
\label{\detokenize{PE104-0:sec-explorando1}}\label{\detokenize{PE104-0:explorando-medidas-de-posicao}}\label{\detokenize{PE104-0::doc}}
No capítulo \sphinxstylestrong{A Natureza da Estatística} trabalhamos com representações gráficas de conjuntos de dados com a finalidade de obter informações sobre estruturas da sua distribuição como estratégia para resumir os dados.
No exemplo dos resgistros de tempo deste capítulo, os 64 dados, no quadro a seguir


\begin{savenotes}\sphinxattablestart
\centering
\sphinxcapstartof{table}
\sphinxcaption{64 registros de tempo de atividade do capítulo \sphinxstylestrong{A Natureza da Estatística}}\label{\detokenize{PE104-0:id3}}
\sphinxaftercaption
\begin{tabulary}{\linewidth}[t]{|T|T|T|T|T|T|T|T|}
\hline
\sphinxstylethead{\sphinxstyletheadfamily 
A
\unskip}\relax &\sphinxstylethead{\sphinxstyletheadfamily 
B
\unskip}\relax &\sphinxstylethead{\sphinxstyletheadfamily 
C
\unskip}\relax &\sphinxstylethead{\sphinxstyletheadfamily 
D
\unskip}\relax &\sphinxstylethead{\sphinxstyletheadfamily 
E
\unskip}\relax &\sphinxstylethead{\sphinxstyletheadfamily 
F
\unskip}\relax &\sphinxstylethead{\sphinxstyletheadfamily 
G
\unskip}\relax &\sphinxstylethead{\sphinxstyletheadfamily 
H
\unskip}\relax \\
\hline
3,03
&
4,37
&
5,04
&
5,73
&
4,03
&
5,37
&
6,04
&
6,74
\\
\hline
3,38
&
4,46
&
5,11
&
5,84
&
4,38
&
5,46
&
6,11
&
6,84
\\
\hline
3,60
&
4,55
&
5,19
&
5,95
&
4,60
&
5,55
&
6,19
&
6,96
\\
\hline
3,78
&
4,63
&
5,29
&
6,08
&
4,78
&
5,64
&
6,29
&
7,08
\\
\hline
3,92
&
4,71
&
5,36
&
6,23
&
4,92
&
5,72
&
6,36
&
7,23
\\
\hline
4,04
&
4,79
&
5,45
&
6,41
&
5,04
&
5,79
&
6,45
&
7,40
\\
\hline
4,16
&
4,87
&
5,54
&
6,62
&
5,16
&
5,87
&
6,54
&
7,63
\\
\hline
4,27
&
4,95
&
5,64
&
6,97
&
5,26
&
5,95
&
6,64
&
7,97
\\
\hline
\end{tabulary}
\par
\sphinxattableend\end{savenotes}

foram organizados em 10 intervalos de classe, como mostra a tabela a seguir


\begin{savenotes}\sphinxattablestart
\centering
\sphinxcapstartof{table}
\sphinxcaption{Registros de tempo agrupados em intervalos de classe}\label{\detokenize{PE104-0:id4}}
\sphinxaftercaption
\begin{tabulary}{\linewidth}[t]{|T|T|}
\hline
\sphinxstylethead{\sphinxstyletheadfamily 
Intervalo de classe
\unskip}\relax &\sphinxstylethead{\sphinxstyletheadfamily 
Número de observações
\unskip}\relax \\
\hline
{[} 3,0 ; 3,5 {[}
&
2
\\
\hline
{[} 3,5 ; 4,0 {[}
&
3
\\
\hline
{[} 4,0 ; 4,5 {[}
&
7
\\
\hline
{[} 4,5 ; 5,0 {[}
&
9
\\
\hline
{[} 5,0 ; 5,5 {[}
&
11
\\
\hline
{[} 5,5 ; 6,0 {[}
&
11
\\
\hline
{[} 6,0 ; 6,5 {[}
&
9
\\
\hline
{[} 6,5 ; 7,0 {[}
&
7
\\
\hline
{[} 7,0 ; 7,5 {[}
&
3
\\
\hline
{[} 7,5 ; 8,0 {[}
&
2
\\
\hline
\end{tabulary}
\par
\sphinxattableend\end{savenotes}

que, por sua vez, foi usada para construir um gráfico, o histograma a seguir.

\begin{figure}[H]
\centering
\capstart

\noindent\sphinxincludegraphics[width=400bp]{{Histograma-resposta_1}.png}
\caption{Histograma dos registros de tempo}\label{\detokenize{PE104-0:fig-histograma-resposta}}\label{\detokenize{PE104-0:id5}}\end{figure}

Observe que os 64 registros de tempo foram resumidos numa representação gráfica que revela o comportamento destes dados: registros de tempo entre 3,0 e 8,0, estrutura simétrica em torno da média dos resgistros de tempo que é 5,5.

O capítulo \sphinxstylestrong{Medidas de Posição e Dispersão} tem como objetivo responder, entre outras, as seguintes perguntas sobre um conjunto de dados quantitativos.
\begin{enumerate}
\item {} 
É possível encontrar valor(es) para resumir as observações? Qual(is) seria(m) este(s) valor(es)? Como encontrá-lo(s)?

\item {} 
Como medir se os dados estão “próximos” ou “afastados” uns dos outros?

\item {} 
Como você classifica a forma do gráfico construído para representar os dados?

\item {} 
Existe algum valor muito diferente dos demais? Como identificá-lo?

\end{enumerate}

Ao longo deste capítulo veremos como resumir a informação dos dados, usando apenas algumas medidas que caracterizam a distribuição em vez de usar toda a coleção de dados para descrevê-la. Por esta razão, tais medidas são chamadas medidas resumo.

Como as distribuições podem apresentar formas variadas é importante conhecer diferentes tipos de medidas resumo, tanto de posição como de dispersão, para usar medidas apropriadas em cada caso.

\phantomsection\label{\detokenize{PE104-0:ativ-notas-de-artes}}
\begin{task}{ notas de Artes}

Ao final de um trimestre, um professor de Artes registrou as seguintes notas de seus 35 alunos, listadas no quadro a seguir, em ordem crescente.


\begin{savenotes}\sphinxattablestart
\centering
\begin{tabulary}{\linewidth}[t]{|T|T|T|T|T|T|T|}
\hline

0,8
&
2,0
&
2,0
&
2,5
&
2,5
&
3,5
&
4,5
\\
\hline
5,0
&
5,4
&
5,5
&
5,5
&
5,5
&
6,0
&
6,0
\\
\hline
6,0
&
6,0
&
6,3
&
6,5
&
6,8
&
6,8
&
7,0
\\
\hline
7,0
&
7,0
&
7,0
&
7,3
&
7,3
&
7,5
&
7,5
\\
\hline
7,5
&
7,5
&
7,8
&
8,0
&
8,0
&
8,0
&
8,0
\\
\hline
\end{tabulary}
\par
\sphinxattableend\end{savenotes}

Este professor verificou que a média da turma foi aproximadamente 5,93 (soma das notas \(S=207,5\)). Como a participação da turma foi muito boa ao longo do trimestre, o professor resolveu dar uma bonificação na nota de cada aluno desta turma, pensando em duas possibilidades:
\begin{enumerate}
\item {} 
acrescentar um ponto para cada aluno da turma;

\item {} 
aumentar em $20\%$ a nota de cada aluno da turma.



Na figura a seguir veja um histograma das notas sem a bonificação.  A tabela com os intervalos de classe considerados na construção do histograma é dada por


\begin{savenotes}\sphinxattablestart
\centering
\sphinxcapstartof{table}
\sphinxcaption{Distribuição de frequências das notas antes de bonificação}\label{\detokenize{PE104-0:id6}}
\sphinxaftercaption
\begin{tabulary}{\linewidth}[t]{|T|T|}
\hline

intervalo
&
frequência absoluta
\\
\hline
{[}0,2{[}
&
1
\\
\hline
{[}2,4{[}
&
5
\\
\hline
{[}4,6{[}
&
6
\\
\hline
{[}6,8{]}
&
23
\\
\hline
\end{tabulary}
\par
\sphinxattableend\end{savenotes}

\begin{figure}[H]
\centering
\capstart

\noindent\sphinxincludegraphics[width=120bp]{{histogramaNotas_E1_1_1}.png}
\caption{Histograma das notas de Artes sem bonificação}\label{\detokenize{PE104-0:fig-histograma-notas-sem-bonificacao}}\label{\detokenize{PE104-0:id7}}\end{figure}

Os dois histogramas a seguir correspondem às notas, após usar cada uma das possibilidades consideradas pelo professor, mantendo quatro intervalos de classe, conforme as tabelas de frequências apresentadas.

\begin{figure}[H]
\centering
\capstart

\noindent\sphinxincludegraphics[width=200bp]{{histogramaNotas_E1_2e3_1}.png}
\caption{Histogramas das notas de Artes com bonificação}\label{\detokenize{PE104-0:fig-histogramas-notas-aleteradas}}\label{\detokenize{PE104-0:id8}}\end{figure}

\begin{minipage}[h]{.45\textwidth}
\begin{savenotes}\sphinxattablestart
\centering
\sphinxcapstartof{table}
\sphinxcaption{Distribuição de frequências das notas após acréscimo de 1 ponto a cada nota}\label{\detokenize{PE104-0:id9}}
\sphinxaftercaption
\begin{tabulary}{\linewidth}[t]{|c|c|}
\hline

intervalo
&
frequência absoluta
\\
\hline
${[}1;3{[}$
&
1
\\
\hline
${[}3;5{[}$
&
5
\\
\hline
${[}5;7{[}$
&
6
\\
\hline
${[}7;9{]}$
&
23
\\
\hline
\end{tabulary}
\par
\sphinxattableend\end{savenotes}
\end{minipage}\hfill \begin{minipage}[h]{.45\textwidth}
\begin{savenotes}\sphinxattablestart
\centering
\sphinxcapstartof{table}
\sphinxcaption{Distribuição de frequências das notas após aumento de 20\% sobre a nota}\label{\detokenize{PE104-0:id10}}
\sphinxaftercaption
\begin{tabulary}{\linewidth}[t]{|c|c|}
\hline

intervalo
&
frequência absoluta
\\
\hline
{[}0 ; 2,4{[}
&
1
\\
\hline
{[}2,4 ; 4,8{[}
&
5
\\
\hline
{[}4,8 ; 7,2{[}
&
6
\\
\hline
{[}7,2 ; 9,6{]}
&
23
\\
\hline
\end{tabulary}
\par
\sphinxattableend\end{savenotes}
\end{minipage}

\item {} 
Compare os histogramas das notas com bonificação com o histograma original. O que mudou em cada um deles em relação ao original?

\item {} 
Considerando os {\hyperref[\detokenize{PE104-0:fig-histogramas-notas-aleteradas}]{\sphinxcrossref{\DUrole{std,std-ref}{Histogramas das notas de Artes com bonificação}}}}, identifique qual deles corresponde ao  acréscimo de 1,0 ponto, assinalando (a) e qual deles corresponde ao aumento de 20\% das notas originais, assinalando (b).

\item {} 
Dada a informação inicial de que a média da turma foi 5,93, de quanto será a média se o professor acrescentar um ponto a cada aluno? E se ele aumentar em 20\% a nota de cada aluno?

\item {} 
Se você fosse um aluno desta turma, que possibilidade de bonificação você escolheria? Para que notas é melhor cada uma das estratégias?

\end{enumerate}
\end{task}


\phantomsection\label{\detokenize{PE104-0:ativ-maratona-de-ny}}
\begin{task}{ a maratona}

A maratona é uma prova de atletismo que consiste em correr uma distância de 42,195 km. Pelas suas características, este tipo de prova é realizada nas ruas de uma grande cidade ou na estrada. As principais cidades do mundo realizam um destes eventos anualmente, recebendo milhares de atletas profissionais e amadores que encaram o desafio e almejam finalizar a corrida ou melhorar o próprio tempo do passado.

Uma das mais famosas é a Maratona da Cidade de Nova Iorque, nos Estados Unidos. Veja na figura {\hyperref[\detokenize{PE104-0:fig-maratona-ny}]{\sphinxcrossref{\DUrole{std,std-ref}{Corredores participando da Maratona de Nova York, Wikipedia}}}} realização de uma maratona em Nova Iorque. Com mais de 50.000 participantes cada ano, é um dos principais eventos do atletismo mundial, junto com as maratonas de Chicago, Londres, Boston, Berlim e Tóquio.

\begin{figure}[H]
\centering
\capstart

\noindent\sphinxincludegraphics[width=200bp]{{New_York_marathon_Verrazano_bridge}.jpg}
\caption{Corredores participando da Maratona de \sphinxstyleemphasis{Nova York}, \sphinxhref{https://commons.wikimedia.org/wiki/File:New\_York\_marathon\_Verrazano\_bridge.jpg}{Wikipedia}}\label{\detokenize{PE104-0:fig-maratona-ny}}\label{\detokenize{PE104-0:id11}}\end{figure}

Os resultados do evento são divididos nas categorias de homens e mulheres, além disso, no evento participam cadeirantes e pessoas usando triciclos de mão (\sphinxstyleemphasis{handcycle}), categorias cujos resultados são premiados e publicados separadamente. Qual das categorias você acha que terá os melhores resultados na maratona? Em quanto tempo você acha que uma pessoa percorre os 42,195 km? O que você acha ser mais rápido: correr em cadeira de rodas ou em triciclo de mão?
\phantomsection\label{\detokenize{PE104-0:handcycle}}
\begin{figure}[H]
\centering

\noindent\sphinxincludegraphics[width=200bp]{{Handcycle_in_Richmond_Park_-_geograph.org.uk_-_1315077}.jpg}
\label{\detokenize{PE104-0:handcycle}}\end{figure}

A seguir analisaremos os tempos de corrida das 100 melhores atletas na categoria de Mulheres da Maratona de Nova York do ano 2017, dados disponíveis no \sphinxhref{http://results.nyrr.org/event/M2017/finishers}{site oficial da competição}.

Observe no quadro a seguir, que os tempos já estão ordenados do menor para o maior e que para identificar o tempo da quadragésima sétima chegada, basta tomar a interseção da linha 7 com a coluna +40 para obter o tempo 2:55:36


\begin{savenotes}\sphinxattablestart
\centering
\sphinxcapstartof{table}
\sphinxcaption{100 melhores tempos de finalização da Maratona de Nova Iorque 2017 para mulheres (hora:minuto:segundo)}\label{\detokenize{PE104-0:id12}}
\sphinxaftercaption
\begin{tabulary}{\linewidth}[t]{|T|T|T|T|T|T|T|T|T|T|T|}
\hline
&
+0
&
+10
&
+20
&
+30
&
+40
&
+50
&
+60
&
+70
&
+80
&
+90
\\
\hline
1
&
2:26:53
&
2:32:01
&
2:42:52
&
2:49:44
&
2:53:59
&
2:56:58
&
2:58:35
&
2:59:36
&
3:01:24
&
3:03:43
\\
\hline
2
&
2:27:54
&
2:32:09
&
2:44:26
&
2:49:59
&
2:54:42
&
2:57:05
&
2:58:36
&
2:59:41
&
3:01:26
&
3:03:46
\\
\hline
3
&
2:28:08
&
2:33:18
&
2:44:48
&
2:50:04
&
2:54:52
&
2:57:10
&
2:58:50
&
2:59:43
&
3:01:28
&
3:04:02
\\
\hline
4
&
2:29:36
&
2:34:10
&
2:45:20
&
2:50:05
&
2:55:04
&
2:57:40
&
2:58:52
&
2:59:46
&
3:01:44
&
3:04:04
\\
\hline
5
&
2:29:39
&
2:34:23
&
2:45:52
&
2:51:11
&
2:55:25
&
2:57:49
&
2:58:56
&
2:59:51
&
3:02:09
&
3:04:17
\\
\hline
6
&
2:29:39
&
2:36:38
&
2:46:45
&
2:53:01
&
2:55:34
&
2:57:49
&
2:59:01
&
2:59:56
&
3:02:15
&
3:04:26
\\
\hline
7
&
2:29:41
&
2:37:22
&
2:47:04
&
2:53:02
&
2:55:36
&
2:57:50
&
2:59:03
&
3:00:02
&
3:02:39
&
3:04:42
\\
\hline
8
&
2:29:56
&
2:37:33
&
2:47:30
&
2:53:02
&
2:55:39
&
2:58:08
&
2:59:10
&
3:00:05
&
3:02:41
&
3:04:49
\\
\hline
9
&
2:31:21
&
2:39:01
&
2:47:35
&
2:53:19
&
2:56:47
&
2:58:23
&
2:59:16
&
3:00:49
&
3:02:56
&
3:04:58
\\
\hline
10
&
2:31:44
&
2:40:09
&
2:49:37
&
2:53:38
&
2:56:57
&
2:58:26
&
2:59:23
&
3:01:18
&
3:03:32
&
3:05:09
\\
\hline
\end{tabulary}
\par
\sphinxattableend\end{savenotes}
\end{task}




\begin{reflection}
\begin{itemize}
\item {} 
Como você calcularia a média de valores em horas, minutos e segundos,  como os da tabela?

\item {} 
Como você construiria um histograma com estes dados? Como você definiria os limites dos intervalos? (Consulte a \DUrole{xref,std,std-ref}{ativ-construcao-histograma} do capítulo \sphinxstylestrong{A Natureza da Estatística} em caso de dúvida.)

\item {} 
Qual o maior tempo em que uma corredora deveria completar a maratona para ficar entre as 25 primeiras? E entre as 50 primeiras?

\end{itemize}
\end{reflection}

Para calcular a média destes dados é conveniente reduzi-los a uma única unidade de medida, pois, caso contrário, seria necessário calcular três médias e, ainda fazer conversões apropriadas para obter a resposta em hora:minuto:segundo. Convertendo todos os tempos para horas, obtemos o seguinte quadro de tempos.


\begin{savenotes}\sphinxattablestart
\centering
\sphinxcapstartof{table}
\sphinxcaption{100 melhores tempos de finalização da Maratona de Nova Iorque 2017 para mulheres (em horas)}\label{\detokenize{PE104-0:id13}}
\sphinxaftercaption
\begin{tabulary}{\linewidth}[t]{|T|T|T|T|T|T|T|T|T|T|T|}
\hline
\sphinxstylethead{\sphinxstyletheadfamily \unskip}\relax &\sphinxstylethead{\sphinxstyletheadfamily 
+0
\unskip}\relax &\sphinxstylethead{\sphinxstyletheadfamily 
+10
\unskip}\relax &\sphinxstylethead{\sphinxstyletheadfamily 
+20
\unskip}\relax &\sphinxstylethead{\sphinxstyletheadfamily 
+30
\unskip}\relax &\sphinxstylethead{\sphinxstyletheadfamily 
+40
\unskip}\relax &\sphinxstylethead{\sphinxstyletheadfamily 
+50
\unskip}\relax &\sphinxstylethead{\sphinxstyletheadfamily 
+60
\unskip}\relax &\sphinxstylethead{\sphinxstyletheadfamily 
+70
\unskip}\relax &\sphinxstylethead{\sphinxstyletheadfamily 
+80
\unskip}\relax &\sphinxstylethead{\sphinxstyletheadfamily 
+90
\unskip}\relax \\
\hline
1
&
2,448
&
2,534
&
2,714
&
2,829
&
2,900
&
2,949
&
2,976
&
2,993
&
3,023
&
3,062
\\
\hline
2
&
2,465
&
2,536
&
2,741
&
2,833
&
2,912
&
2,951
&
2,977
&
2,995
&
3,024
&
3,063
\\
\hline
3
&
2,469
&
2,555
&
2,747
&
2,834
&
2,914
&
2,953
&
2,981
&
2,995
&
3,024
&
3,067
\\
\hline
4
&
2,493
&
2,569
&
2,756
&
2,835
&
2,918
&
2,961
&
2,981
&
2,996
&
3,029
&
3,068
\\
\hline
5
&
2,494
&
2,573
&
2,764
&
2,853
&
2,924
&
2,964
&
2,982
&
2,998
&
3,036
&
3,071
\\
\hline
6
&
2,494
&
2,611
&
2,779
&
2,884
&
2,926
&
2,964
&
2,984
&
2,999
&
3,038
&
3,074
\\
\hline
7
&
2,495
&
2,623
&
2,784
&
2,884
&
2,927
&
2,964
&
2,984
&
3,001
&
3,044
&
3,078
\\
\hline
8
&
2,499
&
2,626
&
2,792
&
2,884
&
2,928
&
2,969
&
2,986
&
3,001
&
3,045
&
3,080
\\
\hline
9
&
2,523
&
2,650
&
2,793
&
2,889
&
2,946
&
2,973
&
2,988
&
3,014
&
3,049
&
3,083
\\
\hline
10
&
2,529
&
2,669
&
2,827
&
2,894
&
2,949
&
2,974
&
2,990
&
3,022
&
3,059
&
3,086
\\
\hline
\end{tabulary}
\par
\sphinxattableend\end{savenotes}
\begin{enumerate}
\item {} 
Construa um histograma dos dados convertidos para horas, completando a tabela a seguir, que indica os intervalos de classe.

\end{enumerate}


\begin{savenotes}\sphinxattablestart
\centering
\sphinxcapstartof{table}
\sphinxcaption{Guia para o registro de frequências dos intervalos}\label{\detokenize{PE104-0:id14}}
\sphinxaftercaption
\begin{tabulary}{\linewidth}[t]{|T|T|}
\hline

Intervalo
&
Frequência
\\
\hline
{[}2,4480 ; 2,5118{[}
&\\
\hline
{[}2,5118 ; 2,5756{[}
&\\
\hline
{[}2,5756 ; 2,6394{[}
&\\
\hline
{[}2,6394 ; 2,7032{[}
&\\
\hline
{[}2,7032 ; 2,7670{[}
&\\
\hline
{[}2,7670 ; 2,8308{[}
&\\
\hline
{[}2,8308 ; 2,8946{[}
&\\
\hline
{[}2,8946 ; 2,9584{[}
&\\
\hline
{[}2,9584 ; 3,0222{[}
&\\
\hline
{[}3,0222 ; 3,0860{[}
&\\
\hline
\end{tabulary}
\par
\sphinxattableend\end{savenotes}

\begin{figure}[H]
\centering
\capstart

\noindent\sphinxincludegraphics[width=200bp]{{Histograma_mulheres}.png}
\caption{Eixos para a construção do histograma}\label{\detokenize{PE104-0:hist-maratona-mulheres}}\label{\detokenize{PE104-0:id15}}\end{figure}
\begin{enumerate}
\setcounter{enumi}{1}
\item {} 
Que características da distribuição dos 100 melhores tempos para mulheres podem ser destacadas, analisando-se o histograma construído?

\item {} 
Calcule o tempo médio dos 100 melhores tempos das corredoras, sabendo que a soma dos tempos é 286,978 horas. Localize o valor encontrado no eixo horizontal do histograma. Em que posição ficaria uma corredora cujo tempo no qual completou a maratona é igual ao tempo médio calculado neste item?

\item {} 
Trace linhas verticais no histograma de modo a separar as classificações em 4 grupos: uma linha vertical que identifica o 25o. lugar, separando os 25 primeiros colocados dos demais; outra, que identifica a 50a. classificação e, por fim, uma que marca o 75o. tempo na classificação geral.

As marcações dos tempos das 25a., 50a. e 75a. posições neste conjunto de 100 observações são chamadas de quartis da distribuição, este conceito será formalizado adiante.

\item {} 
Considerando as marcações realizadas no item anterior, determine aproximadamente as medidas das áreas contando os retângulos da grade do histograma correspondentes aos seguintes intervalos de tempo
\begin{enumerate}
\item {} 
primeiro lugar até o 25o.;

\item {} 
25o. até o 50o.;

\item {} 
50o. até o 75o.;

\item {} 
75o. até o 100o.;

\end{enumerate}

e compare-as.

\item {} 
Calcule os comprimentos dos intervalos de tempo determinados pela proposta de divisão no item (d) e compare-os.

\end{enumerate}


\begin{savenotes}\sphinxattablestart
\centering
\begin{tabular}[t]{|*{2}{\X{1}{2}|}}
\hline

Intervalo
&
comprimento
\\
\hline
1o. a 25o.
&\\
\hline
25o. a 50o  .
&\\
\hline
50o. a 75o.
&\\
\hline\begin{enumerate}
\setcounter{enumi}{749}
\item {} 
a 100o.

\end{enumerate}
&\\
\hline
\end{tabular}
\par
\sphinxattableend\end{savenotes}
\begin{enumerate}
\setcounter{enumi}{6}
\item {} 
O valor obtido para o tempo médio coincide com alguma das outras marcas feitas no histograma?

\item {} 
Observe que o tempo médio dos 100 melhores tempos para mulheres e o tempo da  corredora que chegou em 50o. lugar são diferentes. Qual deles você escolheria como medida resumo destes dados? Por quê?

\end{enumerate}




\arrange{  medidas de posição}
\label{\detokenize{PE104-1:sec-organizando1}}\label{\detokenize{PE104-1::doc}}\label{\detokenize{PE104-1:organizando-as-ideias-medidas-de-posicao}}
Medidas de Posição, como o próprio termo indica, visam a resumir um conjunto de dados em geral numa única medida em algum lugar geométrico entre os extremos observados do conjunto (mínimo e máximo). Veja na figura a seguir, as marcações da média e da mediana das notas de Artes sem bonificação.

\begin{figure}[H]
\centering
\capstart

\noindent\sphinxincludegraphics[width=200bp]{{medidas_posicao_artes}.png}
\caption{Média e mediana assinaladas no Histograma de das notas de Artes}\label{\detokenize{PE104-1:fig-coloque-aqui-o-nome}}\label{\detokenize{PE104-1:id6}}\end{figure}

Só é possível obter medidas como a média e a mediana, se nossas observações são de natureza quantitativa, pois, como vimos no capítulo
\sphinxstylestrong{A Natureza da Estatística}, as variáveis qualitativas estão no domínio da frequência apenas, ou seja, só podemos contar quantas observações ocorrem em cada categoria da variável qualitativa, mas não podemos operar matematicamente com as categorias em si. Por exemplo, na atividade Prática de Atividades Físicas deste capítulo, trabalhamos com a variável modalidade do esporte praticado. As modalidades correspondem à “Futebol”, “Caminhada”, “Fitness”, etc. Observe que são respostas não numéricas e, por isso, não podemos calcular uma média e não existe uma relação de ordem natural das respostas. Apenas podemos ordenar as respostas pela frequência na qual elas ocorreram.

As principais medidas de posição usadas na Estatística são a média, a mediana, a moda e os quartis da distribuição. Outras medidas de posição existem, mas não são tão usuais.

Definiremos a seguir as principais medidas que buscam de alguma forma resumir a informação do conjunto.

Para definir várias medidas a serem estudadas neste capítulo vamos adotar a seguinte notação.

\begin{example}{Idade de pessoas que tomaram a vacina da febre amarela}

Suponha que na primeira segunda-feira do mês de março de 2018, um Posto de Saúde tenha registrado as idades (em anos completos) das seis primeiras pessoas que chegaram para tomar a vacina da febre amarela e, os registros, obtidos foram \(\{55, 22, 30, 14, 25, 40\}\). Neste exemplo dizemos que o número de observações, denotado por \(n\), é \(6\) e que as observações são dadas por \(x_1=55\), \(x_2=22\), \(x_3=30\), \(x_4=14\), \(x_5=25\) e \(x_6=40\).

De um modo geral, sejam \(x_1,x_2, \cdots, x_n\) , os \(n\) valores observados de uma variável quantitativa tal que

\(x_1\) é o primeiro valor observado; \(x_2\) é o segundo valor observado; e, assim por diante, tal que \(x_n\) é o último valor observado.

Os valores observados não ocorrem necessariamente de forma ordenada do menor para o maior. Neste exemplo, das idades das três primeiras pessoas que chegaram para tomar a vacina no Posto de Saúde foram \(x_1=55\), \(x_2=22\) e \(x_3=30\) de modo que \(x_1>x_2\) e \(x_2<x_3\).

Para definir a mediana, será útil usar uma notação para representar os dados ordenados.

Sejam \(x_{(1)}\) o menor valor do conjunto \(\{ x_1,x_2,...,x_n\}\); \(x_{(2)}\), o segundo menor valor do conjunto \(\{ x_1,x_2,...,x_n\}\); e assim sucessivamente até \(x_{(n)}\), o maior valor do conjunto \(\{ x_1,x_2,...,x_n\}\).

Desse modo,
\(x_{(1)}\leq x_{(2)}\leq \cdots\leq x_{(n)}\) são os valores ordenados do conjunto \(\{ x_1,x_2,...,x_n\}\).

No exemplo das idades das seis primeiras pessoas que chegaram para tomar a vacina no Posto de Saúde, os registros obtidos foram \(\{55, 22, 30, 14, 25, 40\}\) tal que

\(x_1=55\), \(x_2=22\), \(x_3=30\), \(x_4=14\), \(x_5=25\) e \(x_6=40\)

e

\(x_{(1)}=14\), \(x_{(2)}=22\), \(x_{(3)}=25\), \(x_{(4)}=30\), \(x_{(5)}=40\) e \(x_{(6)}=55\).

\end{example}

A letra maiúscula sigma \(\left (\Sigma\right )\) é usada para denotar somatório, simplificando algumas fórmulas. Por exemplo,
\begin{equation*}
\begin{split}\sum^n_{i=1} x_i=x_1+x_2+\cdots +x_n\end{split}
\end{equation*}
e
\begin{equation*}
\begin{split}\sum^n_{i=1} x^2_i=x^2_1+x^2_2+\cdots +x^2_n\end{split}
\end{equation*}
Observe que neste exemplo, das idades das seis primeiras pessoas que chegaram para tomar a vacina no Posto de Saúde,
\begin{equation*}
\begin{split}\sum^6_{i=1}x_i=x_1+x_2+x_3+x_4+x_5+x_6=\\
55 + 22 + 30 + 14 + 25 + 40 = 186\end{split}
\end{equation*}
e
\begin{equation*}
\begin{split}\sum^n_{i=1} x^2_i=x^2_1+x^2_2+x^2_3+x^2_4+x^2_5 +x^2_6=\\
55^2+ 22^2+ 30^2+ 14^2+ 25^2+  40^2=6.830\end{split}
\end{equation*}
\sphinxstylestrong{Média}

A definição de média de um conjunto de dados quantitativos já é conhecida desde o Ensino Fundamental e, consiste na soma dos valores do conjunto dividida pelo número de observações. No exemplo das idades das seis primeiras pessoas que chegaram para tomar a vacina no Posto de Saúde, a soma das idades é 186 tal que a média será dada por \(\frac{186}{6}=31\) anos.

De modo mais geral, considere um conjunto contendo \(n\) valores de uma variável quantitativa representado por \(\{x_1,x_2,\cdots,x_n\}\).
A \index{média}média deste conjunto, denotada por \(\bar{x}\),  é definida por
\begin{equation*}
\begin{split}\bar{x}=\frac{\sum^n_{i=1}x_i}{n}=\frac{x_1+x_2+\cdots x_n}{n}\end{split}
\end{equation*}
Observe que a média pode substituir todas as observações sem alterar a  soma dos valores, isto é,
\begin{equation*}
\begin{split}x_1+x_2+\cdots+x_n=\bar{x}+\bar{x}+\cdots+\bar{x} = n\cdot \bar{x}\end{split}
\end{equation*}
fornecendo a expressão que define a média, denotada por \(\bar{x}\) .

Esta é justamente a ideia por trás da definição de qualquer média: uma medida que de alguma forma representa o conjunto de dados, segundo uma formulação, e se situa entre os extremos das observações. É claro que, em geral, haverá valores diferentes no conjunto e, neste caso, a média será um valor pertencente ao intervalo de variação dos valores neste conjunto e não necessariamente, um valor que tenha sido observado.

No exemplo das idades das seis primeiras pessoas que chegaram para tomar a vacina no Posto de Saúde a média é 31 anos, porém não se observou uma idade igual a 31 anos.

Você já calculou a média dos dados das duas primeiras atividades, a saber, \DUrole{xref,std,std-ref}{ativ-Notas-de-Artes} e \DUrole{xref,std,std-ref}{ativ-maratona-de-NY}. Identifique nos histogramas correspondentes a posição em que estas médias ficaram.

\sphinxstylestrong{Média para dados agrupados}

Quando os dados disponíveis estão agrupados em intervalos de classe,  não é possível calcular a soma total exata dos dados. Neste caso, usamos uma aproximação para o cálculo da média como mostra o exemplo a seguir.

Suponha que um coordenador tenha tido acesso apenas ao {\hyperref[\detokenize{PE104-0:fig-histograma-notas-sem-bonificacao}]{\sphinxcrossref{\DUrole{std,std-ref}{Histograma das notas de Artes sem bonificação}}}}, sem conhecer as notas separadamente.  Como este coordenador poderia calcular a média da turma, considerando as notas antes da bonificação?

Temos a seguinte distribuição de frequências das notas antes da bonificação:


\begin{savenotes}\sphinxattablestart
\centering
\sphinxcapstartof{table}
\sphinxcaption{Distribuição de frequências das notas antes de bonificação}\label{\detokenize{PE104-1:id7}}
\sphinxaftercaption
\begin{tabulary}{\linewidth}[t]{|T|T|T|}
\hline

intervalo
&
frequência absoluta
&
ponto médio do intervalo
\\
\hline
{[}0,2{[}
&
1
&
1,0
\\
\hline
{[}2,4{[}
&
5
&
3,0
\\
\hline
{[}4,6{[}
&
6
&
5,0
\\
\hline
{[}6,8{]}
&
23
&
7,0
\\
\hline
\end{tabulary}
\par
\sphinxattableend\end{savenotes}

Apenas sabemos que, por exemplo, entre 2 e 4 existem cinco notas, mas  não conhecemos o valor exato de cada uma destas cinco notas. Portanto, a soma exata destas cinco notas não é conhecida. A estratégia é tomar o ponto médio desta classe \(\left (\frac{2+4}{2}\right )=3\) como a nota representativa das cinco observações, pois espera-se que os erros cometidos para mais e para menos sejam compensados na classe. Desse modo estimamos a soma das notas neste intervalo como \(3+3+3+3+3=5\cdot 3=15\).

Esse procedimento é adotado para todas as classes a fim de obter uma estimativa da soma total dos dados, a saber,
\begin{equation*}
\begin{split}1\cdot 1+5\cdot 3+6\cdot 5+23\cdot 7=207\end{split}
\end{equation*}
Logo, a média correspondente a este agrupamento, a ser considerada pelo coordenador é estimada por
\begin{quote}

\(\textsf{média}=\bar{x}=\frac{1\times 1+5\times 3+6\times 5+23\times 7}{35}=\frac{207}{35}\approx 5,91\)
\end{quote}

Observe que este agrupamento resultou numa soma 207, muito próxima da soma exata dada por 207,5. Por esta razão dizemos que o agrupamento não incorreu em grande perda de informação para efeito de calcular a soma dos dados: em vez de usar as 35 notas, foi possível com cinco intervalos de classe avaliar de forma precisa a soma original dos dados. Consequentemente, a média estimada por este agrupamento (5,91) não se diferencia muito da média considerando os dados brutos (5,93).

Na seção {\hyperref[\detokenize{PE104-A:sec-para-saber-mais}]{\sphinxcrossref{\DUrole{std,std-ref}{Para saber mais}}}} apresenta-se notação e fórmula para o cálculo da média numa situação genérica de dados agrupados.

\sphinxstylestrong{Interpretação da média como ponto de equilíbrio no histograma}

Observe o {\hyperref[\detokenize{PE104-0:fig-histograma-notas-sem-bonificacao}]{\sphinxcrossref{\DUrole{std,std-ref}{Histograma das notas de Artes sem bonificação}}}} , em que as notas dispostas ao longo do eixo horizontal. Suponha que o histograma seja mais do que uma representação da distribuição de frequências, que seja um objeto. Assim, cada ponto que compõe as notas teria massa e poderia ser associado a um peso.  Por exemplo, a nota 1 corresponderia a 1kg, a nota 5 a 5 kg e a nota 6,3 a 6,3 Kg.  esse caso, podemos perguntar onde se encontrará o ponto de equilíbrio (ou centro de massa) do histograma que representa a distribuição de frequências dos dados. É natural pensar na média como o ponto de equilíbrio, como mostra o histograma a seguir, com destaque para a média. Veja adiante a seção sobre desvios da média para reforçar esta noção de ponto de equilíbrio.
\phantomsection\label{\detokenize{PE104-1:id1}}\begin{quote}

\begin{figure}[H]
\centering
\capstart

\noindent\sphinxincludegraphics[width=200bp]{{histogramaNotas_E1_PE_1}.png}
\caption{Histograma com destaque para a média como ponto de equilíbrio}\label{\detokenize{PE104-1:id8}}\end{figure}
\end{quote}

Se fossemos tentar equilibrar o histograma num ponto acima da média, considerando esta interpretação, o mesmo penderia para à esquerda, conforme ilustra a figura a seguir.
\begin{quote}

\begin{figure}[H]
\centering
\capstart

\noindent\sphinxincludegraphics[width=200bp]{{histogramaNotas_esquerda_2}.png}
\caption{Histograma inclinado para à esquerda}\label{\detokenize{PE104-1:id2}}\label{\detokenize{PE104-1:id9}}\end{figure}
\end{quote}

Se fossemos tentar equilibrar o histograma num ponto abaixo da média, considerando esta interpretação, o mesmo penderia para à direita, conforme ilustra a figura a seguir.
\begin{quote}

\begin{figure}[H]
\centering
\capstart

\noindent\sphinxincludegraphics[width=200bp]{{histogramaNotas_direita_1}.png}
\caption{Histograma inclinado para à direita}\label{\detokenize{PE104-1:id3}}\label{\detokenize{PE104-1:id10}}\end{figure}
\end{quote}

Cuidado com esta interpretação: o ponto de equilíbrio corresponde à posição para a qual a soma dos valores, interpretada como peso, é a mesma à esquerda e à direita dela. Esta posição, correspondendo à posição da média, não é necessariamente a posição na qual a área total do histograma é dividida em duas metades (mediana). É claro que, se a forma do histograma for simétrica, estas duas posições serão coincidentes. Veja a figura a seguir, ilustrando uma situação de simetria na qual temos que a média é igual à mediana.

\begin{figure}[H]
\centering
\capstart

\noindent\sphinxincludegraphics[width=300bp]{{registros_de_tempo_simetria_1}.png}
\caption{Histograma dos resgistros de tempo de atividade do Capítulo \sphinxstylestrong{A Natureza da Estatística}}\label{\detokenize{PE104-1:fig-simetria}}\label{\detokenize{PE104-1:id11}}\end{figure}


\begin{example}{O cartão de crédito de supermercado}
Numa tarde, 10 clientes interessados em obter um cartão de crédito oferecido por uma rede de supermercados informaram a uma atendente seus salários (em salários mínimos): \(\{1, 1, 2, 3, 4, 5, 5, 6, 9, 10\}\).

A média destes dados é, então, \(\bar{x}=\frac{46}{10}=4,6\), que representa bem este conjunto, pois nele existem cinco valores acima da média e cinco valores abaixo da média e, estes valores não estão muito afastados do valor da média, conforme ilustrado no Diagrama de Pontos a seguir.

\begin{figure}[H]
\centering
\capstart

\noindent\sphinxincludegraphics[width=200bp]{{ilustrasomedia}.png}
\caption{Diagrama de pontos do conjunto \(\{1, 1, 2, 3, 4, 5, 5, 6, 9, 10\}\) com destaque para a média do conjunto}\label{\detokenize{PE104-1:fig-diagramadepontos-media-sem-outlier}}\label{\detokenize{PE104-1:id12}}\end{figure}

Suponha uma pequena variação do conjunto de dez salários na qual no lugar do salário de 10 salários mínimos, o salário é de 100 salários mínimos. Assim, os registros são \(\{1, 1, 2, 3, 4, 5, 5, 6, 9, 100\}\).  Observe que a única diferença entre os dois conjuntos está no valor extremo: um é 10 e o outro é 100. O que esta única diferença nos dois conjuntos acarreta na média?

Com os dados do segundo conjunto, a média é dada por \(\frac{136}{10}=13,6\), valor maior do que a maioria dos dados observados no conjunto, a saber, apenas uma observação é bem superior a 13,6. Observe, que para representar o diagrama de pontos destes dados usou-se um recurso de quebra do eixo dos dados devido ao valor atípico 100, em relação aos demais valores.

\begin{figure}[H]
\centering
\capstart

\noindent\sphinxincludegraphics[width=400bp]{{ilustrasomediacomoutlier}.png}
\caption{Diagrama de pontos do conjunto \(\{1, 1, 2, 3, 4, 5, 5, 6, 9, 100\}\) com destaque para a média do conjunto e quebra do eixo devido ao valor atípico}\label{\detokenize{PE104-1:fig-diagramadepontos-media-com-outlier}}\label{\detokenize{PE104-1:id13}}\end{figure}

Este exemplo simples mostra que na presença de dados atipicamente altos, deve-se tomar cuidado em escolher a média como medida de posição das observações coletadas. Uma medida pouco afetada para valores atípicos, conhecida como \index{medida robusta}medida robusta,  deverá ser considerada em situações deste tipo. A mediana, que trataremos a seguir, é considerada uma medida robusta.

\end{example}

Desta discussão podemos concluir que deve-se ter cautela em resumir os dados com a média quando sua distribuição, representada pelo histograma, apresenta forma muito assimétrica, como mostram as figuras a seguir.

\begin{figure}[H]
\centering
\capstart

\noindent\sphinxincludegraphics[width=200bp]{{triciclodemao_histogramacdesigual}.png}
\caption{Histograma da distribuição dos tempos de chegada na categoria triciclo de mão revelando assimetria à direita (mediana\textless{}média)}\label{\detokenize{PE104-1:fig-assimetriaadireita}}\label{\detokenize{PE104-1:id14}}\end{figure}

\begin{figure}[H]
\centering
\capstart

\noindent\sphinxincludegraphics[width=200bp]{{histogramacomassimetriaesquerda}.png}
\caption{Histograma de distribuição com assimetria à esquerda}\label{\detokenize{PE104-1:fig-assimetriaaesquerda}}\label{\detokenize{PE104-1:id15}}\end{figure}

Alguns textos usam os termos assimetria positiva para indicar assimetria à direita e assimetria negativa para indicar assimetria à esquerda.

\sphinxstylestrong{Mediana}

A \index{mediana}mediana de um conjundo de valores numéricos é definida como o valor que ocupa a posição central dos dados ordenados.

Se o conjunto de dados tem uma quantidade ímpar de elementos então, considerando os dados ordenados, a mediana ocupará a posição central. Por exemplo, se o conjunto de dados tiver \(n=9\) elementos,  a posição central será a quinta. Nesse caso, haverá, ordenadamente, quatro elementos anteriores e quatro posteriores à mediana.


\begin{example}{Idades de crianças atendidas em Posto de Saúde}
Considere o seguinte conjunto de idades de crianças atendidas (na ordem de atendimento) em um ambulatório pediátrico de um Posto de Saúde na primeira segunda-feira do mês de março no turno da manhã \(\{4,6,9,3,2,3,7,8,7\}\). Temos ao todo 9 observações cujos valores ordenados são
\begin{equation*}
\begin{split}2 \leq 3 \leq 3 \leq 4 \leq \underbrace{\overbrace{6}^{\textsf{valor da quinta posição}}}_{\textsf{mediana}} \leq 7 \leq 7 \leq 8 \leq 9\end{split}
\end{equation*}
\end{example}
Se o conjunto de dados tem uma quantidade par de elementos não será possível identificar “um” elemento central. Nesse caso, para a determinação da mediana serão considerados os dois elementos centrais da sequência ordenada. A mediana é dada pela média aritmética desses elementos. Por exemplo, se o conjunto de dados tiver 10 elementos, então as posições centrais são a 5a. e a 6a. A mediana será a média dos elementos que ocupam essas posições na sequência ordenada.


\begin{example}{cartão de crédito de supermercado (2)}
Considere o conjunto de salários de 10 clientes interessados em obter um cartão de crédito oferecido por uma rede de supermercados e que informaram à atendente seus salários (em salários mínimos):
\begin{equation*}
\begin{split}\{1, 1, 2, 3, \overbrace{4}^{\textsf{5a. posição}}, \underbrace{5}_{\textsf{6a. posição}}, 5, 6, 9, 100\}\end{split}
\end{equation*}
Observe que os valores já estão ordenados e que o salário da 5a. posição é 4 e, o da 6a., é 5. Logo, a mediana dos salários será dada por
\begin{equation*}
\begin{split}\frac{4+5}{2}=4,5\end{split}
\end{equation*}
Lembre que a média destes dados resultou em 13,6. Este exemplo ilustra a propriedade de que a mediana é pouco afetada na presença de valores atipicamente grandes (ou pequenos). Já a média não possui esta propriedade, sendo muito afetada na presença de valores atípicos.
\end{example}

De maneira geral, se \(x_{(1)},x_{(2)},...,x_{(n)}\) são os valores ordenados do conjunto de dados, a mediana será dada por

\(\textsf{Mediana}=\left \{ \begin{array}{lr}
x_{\left (\frac{n+1}{2}\right )}, &\textsf{ se }n \textsf{ for ímpar}\\
\frac{1}{2} [ x_{\left (\frac{n}{2}\right )}+x_{\left (\frac{n}{2}+1\right )} ], &\textsf{ se }n \textsf{ for par.}\end{array}\right.\)

\begin{example}{Determinação da mediana dos conjuntos de dados Notas de Artes e tempos de chegada para as mulheres na maratona de Nova Iorque (2017)}

Considere a \DUrole{xref,std,std-ref}{ativ-notas-de-Artes} na qual tem-se \(n=35\) notas. Como 35 é ímpar, usando a definição anterior, podemos concluir que a mediana das notas será a nota na 18a. posição \(\left (\frac{35+1}{2}=18\right )\), a saber, \(\textsf{mediana}=x_{(18)}=6,5\) .

Considere a \DUrole{xref,std,std-ref}{ativ-maratona-de-NY} na qual tem-se \(n=100\) melhores tempo de chegada entre as mulheres. Como 100 é par, usando a definição anterior, podemos concluir que a mediana dos 100 melhores tempos será dada pela média dos tempos na 50a e na 51a. chegada, a saber,
\begin{equation*}
\begin{split}\textsf{mediana}=\frac{x_{(50)}+x_{(51)}}{2}=\frac{2,949+2,949}{2}=2,949 \textsf{ horas}\end{split}
\end{equation*}
\end{example}

\sphinxstylestrong{Mediana  para dados agrupados}

Voltando à \DUrole{xref,std,std-ref}{ativ-Notas-de-Artes}, suponha novamente que o coordenador tenha tido acesso apenas ao
{\hyperref[\detokenize{PE104-0:fig-histograma-notas-sem-bonificacao}]{\sphinxcrossref{\DUrole{std,std-ref}{Histograma das notas de Artes sem bonificação}}}}, sem conhecê-las separadamente.  Como ele poderia calcular a mediana da turma, considerando as notas antes da bonificação? Sabemos que a posição da mediana deve ser a posição central depois de ter as notas ordenadas. Na tabela de frequências observe que os intervalos já estão ordenados, mas apenas conhecemos a quantidade de notas que ocorreram em cada intervalo e não as notas individualmente. No entanto, é fácil, a partir da tabela, identificar em que intervalo estará a mediana, bastando para isso encontrar o intervalo que compreende a nota da posição 18. Aqui, vamos introduzir o conceito de \index{frequência absoluta acumulada}frequência absoluta acumulada de um intervalo de classe que corresponde à soma da frequência absoluta do intervalo mais a soma acumulada das frequências absolutas  de todos os intervalos anteriores. Veja a tabela a seguir, incluindo as frequências acumuladas.


\begin{savenotes}\sphinxattablestart
\centering
\sphinxcapstartof{table}
\sphinxcaption{Notas de artes agrupadas e frequência absoluta acumulada}\label{\detokenize{PE104-1:id16}}
\sphinxaftercaption
\begin{tabulary}{\linewidth}[t]{|c|c|c|c|}
\hline

intervalo
&
frequência absoluta
&
ponto médio do intervalo
&
freq. absoluta acumulada
\\
\hline
{[}0,2{[}
&
1
&
1,0
&
1
\\
\hline
{[}2,4{[}
&
5
&
3,0
&
$1+5=6$
\\
\hline
{[}4,6{[}
&
6
&
5,0
&
$6+6=12$
\\
\hline
{[}6,8{[}
&
23
&
7,0
&
$12+23=35$
\\
\hline
\end{tabulary}
\par
\sphinxattableend\end{savenotes}

Observe que a nota da posição 18 está no último intervalo, pois até o intervalo anterior, {]}4,6{]}, acumularam-se apenas 12 das 35 notas.

Uma forma de estimar a mediana no caso em que não conhecemos as notas separadamente é tomar o ponto médio do intervalo de classe que compreende o valor da posição central. Neste caso, teríamos que a nota mediana seria 7,0, o ponto médio do intervalo de classe que contém a mediana ({]}6,8{]}). Comparando este valor com o valor da mediana obtido, usando-se as 35 notas individuais, percebe-se que o erro de aproximação é de apenas 0,5 ponto já que sabemos que a nota da posição 18 é 6,5.

Resumindo, quando dispomos dos dados apenas na forma agrupada, para obter uma aproximação da mediana, deve-se identificar o intervalo de classe que compreende o valor da posição central e, então, calcular o ponto médio desta classe como valor aproximado da mediana.

Existem outras formas de avaliar a mediana quando os dados estão agrupados e uma delas foi proposta no exercício 17 do capítulo \sphinxstylestrong{A Natureza da Estatística}.

\sphinxstylestrong{Escolha entre a média e a mediana como valor mais adequado para resumir a informação do conjunto de dados}

Vimos que a média é uma medida muito afetada na presença de valores atípicos (muito afastados da maioria do dados) e de distribuições fortemente assimétricas (caraceterizadas por histogramas alongados para à direita ou para à esquerda). A mediana, por sua vez, é pouco afetada para valores atípicos na distribuição, e por isso é dita ser uma \index{medida robusta}medida robusta.

Por exemplo, vamos voltar ao exemplo sobre as informações de salário entre os interessados para obter um cartão de crédito de uma rede de supermercados. Lembre-se que trabalhamos com dois conjuntos de dados, a saber, \(C_1=\{1, 1, 2, 3, 4, 5, 5, 6, 9, 10\}\) e \(C_2=\{1, 1, 2, 3, 4, 5, 5, 6, 9, 100\}\) .

A média dos dados do conjunto \(C_1\) é \(\bar{x}=\frac{46}{10}=4,6\) e, a \(\textsf{mediana}=\frac{x_{(5)}+x_{(6)}}{2}=\frac{4+5}{2}=4,5\) .

Tanto a média, como a mediana do conjunto \(C_1\) são valores que o representam bem: observe que os demais valores no conjunto \(C_1\) não estão muito afastados dos valores da média e da mediana e, de forma equilibrada, alguns estão abaixo deles e outros, acima deles.

Por outro lado, a média dos dados do conjunto \(C_2\) é \(\frac{136}{10}=13,6\), enquanto que a \(\textsf{mediana}\) é dada por  \(\frac{x_{(5)}+x_{(6)}}{2}=\frac{4+5}{2}=4,5\).  Este último exemplo ilustra como a média é fortemente influenciada pela presença do valor atípico 100, enquanto a mediana não.   Na presença do valor atípico (100), a média é muito afetada, mudando de 4,6 para 13,6, enquanto que a mediana não foi afetada, mantendo-se igual a 4,5.  Observe que apenas um valor no conjunto \(C_2\) está acima da média.

Em distribuições aproximadamente simétricas (veja a {\hyperref[\detokenize{PE104-1:fig-simetria}]{\sphinxcrossref{\DUrole{std,std-ref}{Histograma dos resgistros de tempo de atividade do Capítulo A Natureza da Estatística}}}} ) temos que a média e a mediana são valores próximos um do outro, esta é uma das razões que levam muitas pessoas a confundir estas duas medidas, achando que elas representam a mesma posição na distribuição dos dados qualquer que seja a situação. Mas, vimos que em distribuições com assimetria à direita, veja, por exemplo a figura  {\hyperref[\detokenize{PE104-1:fig-assimetriaadireita}]{\sphinxcrossref{\DUrole{std,std-ref}{Histograma da distribuição dos tempos de chegada na categoria triciclo de mão revelando assimetria à direita (mediana\textless{}média)}}}}, a média é maior do que a mediana e, em distribuições com assimetria à esquerda, veja por exemplo a figura {\hyperref[\detokenize{PE104-1:fig-assimetriaaesquerda}]{\sphinxcrossref{\DUrole{std,std-ref}{Histograma de distribuição com assimetria à esquerda}}}}, a média é menor do que a mediana.

\sphinxstylestrong{Moda}

A \index{moda}moda é a observação mais frequente de um conjunto de dados.

Caso não haja observação mais frequente, ou seja, todos os valores aparecem apenas uma única vez no conjunto de dados, a distribuição é dita amodal. Um conjunto é dito unimodal se houver apenas uma moda; bimodal se houver duas modas; ou multimodal se houver três ou mais modas no conjunto de dados coletados.

Vejamos exemplos das diversas situações possíveis. Considere os conjuntos de notas da prova de Matemática dos alunos de quatro turmas diferentes dadas pela tabela a seguir.


\begin{savenotes}\sphinxattablestart
\centering
\sphinxcapstartof{table}
\sphinxcaption{Exemplos de diversas possibilidades quanto à moda}\label{\detokenize{PE104-1:id17}}
\sphinxaftercaption
\begin{tabulary}{\linewidth}[t]{|c|c|c|c|}
\hline

Turma
&
Notas
&
Moda
&
Distribuição
\\
\hline
I
&
2; 4; 6; 7; 8; 9; 10
&
Não existe
&
Amodal
\\
\hline
II
&
2; 4; 5 ;5; 8; 9; 10
&
5
&
Unimodal
\\
\hline
III
&
2; 4; 5; 5; 8; 9; 9; 10
&
5 e 9
&
Bimodal
\\
\hline
IV
&
2; 2; 4; 5; 5; 8; 9; 9; 10
&
2; 5 e 9
&
Multimodal
\\
\hline
\end{tabulary}
\par
\sphinxattableend\end{savenotes}

O conceito de moda é adequado para conjuntos de dados qualitativos ou quantitativos discretos, pois quando os dados são quantitativos contínuos, potencialmente todas as observações são distintas entre si tal que raramente existirá um valor mais frequente e, mesmo quando um valor se repetir, não necessariamente é por que ele corresponderá a uma moda. Neste último caso, o que fazemos é, agrupar os dados em intervalos de classe para identificar um intervalo de classe modal ou intervalos de classe modais, isto é, o(s) intervalo(s) de classe com maior frequência. Uma vez identificado(s) o(s) intervalo(s) de classe modal(ais), uma estimativa para a(s) moda(s) é dada pelo ponto médio do intervalo de classe modal correspondente.

A pergunta que surge naturalmente agora é: Quando a moda será preferível à média ou à mediana?

Se o histograma da distribuição é aproximadamente simétrico, e há uma única moda, então as três medidas-resumo (média, mediana e moda) serão valores aproximadamente iguais. Nesse caso, em geral, preferiremos usar a média como medida de posição, pois ela possui propriedades relevantes para a inferência estatística.

\begin{figure}[H]
\centering
\capstart

\noindent\sphinxincludegraphics[width=300bp]{{registros_de_tempo_simetria_2}.png}
\caption{Histograma simétrico: distribuição unimodal (Dados: Registros de tempo de atividade do capítulo \sphinxstylestrong{A Natureza da Estatística})}\label{\detokenize{PE104-1:id4}}\label{\detokenize{PE104-1:id18}}\end{figure}

Se, no entanto, a distribuição apresenta forte assimetria com a presença valores atípicos e unimodal, então preferiremos, em geral, tomar a mediana como medida resumo.

\begin{figure}[H]
\centering
\capstart

\noindent\sphinxincludegraphics[width=200bp]{{triciclodemao_histogramacdesigual}.png}
\caption{Histograma de distribuição com assimetria à direita (Tempos de chegada para a categoria Triciclo de mão na maratona de Nova Iorque/2017).}\label{\detokenize{PE104-1:fig-assimetriadireita}}\label{\detokenize{PE104-1:id19}}\end{figure}

Se, por outro lado, o histograma da distribuição é do tipo simétrico e bimodal como na representação esquemática a seguir, então nem a média, nem a mediana serão indicadas como medidas de representação dos dados, pois observe na figura, que elas estarão situadas bem no centro onde há pouca incidência de valores. Assim, neste caso, as duas modas serão mais úteis para descrever de forma resumida este conjunto de dados.

\begin{figure}[H]
\centering
\capstart

\noindent\sphinxincludegraphics[width=200bp]{{histsimbimod}.png}
\caption{Histograma de distribuição simétrica e bimodal}\label{\detokenize{PE104-1:id5}}\label{\detokenize{PE104-1:id20}}\end{figure}

\sphinxstylestrong{Quartis}

Os \index{quartis}quartis são os três valores que dividem a distribuição em quatro partes de frequências iguais.

O primeiro quartil (\(\textsf{Q}_1\)) é o valor da distribuição para o qual a frequência relativa de valores abaixo dele é igual 25\% do número de observações do conjunto de dados e, consequentemente, acima dele, é 75\% do número de observações do conjunto de dados.

O segundo quartil (\(\textsf{Q}_2\)) é a mediana da distribuição ou, equivalentemente, o  valor da distribuição para o qual que a frequência relativa de valores abaixo dele é 50\% do número de observações do conjunto de dados e, consequentemente, acima dele, é 50\% do número de observações do conjunto de dados.

Finalmente o terceiro quartil (\(\textsf{Q}_3\)) é o valor da distribuição
para o qual a frequência relativa de valores abaixo dele é igual 75\% do número de observações do conjunto de dados e, consequentemente, acima dele, é 25\% do número de observações do conjunto de dados.

\begin{example}{quartis da distribuição dos 100 melhores tempos para mulheres na maratona de Nova Iorque (2017)}
Você já determinou os quartis para os dados da \DUrole{xref,std,std-ref}{ativ-maratona-de-NY} referentes aos 100 melhores tempos da maratona para a categoria mulheres.

Como \(n=100\), podemos tomar como o primeiro quartil o tempo da 25a. posição \(\left (\frac{100}{4}=25\right )\), a saber, \(\textsf{Q}1=2,764\) h, já vimos que a mediana é 2,949 h e, para o terceiro quartil podemos tomar o  o valor da 75a. posição \(\left (3\cdot\frac{100}{4}=75\right )\), a saber, \(\textsf{Q}3=2,998\) h.
\end{example}

Já vimos como determinar mediana (ou segundo quartil) de um conjunto de \(n\) dados. Um método simples para obter os demais quartis, Q1 e Q3, é considerar dois novos conjuntos de dados, o primeiro, consistindo da primeira metade dos valores ordenados e, o segundo, consistindo da segunda metade. Depois, basta determinar a mediana de cada um destes dois conjuntos, obtendo Q1 e Q3, respectivamente.


\practice{ }
\label{\detokenize{PE104-2:sec-praticando1}}\label{\detokenize{PE104-2::doc}}\label{\detokenize{PE104-2:praticando}}\phantomsection\label{\detokenize{PE104-2:ativ-maratona-categoria-homens}}
\begin{task}{ categoria homens na maratona}

Considere os dados da categoria Homens da Maratona da Cidade de Nova Iorque do ano 2017 apresentados na tabela a seguir, já convertidos para horas.


\begin{savenotes}\sphinxattablestart
\centering
\sphinxcapstartof{table}
\sphinxcaption{100 melhores tempos de finalização da Maratona de Nova Iorque 2017 para homens}\label{\detokenize{PE104-2:id2}}
\sphinxaftercaption
\begin{tabulary}{\linewidth}[t]{|T|T|T|T|T|T|T|T|T|T|T|}
\hline
\sphinxstylethead{\sphinxstyletheadfamily \unskip}\relax &\sphinxstylethead{\sphinxstyletheadfamily 
+0
\unskip}\relax &\sphinxstylethead{\sphinxstyletheadfamily 
+10
\unskip}\relax &\sphinxstylethead{\sphinxstyletheadfamily 
+20
\unskip}\relax &\sphinxstylethead{\sphinxstyletheadfamily 
+30
\unskip}\relax &\sphinxstylethead{\sphinxstyletheadfamily 
+40
\unskip}\relax &\sphinxstylethead{\sphinxstyletheadfamily 
+50
\unskip}\relax &\sphinxstylethead{\sphinxstyletheadfamily 
+60
\unskip}\relax &\sphinxstylethead{\sphinxstyletheadfamily 
+70
\unskip}\relax &\sphinxstylethead{\sphinxstyletheadfamily 
+80
\unskip}\relax &\sphinxstylethead{\sphinxstyletheadfamily 
+90
\unskip}\relax \\
\hline
1
&
2,181
&
2,258
&
2,457
&
2,500
&
2,526
&
2,551
&
2,573
&
2,602
&
2,616
&
2,631
\\
\hline
2
&
2,182
&
2,311
&
2,461
&
2,501
&
2,528
&
2,552
&
2,575
&
2,606
&
2,621
&
2,631
\\
\hline
3
&
2,192
&
2,341
&
2,469
&
2,502
&
2,53
&
2,554
&
2,577
&
2,608
&
2,621
&
2,631
\\
\hline
4
&
2,198
&
2,358
&
2,471
&
2,507
&
2,531
&
2,555
&
2,578
&
2,610
&
2,622
&
2,634
\\
\hline
5
&
2,200
&
2,377
&
2,472
&
2,508
&
2,531
&
2,557
&
2,588
&
2,610
&
2,623
&
2,635
\\
\hline
6
&
2,211
&
2,379
&
2,474
&
2,514
&
2,533
&
2,562
&
2,588
&
2,612
&
2,625
&
2,635
\\
\hline
7
&
2,213
&
2,394
&
2,478
&
2,518
&
2,542
&
2,563
&
2,591
&
2,613
&
2,626
&
2,636
\\
\hline
8
&
2,223
&
2,398
&
2,487
&
2,520
&
2,546
&
2,568
&
2,592
&
2,613
&
2,627
&
2,636
\\
\hline
9
&
2,233
&
2,426
&
2,495
&
2,523
&
2,548
&
2,571
&
2,595
&
2,613
&
2,628
&
2,639
\\
\hline
10
&
2,249
&
2,453
&
2,496
&
2,524
&
2,549
&
2,573
&
2,597
&
2,614
&
2,629
&
2,639
\\
\hline
\end{tabulary}
\par
\sphinxattableend\end{savenotes}

A figura a seguir mostra um histograma destes dados, considerando-se 10 intervalos de classe.

\begin{figure}[H]
\centering
\capstart

\noindent\sphinxincludegraphics[width=400bp]{{Histograma_homens_1}.png}
\caption{Histograma dos resultados da categoria de Homens da Maratona da Cidade de Nova Iorque do ano 2017}\label{\detokenize{PE104-2:fig-histograma-maratona-homens}}\label{\detokenize{PE104-2:id3}}\end{figure}
\begin{enumerate}
\item {} 
Calcule a média dos 100 melhores tempos na categoria homens, sabendo que a soma dos tempos é dada por 251,1617 horas.

\item {} 
Calcule a mediana dos 100 melhores tempos na categoria homens.

\item {} 
Identifique o intervalo de classe modal dos 100 melhores tempos na categoria homens.

\item {} 
Determine os quartis dos 100 melhores tempos na categoria homens.

\item {} 
Localize no histograma a média e os quartis.

\item {} 
Compare com os resultados obtidos para a categoria homens com os obtidos para a categoria mulheres na \sphinxcrossref{\DUrole{std,std-ref}{ativ-maratona-de-NY}} completando a tabela a seguir.

\end{enumerate}


\begin{savenotes}\sphinxattablestart
\centering
\sphinxcapstartof{table}
\sphinxcaption{Tabela de medidas-resumo para Mulheres e Homens - Maratona de Nova Iorque/2017}\label{\detokenize{PE104-2:id4}}
\sphinxaftercaption
\begin{tabulary}{\linewidth}[t]{|T|T|T|}
\hline
&
Mulheres
&
Homens
\\
\hline
Mínimo
&&\\
\hline
Máximo
&&\\
\hline
Média
&&\\
\hline
Mediana
&&\\
\hline
\(Q1\)
&&\\
\hline
\(Q3\)
&&\\
\hline
\end{tabulary}
\par
\sphinxattableend\end{savenotes}
\end{task}

\


\begin{reflection}

\begin{itemize}
\item {} 
O que seria necessário considerar para poder comparar o histograma da categoria de Homens com o das Mulheres? Observe que os limites dos intervalos são distintos, mas estão na mesma escala.

\item {} 
Como poderiam ser utilizadas a mediana e os quartis para comparar duas distribuições de dados? Pense em alguma forma de comparar esse dados de forma visual e descreva-a.

\end{itemize}
\end{reflection}

\phantomsection\label{\detokenize{PE104-2:ativ-comparacao-de-diferentes-grupos}}

\begin{task}{cadeiras de rodas e triciclos de mão}

Observe os histogramas a seguir referentes as categorias de cadeira de rodas e triciclo de mão da Maratona de Nova Iorque em 2017.

\begin{figure}[H]
\centering
\capstart

\noindent\sphinxincludegraphics[width=300bp]{{Histogramas_cadeira_triciclo}.png}
\caption{Histogramas comparativos das quatro modalidades da maratona de Nova Iorque 2017}\label{\detokenize{PE104-2:id1}}\label{\detokenize{PE104-2:id7}}\end{figure}
\begin{enumerate}
\item {} 
Compare as escalas utilizadas na construção destes histogramas, tanto no eixo horizontal, como no eixo vertical. O que você observou?

\item {} 
Em qual categoria se encontra o atleta que completou a maratona no maior tempo?

\item {} 
Você consegue estimar o tempo médio destas categorias observando os histogramas? Você acha que elas serão muito diferentes das de homens e mulheres (\DUrole{xref,std,std-ref}{ativ-maratona-categoria-homens})?

\item {} 
Observe o quadro a seguir e marque as médias nos histogramas. Comente sobre a posição da média em cada caso e sobre a simetria ou assimetria de cada distribuição de dados.

\begin{savenotes}\sphinxattablestart
\centering
\sphinxcapstartof{table}
\sphinxcaption{Média das quatro categorias da maratona de Nova Iorque 2017}\label{\detokenize{PE104-2:id8}}
\sphinxaftercaption
\begin{tabulary}{\linewidth}[t]{|c|c|c|}
\hline

Categoria
&
Cadeira de rodas
&
Triciclo de mão
\\
\hline
Média
&
2,59
&
2,73
\\
\hline
\end{tabulary}
\par
\sphinxattableend\end{savenotes}

\item {} 
Observe que as médias não são muito diferentes, porém, as distribuições são similares. Se você conhecesse apenas a média, seria capaz de perceber a forma destes histogramas? Por quê?

\item {} 
Comparando os dois histogramas, qual distribuição apresenta maior dispersão? Por quê?

\end{enumerate}
\end{task}




\explore{  medidas de dispersão}
\label{\detokenize{PE104-3:explorando-medidas-de-dispersao}}\label{\detokenize{PE104-3::doc}}\label{\detokenize{PE104-3:sec-explorando2}}\phantomsection\label{\detokenize{PE104-3:ativ-estrategia-de-investimento}}

\bigskip\hrule\bigskip

\begin{quote}

Para investir na bolsa de valores compramos ações de empresas por intermédio de uma corretora a um certo preço e depois de um período de tempo vendemos estas ações na expectativa de que seus preços tenham aumentado. No entanto, também podemos perder com o investimento, caso o preço da ação diminua no período de investimento. Uma ação é a menor parte do capital de uma empresa. Veja na figura a seguir um esquema simplificado do investimento na bolsa de valores.

\begin{figure}[H]
\centering
\capstart

\noindent\sphinxincludegraphics[width=300bp]{{resized001}.png}
\caption{Esquema simplificado de investimento na bolsa de valores}\label{\detokenize{PE104-3:fig-ativ-bolsa-de-valores}}\label{\detokenize{PE104-3:id1}}\end{figure}

Suponha que você tenha a oportunidade de investir um capital, comprando ações de uma de duas  Companhias \(A\) ou \(B\) e para escolher uma das duas, disponha de duas amostras de preços do valor destas ações (em reais) registrados no fechamento da bolsa de valores em dez sextas-feiras consecutivas. Veja na figura e na tabela a seguir a cotação das ações ao longo das últimas 10 semanas.

\begin{figure}[H]
\centering
\capstart

\noindent\sphinxincludegraphics[width=350bp]{{Acoes_medidas_dispersao}.png}
\caption{Gráficos de linha da cotação das ações}\label{\detokenize{PE104-3:fig-coloque-aqui-o-nome}}\label{\detokenize{PE104-3:id2}}\end{figure}


\begin{savenotes}\sphinxattablestart
\centering
\begin{tabulary}{\linewidth}[t]{|T|T|T|}
\hline

Semana
&
\(A\)
&
\(B\)
\\
\hline
1
&
61
&
67
\\
\hline
2
&
56
&
48
\\
\hline
3
&
63
&
52
\\
\hline
4
&
57
&
82
\\
\hline
5
&
67
&
77
\\
\hline
6
&
63
&
33
\\
\hline
7
&
67
&
67
\\
\hline
8
&
58
&
42
\\
\hline
9
&
67
&
90
\\
\hline
10
&
56
&
57
\\
\hline
Total
&
615
&
615
\\
\hline
\end{tabulary}
\par
\sphinxattableend\end{savenotes}
\begin{enumerate}
\item {} 
Obtenha as médias das cotações das ações das companhias A e B nas semanas observadas e compare-as.

\item {} 
Obtenha as medianas das cotações das ações das companhias A e B nas semanas observadas e compare-as, lembrando que os dados da tabela estão apresentados na ordem temporal.

\item {} 
Obtenha as modas das cotações das ações das companhias A e B nas semanas observadas e compare-as.

\item {} 
Analisando apenas as medidas de posição obtidas em (a), (b) e (c), pode-se dizer que as duas companhias diferem uma da outra? Por quê?

\item {} 
Um investimento que apresenta grandes ganhos e perdas pode ser chamado de alto risco, já investimentos cujos valores flutuam pouco são considerados de baixo risco. Se você é um investidor da bolsa de valores avesso ao risco, isto é, você gostaria de escolher o investimento com menores flutuações, em qual das companhias você investiria o seu dinheiro? Por quê?

\end{enumerate}
\end{quote}




\arrange{  medidas de dispersão}
\label{\detokenize{PE104-4:sec-organizando2}}\label{\detokenize{PE104-4::doc}}\label{\detokenize{PE104-4:organizando-as-ideias-medidas-de-dispersao}}
Pela atividade anterior, você deve ter notado que usar apenas medidas de posição para caracterizar uma distribuição não é suficiente. Nos dois conjuntos analisados, vimos que ambos apresentaram média, mediana e moda iguais. No entanto, vimos que um deles apresenta maiores variações de valores do que o outro. A ideia por trás de variação é a noção de dispersão.

Enquanto as medidas de posição procuram resumir o conjunto de dados em alguns valores situados entre dados coletados, as medidas de dispersão buscam avaliar quão dispersos são os dados coletados. Isso é de fundamental importância, pois podemos ter dois conjuntos de dados com as mesmas medidas de posição, como na \DUrole{xref,std,std-ref}{ativ-estrategia-de-investimento}, mas com dispersões diferentes, fazendo com que os valores qualitativos dessas medidas de posição sejam também diferentes.

Há uma piada irônica que conta que o Estatístico é o profissional que diz que uma pessoa, ao se sentar numa cadeira com duas placas de metal, uma aquecida a \(100^o\) C e outra resfriada a \(-40^o\) C, estará em média confortável, pois temperatura média é de \(30^o\) C. Na verdade, um Estatístico jamais diria isso, pois ele não toma decisões apenas por uma medida de posição, mas leva em conta também a dispersão dos dados em torno de uma medida de posição. Uma cadeira com duas placas de metal, uma aquecida a \(35^o\) C e outra a \(25^o\) C, também tem temperatura média de \(30^o\) C, mas há menos dispersão da temperatura nessa cadeira que na outra. Assim, embora quantitativamente iguais, os dois valores de \(30^o\) C não são qualitativamente equivalentes. Há, portanto, que se avaliar a dispersão dos dados coletados, a fim de poder obter conclusões adequadas.

Nesta seção serão apresentadas medidas que buscam caracterizar a dispersão dos dados em um conjunto.

\sphinxstylestrong{Amplitude amostral e distância entre quartis}

Entre as medidas de dispersão mais simples, define-se a \index{amplitude amostral}amplitude amostral (R) como a diferença entre o maior valor e menor valor observados. Usando a notação apresentada anteriormente, dado um conjunto com \(n\) observações, temos
\begin{equation*}
\begin{split}\textsf{Amplitude amostral}=\textsf{R}= \underbrace{x_{(n)}}_{\textsf{maior valor do conjunto}}-\underbrace{x_{(1)}}_{\textsf{menor valor do conjunto}}\end{split}
\end{equation*}
Uma desvantagem desta medida é que ela considera apenas os dois extremos do conjunto. Ainda é possível que dois conjuntos, tendo mesmas média, moda e mediana, apresentem a mesma amplitude e, no entanto, eles tenham comportamentos diferentes. 

\begin{example}{Notas de Matemática}
Supondo os seguintes conjuntos de notas de Matemática de duas turmas de reforço, cada uma com 10 alunos.

\(\textsf{Notas da turma A}=\{ 1,1,1,5,5,5,5,9,9,9\}\) e \(\textsf{Notas da turma B}=\{1,3,3,5,5,5,5,7,7,9\}\)

Verifique que para esses dois conjuntos tem-se média, moda, mediana e amplitude amostral iguais. No entanto, comparando os diagramas de pontos correspondentes a cada um deles, ilustrados na figura a seguir, é possível perceber diferenças quanto à dispersão das notas em torno da média 5,0 nos dois conjuntos.

\begin{figure}[H]
\centering
\capstart

\noindent\sphinxincludegraphics[width=200bp]{{diagrama_notas_1}.png}
\caption{Diagramas de pontos das notas nas turmas A e B}\label{\detokenize{PE104-4:fig-diagrama-de-pontos-notas}}\label{\detokenize{PE104-4:id2}}\end{figure}


Neste caso, uma medida um pouco mais refinada, mas ainda sem considerar todos os valores no conjunto, é a \index{distância entre quartis}distância entre quartis (DQ), definida como a diferença entre o terceiro e primeiro quartis da distribuição. Usando a notação apresentada anteriormente,
\begin{equation*}
\begin{split}\textsf{DQ}=\textsf{Q}3-\textsf{Q}1\end{split}
\end{equation*}
\end{example}

No exemplo anterior, como cada conjunto tem 10 observações, podemos dividi-los em duas metades com cinco observações e tomar as medianas para identificar os primeiro e terceiros quartis.

\(\textsf{Notas da turma A}= \{ \overbrace{1,1,1,5,5}^{\textsf{primeira metade}},\underbrace{5,5,9,9,9}_{\textsf{segunda metade}}\}\)

Deste modo, temos para a turma \(A\), Q1=1 (mediana da primeira metade) e Q3=9 (mediana da segunda metade) tal que DQ=9-1=8 e, para a turma \(B\), usando o mesmo raciocínio, DQ=7-3=4, indicando que na turma \(B\), considerando a distância entre quartis, temos menor dispersão, comparada à turma \(A\), observação que pode ser verificada nos diagramas de pontos da figura {\hyperref[\detokenize{PE104-4:fig-diagrama-de-pontos-notas}]{\sphinxcrossref{\DUrole{std,std-ref}{Diagramas de pontos das notas nas turmas A e B}}}}.

De fato, a distância entre quartis (DQ) também apresenta a desvantagem de somente considerar o primeiro e terceiro quartis, não considerando todas as observações do conjunto. A seguir, serão definidas medidas de dispersão que levam em conta todas as observações realizadas.

\sphinxstylestrong{Desvios da Média}

Considerando o conjunto \(\{ x_1,x_2,\cdots, x_n\}\) com \(n\) observações, seja \(\bar{x}\) a média deste conjunto.  Define-se como um \index{desvio da média}desvio da média, a diferença entre uma observação e a média, a saber,
\begin{equation*}
\begin{split}d_i=x_i-\bar{x}, \quad i=1,2,\cdots n\end{split}
\end{equation*}
Na \DUrole{xref,std,std-ref}{ativ-estrategia-de-investimento} os desvios da média, para cada uma das Companhias estão registrados na tabela a seguir.


\begin{savenotes}\sphinxattablestart
\centering
\begin{tabulary}{\linewidth}[t]{|T|T|T|}
\hline

Semana
&
Cia A
&
Cia B
\\
\hline
1
&
-0,5
&
5,5
\\
\hline
2
&
-5,5
&
-13,5
\\
\hline
3
&
1,5
&
-9,5
\\
\hline
4
&
-4,5
&
20,5
\\
\hline
5
&
5,5
&
15,5
\\
\hline
6
&
1,5
&
-28,5
\\
\hline
7
&
5,5
&
5,5
\\
\hline
8
&
-3,5
&
-19,5
\\
\hline
9
&
5,5
&
28,5
\\
\hline
10
&
-5,5
&
-4,5
\\
\hline
soma
&
0
&
0
\\
\hline
\end{tabulary}
\par
\sphinxattableend\end{savenotes}

Poderíamos pensar em usar os desvios da média para definir uma medida de dispersão dos dados em relação à média do conjunto, no entanto, a não ser que todos os valores sejam iguais, teremos valores acima da média e valores abaixo da média de tal modo que os desvios da média poderão apresentar sinais positivos ou negativos. Vimos que a média pode ser interpretada como o centro de massa (ponto de equilíbrio) dos dados e, esta propriedade pode ser descrita da seguinte forma: a soma dos desvios da média de qualquer conjunto de dados é sempre nula.

Com os dados da \DUrole{xref,std,std-ref}{ativ-Estrategia-de-Investimento} você pôde comprovar esta propriedade. Veja na figura a seguir a ilustração dos desvios da média das duas companhias na qual a linha pontilhada representa a cotação média da companhia e os segmentos em vermelho indicam o tamanho do desvio da média.

\begin{figure}[H]
\centering
\capstart

\noindent\sphinxincludegraphics[width=450bp]{{desviosdamedialadoalado}.png}
\caption{Desvios da média das cotações nas companhias A e B}\label{\detokenize{PE104-4:fig-desvios-da-media}}\label{\detokenize{PE104-4:id3}}\end{figure}

O gráfico {\hyperref[\detokenize{PE104-4:fig-desvios-da-media}]{\sphinxcrossref{\DUrole{std,std-ref}{Desvios da média das cotações nas companhias A e B}}}} reforça a conclusão anterior, da \DUrole{xref,std,std-ref}{ativ-estrategia-de-investimento}, de que as cotações da companhia A variam bem menos em torno da média do que as cotações da companhia B.

Em símbolos, a propriedade de que a soma dos desvios da média é sempre nula, pode ser traduzida em

\(\displaystyle{\sum^n_{i=1}} d_i=\displaystyle{\sum^n_{i=1}} (x_i-\bar{x})=0\), qualquer que seja o conjunto \(\{ x_1,x_2,\cdots, x_n\}\)

Portanto, não será possível usar a soma dos desvios da média como medida de dispersão de um conjunto de dados, pois ela sempre resultará em zero. Isso se deve ao fato de que a soma em valor absoluto dos desvios de sinal negativo é sempre igual a soma dos desvios de sinal positivo, uma consequência da propriedade da média como centro de massa.

Na Companhia A a soma dos desvios negativos é -19,5 e, dos desvios positivos, 19,5. Na Companhia B a soma dos desvios negativos é -75,5 e, dos desvios positivos, 75,5.

Uma forma de  contornar esta situação, de modo a usar os desvios da média para definir uma medida de dispersão, é eliminar o sinal negativo dos desvios da média de tal forma que a soma nula destes desvios transformados ocorra apenas quando todos os dados são iguais, ou seja, quando qualquer medida de dispersão bem definida deve ser nula.

Veja na seção {\hyperref[\detokenize{PE104-A:sec-para-saber-mais}]{\sphinxcrossref{\DUrole{std,std-ref}{Para saber mais}}}} a demonstração da propriedade de que a soma dos desvios da média é sempre nula.

\sphinxstylestrong{Desvio Médio Absoluto}

Considerando os desvios da média em valor absoluto (\(|x_i-\bar{x}|\)) observe que todos serão não-negativos tal que a soma dos desvios da média em valor absoluto (\(\displaystyle{\sum^n_{i=1}}|x_i-\bar{x}|\)) será nula apenas quando todos os valores do conjunto forem iguais.

Com base na observação anterior, pode-se definir uma medida de dispersão dos dados, considerando todas as observações, chamada \index{desvio médio absoluto}desvio médio absoluto (DM) que é definida como a média dos desvios da média tomados em valor absoluto.

Na tabela a seguir são apresentados os desvios da média em valor absoluto das cotações nas companhias A e B e, a respectiva soma.


\begin{savenotes}\sphinxattablestart
\centering
\sphinxcapstartof{table}
\sphinxcaption{Desvios da média em valores absolutos para as companhias A e B}\label{\detokenize{PE104-4:id4}}
\sphinxaftercaption
\begin{tabulary}{\linewidth}[t]{|T|T|T|}
\hline

semana
&
A
&
B
\\
\hline
1
&
0,5
&
5,5
\\
\hline
2
&
5,5
&
13,5
\\
\hline
3
&
1,5
&
9,5
\\
\hline
4
&
4,5
&
20,5
\\
\hline
5
&
5,5
&
15,5
\\
\hline
6
&
1,5
&
28,5
\\
\hline
7
&
5,5
&
5,5
\\
\hline
8
&
3,5
&
19,5
\\
\hline
9
&
5,5
&
28,5
\\
\hline
10
&
5,5
&
4,5
\\
\hline
soma
&
39,0
&
151,0
\\
\hline
\end{tabulary}
\par
\sphinxattableend\end{savenotes}

Logo, concluímos que o desvio médio absoluto na companhia A é DM= \(\frac{39}{10}=3,9\) reais e, na companhia B, DM= \(\frac{151}{10}=15,1\) reais, indicando que, de fato, a dispersão em torno da média na companhia B é cerca de 4 vezes maior do que na companhia A com relação ao desvio médio (\({15,1}/{3,9}\approx 3,89\)).

De maneira geral, o desvio médio absoluto do conjunto de dados \(\{ x_1,x_2, \cdots, x_n\}\) é
\begin{equation*}
\begin{split}\textsf{DM} = \frac{1}{n}\cdot \sum^n_{i=1}|x_i-\bar{x}|=\frac{|x_1-\bar{x}|+|x_2-\bar{x}|+\cdots+|x_n-\bar{x}|}{n}\end{split}
\end{equation*}
\sphinxstylestrong{Variância e Desvio Padrão}

Uma outra forma de eliminar o sinal negativo dos desvios da média é elevar ao quadrado cada um deles, tornando-os não-negativos. A \index{variância}variância é definida como uma média dos desvios da média elevados ao quadrado.
\begin{equation*}
\begin{split}\textsf{variância} = \frac{1}{n}\cdot \sum^n_{i=1} (x_i-\bar{x})^2=\frac{(x_1-\bar{x})^2+(x_2-\bar{x})^2+\cdots+(x_n-\bar{x})^2}{n}\end{split}
\end{equation*}
Na tabela a seguir são apresentados os desvios da média elevados ao quadrado das cotações nas companhias A e B e, a respectiva soma.


\begin{savenotes}\sphinxattablestart
\centering
\sphinxcapstartof{table}
\sphinxcaption{Desvios da média elevados ao quadrado para as companhias A e B}\label{\detokenize{PE104-4:id5}}
\sphinxaftercaption
\begin{tabulary}{\linewidth}[t]{|T|T|T|}
\hline

semana
&
A
&
B
\\
\hline
1
&
0,25
&
30,25
\\
\hline
2
&
30,25
&
182,25
\\
\hline
3
&
2,25
&
90,25
\\
\hline
4
&
20,25
&
420,25
\\
\hline
5
&
30,25
&
240,25
\\
\hline
6
&
2,25
&
812,25
\\
\hline
7
&
30,25
&
30,25
\\
\hline
8
&
12,25
&
380,25
\\
\hline
9
&
30,25
&
812,25
\\
\hline
10
&
30,25
&
20,25
\\
\hline
soma
&
188,5
&
3018,5
\\
\hline
\end{tabulary}
\par
\sphinxattableend\end{savenotes}

Logo, concluímos que a variância na companhia A é \(\frac{188,5}{10}=18,85\textsf{ reais}^2\) e, na companhia B, \(\frac{3018,5}{10}=301,85\textsf{ reais}^2\) , indicando que a dispersão em torno da média na companhia B é cerca de 16 vezes maior do que na companhia A com relação à variância  (\(301,85/18,85\approx 16\)).

Quando lidamos com grande quantidade de dados, calcular a variância usando a definição apresentada será uma tarefa maçante, pois após calcular a média de muitos dados, teremos que calcular cada desvio da média, elevá-los ao quadrado e, finalmente, somá-los. Para conjuntos de dados com  mais de 10 elementos será, em geral, muito trabalhoso calcular a variância desta forma. Um modo mais simples para calcular a variância é apresentado a seguir.  Pode-se mostrar que o numerador da fórmulada variância é dado por
\begin{equation*}
\begin{split}\sum^n_{i=1} (x_i-\bar{x})^2 = \sum^n_{i=1} x^2_i-n\cdot \bar{x}^2\end{split}
\end{equation*}
Assim, basta conhecer a soma simples (\(\displaystyle{\sum^n_{i=1}}x_i\)), para determinar a média \(\bar{x}\), e a soma de quadrados (\(\displaystyle{\sum^n_{i=1}}x^2_i\)) para calcular a variância.

A demonstração desta igualdade está na Seção {\hyperref[\detokenize{PE104-A:sec-para-saber-mais}]{\sphinxcrossref{\DUrole{std,std-ref}{Para saber mais}}}}.

Na \DUrole{xref,std,std-ref}{ativ-estrategia-de-investimento} , podemos verificar que na companhia A, \(\bar{x}=61,5\) e \(\displaystyle{\sum^{10}_{i=1}} x^2_i=38.011\) tal que a variância em A pode ser calculada por
\begin{equation*}
\begin{split}\textsf{variância}=\frac{1}{10}\cdot (38.011-10\cdot 61,5^2)=18,85\textsf{ reais}^2\end{split}
\end{equation*}
e, na companhia B,

\(\bar{x}=61,5\) e \(\displaystyle{\sum^{10}_{i=1}} x^2_i=40.841\) tal que a variância em B pode ser calculada por
\begin{equation*}
\begin{split}\textsf{variância}=\frac{1}{10}\cdot (40.841-10\cdot 61,5^2)=301,85\textsf{ reais}^2\end{split}
\end{equation*}
Vimos que o desvio médio absoluto da companhia B foi aproximadamente 4 vezes maior do que o da companhia A. Na comparação de variâncias, a variância da companhia B foi cerca de 16 vezes maior do que a da companhia A. Este grande aumento deve-se ao fato de que consideramos os desvios da média elevados ao quadrado no cálculo da variância. Observe que a unidade de medida na variância é o quadrado da unidade de medida das observações. Para retornar à escala de medida das observações, basta extrair a raiz quadrada da variância, levando a definição de desvio padrão, uma medida de dispersão em torno da média, na mesma unidade das observações.
\begin{equation*}
\begin{split}\textsf{desvio padrão}=\sqrt{\textsf{variância}}\end{split}
\end{equation*}
No exemplo das cotações, podemos verificar que na companhia A,
\begin{equation*}
\begin{split}\textsf{desvio padrão}=\sqrt{18,85} \approx 4,34 \textsf{ reais}\end{split}
\end{equation*}
e, na companhia B,
\begin{equation*}
\begin{split}\textsf{desvio padrão}=\sqrt{301,85}\approx 17,37\textsf{ reais}\end{split}
\end{equation*}
Verifique que o desvio padrão da companhia B é aproximadamente 4 vezes maior do que o da companhia A.

\begin{observation}{Por que o desvio padrão é preferível ao desvio médio?}

Você deve estar se perguntando por que se utiliza o desvio padrão na Estatística em detrimento do desvio médio, cujo cálculo é bem mais simples. A resposta é um tanto complexa para o nível em que estamos, mas ela está associada à necessidade na Estatística de se minimizar estruturas de maneira simples. O desvio médio faz uso da função modular \(f(x)=|x|\), que não possui boas propriedades matemáticas para a minimização, por possuir na sua forma uma mudança abrupta em torno de \(x=0\),  enquanto que a variância faz uso da função quadrática \(f(x)=x^2\), representando parábolas de vértice suave e cujas propriedades analíticas são bem conhecidas. Veja a figura a seguir.

\begin{figure}[H]
\centering
\capstart

\noindent\sphinxincludegraphics[width=300bp]{{funcoesmoduloequadratica_2}.png}
\caption{Funções modular e quadrática com destaque para o comportamento em torno de x=0.}\label{\detokenize{PE104-4:fig-coloque-aqui-o-nome}}\label{\detokenize{PE104-4:id6}}\end{figure}

Muitos problemas de estimação de posição de astros na Física são resolvidos por funções quadráticas por esse motivo, um legado deixado pelo matemático alemão Carl Friedrich Gauss (1777 - 1855) no chamado  \sphinxhref{https://pt.wikipedia.org/wiki/M\%C3\%A9todo\_dos\_m\%C3\%ADnimos\_quadrados}{Método dos Mínimos Quadrados}.

\begin{figure}[H]
\centering
\capstart

\noindent\sphinxincludegraphics[width=100bp]{{gauss}.png}
\caption{Carl Friedrich Gauss}\label{\detokenize{PE104-4:id1}}\label{\detokenize{PE104-4:id7}}\end{figure}
\end{observation}

\sphinxstylestrong{Variância populacional e amostral, desvio padrão populacional e amostral}

No capítulo \sphinxstylestrong{A Natureza da Estatística} foram apresentados os conceitos \index{parâmetro}parâmetro e \index{estimador}estimador. Parâmetro é uma característica numérica da população, em geral desconhecida; enquanto estimador é uma função dos dados da amostra (subconjunto da população), usada para estimar o parâmetro. Em geral, usam-se letras gregas para denotar parâmetros.

Se dispomos de uma amostra da população, de fato, calculamos a média amostral e a variância amostral (funções dos dados da amostra) e usamos estes resultados como estimativas da média populacional e da variância populacional. Como já foi comentado anteriormente, a média amostral tem boas propriedades como estimador da média populacional. No entanto, é possível mostrar que a variância calculada pela fórmula apresentada no início deste capítulo é um estimador que tende a produzir valores menores do que o valor da variância da população. Dizemos que é um estimador viesado por essa razão.

Para contornar este defeito do estimador, usamos o denominador \(n-1\) no lugar de \(n\). Observe que com isto os valores produzidos serão um pouco maiores, pois o denominador é um pouco menor.

Assim, as expressões que deverão ser usadas quando o conjunto de dados sob estudo é uma amostra da população são dadas por
\begin{align*}\!\begin{aligned}
\textsf{variância amostral}=s^2=\frac{1}{n-1}\sum^n_{i=1}(x_i-\bar{x})^2\\
\textsf{desvio padrão amostral}=\sqrt{s^2}=s\\
\end{aligned}\end{align*}
Na maioria das vezes trabalhamos com amostras. Assim, neste capítulo, salvo menção em contrário, estaremos sempre calculando a variância amostral (\(s^2\)) e o desvio padrão amostral (\(s\)), mesmo que o termo “amostral” esteja omitido.

Se você estiver trabalhando com uma amostra e usar o denominador \(n\) para calcular a variância, isso implicará que você escolheu um estimador viesado, pois tende a produzir estimativas que são menores do que o verdadeiro valor da variância. Observe que se você estiver trabalhando com amostras muito grandes, essa diferença não será importante, pois haverá pouca diferença entre dividir por \(n\) ou por \(n-1\).

Expressões que deverão ser consideradas quando o conjunto de dados sob estudo refere-se à população com \(N\) elementos:
\begin{align*}\!\begin{aligned}
\textsf{variância populacional} = \sigma^2=\frac{1}{N}\sum^n_{i=1}(x_i-\mu)^2\\
\textsf{desvio padrão populacional}=\sqrt{\sigma^2}=\sigma\\
\end{aligned}\end{align*}
em que \(\mu\) representa a média populacional.

Veja na figura a seguir uma saída do GeoGebra de análise descritiva do conjunto de Notas de Artes sem bonificação.

\begin{figure}[H]
\centering
\capstart

\noindent\sphinxincludegraphics[width=100bp]{{summary_NArtes}.png}
\caption{Medidas-resumo no GeoGebra das notas de Artes}\label{\detokenize{PE104-4:fig-medidas-resumo-geogebra}}\label{\detokenize{PE104-4:id8}}\end{figure}

Observe que o GeoGebra usa como separador decimal o ponto e não a vírgula. Logo após a informação da média (com quatro casas decimais), tem-se a letra grega \(\sigma\), usada para representar desvio padrão populacional. Em seguida, tem-se a letra \(s\), usada para representar o desvio padrão amostral.
\phantomsection\label{\detokenize{PE104-4:ativ-inflacao-anual}}

\bigskip\hrule\bigskip

\begin{task}{inflação de dois países}

A seguir são apresentados dados sobre as inflações anuais em dois países. Antes de trabalhar com os dados, vamos tentar explicar o que é \index{inflação}inflação. De uma maneira bem simples, pode-se dizer que a inflação é o aumento contínuo nos preços de produtos e serviços. Esse aumento costuma ser avaliado de forma mensal, gerando os índices de inflação, que refletem a variação nos preços.

A inflação pode ser medida de várias formas. O índice oficial de inflação no Brasil é o IPCA (Índice de Preços ao Consumidor Amplo), que mede a variação mensal de preços de produtos considerando o consumo de famílias com renda mensal entre 1 e 40 salários mínimos. O IBGE (Instituto Brasileiro de Geografia e Estatística) é o orgão responsável pela medição e divulgação do IPCA. Veja neste
\sphinxhref{https://www.youtube.com/watch?v=JVcDZOlIMBk}{link}, um vídeo produzido pelo IBGE, explicando o IPCA.

Foram observadas as inflações anuais de dois países A e B para os anos de 2011 a 2015, conforme tabela a seguir.


\begin{savenotes}\sphinxattablestart
\centering
\sphinxcapstartof{table}
\sphinxcaption{Inflação anual}\label{\detokenize{PE104-4:id9}}
\sphinxaftercaption
\begin{tabulary}{\linewidth}[t]{|T|T|T|T|T|T|T|}
\hline

País
&
2011
&
2012
&
2013
&
2014
&
2015
&
soma
\\
\hline
A
&
2,00\%
&
1,80\%
&
2,10\%
&
2,20\%
&
1,90\%
&
10,00\%
\\
\hline
B
&
0,01\%
&
-0,19\%
&
-0,09\%
&
0,21\%
&
0,11\%
&
0,05\%
\\
\hline
\end{tabulary}
\par
\sphinxattableend\end{savenotes}
\begin{enumerate}
\item {} 
Calcule as médias das inflações anuais dos dois países. Há diferenças entre elas?

\item {} 
Calcule as variâncias das inflações anuais dos dois países, sabendo que para o país A, \(\displaystyle{\sum^5_{i=1}}x^2_i=20,1\)  (\% \(^2\) ) e para o país B,  \(\displaystyle{\sum^5_{i=1}}x^2_i=0,1005\)  (\% \(^2\) ). Há diferença entre elas?

\item {} 
Qual dos países apresenta maior variação inflacionária quando comparada à média inflacionária?

\end{enumerate}
\end{task}


\sphinxstylestrong{Coeficiente de variação}

Nem sempre uma variância pequena (e consequentemente desvio-padrão pequeno) significa pouca dispersão. Tampouco uma variância grande é sempre indicador de alta dispersão. Esses valores podem ser altos ou baixos devido à magnitude (ordem de grandeza) dos dados observados. Se medimos observações em microscópio, por exemplo, teremos inevitavelmente valor numericamente baixo de variância, podendo no entanto haver alta dispersão dos dados no nível microscópico. Da mesma maneira, ao medir os produtos internos brutos brasileiros em dólares em vários anos teremos valores observados de alta magnitude, gerando variância numericamente grande, mas não necessariamente indicando alta dispersão.

Na \DUrole{xref,std,std-ref}{ativ-inflacao-anual}, estudamos dois conjuntos de dados que apresentam médias diferentes, mas variâncias iguais. Podemos dizer que o impacto da variância em relação à média é o mesmo para os dois conjuntos? Comparando o valor do desvio padrão de cerca de 0,16\% à média do país A de 2\%, vemos que ele é pequeno em relação à média. Comparando o valor do desvio padrão 0,16\% em relação à média do país B de 0,01\%, vemos que ele é muito grande em relação à média. Neste caso dizemos que no país A os dados apresentam variação relativa em torno da média pequena. Já, no país B, os dados apresentam variação relativa em torno da média grande.

O \index{coeficiente de variação}coeficiente de variação é uma medida usada para calcular a variação relativa dos dados de um conjunto em torno da média: quanto maior seu valor, maior é a variação relativa em torno da média.
\begin{description}
\item[{Coeficiente de variação\index{Coeficiente de variação|textbf}}] \leavevmode\phantomsection\label{\detokenize{PE104-4:term-coeficiente-de-variacao}}
é a razão entre o desvio padrão e a média. Em geral, ele é descrito em termos percentuais.

\end{description}

O coeficiente de variação amostral, em termos percentuais, é calculado  por
\begin{equation*}
\begin{split}CVA=\frac{s}{\bar{x}}\cdot 100 \%\end{split}
\end{equation*}
em que \(s\) é o desvio padrão amostral e \(\bar{x}\) é a média amostral.

Esta expressão é usada quando dispomos de uma amostra da população. Se, dispomos dos dados da população, então o coeficiente de variação populacional é dado por
\begin{equation*}
\begin{split}CVP=\frac{{\sigma}}{\mu}\cdot 100\%\end{split}
\end{equation*}
em que \(\sigma\) é o desvio padrão populacional e \(\mu\) é a média populacional.

Observe que o coeficiente de variação só é definido para conjuntos cuja média é diferente de zero.


\practice{ }
\label{\detokenize{PE104-5:sec-praticando2}}\label{\detokenize{PE104-5::doc}}\label{\detokenize{PE104-5:praticando}}\phantomsection\label{\detokenize{PE104-5:ativ-compara-categorias}}
\begin{task}{ comparação de conjuntos de dados}

Para realizar esta atividade será necessário coletar dois conjuntos de dados da mesma natureza, correspondentes a grupos distintos, os quais queremos comparar. Por exemplo:
\begin{itemize}
\item {} 
alturas de homens e mulheres;

\item {} 
alturas de alunos de 1º e de 9º ano do Ensino Fundamental;

\item {} 
notas de disciplinas distintas;

\item {} 
notas de turmas distintas na mesma disciplina;

\item {} 
medições de produtos naturais: comprimento das folhas de vegetais (alface, rúcula, etc) comprados em lojas distintas, altura de árvores ou plantas similares locais da cidade distintos;

\end{itemize}

entre outros que podem ser escolhidos dependendo da região e dos recursos disponíveis na escola.

No seu caderno ou em uma planilha eletrônica, registre os dados coletados, como indicado no modelo de tabela a seguir, lembrando que quanto mais dados você coletar com os critérios definidos, os resultados do experimento terão maior chance de refletir a realidade.

Para calcular as medidas de posição e dispersão, utilize de forma cuidadosa as  fórmulas apresentadas. De forma alternativa, você pode digitar os dados no \sphinxhref{https://ggbm.at/KbYqnQ6Q}{Aplicativo de medidas de posição e dispersão do Livro Aberto} e obter as medidas resumo dos dados.


\begin{savenotes}\sphinxattablestart
\centering
\sphinxcapstartof{table}
\sphinxcaption{Registre os seus resultados}\label{\detokenize{PE104-5:id1}}
\sphinxaftercaption
\begin{tabulary}{\linewidth}[t]{|T|T|T|}
\hline
\sphinxstylethead{\sphinxstyletheadfamily \unskip}\relax &\sphinxstylethead{\sphinxstyletheadfamily 
Grupo   A
\unskip}\relax &\sphinxstylethead{\sphinxstyletheadfamily 
Grupo B
\unskip}\relax \\
\hline
Nome da categoria
&&\\
\hline
Mínimo (\(x_{(1)}\))
&&\\
\hline
Máximo  (\(x_{(n)}\))
&&\\
\hline
Média
&&\\
\hline
Q1
&&\\
\hline
Mediana
&&\\
\hline
Q3
&&\\
\hline
Amplitude amostral (R)
&&\\
\hline
Dist. entre quartis (DQ)
&&\\
\hline
Desvio médio absoluto (DM)
&&\\
\hline
Variância amostral (\(s^2\))
&&\\
\hline
Desvio padrão amostral (\(s\))
&&\\
\hline
\end{tabulary}
\par
\sphinxattableend\end{savenotes}

Sugere-se a construção dos histogramas para comparar os dois grupos. Você pode usar o GeoGebra para esta construção.
\end{task}
\begin{enumerate}
\setcounter{enumi}{1}
\item {} 
Analisando os dois conjuntos de dados obtidos, que medida de posição você julga mais adequada para resumir a informação do conjunto? Por quê?

\item {} 
Os resultados que você obteve parecem refletir a realidade? Existe algum resultado científico que suporte estas observações? Consulte  professores de outras áreas sobre suas conclusões.

\end{enumerate}
\phantomsection\label{\detokenize{PE104-5:ativ-aproxima-dpa-usando-r}}

\bigskip\hrule\bigskip

\begin{quote}

Nos conjuntos de dados, quando não há valores atípicos (valores muito altos ou muito baixos em relação à maior parte dos valores no conjunto), a maior parte dos valores se situará no intervalo centrado na média distando 2 desvios padrões à esquerda e à direita da média. A partir desta suposição, pode-se obter uma fórmula para estimar o valor do desvio padrão amostral \(s\) .
\begin{equation*}
\begin{split}\left \{ \begin{array}{l} \textsf{Max}=x_{(n)}\approx \bar{x}+2\cdot s \\ \textsf{Min}=x_{(1)}\approx \bar{x}-2\cdot s\end{array}\right.\end{split}
\end{equation*}
Tomando a diferença das primeiras expressões apresentadas, obtemos
\begin{equation*}
\begin{split}R= \textsf{Max-Min} \approx 4\cdot s\end{split}
\end{equation*}
tal que
\begin{equation*}
\begin{split}s\approx \frac{R}{4}\end{split}
\end{equation*}\begin{enumerate}
\item {} 
Use esta fórmula para estimar o valor do desvio padrão amostral dos dados da \DUrole{xref,std,std-ref}{ativ-Notas-de-Artes} e compare o valor obtido com o desvio padrão amostral \(s\). Use os dados na figura a seguir, produzidos pelo GeoGebra.

\end{enumerate}

\begin{figure}[H]
\centering
\capstart

\noindent\sphinxincludegraphics[width=100bp]{{summary_NArtes}.png}
\caption{Estatísticas resumo das Notas de Artes}\label{\detokenize{PE104-5:fig-resumonartes}}\label{\detokenize{PE104-5:id2}}\end{figure}
\begin{enumerate}
\setcounter{enumi}{1}
\item {} 
Idem para estimar o valor do desvio padrão amostral dos dados da \DUrole{xref,std,std-ref}{ativ-Maratona-de-NY} e compare o valor obtido com o desvio padrão amostral \(s\). Use os dados na figura a seguir, produzidos pelo GeoGebra.

\end{enumerate}

\begin{figure}[H]
\centering
\capstart

\noindent\sphinxincludegraphics[width=100bp]{{summary_MaratonaNYMulheres}.png}
\caption{Estatísticas resumo dos 100 melhores tempos para mulheres - Maratona de Nova Iorque/2017}\label{\detokenize{PE104-5:fig-summarymaratonamulheres}}\label{\detokenize{PE104-5:id3}}\end{figure}
\begin{enumerate}
\setcounter{enumi}{2}
\item {} 
Idem para estimar o valor de desvio padrão amostral dos dados da \DUrole{xref,std,std-ref}{ativ-Estrategia-de-investimento}. Use os dados na figura a seguir, produzidos pelo GeoGebra.

\end{enumerate}

\begin{figure}[H]
\centering
\capstart

\noindent\sphinxincludegraphics[width=200bp]{{summary_estrategiainvestimento}.png}
\caption{Estatísticas resumo das cotações das ação nas Companhias A e B.}\label{\detokenize{PE104-5:fig-estrategia}}\label{\detokenize{PE104-5:id4}}\end{figure}
\end{quote}


\phantomsection\label{\detokenize{PE104-5:ativ-mediamaisoumenosdoisdesvios}}
\begin{task}{média mais ou menos 2 desvios padrões}

Para os conjuntos de dados considerados na \DUrole{xref,std,std-ref}{ativ-aproxima-dpa-usando-R}, calcule a frequência absoluta de dados que estão no intervalo \([\bar{x}-2\cdot s,\bar{x}+2\cdot s]\) e comente sobre os resultados obtidos.
\end{task}



\begin{task}{comparação das bonificações nas Notas de Artes}

Vamos retornar a \DUrole{xref,std,std-ref}{ativ-Notas-de-Artes} e às duas possibilidades de bonificação das notas: acrescentar um ponto a todos os alunos ou aumentar em 20\% a nota de cada aluno. Suponha, que o professor deseja que o resultado geral de sua turma apresente o menor coeficiente de variação. Partindo deste ponto de vista, qual das duas possibilidades é mais interessante para o professor adotar?

Para facilitar, use as informações a seguir.


\begin{savenotes}\sphinxattablestart
\centering
\sphinxcapstartof{table}
\sphinxcaption{Dados sobre as somas simples e somas de quadrados das notas antes da bonificação (antes), após serem acrescidas de um ponto (1 pt) e após serem aumentadas em 20\% (20\%)}\label{\detokenize{PE104-5:id5}}
\sphinxaftercaption
\begin{tabulary}{\linewidth}[t]{|T|T|T|T|}
\hline

\(n=35\)
&
antes
&
1 pt
&
20\%
\\
\hline
\(\sum x\)
&
207,5
&
242,5
&
249,0
\\
\hline
\(\sum x^2\)
&
1361,39
&
1811,39
&
1960,402
\\
\hline
\end{tabulary}
\par
\sphinxattableend\end{savenotes}
\end{task}




\explore{  Boxplot}
\label{\detokenize{PE104-6::doc}}\label{\detokenize{PE104-6:explorando-boxplot}}\label{\detokenize{PE104-6:sec-explorando3}}

\begin{task}{homens e mulheres na maratona de Nova Iorque}
\label{\detokenize{PE104-6:atividade-homens-e-mulheres-na-maratona-de-nova-iorque}}\label{\detokenize{PE104-6:ativ-construcao-do-boxplot}}

No quadro a seguir, são apresentadas algumas informações sobre os 100 melhores tempos na maratona de Nova Iorque em 2017 para o grupo dos homens e para o grupo das mulheres.


\begin{savenotes}\sphinxattablestart
\centering
\sphinxcapstartof{table}
\sphinxcaption{Medidas resumo para os 100 melhores tempos de mulheres e homens na maratona de Nova Iorque/2017}\label{\detokenize{PE104-6:id1}}
\sphinxaftercaption
\begin{tabulary}{\linewidth}[t]{|T|T|T|}
\hline

medida
&
Mulheres
&
Homens
\\
\hline
Min
&
2,448
&
2,181
\\
\hline
Q1
&
2,772
&
2,473
\\
\hline
mediana
&
2,949
&
2,550
\\
\hline
Q3
&
2,998
&
2,611
\\
\hline
Max
&
3,086
&
2,639
\\
\hline
\end{tabulary}
\par
\sphinxattableend\end{savenotes}
\begin{enumerate}
\item {} 
Identifique entre as 10 medidas calculadas para as duas amostras, o menor e o maior valores obtidos.

\item {} 
Construa em um eixo, que pode ser vertical ou horizontal, uma escala que comprenda este intervalo do menor valor até o maior valor identificados no item anterior.

\item {} 
Marque no eixo os valores de Q1 e Q3  para as mulheres e desenhe um retângulo cujas bases correspondam a estas duas medidas.

\item {} 
Dentro do retângulo desenhado, trace um segmento paralelo às bases que corresponda ao valor da Mediana das mulheres.

\item {} 
Para terminar, trace um segmento partindo do ponto médio da base correspondente a Q1 até o menor valor observado e, um segmento partindo do ponto médio da base correspondente a Q3 até o maior valor observado.

\item {} 
Ao lado ou acima da figura construída (dependendo de seu eixo ter orientação vertical ou horizontal), repita os intens anteriores para a categoria homens.

\item {} 
Compare as duas figuras obtidas. Que diferenças você pode destacar entre as duas categorais, a partir da figura construída?

\end{enumerate}
\end{task}




\arrange{  Boxplot}
\label{\detokenize{PE104-6:sec-organizandoasideias3}}\label{\detokenize{PE104-6:organizando-as-ideias-boxplot}}
O gráfico construído na {\hyperref[\detokenize{PE104-6:ativ-construcao-do-boxplot}]{\sphinxcrossref{\DUrole{std,std-ref}{Atividade: homens e mulheres na maratona de Nova Iorque}}}} é uma versão simplificada do gráfico conhecido como \sphinxstylestrong{boxplot} na qual não se consideram valores atípicos. Este gráfico, muito simples de ser construído, usa a informação das medidas Mínimo, Q1, Mediana, Q3 e Máximo.

A construção do \sphinxstylestrong{boxplot} é baseada em  cinco medidas de posição, que compõem o \index{esquema dos cinco números}esquema dos cinco números, a saber,
\begin{enumerate}
\item {} 
mínimo (\(\textsf{Min}=x_{(1)}\)),

\item {} 
primeiro quartil (\(\textsf{Q}1\)),

\item {} 
mediana (\(\textsf{Q}2\)),

\item {} 
terceiro quartil (\(\textsf{Q}3\)) e

\item {} 
máximo (\(\textsf{Max}=x_{(n)}\)).

\end{enumerate}

Por exemplo, veja na figura a seguir o boxplot dos 100 melhores tempos das mulheres na maratona de Nova Iorque/2017, considerando a orientação do  eixo vertical.

\begin{figure}[H]
\centering
\capstart

\noindent\sphinxincludegraphics[width=200bp]{{boxplotmulheres}.png}
\caption{Boxplot dos 100 melhores tempos na Maratonona de Nova Iorque/2017 na categoria mulheres}\label{\detokenize{PE104-6:fig-boxplotmulheres}}\label{\detokenize{PE104-6:id3}}\end{figure}

O primeiro passo na construção do boxplot é traçar um eixo na escala dos dados observados, que pode ter orientação tanto vertical como horizontal, e, depois, desenhar um retângulo cujas bases correspondem ao primeiro e ao terceiro quartis, de acordo com o eixo. Em seguida, traça-se um segmento paralelo às bases, correspondendo ao valor da mediana. Veja a figura a seguir, considerando o eixo na escala dos dados com orientação vertical.

\begin{figure}[H]
\centering
\capstart

\noindent\sphinxincludegraphics[width=200bp]{{boxplotcaixa_2}.png}
\caption{Caixa do boxplot}\label{\detokenize{PE104-6:fig-caixadoboxplot}}\label{\detokenize{PE104-6:id4}}\end{figure}

A distância entre quartis (\(\textsf{DQ}=\textsf{Q}3-\textsf{Q}1\)) é a medida de dispersão utilizada na classificação de valores da distribuição como  \index{valores atípicos}valores atípicos, isto é, valores que destoam dos demais no conjunto de dados.

O critério adotado para classificar um valor como atípico na construção do boxplot é descrito a seguir.

Defina
\begin{equation*}
\begin{split}\textsf{cerca inferior}=\textsf{Q}1-1,5\cdot \textsf{DQ}\textsf{ e }\textsf{cerca superior}=\textsf{Q}3+1,5\cdot \textsf{DQ}\end{split}
\end{equation*}
Se \(x_i< \textsf{cerca inferior}\) ou \(x_i> \textsf{cerca superior}\) , então \(x_i\) é classificado como valor atípico, e assinalado no boxplot com um asterisco ou algum outro caracter, de acordo com o eixo na escala dos dados. Veja figura a seguir.

\begin{figure}[H]
\centering
\capstart

\noindent\sphinxincludegraphics[width=300bp]{{boxplotdq_2}.png}
\caption{Valores atípicos no boxplot}\label{\detokenize{PE104-6:fig-valoresatipicosnoboxplot}}\label{\detokenize{PE104-6:id5}}\end{figure}

Na finalização da contrução do boxplot, traçam-se segmentos paralelos ao eixo considerado (vertical ou horizontal) partindo dos pontos médios das bases do retângulo e terminando nos maior e menor valores não atípicos observados. Veja figura a seguir.

\begin{figure}[H]
\centering
\capstart

\noindent\sphinxincludegraphics[width=300bp]{{boxplotcompl_1}.png}
\caption{Ilustração do boxplot}\label{\detokenize{PE104-6:fig-finalizacaodoboxplot}}\label{\detokenize{PE104-6:id6}}\end{figure}

A figura a seguir ilustra um boxplot quando não há valores atípicos no conjunto de dados. Observe que neste caso, destacam-se no gráfico as medidas do esquema dos cinco números.

\begin{figure}[H]
\centering
\capstart

\noindent\sphinxincludegraphics[width=200bp]{{boxplotx_1}.png}
\caption{Boxplot quando não há valores atípicos}\label{\detokenize{PE104-6:fig-boxplotsemvaloratipico}}\label{\detokenize{PE104-6:id7}}\end{figure}

O retângulo do boxplot corresponde aos 50\% valores centrais da distribuição, ou seja, metade dos dados estão no intervalo delimitado pela  caixa (retângulo) e, a outra metade, está nos dois intervalos delimitados fora da caixa, sendo 25\% acima e 25\% abaixo da caixa.

As medidas do esquema dos cinco números nos permitem avaliar o grau de assimetria da distribuição. Por exemplo, se
\begin{enumerate}
\item {} 
\(\textsf{mediana} -\textsf{Q}1\approx \textsf{Q}3-\textsf{mediana}\)

\item {} 
\(\textsf{Q}1-x_{(1)} \approx x_{(n)}-\textsf{Q}3\)

\item {} 
\(\textsf{mediana}-x_{(1)}\approx x_{(n)}-\textsf{mediana}\)

\end{enumerate}

podemos concluir que a distribuição é aproximadamente simétrica, porém se alguns destes pares de intervalos apresentarem comprimentos muito diferentes, isso indica que a distribuição apresenta algum tipo de assimetria.

Afinal, para que servem os quartis da distribuição?

Os quartis servem para
\begin{enumerate}
\item {} 
identificar \index{valores atípicos}valores atípicos da distribuição (se houver), também conhecidos como  valores discrepantes ou \sphinxstyleemphasis{outliers};

\item {} 
avaliar o grau de assimetria da distribuição empírica do conjunto de dados e

\item {} 
construir um gráfico alternativo ao histograma para representar dados quantitativos conhecido como \sphinxstyleemphasis{boxplot} ou gráfico-caixa.

\end{enumerate}

Analisando o {\hyperref[\detokenize{PE104-6:fig-boxplotmulheres}]{\sphinxcrossref{\DUrole{std,std-ref}{Boxplot dos 100 melhores tempos na Maratonona de Nova Iorque/2017 na categoria mulheres}}}} podemos observar que
\begin{enumerate}
\item {} 
não existem valores atípicos;

\item {} 
o melhor tempo é ligeiramente inferior a 2,5 h e, o centésimo tempo, é ligeiramente inferior a 3,1 h;

\item {} 
o tempo que corresponde à mediana está entre 2,9 h e 3,0 h;

\item {} 
o primeiro quartil está próximo de 2,8 h e, o terceiro, próximo de 3,0 h e que

\item {} 
a distribuição dos 100 melhores tempos na categoria mulheres apresenta assimetria à esquerda. Verifique que

\end{enumerate}

\(\textsf{mediana} -\textsf{Q}1 > \textsf{Q}3-\textsf{mediana}\)

\(\textsf{Q}1-x_{(1)} >> x_{(n)}-\textsf{Q}3\)

\(\textsf{mediana}-x_{(1)}>> x_{(n)}-\textsf{mediana}\)  em que o símbolo \(>>\) é usado para representar “bem  maior do que”.

Os valores exatos destas medidas estão na figura {\hyperref[\detokenize{PE104-7:fig-medidasresumo4categorias}]{\sphinxcrossref{\DUrole{std,std-ref}{Medidas resumo para as quatro categorias da maratona de Nova Iorque/2017}}}}.

Vimos que o boxplot é útil para avaliar a forma da distribuição quanto ao grau de assimetria e também revela valores atípicos, se houver.

Uma regra empírica para avaliar frequências de valores em intervalos em torno da média que pode ser útil, é obtida a partir das propriedades de um modelo teórico conhecido como densidade normal de probabilidades. Entre várias propriedades desta densidade, destaca-se que ela é simétrica e unimodal tal que média, mediana e moda são iguais. Veja na figura a seguir uma ilustração da densidade normal com média \(\mu\) e desvio padrão \(\sigma\), também conhecida como a curva em forma de sino.

\begin{figure}[H]
\centering
\capstart

\noindent\sphinxincludegraphics[width=300bp]{{densidadenormal_1}.png}
\caption{Densidade Normal com região colorida no intervalo entre \(\mu-\sigma\) e \(\mu+\sigma\) , cuja área corresponde a aproximadamente 0,67 da área total igual a 1.}\label{\detokenize{PE104-6:fig-densidade-normal}}\label{\detokenize{PE104-6:id8}}\end{figure}

A regra empírica estabelece que em distribuições aproximadamente simétricas para as quais a presença de valores atípicos é muito rara ou não existem valores atípicos,
\begin{enumerate}
\item {} 
a frequência relativa de valores no intervalo \([\bar{x}-s;\bar{x}+s]\) é aproximadamente 67\%,

\item {} 
a frequência relativa de valores no intervalo \([\bar{x}-2\cdot s; \bar{x}+2\cdot s]\) é aproximadamente 95\%.

\end{enumerate}

No caso dos dados da \DUrole{xref,std,std-ref}{ativ-Maratona-de-NY} vimos que não existem valores atípicos, mas a distribuição apresenta assimetria à esquerda. Ainda assim, contando frequência de observações que nos intervalos definidos por \([\bar{x}-s;\bar{x}+s]\) e  \([\bar{x}-2\cdot s; \bar{x}+2\cdot s]\),  obtém-se 69\% e 93\%, respectivamente. Observe que este valores estão próximos dos valores estipulados pela regra empírica, mesmo com este conjunto apresentando assimetria à esquerda.

O boxplot é muito útil na comparação de diferentes grupos, como veremos na atividade a seguir.


\practice{ }
\label{\detokenize{PE104-7:sec-praticando3}}\label{\detokenize{PE104-7::doc}}\label{\detokenize{PE104-7:praticando}}

\begin{task}{modalidades da maratona de Nova Iorque 2017}
\label{\detokenize{PE104-7:ativ-comparacaodegruposusandoboxplot}}\label{\detokenize{PE104-7:atividade-modalidades-da-maratona-de-nova-iorque-2017}}

Nas figuras a seguir apresentam-se os boxplots dos 100 melhores tempos para na maratona de Nova Iorque no ano de 2017 para as categorias homens e mulheres e os melhores tempos para as categorias cadeira de rodas e triciclo de mão e as medidas resumo calculadas pelo GeoGebra para as quatro categorias.

\begin{figure}[H]
\centering
\capstart

\noindent\sphinxincludegraphics[width=450bp]{{boxplots_maratona}.png}
\caption{Boxplots para os 100 melhores tempos das categorias homens e mulheres e dos melhores tempos das categorias cadeira de rodas e triciclo de mão da maratona de Nova Iorque/2017}\label{\detokenize{PE104-7:fig-boxplotsmaratona}}\label{\detokenize{PE104-7:id1}}\end{figure}

\begin{figure}[H]
\centering
\capstart

\noindent\sphinxincludegraphics[width=450bp]{{resumo-quatrocategorias}.png}
\caption{Medidas resumo para as quatro categorias da maratona de Nova Iorque/2017}\label{\detokenize{PE104-7:fig-medidasresumo4categorias}}\label{\detokenize{PE104-7:id2}}\end{figure}
\begin{enumerate}
\item {} 
Qual das modalidades apresentou maior dispersão?

\item {} 
Qual(ais) modalidade(s) apresentaram valores atípicos?

\item {} 
Como você avalia, em relação à simetria, cada uma das distribuições?

\item {} 
Faça uma análise comparativa das distribuições das modalidades homens e mulheres, usando a figura a seguir.

\end{enumerate}

\begin{figure}[H]
\centering
\capstart

\noindent\sphinxincludegraphics[width=400bp]{{bphm_1}.png}
\caption{Boxplot dos 100 melhores tempos para homens e mulheres na maratona de Nova Iorque/2017}\label{\detokenize{PE104-7:fig-boxplothm}}\label{\detokenize{PE104-7:id3}}\end{figure}
\begin{enumerate}
\setcounter{enumi}{4}
\item {} 
Faça uma análise comparativa das distribuições das modalidades cadeira de rodas e triciclo de mão.

\end{enumerate}
\end{task}




\know{Cálculos para dados agrupados}
\label{\detokenize{PE104-A:sec-para-saber-mais}}\label{\detokenize{PE104-A::doc}}\label{\detokenize{PE104-A:para-saber-mais}}


\sphinxstylestrong{Média}

Considere um conjunto de \(n\) dados agrupados em \(c\) intervalos de classe.

Sejam \(\tilde{x}_{1}\), \(\tilde{x}_{2}\), …, \(\tilde{x}_{c}\) os pontos médios dos \(c\) intervalos de classe e, \(n_1\), \(n_2\), …, \(n_c\) ,  as frequências absolutas dos \(c\) intervalos de classe, respectivamente. Lembre que o ponto médio de um intervalo de classe  corresponde à média aritmética dos extremos do intervalo. Neste caso a média é calculada por

\(\textsf{média}=\bar{x}=\frac{n_1\cdot \tilde{x}_{1}+n_2\cdot \tilde{x}_{2}+\cdots+n_c\cdot \tilde{x}_{c}}{\underbrace{n_1+n_2+\cdots+n_c}_{=n}}=\frac{1}{n}\cdot \displaystyle{\sum^c_{i=1}}n_i\cdot \tilde{x}_i\)

Denotando por \(f_i=\frac{n_i}{n}\) a frequência relativa do \(i\)-ésimo intervalo classe, temos

\(\textsf{média}=\bar{x}=f_1\cdot \tilde{x}_{1}+f_2\cdot \tilde{x}_{2}+\cdots +f_c\cdot \tilde{x}_{c}=\displaystyle{\sum^c_{i=1}}f_i\cdot \tilde{x}_i\)

Quando os dados estão agrupados em intervalos de classe, a média é calculada como uma média ponderada dos pontos médios das classes em que os pesos são dados pelas frequências absolutas (ou relativas) das classes.

\sphinxstylestrong{Mediana}

Para obter uma aproximação da mediana quando os dados estão agrupados, deve-se primeiro determinar as frequências acumuladas (absoluta ou relativa) associadas a cada intervalo. Se as frequências forem absolutas, deve-se identificar em qual intervalo encontra-se a observação na posição central (\(\frac{n+1}{2}\) se \(n\) for ímpar, ou as duas posições centrais (\(\frac{n}{2}\) e \(\frac{n}{2}+1\)) se \(n\) for par. Depois, como foi sugerido anteriormente, tome como mediana o ponto médio do intervalo de classe que compreende a(s) posição(ões) central(is).

\sphinxstylestrong{Variância e desvio padrão amostrais}
\begin{equation*}
\begin{split}s^2 = \frac{1}{n-1}\sum^c_{i=1}n_i(\tilde{x}_i-\bar{x})^2=\frac{1}{n-1}\left(\sum^c_{i=1}n_i\tilde{x}^2_i- n\bar{x}^2\right )\end{split}
\end{equation*}
em que \(\bar{x}\) é a média amostral. Se conhecemos apenas as frequências relativas do conjunto de dados, também podemos calcular a variância amostral por \(s^2=\displaystyle{\sum^c_{i=1}}f_i(\tilde{x}_i-\bar{x})^2=\displaystyle{\sum^c_{i=1}}f_i\tilde{x}^2_i -\bar{x}^2\).

O desvio padrão amostral é, então, calculado por \(s=\sqrt{s^2}\).
\phantomsection\label{\detokenize{PE104-A:ativ-dadosagrupados}}
\begin{task}{ medidas para dados agrupados}

Os resultados obtidos na prova de seleção para vagas de estágio numa empresa estão representados no histograma a seguir.
\phantomsection\label{\detokenize{PE104-A:fig-hist-vagas-estagio}}
\begin{figure}[H]
\centering

\noindent\sphinxincludegraphics[width=150pt]{{exercicio9}.png}
\caption{Histograma das notas na prova de seleção para vagas de estágio}
\label{\detokenize{PE104-A:fig-hist-vagas-estagio}}\end{figure}

\begin{enumerate}
\item {} 
Com base neste histograma, calcule a média, a variância, a mediana, a moda, o primeiro quartil e o terceiro quartil.

\item {} 
Usando a informação do histograma, faça um esboço do boxplot destes dados.

\end{enumerate}
\end{task}



\subsection{Um método para a determinação dos quartis}

Existem métodos diferentes para determinar os quartis de um conjunto \(\{x_1,x_2,\cdots,x_n\}\) de \(n\) observações. Um método simples será descrito a seguir.

Tome \(\textsf{Q}1\) como o valor correspondente à posição \(\frac{n+1}{4}\) depois de ordenar os dados.

Tome \(\textsf{Q}2\) como a mediana do conjunto de dados, calculada pelo método apresentado para o cálculo da mediana.

Tome \(\textsf{Q}3\) como o valor correspondente à posição \(\frac{3n+1}{4}\) depois de ordenar os dados.

Se os resultados de  \(\frac{n+1}{4}\) e \(\frac{3n+1}{4}\) não forem números inteiros, arredonde-os para o inteiro mais próximo. Se a parte decimal do resultado destas operações for 0,5; calcule a média dos dois valores nas posições correspondentes. Por exemplo, suponha \(n=21\) tal que \((21+1)/4=5,5\). Assim, neste caso, para obter o primeiro quartil, calcule a média dos valores nas posições 5 e 6.

Vamos voltar aos dados da \DUrole{xref,std,std-ref}{ativ-Notas-de-Artes}. Como \(n=35\), para o primeiro quartil tomaremos o valor da posição \(\frac{35+1}{4}=9\), a saber, \(\textsf{Q}1=5\), já vimos que a mediana é 6,5 e, para o terceiro quartil tomaremos o valor da posição \(\frac{3\cdot 35+1}{4}=26,5\). Como 26,5 é equidistante das posições 26 e 27, tomaremos o terceiro quartil como a média dos dois valores nestas duas posições, a saber, \(\textsf{Q}3=\frac{7,3+7,5}{2}=7,4\). Logo, podemos dizer que na turma cerca de 25\% das notas foram menores do que 5 e cerca de 25\% das notas foram maiores do que 7,4.

\subsection{Soma dos desvios da média}

Considerando o conjunto \(\{ x_1,x_2,\cdots, x_n\}\) com \(n\) observações, seja \(\bar{x}\) a média deste conjunto.  Define-se como um \index{desvio da média}desvio da média, a diferença entre uma observação e a média, a saber,
\begin{equation*}
\begin{split}d_i=x_i-\bar{x}, \quad i=1,2,\cdots, n\end{split}
\end{equation*}
Uma propriedade dos desvios da média é dada por
\begin{equation*}
\begin{split}\sum^n_{i=1}d_i=\sum^n_{i=1}(x_i-\bar{x})=0,\end{split}
\end{equation*}
qualquer que seja o conjunto \(\{ x_1,x_2,\cdots, x_n\}\).

Demonstração:

\(\displaystyle{\sum^n_{i=1}} (x_i-\bar{x})=(x_1-\bar{x})+(x_2-\bar{x})+\cdots+(x_n-\bar{x})=\\ \underbrace{(x_1+x_2+\cdots +x_n)}_{=n\cdot \bar{x}} - n\cdot \bar{x}=0\)

lembrando que \(\bar{x}=\frac{x_1+x_2+\cdots+x_n}{n}\).

Veja um exemplo na seção \DUrole{xref,std,std-ref}{sub-desviosdamedia}.

\sphinxstylestrong{Fórmula para o cálculo da variância amostral}

Vimos que a variância amostral do conjunto de dados \(\{x_1,x_2,\cdots,x_n\}\) é definida por
\begin{equation*}
\begin{split}s^2 = \frac{1}{n-1}\cdot \sum^n_{i=1} (x_i-\bar{x})^2=\frac{(x_1-\bar{x})^2+(x_2-\bar{x})^2+\cdots+(x_n-\bar{x})^2}{n-1}\end{split}
\end{equation*}
De fato, é possível mostrar que
\begin{equation*}
\begin{split}s^2 = \frac{1}{n-1}\cdot \left (\sum^n_{i=1} x^2_i-n\cdot \bar{x}^2\right )\end{split}
\end{equation*}
Demonstração:  Expandindo a soma no numerador da fórmula da variância é possível concluir que
\begin{equation*}
\begin{split}\sum^n_{i=1}(x_i-\bar{x})^2= \sum^n_{i=1} x^2_i -n\cdot \bar{x}^2\end{split}
\end{equation*}
Lembre que \((x_i-\bar{x})^2=x^2_i-2\cdot \bar{x}\cdot x_i+\bar{x}^2\). Assim,
\begin{equation*}
\begin{split}\small {\sum^n_{i=1}(x_i-\bar{x})^2=\sum^n_{i=1}(x^2_i-2\cdot \bar{x}\cdot x_i+\bar{x}^2)=\\ (x^2_1-2\cdot\bar{x}\cdot x_1+\bar{x}^2)+(x^2_2-2\cdot\bar{x}\cdot x_2+\bar{x}^2)+ \cdots + (x^2_n-2\cdot\bar{x}\cdot x_n+\bar{x}^2)}\end{split}
\end{equation*}
Como a soma é finita, podemos reunir os termos semelhantes, obtendo
\begin{equation*}
\begin{split}\sum^n_{i=1}(x_i-\bar{x})^2= \\ (x^2_1+x^2_2+\cdots x^2_n)\underbrace{-2\cdot \bar{x}\cdot \overbrace{(x_1+x_2+\cdots+x_n)}^{=n\cdot \bar{x}}}_{=-2\cdot n\cdot \bar{x}^2}+n\cdot \bar{x}^2= \\ \sum^n_{i=1} x^2_i-n\cdot\bar{x}^2\end{split}
\end{equation*}
Vamos voltar aos dados da \DUrole{xref,std,std-ref}{ativ-Notas-de-Artes}. Temos \(n=35\), \(\displaystyle{\sum^{35}_{i=1}}x_i=207,5\) e \(\displaystyle{\sum^{35}_{i=1}}x^2_i=1361,39\)  tal que \(\bar{x}=\frac{207,5}{35}\approx 5,93\) e
\begin{equation*}
\begin{split}s^2=\frac{1}{34}\left ( 1361,39-35\cdot 5,93^2\right )\approx 3,8417\end{split}
\end{equation*}
tal que o desvio padrão amostral é, aproximadamente, 1,96.


\exercise

\label{\detokenize{PE104-E:sec-exercicos}}\label{\detokenize{PE104-E::doc}}\label{\detokenize{PE104-E:exercicios}}

\begin{enumerate}
\item Numa Escola de Ensino Médio os estudantes precisam fazer um exame no final do ano, se a média dos bimestres for inferior a 7. Um estudante de Ensino Médio desta Escola gostaria de saber em quantas disciplinas ele pode ser aprovado sem fazer exame final. No final do segundo bimestre ele obteve as notas registradas no quadro a seguir. Indique quanto deverão somar, no mínimo, as duas notas dos dois últimos bimestres para evitar o exame final, conclua quando isto ainda é possível.
\begin{quote}


\begin{savenotes}\sphinxattablestart
\centering
\begin{tabular}[t]{|c|c|c|c|c|}
\hline
Disciplina & $1^\circ$ & $2^\circ$  &  Soma mínima das notas &  Exemplo de notas possíveis\\
\hline
Língua portuguesa & 7 & 4 & &\\
\hline
Física & 5 & 4 & & \\
\hline
Matemática & 8 & 8 &&\\
\hline
História & 3 & 4 && \\
\hline
Geografia
&
5
&
5
&&
\\
\hline
Filosofia
&
7
&
9
&&\\
\hline
Educação Física
&
9
&
8
&&\\
\hline
Inglês
&
7
&
5
&&\\
\hline
Química
&
3
&
7
&&\\
\hline
Biologia
&
8
&
6
&&\\
\hline
\end{tabular}
\par
\sphinxattableend\end{savenotes}
\end{quote}

\item Suponha que o aluno do exercício anterior conseguiu evitar o exame final das disciplinas de Língua Portuguesa, Física e Biologia, por ter obtido média sete. As notas deste aluno dos quatro bimestres estão indicadas no quadro a seguir.
\begin{quote}


\begin{savenotes}\sphinxattablestart
\centering
\begin{tabulary}{\linewidth}[t]{|T|T|T|T|T|}
\hline
\sphinxstylethead{\sphinxstyletheadfamily 
Disciplina
\unskip}\relax &\sphinxstylethead{\sphinxstyletheadfamily 
1o.
\unskip}\relax &\sphinxstylethead{\sphinxstyletheadfamily 
2o.
\unskip}\relax &\sphinxstylethead{\sphinxstyletheadfamily 
3o.
\unskip}\relax &\sphinxstylethead{\sphinxstyletheadfamily 
4o.
\unskip}\relax \\
\hline
Língua portuguesa
&
7
&
4
&
8
&
9
\\
\hline
Física
&
5
&
4
&
9
&
10
\\
\hline
Biologia
&
8
&
6
&
7
&
7
\\
\hline
\end{tabulary}
\par
\sphinxattableend\end{savenotes}
\begin{enumerate}
\item {} 
Complete o quadro seguir com os desvios da média de cada disciplina.


\begin{savenotes}\sphinxattablestart
\centering
\begin{tabulary}{\linewidth}[t]{|T|T|T|T|T|T|}
\hline
\sphinxstylethead{\sphinxstyletheadfamily 
Disciplina
\unskip}\relax &\sphinxstartmulticolumn{4}%
\begin{varwidth}[t]{\sphinxcolwidth{4}{6}}
\sphinxstylethead{\sphinxstyletheadfamily Desvios da média
\unskip}\relax \par
\vskip-\baselineskip\vbox{\hbox{\strut}}\end{varwidth}%
\sphinxstopmulticolumn
&\sphinxstylethead{\sphinxstyletheadfamily 
Soma
\unskip}\relax \\
\hline
Língua portuguesa
&&&&&\\
\hline
Física
&&&&&\\
\hline
Biologia
&&&&&\\
\hline
\end{tabulary}
\par
\sphinxattableend\end{savenotes}

\item {} 
Em qual das disciplinas foi maior o desvio padrão das notas? E o menor?

\item {} 
Você acha que a mediana das notas seria um bom critério para a aprovação? Apresente exemplos para os quais a mediana das notas é 7 e a média é:
\begin{enumerate}
\item {} 
inferior a 7;

\item {} 
igual a 7;

\item {} 
superior a 7.

\end{enumerate}

\end{enumerate}
\end{quote}

\item (UFRJ - 2005 - adaptado)  A altura média de um grupo de 53 recrutas é 1,81 m. Sabe-se também que nem todos os recrutas do grupo têm a mesma altura. Diga se cada uma das afirmações a seguir é verdadeira, falsa ou se os dados são insuficientes para uma conclusão. Em cada caso, justifique a sua resposta.
\begin{enumerate}
\item {} 
“Há, no grupo em questão, pelo menos um recruta que mede mais de 1,81 m e pelo menos um que mede menos de 1,81 m.”

\item {} 
“Há, no grupo em questão, mais de um recuta que mede mais de 1,81 m e mais de um que mede menos de 1,81 m.”

\end{enumerate}

\item Seja \(\{x_1,x_2,\cdots x_n\}\) uma amostra de tamanho \(n\) de uma população, em que a média amostral é dada por \(\bar{x}\), o desvio padrão amostral é dado por \(s\) e o coeficiente de variação amostral é dado por \(\textsf{CV}=\frac{s}{\bar{x}}\cdot 100\) \%.

Defina um novo conjunto de dados \(\{y_1,y_2,\cdots y_n\}\) em que
\begin{equation*}
\begin{split}y_i=x_i+a,\quad  i=1,2,\cdots, n\end{split}
\end{equation*}
e \(a\) é um número real fixado, ou seja, o novo conjunto compreende todos os elementos do conjunto inicial acrescidos de uma constante \(a\) . Na \DUrole{xref,std,std-ref}{ativ-Notas-de-artes} essa transformação será realizada sobre o conjunto das notas, se o professor acrescentar 1,0 ponto às notas dos alunos da turma.
\begin{enumerate}
\item {} 
Em função da média do conjunto inicial, \(\bar{x}\), determine a média do novo conjunto.

\item {} 
Em função do desvio padrão do conjunto inicial, \(s\), determine o desvio padrão do novo conjunto.

\item {} 
Compare o coeficiente de variação do novo conjunto com o do conjunto inicial. São iguais? Por quê?

\end{enumerate}

\item Seja \(\{x_1,x_2,\cdots x_n\}\) uma amostra de tamanho \(n\) de uma população, em que a média amostral é dada por \(\bar{x}\), o desvio padrão amostral é dado por \(s\) e o coeficiente de variação amostral é dado por \(\textsf{CV}=\frac{s}{\bar{x}}\cdot 100\) \%. Defina um novo conjunto de dados \(\{y_1,y_2,\cdots y_n\}\) em que \(y_i=c\cdot x_i\), \(i=1,2,\cdots, n\) e \(c\) é um número real fixado e \(c>0\), ou seja, o novo conjunto compreende todos os elementos do conjunto inicial multiplicados por uma constante \(c>0.\) Na \DUrole{xref,std,std-ref}{ativ-Notas-de-artes} essa transformação será realizada sobre o conjunto das notas, se o professor aumentar em 20\% a nota de cada aluno, isto é, multiplicar cada nora pelo fator 1,2.
\begin{enumerate}
\item {} 
Em função da média do conjunto inicial, \(\bar{x}\), determine a média do novo conjunto.

\item {} 
Em função do desvio padrão do conjunto inicial, \(s\), determine o desvio padrão do novo conjunto.

\item {} 
Compare o coeficiente de variação do novo conjunto com o do conjunto inicial. Houve alguma alteração? Por quê?

\end{enumerate}

\item (ENEM 2015) Em uma seletiva para a final dos 100 metros livres de natação, numa olimpíada, os atletas, em suas respectivas raias, obtiveram os tempos no quadro a seguir. Escolha a opção que indica o valor da mediana dos tempos apresentados.
\begin{enumerate}
\item {} 
20,70 s.

\item {} 
20,77 s.

\item {} 
20,80 s.

\item {} 
20,85 s.

\item {} 
20,90 s.

\end{enumerate}


\begin{savenotes}\sphinxattablestart
\centering
\sphinxcapstartof{table}
\sphinxcaption{Tempos em segundos}\label{\detokenize{PE104-E:id4}}
\sphinxaftercaption
\begin{tabulary}{\linewidth}[t]{|T|T|T|T|T|T|T|T|T|}
\hline
\sphinxstylethead{\sphinxstyletheadfamily 
Raia
\unskip}\relax &\sphinxstylethead{\sphinxstyletheadfamily 
1
\unskip}\relax &\sphinxstylethead{\sphinxstyletheadfamily 
2
\unskip}\relax &\sphinxstylethead{\sphinxstyletheadfamily 
3
\unskip}\relax &\sphinxstylethead{\sphinxstyletheadfamily 
4
\unskip}\relax &\sphinxstylethead{\sphinxstyletheadfamily 
5
\unskip}\relax &\sphinxstylethead{\sphinxstyletheadfamily 
6
\unskip}\relax &\sphinxstylethead{\sphinxstyletheadfamily 
7
\unskip}\relax &\sphinxstylethead{\sphinxstyletheadfamily 
8
\unskip}\relax \\
\hline
Tempo (s)
&
20,90
&
20,90
&
20,50
&
20,80
&
20,60
&
20,60
&
20,90
&
20,96
\\
\hline
\end{tabulary}
\par
\sphinxattableend\end{savenotes}

\item (ENEM 2016-adaptado) Em uma cidade, o número de casos de dengue confirmados aumentou consideravelmente nos últimos dias. A prefeitura resolveu desenvolver uma ação, contratando funcionários para ajudar no combate à doença, os quais orientarão os moradores a eliminarem criadouros do mosquito Aedes aegypti, transmissor da dengue. A tabela a seguir apresenta o número atual de casos confirmados, por região da cidade.

A prefeitura optou pela seguinte quantidade de funcionários a serem contratados: (I) 10 funcionários para cada região da cidade cujo número de casos seja maior que a média dos casos confirmados e (II) 7 funcionários para cada região da cidade cujo número de casos seja menor ou igual à média dos casos confirmados. Quantos funcionários a prefeitura deverá contratar para efetivar a ação?
\begin{multicols}{5}
\begin{enumerate}
\item {} 
59

\item {} 
65

\item {} 
68

\item {} 
71

\item {} 
80
\end{enumerate}
\end{multicols}

\begin{savenotes}\sphinxattablestart
\centering
\sphinxcapstartof{table}
\sphinxcaption{Número atual de casos por região da cidade}\label{\detokenize{PE104-E:id5}}
\sphinxaftercaption
\begin{tabulary}{\linewidth}[t]{|T|T|}
\hline
\sphinxstylethead{\sphinxstyletheadfamily 
Região
\unskip}\relax &\sphinxstylethead{\sphinxstyletheadfamily 
Casos confirmados
\unskip}\relax \\
\hline
Oeste
&
237
\\
\hline
Centro
&
262
\\
\hline
Norte
&
158
\\
\hline
Sul
&
159
\\
\hline
Noroeste
&
160
\\
\hline
Leste
&
278
\\
\hline
Centro-Oeste
&
300
\\
\hline
Centro-Sul
&
278
\\
\hline
Soma
&
1.832
\\
\hline
\end{tabulary}
\par
\sphinxattableend\end{savenotes}

\item O \sphinxstyleemphasis{boxplot} a seguir representa a distribuição do número de gols da artilharia nas Copas do Mundo desde a Copa de 1930 até a Copa de 2006. Vamos chamar este número de \sphinxstylestrong{recorde}. Observe que só é considerado o \sphinxstylestrong{recorde}, sem levar em conta se houve mais de um artilheiro na Copa. Desse modo, nestas 18 Copas do Mundo, a figura leva em consideração os 18 \sphinxstylestrong{recordes} observados.
\begin{quote}
\phantomsection\label{\detokenize{PE104-E:fig-boxplotgols}}\begin{quote}

\begin{figure}[H]
\centering
\capstart

\noindent\sphinxincludegraphics[width=100bp]{{boxpltgols}.png}
\caption{Boxplot dos \sphinxstylestrong{recordes} das Copas do Mundo de 1936 a 2006.}\label{\detokenize{PE104-E:id6}}\end{figure}
\end{quote}
\end{quote}

Com base neste gráfico, as seguintes afirmações foram feitas a cerca da distribuição dos \sphinxstylestrong{recordes}  nestas Copas do Mundo.
\begin{enumerate}
\item {} 
A distribuição apresenta assimetria à direita.

\item {} 
A média dos \sphinxstylestrong{recordes} é maior do que a mediana dos \sphinxstylestrong{recordes}.

\item {} 
O boxplot não nos permite avaliar a existência de moda.

\item {} 
Uma aproximação grosseira para o valor do desvio padrão dos \sphinxstylestrong{recordes} nestas Copas é dada por 2,25 gols.

\item {} 
A distância entre quartis desta distribuição é 3 gols.

\item {} 
Esta distribuição não apresentou valores atípicos.

\item {} 
Uma aproximação para o valor da média dos \sphinxstylestrong{recordes} pode ser calculada por \(0,25\cdot (5+6,25+7,75+11)=7,5\) gols.

\end{enumerate}

Responda se concorda ou não com cada uma destas afirmações, justificando cada resposta.

\item Na questão anterior foram consideradas 18 Copas do Mundo. Sabe-se que a soma exata dos \sphinxstylestrong{recordes} destas Copas é dada por \(\displaystyle{\sum^{18}_{i=1}}x_i=132\) e que a soma dos quadrados dos \sphinxstylestrong{recordes} é dada por \(\displaystyle{\sum^{18}_{i=1}}x^2_i=1060\).
\begin{enumerate}
\item {} 
Com base nestas informações, calcule a média e o desvio padrão dos \sphinxstylestrong{recordes} e compare com as aproximações obtidas no exercício anterior.

\item {} 
Consultando os \sphinxstylestrong{recordes} referentes às Copas de 2010 e 2014, verificou-se que eles foram 5 e 6, respectivamente. Determine a média e o desvio padrão dos \sphinxstylestrong{recordes}, considerando as 20 Copas do Mundo até 2014.

\end{enumerate}

\item (ENEM-2010) O quadro seguinte mostra o desempenho de um time de futebol no último campeonato. A coluna da esquerda mostra o número de gols marcados e a coluna da direita informa em quantos jogos o time marcou aquele número de gols.


\begin{savenotes}\sphinxattablestart
\centering
\sphinxcapstartof{table}
\sphinxcaption{Desempenho de um time}\label{\detokenize{PE104-E:id7}}
\sphinxaftercaption
\begin{tabulary}{\linewidth}[t]{|T|T|}
\hline

Gols marcados
&
Quantidade de partidas
\\
\hline
0
&
5
\\
\hline
1
&
3
\\
\hline
2
&
4
\\
\hline
3
&
3
\\
\hline
4
&
2
\\
\hline
5
&
2
\\
\hline
7
&
1
\\
\hline
\end{tabulary}
\par
\sphinxattableend\end{savenotes}

Se X, Y e Z são, respectivamente, a média, a mediana e a moda desta distribuição, então:
\begin{enumerate}
\item {} 
X = Y \textless{} Z                 b) Z \textless{} X = Y    c) Y \textless{} Z \textless{} X            d) Z \textless{} X \textless{} Y    e) Z \textless{} Y \textless{} X

\end{enumerate}

\item Um professor de Matemática suspeita que seus alunos do turno da tarde são mais fracos do que os seus alunos do turno da manhã. Para verificar sua suspeita, logo no início do ano letivo ele aplicou um teste básico de questões envolvendo conteúdos básicos e esperados para o nível a ser iniciado em duas amostras, uma de alunos do turno da manhã e outra de alunos do turno da tarde. A seguir, estão os resultados para as duas amostras.

\begin{minipage}{.45\textwidth}
\begin{savenotes}\sphinxattablestart
\centering
\sphinxcapstartof{table}
\sphinxcaption{Notas de uma amostra de alunos do turno da manhã}\label{\detokenize{PE104-E:id8}}
\sphinxaftercaption
\begin{tabulary}{\linewidth}[t]{|T|T|T|T|T|}
\hline

7,4
&
7,3
&
6,2
&
6,3
&
4,1
\\
\hline
5,7
&
10,0
&
6,2
&
4,9
&
6,0
\\
\hline
8,7
&
6,5
&
3,0
&
5,8
&
7,0
\\
\hline
8,0
&
8,0
&
4,9
&
7,4
&
6,8
\\
\hline
6,7
&
7,6
&
6,1
&
6,2
&
8,5
\\
\hline
7,4
&
4,4
&
8,1
&
5,8
&
6,6
\\
\hline
4,2
&
5,3
&
4,9
&
8,1
&
6,8
\\
\hline
6,8
&
4,4
&
5,4
&
7,1
&
6,1
\\
\hline
5,3
&
5,2
&
5,7
&
9,9
&
8,3
\\
\hline
\end{tabulary}
\par
\sphinxattableend\end{savenotes}
\end{minipage}\hfill\begin{minipage}{.45\textwidth}

\begin{savenotes}\sphinxattablestart
\centering
\sphinxcapstartof{table}
\sphinxcaption{Notas de uma amostra de alunos do turno da tarde}\label{\detokenize{PE104-E:id9}}
\sphinxaftercaption
\begin{tabulary}{\linewidth}[t]{|T|T|T|T|T|}
\hline

5,1
&
4,7
&
5,7
&
4,7
&
5,0
\\
\hline
4,2
&
4,9
&
6,0
&
4,4
&
4,4
\\
\hline
6,0
&
4,9
&
5,6
&
6,2
&
6,6
\\
\hline
6,2
&
4,7
&
6,0
&
4,6
&
3,6
\\
\hline
5,4
&
5,2
&
5,6
&
5,5
&
5,2
\\
\hline
5,8
&
4,5
&
5,0
&
3,8
&
4,6
\\
\hline
4,1
&
4,7
&
4,2
&
6,8
&
5,6
\\
\hline
5,3
&
4,5
&
4,7
&
5,1
&
5,2
\\
\hline
\end{tabulary}
\par
\sphinxattableend\end{savenotes}
\end{minipage}

Usando todas as ferramentas estudadas neste capítulo, ajude este professor, fazendo um relatório detalhado e comparativo sobre os dois turnos. Se preferir, você poderá baixar estes dados no \sphinxstylestrong{link}, mas lembre-se que como eles estão registrados no GeoGebra, a vírgula foi trocada por ponto.

\item Quando comparou-se a média com a mediana falou-se em grau de assimetria da distribuição ({\hyperref[\detokenize{PE104-1:sec-organizando1}]{\sphinxcrossref{\DUrole{std,std-ref}{Organizando as ideias: medidas de posição}}}}). Na seção {\hyperref[\detokenize{PE104-A:sec-para-saber-mais}]{\sphinxcrossref{\DUrole{std,std-ref}{Para saber mais}}}} falou-se novamente em grau de assimetria. A assimetria pode ser medida pelo \sphinxstylestrong{índice de assimetria de Pearson}
\begin{equation*}
\begin{split}I=\frac{3\cdot(\bar{x}-\textsf{mediana})}{s}\end{split}
\end{equation*}
Se \(I\approx 0\), os dados são considerados aproximadamente simétricos. Um valor de \(I\) negativo, indica assimetria à esquerda e, um valor de \(I\) positivo, assimetria à direita.

Se \(I\geq 1,00\) ou \(I\leq -1,00\) , os dados podem ser considerados fortemente assimétricos à direita ou à esquerda, respectivamente. Calcule o índice de assimetria de Pearson, para os dados de
\begin{enumerate}
\item {} 
\DUrole{xref,std,std-ref}{ativ-Notas-de-Artes};

\item {} 
{\hyperref[\detokenize{PE104-7:ativ-comparacaodegruposusandoboxplot}]{\sphinxcrossref{\DUrole{std,std-ref}{Atividade: modalidades da maratona de Nova Iorque 2017}}}};

\item {} 
exercício 10.

\end{enumerate}

\item Em provas aplicadas em grande escala é comum divulgar as notas transformadas da seguinte forma
\begin{equation*}
\begin{split}y_i = 500+100\cdot \frac{(x_i-\bar{x})}{s}, \quad i=1,2,...,n\end{split}
\end{equation*}
em que \(x_i\) é a nota obtida pelo \(i\)-ésimo candidato, \(\bar{x}=\frac{1}{n}\displaystyle{\sum^n_{i=1}}x_i\) , \(s\) é o desvio padrão amostral das notas do conjunto \(\{ x_1,x_2, ..., x_n\}\) e \(y_i\) é a nota transformada do \(i\)-ésimo candidato.

Considere as afirmações a seguir.
\begin{enumerate}
\item {} 
A média das notas transformadas é 500.

\item {} 
O desvio padrão das notas transformadas é 100.

\item {} 
Se a distribuição de notas é aproximadamente simétrica e com poucas notas atípicas, cerca de 67\% dos candidatos obtiveram notas transformadas entre 400 e 600.

\item {} 
Se a distribuição de notas é aproximadamente simétrica e com poucas notas atípicas, cerca de 95\% dos candidatos obtiveram notas transformadas entre 300 e 700.

\end{enumerate}

Responda se concorda ou não com cada uma destas afirmações, justificando cada resposta.

\item (Dados trabalhados na Atividade “Comparação de Medicamentos” no Capítulo \sphinxstylestrong{A Natureza da Estatística})

Deseja-se comparar três medicamentos, X, Y e Z, no tratamento da dor de cabeça. Para isso 60 pacientes com perfis similares foram separados aleatoriamente em três grupos de 20 cada. Para cada grupo,  será ministrado um dos medicamentos e observado o tempo de cura da dor de cabeça (em minutos). No quadro a seguir estão dispostos os dados obtidos.
\begin{quote}


\begin{savenotes}\sphinxattablestart
\centering
\sphinxcapstartof{table}
\sphinxcaption{Dados do tempo de cura (em minutos) para os medicamentos X, Y e Z}\label{\detokenize{PE104-E:tabela-medicamentos}}\label{\detokenize{PE104-E:id10}}
\sphinxaftercaption
\begin{tabulary}{\linewidth}[t]{|T|T|T|T|}
\hline

dados ordenados
&
X
&
Y
&
Z
\\
\hline
1
&
7
&
7
&
11
\\
\hline
2
&
8
&
8
&
11
\\
\hline
3
&
8
&
9
&
11
\\
\hline
4
&
9
&
9
&
11
\\
\hline
5
&
9
&
10
&
11
\\
\hline
6
&
9
&
10
&
12
\\
\hline
7
&
9
&
11
&
12
\\
\hline
8
&
10
&
11
&
12
\\
\hline
9
&
10
&
11
&
12
\\
\hline
10
&
10
&
12
&
12
\\
\hline
11
&
10
&
12
&
12
\\
\hline
12
&
10
&
12
&
12
\\
\hline
13
&
10
&
13
&
12
\\
\hline
14
&
11
&
13
&
12
\\
\hline
15
&
11
&
14
&
12
\\
\hline
16
&
11
&
14
&
13
\\
\hline
17
&
11
&
15
&
13
\\
\hline
18
&
12
&
15
&
13
\\
\hline
19
&
12
&
16
&
13
\\
\hline
20
&
13
&
18
&
13
\\
\hline
soma simples
&
200
&
240
&
240
\\
\hline
soma de quadrados
&
2042
&
3030
&
2890
\\
\hline
\end{tabulary}
\par
\sphinxattableend\end{savenotes}
\begin{enumerate}
\item {} 
Complete o quadro a seguir.

\end{enumerate}


\begin{savenotes}\sphinxattablestart
\centering
\begin{tabulary}{\linewidth}[t]{|T|T|T|T|}
\hline

medida
&
X
&
Y
&
Z
\\
\hline
média
&&&\\
\hline
moda
&&&\\
\hline
s
&&&\\
\hline
Min
&&&\\
\hline
Q1
&&&\\
\hline
mediana
&&&\\
\hline
Q3
&&&\\
\hline
Max
&&&\\
\hline
\end{tabulary}
\par
\sphinxattableend\end{savenotes}
\begin{enumerate}
\setcounter{enumi}{1}
\item {} 
Construa os boxplots para os três conjuntos de dados.

\item {} 
Como você avalia a forma das distribuições quanto à assimetria? Por quê?

\item {} 
Com base nas informações obtidas, que medicamento você escolheria? Por quê?

\end{enumerate}
\end{quote}
\end{enumerate}





\section{Material Suplementar}
\label{\detokenize{PE104-E:material-suplementar}}\label{\detokenize{PE104-E:sec-applet-medidas}}
Como material de suporte para este capítulo foi desenhado um aplicativo interativo de Geogebra para a visualização de medidas de posição e dispersão de uma distribuição, que pode ser encontrado \sphinxhref{https://ggbm.at/KbYqnQ6Q}{aqui}. O aplicativo pode ser usado diretamente no explorador de internet de sua preferência ou baixado e usado em computadores e celulares com \sphinxhref{https://www.geogebra.org/}{Geogebra} instalado.

O aplicativo gera dados de forma aleatória, mas você pode inserir seus próprios dados na primeira coluna da planilha e verá o histograma correspondente na área gráfica, escolhendo a quantidade de partições do intervalo que você deseja.

O aplicativo permite visualizar, além do histograma, as medidas de posição da distribuição além das medidas de dispersão, mostrando: mínimo, máximo, média, mediana, Q1, Q3, variância e desvio padrão amostrais e populacionais.

Finalmente, é possível construir o boxplot na mesma área gráfica para que o estudante se familiarize visualmente com a relação entre o histograma e o boxplot.

\begin{figure}[H]
\centering
\capstart

\noindent\sphinxincludegraphics[width=400bp]{{Aplicativo_Medidas}.png}
\caption{\sphinxhref{https://ggbm.at/KbYqnQ6Q}{Aplicativo interativo em Geogebra para a visualização de medidas de posição e dispersão de uma distribuição}}\label{\detokenize{PE104-E:fig-aplicativo-medidas}}\label{\detokenize{PE104-E:id17}}\end{figure}



% \ifnum\aluno=1
\renewcommand\chapterillustration{abertura-probabilidade.jpg}
\else
\renewcommand\chapterillustration{abertura-probabilidade-professor.jpg}
\fi
\renewcommand\chapterwhat{Reconhecimento de fenômenos determinísticos e aleatórios. Interpretações de probabilidade: clássica, frequentista e subjetiva. Conceitos básicos. Definição matemática deprobabilidade.
Propriedades da probabilidade. Probabilidade condicional. Inde-
pendência.}
\renewcommand\chapterbecause{Porque grande parte das decisões científicas de nossa era se dá em ambiente de incerteza e a Teoria das Probabilidades é a área da matemática que fornece estruturas para a quantificação da aleatoriedade associada a determinados fenômenos de interesse, para uma tomada de decisão adequada sob incerteza.}

\makeatletter
\ifnum\aluno=1
\else
\renewcommand*{\toclevel@section}{1}
\renewcommand*{\toclevel@subsection}{4}
\renewcommand*{\toclevel@paragraph}{5}
\renewcommand*{\toclevel@subparagraph}{6}

\renewcommand*{\toclevel@exploresec}{2}
\renewcommand*{\toclevel@practicesec}{2}
\renewcommand*{\toclevel@arrangesec}{2}
\renewcommand*{\toclevel@knowsec}{1}
\renewcommand*{\toclevel@exercisesec}{1}

\setcounter{tocdepth}{2}
\fi
\makeatother

\chapter{Probabilidade}
\ifdefined\estchapum
\else
\label{est1-chap}
\fi

\ifdefined\estchapdois
\else
\label{est2-chap}
\fi


\mbox{}\thispagestyle{empty}\clearpage

\thispagestyle{empty}

\begin{center}
Projeto: LIVRO ABERTO DE MATEMÁTICA

\noindent \begin{tabular}{lcccr}
\includegraphics[scale=.15]{impa}& \quad\quad& \includegraphics[width=3cm]{logo} & \quad\quad& \includegraphics[scale=.24]{obmep} 
\end{tabular}
\end{center}

\vspace*{.3cm}

Cadastre-se como colaborador no site do projeto: \url{umlivroaberto.org}

Versão digital do capítulo:

\url{https://www.umlivroaberto.org/BookCloud/Volume_1/master/view/PE511.html}

% \begin{center}
%   \includegraphics[width=2cm]{canvas}
% \end{center}

\begin{tabular}{p{.15\textwidth}p{.7\textwidth}}
Título: & Probabilidade\\
\\
Ano/ Versão: & 2020 / versão 1.2 de \today\\
\\
Editora & Instituto Nacional de Matem\'atica Pura e Aplicada (IMPA-OS)\\
\\
Realização:& Olimp\'iada Brasileira de Matem\'atica das Escolas P\'ublicas (OBMEP)\\
\\
Produção:& Associação Livro Aberto\\
\\
Coordenação: & Fabio Simas e Augusto Teixeira (livroaberto@impa.br)\\
\\
  Autores: & Flávia Landim (coordenadora da equipe - UFRJ),\\
        & Alexandre Silva (UNIRIO),\\
        & Nei Rocha (UFRJ),\\
             & Vanessa Matos (SEduc Angras dos Reis e Mesquita).\\
\\
Revisora: &  Cydara Ripoll  \\
\\
Design: & Andreza Moreira (Tangentes Design) \\
\\
  Ilustrações: & Miller  Guglielmo \\ 
\\
Gráficos: & Beatriz Cabral e Tarso Caldas (Licenciandos da UNIRIO)\\
\\
  Capa: & Foto de Karsten Winegeart, no Unsplash \\
        & https://unsplash.com/photos/EI-iNlUGfzI \\

\end{tabular}

\begin{figure}[b]
\begin{minipage}[l]{5cm}
\centering

{\large Licença:}

  \includegraphics[width=3.5cm]{cc-by-sa1}
\end{minipage}\hfill
\begin{minipage}[c]{5cm}
\centering
{\large Desenvolvido por}

\includegraphics[width=2.5cm]{logo-associacao.jpg}
\end{minipage}
\begin{minipage}[r]{5cm}
\centering

{\large Patrocínio:}
  \vspace{1em}
  \includegraphics[width=3.5cm]{itau}
\end{minipage}
\end{figure}

\mainmatter

\begin{apresentacao}{Probabilidade}
Previsão de de alunas necessárias: 8 a 10 tempos de aula

\subsection{Objetivo Geral}
Apresentar a noção de probabilidade sob os pontos de vista clássico, frequentista e subjetivo. (A noção frequentista de probabilidade está prevista para o Ensino Fundamental pela BNCC.) Apresentar regras básicas (axiomas) da probabilidade independentes da interpretação adotada, introduzindo uma teoria matemática da probabilidade e, como consequência, trabalhar propriedades decorrentes das regras básicas, definir probabilidade condicional e independência entre dois eventos. Finalmente, trabalhar com o cálculo de probabilidades de eventos em experimentos sucessivos, usando a regra da multiplicação, obtida a partir da definição de probabilidade condicional.

Ao longo do capítulo serão propostas atividades nas quais o aluno deverá propor como especificar probabilidades para determinados eventos em situações nas quais não se aplica a definição clássica de probabilidade.

\paragraph{Habilidades da BNCC}
\begin{habilities}{EM13MAT311}
Identificar e descrever o espaço amostral de eventos aleatórios, realizando contagem das possibilidades, para resolver e elaborar problemas que envolvem o cálculo da probabilidade.
\columnbreak

\tcbsubtitle{EM13MAT312}
Resolver e elaborar problemas que envolvem o cálculo de probabilidade de eventos em experimentos aleatórios sucessivos.

\tcbsubtitle{EM13MAT511}
Reconhecer a existência de diferentes tipos de espaços amostrais, discretos ou não, e de eventos, equiprováveis ou não, e investigar implicações no cálculo de probabilidades.
\end{habilities}

\paragraph{Por que estudar o assunto?}
“A Probabilidade fornece uma maneira única de se envolver, pensar e interagir com uma gama diversa de situações do mundo real, no qual aleatoriedade e incerteza estão onipresentes.”{} \citep{budgett2016}

\paragraph{Desfios do capítulo}
Apresentar o conteúdo de Probabilidade antes do conteúdo de Análise Combinatória, relacionando-o mais à Estatística do que à noção clássica de probabilidade.

\paragraph{Conteúdos abordados}
\begin{enumerate}
\item Interpretações clássica, frequentista e subjetiva da probabilidade.
\item Lei dos Grandes Números.
\item Definição axiomática da probabilidade. (Regras básicas da Probabilidade)
\item Propriedades da probabilidade.
\item Probabilidade condicional, Regra da Multiplicação, Eventos Independentes.
\item Aplicações da Lei dos Grandes Números, usando simulações no GeoGebra.(Para saber mais)
\end{enumerate}



\paragraph{Pré-requisitos}
\begin{habilities}{EF09MA19}
Reconhecer, em experimentos aleatórios, eventos independentes e dependentes, e calcular a probabilidade de sua ocorrência nos dois casos.

\tcbsubtitle{EF08MA19}
Calcular a probabilidade de eventos com base na construção do espaço amostral, utilizando o princípio multiplicativo, e reconhecer que a soma das probabilidades de todos os elementos do espaço amostral é igual a 1.

\tcbsubtitle{EF07MA28}
Planejar e realizar experimentos aleatórios ou simulações que envolvem cálculo de probabilidades ou estimativas por meio de frequência de ocorrências.
\end{habilities}

\textbf{Observação}: Como a BNCC ainda não entrou em vigor, os pré-requisitos acima, não necessariamente foram contemplados no Ensino Fundamental. Por essa razão, eles serão abordados neste capítulo.

\paragraph{Desdobramentos imediatos}

\begin{habilities}{EM13CNT205}
Utilizar noções de probabilidade e incerteza para interpretar previsões sobre atividades experimentais, fenômenos naturais e processos tecnológicos, reconhecendo os limites explicativos das ciências.

\tcbsubtitle{EM13MAT106}
Identificar situações da vida cotidiana nas quais seja necessário fazer escolhas levando-se em conta os riscos probabilísticos (usar este ou aquele método contraceptivo, optar por um tratamento médico em detrimento de outro etc.)

\tcbsubtitle{EM13MAT203}
Aplicar conceitos matemáticos no planejamento, na execução e na análise de ações envolvendo a utilização de aplicativos e a criação de planilhas (para o controle de orçamento familiar, simuladores de cálculos de juros simples e compostos, entre outros), para tomar decisões.
\end{habilities}

\subsection{Abordagem do capítulo}
Neste capítulo serão apresentadas várias atividades iniciais, solicitando ao aluno determinar a probabilidade de ocorrer um certo evento, para posterior discussão e introdução a diferentes tipos de interpretação de Probabilidade. Na discussão sobre a “definição”{} de probabilidade pretende-se deixar claro que a expressão $P(A)=\frac{\#(A)}{\#(S)}$ em que $A$ é um evento, $S$ é o espaço amostral, $\#(A)$ é o número de elementos do evento $A$ e $\#(S)$ é o número de elementos do espaço amostral $S$, é restrita a um contexto particular e que as noções frequentista e subjetiva levam a designações da probabilidade de um evento $A$ ocorrer que não fazem uso desta expressão.

A partir desta discussão, as regras básicas (axiomas) da probabilidade são apresentadas para introduzir uma teoria matemática da probabilidade que independe da interpretação adotada nas designações de probabilidade.

Na BNCC do Ensino Fundamental, a noção frequentista de probabilidade está prevista.

Na sequência serão propostas atividades que demandarão o uso das propriedades da probabilidade tais como a probabilidade de um evento complementar, a verificação de que a probabilidade é uma função não-decrescente no sentido de que se $A\subset B$, então $P(A)\leq P(B)$ e a probabilidade da união de dois eventos quaisquer.

Para trabalhar a probabilidade de eventos em experimentos sucessivos (\textbf{EM13MAT312}), primeiro será proposta uma atividade para calcular probabilidades condicionais e explorar a noção de independência entre eventos. Em seguida, a definição de probabilidade condicional será apresentada, levando à regra da multiplicação, a ser explorada em atividades da seção praticando subsequente.

\paragraph{Diferencial do capítulo}
Dar menos ênfase à noção clássica de Probabilidade, apresentando situações nas quais esta noção não se aplica.

\paragraph{Dificuldades típicas dos estudantes (distratores)}
\begin{enumerate}
\item Dificuldade de interpretar “probabilidade”.
\item Existe forte inclinação dos estudantes a assumir que os resultados em um espaço amostral são equiprováveis mesmo quando não é apropriado fazer esta suposição \citep{albert2003}
\item Confusão entre os conceitos de eventos independentes e eventos disjuntos \citep{rodrighes1984}.
\end{enumerate}

\paragraph{Estratégia pedagógica}
Pretende-se ao longo do capítulo propor atividades em variados contextos que estimulem

\begin{enumerate}
\item a compreensão do pensamento probabilístico;
\item a interpretação de resultados de pesquisas estatísticas baseadas em amostras aleatórias;
\item a necessidade de propor um modelo para explicar um fenômeno aleatório.
\end{enumerate}

\subsection{Estrutura}
Como nos demais capítulos do livro, todas as seçções são compostas pelas subseções explorando (atividades), organizando as ideias (apresentação do conteúdo tratado nas atividades) e praticando (atividades de exploração dos conteúdos trabalhados na seção).

\paragraph{Seção 1: Conceitos básicos}
Nesta seção incluem-se
\begin{enumerate}
\item breve histórico do desenvolvimento da teoria das probabilidades,
\item os conceitos de fenômenos determinísticos e não determinísticos (aleatórios),
\item a interpretação de uma medida de incerteza,
\item conceitos básicos como espaço amostral e evento;
\item revisão de operações entre conjuntos (união, interseção e complementariedade),
\item interpretações clássica, frequentista e subjetiva da probabilidade.
\end{enumerate}

\textbf{Objetivos específicos da seção 1:}
\begin{OES}
\item Aleatoriedade - Reconhecer fenômenos aleatórios, distinguindo-os de fenômenos determinísticos.
\item Probabilidade - Reconhecer os conceitos de espaço amostral e evento.
\item Medida de incerteza - Reconhecer que toda probabilidade se traduz como uma taxa de ocorrência de um evento.
\item Interpretações da probabilidade - Reconhecer diferentes interpretações da probabilidade (clássica, frequentista e subjetiva).
\item Álgebra de conjuntos - Aplicar as operações de união, interseção e complementariedade da teoria de conjuntos para descrever eventos compostos.
\end{OES}

\paragraph{Seção 2: Regras básicas e propriedades da probabilidade}

Nesta seção incluem-se
\begin{enumerate}
\item os axiomas da probabilidade, chamados no livro de regras básicas da probabilidade,
\item uma discussão sobre a validade das regras básicas sob cada interpretação,
\item propriedades da probabilidade decorrentes das regras básicas,
\item a definição de probabilidade geométrica.
\end{enumerate}

\textbf{Objetivos específicos da seção 2:}
\begin{OES}\setcounter{enumi}{5}
\item Definição axiomática - Aplicar as regras básicas da probabilidade nas diferentes interpretações de probabilidade.
\item  Ddefinição axiomática - Aplicar as regras básicas da probabilidade para obter as regras da probabilidade do evento complementar e a probabilidade da união de dois eventos (propriedades da probabilidade).
\item  Modelagem probabilística - Avaliar, ainda que de forma incipiente, modelos para fenômenos aleatórios.
\item  Aplicação - Aplicar as regras básicas da probabilidade em espaços amostrais contínuos (probabilidade geométrica).
\end{OES}

\paragraph{Seção 3: Probabilidade condicional e independência}
Nesta seção incluem-se
\begin{enumerate}
\item a definição de probabilidade condicional;
\item a regra da multiplicação para a probabilidade da ocorr~encia simultânea de dois eventos;
\item a extensão da regra da multiplicação para a ocorrência simultânea de mais de dois eventos;
\item a definição de independência entre dois eventos e sua extensão para mais de dois eventos.
\end{enumerate}

\textbf{Objetivos específicos da seção 3:}
\begin{OES}\setcounter{enumi}{9}
\item Probabilidade condicional - Reconhecer que a probabilidade de um evento pode ser alterar dada a ocorrência prévia de outro evento
\item Independência de eventos - Aplicar a definição de probabilidade condicional para reconhecer eventos independentes e eventos dependentes.
\item Independência de eventos - Aplicar a definição de probabilidade condicional no cálculo da probabilidade da interseção de dois eventos quaisquer.
\item Eventos sequenciais - Entender que a probabilidade da ocorrência simultânea de um número finito de eventos pode ser calculada como o produto de probabilidades adequadas.
\end{OES}

\paragraph{Para saber mais}
Simulações de experimentos simples usando o GeoGebra. Ilustrações da Lei dos Grandes Números.

\paragraph{Exercícios}
Serão propostos exercícios do ENEM, Vestibulares entre outros abordando os conteúdos deste capítulo. Nos exercícios serão tratados os distratores.

\end{apresentacao}

\def\currentcolor{session1}
\begin{paginatexto}{Conceitos Básicos}{

  Como na nossa vida quase tudo é incerto, o conhecimento sobre a quantidade de incerteza associada a certos eventos (probabilidade) é fundamental para tomar decisões adequadas.

  A abordagem tradicional restringe o ensino de probabilidade à interpretação clássica de probabilidade, definindo-a como a razão de número de casos favoráveis sobre número de casos possíveis, pois nesta interpretação eventos elementares (subconjuntos unitários) do espaço amostral (conjunto de todos os resultados possíveis em um experimento aleatório). Um experimento aleatório é um processo cujo resultado final só é conhecido após a sua realização) são considerados equiprováveis. No entanto, a maior parte dos problemas que envolvem modelagem de fenômenos aleatórios não se encaixa na interpretação clássica. Por exemplo,

  problemas nos quais o espaço amostral é finito, mas os eventos elementares não são equiprováveis, por exemplo, uma moeda é viciada de tal modo que a probabilidade de resultar em “cara”{} é $0{,}7$ e de resultar em “coroa”{} é $0{,}3$. Se essa moeda for lançada e estamos interessados na face da moeda voltada para cima, o espaço amostral terá dois elementos, a saber, “cara”{} e “coroa”. No entanto, nesse caso, não é adequado atribuir probabilidades iguais para os dois tipos de face.

  problemas nos quais o espaço amostral não é finito. Considere o experimento que consiste em observar o tempo de vida de uma lâmpada. Nesse caso o espaço amostral está contido na semirreta não-negativa, $[0,\infty)$.

  Apesar de a Análise Combinatória ser muito relacionada à Probabilidade, de fato, ela é meramente uma ferramenta útil no cálculo de probabilidades em contextos nos quais a interpretação clássica de probabilidade se encaixa. Por essa razão optamos por não incluí-la como pré-requisito na abordagem de probabilidade dessa unidade temática. É claro que, quando for se trabalhar com a Análise Combinatória, é importante reforçá-la em problemas que envolvam cálculo de probabilidades. Acreditamos que a estratégia adotada nesta unidade para abordar a Probabilidade será útil para desmistificá-la como um tópico da Matemática muito difícil.

  Na unidade temática de probabilidade pretende-se reforçar a importância da probabilidade à estatística. Na estatística, em geral, adotam-se as interpretações frequentista e subjetiva de probabilidade. Por isso o objetivo principal desta seção inicial da unidade é a compreensão das interpretações de probabilidade mais utilizadas, a saber, clássica, frequentista e subjetiva.

  Outro objetivo nesta seção é a interpretação de uma medida de incerteza (probabilidade) como uma taxa de ocorrência de um evento, ou seja, como uma média de ocorrências desse evento, quando repetimos o experimento um grande número de vezes. Por exemplo, uma probabilidade $0{,}95$ para a ocorrência de um evento, não implica necessariamente que esse evento ocorrerá. Essa probabilidade é interpretada da seguinte forma: se o experimento for repetido $100$ vezes sob as mesmas condições, espera-se que em $95$ delas o evento ocorra.

  Nesta seção serão apresentados os conceitos básicos de espaço amostral e evento, adequados para esse nível do ensino, ou seja, o espaço amostral como a coleção contendo todos os resultados possíveis e o evento como um subconjunto do espaço amostral. A seção é encerrada apresentando-se uma breve revisão de operações com conjuntos (união, interseção e complementariedade), pois como a probabilidade é uma função na qual seu domínio é um conjunto cujos elementos são conjuntos (eventos), conhecer as operações básicas de união, interseção e complementariedade entre conjuntos (eventos) serão necessárias. (Entendemos que a definição formal de medida de probabilidade não é necessária para esse nível do ensino de modo que considerações sobre sigmas-álgebras, necessárias para essa definição formal, não serão consideradas. Sugerimos o texto de Marcos N. Magalhães, Probabilidade e Variáveis Aleatórias, da EDUSP e o texto do Barry James, Probabilidade: um curso em nível intermediário, Projeto Euclides, IMPA, para maiores detalhes.)

  São objetivos específicos da seção 1:

  \begin{OES}
  \item Aleatoriedade - Reconhecer fenômenos aleatórios, distinguindo-os de fenômenos determinísticos.
  \item Probabilidade - Reconhecer os conceitos de espaço amostral e evento.
  \item Medida de incerteza - Reconhecer que toda probabilidade se traduz como uma taxa de ocorrência de um evento.
  \item Interpretações da probabilidade] - Reconhecer diferentes interpretações da probabilidade (clássica, frequentista e subjetiva).
  \item Álgebra de conjuntos - Aplicar as operações de união, interseção e complementariedade da teoria de conjuntos para descrever eventos compostos.
  \end{OES}

  \textbf{Observação}: Embora esta seção seja mais conceitual, a atividade que envolve a introdução de diferentes interpretações de probabilidade demanda calcular probabilidades. No entanto, trata-se de uma atividade cujas respostas, algumas delas abertas, não são o mais importante. Pretende-se nesta atividade apresentar situações que levam às diferentes interpretações de probabilidade. Lembramos que a noção de probabilidade é intuitiva e já é trabalhada no Ensino Fundamental II. Com a implantação da BNCC do Ensino Fundamental, probabilidade será trabalhada desde os anos iniciais do Ensino Fundamental.
}
\end{paginatexto}
\explore{Conceitos Básicos}\label{conceitosbasicos}
Neste capítulo iremos explorar a noção de probabilidade para, em seguida, apresentar uma teoria matemática útil para calcular probabilidades de eventos associados a experimentos aleatórios ou fenômenos aleatórios, isto é, experimentos cujos resultados finais são conhecidos somente após a realização dos mesmos. Por exemplo,

\begin{enumerate}
\item {} 
o número de \textit{likes} que você irá receber no período de 24h após a sua postagem em uma rede social;

\item {} 
a quantidade de metros cúbicos de gás consumida na sua residência no primeiro semestre do próximo ano;

\item {} 
o tempo, contado a partir de hoje, que a lâmpada do seu quarto levará para queimar;

\item {} 
o número de quilowatts consumidos na sua residência no próximo mês.
\end{enumerate}
Em contraposição aos fenômenos aleatórios existem os \index{fenômenos determinísticos}fenômenos determinísticos, quando é possível determinar seu resultado mesmo antes de realizá-lo, conhecendo-se determinadas condições. Na natureza existem muitos exemplos de experimentos determinísticos. Por exemplo, na Física há vários modelos determinísticos, como

\begin{enumerate}\setcounter{enumi}{4}
\item {} 
a primeira lei de Newton que estabelece a força, conhecendo-se massa e aceleração;

\item {} 
a lei do movimento retilíneo uniforme em que é possível calcular a distância percorrida pelo móvel, conhecendo-se a velocidade e tempo transcorrido;

\item {} 
a lei do movimento uniformemente variado em que é possível calcular a distância percorrida pelo móvel, conhecendo-se a aceleração, a velocidade e o tempo transcorrido.

\item {} 
a lei da gravitação universal em que é possível calcular o tempo de queda de um objeto que é lançado em queda livre, conhecendo-se a altura, a aceleração da gravidade, desprezando a resistência do ar.

\end{enumerate}

Tais modelos da Física são chamados modelos matemáticos determinísticos, uma vez que é possível determinar quantidades de interesse, conhecendo-se certas condições, mesmo sem a realização do experimento.

Para explicar fenômenos aleatórios como exemplificados nos itens \titem{a)} a \titem{d)}, usamos modelos matemáticos não determinísticos chamados modelos probabilísticos. Neste caso, mesmo conhecendo algumas condições, não é possível determinar qual será o resultado antes da realização do experimento.

\subsection{Um pouco de história da Probabilidade}

Antes de começar o estudo um pouco mais formal de probabilidade, apresentaremos um breve resumo sobre a história da probabilidade.

A noção de acaso e ocorrências de fenômenos aleatórios foram percebidas   sensorialmente pela humanidade bem antes de sermos capazes de utilizar a Matemática como forma de descrição do mundo. No entanto, a percepção antiga é de que havia uma razão mítica para o aparecimento de tais fenômenos. Há vários registros históricos de 2700 A.C. do uso de dados antigos (como os ossos astrágalos e dados egípcios, ilustrados nas \hyperref[astragalos]{figuras \ref{astragalos} e \ref{dadosegipcios}}), usados para uma tomada de decisão regida pelos Deuses do Acaso, quando o homem queria se eximir de sua responsabilidade na escolha e tomada de decisão.

\begin{minipage}{0.5\textwidth}
\begin{figure}[H]
\centering
\capstart

\noindent\includegraphics[trim=1.2mm 0 1.75mm .2mm, clip, height=110bp]{{astragalos}.png}
\caption{Astrágalos}\label{astragalos}\end{figure}
\end{minipage}
\begin{minipage}{0.5\textwidth}
\begin{figure}[H]
\centering
\capstart

\noindent\includegraphics[height=110bp]{{dadosegipcios}.png}
\caption{Dados egípcios}\label{dadosegipcios}\end{figure}
\end{minipage}

A própria Bíblia nos informa que  “não cai uma folha de uma árvore sem que o Pai não deseje”. Essa crença de que deuses (no mundo panteísta) ou Deus (no mundo monoteísta) eram os regentes desses fenômenos, acarretou um atraso histórico na matematização do acaso e na criação da Teoria das Probabilidades, uma área considerada cognitivamente desafiadora até hoje na Ciência. Como bem colocou Piaget em seu famoso livro A Origem da Ideia do Acaso na Criança:  “Em contraste com as operações lógicas e aritméticas, a probabilidade é descoberta gradualmente.”

Por isso, foi preciso esperar os séculos XVI e XVII para que matemáticos como Cardano, Tartaglia, Pascal e Fermat (\hyperref[rostos_historia]{figura \ref{rostos_historia}}), para citar alguns, conseguissem dar uma explicação mais consistente do conceito de acaso/aleatoriedade no seio da Matemática, a partir, primordialmente, do estudo de jogos de azar e de sua conexão estreita com a Análise Combinatória.

\begin{figure}[H]
\centering
\capstart

\noindent\includegraphics[height=300bp]{{rostos_historia}.png}
\caption{Alguns matemáticos que originaram a discussão do conceito de acaso}\label{rostos_historia}\end{figure}{}

No entanto, poderíamos dizer que a ideia fundamental por trás da matematização do acaso reside essencialmente na Estatística, quando esta reconhece, pela sua própria natureza, que fenômenos aleatórios, embora sem explicação determinística, tendem a demonstrar uma certa taxa regular de ocorrência conforme são realizados vários experimentos similares ao longo do tempo. A busca de um modelo que explique tais regularidades de ocorrência do fenômeno em estudo é a ideia central da Teoria das Probabilidades e sua utilidade hoje em vários campos científicos, como Economia, Medicina, Robótica, Engenharia, Computação, Biologia, etc, demonstra como a teoria está mais perto da Estatística do que da abordagem feita por meio do diálogo com a Análise Combinatória durante os Séculos das Luzes.

É somente na primeira metade do século XX que a teoria das probabilidades vai adquirir uma base axiomática rigorosa por meio da construção teórica estabelecida pelo matemático russo Kolmogorov (\hyperref[kolmogorov]{figura \ref{kolmogorov}}). Desde então a teoria das probabilidades tem sido vista como uma das áreas mais promissoras da Matemática e a ferramenta por excelência para modelar e explicar os mais variados fenômenos aleatórios presentes no mundo contemporâneo.

\begin{figure}[H]
\centering
\capstart

\noindent\includegraphics[trim = 0 10mm 0 0, clip, width=150bp]{{kolmogorov}.png}
\caption{Andrei Kolmogorov (\(\star\)\(1903\) — \(\dagger\)\(1987\))}\label{kolmogorov}\end{figure}

Veja na \hyperref[linhadotempo]{figura \ref{linhadotempo}} uma linha do tempo destacando acontecimentos importantes no desenvolvimento da teoria das probabilidades.

\begin{figure}[H]
\centering
\capstart
\resizebox{\linewidth}{!}
{
\begin{tikzpicture}[scale=.4, every node/.style={scale=.7}]
     \tikzstyle{quadro}=[rectangle,draw, minimum width=1cm, minimum height=0.5, align=left]
         \tikzstyle{circulo}=[circle, draw, minimum size=0.05cm, fill=red]
         
         \draw (-9,-.5) -- (-6,-.50);
         \draw (-1,-.5) -- (0,-.5) -- (25,-.5);
         \draw [dotted] (-6,-.5) -- (-1,-.5);
         \foreach \x/\y in {-7/3000 A.C.,0/0,5/500,10/1000,15/1500,20/2000} \draw (\x,-0.5) -- (\x,-1) node [below] {\y};

         \node (qRegistros) at (-6,7) [quadro,align=center] {Registros de objetos \\ de ossos similares \\ a dados};
         \node (qCardano) at (11,6) [quadro, align=center] {Cardano \\ (número \\ combinatório)};
         \node (qFermat) at (14,2) [quadro,align=center] {Fermat e Pascal \\ (Princípios do \\ cálculo de \\ probabilidades)};
         \node (qBernoulli) at (16.5,7) [quadro,align=center] {Bernoulli \\ (distribuição \\ binomial)};
         \node (qBayes) at (20,2) [quadro,align=center] {Probabilidade \\ condicional e \\ teorema de Bayes)};
         \node (qKolmogorov) [quadro,align=center] at (22,6.5) {Kolmogorov \\ (Fundamentos de \\ Probabilidade)};
         
         
         \node (registros) at  (-7,4.5) [circulo] {};
         \node (Cardano) at  (15.64,4.5) [ circulo] {};
         \node (Fermat) at (17.04,4.5) [circulo] {};
         \node (Bernoulli) at (16.54,4.5) [circulo] {};
         \node (Bayes) at (17.63,4.5) [circulo] {};
         \node (Kolmogorov)  at(19.33,4.5) [circulo] {};
         
         \path
         (registros) edge (qRegistros)
         (Cardano) edge (qCardano)
         (Fermat) edge (qFermat)
         (Bernoulli) edge (qBernoulli)
         (Bayes) edge (qBayes)
         (Kolmogorov) edge (qKolmogorov);
\end{tikzpicture}
}
\caption{Linha do tempo}\label{linhadotempo}
\end{figure}

Nos capítulos \hyperref[est1-chap]{\textbf{A Natureza da Estatística}} e \hyperref[est2-chap]{\textbf{Medidas de Posição e Dispersão}}, vimos como resumir a informação de dados aleatórios amostrais com o objetivo de  entender estruturas úteis para uma tomada de decisão sob incerteza. Neste capítulo, analisaremos as características extraídas dos dados aleatórios amostrais a fim de revelar como a Estatística nos auxilia a descrever as regularidades de ocorrências de determinados eventos aleatórios.
\clearpage
\begin{objectives}{Não determinístico (aleatório) ou determinístico?}
{
Reconhecer fenômenos aleatórios, distinguindo-os de fenômenos determinísticos.
}{1}{1}
\end{objectives}

\begin{answer}{Não determinístico (aleatório) ou determinístico?}
{
\begin{enumerate}
\item determinísitco
\item Dleatório
\item Determinísitico
\item Aleatório
\item Aleatório
\end{enumerate}
}{1}
\end{answer}
\begin{objectives}{Interpretando medida de incerteza}
{
Reconhecer que toda probabilidade se traduz como uma taxa de ocorrência de um evento.
}{1}{1}
\end{objectives}
\begin{sugestions}{Interpretando medida de incerteza}
{
Algumas afirmações envolvendo uma probabilidade serão apresentadas para depois explorar a interpretação dessa informação.
}{1}{1}
\end{sugestions}
\begin{answer}{Interpretando medida de incerteza}
{
\begin{enumerate}
\item Não. A probabilidade $0{,}5$ indica uma taxa de ocorrência de modo que se, de fato, a probabilidade é $0{,}5$ de ocorrer cara, isso significa que se lançarmos a moeda muitas vezes ($N$), esperamos observar um número de caras que seja próximo de $0{,}5\cdot N$ ($50\%$ do número de lançamentos). Assim, se a moeda é lançada $100$ vezes, espera-se observar um número de caras próximo a $50$, mas não necessariamente igual a $50$. Vocês podem rapidamente fazer uma experiência, reunindo-se em $10$ grupos de cerca de $4$ alunos e cada grupo deverá lançar uma moeda $10$ vezes e registrar o número de caras. Depois, a informação dos $10$ grupos deverá ser reunida, totalizando $100$ lançamentos. Quantas caras foram observadas?
\item Se a probabilidade de chover amanhã é de $30\%$, isso significa que para cada $10$ dias com as mesmas características, espera-se que em $3$ deles chova e em $7$ não. Se é muito trabalhosos carregar um guarda-chuva, talvez a melhor decisão seja a de não levar o guarda chuva, dado que a probabilidade é inferior a $50\%$. Por outro lado, se você está resfriado e quer evitar de todo o modo o risco de se molhar, carregar o guarda-chuva, mesmo que ele venha a não ser útil, pode ser a melhor decisão. O importante aqui, é que não há como não correr riscos, qualquer que seja a sua decisão, mas a informação da probabilidade de chover é útil para tomarmos uma decisão com base nas nossas necessidades e expectativas.
\item Não. Aqui podemos usar a mesma ideia da moeda: como a probabilidade é de $10\%$, espera-se que ao observar um grande número de pessoas ($N$), cerca de $0{,}1\cdot N$ ($10\%)$ delas irão apresentar o Diabetes. Em $500$ pessoas, espera-se que $10\%$ de $500=50$ pessoas irão ter o Diabetes. Observe também que nesse último caso, a introdução de políticas públicas na área da saúde e de campanhas sobre hábitos saudáveis, poderia reduzir essa probabilidade.
\end{enumerate}
}{0}
\end{answer}
\clearmargin
\begin{objectives}{Avaliando probabilidades}
{
Reconhecer diferentes interpretações da probabilidade (clássica, frequentista e subjetiva).
}{1}{2}
\end{objectives}
\begin{sugestions}{Avaliando probabilidades}
{
Nesta atividade serão apresentados três blocos de três itens cada. Em cada item será solicitada a probabilidade de um determinado evento. No bloco I será adequado adotar a interpretação clássica da probabilidade, de modo que as respostas deverão surgir de forma natural e sem problemas. É importante discutir com os alunos como foram obtidas as respostas do bloco I. No bloco II, a interpretação adequada de probabilidade será a frequentista. Neste bloco, não existe a resposta certa. O objetivo neste caso é fazer o aluno pensar, pois tratam-se de situações aleatórias para as quais faz sentido atribuir uma probabilidade (chance). No bloco III, a interpretação adequada de probabilidade é a subjetiva e, portanto, também não haverá a resposta certa. O objetivo principal é levar o aluno a pensar em como atribuir probabilidades para eventos aleatórios. Esta atividade serve como estímulo à discussão do conceito de probabilidade. No item b, do bloco II, discuta com seus alunos sobre como investigar a proporção de nascimentos de meninos e meninas.
}{1}{2}
\end{sugestions}
\begin{answer}{Avaliando probabilidades}
{No primeiro bloco de itens pode-se pensar que cada resultado possível tenha a mesma chance de ocorrer. Assim temos,
\begin{enumerate}
\item $\frac{1}{4}=0{,}4-40\%$, pois são $10$ cartões e quatro deles apresentam "triângulos"{} que representam meninas.
\item Novamente,
\begin{enumerate}
\item $\frac{2}{10}=0{,}2=20\%$, pois são $10$ casas e duas delas têm exatamente $4$ moradores.
\item $\frac{4}{10}=0{,}4=40\%$, pois são $10$ casas e quatro delas têm mais de $4$ moreadores
\end{enumerate}
\item $\frac{1}{4}=0{,}25=40\%$, pois a área da região pintada corresponde a um quarto da área do círculo, isto é, a um setor circular de ângulo reto.
\end{enumerate}
}{1}
\end{answer}
\clearmargin
\begin{answer}{Avaliando probabilidades}
{
  No segundo bloco de itens o objetivo é determinar probabilidades, usando o enfoque frequentista de probabilidade, atribuindo como resposta a frequência relativa de ocorrência do evento.
  \begin{enumerate}
  \item $\frac{19}{20}=0{,}95$ que corresponde à frequência relativa de caras obtidas nos $20$ lançamentos. Outras respostas podem ser consideradas neste item, por exemplo, $0,50$, supondo-se, apesar das evidências em contrário, que a moeda é honesta; ou $0{,}725$, calculando-se uma média entre a frequência de caras observadas e a frequência esperada, supondo que a moeda é honesta ($\frac{0{,}95+0{,}5}{2}$).
  \item Uma resposta possível é $\frac{1}{2}=0{,}5$, pois da experiência é possível verificar que meninas e meninos nascem na mesma proporção.
  \item $\frac{8500}{11000}\approx0{,}77$, pois $8500$ estudantes de $11$ mil responderam que esse é o seu principal meio de acesso à internet na pesquisa realizada e supondo que a pesquisa represente bem o comportamento na população de todos os estudantes de Ensino Fundamental II e Médio.
  \end{enumerate}

  No terceiro bloco de itens o objetivo é determinar probabilidades, usando o enfoque subjetivo da probabilidade. Certos eventos são únicos e não podem ser reproduzidos, mas por serem aleatórios, faz sentido atribuir a eles probabilidades. As respostas aqui dependerão da experiência e objetivos de cada um. As perguntas neste bloco têm o objetivo de estimular a discussão sobre o conceito de probabilidade.
  \begin{enumerate}
  \item Uma resposta possível é maior do que $0,5$ ($50\%$), pois o Brasil, além de ser o único país a ter participado de todas as Copas do Mundo até a Copa de 2018, classificou-se na fase de grupos nas últimas $9$ edições desse campeonato.
  \item A resposta dependerá de cada aluno. Por exemplo, se for um um bom aluno do Ensino Médio que pretende cursar Engenharia, curso que dura cinco anos, uma resposta possível seria uma probabilidade maior do que $0{,}5$ ($50\%$). Por outro lado, se for um aluno que tem outros planos e não pretende cursar o nível superior, uma resposta possível seria uma probabilidade menor do que $0{,}5$ ($50\%$).
  \end{enumerate}

  \textbf{Observação}: A razão pela qual usamos $0{,}5$ ($50\%$) como valor de comparação para avaliar probabilidades se dá pelo fato de ser exatamente o centro da escala da probabilidade que varia de $0$ a $1$ ($0$ a $100\%$). Assim, se temos a percepção de que é mais provável que certo evento ocorra do que ele não ocorra, atribuímos a ele uma probabilidade maior do que $0{,}5$ ($50\%$). Por outro lado, se temos a percepção de que é menos provável que certo evento ocorra do que ele não ocorra, atribuímos a ele uma probabilidade inferior a $0{,}5$ ($50\%$). Se temos a percepção de que não existe favorecimento entre a ocorrência ou não de certo evento, ou mesmo se não sabemos nada sobre ele, atribuímos a ele uma probabilidade $0{,}5$ ($50\%$).
}{1}
\end{answer}

\begin{task}{não determinístico (aleatório) ou determinístico?}

Classifique cada experimento a seguir em aleatório ou determinístico.

Deseja-se observar:
\begin{enumerate}[rightmargin=3mm]
\item 
O valor constante de cada prestação quando se financia um eletrodoméstico, estabelecendo-se a taxa de juros efetiva ao mês, a quantidade de meses do financiamento e o pagamento da primeira prestação no ato da compra.

\item 
A quantidade de metros cúbicos de água consumida em sua residência no primeiro semestre do próximo ano.


\item 
A distância percorrida por um objeto em movimento, conhecendo-se a velocidade e o tempo transcorrido.

\item 
O valor a ser pago na conta de luz da sua residência no próximo mês.

\item 
A sua média final em Matemática desse ano.

\end{enumerate}
\end{task}
\begin{task}{interpretando medida de incerteza}


Responda aos itens a seguir.
\begin{enumerate}
\item {} 
A probabilidade de ocorrer cara quando lançamos uma moeda honesta é $0{,}5$. Isso significa que toda vez que lançarmos essa moeda $100$ vezes, ocorrerão $50$ caras? Por quê?

\item {} 
Foi publicada a previsão do tempo, indicando que a probabilidade de chover amanhã na região onde você mora e estuda é de $30\%$. Que decisão você tomaria com base nessa previsão: levar ou não um guarda-chuva para a escola? Por quê? Como você interpreta essa previsão?

\item {} 
Um estudo na área de Saúde indicou que a probabilidade de uma pessoa vir a ter o Diabetes é $10\%$. Isso significa que ao acompanhar um grupo de $500$ pessoas, $50$ delas terão Diabetes? Por quê?

\end{enumerate}
\end{task}
\clearpage
\begin{task}{avaliando probabilidades}
\label{avaliando-probabilidades}


Bloco I - Probabilidade clássica
\begin{enumerate}
\item {} 
De um grupo de 10 estudantes, um será sorteado para ser o representante de turma. Como são 4 meninas e seis meninos, decidiu-se, para fazer o sorteio, representar as meninas por cartões ilustrados com triângulos e os meninos por cartões ilustrados com círculos. Os cartões foram colocados numa caixa e um será sorteado (\hyperref[cartoesilustrados]{figura \ref{cartoesilustrados}}).

\begin{figure}[H]
\centering

\begin{tikzpicture}
[scale=0.6]
\draw [,rounded corners=8pt, -] (0,-0) -- (0,10) -- (10,10) -- (10,0) -- cycle;
       \draw (0.25,6.5) rectangle (1.75,8.5);
       \draw (2.25,6.5) rectangle (3.75,8.5);
       \draw (4.25,6.5) rectangle (5.75,8.5);
       \draw (6.25,6.5) rectangle (7.75,8.5);
       \draw (8.25,6.5) rectangle (9.75,8.5);
       \draw (0.25,1.5) rectangle (1.75,3.5);
       \draw (2.25,1.5) rectangle (3.75,3.5);
       \draw (4.25,1.5) rectangle (5.75,3.5);
       \draw (6.25,1.5) rectangle (7.75,3.5);
       \draw (8.25,1.5) rectangle (9.75,3.5);
       \draw [color=primario, fill=primario] (1,7.5) circle (12pt);
       \draw [color=primario, fill=primario] (5,7.5) circle (12pt);
       \draw [color=primario, fill=primario] (7,7.5) circle (12pt);
       \draw [color=primario, fill=primario] (3,2.5) circle (12pt);
       \draw [color=primario, fill=primario] (5,2.5) circle (12pt);
       \draw [color=primario, fill=primario] (9,2.5) circle (12pt);
       \draw [color=atento, fill=atento] (3,8) -- (3.5,7) -- (2.5,7) -- (3,8) --  cycle;
       \draw [color=atento, fill=atento] (9,8) -- (9.5,7) -- (8.5,7) -- (9,8) --  cycle;
\draw [color=atento, fill=atento, ] (1,3) -- (1.5,2) -- (.5,2) -- (1,3) --  cycle;
       \draw [color=atento, fill=atento] (7,3) -- (7.5,2) -- (6.5,2) -- (7,3) --  cycle;
\end{tikzpicture}
\caption{Cartões ilustrados}
\label{cartoesilustrados}
\end{figure}

Qual é a probabilidade (chance) de ser escolhida uma menina como representante de turma? Por quê?

\item {} 
Numa rua há $10$ casas. O número de moradores por casa está representado na \hyperref[moradorescasa]{figura \ref{moradorescasa}}. Suponha que você irá escolher ao acaso uma casa desta rua.
\begin{figure}[H]
\centering

\begin{tikzpicture}[triangle/.style = {fill=session3!80, shape=isosceles triangle, node distance=.75cm, minimum width=1.5cm, minimum height=.75cm, rotate=90,isosceles triangle stretches, draw=black},
casa/.style={fill, draw, primario, rectangle, minimum height=1cm, minimum width=1.25cm, node distance=2cm, text={white}, font=\bfseries,draw=black}]


\node (a) [casa] {5};
\node (b) [casa, right of=a] {4};
\node (c) [casa, right of=b] {8};
\node (d) [casa, right of=c] {8};
\node (e) [casa, right of=d] {4};

\node [above of=a, triangle] {};
\node [above of=b, triangle] {};
\node [above of=c, triangle] {};
\node [above of=d, triangle] {};
\node [above of=e, triangle] {};

\node (f) [casa, below of=a, node distance=2.5cm] {1};
\node (g) [casa, right of=f] {1};
\node (h) [casa, right of=g] {3};
\node (i) [casa, right of=h] {2};
\node (j) [casa, right of=i] {6};

\node [above of=f, triangle] {};
\node [above of=g, triangle] {};
\node [above of=h, triangle] {};
\node [above of=i, triangle] {};
\node [above of=j, triangle] {};


\end{tikzpicture}
\caption{Ilustração dos números de moradores por casa}
\label{moradorescasa}
\end{figure}

\begin{enumerate}
\item
Qual é a probabilidade de que a casa escolhida tenha exatamente $4$ moradores? Por quê?

\item {} 
Qual é a probabilidade de que a casa escolhida tenha mais de $4$ moradores? Por quê?
\end{enumerate}

\item {} 
Suponha que você vá girar a roleta ilustrada na \hyperref[roleta]{figura \ref{roleta}}.

\begin{figure}[H]
\centering

\begin{tikzpicture}[scale=.75] 

  
  \draw[fill=\currentcolor!70, smooth] (0,0) -- +(90:3) arc (90:180:3cm);
  \draw (0,0) -- (-3,0);
  \draw [thick](0,0) circle (3cm); 
  \draw (-.5,0) -- (-.5,.5) -- (0,.5);
  \draw [fill = black, smooth] (-.25,.25) circle (1pt);
  \draw [->] (1,-1.2) arc (0:-100:.75cm);
  \draw [thick, ->] (0,0) -- (2,-2);
  \draw [color=destacado, fill = destacado] (0,0) circle (.1cm);

\end{tikzpicture}
\caption{Roleta}
\label{roleta}
\end{figure}


\end{enumerate}

Qual é a probabilidade de que a seta pare na região pintada de azul? Por quê?

Bloco II - Probabilidade frequentista
\begin{enumerate}
\item {} 
Suponha que você tenha lançado uma moeda $20$ vezes e que tenha observado a face “cara”{} 19 vezes e a face “coroa”{} uma vez. Se você lançar esta moeda mais uma vez, qual é a probabilidade (chance) de a face voltada para cima resultar em “cara”? Por quê?

\item {} 
Suponha que um bebê tenha nascido na maternidade mais próxima de sua casa na manhã de hoje. Qual é a probabilidade de que este bebê seja um menino? Por quê?

\item {} 
A pesquisa TIC Educação 2016, do Centro de Estudos sobre as Tecnologias da Informação e da Comunicação (Cetic), coletou dados de cerca de $11$ mil estudantes do segundo segmento do Ensino Fundamental e do Ensino Médio. Entre várias informações, verificou-se que cerca de $8.500$ estudantes usam smartphones como seu principal meio de acesso à internet. A pesquisa aconteceu entre agosto e dezembro de 2016.”{} (Leia a reportagem publicada no \href{https://g1.globo.com/educacao/noticia/52-das-instituicoes-de-educacao-basica-usam-celular-em-atividades-escolares-aponta-estudo-da-cetic.ghtml}{G1.com.br}).

\end{enumerate}

Qual é a probabilidade de que um estudante de Ensino Fundamental II ou Médio, escohido ao acaso, use como seu principal meio de acesso à internet um smartphone, usando os dados dessa pesquisa? Por quê?

Bloco III - Probabilidade subjetiva
\begin{enumerate}
\item {} 
Qual é a probabilidade de que o Brasil se classifique na fase de grupos na próxima Copa do Mundo que irá competir?

\item {} 
Qual é a probabilidade de que daqui a 8 anos você tenha concluído um curso de nível superior?

\item {} 
Qual é a probabilidade de que você esteja casado(a) aos 25 anos?

\end{enumerate}
\end{task}


\arrange{Conceitos Básicos}
\label{organizandoconceitosbasicos}
Como você fez para determinar as probabilidades no bloco I da atividade \hyperref[avaliando-probabilidades]{Avaliando probabilidades}? E no bloco II? E no bloco III? Você deve ter percebido que, dependendo da situação, nem sempre é possível atribuir probabilidades a um evento, usando o mesmo tipo de raciocínio.

Por exemplo, considere o experimento “observar se um corpo celeste cairá sobre a casa onde você mora dentro de uma hora”.

\begin{figure}[H]
\centering

\noindent\includegraphics[width=200bp]{{corpoceleste}.png}
\caption{Corpo celeste}
\end{figure}


Há dois resultados possíveis: cair ou não cair. Você acha que é razoável atribuir probabilidades iguais a estes dois resultados?

Certamente não, pois sabemos que a situação “cair”{} é um evento raríssimo. Em toda a sua vida, você já observou um evento deste tipo?

Neste caso, não faz sentido atribuir probabilidades iguais para os dois resultados possíveis. Uma probabilidade razoável para o resultado “cair”{} seria bem pequena, próxima de zero e não igual a probabilidade do resultado “não cair”.

Antes de apresentarmos diferentes interpretações de probabilidade, vamos começar definindo os termos espaço amostral e evento.


\begin{observationtitle}{Espaço amostral}
Conjunto que compreende todos os resultados possíveis de um experimento aleatório.
\end{observationtitle}

Usaremos a letra maiúscula \(S\) para denotar espaço amostral.

Por exemplo, no item \titem{b)}, do bloco I da atividade \hyperref[avaliando-probabilidades]{Avaliando probabilidades}, um espaço amostral pode ser representado por \(S=\{1,2,3,4,5,6,8\}\), pois este conjunto compreende todos os números possíveis de serem obtidos, quando sorteamos uma casa do bairro. Observe novamente que, neste caso, também não é razoável atribuir probabilidades iguais aos elementos deste espaço amostral, pois na rua há duas casas com exatamente 1 morador, duas casas com exatamente 4 moradores e duas casas com exatamente 8 moradores, enquanto que há apenas uma casa com exatamente 2 moradores, uma casa com exatamente 3 moradores, uma casa com exatamente 5 moradores e uma casa com exatamente 6 moradores.

\begin{observationtitle}{Evento}
É qualquer subconjunto \(A\) do espaço amostral \(S\) para o qual faz sentido atribuir uma probabilidade.
\end{observationtitle}

Veja, na \hyperref[diagramavenn1]{figura \ref{diagramavenn1}}, uma representação de um evento (conjunto) \(A\) em um espaço amostral \(S\) (conjunto universo), usando diagrama de Venn.
\begin{figure}[H]
\centering

\begin{tikzpicture}

[scale=0.6]
\draw [,rounded corners=8pt, -] (0,-0) -- (0,5) -- (8,5) -- (8,0) -- cycle;
\draw (4,2.5) ellipse (2.5cm and 2cm);
\node [] at (6.25,4) {$A$};
\node [] at (8,5.25) {$S$};
\end{tikzpicture}
\caption{Representação de conjuntos no diagrama de Venn}
\label{diagramavenn1}
\end{figure}

Ao realizar um experimento aleatório, dizemos que um evento \(A\) ocorreu, se o resultado tiver sido um elemento de \(A\). Por exemplo, se ao lançarmos um dado com as faces numeradas de $1$ a $6$, tiver ocorrido face $2$ e \(A\) é o evento “a face voltada para cima corresponde a um número par”, isto é, \(A=\{2,4,6\}\), dizemos que o evento \(A\) ocorreu.

\begin{figure}[H]
\centering

\noindent\includegraphics[width=200bp]{{dados}.png}
\caption{Dados comuns de seis faces numeradas de 1 a 6}
\end{figure}

\begin{observationtitle}{Evento elementar}
Subconjunto unitário do espaço amostral \(S\); ou, equivalentemente, subconjunto do espaço amostral \(S\) no qual há apenas um resultado possível.
\end{observationtitle}

Por exemplo, no lançamento de um dado, podemos representar o espaço amostral por \(S=\{1,2,3,4,5,6\}\). Nesse caso, os eventos elementares são os conjuntos unitários:
\begin{equation*}
\begin{split}\{1\}, \{2\}, \{3\}, \{4\}, \{5\} \text{ e } \{6\}.\end{split}
\end{equation*}
\paragraph{Operações de união, interseção e complementariedade}

Faremos agora uma rápida revisão sobre operações entre conjuntos (união, interseção e complementariedade), pois eventos são conjuntos e nós estamos interessados em calcular probabilidades de eventos.

\begin{observationtitle}{União}
O conjunto \(A\cup B\) (lê-se \(A\)  união \(B\)) corresponde à reunião de todos os elementos de \(A\) e de \(B\). Veja na \hyperref[aub]{figura \ref{aub}}, uma representação de \(A\cup B\), usando diagrama de Venn, em que o conjunto \(A\cup B\) corresponde à região pintada.

\begin{figure}[H]
\centering

\begin{tikzpicture}[scale=0.9]
\draw [rounded corners=8pt, -] (0,-0) -- (0,5) -- (8,5) -- (8,0) -- cycle;

\node at (7,4) {$B$};
\node at (.5,4) {$A$};
\node at (7,5.25) {$S$};
\draw [fill=\currentcolor!80, smooth] (5,2.5) circle (2cm);

\draw [fill=\currentcolor!80, smooth] (2.5,2.5) circle (2cm);
\clip [draw] (2.5,2.5) circle (2cm);
\draw [fill= \currentcolor!80, smooth] (5,2.5) circle (2cm);


\end{tikzpicture}
\caption{\(A\cup B\)}
\label{aub}
\end{figure}


Dizemos que o evento \(A\cup B\) ocorreu se pelo menos um dos dois eventos, \(A\) ou \(B\), tiver ocorrido.
\end{observationtitle}

\begin{observationtitle}{Interseção}
O conjunto \(A\cap B\) (lê-se \(A\) interseção \(B\)) corresponde à coleção de todos os elementos que pertencem simultaneamente ao conjunto \(A\) e ao conjunto \(B\). Veja na \hyperref[intersecao]{figura \ref{intersecao}} uma representação de \(A\cap B\), usando diagrama de Venn, em que o conjunto \(A\cap B\) corresponde à região pintada.
\begin{figure}[H]
\centering

\begin{tikzpicture}[scale=0.9]
\draw [,rounded corners=8pt, -] (0,-0) -- (0,5) -- (8,5) -- (8,0) -- cycle;
\node at (7,4) {$B$};
\node at (.5,4) {$A$};
\node at (7,5.25) {$S$};
\draw (5,2.5) circle (2cm);
\clip [draw] (2.5,2.5) circle (2cm);
\draw [fill=\currentcolor!80] (5,2.5) circle (2cm);
\end{tikzpicture}
\caption{\(A\cap B\)}
\label{intersecao}
\end{figure}


Dizemos que o evento \(A \cap B\) ocorreu, se os eventos \(A\) e \(B\) tiverem ocorrido simultaneamente.
\end{observationtitle}


\begin{observationtitle}{Complementariedade}

O conjunto \(\overline{A}\) (lê-se \(A\) complementar) corresponde à coleção de todos os elementos do conjunto universo (espaço amostral \(S\)) que não pertencem ao conjunto \(A\). Veja na \hyperref[complementar]{figura \ref{complementar}} uma representação de \(\overline{A}\), usando diagrama de Venn, em que o conjunto \(\overline{A}\) corresponde à região pintada.

\begin{figure}[H]
\centering

\begin{tikzpicture}
\draw [,fill=\currentcolor!80,rounded corners=8pt, -] (0,-0) -- (0,5) -- (8,5) -- (8,0) -- cycle;
\node at (7,4) {$\overline{A}$};
\node at (.5,4) {$A$};
\node at (7,5.25) {$S$};
\draw[fill=white] (2.5,2.5) circle (2cm);
\end{tikzpicture}
\caption{Evento complementar de \(A: \overline{A}\)}
\label{complementar}
\end{figure}

Dizemos que o evento \(\overline{A}\) ocorreu, se o evento \(A\) não tiver ocorrido.
\end{observationtitle}

Pela definição das operações de união, interseção e complementareidade dizemos que
\begin{itemize}
\item {} 
o evento \(A\cup B\)  ocorreu se, e somente se, pelo menos um dos dois eventos \(A\) ou \(B\) tiverem ocorrido;

\item {} 
o evento \(A\cap B\) ocorreu se, e somente se, os dois eventos \(A\) e \(B\) tiverem ocorrido simultanemante.

\item {} 
o evento \(\overline{A}\) ocorreu se, e somente se, o evento \(A\) \textbf{não} tiver ocorrido.

\end{itemize}

\begin{observation}{}
Dois eventos \(A\)  e \(B\) são ditos eventos disjuntos se $A\cap B=\emptyset$, ou seja, se \(A\) e \(B\) não tiverem elementos em comum.
\end{observation}

Dado qualquer espaço amostral \(S\), os conjuntos \(S\) e \(\emptyset\) (conjunto vazio) são considerados eventos especiais e chamados de \textbf{evento certo} e \textbf{evento impossível}, respectivamente.

Para o evento certo (\(S\)) atribui-se probabilidade $1$ e, para o evento impossível (\(\emptyset\)), atribui-se probabilidade zero. Para qualquer outro evento, a probabilidade deverá ser um número real no intervalo $[0,1]$.

Para concluir essa breve revisão de operações com conjuntos, vamos apresentar a propriedade distributiva da operação de interseção com a união de dois eventos, a saber,

\begin{equation*}
\begin{split}(A\cup B)\cap C=(A\cap C)\cup (B\cap C)\end{split}
\end{equation*}

Para visualizar melhor essa propriedade, considere $A$, $B$ e $C$ eventos em um espaço amostral $S$, representados nos diagramas de Venn da \hyperref[diagramasvenn2]{figura \ref{diagramasvenn2}}, lembrando que as interseções nesses diagramas podem ser conjuntos vazios.

\begin{figure}[H]
\centering
\begin{minipage}{0.35\textwidth}
\centering
\begin{tikzpicture}[scale=0.5, every node/.style={scale=1}]
\draw [,rounded corners=8pt,] (0,0) -- (0,7.5) -- (10,7.5) -- (10,0) -- cycle;
\node  at (9,8) {$S$};
\draw (3.5,2.5) circle (2cm);
\node  at (1,2) {$B$};
\draw (6.5,2.5) circle (2cm);
\node  at (9,2) {$C$};
\draw (5,5) circle (2cm);
\node  at (7.5,5.5) {$A$};
\end{tikzpicture}

\end{minipage}
\begin{minipage}{0.35\textwidth}
\centering
\begin{tikzpicture}[scale=0.5, every node/.style={scale=1}]

\draw [,rounded corners=8pt,] (0,0) -- (0,7.5) -- (10,7.5) -- (10,0) -- cycle;
\node at (9,8) {$S$};
\draw (3.5,2.5) circle (2cm);
\node at (1,2) {$B$};
\draw (6.5,2.5) circle (2cm);
\node at (9,2) {$C$};
\draw (5,5) circle (2cm);
\node at (7.5,5.5) {$A$};
\end{tikzpicture}

\end{minipage}
\caption{Diagramas de Venn com três conjuntos: \(A\), \(B\)  e \(C\)}
\label{diagramasvenn2}
\end{figure}


No primeiro diagrama, pinte de uma cor \(A\cup B\) e com outra cor, pinte o conjunto \(C\), destacando a região que foi pintada pelas duas cores. No segundo diagrama, pinte o conjunto \(A\cap C\) e, com a mesma cor,  o conjunto \(B\cap C\). A região pintadada corresponde ao conjunto dado no lado direito da igualdade. Finalmente, verifique que as regiões destacadas correspondem ao mesmo conjunto.

A seguir, serão apresentadas três interpretações da probabilidade.


\subsection{Interpretação clássica de probabilidade}

Na interpretação clássica de probabilidade todos os eventos elementares são considerados igualmente prováveis (equiprováveis).

Esta interpretação costuma ser usada em problemas envolvendo lançamento de dados, sorteios de cartas de um baralho e outros jogos. De fato, os primeiros trabalhos teóricos publicados envolvendo probabilidades no século XVII, fazem uso desta interpretação e envolvem cálculos de probabilidades de eventos em jogos de azar.

No entanto, nem sempre a interpretação clássica será adequada: lembre-se do exemplo da queda de um corpo celeste.

Um outro problema com esta interpretação é a circularidade do conceito de probabilidade para definir a própria probabilidade, pois considera em sua definição “resultados igualmente prováveis”{} que depende do conceito de probabilidade.

\subsection{Interpretação frequentista de probabilidade}

Na interpretação frequentista de probabilidade, a probabilidade de um evento é definida como a frequência relativa de ocorrência deste evento, se o experimento for repetido, sob as mesmas condições, um grande número de vezes.

Problemas com esta definição envolvem falta de clareza: o que siginificam
\begin{itemize}
\item {} 
“sob as mesmas condições”? e

\item {} 
“um grande número de vezes”?

\end{itemize}

Além disso, existem fenômenos únicos para os quais não é possível realizar repetições, por exemplo, o experimento que envolve verificar se daqui a 8 anos seu nível de instrução será superior completo ou não.

No entanto, esta interpretação é muito útil e amplamente usada em modelagens probabilísticas. De fato, a interpretação frequentista de probabilidade tem suas origens com a Lei dos Grandes Números, importante resultado da teoria das probabilidades estabelecido pelo matemático suíço Jakob Bernoulli (\(\star\)\(1654\) — \(\dagger\)\(1705\)). Bernoulli levou mais de vinte anos para provar a fórmula matemática, que foi publicada em seu livro “A Arte da Conjectura”{} (Ars Conjectandi) por seu sobrinho Nicolau Bernoulli em 1713. Bernoulli afirmou que quanto maior o número de tentativas (repetições do experimento), mais a proporção de tentativas bem-sucedidas (frequência relativa de ocorrência do evento de interesse) se aproxima de \(p\) (probabilidade do evento de interesse ocorrer).

\begin{figure}[H]
\centering

\noindent\includegraphics[width=125bp]{{jakob_bernoulli}.png}
\caption{Jakob Bernoulli (1654-1705)}
\end{figure}


Veja na \hyperref[1000_lancamentos2]{figura \ref{1000_lancamentos2}} uma ilustração da Lei dos Grandes Números na qual mostra-se uma \index{simulação}simulação do lançamento de uma moeda honesta (probabilidades iguais de cara e coroa) $1000$ vezes. Os gráficos ilustram a frequência relativa de caras, ou seja, número de caras obtidas sobre o número de lançamentos da moeda (eixo vertical) em função do número de lançamentos da moeda (eixo horizontal). A linha horizontal indica o valor \(\frac{1}{2}=0{,}5\), a probabilidade teórica de ocorrer cara para uma moeda honesta. Observe como rapidamente a frequência relativa se aproxima do valor \(\frac{1}{2}\).

\begin{figure}[H]
\centering

\noindent\includegraphics[width=250bp]{{1000_lancamentos2}.png}
\caption{Simulação do lançamento de uma moeda honesta 1000 vezes com destaque para  os 100 primeiros lançamentos da moeda.}
\label{1000_lancamentos2}
\end{figure}



\begin{figure}[H]
\centering

\noindent\includegraphics[width=245bp]{{100_lancamentos2}.png}
\caption{100 primeiros lançamentos da moeda com destaque para os 10 primeiros lançamentos
}
\end{figure}


\begin{figure}[H]
\centering

\noindent\includegraphics[width=245bp]{{10_lancamentos2}.png}
\caption{10 primeiros lançamentos da moeda}
\label{10_lancamentos2}
\end{figure}


Observe que nesta simulação ocorreu cara no primeiro lançamento da moeda de tal modo que a frequência relativa de caras inicial é $1$. Além disso, nota-se que nas repetições iniciais a frequência relativa de caras oscila muito mais, no entanto, rapidamente ela se aproxima de $0{,}5$ com uma oscilação desprezível.


Observando a \hyperref[10_lancamentos2]{figura \ref{10_lancamentos2}}, o que você diria que ocorreu (cara ou coroa), no terceiro lançamento da moeda? Por quê?

\subsection{Interpretação subjetiva de probabilidade}

Na \index{interpretação subjetiva de probabilidade}interpretação subjetiva de probabilidade, probabilidades de eventos são designadas de acordo com a experiência que o pesquisador tem sobre o fenômeno em investigação. No primeiro exemplo desta seção (queda de um corpo celeste) pode-se dizer que adotou-se a interpretação subjetiva quando atribui-se uma probabilidade pequena, próxima de zero, para o evento “não cair”.

Uma crítica a esta interpretação é a de que pessoas diferentes podem atribuir probabilidades diferentes para um mesmo evento. No entanto, observe que as outras duas interpretações também são subjetivas.

O importante, quando adota-se a interpretação subjetiva, é ter coerência. Por exemplo se sabemos que um evento \(A\) ocorre com frequência quatro vezes maior do que um evento \(B\), então \(P(A)=4\cdot P(B)\).

Se temos a percepção de que é mais provável que certo evento ocorra do que ele não ocorra, atribuímos a ele uma probabilidade maior do que 0,5. Por outro lado, se temos a percepção de que é menos provável que certo evento ocorra do que ele não ocorra, atribuímos a ele uma probabilidade inferior a 0,5. Se temos a percepção de que não existe favorecimento entre a ocorrência ou não de certo evento, ou mesmo se não sabemos nada sobre ele, atribuímos a ele uma probabilidade de 0,5. A razão pela qual usamos o valor 0,5 como referência se dá pelo fato de que 0,5 é exatamente o centro da escala da probabilidade que varia de 0 a 1 (0 a 100\%) como será formalizado na seção \hyperref[regrasbasicaspropriedades]{Organizando: Probabilidade - regras básicas e propriedades}.

\clearpage
\def\currentcolor{session2}
\begin{objectives}{Espaço amostral não é único!}
{
Reconhecer a não unicidade do espaço amostral.
}{1}{2}
\end{objectives}
\begin{sugestions}{Espaço amostral não é único!}
{
Nesta atividade pretende-se discutir com o aluno a não unicidade do espaço amostral. Esta discussão é importante, pois dependendo da forma como o espaço amostral é especificado, pode-se ter eventos elementares que são equiprováveis ou não. Além disso, algumas representações do espaço amostral de um experimento poderão responder a mais perguntas do que outras. Por exemplo, no primeiro item, a discriminação de todas as sequências possíveis de ordem de nascimentos permite responder perguntas sobre o sexo do filho mais velho, etc. Já no segundo item, perguntas deste tipo nem sempre podem ser respondidas.
}{1}{2}
\end{sugestions}
\begin{answer}{Espaço amostral não é único!}
{
\begin{enumerate}
\item Usando $a$ para menina e $o$ para menino, pode-se identificar todas as possibilidades, construindo-se o seguinte diagrama, chamado diagrama de árvore.

\begin{figure}[H]
\centering
\begin{tikzpicture}[scale=1, every node/.style={scale=.9}]

\draw (0,0) -- (30:3) node [right] {$a$} node [above, midway, rotate=30, scale=0.7] {};
\draw (0,0) -- (-30:3) node [right] {$o$} node [below, midway, rotate=-30, scale=0.7] {};
\draw (3.159807,1.5) -- ++(20:3) node [right] {$a$} node [above, midway, rotate=20, scale=0.7] {};
\draw (3.159807,1.5) -- ++(-20:3) node [right] {$o$} node [below, midway, rotate=-20, scale=0.7] {};
\draw (3.159807,-1.5) -- ++(20:3) node [right] {$a$} node [above, midway, rotate=20, scale=0.7] {};
\draw (3.159807,-1.5) -- ++(-20:3) node [right] {$o$} node [below, midway, rotate=-20, scale=0.7] {};
\end{tikzpicture}
\caption{Diagrama de árvore: representação das oito possibilidades de nascimentos de três filhos}
\end{figure}

{\small$S=\{(a,a,a), (a,a,o), (a,o,a), (a,o,o), (o,a,a), (o,a,o), (o,o,a), (o,o,o)\}$} com, por exemplo, $(a,o,a)$ indicando que dos três filhos, o primeiro foi uma menina, o segundo foi um menino e, o terceiro, uma menina. Observe que neste caso, $\#(S)=8$ e, se as probabilidades de nascer um menino e de nascer uma menina são iguais, é natural usar a interpretação clássica de probabilidade, atribuindo probabilidades iguais a cada um dos 8 eventos elementares deste espaço amostral. Lembre que eventos elementares são os subconjuntos unitários do espaço amostral.

\item Se formos pensar na quantidade de meninas do casal tem-se $S=\{0,1,2,3\}$. Observe que neste caso $\#(S)=4$, mas neste caso não será adequado atribuir probabilidades iguais aos eventos elmentares $\{0\}, \{1\}, \{2\}$ e $\{3\}$, pois claramente, os eventos $\{1\}$ e $\{2\}$ ocorrem com maior frequência e, portanto, com maior probabilidade. Veja as oito situações possíveis no item anterior e quantas delas correspondem a estes dois eventos elementares.
\end{enumerate}
}{0}
\end{answer}
\begin{objectives}{Avaliando probabilidades a partir de um histograma}
{
Avaliar probabilidades de eventos usando dados quantitativos coletados, representados em um histograma e em uma tabela de frequências.
}{1}{2}
\end{objectives}
\begin{sugestions}{Avaliando probabilidades a partir de um histograma}
{
A partir de dados coletados e representados por meio de um histograma, os alunos deverão responder perguntas referentes a probabilidades de certos eventos e para isso deverão usar a noção frequentista de probabilidade.
}{1}{2}
\end{sugestions}
\clearmargin
\marginpar{\vspace{.5em}}
\begin{answer}{Avaliando probabilidades a partir de um histograma}
{
Para responder as questões desta atividade será usada a interpretação frequentista de probabilidade.

\begin{enumerate}
\item Pela tabela de frequências observa-se que $19$ tempos de chegada dos $100$ melhores estão no intervalo dado, portanto, a probabilidade será $\frac{19}{100}=0{,}19$.

\item Pela tabela de frequências observa-se que há $100−24−19=57$ tempos de chegada entre os $100$ melhores que são inferiores a $154{,}0$ minutos. Portanto a probabilidade será $\frac{57}{100}=0{,}57$. A resposta poderia ser obtida também, somando-se as frequências dos intervalos de classe do início até o intervalo $[151{,}0 ; 154{,}0[$: $7+4+1+2+4+4+14+21=57$, obtendo-se $0{,}57$ como probabilidade.

\item Pela tabela, vemos que $152{,}5$ minutos não é extremo de intervalo na construção apresentada, desse modo podemos concluir que a probabilidade solicitada será um número no intervalo $]0{,}43 ; 0{,}64[$. Observe que $152{,}5$ está no intervalo $[151{,}0 ; 154{,}0[$ de comprimento 3, cuja frequência absoluta é $21$. Observe também que o subintervalo $[152{,}5; 154{,}0[$ de $[151{,}0 ; 154{,}0[$ tem comprimento $1{,}5$. Usando a suposição de proporcionalidade da frequência em cada intervalo por unidade de comprimento do intervalo, podemos a aproximar a frequência do subintervalo por $21\times\frac{1{,}5}{3}=10{,}5$. Logo, a probabilidade será obtida por $\frac{10{,}5+24+19}{100}=\frac{53{,}5}{100}=0{,}535$ que de fato é um número no intervalo inicialmente obtido.

\item Pela tabela, nem $152,5$ nem $158,0$ são extremos de intervalo, porém podemos obter um limite superior para a probabilidade solicitada, que neste caso será um número menor do que $0,64$ $\big(\frac{21+24+19}{100}\big)$. Para um valor pontual, serão necessárias duas aproximações de frequências. A primeira no intervalo $[152{,}5; 154{,}0[$ já realizada no item anterior que foi aproximada para $10{,}5$. A frequência do intervalo $[154{,}0 ; 157{,}0 [$ é $24$. Para o intervalo $[157{,}0 ; 158{,}0[$ calcularemos uma frequência aproximada, usando o argumento da proporcionalidade. O intervalo $[157{,}0 ; 160{,}0[$ tem comprimento $3$ com frequência $10$, então, uma aproximação para o subintervalo $[157{,}0 ; 158{,}0[$ de $[157{,}0 ; 160{,}0[$ de comprimento $1$ é dada por $19\times\frac{1}{3}\approx6{,}3$. Assim, a probabilidade será $\frac{10{,}5+24+6{,}3}{100}=\frac{40{,}8}{100}=0{,}408$.
\end{enumerate}
}{1}
\end{answer}
\begin{objectives}{Leis de De Morgan}
{
Reconhecer propriedades importantes de operações com conjuntos, envolvendo união, interseção e complementação.
}{1}{1}
\end{objectives}
\begin{answer}{Leis de De Morgan}
{
Veja a figura a seguir. No primeiro diagrama foi pintada a região correspondente ao complementar da interseção de $A$ e $B$. No segundo diagrama foi realçada em listras vermelhas a região que corresponde ao evento $\overline{A}$ e em listras azuis a região que corresponde ao evento $\overline{B}$. Depois, em cinza foi destacada a região correspondendo à união dos dois. Verifique que correspondem à mesma região.
\begin{figure}[H]
\centering

\includegraphics[width=\linewidth]{demorgan1.png}

\includegraphics[width=\linewidth]{demorgan2.png}
\caption{As leis de De Morgan podem ser enunciadas da seguinte forma:}
\label{}
\end{figure}

\begin{enumerate}
\item O complementar da interseção de dois eventos é igual a união dos complementares desses dois eventos.
\item O complementar da união de dois eventos é igual a interseção dos complementares desses dois eventos.
\end{enumerate}
}{0}
\end{answer}
\begin{objectives}{Distributividade}
{
Verificar a propriedade distribuitiva da operação de interseção de conjuntos com a operação de união de conjuntos.
}{1}{2}
\end{objectives}
\begin{sugestions}{Distributividade}
{
Dois diagramas de venn são apresentados. A ideia é realizar a verificação pintando os diagramas conforme a operação à esquerda e à direita da igualdade, para concluir que se referem ao mesmo conjunto.
}{1}{2}
\end{sugestions}
\begin{answer}{Distributividade}
{
\begin{figure}[H]
\centering

\includegraphics[width=\linewidth]{distributividade.png}
\caption{Diagramas de Venn: $(A\cap B)\cup C=(A\cup C)\cap(B\cup C)$}
\label{}
\end{figure}

No diagrama da esquerda tem-se em amarelo a região indicada pela expressão à esquerda na igualdade a ser verificada. No diagrama da direita tem-se $A\cup C$ em listras vermelhas, $B\cup C$ em listras azuis. A região da interseção dos dois é a que contêm listras azuis e vermelhas e está destacada na cor cinza. Observe que é a mesma região em amarelo do diagrama da esquerda.
}{1}
\end{answer}
\practice{Conceitos Básicos}
\begin{task}{espaço amostral não é único!}
\label{espaço-nao-unico}


Considere as famílias com três filhos no bairro onde você mora. Suponha que deseja-se calcular probabilidades do tipo: “qual a probabilidade de que uma dessas famílias com três filhos tenha dois meninos e uma menina?”.
\begin{enumerate}
\item {} 
Construa um espaço amostral adequado para calcular esta probabilidade, considerando as possíveis sequências de nascimentos dos três filhos na família.

\item {} 
Construa um outro espaço amostral, considerando a quantidade de meninas em cada casal de três filhos.

\end{enumerate}
\end{task}

\begin{task}{avaliando probabilidades a partir de um histograma}


No capítulo \hyperref[est2-chap]{\textbf{Medidas de posição e dispersão}} foram trabalhados os dados sobre os 100 melhores tempos atingidos na Maratona de Nova Iorque (2017) para as categorias homens e mulheres. Na \hyperref[maratona-homens-prob]{figura \ref{maratona-homens-prob}}, apresenta-se um histograma construído para os 100 melhores tempos na maratona de Nova Iorque (2017) para a categoria homens, após a conversão destes tempos para minutos.

\begin{figure}[H]
\centering

\begin{tikzpicture}[xscale=0.5,yscale=.8, scale=0.3]

\draw (-0.2,0) -- (33.5,0);
\draw (0,0) -- (0,25);

\foreach \x in {0,5,10,15,20,25}  \draw (0,\x) -- (-0.5,\x) node [above, rotate=90] at (-0.3,\x) {\x}  
;


\foreach \x/\y in {3/7,6/4,9/1,12/2,15/4,18/4,21/14,24/21,27/24,30/19}{ \draw [fill=\currentcolor!80] (\x,0) rectangle (\x+3,\y);
\node [above, align=center, scale=.9] at (\x+1.5,\y) {$(\y)$};}

\foreach \x/\y in {3/130,6/\quad,9/136,12/\quad,15/142,18/\quad,21/148,24/\quad,27/153,30/\quad,33/160} \draw (\x,0) -- (\x,-0.25) node [below] {\y};

\foreach \x in {0,5,10,15,20,25}  \draw [dashed] (0,\x) -- (33.5,\x)
;

\node [rotate=90] at (-4,12.5) {frequência absoluta};
\node  at (16.5,-3) {tempos em minutos};
\node [align=center] at (15.6, 27.5) {Histograma dos 100 melhores tempos na categoria \\ Maratona de Nova Iorque - 2017};


\end{tikzpicture}
\caption{Histograma dos 100 melhores tempos para homens na Maratona de Nova Iorque (2017), destacando a frequência absoluta de cada intervalo de classe.}
\label{maratona-homens-prob}
\end{figure}


Na \hyperref[maratonatabela]{
tabela \ref{maratonatabela}} são apresentados os intervalos de classe e suas respectivas frequências, usados na construção do histograma da \hyperref[maratona-homens-prob]{figura \ref{maratona-homens-prob}}. Os intervalos considerados são fechados à esquerda e abertos à direita.

\begin{table}[H]
\centering
\begin{tabu} to \textwidth{|c|c|c|}
\hline
\thead
\parbox[c][1cm]{3.5cm}{\centering Intervalo de classe} & \parbox[c][1cm]{3.5cm}{\centering Frequência Absoluta} & \parbox[c][1cm]{3.5cm}{\centering Frequência Relativa} \\
\hline\relax
$[130,0;133,0[$ & $7$ & $0{,}07$ \\
\hline
$[133,0;136,0[$ & $4$ & $0{,}04$ \\
\hline
$[136,0;139,0[$ & $1$ & $0{,}01$ \\
\hline
$[139,0;142,0[$ & $2$ & $0{,}02$ \\
\hline
$[142,0;145,0[$ & $4$ & $0{,}04$ \\
\hline
$[145,0;148,0[$ & $4$ & $0{,}04$ \\
\hline
$[148,0;151,0[$ & $14$ & $0{,}14$ \\
\hline
$[151,0;154,0[$ & $21$ & $0{,}21$ \\
\hline
$[154,0;157,0[$ & $24$ & $0{,}24$ \\
\hline
$[157,0;160,0[$ & $19$ & $0{,}19$ \\
\hline
\end{tabu}
\caption{Distribuição de frequências dos 100 melhores tempos na categoria homens da maratona de Nova Iorque (2017)}
\label{maratonatabela}
\end{table}

Suponha que o comportamento dos 100 melhores tempos para homens na Maratona de Nova Iorque (2017) represente bem os 100 melhores tempos para homens em qualquer Maratona de Nova Iorque.

Com base nessa suposição, estime a probabilidade de que na próxima maratona de Nova Iorque o tempo de conclusão da corrida, entre os 100 melhores na categoria homens,
\begin{enumerate}
\item {} 
ocorra entre $157{,}0$ e $160{,}0$ minutos;

\item {} 
seja inferior a $154{,}0$ minutos;

\item {} 
seja superior a $152{,}5$ minutos;

\item {} 
caia entre $152{,}5$ e $158$ minutos.

Observação: Para responder os dois últimos itens, suponha, em cada intervalo, que as frequências obervadas são proporcionais aos comprimentos dos intervalos, para poder avaliar frequências em subintervalos.

\end{enumerate}
\end{task}

\begin{task}{Leis de De Morgan}


Verifique, usando diagrama de Venn, as seguintes igualdades, conhecidas como as Leis de De Morgan. Sejam \(A\) e \(B\) dois conjuntos, então
\begin{enumerate}
\item {} 
\(\displaystyle{\overline{A\cap B}}=\overline{A}\cup \overline{B}\)

\item {} 
\(\overline{A\cup B}=\overline{A}\cap \overline{B}\)

\end{enumerate}
\end{task}
\begin{task}{Distributividade}


Verifique, usando os diagramas de Venn na \hyperref[diagramavenn3]{figura \ref{diagramavenn3}}, a propriedade distributiva da operação de união com a interseção de dois conjuntos.
\begin{equation*}
\begin{split}(A\cap B)\cup C=(A\cup C)\cap (B\cup C)\end{split}
\end{equation*}

\begin{figure}[H]
\centering
\begin{minipage}{0.35\textwidth}
\centering
\begin{tikzpicture}[scale=0.5, every node/.style={scale=1}]
\draw [,rounded corners=8pt,] (0,0) -- (0,7.5) -- (10,7.5) -- (10,0) -- cycle;
\node  at (9,8) {$S$};
\draw (3.5,2.5) circle (2cm);
\node  at (1,2) {$B$};
\draw (6.5,2.5) circle (2cm);
\node  at (9,2) {$C$};
\draw (5,5) circle (2cm);
\node  at (7.5,5.5) {$A$};
\end{tikzpicture}

\end{minipage}
\begin{minipage}{0.35\textwidth}
\centering
\begin{tikzpicture}[scale=0.5, every node/.style={scale=1}]

\draw [,rounded corners=8pt,] (0,0) -- (0,7.5) -- (10,7.5) -- (10,0) -- cycle;
\node at (9,8) {$S$};
\draw (3.5,2.5) circle (2cm);
\node at (1,2) {$B$};
\draw (6.5,2.5) circle (2cm);
\node at (9,2) {$C$};
\draw (5,5) circle (2cm);
\node at (7.5,5.5) {$A$};
\end{tikzpicture}

\end{minipage}
\caption{Diagramas de Venn com três conjuntos: \(A\), \(B\)  e \(C\)}
\label{diagramavenn3}
\end{figure}
\end{task}

\clearpage
\def\currentcolor{session1}
\begin{paginatexto}{Regras Básicas e Propriedades}{

Esta seção da unidade temática de Probabilidade tem como objetivo principal apresentar uma definição matemática de probabilidade, independente das interpretações apresentadas na seção 1, conhecida como definição axiomática da probabilidade. Os axiomas dessa definição serão, no livro, chamados de regras básicas da probabilidade, a saber,

\begin{enumerate}
\item a probabilidade de um evento é um número não-negativo,

\item a probabilidade do evento certo (espaço amostral) é $1$ (Dizemos que um evento ocorreu, se o resultado observado for um dos elementos do evento. Observe que sempre ocorrerá um elemento do espaço amostral que contém todos os resultados possíveis, por essa razão, ele costuma ser chamado de evento certo. Lembre-se que todo conjunto é subconjunto de si próprio e, portanto, o espaço amostral pode ser considerado como um evento, assim como o conjunto vazio.),

\item dados dois eventos disjuntos $A$ e $B$ ($A\cap B=\emptyset$), a probabilidade da união dos dois é dada pela soma das probabilidades $P(A)+P(B)$. De fato, na definição formal de medida de probabilidade, este terceiro axioma deve incluir uma coleção enumerável de eventos dois a dois disjuntos tal que a probabilidade da união enumerável desses eventos é dada pela soma enumerável das probabilidades de cada um deles. Este axioma é chamado axioma da sigma-aditividade da probabilidade. No entanto, entendemos que nesse nível de ensino, não será necessário apresentar a definição axiomática completa. Este último axioma, será estendido para uniões disjuntas (uniões de eventos disjuntos) de um número finito de eventos disjuntos dois a dois.
\end{enumerate}

A introdução da definição axiomática da probabilidade, a mesma será considerada sob cada uma das interpretações trabalhadas na seção inicial da unidade, de modo que na interpretação clássica, obteremos a probabilidade de um evento como a razão do número de elementos do evento sobre número de elementos do espaço amostral; na interpretação frequentista, obteremos a probabilidade de um evento como a frequência relativa de ocorrência desse evento quando repetimos o experimento um grande número de vezes sob as mesmas condições e na interpretação subjetiva destacaremos a importância de sermos coerentes com os axiomas, no sentido de não atribuir probabilidades maiores que 1 ou negativas, lembrando também do axioma 3. O mais importante nessa discussão é perceber que em qualquer uma delas valerão os três axiomas.

Em seguida trabalharemos com propriedades derivadas dos axiomas, a saber, 
\begin{enumerate}
\item a probabilidade do evento vazio ($P(\emptyset)=0$),
\item a probabilidade do evento complementar $(P(\overline{A})=1-P(A))$, 
\item se $A\subset B$, então $P(A)\leq P(B)$ e 
\item a probabilidade da união de dois eventos quaisquer $(P(A\cup B)=P(A)+P(B)-P(A\cap B))$.
\end{enumerate}

Nesta seção, será introduzida, embora de forma superficial, a questão da modelagem probabilística de um fenômeno, a partir de levantamento de dados. O intuito é mostrar como podemos usar dados de pesquisas para avaliar probabilidades. Basicamente, uma distribuição de frequências será usada para calcular probabilidades. Reforça-se a importância de, em tais situações, sempre destacar a suposição de que os dados observados refletem bem o comportamento da população cujas probabilidades ou estimativas serão calculadas a partir dos dados observados.

A seção é encerrada trabalhando-se com a noção de probabilidade geométrica para espaços amostrais contínuos. Nesse caso específico é importante frisar que estamos considerando como espaços amostrais contínuos conjuntos infinitos não-enumeráveis para os quais estão definidas medidas de comprimento ou área ou volume. Além disso, também estamos considerando que o modelo uniforme de probabilidade (função de densidade de probabilidade constante) é adequado para esses espaços. É claro que para o aluno, bastará mencionar que um ponto desse espaço será escolhido ao acaso de modo que subconjuntos de igual comprimento ou área ou volume do espaço amostral são equiprováveis. Nesses casos, dependendo da dimensão do espaço amostral (1, 2 ou 3), probabilidades serão calculadas como razão de comprimentos ou áreas ou volumes, respectivamente, observando-se a validade das regras básicas da probabilidade apresentadas no início da seção. É importante explorar nesse contexto as situações de eventos com infinitos elementos, mas cuja probabilidade é zero. Por exemplo, se o espaço amostral é um círculo de raio R, a probabilidade de sortearmos um ponto de um diâmetro fixado será zero, pois a medida da área definida pelo diâmetro é nula.

São objetivos específicos da seção 2:

\begin{OES}\setcounter{enumi}{5}
\item Definição axiomática - Aplicar as regras básicas da probabilidade nas diferentes interpretações de probabilidade.
\item Definição axiomática - Aplicar as regras básicas da probabilidade para obter as regras da probabilidade do evento complementar e a probabilidade da união de dois eventos (propriedades da probabilidade).
\item Modelagem probabilística - Avaliar, ainda que de forma incipiente, modelos para fenômenos aleatórios.
\item Aplicação - Aplicar as regras básicas da probabilidade em espaços amostrais contínuos (probabilidade geométrica).
\end{OES}
}
\end{paginatexto}
\begin{objectives}{Censo Educação Física}
{
Aplicar as regras básicas da probabilidade, usando a interpretação frequentista de probabilidade.
}{1}{1}
\end{objectives}
\begin{sugestions}{Censo Educação Física}
{
Aplicar as regras básicas da probabilidade para obter as regras da probabilidade do evento complementar e da probabilidade da união de dois eventos quaisquer.
}{1}{1}
\end{sugestions}
\clearmargin
\begin{answer}{Censo Educação Física}
{
\begin{enumerate}
\item $P(A)=\dfrac{270}{800}=0{,}3375, P(B)=\dfrac{240}{800}=0{,}3, P(C)=\dfrac{150}{800}=0{,}1875$ e $P(D)=\dfrac{140}{800}=0{,}175$.

\item De fato, $S=A\cup B\cup C\cup D$, com $A, B, C $ e $D$ dois a dois disjuntos. $1=P(S)=P(A\cup B\cup C \cup D)=P(A)+P(B)+P(C)+P(D)=0{,}3375$, o que era de se esperar em função das regras básicas da probabilidade.

\item Observe que podemos usar a interpretação clássica de probabilidade, considerando cada aluno da escola igualmente provável de ser escolhido. Como queremos a probabilidade de que o aluno pratique algum tipo de atividade física regular fora do período escolar, devemos contar quantos são estes alunos. Uma maneira mais simples de contar estes alunos é calcular a diferença entre o total de alunos ($800$) e o número de alunos que \textbf{não} praticam atividade física ($270$), obtendo-se $530$. Logo, a probabilidade solicitada é dada por $\frac{530}{800}=0{,}6625$. É claro que também poderíamos somar os números de alunos que praticam cada um dos tipos de atividade física: $240+150+140=530$. Mas, observe que a contagem obtida pelo cálculo da diferença entre o total e o número de alunos que não praticam atividade física foi mais simples: $800−270=530$.

\item  Sejam os eventos $B$: “jogar futebol”{} e $M$: “ser do turno da manhã”. Observe que $B\cap M\subset B$ tal que o número de elementos da interseção é menor ou igual ao número de elementos de $B$ de modo que o evento $B$ parece ser mais provável do que o evento $B\cap M$. Olhando os dados, temos que a probabilidade de que o aluno jogue futebol é dada por $P(B)=\frac{240}{800}=0{,}3$ e a probabilidade de jogar futebol e ser do turno da manhã é $P(B\cap M)=\frac{160}{800}=0{,}2$. Logo, é mais provável escolher um aluno que jogue futebol do que escolher um aluno do turna da manhã que jogue futebol. Essa é uma propriedade importante da probabilidade: se $A\subset B$, então $P(A)\leq P(B)$.

\item Chamando de $E$ o evento “praticar outra atividade diferente de futebol”{} e $T$ o evento “ser do turno da tarde”, queremos calcular $P(E\cup T)$. Da tabela de dados temos que o total de alunos que praticam outra atividade diferente de futebol é $150+140=290$ e o total dos alunos do turno da tarde é $350$. No entanto, a soma $290+350=640$ é superior ao número de alunos que tem pelo menos uma dessas duas características. Isso ocorre pois nos dois totais considerados, estamos contando os alunos que têm simultaneamente as duas características. Logo, devemos subtrair de $640$ o valor $70+70=140$ que representa o total de alunos que pratica outra atividade e é ao mesmo tempo do turno da tarde. Portanto, podemos escrever que
\begin{equation*}
P(E\cup T)=P(E)+P(T)-P(E\cap T)=\frac{290}{800}+\frac{350}{800}-\frac{140}{800}=\frac{500}{800}=0{,}625
\end{equation*}
\end{enumerate}
}{1}
\end{answer}
\explore{Regras Básicas e Propriedades}

No início do século XX, o matemático russo Kolmogorov, como já comentado na seção \hyperref[conceitosbasicos]{Explorando: Probabilidade \textendash{} conceitos básicos}, estabeleceu regras básicas para a probabilidade que independem da interpretação adotada, possibilitando assim, a construção de uma teoria matemática de probabilidade.

De maneira simplificada, essas regras básicas serão apresentadas a seguir.

Seja \(S\) um espaço amostral. Uma probabilidade é uma função \(P\) que associa a cada subconjunto de \(S\) (evento)  um número real, tal que
\begin{itemize}
\item {} 
ela é sempre um número não negativo,

\item {} 
a probabilidade do evento certo é igual a $1$ e,

\item {} 
dados dois eventos disjuntos, a probabilidade da união dos dois é dada pela soma das probabilidades individuais.

\end{itemize}

Em símbolos, essas regras podem ser apresentadas da seguinte forma:
\begin{itemize}
\item {} 
\(P(A)\geq 0\) qualquer que seja \(A\subset S\), ou seja, a probabilidade de qualquer evento \(A\) é um número não-negativo.

\item {} 
\(P(S)=1\), ou seja, a probabilidade do evento certo é igual a $1$.

\item {} 
Se \(A,B\subset S\) com \(A\) e \(B\) eventos disjuntos (\(A\cap B=\emptyset\)), então \(P(A\cup B)=P(A)+P(B)\).

\end{itemize}
\begin{task}{censo Educação Física}
\label{censo-educacao-fisica}

Em uma escola de Ensino Médio há dois turnos: manhã e tarde. No turno da manhã há $450$ alunos e, no turno da tarde, $350$ alunos. Os professores de Educação Física realizaram um censo para saber se os alunos da escola praticavam algum tipo de atividade física regular fora do período escolar. A pergunta principal do questionário da pesquisa foi:

\textbf{Qual é a sua atividade física principal fora do período escolar? Marque apenas uma opção.}


\begin{center}\textbf{({ }) Não pratica  ({ }) Futebol ({ }) Outra}\end{center}

Na \hyperref[frequenciaatividade]{tabela \ref{frequenciaatividade}} estão os resultados obtidos.
\begin{quote}

\end{quote}

\begin{table}[H]
\centering
\begin{tabu} to \textwidth{|c|c|c|c|}
\hline
\thead
Atividade Física & Manhã & Tarde & Total \\
\hline
Não pratica & $140$ & $130$ & $270$ \\
\hline
Futebol & $160$ & $80$ & $240$ \\
\hline
Natação & $80$ & $70$ & $150$ \\
\hline
Outra atividade & $70$ & $70$ & $140$ \\
\hline
Total & $450$ & $350$ & $800$ \\
\hline
\end{tabu}
\caption{Distribuição de frequências por atividade, segundo o turno.}
\label{frequenciaatividade}
\end{table}

Um aluno desta escola será escolhido ao acaso.
\begin{enumerate}
\item {} 
Considere os eventos  \(A\):{}”o aluno escolhido não pratica atividade física”, \(B\):”{} o aluno escolhido pratica Futebol como atividade física principal”, \(C\): ”o aluno escolhido pratica natação”{} e \(D:\):“o aluno pratica outro tipo de atividade física principal”. Determine a probabilidade de cada um desses eventos.

\item {} 
Observe que o espaço amostral nesse experimento corresponde à união dos eventos considerados no item anterior. Calcule a soma das probabilidades determinadas no item anterior. O resultado obtido é compatível com a segunda regra básica, a saber, \(P(S)=1\)? Por quê?

\item {} 
Qual é a probabilidade de que este aluno pratique algum tipo de atividade física regular fora do período escolar?

\item {} 
O que é mais provável: que o aluno escolhido seja do turno da manhã e jogue futebol ou que o aluno jogue futebol?

\item {} 
Qual é a probabilidade de que o aluno escolhido seja do turno da tarde \textbf{ou} pratique atividade física diferente de futebol, isto é, que o aluno escolhido tenha pelos menos uma dessas duas características?

\end{enumerate}
\end{task}


\arrange{Regras Básicas e Propriedades}

A última regra básica é chamada propriedade aditiva da probabilidade para eventos disjuntos e, de fato, é um caso particular da propriedade de aditividade da probabilidade.

Suponha três eventos \(A\), \(B\) e \(C\) disjuntos 2 a 2, ou seja, \(A\cap B=A\cap C=B\cap C=\emptyset\). A regra da aditividade para a união destes três eventos resultará em \(P(A\cup B\cup C)=P(A)+P(B)+P(C)\).
\begin{figure}[H]
\centering

\begin{tikzpicture}[scale=0.5]

\draw [,rounded corners=8pt, -] (0,0) -- (0,8) -- (10,8) -- (10,0) -- cycle;
\node at (9,8.5) {$S$};
\draw [fill=\currentcolor!80](2.5,2.2) circle (2cm);
\node  at (.75,.5) {$B$};
\draw [fill=\currentcolor!80](7.5,2.2) circle (2cm);
\node  at (9.25,.5) {$C$};
\draw [fill=\currentcolor!80](5,5.75) circle (2cm);
\node  at (7.5,5.5) {$A$};
\end{tikzpicture}

\caption{Três eventos A, B e C disjuntos dois a dois ilustrados no diagrama de Venn}

\end{figure}

A propriedade de aditividade da probabilidade para eventos disjuntos vale para qualquer coleção de eventos disjuntos $2$ a $2$.


\subsection{Interpretações da probabilidade e as regras básicas}

Como determinar probabilidades sob cada uma das interpretações apresentadas na seção \hyperref[organizandoconceitosbasicos]{Organizando as ideias: Probabilidade \textendash{} conceitos básicos}, considerando as regras básicas?

\paragraph{Interpretação clássica}

\begin{example}{}

Considere o lançamento de um dado honesto (todas as faces ocorrem com probabilidades iguais). Neste caso o espaço amostral é dado por

\(S=\{ 1,2,3,4,5,6\}\) e, os eventos elementares, são dados por
$$A_1=\{1\}, A_2=\{2\}, A_3=\{3\}, A_4=\{4\}, A_5=\{5\} \text{ e } A_6=\{6\}.$$

Faça \(P(A_i)=k\), \(i=1,2,3,4,5,6\).

Observe que \(S=A_1\cup A_2\cup A_3\cup A_4\cup A_5 \cup A_6\) e que os eventos \(A_1\), \(A_2\), \(A_3\), \(A_4\) , \(A_5\) e \(A_6\) são disjuntos.

Usando as regras básicas, tem-se
$$1=P(S)=P(A_1\cup A_2\cup A_3\cup A_4\cup A_5 \cup A_6)=P(A_1)+P(A_2)+\cdots +P(A_6)=6\cdot k$$

Logo, \(k=\dfrac{1}{6}\).
\end{example}

De modo mais geral, sob a interpretação clássica na qual o espaço amostral \(S\) é finito e todos os eventos elementares são equiprováveis, se o número de elementos do conjunto \(S\) é \(n\), \(n\in \mathbb{N}\), então a probabilidade de um evento elementar é dada por \(\dfrac{1}{n}\).

Nesse caso, se \(A\subset S\) tem-se $\displaystyle{P(A)=\frac{\#(A)}{n}}$, em que a notação \(\#(A)\) representa o número de elementos do conjunto \(A\).

\begin{observation}{ }

\textbf{Observação} Usando a interpretação clássica, na qual o espaço amostral \(S\) é um conjunto finito e todos os eventos elementares são equiprováveis, tem-se \(P(A)=\frac{\#(A)}{\#(S)}\).

Observe que as regras básicas são satisfeitas, pois
\begin{enumerate}
\item {} 
\(P(A)=\displaystyle{\frac{\#(A)}{\#(S)}}\geq 0\), qualquer que seja \(A \subset S\);

\item {} 
\(P(S)=\displaystyle{\frac{\#(S)}{\#(S)}=1}\) e,

\item {} 
se \(A\cap B=\emptyset\), tem-se que \(\#(A\cup B)=\#(A)+\#(B)\) tal que \(P(A\cup B)=P(A)+P(B)\).

\end{enumerate}

\textbf{Atenção:} Antes de sair usando essa interpretação, é necessário verificar se a suposição de eventos elementares equiprováveis é adequada. Por exemplo, vimos no item \titem{b)} da atividade \hyperref[espaço-nao-unico]{Espaço amostral não é único}, que \(S=\{0,1,2,3\}\) com quatro elementos. No entanto, esses elementos não são igualmente prováveis, pois os eventos elementares \(\{1\}\) e \(\{ 2\}\) são três vezes mais prováveis de ocorrer comparados aos eventos elementares \(\{0\}\) e \(\{3\}\).
\end{observation}

\paragraph{Interpretação frequentista}

Na interpretação frequentista atribuem-se probabilidades, usando-se as frequências relativas de ocorrência do evento depois de observar o mesmo experimento um grande número de vezes. Também podemos perceber com esta interpretação, que as regras da probabilidade valem, pois uma frequência relativa é sempre um número não-negativo. Se considerarmos o evento certo, é claro que sua frequência relativa de ocorrência será sempre 1, independente até da quantidade de vezes na qual o experimento é repetido. Finalmente, dados dois eventos disjuntos, a frequência relativa de ocorrência da união dos dois será dada pela soma das frequências relativas dos dois.

\paragraph{Interpretação subjetiva}

A validade das regras básicas na interpretação subjetiva de probabilidade depende de coerência com as regras básicas, quando se atribuem probabilidades. Por exemplo, para um espaço amostral \(S\) finito, as probabilidades atribuídas aos eventos elementares não devem ser valores fora do intervalo \([0,1]\). Além disso, a soma das probabilidades dos eventos elementares deverá ser igual a $1$.


\subsection{Propriedades da probabilidade}

A seguir, serão enumeradas algumas propriedades úteis no cálculo de probabilidades. Estas propriedades são consequências das regras básicas da probabilidade.

\begin{observationtitle}{Propriedade 1}
A probabilidade do evento vazio (\(\emptyset\)) é zero.
\end{observationtitle}

Esta propriedade é obtida das regras básicas \(P(S)=1\) e aditividade da probabilidade para eventos disjuntos, lembrando que \(S\cup \emptyset =S\) e \(S\cap \emptyset =\emptyset\).

Observe que de fato é natural que a probabilidade do evento vazio seja zero, pois um evento vazio nunca irá ocorrer.

\begin{observationtitle}{Propriedade 2} A probabilidade de um evento \(A\) pode ser calculada por \(P({A})= 1-P(\overline{A})\).
\end{observationtitle}

Muitas vezes pode ser complicado calcular diretamente a probabilidade de um evento. Uma possível simplificação será calcular a probabilidade do evento complementar. Um exemplo comum, é o problema dos aniversários que será apresentado na seção \hyperref[regrasbasicaspropriedades]{Praticando: Probabilidade - regras básicas e propriedades}. Uma aplicação desta propriedade está ilustrada no item \titem{c)} da atividade \hyperref[censo-educacao-fisica]{Censo Educação Física}.

\begin{observationtitle}{Propriedade 3} Dados \(A\) e \(B\) eventos em um espaço amostral \(S\) tais que \(A\subset B\) (\(A\) está contido em \(B\)), então \(P(A)\leq P(B)\).
\end{observationtitle}

Essa propriedade, ilustrada no item \titem{d)} da atividade \hyperref[censo-educacao-fisica]{Censo Educação Física}, pode ser obtida a partir das regras básicas:
como \(A\subset B\), podemos escrever o evento \(B\) como a união de dois eventos disjuntos, a saber,  \(B=A\cup (B\cap \overline{A})\). Veja uma ilustração na \hyperref[eventosdijuntos]{figura \ref{eventosdijuntos}}.
\begin{figure}[H]
\centering

\begin{tikzpicture}[scale=.75]

\draw [,rounded corners=8pt, -] (0,0) -- (0,7) -- (7,7) -- (7,0) -- cycle;
\node  at (6,7.3) {$S$};
\draw [fill=\currentcolor!80](3.5,3.5) circle (3cm);
\draw [fill=destacado!70] (5,3.5) circle (1cm);
\node [above right] at (1,6) {$B$};
\node at (5,3.5) {$A$};
\node at (3,2) {$\overline{A}\cap B$};
\end{tikzpicture}
\caption{\(B=A\cup (\overline{A}\cap B)\) como a união de dois eventos disjuntos}
\label{eventosdijuntos}
\end{figure}

\clearpage

Assim, usando a regra de que toda probabilidade é um número não-negativo e a regra básica de aditividade da probabilidade, tem-se
\begin{equation*}
\begin{split}P(B)=P(A\cup(\overline{A}\cap B))=P(A)+\underbrace{P(\overline{A}\cap B)}_{\geq 0}\geq P(A)\end{split}
\end{equation*}
\begin{observationtitle}{Propriedade 4}
\(P(A\cup B)=P(A)+P(B)-P(A\cap B)\), quaisquer que sejam os eventos \(A\) e \(B\) de um espaço amostral \(S\).
\end{observationtitle}

Essa propriedade, utilizada no item \titem{e)} da atividade \hyperref[censo-educacao-fisica]{Censo Educação Física}, pode ser obtida, escrevendo-se o evento \(A\cup B\) como a união de dois eventos disjuntos, a saber, \(A\cup B=A\cup (\overline{A}\cap B)\).

Veja uma ilustração na \hyperref[eventosdijuntos2]{figura \ref{eventosdijuntos2}}.
\begin{figure}[H]
\centering

\begin{tikzpicture}[scale=.85]
\node at (7,5.3) {$S$};
\draw [=,rounded corners=8pt, -] (0,-0) -- (0,5) -- (8,5) -- (8,0) -- cycle;
\node at (7,4) {$B$};
\node at (.5,4) {$A$};
\node at (4,-.5) {$A \cup B$};
\draw [fill=\currentcolor!80] (5,2.5) circle (2cm);
\node at (5.5 ,2) {$\overline{A}  \cap B$};
\draw [fill=\currentcolor!80] (2.5,2.5) circle (2cm) ;
\clip[draw] (5,2.5) circle (2cm);
\end{tikzpicture}

\caption{\(A\cup B\) como uma união de dois eventos disjuntos: \(A\cup (\overline{A}\cap B)\)}
\label{eventosdijuntos2}
\end{figure}

Assim, \(P(A\cup B)=P(A)+P(\overline{A}\cap B)\) , pois \(A\) e \(\overline{A}\cap B\) são disjuntos.

Mas, como \(B=(A\cap B) \cup (\overline{A}\cap B)\) e os eventos \(A\cap B\) e \(\overline{A}\cap B\) são disjuntos, segue que \(P(B)=P(A\cap B)+P(\overline{A}\cap B)\) (Veja a \hyperref[eventosdijuntos3]{figura \ref{eventosdijuntos3}}).

\begin{figure}[H]
\centering

\begin{tikzpicture}[scale=.85]

\node at (7,5.3) {$S$};
\draw [ ,rounded corners=8pt, -] (0,-0) -- (0,5) -- (8,5) -- (8,0) -- cycle;
\node at (6.5,4) {$B$};
\node at (1,4) {$A$};
\draw [fill= \currentcolor!80] (4.5,2.5) circle (2cm);
\node at (5.75,2.5) {$\overline{A}  \cap B$};
\clip [draw]  (3,2.5) circle (2cm) ;
\draw [fill= \currentcolor!40] (4.5,2.5) circle (2cm);
\node at (3.75,2.5) {$A\cap B$};
\end{tikzpicture}
\caption{\(B\) como a união de dois eventos disjuntos.}
\label{eventosdijuntos3}
\end{figure}

Logo, podemos escrever \(P(\overline{A}\cap B)=P(B)-P(A\cap B)\) tal que
\begin{equation*}
\begin{split}P(A\cup B)=P(A)+P(B)-P(A\cap B)\end{split}
\end{equation*}


\clearpage
\def\currentcolor{session2}
\begin{objectives}{O problema dos bodes}
{
Calcular a probabilidade de um evento considerando duas estratégias distintas.
}{1}{1}
\end{objectives}
\begin{sugestions}{O problema dos bodes}
{
Nesta atividades as interpretações clássica e frequentista podem ser usadas. De fato, a noção frequentista de probabilidade é usada muitas vezes para validar a suposição de equiprobabilidade de eventos elementares. Este problema gerou muita polêmica nos anos 1990 do século XX e é conhecido como problema de Monty Hall (incluir referência). Neste link é possível visualizar uma simulação do problema. (incluir link)

Uma sugestão para lidar com esta atividade é realizar a simulação do programa de televisão em sala de aula. Peça a turma para se dividir em dois grupos: um que adotará a estratégia de nunca trocar a escolha inicial e, o outro, de sempre trocar. A cada simulação do jogo deverá ser registrado, em dada grupo, se houve vitória ou derrota. Esta simulação pode ser feita com três cartas de versos idênticos, por exemplo. Sugere-se repetir a simulação pelo menos 30 vezes em cada grupo. No final compare com os alunos a frequência de vitórias em cada grupo e depois discuta sobre a melhor estratégia.

Caso esteja disponível internet e projetor, neste \href{https://www.geogebra.org/m/Ec9xubPJ}{link} será possível simular várias partidas do jogo e ir registrando os resultados em cada equipe.
}{1}{1}
\end{sugestions}
\begin{answer}{O problema dos bodes}
{
A melhor estratégia será trocar de porta. Observe que inicialmente, o candidato tem probabilidade $1/3$ de escolher a porta com o carro e $2/3$ de escolher uma porta com um bode. Assim, se a estratégia do candidato for manter a escolha inicial, sua chance de ganhar o carro será $1/3$.

Por outro lado, se a estratégia do candidato for trocar de porta, sua chance de ganhar o carro será $\frac{2}{3}$: se ele escolher a porta que tem o carro (com probabilidade $\frac{1}{3}$) e trocar de porta ele não ganhará. No entanto se ele escolher uma porta com bode (cuja probabilidade é $\frac{2}{3}$) e trocar de porta, ele ganhará. Logo, sob a estratégia “trocar de porta”, a probabilidade de ganhar o carro é $\frac{2}{3}$.

Veja neste \href{https://www.geogebra.org/m/Ec9xubPJ}{link} uma simulação deste problema
}{1}
\end{answer}
\begin{objectives}{Jogo de dardos}
{
Calcular probabilidades de eventos em situações cujo espaço amostral é uma região do plano, estendendo a noção clássica de probabilidade para um espaço amostral não discreto.
}{1}{2}
\end{objectives}
\begin{sugestions}{Jogo de dardos}
{
O objetivo desta atividade é estender a noção de probabilidade para uma situação envolvendo um espaço amostral não discreto e induzir à noção de probabilidade geométrica como razão de áreas em que uma região de área bem definida e finita do plano é fixada como o espaço amostral e, os eventos são tomados como sub-regiões de área bem definida do espaço amostral. A situação a ser tratada aqui é bem restrita, pois usará a noção clássica de probabilidade, adotando a suposição de sub-regiões do espaço amostral de áreas iguais têm probabilidades iguais.

Note que a frase “Suponha que você seja suficientemente experiente de modo que sempre atinja o tabuleiro de darodos”{} especifica que o espaço amostral resume-se ao tabuleiro de dardos.
}{1}{2}
\end{sugestions}
\clearmargin
\marginpar{\vspace{.5em}}
\begin{answer}{Jogo de dardos}
{
\begin{enumerate}
\item Para ganhar $100$ pontos o dardo deve cair no círculo menor de raio $5$cm (em verde). Logo, 
$$P(A)=\dfrac{\pi\cdot5^2}{\pi\cdot20^2}=\dfrac{1}{16}=0{,}0625$$.
\item Para ganhar $20$ pontos o dardo deve cair no anel circular verde mais externo. Logo, usando a definição 
\begin{equation*}
P(B)=\frac{\text{área de $B$}}{\text{área de $S$}}=\frac{\pi(15^2-10^2)}{\pi\cdot20^2}=\frac{125}{400}=0{,}3125
\end{equation*}
\item Para ganhar no máximo $50$ pontos, o dardo deve cair em qualquer ponto exceto no círculo menor em verde onde se ganha $100$ pontos. Logo, usando a propriedade que explicita a probabilidade do evento complementar, \begin{equation*}
P(C)=P(\overline{A})=1-P(A)=1-0{,}0625\approx0{,}9375
\end{equation*}.
\item Considerando o semicírculo destacado na \hyperref[dardos]{figura \ref{dardos}}:\begin{enumerate}
\item A região que determina o evento de interesse é a reunião de duas regiões, a saber, o semicírculo $(A)$ e o anel circular correspondente a faixa de $50$ pontos $(B)$. Logo, usando a propriedade da probabilidade da união de dois eventos tem-se 

\begin{align*}
P(A\cup B)&=P(A)+P(B)-P(A\cap B)\\
&=\frac{1}{2}+\frac{\pi\cdot(10^2-5^2)}{\pi\cdot20^2}-\frac{(\pi/2)\cdot(10^2-5^2)}{\pi\cdot20^2}\\
&=\frac{1}{2}+\frac{75}{400}-\frac{75}{800}\approx0{,}59.
\end{align*}

\item A região que determina o evento de interesse $(D)$ corresponde a um semicírculo de raio $15$cm. $P(D)=\dfrac{1}{2}\cdot\dfrac{\pi\cdot15^2}{\pi\cdot20^2}=\dfrac{9}{32}\approx0{,}28$.
\end{enumerate}
\end{enumerate}
}{1}
\end{answer}
\clearmargin
\begin{objectives}{O problema dos aniversários}
{
Calcular a probabilidade de um evento usando a propriedade do evento complementar.
}{1}{2}
\end{objectives}
\begin{sugestions}{O problema dos aniversários}
{
Para resolver esse problema será necessário fazer algumas suposições: 
\begin{itemize}
\item considerar apenas anos não bissextos; 
\item supor que os $365$ dias do ano são igualmente prováveis como datas de aniversário. Com essas suposições, será adotada a interpretação clássica de probabilidade para resolver o problema.
\end{itemize}

Faça uma enquete perguntando quem são os nascidos em janeiro, fevereiro, etc., até obter uma coincidência (ou não) de aniversários. Se a sua turma tem $35$ ou mais alunos, é muito mais provável que exista uma coincidência de aniversários do que não exista. A explicação para isso envolve o fato de que com $35$ pessoas pode-se formar $595$ pares de datas de aniversário de duas pessoas diferentes, ao passo que no ano há $365$ dias como possíveis datas de aniversário. Para calcular a probabilidade também será necessária pelo menos uma calculadora científica básica.

É importante destacar que a escolha do número $35$ alunos deveu-se à restrição numérica nas calculadoras científicas básicas. Para números maiores do que $35$, as calculadoras apontam “Erro matemático”{} por conta da magnitude de valores manipulados no cálculo da probabilidade, que superam a capacidade de uma calculadora simples. No entanto, usando programação tais probabilidades podem ser facilmente obtidas para números maiores do que $4$. No link há uma ilustração sobre o comportamento dessa probabilidade em função do tamanho do grupo.
}{1}{2}
\end{sugestions}
\begin{answer}{O problema dos aniversários}
{
Calcular diretamente essa probabilidade é muito complicado, pois existem várias configurações possíveis de coincidências de aniversários. Por outro lado, podemos pensar no evento complementar ao evento cuja probabilidade queremos calcular. O evento complementar corresponde ao evento “todos os alunos da turma nasceram em dias diferentes”. Vamos usar a interpretação clássica de probabilidade supondo que todas as configurações possíveis de aniversário para as 40 pessoas são igualmente prováveis e, assim, 
\begin{equation*}
P(\overline{A})=\frac{\#(\overline{A})}{\#(S)}
\end{equation*}. 

Observe que para cada pessoa existem $365$ possibilidades de datas. Como são 35 pessoas, tem-se $\#(S)=365^35$. Agora o número de elementos do evento $\overline{A}$, que corresponde a todos terem nascido em dias diferentes, pode ser calculado, usando o princípio multiplicativo, da seguinte forma: há $365$ possibilidades para o aluno número $1$ da chamada, logo há $(365-1)=364$ possibilidades para o aluno número $2$ da chamada. Continuando, há $(365-2)=363$ possibilidades para o aluno número $3$ da chamada, até $(365-34)=331$ possibilidades para o aluno de número 35 da chamada. Assim, 
\begin{equation*}
\#(\overline{A})=365\cdot364\cdot363\cdots331=\frac{365!}{(365-34)!}.
\end{equation*}

Com uma calculadora científica básica, é possível obter o valor de 
\begin{equation*}
P(\overline{A})=\frac{365!}{(365−35)!}{365^35}
\end{equation*}
que é, aproximadamente, $0{,}814$. A título de informação veja na tabela a seguir as probabilidades de coincidência em função do tamanho do grupo $k$.

\begin{table}[H]
\centering

\setlength\tabcolsep{7.5pt}
\begin{tabular}{|f|f|}
\hline
\tmat{k} & \tmat{P(A)} \\
\hline
5 & 0{,}027 \\
\hline
10 & 0{,}117 \\
\hline
20 & 0{,}411 \\
\hline
22 & 0{,}476 \\
\hline
23 & 0{,}507 \\
\hline
30 & 0{,}569 \\
\hline
40 & 0{,}891 \\
\hline
50 & 0{,}970 \\
\hline
60 & 0{,}994 \\
\hline
\end{tabular}
\end{table}
}{0}
\end{answer}

\practice{Regras Básicas e Propriedades}
\label{regrasbasicaspropriedades}\begin{task}{o problema dos bodes}


Em um programa de televisão semanal, um jogo oferece como prêmio um automóvel a um espectador escolhido da plateia. O candidato a ganhar o automóvel é convidado pelo apresentador do programa a escolher uma entre três portas idênticas, atrás das quais há um carro em uma delas e, nas outras duas, há um bode em cada uma.

\begin{figure}[H]
\centering

\noindent\includegraphics[width=250bp]{bode}
\caption{Problema dos bodes}
\end{figure}


Depois de o candidato escolher a porta, o apresentador, que sabe o que tem atrás de cada uma delas, abre uma das portas não escolhidas, mostrando que atrás dela tem um bode. Então, o apresentador oferece ao candidato decidir entre manter sua escolha inicial ou trocar de porta.

Qual deve ser a melhor estratégia para o candidato (trocar ou não trocar de porta) de modo que a sua probabilidade de ganhar o automóvel seja a maior possível?
\end{task}
\begin{task}{jogo de dardos}



No jogo de dardos o vencedor é quem zera os seus pontos mais rapidamente. Você começa, por exemplo, com um total de $200$ pontos. A cada lançamento do dardo, dependendo do local atingido, você ganha uma certa pontuação que é descontada do seu total. Se você for o primeiro a zerar, será o vencedor do jogo.

Quanto mais próximo do centro do tabuleiro de dardos (um tabuleiro circular conforme a \hyperref[dardos]{figura \ref{dardos}}), mais pontos você ganha.

Suponha que você seja suficentemente experiente de modo que todos os seus lançamentos atingem o tabuleiro de dardos.

\begin{figure}[H]
\centering


\begin{tikzpicture}[scale=1.5]

\draw [fill= secundario!70] (0,0) circle (2);
\draw [fill= white!70] (0,0) circle (1.85);
\draw [fill= \currentcolor!70] (0,0) circle (1.55);
\draw [fill= white!70] (0,0) circle (1);
\draw [fill= \currentcolor!70] (0,0) circle (.5);
\draw [fill= destacado!70] (0,0) circle (.1);
\draw [fill=secundario] (1.4142,1.4142) arc (45:-135:2);
\draw [fill=secundario!7] (1.30814,1.30814) arc (45:-135:1.85);
\draw [fill=\currentcolor] (1.0960,1.0960) arc (45:-135:1.55);
\draw [fill=secundario!7] (.70710,.70710) arc (45:-135:1);
\draw [fill=\currentcolor] (.35355,.35355) arc (45:-135:.5);
\draw [fill=destacado] (.070710,.070710) arc (45:-135:.1);
\draw [,color=blue!30!\currentcolor, fill=blue!30!\currentcolor] (1.8,1.8)--++(15:.2) --(2.19318,2.05176) -- (2,2);
\draw [,color=blue!30!\currentcolor, fill=blue!30!\currentcolor] (1.8,1.8)--++(75:.2) --(2.05176,2.19318) --(2,2);
\draw [] (0,0) -- (2.04,2.04);
\draw [secundario!7] (-1.29814,-1.29814)--(-1.1060,-1.1060);
\draw [secundario!7] (-.69710,-.69710)--(-.36355,-.36355);
\draw [destacado] (-.066710,-.066710)--(0,0);
\end{tikzpicture}
\caption{Tabuleiro de jogo de dardos}
\label{dardos}
\end{figure}

Suponha que a medida do raio do tabuleiro de dardos seja $20$cm e que a medida do menor raio (círculo em verde no centro do tabuleiro) seja $5$cm, e que os acréscimos de comprimento do raio nas faixas branca, verde e branca do tabuleiro sejam iguais a $5$cm. A moldura em preto não faz parte do alvo. Suponha também que atingindo o
\begin{itemize}
\item {} 
círculo de raio $5$cm (em verde), você ganha $100$ pontos;

\item {} 
o anel cicular mais próximo ao centro (em branco), você ganha $50$ pontos;

\item {} 
o anel circular em verde subsequente, você ganha $20$ pontos e

\item {} 
a anel circular mais externo (em branco), você ganha $10$ pontos.

\end{itemize}


Observe que, neste caso, não é possível usar a interpretação clássica de probabilidade, pois existem infinitos eventos elementares. No entanto, é razoável supor uniformidade de probabilidades se, de fato, o jogador acerte em qualquer ponto do tabuleiro de dardos ao acaso. Neste espaço amostral, o círculo de raio $20$cm, se os pontos são obtidos ao acaso, ao considerar regiões de mesma área, contidas no círculo, as probabilidades de se obter pontos nestas regiões devem ser iguais.

Assim, calcula-se a probabilidade do dardo cair numa região dentro do círculo como o quociente entre a medida da área da região sobre a medida da área do círculo (espaço amostral), isto é, se
\begin{equation*}
\begin{split}A\subset S \text{, então } P(A)=\displaystyle{\frac{\text{Área de }A}{\text{Área de }S}}\end{split}
\end{equation*}

\begin{figure}[H]
\centering

\begin{tikzpicture}[scale=0.6]  

\draw [thick] (0,0) circle (5cm);
\draw [fill = \currentcolor!50, thick]  (1.5,1.5) circle (2cm);
\node at (2,2) {$A$};

\node at (2.5,5) {$S$};
\end{tikzpicture}


\caption{Exemplo de um evento \(A\) no tabuleiro de dardos}
\end{figure}

Observações:
\begin{itemize}
\item {} 
Nesta situação, a probabilidade do dardo atingir um ponto fixado no círculo será sempre zero, pois a medida de área correspondente a um ponto é nula.

\item {} 
Esta forma de calcular probabilidades costuma ser denominada como \textit{probabilidade geométrica} e pode ser considerada como uma extensão da interpretação clássica de probabilidade para espaços amostrais representados por uma região do plano com área definida. Esta mesma noção poderá ser usada para espaços amostrais representados por intervalos da reta limitados de comprimento definido, neste caso, calculando-se probabilidades como uma razão de comprimentos de intervalos.

\end{itemize}

Calcule a probabilidade de que em um lançamento você ganhe
\begin{enumerate}
\item {} 
exatamente $100$ pontos;

\item {} 
exatamente $20$ pontos;

\item {} 
no máximo $50$ pontos.

\item {} 
Suponha também que pode ser combinado, antes do início do jogo, conceder um bônus adicional de $10\%$ da pontuação, se o dardo atingir o semicírculo, destacado na \hyperref[dardos]{figura \ref{dardos}}. Calcule a probabilidade de que em um lançamento você atinja
\begin{enumerate}
\item {} 
o semicírculo destacado ou uma faixa de exatamente 50 pontos;

\item {} 
o semicírculo destacado e uma faixa de pelo menos 20 pontos.

\end{enumerate}

\end{enumerate}
\end{task}

\begin{task}{o problema dos aniversários}


Numa turma de seu colégio há $35$ alunos. Calcule a probabilidade de que haja pelo menos uma coincidência de datas de aniversário (dia e mês) entre os alunos dessa turma. Considere apenas anos não bissextos e suponha que todos os $365$ dias do ano são igualmente prováveis como datas de aniversário.

\begin{figure}[H]
\centering

\noindent\includegraphics[width=200bp]{bolo.jpg}
\end{figure}
\end{task}

\clearpage
\begin{paginatexto}{Probabilidade Condicional}

Nesta seção da unidade de probabilidade todas as regras e propriedades da probabilidade trabalhadas na seção anterior serão retomadas, porém em um contexto de revisão da probabilidade de um evento, sabendo-se que algum outro evento tenha ocorrido. Após explorar uma situação que envolva essa revisão, apresentaremos definição de probabilidade condicional que sirva para todas as interpretações trabalhadas dada por $P(A|E)=\dfrac{P(A\cap E)}{P(E)},  P(E)>0.$

É importante verificar que, de fato, essa definição, satisfaz a definição axiomática de probabilidade $(P(A|E)\geq0$, para todo evento $A\subset S$, $P(S|E)=1$, se $A\cap B=\emptyset$, $P(A\cup B|E)=P(A|E)+P(B|E))$. Assim, a partir de tal verificação, será possível concluir que as propriedades (evento vazio, evento complementar e união de dois eventos quaisquer) serão válidas para a probabilidade condicional fixado o evento ocorrido tal que ($P(\emptyset|E)=0$, $P(\overline{A}|)=1−P(A|E)$, se $A\subset B$, então $P(A|E)\leq P(B|E)$ e $P(A\cup B|E)=P(A|E)+P(B|E)−P(A\cap B|E$)).

Na discussão sobre probabilidade condicional será importante mostrar para o aluno que a probabilidade de um evento $A$ condicionada a um evento $E$ poderá não se alterar ou se alterar, aumentando ou diminuindo, ou $P(A|E)=P(A)$ ou $P(A|E)>P(A)$ ou $P(A|E)<P(A)$. Essa discussão será usada para definir eventos independentes e dependentes. Se a probabilidade de $A$ não se altera dado que E ocorreu $[P(A|E)=P(A)]$, dizemos que os eventos $A$ e $E$ são independentes, pois a ocorrência de $E$ não interfere na incerteza a cerca do evento $A$. É importante explorar a simetria dessa definição, ou seja, se $P(A|E)=P(A)$, então $P(E|A)=P(E)$ tal que se a ocorrência de $E$ não altera a probabilidade de ocorrência de $A$, os eventos $E$ e $A$ são ditos eventos independentes.

Se $P(A|E)<P(A)$, dizemos que o evento $E$ é desfavorável a ocorrência de $A$, pois $P(A|E)<P(A)$ (Observe que essa relação também é simétrica, pois se $E$ é desfavorável ao evento $A$, então $A$ é desfavorável ao evento $E$.)

Se $P(A|E)>P(A)$, dizemos que o evento E é favorável a ocorrência de $A$, pois $P(A|E)>P(A)$. (Observe que essa relação também é simétrica, pois se E é favorável ao evento $A$, então $A$ é favorável ao evento $E$.)

Da definição de probabilidade condicional, podemos obter a regra da multiplicação para calcular a probabilidade da ocorrência simultânea de dois eventos quaisquer, a saber, $P(A\cap B)=P(B)\cdot P(A|B)$.

Para explorar essa regra, recomenda-se fortemente o uso de diagramas de árvore. Por exemplo, suponha o sorteio, sem reposição, de dois alunos de uma turma na qual $5$ são meninas e $11$ são meninos. Deseja-se calcular a probabilidade de que o primeiro aluno sorteado seja uma menina e, que o segundo, seja um menino.

Defina os eventos $A_1$: “menina no primeiro sorteio”, $A_2$: “menina no segundo sorteio”, $B_1$: “menino no primeiro sorteio”{} e $B_2$: “menino no segundo sorteio”.

Defina os eventos $A_1$: “menina no primeiro sorteio”, $A_2$: “menina no segundo sorteio”, $B_1$: “menino no primeiro sorteio”{} e $B_2$: “menino no segundo sorteio”.

Pela regra da multiplicação, $P(A_1\cap B_2)=P(A_1)\cdot P(B_2|A_1)=\frac{5}{16}\cdot\frac{11}{15}=\frac{11}{48}$.

Na \hyperref[exemploarvore1]{figura \ref{exemploarvore1}} 1, há uma iluistração desse problema, usando o diagrama de árvore.
\begin{figure}[H]
\centering

\includegraphics[width=.8\linewidth]{exemploarvore1.png}
\caption{Exemplo do uso do diagrama de árvore para resolver um problema de probabilidade}
\label{exemploarvore1}
\end{figure}

Observe que as probabilidades envolvendo as ramificações a partir de um ponto no diagrama da árvore sempre somam $1$ (axioma $2$ da probabilidade). No primeiro sorteio, a probabilidade de ser uma menina é $5/16$ e de ser um menino é $11/16$. Para o segundo sorteio, consideram-se as probabilidades condicionais, dado o ocorrido no primeiro sorteio:

\begin{itemize}
\item se foi menina, sobraram $15$ alunos, sendo $4$ meninas e $11$ meninos.
\item se foi menino, sobraram $15$ alunos, sendo $5$ meninas e $10$ meninos.
\end{itemize}

Nesse exemplo, podemos perguntar aos alunos se os eventos $A_1$ e $B_2$ são independentes. Observe que $P(A_1)=\frac{5}{16}$ e $P(B_2)=P(A_1\cap B_2)+P(B_1\cap B_2)=\frac{5}{16}\cdot \frac{11}{15}+\frac{11}{16}\cdot\frac{10}{15}=\frac{11}{16}$ (de fato, a probabilidade incondicional de ser um menino no segundo sorteio é dada pela proporção de meninos na turma) e $P(A_1\cap B_2)=\frac{11}{48} $tal que $P(B_2|A_1)=\frac{11/48}{5/16}=\frac{11}{15}\neq P(B_2)=\frac{11}{16}$. Logo, conclui-se que os eventos $A_1$ e $B_2$ não são independentes.

Esse exemplo é importante, para mostrar que a regra da multiplicação vale para quaisquer dois eventos. Se, no entanto, os eventos são independentes, então teremos o caso especial em que $P(A\cap B)=P(A)\cdot P(B)$. Deve-se ter cuidado com esta última expressão, adequada apenas para dois eventos independentes. A regra da multiplicação é dada por $P(A\cap B)=P(B)\cdot P(A|B)$ . (Usar como regra geral que $P(A\cap B)=P(A)\cdot P(B)$ é um distrator, possivelmente causado pela forma como os textos costumam apresentar o conceito de independência, priorizando essa expressão, sem olhar a regra geral, derivada da definição de probabilidade condicional.)

No encerramento desta seção, apresenta-se extensão da regra da multiplicação para mais de dois eventos quaisquer.

É possível, por indução, generalizar a regra da multiplicação para uma coleção de n eventos $(A_1,A_2,...,A_n)$ tais que $P(A_1\cap A_2\cap \cdots A_n)>0$, n∈N. Existem formas diferentes de apresentar a regra, mas utilizaremos aqui a sequência crescente de índices dos eventos

{\small
\begin{align*}
&P(A_1\cap A_2\cap \cdots\cap An)=P(A_1)\cdot P(A_2|A_1)\cdot P(A_3|A_1\cap A_2)\\
&\cdots P(A_n|A_1\cap A_2\cap\cdots\cap A_n−1)
\end{align*}}

Veja na \hyperref[exemploarvore2]{figura \ref{exemploarvore2}} o diagrama de árvore correspondente.
\begin{figure}[H]
\centering

\includegraphics[width=.75\linewidth]{exemploarvore2.png}
\caption{Exemplo do uso do diagrama de árvore para calcular probabilidades conjuntas}
\label{exemploarvore2}
\end{figure}

A extensão da regra da multiplicação para uma coleção finita de eventos pode ser simplificada para o caso especial de uma coleção de eventos independentes. Nesse caso, tem-se a expressão $P(A_1\cap A_2\cap\cdots\cap A+n)=P(A_1)\cdots P(A_2)\cdots P(A_n)$, útil para calcular probabilidades de experimentos repetidos independentemente, como, por exemplo, o lançamento de uma moeda honesta $10$ vezes consecutivas.

São objetivos específicos da seção 3:

\begin{OES}
\item Probabilidade condicional - Reconhecer que a probabilidade de um evento pode ser alterar dada a ocorrência prévia de outro evento
\item Independência de eventos - Aplicar a definição de probabilidade condicional para reconhecer eventos independentes e eventos dependentes.
\item Independência de eventos - Aplicar a definição de probabilidade condicional no cálculo da probabilidade da interseção de dois eventos quaisquer.
\item Eventos sequenciais - Entender que a probabilidade da ocorrência simultânea de um número finito de eventos pode ser calculada como o produto de probabilidades adequadas.
\end{OES}
\end{paginatexto}


\def\currentcolor{session1}
\begin{objectives}{Uso de óculos e sexo de estudandes}
{
\begin{itemize}
\item Reconhecer que a probabilidade de um evento pode se alterar, conhecendo-se uma informação parcial do fenômeno sob investigação.
\item Aplicar a definição de probabilidade condicional para reconhecer eventos independentes e eventos dependentes.
\end{itemize}
}{1}{1}
\end{objectives}
\begin{sugestions}{Uso de óculos e sexo de estudandes}
{
Nesta atividade uma tabela de dupla entrada será fornecida para verificar uma possível relação entre usar óculos e sexo de um estudante do Ensino Médio. Recomenda-se construir esta mesma tabela com os dados dos alunos de sua turma e responder aos itens, usando esses dados.

Como sugestão de discussão, sugere-se uma pesquisa na internet para investigar a proporção de jovens que usa óculos. Por exemplo, em 21 de maio de 2018, foi publicada a seguinte reportagem "\href{https://www.dn.pt/sociedade/interior/miopia-aumenta-nos-jovens-e-a-culpa-e-da-falta-de-sol-e-dos-computadores-5656709.html}{Miopia aumenta nos jovens e a culpa é da falta de sol e dos computadores}”
}{1}{1}
\end{sugestions}
\begin{answer}{Uso de óculos e sexo de estudandes}
{
Defina os seguintes eventos

\begin{itemize}
\item $M$: “estudante do gênero masculino”, 
\item $F$: “estudante do sexo feminino”,
\item $O$: “estudante usa óculos”{} e 
\item $\overline{O}$: “estudante não usa óculos”.
\end{itemize}

\begin{enumerate}
\item $P(O)=\frac{11}{40}=0{,}275$
\item Há $22$ alunos do gênero feminino e 6 usam óculos. Assim, a probabilidade é $\frac{6}{22}\approx0{,}273$.
\item Há $18$ alunos do gênero masculino e $5$ usam óculos. Assim, a probabilidade é $\frac{5}{18}\approx0{,}278$.
\item $P(F)=\frac{22}{40}=0{,}55$
\item Há $11$ alunos que usam óculos e 6 são do gênero feminino. Assim, a probabilidade é $\frac{6}{11}\approx0{,}545$.
\item Há $29$ alunos que não usam óculos e $16$ são do gênero feminino. Assim, a probabilidade é $\frac{16}{29}\approx0{,}552$.
\item Sem discriminar por gênero, a probabilidade de usar óculos é $0{,}275$. Discriminando por sexo, observa-se que a probabilidade de uma menina usar óculos é $0{,}272$ e de um menino usar óculos é $0{,}278$. Sem discriminar por uso de óculos, a probabilidade de ser uma menina é $0{,}55$. Discriminando por uso de óculos, observa-se que a probabilidade de ser uma menina entre os alunos que usam óculos é $0{,}545$ e a probabilidade de ser uma menina entre os que não usam óculos é $0{,}552$, ou seja, ambas aproximadamente iguais a probabilidade de ser uma menina sem levar em conta o uso de óculos. Portanto, independentemente, de sexo, a probabilidade de usar óculos é aproximadamente $0{,}275$ e independentemente de uso de óculos, a probabilidade de ser uma menina é aproximadamente $0{,}55$. Como estamos lidando com uma amostra da população, probabilidades gerais, por exemplo “uso de óculos”{} e probabilidades parciais “uso de óculos segundo o sexo”{} aproximadamente iguais revelam a independência no sentido estatístico dos eventos considerados.
\end{enumerate}
}{0}
\end{answer}
\explore{ Probabilidade Condicional}


\begin{task}{uso de óculos e sexo de estudandes}
\label{uso-oculos}

Na tabela a seguir estão os dados de uma turma de segundo ano do Ensino Médio com 40 alunos quanto ao gênero e se ele usa ou não óculos.

\begin{table}[H]
\centering
\begin{tabu} to \textwidth{|l|c|c|c|}
\hline
\thead
gênero & usa óculos & não usa óculos & total \\
\hline
feminino & $6$ & $16$ & $22$ \\ 
\hline
masculino & $5$ & $13$ & $18$ \\
\hline
total & $11$ & $29$ & $40$ \\
\hline
\end{tabu}
\end{table}

Se um estudante desta turma é sorteado, pede-se determinar a probabilidade de que ele
\begin{enumerate}
\item {} 
use óculos;

\item {} 
use óculos, sabendo que é do gênero feminino;

\item {} 
use óculos, sabendo que é do gênero masculino;

\item {} 
seja do gênero feminino;

\item {} 
seja do gênero feminino, sabendo que usa óculos;

\item {} 
seja do gênero feminino, sabendo que não usa óculos.

\item {} 
Analisando os dados da tabela e as respostas obtidas, há razões para supor que gênero é independente de uso de óculos ou não? Por quê?

\end{enumerate}
\end{task}


\arrange{Probabilidade Condicional}


\paragraph{Definição de probabilidade condicional}

Em Ciência, “informação disponível”{}é certamente uma matéria-prima preciosa, pois através dela é possível construir modelos mais realísticos para descrever fenômenos tanto determinísticos quanto aleatórios e obter resultados mais fidedignos de um ponto de vista da aplicação. Assim, quanto mais informação dispomos sobre determinados fenômenos, mais acurados serão potencialmente nossos modelos. É nesse sentido que surge historicamente o conceito de probabilidade condicional, tema dessa seção.

A ideia central da probabilidade condicional é estabelecer uma estrutura matemática para reavaliar a probabilidade de um evento à luz de uma informação disponível relacionada a este. Por exemplo, um geólogo, ao examinar uma bacia, avaliará a probabilidade de potencial petrolífero de forma diferente a depender de informações disponíveis, tais como porosidade da rocha, estruturas sísmicas, etc. Quanto mais informação ele tenha, tanto mais próxima da realidade será potencialmente sua avaliação da probabilidade de encontrar óleo na região.

\begin{figure}[H]
\centering

\noindent\includegraphics[width=200bp]{{bacia_petroleo}.png}

\caption{Bacias de petróleo na costa brasileira}
\end{figure}

O mesmo se dá em avaliações médicas: a partir da anamnese do paciente, um médico melhorará sua avaliação sobre a probabilidade de um paciente ter ou não determinada patologia.

\begin{figure}[H]
\centering

\noindent\includegraphics[width=200bp]{{anamnese}.png}

\caption{Medição da pressão arterial para anamnese do paciente}
\end{figure}


A questão que se coloca é: como incorporar matematicamente a informação de que um evento $B$ ocorreu para se reavaliar a ocorrência de um evento de interesse $A$?

A definição de probabilidade condicional é a resposta para essa questão.

\begin{observation}{}
A probabibilidade condicional de o evento \(A\) ocorrer, dado que sabemos que o evento \(B\)  ocorreu, denotada por \(P(A|B)\),  é definida por
\begin{equation*}
\begin{split}P(A|B)=\frac{P(A\cap B)}{P(B)}, \quad P(B)>0\end{split}
\end{equation*}
\end{observation}

Retomando a a atividade \hyperref[uso-oculos]{uso de óculos e sexo de estudantes}, a probabilidade de o aluno sorteado usar óculos, sabendo que ele é do gênero masculino foi calculada pela razão do número de estudantes que usam óculos e são do gênero masculino e o número de estudantes do gênero masculino, a saber, \(\frac{5}{18}\approx 0{,}278\).

Observe que esse quociente, pode ser também obtido a partir da definição de probabilidade condicional, calculando-se o quociente da probabilidade de “usar óculos e ser do gênero masculino”{} $(5/40=0{,}125)$ e da probabilidade de ser do gênero masculino $(18/40=0{,}45)$, obtendo-se \(\displaystyle{\frac{0,125}{0,45}=\frac{5}{18}\approx 0{,}278}\).

Repita essa verificação para as demais probabilidades condicionais calculadas na atividade \hyperref[uso-oculos]{uso de óculos e sexo de estudantes}.

\begin{example} {A probabilidade condicional é uma probabilidade?}

É possível verificar que a probabilidade condicional, \textbf{dado o conhecimento da ocorrência do evento B}, satisfaz as regras básicas da probabilidade e, portanto, também satisfaz as demais propriedades da probabilidade trabalhadas na seção anterior.

A primeira regra básica é a de que toda probabilidade é um número não negativo. De fato, tem-se que dado um evento \(A\subset S\) qualquer,
\begin{equation*}
\begin{split}P(A|B)=\frac{\overbrace{P(A\cap B)}^{\geq 0}}{\underbrace{P(B)}_{>0}}\geq 0\end{split}
\end{equation*}
Além disso, a segunda propriedade básica \(P(S)=1\) pode ser adaptada.  Observe que dado que o evento \(B\) ocorreu, o natural é passar a considerá-lo como o “novo”{} espaço amostral à luz dessa informação. Assim,
\begin{equation*}
\begin{split}P(B|B)=\frac{P(B)}{P(B)}=1\end{split}
\end{equation*}
de modo que vale a segunda regra básica.

Finalmente, dados \(A_1\) e \(A_2\)  eventos disjuntos,
\begin{equation*}
\begin{split}P(A_1\cup A_2|B)=\frac{P(A_1\cup A_2)\cap B)}{P(B)}=\frac{P((A_1\cap B)\cup(A_2\cap B))}{P(B)}\end{split}
\end{equation*}
Na expressão anterior, observe que foi usada a propriedade distributiva da operação de interseção com a união de dois eventos. Observe também que, no numerador do termo mais à direita da expressão obtida, tem-se a probabilidade da união de dois eventos, a saber, \(A_1\cap B\) e \(A_2\cap B\).

Como os eventos \(A_1\) e \(A_2\) são disjuntos, consequentemente, os eventos \(A_1\cap B\subset A_1\) e \(A_2\cap B\subset A_2\)  também são disjuntos tal que \(P((A_1\cap B)\cup(A_2\cap B))=P(A_1\cap B)+P(A_2\cap B)\) (regra básica 3 da probabilidade).

Logo,
\begin{equation*}
\begin{split}P(A_1\cup A_2|B)=\frac{P((A_1\cup A_2)\cap B)}{P(B)}=\frac{P(A_1\cap B)+P(A_2\cap B)}{P(B)}=P(A_1|B)+P(A_2|B)\end{split}
\end{equation*}
Portanto, as demais propriedades da probabilidade estudadas, também valem para a probabilidade condicional, a saber,
\begin{enumerate}
\item {} 
\(P(\emptyset |B)=0\)

\item {} 
Se \(A_1\subset A_2\),  então \(P(A_1|B)\leq P(A_2|B)\).

\item {} 
\(P(A|B)=1-P(\overline{A}|B)\).

\item {} 
\(P(A_1 \cup A_2|B)=P(A_1|B)+P(A_2|B)-P(A_1\cap A_2|B)\).

\end{enumerate}
\end{example}


\paragraph{Regra da Multiplicação}

A partir da definição de probabilidade condicional é possível obter uma regra para calcular a probabilidade da ocorrência simultânea de dois eventos \(A\)  e \(B\). Como \(\displaystyle{P(A|B)=\frac{P(A\cap B)}{P(B)}}\), segue que
\begin{quote}
\begin{equation*}
\begin{split}P(A\cap B)=P(B)\cdot P(A|B)\end{split}
\end{equation*}\end{quote}

Essa expressão é fundamental para entender resoluções de problemas de cálculo de probabilidades em experimentos sequenciais. Veja o exemplo a seguir.

\begin{example} {Diagrama de árvore}

Em um grupo de 12 pessoas, sabe-se que 8 delas votarão no candidato à prefeito  \(ABC\) e as outras 4 votarão no candidato a prefeito \(XYZ\). Suponha que duas pessoas serão escolhidas sequencialmente, ao acaso e sem reposição desse grupo. Deseja-se calcular a probabilidade de que as duas pessoas sorteadas votarão em candidatos a prefeito distintos.

Primeiro vamos apresentar uma solução analítica desse problema, para em seguida mostrar a solução, muito mais simples, usando diagrama de árvore. A solução analítica será útil para compreender melhor os elementos da árvore e quando usar multiplicação e adição de probabilidades.

Como são apenas dois candidatos, vamos chamar de \(A_1\) o evento “a primeira pessoa sorteada votará em \(ABC\) ”{} e, assim, \(\overline{A_1}\) corresponderá ao evento “a primeira pessoa sorteada votará em \(XYZ\) “. Similarmente, vamos chamar de \(A_2\) o evento “a segunda pessoa sorteada votará em \(ABC\) ”{} e, assim, \(\overline{A_2}\) corresponderá ao evento “a segunda pessoa sorteada votará em \(XYZ\) “. O evento cuja probabilidade queremos calcular é \(E\): “as duas pessoas sorteadas votarão em candidatos distintos.”

Observe que \(E=(A_1\cap \overline{A_2})\cup (\overline{A_1}\cap A_2)\)  e que os dois eventos do lado direito são disjuntos de modo que podemos calcular \(P(E)\) como a soma \(P(A_2\cap \overline{A_1})+P(\overline{A_2}\cap A_1)\). Observe que cada uma dessas duas probabilidades envolve a ocorrência simultânea de dois eventos, de modo que podemos usar a regra da multiplicação:

$$P(A_1\cap \overline{A_2})=P({A_1})\cdot \underbrace{P(\overline{A_2}|{A_1})}_{\mathclap{\substack{\text{prob. da 2a. pessoa não votar em ABC,} \\ \text{dado que a primeira vota em ABC}}}}=\frac{8}{12}\cdot\frac{4}{11}=\frac{8}{33}$$

$$P(\overline{A_1}\cap {A_2})=P(\overline{A_1})\cdot \underbrace{P({A_2}|\overline{A_1})}_{\mathclap{\substack{\text{prob. da 2a. pessoa votar em ABC,} \\ \text{dado que a primeira não vota em ABC}}}}=\frac{4}{12}\cdot\frac{8}{11}=\frac{8}{33}$$

Logo, \(\displaystyle{P(E)=\frac{8}{33}+\frac{8}{33}=\frac{16}{33}\approx 0,485}\)

\textbf{Solução via diagrama de árvore:} Cada ponto de ramificação da árvore desdobra-se nas possibilidades. Observe que nesse exemplo há apenas duas, de modo que do primeiro ponto partem duas possibilidades e a partir de cada possibilidade, partirão mais duas possibilidades, como no esquema a seguir.
\begin{figure}[H]
\centering

\begin{tikzpicture}[scale=.9, every node/.style={scale=.8}]

\draw (0,0) -- (30:3) node [right] {$A_1$} node [above, midway, rotate=30, scale=0.7] {};
\draw (0,0) -- (-30:3) node [right] {$\overline{A_1}$} node [below, midway, rotate=-30, scale=0.7] {};
\draw (3.159807,1.5) -- ++(20:3) node [right] {$A_2$} node [above, midway, rotate=20, scale=0.7] {};
\draw (3.159807,1.5) -- ++(-20:3) node [right] {$\overline{A_2}$} node [below, midway, rotate=-20, scale=0.7] {};
\draw (3.159807,-1.5) -- ++(20:3) node [right] {$A_2$} node [above, midway, rotate=20, scale=0.7] {};
\draw (3.159807,-1.5) -- ++(-20:3) node [right] {$\overline{A_2}$} node [below, midway, rotate=-20, scale=0.7] {};
\end{tikzpicture}
\caption{Ilustração das quatro configurações possíveis via diagrama de árvore}
\end{figure}

Em cada ramificação, assinalamos a respectiva probabilidade. Veja na \hyperref[arvore2]{figura \ref{arvore2}}, o diagrama de árvore com as respectivas probabilidades destacadas.

\begin{figure}[H]
\centering


\begin{tikzpicture}[scale=.9, every node/.style={scale=.8}]

\draw (0,0) -- (30:3) node [right] {$A_1$} node [above, midway, rotate=30] {$8/12$};
\draw (0,0) -- (-30:3) node [right, ] {$\overline{A_1}$} node [below, midway, rotate=-30] {$4/12$};
\draw (3.159807,1.5) -- ++(20:3) node [right] {$A_2$} node [above, midway, rotate=20] {$7/11$};
\draw (3.159807,1.5) -- ++(-20:3) node [right] {$\overline{A_2}$} node [below, midway, rotate=-20] {$4/11$};
\draw (3.159807,-1.5) -- ++(20:3) node [right] {$A_2$} node [above, midway, rotate=20] {$8/11$};
\draw (3.159807,-1.5) -- ++(-20:3) node [right] {$\overline{A_2}$} node [below, midway, rotate=-20] {$3/11$};
\end{tikzpicture}
\caption{Diagrama de árvore do exemplo com as respectivas probabilidades}
\label{arvore2}
\end{figure}

Na \hyperref[arvore3]{figura \ref{arvore3}}, destacam-se as quatro configurações possíveis e respectivas probabilidades.
\begin{figure}[H]
\centering

\begin{tikzpicture}[scale=.9, every node/.style={scale=.8}]

\draw (0,0) -- (30:3) node [right,] {$A_1$} node [above, midway, rotate=30, ] {$8/12$};
\draw (0,0) -- (-30:3) node [right, ] {$\overline{A_1}$} node [below, midway, rotate=-30, ] {$4/12$};
\draw (3.159807,1.5) -- ++(20:3) node [right, ] {$A_2$} node [above, midway, rotate=20, ] {$7/11$};
\draw (3.159807,1.5) -- ++(-20:3) node [right, ] {$\overline{A_2}$} node [below, midway, rotate=-20, ] {$4/11$};
\draw (3.159807,-1.5) -- ++(20:3) node [right, ] {$A_2$} node [above, midway, rotate=20, ] {$8/11$};
\draw (3.159807,-1.5) -- ++(-20:3) node [right, ] {$\overline{A_2}$} node [below, midway, rotate=-20, ] {$3/11$};
\draw [->] (6.57888,2.5260) -- ++(0:0.7) node [right, ] {$\displaystyle{A_1 \cap A_2 \qquad \frac{14}{33}}$};
\draw [->] (6.57888,0.4739) -- ++(0:0.7) node [right, ]   {$\displaystyle{A_1 \cap \overline{A_2} \qquad \frac{8}{33}}$};
\draw [->] (6.57888,-0.4739) -- ++(0:0.7) node [right, ]   {$\displaystyle{\overline{A_1} \cap A_2 \qquad \frac{8}{33}}$};
\draw [->] (6.57888,-2.5260) -- ++(0:0.7) node [right, ]   {$\displaystyle{\overline{A_1} \cap \overline{A_2} \qquad \frac{3}{33}}$};
\end{tikzpicture}
\caption{Diagrama de árvore indicando os quatro casos possíveis após o sorteio}
\label{arvore3}
\end{figure}

Logo, pela árvore, \(\displaystyle P(E)=\frac{8}{33}+\frac{8}{33}=\frac{16}{33}\).
\end{example}


\paragraph{Eventos Independentes}

Na atividade \hyperref[uso-oculos]{uso de óculos e sexo de estudantes}, concluímos que uso de óculos independe de gênero, pois  comparando as probabilidades incondicionais de ser do gênero feminino $(0{,}55)$ e de ser do gênero masculino $(0{,}45)$ com as probabilidades condicionais de ser do gênero feminino $(0{,}545)$ e do gênero masculino $(0{,}455)$ dado que usa óculos, percebe-se que elas são aproximadamente iguais. Na teoria, para que os eventos sejam independentes, essas probabilidades deveriam ser iguais. Na prática esse tipo de análise é usado em Estatística para avaliar a hipótese de independência entre dois eventos.
\begin{observation}{}
Dois eventos \(A\) e \(B\) são ditos independentes se a ocorrência de um deles não muda a incerteza sobre a ocorrência do outro.
\end{observation}

Em símbolos, \(A\) e \(B\)  são ditos eventos independentes se \(P(A|B)=P(A)\), ou equivalentemente, se \(P(B|A)=P(B)\).

Decorre, da definição de probabilidade condicional,
\(P(A|B)=\displaystyle\frac{P(A\cap B)}{P(B)}\), que se \(A\) e \(B\)
são eventos independentes, então \(P(A\cap B)=P(A) \cdot P(B)\).

É preciso ter cuidado ao usar este último resultado. Se não sabemos que os eventos \(A\) e \(B\) são independentes, então devemos usar a regra da multiplicação, a saber, \(P(A\cap B)=P(B)\cdot P(A|B)\) que vale para quaisquer dois eventos.

Observe que há uma simetria no conceito de independência de tal modo que se
\(P(A|B)=P(A)\), então \(P(B|A)=P(B)\).

De fato, se \(P(A|B)=P(A)\), segue que \(P(A\cap B)=P(A) \cdot P(B)\) e, assim,
\begin{equation*}
\begin{split}P(B|A)=\frac{P(A\cap B)}{P(A)}=\frac{P(A)\cdot P(B)}{P(A)}=P(B)\end{split}
\end{equation*}
\begin{example}{Eventos independentes versus eventos disjuntos}

Considere o experimento que consiste em lançar um dado honesto e, em seguida, lançar uma moeda honesta. Defina os eventos \(A:\) “ocorre uma face par”{} e \(B:\) “ocorre uma cara”.
\begin{enumerate}
\item {} 
\(A\) e \(B\) são eventos disjuntos?

\item {} 
\(A\) e \(B\) são eventos independentes?

\end{enumerate}

Observe que estamos lidando com um experimento sequencial de modo que os resultados possíveis envolvem a combinação de dois casos, a saber, número do dado e tipo de face da moeda. Como são seis números possíves para o dado e duas faces possíveis para a moeda, segue que o espaço amostral desse experimento consiste de 12 pares ordenados:

\(S=\{(1,ca),(1,co),(2,ca),(2,co),(3,ca),(3,co),(4,ca),(4,co),(5,ca),(5,co),(6,ca),(6,co)\}\)

Embora seja comum pensar que os eventos \(A\) e \(B\) são disjuntos, depois de descrever os elementos do espaço amostral é fácil perceber que $A$ e  $B$ não são disjuntos (\(A\cap B\neq \emptyset\)). De fato,
\begin{align*}
A&=\{(2,ca),(2,co),(4,ca),(4,co),(6,ca),(6,co)\}\text{, tal que } &P(A)&=\frac{6}{12}=0{,}5\\
B&=\{(1,ca),(2,ca),(3,ca),(4,ca),(5,ca),(6,ca)\} \text{, tal que } &P(B)&=\frac{6}{12}=0{,}5\\
A\cap B&=\{(2,ca),(4,ca),(6,ca)\} \text{, tal que }&P(A\cap B)&=\frac{3}{12}=\frac{1}{4}=0{,}25\\
\end{align*}

É natural pensarmos que os lançamentos do dado e da moeda sejam independentes, pois o resultado de um não interfere no resultado de outro. Podemos confirmar esse raciocínio, observando que \(P(A\cap B)=0{,}25\) é igual ao produto das probabilidades incondicionais de \(A\) e de \(B\)  ambas iguais a $0{,}5$.
\end{example}

Do exemplo anterior vale destacar que, dados dois eventos com probabilidade positiva, não será possível que eles sejam ao mesmo tempo disjuntos e independentes.

\begin{example} {Eventos disjuntos versus eventos independentes}

Sejam \(A\) e \(B\) dois eventos em um espaço amostral \(S\) tais que \(P(A)>0\) e \(P(B)>0\).
\begin{enumerate}
\item {} 
Se \(A\) e \(B\) são eventos disjuntos, então \(A\) e \(B\) não são eventos independentes.

Se \(A\) e \(B\) são disjuntos, então \(P(A\cap B)=0\neq P(A)\cdot P(B)>0\), pois \(P(A)>0\) e \(P(B)>0\). Assim, \(A\) e \(B\) não são independentes.

\item {} 
Se \(A\) e \(B\) são eventos independentes, então $A$ e $B$ não são eventos disjuntos.

Se \(A\) e \(B\) são independentes, então \(P(A\cap B)=P(A)\cdot P(B)>0\)o que implica que \(A\cap B\neq \emptyset\). Logo, \(a\) e \(B\) não são disjuntos.

\end{enumerate}

Observe que a única situação em que se pode ter \(A\)  e \(B\)  simultaneamente disjuntos e independentes ocorre se um dos eventos \(A\) ou \(B\) tiver probabilidade nula.
\end{example}

\begin{observation}{}

Três eventos \(A\), \(B\)  e  \(C\) são independentes se, e somente se, eles são dois a dois independentes e \(P(A\cap B\cap C)=P(A)\cdot P(B)\cdot P(C)\).
\end{observation}

\begin{example} {eventos dois a dois independentes não independentes}

Considere um dado em forma de tetraedro regular cujos números de suas faces são 1, 2, 3 e 4.

\begin{figure}[H]
\centering

\noindent\includegraphics[width=125bp]{{tetraedro_dado}.png}

\caption{Dado em forma de tetraedro com a face 2 oculta}
\end{figure}



Defina os eventos \(A=\{1,2\}\), \(B=\{1,3\}\) e \(C=\{1,4\}\).
\begin{enumerate}
\item {} 
Verifique que \(A\) e \(B\)  são eventos independentes.

\item {} 
Idem para \(A\) e \(C\).

\item {} 
Idem para \(B\) e \(C\).

\item {} 
Calcule \(P(A|B\cup C)\), compare o resultado obtido com \(P(A)\) e diga se \(A\) e \(B\cup C\) são eventos independentes.

\end{enumerate}

É fácil perceber que os eventos \(A\), \(B\) e \(C\) tomados dois a dois são independentes, pois a interseção entre dois dos três é sempre \(\{1\}\) cuja probabilidade é \(1/4\) e como cada um dos eventos \(A\), \(B\) e \(C\) têm probabilidade \(1/2\), segue que

\begin{align*}
P(A\cap B)&=P(A)\cdot P(B)=\frac{1}{2}\cdot\frac{1}{2}=\frac{1}{4}\\
P(A\cap C)&=P(A) \cdot P(C)=\frac{1}{2}\cdot\frac{1}{2}=\frac{1}{4} \\
P(B\cap C)&=P(B)\cdot P(C)=\frac{1}{2}\cdot\frac{1}{2}=\frac{1}{4}\\
\end{align*}
Mas 
\begin{equation*}
\displaystyle P(A|B\cup C)=\frac{P(A\cap (BUC))}{P(B\cup C)}=\frac{P((A\cap B)\cup (A\cap C))}{P(B\cup C)}=\frac{\frac{1}{4}}{\frac{3}{4}}=\frac{1}{3}\neq P(A),
\end{equation*}
e portanto, \(A\) e \(B\cup C\)  não são independentes apesar de \(A\) e \(B\) serem independentes e de \(A\) e \(C\) serem indepedentes.

Portanto, se temos três eventos dois a dois independentes, isso não implicará que os três eventos sejam independentes.
\end{example}

\clearpage
\def\currentcolor{session2}
\begin{objectives}{Desempenho de exames diagnósticos}
{
Reconhecer uma probabilidade condicional.
}{1}{2}
\end{objectives}
\begin{sugestions}{Desempenho de exames diagnósticos}
{
Essa atividade envolve uma adaptação de uma questão do ENEM (2014).
}{1}{2}
\end{sugestions}
\clearmargin
\begin{answer}{Desempenho de exames diagnósticos}
{
Defina 
\begin{itemize}
\item $A$: o evento "a pessoa sorteada tem a doença $X$"{} e 
\item $B$: o evento "o resultado do teste da pessoa sorteada foi positivo".
\end{itemize}

\begin{enumerate}
\item Entre os duzentos indivíduos, $100$ têm a doença $X$, logo $P(A)=\frac{100}{200}=0{,}5$.
\item $P(\overline{A})=1-0,5=0,5$.
\item $P(A|B)=\frac{P(A\cap B)}{P(B)}=\frac{95}{110}\approx0{,}864$.
\item $P(A|\overline{B})=\frac{5}{90}\approx0,056$.
\item Entre as $100$ pessoas doentes, $95$ resultados foram positivos, logo $P(B|A)=0{,}95$.
\item Entre as $100$ pessoas sadias, 85 resultados foram negativos, logo $P(\overline{B}|\overline{A})=0,85$.
\item $0{,}95$ e $0{,}85$, respectivamente, conforme os itens \titem{c)} e \titem{d)}.
\end{enumerate}
}{1}
\end{answer}
\begin{objectives}{Produção de peças}
{
Aplicar a definição de probabilidade condicional para reconhecer eventos independentes e eventos dependentes.
}{1}{1}
\end{objectives}
\begin{answer}{Produção de peças}
{
Defina os eventos $A$: “peça produzida pela máquina $I”$ e $B$: “peça defeituosa”. Observe que os respectivos eventos complementares são $\overline{A}$: “peça produzida pela máquina $II$”, pois são apenas duas máquinas e $\overline{B}$: “peça boa”, pois são apenas duas classificações das peças (boas ou defituosas).

Se os eventos são independentes, deve-se ter $P(B|A)=P(B)$, por exemplo. Tem-se que $P(A)=\frac{200}{200+600}=0{,}25$, $P(B)=\frac{20}{800}=0{,}025$ e $P(B|A)=P(A\cap B)P(A)=\frac{5/800}{200/800}=\frac{1}{40}=0{,}025$. Logo, conlui-se que a produção de defeituosas é independente do tipo de máquina, pois $P(B|A)=P(B)$.

Observe que tanto faz a escolha dos eventos máquina $I$ ou máquina $II$ e peça defeituosa ou peça boa. Por exemplo,

$P(\overline{A})=\frac{3}{4}=0{,}75, P(B)=\frac{20}{800}=0{,}025$ e $P(\overline{B})=0{,}975$. Além disso, $P(\overline|B)=0{,}75=P(\overline{A}), P(A|\overline{B})=0{,}25=P(A), P(\overline{A}|\overline{B})=0{,}75=P(\overline{A})$.

De fato, essa propriedade não é um caso específico desse problema e ela é explorada na próxima atividade.
}{1}
\end{answer}
\begin{objectives}{Independência e complementariedade}
{
Aplicar o conceito de independência entre dois eventos para entender que o respectivos eventos complementares herdam essa a condição de eventos independentes.
}{1}{2}
\end{objectives}
\begin{sugestions}{Independência e complementariedade}
{
Essa atividade é um exercício teórico de dedução muito simples que revela uma propriedade importante entre eventos independentes: dada uma coleção de eventos independentes, se para alguns eventos (ou todos) considerarmos os seus complementares em vez do próprio, a coleção continua independente. Com fins de simplificação e não tornar o processo complicado, consideraremos nesta atividade apenas o caso para dois eventos independentes. Mas, de fato, o resultado vale para quaisquer coleções de eventos independentes.
}{1}{2}
\end{sugestions}
\begin{answer}{Independência e complementariedade}
{
\begin{enumerate}
\item Por hipótese temos que $P(A\cap B)=P(A)\cdot P(B)$, pois $A$ e $B$ são independentes. Queremos provar que $\overline{A}$ e $\overline{B}$ também são independentes, ou equivalentemente, que $P(\overline{A}\cap \overline{B})=P(\overline{A})\cdot P(\overline{B})$.
Pelas Leis de De Morgan trabalhadas em atividade anterior, sabemos que $\overline{A}\cap \overline{B}=(\overline{A\cup B})$. Portanto, usando a propriedade do evento complementar, podemos escrever $P(\overline{A}\cap \overline{B})=1−P(A\cup B)$.

Mas, 
\begin{equation*}
P(A\cup B)=P(A)+P(B)-P(A\cap B)=P(A)+P(B)-P(A)\cdot P(B)
\end{equation*}
ela independência entre $A$ e $B$. Assim,
\begin{align*}
P(\overline{A}\cap \overline{B})&=1-P(A)-P(B)+P(A)\cdot P(B)\\ 
&=1-P(A)-P(B)\cdot(1−P(A))\\
&=(1−P(A))\cdot(1-P(B))\\
&=P(\overline{A})\cdot P(\overline{B})
\end{align*}.
Portanto, se $A$ e $B$ são independentes, então $\overline{A}$ e $\overline{B}$ também são independentes.

\item Por hipótese temos que $P(A\cap B)=P(A)\cdot P(B)$, pois $A$ e $B$ são independentes. Queremos provar que $A$ e $\overline{B}$ também são independentes, ou equivalentemente, que $P(A\cap \overline{B})=P(A)\cdot P(\overline{B})$. Observe que podemos escrever $A=(A\cap B)\cup (A\cap\overline{B})$ com os dois eventos do lado direito sendo disjuntos. Assim, $P(A)=P(A\cap B)+P(A\cap \overline{B})$ implicando que $P(A\cap \overline{B})=P(A)-P(A\cap B)=P(A)-P(A)\cdot P(B)$ pela independedência de $A$ e $B$. Logo, $P(A\cap \overline{B})=P(A)\cdot(1−P(B))=P(A)\cdot P(\overline{B})$ e, portanto, se $A$ e $B$ são independentes, então $A$ e $\overline{B}$ também são.
\item Idem ao item anterior, bastando trocar de posição as letras $A$ e $B$.
\end{enumerate}
}{0}
\end{answer}
\begin{objectives}{Escolha da chave correta}
{
Estender a regra da multiplicação para o caso de mais de dois eventos.
}{1}{2}
\end{objectives}
\begin{answer}{Escolha da chave correta}
{
\begin{enumerate}
\item Sejam os eventos $A_i$: “o cadeado é aberto na $i$-ésima tentativa”, $i=1,2,3$. Queremos calcular $P(\overline{A_1}\cap \overline{A_2}\cap A_3)$ , pois por hipótese o cadeado não foi aberto nem na primeira nem na segunda tentetiva.

$P(\overline{A_1}∩\overline{A_2}\cap A_3)=P(\overline{A_1})\cdot P(\overline{A_2}|\overline{A_1})\cdot P(A_3|\overline{A_1}\cap \overline{A_2})=\frac{4}{5}\cdot\frac{4}{5}\cdot\frac{1}{5}=\frac{16}{125}=0{,}128$. Essa solução pode ser visualizada por meio do diagrama de árvore, ilustrado na figura a seguir.
\begin{figure}[H]
\centering

\resizebox{.95\linewidth}{!}
{
\begin{tikzpicture}[every node/.style={scale=3}]
\draw (0,0) -- (30:3) node [right, scale=0.28] {$A_1$} node [above, midway, rotate=30, scale=0.28] {1/5};
\draw (0,0) -- (-30:3) node [right, scale=0.28] {$\overline{A_1}$} node [below, midway, rotate=-30, scale=0.28] {4/5};

\draw [->] (3.159807,1.5) -- ++(0:0.7) node [right, scale=0.28] {Abre na Primeira $\quad P(A_1) = \dfrac{1}{5}$} ;

\draw (3.159807,-1.5) -- ++(30:3) node [right, scale=0.28] {$A_2$} node [above, midway, rotate=20, scale=0.28] {1/5};
\draw (3.159807,-1.5) -- ++(-30:3) node [right, scale=0.28] {$\overline{A_2}$} node [below, midway, rotate=-20, scale=0.28] {1/5};

\draw [->] (6.319614,0) -- ++(0:0.7) node [right, scale=0.28] {Abre na Segunda $\quad P(A_1 \cap A_2) = \dfrac{4}{5} \cdot \dfrac{1}{5}$} ;


\draw (6.319614,-3) -- ++(30:3) node [right, scale=0.28] {$A_3$} node [above, midway, rotate=20, scale=0.28] {1/5};
\draw (6.319614,-3) -- ++(-30:3) node [right, scale=0.28] {$\overline{A_3}$} node [below, midway, rotate=-20, scale=0.28] {1/5};

\draw [->, align=left] (9.479421,-1.5) -- ++(0:0.28) node [ right, scale=0.28] {Abre na Segunda \\ $P(A_1 \cap A_2 \cap A_3) = \dfrac{4}{5} \cdot \dfrac{4}{5} \cdot \dfrac{1}{5}$} ;

\end{tikzpicture}
}
\item 
\end{figure}
\item Considerando agora o caso sem reposição tem-se
$P(\overline{A_1}\cap \overline{A_2}\cap A_3)=P(\overline{A_1})\cdot P(\overline{A_2}|\overline{A_1})\cdot P(A_3|\overline{A_1}\cap\overline{A_2})=\frac{4}{5}\cdot\frac{3}{4}\cdot\frac{1}{3}=\frac{1}{5}=0{,}2$. Essa solução pode ser visualizada por meio do diagrama de árvore, ilustrado na figura a seguir.

\begin{figure}[H]
\centering

\begin{tikzpicture}[every node/.style={scale=3}]

\draw (0,0) -- (30:3) node [right, scale=0.28] {$A_1$} node [above, midway, rotate=30, scale=0.28] {1/5};
\draw (0,0) -- (-30:3) node [right, scale=0.28] {$\overline{A_1}$} node [below, midway, rotate=-30, scale=0.28] {4/5};

\draw [->] (3.159807,1.5) -- ++(0:0.7) node [right,  scale=0.28] {Abre na Primeira $\quad P(A_1) = \dfrac{1}{5}$} ;

\draw (3.159807,-1.5) -- ++(30:3) node [right, scale=0.28] {$A_2$} node [above, midway, rotate=20, scale=0.28] {1/4};
\draw (3.159807,-1.5) -- ++(-30:3) node [right, scale=0.28] {$\overline{A_2}$} node [below, midway, rotate=-20, scale=0.28] {3/4};

\draw [->] (6.319614,0) -- ++(0:0.7) node [right, scale=0.28] {Abre na Segunda $\quad P(A_1 \cap A_2) = \dfrac{4}{5} \cdot \dfrac{3}{4}$} ;


\draw (6.319614,-3) -- ++(30:3) node [right, scale=0.28] {$A_3$} node [above, midway, rotate=20, scale=0.28] {1/3};
\draw (6.319614,-3) -- ++(-30:3) node [right, scale=0.28] {$\overline{A_3}$} node [below, midway, rotate=-20, scale=0.28] {2/3};

\draw [->, align=left] (9.479421,-1.5) -- ++(0:0.7) node [ right, scale=0.28] {Abre na Segunda \\ $P(A_1 \cap A_2 \cap A_3) = \dfrac{4}{5} \cdot \dfrac{3}{4} \cdot \dfrac{1}{3}$} ;
\end{tikzpicture}
\end{figure}
\end{enumerate}
}{0}
\end{answer}
\clearmargin
\begin{objectives}{Probabilidade total}
{
Aplicar a definição de probabilidade condicional e propriedades da probabilidade para resolver um problema cuja solução faz uso do teorema de Bayes.

}{2}{2}
\end{objectives}
\begin{sugestions}{Probabilidade total}
{
Com o material apresentado até aqui é possível resolver problemas cuja solução faz uso do teorema de Bayes. No entanto, optou-se nesse livro, a não apresentar a fórmula do teorema de Bayes, que só traz dificuldade para o aluno e, no entanto, é muito simples de ser deduzida na prática.

Teorema de Bayes: Sejam $A_1,A_2,\cdot,A_n$ eventos tais que $A_1\cup A_2\cup\cdots\cup A_n=S$ que são dois a dois disjuntos, ou seja, $A_i\cap A_j=\emptyset$ sempre que $i,j\in\{1,2,3,\cdots,n\}$ com $i\neq j$. Seja $B\subset S$. Então,
\begin{equation*}
P(A_k|B)=\dfrac{P(A_k)\cdot P(B|A_k)}{\displaystyle\sum^n_{i=1}P(A_i)\cdot P(B|A_i)}, k=1,2,...,n.
\end{equation*}
}{2}{2}
\end{sugestions}
\begin{answer}{Probabilidade total}
{
\begin{enumerate}
\item Defina os seguintes eventos:
\begin{itemize}
\item $A_1$: "bloco I é sorteaco",
\item $A_2$: "bloco II é sorteado" e
\item $A_3$: "bloco III é sorteado".
\end{itemize}
Defina també o evento 
\begin{itemize}
\item B: "um menor de até 12 anos é sorteado". Observe que os eventos $A_1,A_2$ e $A_3$ são disjuntos e que um menor sorteado pode ser de um, e somente um, dos três blocos. Logo, podemos escrever
\begin{equation*}
B=(A_1\cap B)\cup(A_2\cap B)\cup(A_3\cap B).
\end{equation*}
Como os eventos do lado direito da equação acima são disjuntos, tem-se\begin{equation*}
P(B)=(A_1\cap B)+P(A_2\cap B)+P(A_3\cap B)
\end{equation*}
Usando-se a regra da multiplicação, tem-se
\begin{align*}
P(B)&=P(A_1)\cdot P(B|A_1)+P(A_2)\cdot P(B|A_2)+P(A_3)\cdot P(B|A_3)\\
&=\frac{1}{3}+\frac{27}{180}+\frac{1}{3}\cdot\frac{36}{200}+\frac{1}{3}\cdot\frac{24}{100}\\
&=\frac{1}{20}+\frac{3}{50}+\frac{1}{15}\\
&=\frac{15+18+20}{300}\\
&=\frac{53}{300}\approx0{,}177.
\end{align*}

Nesse item, deseja-se calcular a probabilidade condicional de ter ocorrido o evento $A_2$, dado que ocorreu o evento $B$, ou seja, $P(A_2|B)$. Usando a definição de probabilidade condicional tem-se
\begin{align*}
P(A_2|B)&=\frac{P(A_2\cap B)}{P(B)}\\
&=\frac{P(A_2)\cdot P(B|A_2)}{53/300}\\
&=\frac{3}{50}\cdot\frac{300}{53}\\
&=\frac{18}{53}\approx0{,}34.
\end{align*}
\end{itemize}
\end{enumerate}
}{0}
\end{answer}

\practice{Probabilidade Condicional}
\begin{task}{desempenho de exames diagnósticos}


Testes diagnósticos para detectar uma doença não são infalíveis. Para analisar o desempenho de um desses testes, realizam-se estudos em populações, contendo pessoas sãs e portadoras da doença.

\begin{figure}[H]
\centering

\noindent\includegraphics[width=200bp]{{amostras_sangue}.png}


\caption{Amostras de sangue para realização de exame}
\end{figure}


Quatro situações distintas podem ocorrer
\begin{enumerate}
\item {} 
a pessoa TEM a doença e o resultado do teste é POSITIVO.

\item {} 
a pessoa TEM a doença e o resultado do teste é NEGATIVO.

\item {} 
a pessoa NÃO TEM a doença e o resultado do teste é POSITIVO.

\item {} 
a pessoa NÃO TEM a doença e o resultado do teste é NEGATIVO.

\end{enumerate}

Observe que nas situações \titem{b)} e \titem{c)}, o teste falha, pois deveria ser positivo quando a pessoa tem a doença e, negativo, quando a pessoa não tem a doença. Já nas situações \titem{a)} e \titem{d)} o teste acerta o diagnóstico.

Dois índices de desempenho para avaliação de um teste diagnóstico costumam ser usados: a sensibilidade e especificidade.

A \textbf{sensibilidade} é definida como a probabilidade de o resultado do teste ser POSITIVO, dado que a pessoa examinada tem a doença. Já a \textbf{especificidade} é a probabilidade do teste ser NEGATIVO, dado que a pessoa examinada não tem a doença.


O quadro a seguir refere-se a um teste diagnóstico para a doença \(X\), aplicado em uma amostra composta por duzentas pessoas, sendo 100 sadias e 100 portadoras da doença \(X\).

\begin{table}[H]
\centering
\begin{tabu} to \textwidth{|l|l|l|}
\hline
\thead
Resultado do teste & doente & sadia \\
\hline
Positivo & 95 & 15 \\
\hline
Negativo & 5 & 85 \\
\hline
\end{tabu}
\end{table}


Uma pessoa entre as duzentas dessa amostra será sorteada.
\begin{enumerate}
\item {} 
Qual a probabilidade de ela tenha a doença \(X\)?

\item {} 
Qual a probabilidade de que ela NÃO tenha a doença \(X\)?

\item {} 
Se o resultado do teste da pessoa sorteada foi positivo, calcule a probabilidade de que ela tenha a doença.

\item {} 
Se o resultado do teste da pessoa sorteada foi negativo, calcule a probabilidade de que ela tenha a doença.

\item {} 
Sabendo que a pessoa sorteada tem a doença, qual a probabilidade de seu teste ter resultado positivo?

\item {} 
Sabendo que a pessoa sorteada  NÃO tem a doença, qual a probabilidade de seu teste ter resultado negativo?

\item {} 
Determine uma estimativa da sensibilidade e da especificidade desse teste, usando a informação do quadro acima.

\end{enumerate}

\end{task}

\begin{task}{produção de peças}


\begin{figure}[H]
\centering

\noindent\includegraphics[width=375bp]{maquinas}

\caption{Produção de peças}
\end{figure}



Para fins de controle de qualidade das peças produzidas em uma fábrica, foram analisadas amostras de peças produzidas por máquinas diferentes. Os resultados obtidos estão resumidos no quadro a seguir.

\begin{table}[H]
\centering
\begin{tabu} to \textwidth{|c|c|c|}
\hline
\thead
Máquina
&
Defeituosas
&
Boas
\\
\hline
$I$
&
5
&
195
\\
\hline
$II$
&
15
&
585
\\
\hline
\end{tabu}
\end{table}


Pela informação das amostras coletadas, há razão para se acreditar que a produção de defeituosas ou boas é independente do tipo de máquina utilizada na produção?
\end{task}

\begin{task}{independência e complementariedade}


Sejam \(A\) e \(B\) dois eventos independentes.  Mostre que os pares de eventos a seguir também são independentes:
\begin{enumerate}
\item {} 
\(\overline{A}\) e \(\overline{B}\);

\item {} 
\(A\) e \(\overline{B}\); e

\item {} 
\(\overline{A}\) e \(B\).

\end{enumerate}
\end{task}

\begin{task}{escolha da chave correta}
Numa caixa há 5 chaves das quais apenas uma abre um cadeado.

\begin{figure}[H]
\centering
\includegraphics[width=175bp]{cadeado}

\caption{Chaves e cadeado}
\label{}
\end{figure}
Retirando-se chaves sequencialmente da caixa, qual a probabilidade de se abrir o cadeado apenas na terceira tentativa se:
\begin{enumerate}
\item a cada tentativa a chave extraída é recolocada na caixa?
\item a cada tentativa a chave extraída não é recolocada na caixa?
\item Em qual dos contextos (a) e (b) a probabilidade de se abrir a caixa é maior? Por quê?
\end{enumerate}

\end{task}

\begin{task}{probabilidade total}


Em um condomínio há três prédios: bloco I, bloco II e bloco III. No bloco I há 180 moradores, no bloco II há 200 moradores e no bloco III há 120 moradores. Entre os moradores do bloco I, há 27 menores de até 12 anos. Entre os moradores do bloco II há 36 menores de até 12 anos. Entre os moradores do bloco III há 24 menores de até 12 anos. Um residente do condomínio será sorteado da seguinte forma: primeiro será sorteado um bloco, tendo os blocos probabilidades iguais.  Depois, um residente do bloco sorteado será escolhido ao acaso, usando-se um cadastro dos moradores do bloco. Pede-se calcular a probabilidade
\begin{enumerate}
\item {} 
a probabilidade de que a pessoa sorteada seja um menor de até 12 anos;

\item {} 
a probabilidade de ter sido sorteado o bloco II, sabendo que a pessoa sorteada é um menor de até 12 anos.
  
\end{enumerate}

\begin{figure}[H]
\centering

\noindent\includegraphics[width=\linewidth]{5.jpg}

\caption{Condomínio com três blocos}
\end{figure}

\end{task}
\know{ }

Em atividades anteriores foi sugerido lançar uma moeda repetidas vezes e registrar o resultado observado.  Veremos nesta seção como simular o lançamento de uma moeda um grande número de vezes com o auxílio da tecnologia.

Todos as linguagens de programação de computadores, planilhas e aplicativos costumam ter uma função interna usada para a geração de números aleatórios, na verdade, números pseudo-aleatórios, pois existe um mecanismo determinístico por trás da geração desses números.  Para mais detalhes sobre números (pseudo) aleatórios sugerimos assistir ao vídeo nesse  \href{https://www.youtube.com/watch?v=f4sE1r3UL4E}{link}.

Nessa seção serão apresentadas algumas ferramentas úteis para fazer simulações de experimentos simples como o lançamento de uma ou mais moedas e  o lançamento de um ou mais dados.

Uma simulação de um experimento é um processo que tem o mesmo comportamento do experimento, de tal modo que resultados similares à realidade sejam gerados pelo processo.

As atividades a seguir consistirão em simulações de fenômenos aleatórios simples. Para sua realização será necessário o uso de tecnologia. A seguir, duas possibilidades serão detalhadas: as planilhas do  \href{https://pt-br.libreoffice.org/}{LibreOffice} e do \href{https://wiki.geogebra.org/en/Reference:GeoGebra\_Installation}{GeoGebra}.

A função \textit{=ALEATÓRIOENTRE(1;N)} do LibreOffice, com \(N\) um número natural, produz um número do conjunto \(\{1,2,3,,\cdots, N\}\) de tal modo que a probabilidade de cada um dos números é dada por \(\frac{1}{N}\). Assim, ao usarmos \textit{=ALEATÓRIOENTRE(1;2)}, o resultado será equivalente a sortear um número do conjunto \(\{1,2\}\) cada um com probabilidade igual a \(\frac{1}{2}\). Veja na \hyperref[simulacao1]{figura \ref{simulacao1}} uma simulação de 20 números com essa propriedade, usando o LibreOffice.

\begin{figure}[H]
\centering

\noindent\includegraphics[width=370bp]{{libreoffice1}.png}

\caption{Vinte simulações do sorteio dos números 1 e 2 com probabilidades iguais, usando o LibreOffice}
\label{simulacao1}
\end{figure}
 

Para simular os 20 números, basta digitar na célula A1 \textit{=aleatórioentre(1;2)} e apertar a tecla \textit{Enter}. Depois, posicione o cursor no canto esquerdo inferior da célula e arraste para baixo até a linha 20. Os 20 números gerados neste exemplo estão na coluna A da planilha. É possível gerar rapidamente, muito mais do que apenas 20 números.

Essa função, \textit{aleatórioentre(1;2)} pode ser usada para simular o lançamento de uma moeda honesta, associando, por exemplo, o número 1 para a ocorrência de “cara”{} e, o número 2 para a ocorrência de  “coroa”, pois o modelo probabilístico da função \textit{=aleatórioentre(1;2)} é tal que os dois números são gerados com probabilidades iguais. Assim, espera-se, em média, que as quantidades de números 1 (1’s) e de números 2 (2’s) sejam aproximadamente iguais.

Como fazer para obter de forma rápida essas quantidades, sem ter que contar um a um?

No LibreOffice a função \textit{=cont.se(lista de células; número pesquisado)} faz essa contagem.

No exemplo anterior, a função \textit{=cont.se(A1:A20;1)} irá retornar o número de 1’s encontrados nas células da coluna A das linhas 1 a 20. Já a  a função \textit{=cont.se(A1:A20;2)} irá retornar o número de 2’s encontrados nas células da coluna A das linhas 1 a 20. Veja uma ilustração do uso da função na \hyperref[simulacao2]{figura \ref{simulacao2}}. Nessa figura, vemos que na simulação foram gerados doze (12) números 1 (caras) e oito (8) números 2 (coroas).

\begin{figure}[H]
\centering

\noindent\includegraphics[width=400bp]{{Libreoffice2}.png}

\caption{Exemplo do uso da função cont.se do LibreOffice}
\label{simulacao2}
\end{figure}

 

No Geogebra as funções para realizar as mesmas tarefas são similares às utilizadas no LibreOffice. Veja na \hyperref[simulacao3]{figura \ref{simulacao3}} a simulação do lançamento de uma moeda honesta 20 vezes, usando a função \textit{=NúmeroAleatório(1,2)} do GeoGebra.

\begin{figure}[H]
\centering

\noindent\includegraphics[width=400bp]{{GeoGebra1_1}.png}
\caption{GeoGebra: simulação de 20 lançamentos de uma moeda honesto, associando 1 para cara e 2 para coroa}
\label{simulacao3}
\end{figure}


Para a realização da contagem de caras e coroas, a função correspondente no GeoGebra é a função \textit{ContarSe(números, células)}.

A quantidade de 1’s na coluna A das linhas 1 a 20 é dada por \textit{=ContarSe(0\textless{}x\textless{}2,A1:A20)} que produzirá a quantidade de números inteiros no intervalo aberto {]}0,2{[} que ocorrem nas celas da coluna A das linhas 1 a 20 (A1:A20). Observe que o único número inteiro no intervalo aberto {]}0,2{[} é o número 1. Para contar a quantidade de 2’s na coluna A, devemos digitar \textit{=ContarSe(1\textless{}x\textless{}3,A1:A20)}, lembrando que o único número inteiro no intervalo aberto {]}1,3{[} é o número 2.

\begin{figure}[H]
\centering

\noindent\includegraphics[width=400bp]{{GeoGebra2_1}.png}

\caption{Geogebra: exemplo de uso da função \textit{ContarSe}}
\end{figure}


Observe que na simulação realizada com o GeoGebra também foram obtidos doze números 1 (caras) e oito (8) números 2 (coroas). De fato, trata-se de mera coincidência, pois espera-se que, em média sejam produzidos a cada simulação, 10 caras.
\clearmargin
\begin{objectives}{Simulação do lançamento de uma moeda honesta}
{
Aplicar o modelo probabilístico equiprovável e usar tecnologia para realizar simulações de um fenômeno aleatório.
}{1}{2}
\end{objectives}
\begin{sugestions}{Simulação do lançamento de uma moeda honesta}
{
Essa atividade demanda o uso de tenologia. Sugere-se realizá-la em laboratório de informática. Sugere-se também que os alunos trabalhem em pequenos grupos de dois ou três alunos para cada computador disponível. As atividades poderão ser adaptadas de acordo com o conhecimento prévio dos estudantes. Nesssa atividade poderá ser usado o exemplo dado anteriormente, atribuindo um dos dois números para “cara”{} e, o outro, para “coroa”.
}{1}{2}
\end{sugestions}
\begin{answer}{Simulação do lançamento de uma moeda honesta}
{
\begin{enumerate}
\item Você pode realizar a simulação usando a planilha do Geogebra e a função \textit{=NúmeroAleatório(1,2)} que irá gerar com probabilidades iguais ou o número $1$ ou o número $2$. Escolha um dos números para representar a ocorrência de cara e, o outro, para a ocorrência de coroa. Depois, arraste, copiando esta função para mais $19$ células, obtendo as $20$ simulações. Veja exemplo no início dessa seção. No GeoGebra ocorreram doze $1$’s (caras) e oito $2$’s (coroas) tal que a frequência relativa de caras observadas nessas 20 simulações foi $\frac{12}{20}=0{,}6$. É claro que as respostas irão variar, dependendo da simulação. Mas, espera-se que o número de $1$’s (caras) obtidos oscile em torno de $10$, pois a função produz os dois números com probabilidades iguais e geramos $20$ números.

\item Idem ao item anterior, só que agora os números de células a serem considerados na planilha são $50$, $100$, $250$ e $1000$, respectivamente.
\end{enumerate}
}{1}
\end{answer}
\begin{answer}{Simulação do lançamento de uma moeda honesta}
{
\begin{enumerate}\setcounter{enumi}{2}
\item No preenchimento da tabela você deverá perceber que a medida que o número de simulações é maior, a frequência relativa de caras se aproxima da probabilidade teórica de obter uma cara $(0{,}5)$. Se de fato o gerador de números aleatórios do programa que você está usando é bom, esse é o resultado esperado. Por exemplo, em uma simulação com o Geogebra foram observadas as seguintes frequências relativas de caras conforme o número de lançamentos:
\begin{table}[H]
\centering

\begin{tabular}{|f|f|}
\hline
$\tcolor{Número de Observações}$ & $\tcolor{Frequência relativa de 6}$ \\
\hline
20 & 0{,}45 \\
\hline
50 & 0{,}58 \\
\hline
100 & 0{,}51 \\
\hline
250 & 0{,}52 \\
\hline
1000 & 0{,}49 \\
\hline
\end{tabular}
\end{table}
\end{enumerate}
}{1}
\end{answer}
\begin{objectives}{Simulação do lançamento de um dado honesto}
{
Aplicar o modelo probabilístico equiprovável e usar tecnologia para realizar simulações de um fenômeno aleatório.
}{1}{1}
\end{objectives}
\begin{sugestions}{Simulação do lançamento de um dado honesto}
{
Essa atividade demanda o uso de tenologia. Sugere-se realizá-la em laboratório de informática. Sugere-se também que os alunos trabalhem em pequenos grupos de dois ou três alunos para cada computador disponível. As atividades poderão ser adaptadas de acordo com o conhecimento prévio dos estudantes. Nesssa atividade poderá ser usado exemplo dado anteriormente, gerando-se um número aleatório entre $1$ e $6$ e, depois, contando as quantidades obtidas de cada face.
}{1}{1}
\end{sugestions}
\begin{answer}{Simulação de um dado honesto}
{
\begin{enumerate}
\item Você pode realizar a simulação usando a planilha do GeoGebra e a função \textit{=NúmeroAleatório(1,6)} na célula A1. Essa função retornará, com probabilidades iguais, um entre os números 1,2,3,4,5 e 6. Depois, arraste, copiando esta função para mais 29 células A2 até A30, obtendo as 30 simulações. Veja exemplo no início dessa seção. Depois use a função \textit{=ContarSe(5<x<7,A1:A30)}. É claro que as respostas irão variar, dependendo da simulação. Mas, espera-se que o número de faces “6” obtidas oscile em torno de 5, pois a função gera o número $6$ com probabilidade $\frac{1}{6}$ e assim, em média espera-se obter $30\cdot16=5$ dígitos $6$.

\item Idem ao item anterior, só que agora o número de células a ser considerado na planilha será $60$, $120$, $300$ e $1500$.

\item No preenchimento da tabela você deverá perceber que a medida que o número de simulações é maior, a frequência relativa de faces 6 se aproxima da probabilidade teórica de obter uma face 6 ($\approx0{,}167$). Se de fato o gerador de números aleatórios do programa que você está usando é bom, esse é o resultado esperado. Por exemplo, em uma simulação com o GeoGebra foram observadas as seguintes frequências relativas de faces 6 conforme o número de lançamentos:

\begin{table}[H]
\centering

\begin{tabular}{|f|f|}
\hline
$\tcolor{Número de Observações}$ & $\tcolor{Frequência relativa de 6's}$ \\
\hline
30 & 0{,}17 \\
\hline
60 & 0{,}18 \\
\hline
120 & 0{,}18 \\
\hline
300 & 0{,}16 \\
\hline
1500 & 0{,}17 \\
\hline
\end{tabular}
\end{table}
\end{enumerate}
}{0}
\end{answer}
\clearmargin
\begin{objectives}{Simulação do lançamento de um dado desequilibrado}
{
Aplicar modelo probabilístico não equiprovável e usar tecnologia para realizar simulações de um fenômeno aleatório.
}{1}{2}
\end{objectives}

\begin{sugestions}{Simulação do lançamento de um dado desequilibrado}
{
Nessa atividade será proposta a simulação de um dado desequilibrado de tal modo que as faces não ocorrem com probabilidades iguais. Inicialmente será necessária uma discussão sobre com os alunos desequilibrados. Em particular, nessa atividade será suposto que a probabilidade da face é proporcional ao número da face. Assim, a primeira pergunta envolverá obter as probabilidades de cada face. Fazendo $P({i})=k\cdot i$ em que $i=1,2,3,4,5,6$ e $k$ é a constante de proporcionalidade, lembre com os alunos a segunda regra básica de que $P(S)=1$ e que $\{1\}\cup\{2\}\cup\cdots\cup\{6\}=S$. Além disso, como os eventos elementares são disjuntos segue que $1=P(\{1\})+P(\{2\})+\cdots+P(\{6\})=k+2k+3k+4k+5k+6k=21k$ tal que $k=121$. Depois dessa dedução será necessária uma discussão sobre como usar o gerador de números aleatórios para simular os resultados desse dado, para finalmente realizar a simulação de um número aleatório entre $1$ e $21$ e atribuido o $1$ à face $1$, depois o $2$ e o $3$ à face $2$, o $4$, o $5$ e o $6$ à face $3$, o $7$, o $8$, o $9$ e o $10$ à face $4$, o $11$, o $12$, o $13$, o $14$ e o $15$ à face $5$ e, finalmente, o $16$, o $17$, o $18$, o $19$, o $20$ e o $21$ à face $6$. Leve o aluno a observar que desse modo estaremos respeitando as probabilidades desiguais, a saber, $1/21$ para a face $1$, $2/21$ para a face $2,..., 6/21$ para a face $6$.
}{0}{2}
\end{sugestions}
\begin{answer}{Simulação do lançamento de um dado desequilibrado}
{
\begin{enumerate}
\item Você pode realizar a simulação usando a planilha do Geogebra e a função \textit{=NúmeroAleatório(1,21)} na célula A1. Essa função retornará, com probabilidades iguais, um entre os números 1,2,3,4,5, …,21. Depois, arraste, copiando esta função para mais 41 células A2 até A42, obtendo as 42 simulações. Observe que nesse caso, você deverá usar apenas um número para atribuir à ocorrência da face 1, dois números para a face 2, e assim por diante, com seis números para a face 6. Se consideraros os seis últimos, a saber, $21$, $20$, $19$, $18$, $17$ e $16$, use a função \textit{=ContarSe(15<x<22,A1:A42)}, para obter a quantidade de faces 6 obtidas nos $42$ lançamentos. É claro que as respostas irão variar de um para outro. Mas, espera-se que o número de faces $“6”$ obtidas oscile em torno de $12$, pois a probabilidade de gerar uma face $“6”$ com esse procedimento é $\frac{6}{21}\approx0{,}286$ e estamos simulando $42$ lançamentos.

\item Idem ao item anterior, só que agora o número de células a ser considerado na planilha será $210$, $630$, $840$ e $160$.

\item No preenchimento da tabela você deverá perceber que a medida que o número de simulações é maior, a frequência relativa de faces $6$ se aproxima da probabilidade teórica de obter uma face $6$ ($\approx0,286$). Se de fato o gerador de números aleatórios do programa que você está usando é bom, esse é o resultado esperado. Por exemplo, em uma simulação com o GeoGebra foram obserdas as seguintes frequências relativas de faces 6 conforme o número de lançamentos:
\begin{table}[H]
\centering

\begin{tabular}{|f|f|}
\hline
$\tcolor{Número de Observações}$ & $\tcolor{Frequência relativa de 6}$ \\
\hline
42 & 0{,}29 \\
\hline
84 & 0{,}31 \\
\hline 
210 & 0{,}30 \\
\hline
630 & 0{,}28 \\
\hline
840 & 0{,}29 \\
\hline
1680 & 0{,}29 \\
\hline
\end{tabular}
\end{table}
\end{enumerate}
}{0}
\end{answer}
\begin{objectives}{Simulação do lançamento de um dado diferente}
{
Aplicar modelo probabilístico e usar tecnologia para realizar simulações de um fenômeno aleatório.
}{1}{2}
\end{objectives}
\begin{sugestions}{Simulação do lançamento de um dado diferente}
{
Nessa atividade será proposta a simulação de um dado diferente. Apesar das faces ocorrerem com probabilidades iguais, esse dados tem registrado em suas faces o número 1 em uma delas, o núermo 2 em duas delas e o número 3 em três delas. Discuta com seus alunos como adaptar a função de geração de números aleatórios para simular resultados de lançamentos desse dado. É fácil perceber, da atividade anterior que a probabilidade de 3 é 3/6, de 2 é 2/6 e, de 1, 1/6
}{1}{2}
\end{sugestions}
\begin{answer}{Simulação do lançamento de um dado diferente}
{
\begin{enumerate}
\item A probabilidade de obter face 1 é dada por 16, de obter face 2 é dada por 26 e, de obter face 3 é dada por 36, pois o dado é equilibrado, mas há uma face 1, duas faces 2 e três faces 3.
Você pode realizar a simulação usando a planilha do Geogebra e a função =NúmeroAleatório(1,6) na célula A1. Essa função retornará, com probabilidades iguais, um entre os números 1,2,3,4,5,6. Depois, arraste, copiando esta função para mais 29 células A2 até A30, obtendo as 30 simulações. Observe que nesse caso, você deverá atribuir apenas um número para a ocorrência da face 1, dois números para a face 2 e três números para a face 3. Se considerarmos os números 2 e 3 para a ocorrência de face 2, use a função =ContarSe(1<x<4,A1:A30). É claro que as respostas irão variar, dependendo da simulação. Mas, espera-se que o número de faces “2” obtidas oscile em torno de 10, pois a probabilidade de obter uma face “2” é 2/6 e estamos simulando 30 lançamentos.

\item Idem ao item anterior, só que agora o número de células a ser considerado na planilha será 300, 600, 900 e 1800.

\item No preenchimento da tabela você deverá perceber que a medida que o número de simulações é maior, a frequência relativa de faces 2 se aproxima da probabilidade teórica de obter uma face 2 (≈0,333). Se de fato o gerador de números aleatórios do programa que você está usando é bom, esse é o resultado esperado. Por exemplo, em uma simulação com o GeoGebra foram observadas as seguintes frequências relativas de faces 2 conforme o número de lançamentos
\end{enumerate}
}{0}
\end{answer}

\begin{example}{experimento cujo espaço amostral é equiprovável.}

Suponha que o espaço amostral \(S\) é dado por \(\{ 1,2,3,4,5,6,7,8,9,10,11,12\}\) e que os eventos elementares são equiprováveis.

A função do LibreOffice \textit{=AleatórioEntre(1;12)} retornará com probabilidade \(\frac{1}{12}\) um dos elementos de \(S\).

A função do GeoGebra \textit{=NúmeroAleatório(1,12)} retornará com probabilidade \(\frac{1}{12}\) um dos elementos de \(S\).

\begin{figure}[H]
\centering

\noindent\includegraphics[width=400bp]{{geo1_1}.png}

\caption{Funções no Geogebra e LibreOffice}
\end{figure}

Observação:  Você poderá simular o lançamento de um dado ou de mais de um dado honesto e muitos outros experimentos simples, usando essas funções do Geogebra e do LibreOffice.
\end{example}
\begin{task}{simulação do lançamento de uma moeda honesta}


Deseja-se simular o lançamento de uma moeda honesta uma grande quantidade de vezes e comparar a frequência relativa de caras com a probabilidade teórica 0,5 de obter uma cara quando a moeda é honesta.

\begin{figure}[H]
\centering

\noindent\includegraphics[width=125bp]{moeda.jpg}
\end{figure}
\begin{enumerate}
\item {} 
Usando o GeoGebra ou algum outro recurso tecnológico, simule 20 lançamentos da moeda e observe a quantidade de caras, calculando a frequência relativa.

\item {} 
Repita a simulação para 50, 100, 250 e 1000 lançamentos da moeda.

\clearpage
\item {} 
Complete o quadro a seguir e comente sobre os resultados obtidos.

\end{enumerate}

\begin{table}[H]
\centering
\begin{tabu} to \textwidth{|c|c|}
\hline
\thead
Número de Observações & Frequência relativa de caras \\
\hline
20 &\\
\hline
50 &\\
\hline
100 &\\
\hline
250 &\\
\hline
1000 &\\
\hline
\end{tabu}
\end{table}

\end{task}

\begin{task}{simulação do lançamento de um dado honesto}


Deseja-se simular o lançamento de um dado honesto uma grande quantidade de vezes e comparar a frequência relativa de faces “6”{} com a probabilidade teórica \(\frac{1}{6}\approx 0,167\) de obter uma face “6”{} quando o dado é honesto.

\begin{figure}[H]
\centering

\noindent\includegraphics[width=225bp]{{lancamento_dado}.png}
\end{figure}
\begin{enumerate}
\item {} 
Usando o GeoGebra ou algum outro recurso tecnológico, simule 30 lançamentos do dado e observe a quantidade de faces “6”{} obtidas, calculando a frequência relativa.

\item {} 
Repita a simulação para 60, 120, 300 e 1500 lançamentos do dado.

\item {} 
Complete o quadro a seguir e comente sobre os resultados que você obteve.

\end{enumerate}
\begin{table}[H]
\centering
\begin{tabu} to \textwidth{|c|c|}
\hline
\thead
\centering
Número de Observações & Frequência relativa de 6 \\
\hline
30 &\\
\hline
60 &\\
\hline
120 &\\
\hline
300 &\\
\hline
1500 &\\
\hline
\end{tabu}
\end{table}
\end{task}
\clearpage
\begin{task}{simulação do lançamento de um dado desequilibrado}


Um dado é desequilibrado quando suas faces ocorrem com probabilidades desiguais. Suponha que um dado seja desequilibrado de tal modo que a probabilidade de ocorrer cada uma de suas faces, entre os números 1, 2, 3, 4, 5 e 6, sejam proporcionais aos respectivos números das faces.
\begin{enumerate}
\item {} 
Determine as probabilidades de cada face no caso desse dado.

\item {} 
Usando as funções de geração de números aleatórios, simule o lançamento desse dado 42 vezes e compare a frequência relativa de faces 6 obtidas com a probabilidade teórica da face 6 obtida no item anterior.

\item {} 
Repita o item anterior para 84, 210, 630, 840 e 1680 lançamentos do dado desequilibrado e registre o número de vezes que você obteve a face 6.

\item {} 
Complete a tabela a seguir e comente sobre os resultados que você obteve.

\end{enumerate}

\begin{table}[H]
\centering
\begin{tabu} to \textwidth{|c|c|}
\hline
\thead
Número de Observações & Frequência relativa de 6 \\
\hline
42 &\\
\hline
210 &\\
\hline
630 &\\
\hline
840 &\\
\hline
1680 &\\
\hline
\end{tabu}
\end{table}

\end{task}

\begin{task}{simulação do lançamento de um dado diferente}


Um dado é equilibrado, mas suas faces foram pintadas de tal modo que há uma face 1, duas faces 2 e três faces 3.
\begin{enumerate}
\item {} 
Determine as probabilidades de se obter face 1, face 2 e face 3 com esse dado.

\item {} 
Usando as funções de geração de números aleatórios, simule o lançamento desse dado 30 vezes e compare a frequência relativa de faces 2 obtidas com a probabilidade teórica da face 2 obtida no item anterior.

\item {} 
Repita o item anterior para 300, 600, 900 e 1800 lançamentos do dado equilibrado e registre o número de vezes que você obteve a face 2.

\item {} 
Complete a tabela a seguir e comente sobre os resultados que você obteve.

\end{enumerate}

\begin{table}[H]
\centering
\begin{tabu} to \textwidth{|c|c|}
\hline
\thead
Número de Observações & Frequência relativa de 2 \\
\hline
30 &\\
\hline
300 &\\
\hline
600 &\\
\hline
900 &\\
\hline
1800 &\\
\hline
\end{tabu}
\end{table}
\end{task}
\clearpage
\begin{observationtitle}{Teorema: Lei dos Grandes Números}

À medida que um experimento é repetido várias vezes, a probabilidade dada pela frequência relativa de um evento tende a se aproximar da probabilidade teórica.
\end{observationtitle}

A lei dos grandes números nos diz que as estimativas da probabilidade de um evento dadas pelas frequências relativas tendem a ficar melhores com mais observações. Esse resultado, como já comentado anteriormente, é a base teórica para a interpretação frequentista de probabilidade. Veja na \hyperref[1000_lancamentos2]{figura \ref{1000_lancamentos2}}, uma ilustração da Lei dos Grandes Números.

\begin{example}{Cálculo de probabilidade, usando simulação}

Deseja-se calcular a probabilidade de se obter pelo menos uma face 6, quando um dado honesto é lançado quatro vezes.

Definindo \(A\) como sendo o evento ocorreu pelo menos um seis nos quatro lançamentos, então \(\overline{A}\) é o evento nenhum 6 ocorreu nos quatro lançamentos.

Nesse caso, podemos usar a propriedade do evento complementar para calcular \(P(A)=1-P(\overline{A})\).
O evento \(\overline{A}\)  ocorre se, e somente se, em cada um dos quatro lançamentos não ocorre face 6. Como os lançamentos são independentes, a probabilidade de não ocorrer face 6 nos quatro lançamentos será dada por
\begin{equation*}
\begin{split}\frac{5}{6}\cdot \frac{5}{6}\cdot \frac{5}{6}\cdot \frac{5}{6}=\left (\frac{5}{6}\right )^4\approx 0,48\end{split}
\end{equation*}
Assim \(P(A) \approx 1- 0,48=0,52\).

Vamos estimar essa probabilidade, usando a Lei dos Grandes Números com o auxílio do GeoGebra.

Para simular o experimento podemos usar a função \textit{=Número Aleatório(1,6)} nas céluas A1 até A4.

\begin{figure}[H]
\centering

\noindent\includegraphics[width=370bp]{{GeoGebra_e1}.png}

\caption{Simulação de quatro lançamento de um dado com o GeoGebra}
\end{figure}

\hyperref[geogebra2]{figura \ref{geogebra2}}: Simulação de quatro lançamento de um dado com o GeoGebra

Na cela B4 usaremos a função \textit{=ContarSe(5\textless{}x\textless{}7,A1:A4)} para obter o número de faces 6 obtidas.

\begin{figure}[H]
\centering

\noindent\includegraphics[width=325bp]{{GeoGebra_e2}.png}
\caption{Contagem do número de faces 6 obtidas, usando o GeoGebra}
\label{geogebra2}
\end{figure}


Depois, marque as células A1:B4 e arraste-as até a linha 400, de modo a repetir 100 vezes esse experimento.

\begin{figure}[H]
\centering

\noindent\includegraphics[width=330bp]{{GeoGebra_e3_1}.png}

\caption{Simulação de 100 experimentos (lançar um dado 4 vezes), usando o GeoGebra}
\end{figure}


Vamos então usar a função \textit{ContarSe(-1\textless{}x\textless{}1,B1:B400)} para obter a quantidade de zeros, ou seja, o número de repetições do experimento em que a face 6 não ocorreu.

\begin{figure}[H]
\centering

\noindent\includegraphics[width=\linewidth]{{GeoGebra_e4_1}.png}

\caption{Contagem do número de ocorrências do evento “nenhuma face 6”}
\end{figure}


Observe que foram 43 ocorrências sem nenhuma face 6 de modo que nessa simulação a probabilidade estimada de obter pelo menos um 6 é dada por \(1-0,43=0,57\). Observe que essa estimativa ficou ligeiramente afastada da probabiliade teórica (\(\approx 0,52\)).

Duplicando o número de repetições do experimento, espera-se obter uma estimativa mais próxima da probabilidade teórica. Veja na \hyperref[geogebra5]{figura \ref{geogebra5}} um resultado, usando o GeoGebra. Agora, com 200 repetições, a estimativa  foi $0,54$ e, portanto, mais perto da probabilidade teórica.

\begin{figure}[H]
\centering

\noindent\includegraphics[width=\linewidth]{{GeoGebra_e5}.png}

\caption{Simulação de 200 experimentos (lançar um dado 4 vezes), usando o GeoGebra}
\label{geogebra5}
\end{figure}

\end{example}

\clearpage
\def\currentcolor{cor1}

\begin{answer}{Exercícios}
{\exerciselist
\begin{enumerate}
\item A sentença \textit{c)} é a única correta. Se a probabilidade de ocorrer um terremoto nos próximos $20$ anos é $\frac{2}{3}$, então a probabilidade de não ocorrer um terremoto nos próximos $20$ anos é $1-\frac{2}{3}=\frac{1}{3}$ (regra do evento complementar) tal que é duas vezes mais provável ele ocorrer do que não ocorrer nos próximos $20$ anos. A alternativa \textit{d)} não faz sentido, pois considera a fração $\frac{2}{3}$ de $20$ anos, quando a interpretação correta da probabilidade é a de que se pudessemos observar 60 períodos similares de $20$ anos na cidade de Zed, em cerca de 40 desses perídos haveria um terremoto. A alternativa \textit{b)} também está errada, pois uma medida de probabilidade maior que zero e menor que $1$ não nos fornece “certeza”. A alternativa \textit{d)} també está incorrreta, pois sabe-se que há maior chance de ocorrer um terremoto do que não ocorrer, apesar de não sermos capazes de afirtmar se ele irá ocorrer ou não e quando ele irá ocorrer.

\item A alternativa que melhor corresponde à afirmação é \textit{d)}. A probabilidade indica uma taxa de ocorrência do evento, caso ele seja repetido um grande número de vezes. Assim, se observarmos $100$ dias similares, em cerca de $30$ desses dias irá chover. Observe que esse número de fato pode não ser exatamente $30$, apenas espera-se que ele seja próximo de $30$. Lembre-se que se você lançar uma moeda honesta $20$ vezes, não necessariamente irá observar $10$ caras, mesmo sabendo que a probabilidade de ocorrer cara em cada lançamento é $50\%$.

\item Se a probabilidade de um casal com três filhos ter duas meninas é $\frac{3}{8}$,
\begin{enumerate}
\item espera-se que entre $160$ casais com três filhos cerca de $160\cdot\frac{3}{8}=60$ casais tenham duas meninas.

\item espera-se que entre $400$ casais com três filhos cerca de $400\cdot\frac{3}{8}=150$ casais tenham duas meninas.

\item Não, pois espera-se que entre $80$ casais com três filhos cerca de $80\cdot\frac{3}{8}=30$ casais tenham duas meninas, podendo ocorrer um número de casais dferente de $30$.
\end{enumerate}
\item As interpretações mais adequadas são

\begin{enumerate}
\item clássica, pois sendo o dado honesto atribuem-se probabiliades iguais para cada uma das $6$ faces possíveis, a saber, $\frac{1}{6}$.
\item frequentista. Nesse caso, a partir da observação de um grande número de tempos de espera, o gerente poderá usar a frequência relativa de ocorrência dos tempos de espera de mais de 30 minutos como a probabilidade do evento “esperar mais de $30$ inutos para ser atendido”.
\item subjetiva.
\end{enumerate}
\end{enumerate}
}{1}
\end{answer}
\clearmargin

%Pagina 2
\begin{answer}{Exercícios}
{\exerciselist
\begin{enumerate}\setcounter{enumi}{4}
\item Observe que os eventos complementares são $\overline{A}$: “ser do turno da tarde” e $\overline{B}$: “não pratica atividade física fora do horário escolar”. Descrições

\begin{enumerate}
\item Ocorre pelo menos um dos eventos entre “ser do turno da manhã” ou “praticar atividade física fora do horário escolar”, podendo ocorrer apenas um dos dois ou os dois simultaneamente.
\item Ocorrem os dois eventos “ser do turno da manhã” e “praticar atividade física fora do horário escolar” simultaneamente.
\item Ocorrem os dois eventos “ser do turno da tarde” e “não praticar atividade física fora do horário escolar” simultaneamente.
\item Observe que pelas Leis de De Morgan $\overline{A\cap B}=\overline{A}\cup \overline{B}$ tal que uma descrição para esse evento é Ocorre pelo menos um dos eventos entre “ser do turno da tarde” ou “não praticar atividade física fora do horário escolar”, podendo ocorrer apenas um dos dois ou os dois simultaneamente.
\end{enumerate}

\item Sejam os eventos $A$: “ter 15 anos”; $B$: “ter 16 anos” e $C$: “ter 17 anos”. Como só há essas três idades, segue que $S=A\cup B\cup C$ e como $A$, $B$ e $C$ são disjuntos, tem-se $1=P(S)=P(A)+P(B)+P(C)$. Sabe-se que $P(A)=0{,}3$ e $P(C)=0{,}2$, logo $P(B)=1-0{,}3-0{,}2=0{,}5$.
\begin{enumerate}
\item $P(C)=0{,}2$.
\item $P(B)=0{,}5$.
\item $P(S)=1$.
\item Como $A$ e $B$ são disjuntos, tem-se $P(A\cup B)=P(A)+P(B)=0{,}3+0{,}5=0{,}8$.
\end{enumerate}
\item Definindo o evento $R$: “ter Rh+” tal que $\overline{R}$: “ter Rh-” e usando os símbolos do tipo de sangue, tem-se
\begin{enumerate}[wide]
\item Como $P(R)=0{,}75$, segue que o número esperado de pessoas com fator Rh+ é $1.000.000\cdot 0{,}75=750.000$
\item Como $P(A)=0{,}30$, segue que o número esperado de pessoas sangue tipo A é $300$ mil.
\item Como $P(AB\cap R)=0{,}16$, o número esperado de pessoas com tipo AB é 160 mil.
\item Como $P(\overline{O})=1-0{,}40=0{,}60$, o número esperado de pessoas que não têm tipo O é $600$ mil.
\item Como $P(\overline{O}|R)=\frac{0{,}75-0{,}32}{0{,}75}\approx0{,}57$, o número de esperado de pessoas com fator Rh+ que têm sangue tipo O é $0{,}57\cdot750.000=427.500$.
\item Como $P(\overline{O}|\overline{R})=0{,}25-0{,}080{,}25=0{,}68$, o número esperado de pessoas que não têm sangue tipo O entre as pessoas com fator Rh− é $0{,}68\cdot250.000=170.000$.
\end{enumerate}

\end{enumerate}
}{1}
\end{answer}
\clearmargin

%Pagina 3
\begin{answer}{Exercícios}
{\exerciselist
\begin{enumerate}\setcounter{enumi}{7}
\item Opção correta: \textit{c)}

Defina os eventos $U1$ e $U2$ para suco de uva na primeira seleção e suco de uva na segunda seleção. Similarmente, defina os eventos $L1$ e $L2$ e $P1$ e $P2$ para os sabores laranja e pêssego, respectivamente. Deseja-se calcular a probabilidade de que os dois sucos selecionados sejam do mesmo sabor, $P\big[(U1\cap U2)\cup (L1\cap L2)\cup (P1\cap P2)\big]=P(U1\cap U2)+P(L1\cap L2)+P(P1\cap P2)$, pois os eventos são disjuntos. Usando a regra da multiplicação tem-se que $P(U1\cap U2)=P(U1)\cdot P(U2|U1)=412\cdot 311=111$. Como as quantidades de sucos dos tr~es sabores são iguais, segue que a resposta é $3\cdot 111=311\approx0{,}273=27{,}3$.

Essa solução é facilmente visualisada, usando-se o diagrama de árvore ilustrado na \hyperref[arvore-suco]{figura \ref{arvore-suco}}.

\notas{\begin{figure}[H]
\centering

\resizebox{.85\linewidth}{!}
{
\begin{tikzpicture}[scale=.5]
\draw (0,0) -- (45:3) node [right, scale=0.3 ] {Uva} node [midway, above, scale=0.3, rotate=45] {4/12};
\draw (0,0) -- (-45:3) node [right, scale=0.3] {Pêssego} node [midway, below, scale=0.3, rotate=-45] {4/12};
\draw (0,0) -- (2.12132,0) node [right, scale = 0.3] {Laranja} node [midway, above, scale=0.3] {4/12};

\draw (3.12132,2.12132) -- ++(20:2) node [right, scale=0.3 ] {Uva} node [near end, above, scale=0.2, rotate=20] {3/11};
\draw (3.12132,2.12132) -- ++(0:1.8793) node [right, scale = 0.3]  {Laranja} node [near end, above, scale=0.2] {4/11};
\draw (3.12132,2.12132) -- ++(-20:2) node [right, scale=0.3] {Pêssego} node [near end, below, scale=0.2, rotate=-20] {4/11};

\draw (3.12132,0) -- ++(20:2) node [right, scale=0.3 ] {Uva} node [near end, above, scale=0.2, rotate=20] {4/11};
\draw (3.12132,0) -- ++(0:1.8793) node [right, scale = 0.3] {Laranja} node [near end, above, scale=0.2] {3/11};
\draw (3.12132,0) -- ++(-20:2) node [right, scale=0.3] {Pêssego} node [near end, below, scale=0.2, rotate=-20] {4/11};


\draw (3.12132,-2.12132) -- ++(20:2) node [right, scale=0.3 ] {Uva} node [near end, above, scale=0.2, rotate=20] {4/11};
\draw (3.12132,-2.12132) -- ++(0:1.8793) node [right, scale = 0.3] {Laranja}  node [near end, above, scale=0.2] {4/11};
\draw (3.12132,-2.12132) -- ++(-20:2) node [right, scale=0.3] {Pêssego} node [near end, below, scale=0.2, rotate=-20] {3/11};

\draw [->] (6.250062,0) -- ++(0:2) node [above, midway, align=center,scale=0.3] {sabores\\iguais} node [right, scale=0.5] {$\frac{4}{12} \cdot \frac{3}{11}$};

\draw [->] (6.250062,2.8053) -- ++(0:2) node [above, midway, align=center,scale=0.3] {sabores\\iguais} node [right, scale=0.5] {$\frac{4}{12} \cdot \frac{3}{11}$};

\draw [->] (6.250062,-2.8053) -- ++(0:2) node [above, midway, align=center,scale=0.3] {sabores\\iguais} node [right, scale=0.5] {$\frac{4}{12} \cdot \frac{3}{11}$};
\end{tikzpicture}
}

\caption{Solução do exercício 8}
\label{arvore-suco}
\end{figure}}

\item Opção correta: \textit{b)}

Basta verificar que dos $1000$ consumidores pesquisados $50+100+200=350$ não pretendem mudar de modelo.

\item Opção correta: \textit{a)}

Primeiro será necessário calcular a probabilidade de um aluno qualquer dessa turma acertar a questão. Temos que $20\%$ dos alunos a acertam com probabilidade $1$, mas $80\%$ chutam, acertando com probabilidade $14$, pois há quatro opções de resposta. Assim, a probabilidade de um aluno dessa turma acertar a questão é dada por $0{,}2\cdot1+0{,}8\cdot\frac{1}{4}=0{,}4$.

Usando o diagrama de árvore na \hyperref[prova-aluno]{figura \ref{prova-aluno}}, podemos escrever para os dois alunos selecionados:

\begin{figure}[H]
\centering

\resizebox{.95\linewidth}{!}
{
\begin{tikzpicture}[scale=0.4, ever node/.style=scale{2}]

         \draw (0,0) -- (30:3) node [right, scale=0.28] {Acerta} node [above, midway, rotate=30, scale=0.28] {0,3};
         \draw (0,0) -- (-30:3) node [right, scale=0.28] {Erra} node [below, midway, rotate=-30, scale=0.28] {0,7};
         
         \draw (4.159807,1.5) -- ++(20:3) node [right, scale=0.28] {Acerta} node [above, midway, rotate=20, scale=0.28] {0,5};
         \draw (4.159807,1.5) -- ++(-20:3) node [right, scale=0.28] {Erra} node [below, midway, rotate=-20, scale=0.28] {0,5};
         
         \draw (4.159807,-1.5) -- ++(20:3) node [right, scale=0.28] {Acerta} node [above, midway, rotate=20, scale=0.28] {0,25};
         \draw (4.159807,-1.5) -- ++(-20:3) node [right, scale=0.28] {Erra} node [below, midway, rotate=-20, scale=0.28] {0,75};
         
         \node [right, scale=0.28] at (5.77888,3.5260) {Aluno 2}; 
         \node [right, scale=0.28] at (1.27888,3.5260) {Aluno 1}; 
                     
         \node [right, scale=0.28] at (8.77888,3.5260) {Situação}; 
         \node [right, scale=0.28] at (8.77888,2.5260) {Dois Acertos}; 
         \node [right, scale=0.28] at (8.77888,0.4739) {Um Acerto}; 
         \node [right, scale=0.28] at (8.77888,-0.4739) {Um Acerto}; 
         \node [right, scale=0.28] at (8.77888,-2.5260) {Dois Erros}; 
         
         \node [right, scale=0.28] at (10.77888,3.5260) {Probabilidade};
         \node [right, scale=0.28] at (10.77888,2.5260) {$0,4 \cdot 0,4=0,16$}; 
         \node [right, scale=0.28] at (10.77888,0.4739) {$0,4 \cdot 0,6=0,24$}; 
         \node [right, scale=0.28] at (10.77888,-0.4739) {$0,6 \cdot 0,4=0,24$}; 
         \node [right, scale=0.28] at (10.77888,-2.5260) {$0,6 \cdot 0,6=0,36$};
\end{tikzpicture}
}
\caption{Solução do exercício 10}
\label{prova-aluno}
\end{figure}
Logo, a probabilidade de que exatamente um aluno tenha marcado a opção correta é dada por $0{,}24+0{,}24=0{,}48$.

\end{enumerate}
}{1}
\end{answer}

%Pagina 4
\begin{answer}{Exercícios}
{\exerciselist
\begin{enumerate}\setcounter{enumi}{10}
\item Opção correta: \textit{d)}

No método I, a probabilidade de selecionar um aluno do diurno será dada pela probabilidade de selecionar o turno diurno ($\frac{1}{3}$) vezes a probabilidade de selelcionar um aluno do diurno ($\frac{1}{300}$), ou seja, será dada por $\frac{1}{600}$. Já a probabilidade de selecionar um aluno do noturno será dada por $\frac{1}{2}\cdot\frac{1}{240}=\frac{1}{480}$. Assim, no método I, a probabilidade de selecionar um aluno do noturno é maior do que a probabilidade de selecionar um aluno do diurno. Portanto, as alternativas \textit{a)}, \textit{b)}, \textit{c)} e \textit{d)} são falsas.

No método II, a probabilidade de selecionar um aluno do diurno é dada por $\frac{10}{16}\cdot\frac{1}{30}=\frac{1}{48}=\frac{20}{960}$. Já a probabilidade de selecionar um aluno do noturno é $\frac{6}{16}\cdot\frac{1}{40}=\frac{3}{320}=\frac{9}{960}$. Assim, no método II, a probabilidade de selecionar um aluno do noturno é menor do que a probabilidade de selecionar um aluno do diurno. Portanto, a alternativa \textit{d)} está correta.

\item Opcão correta: \textit{b)}

Para calcular a soma das áreas dos setores circulares na figura dada é necessário conhecer os respectivos ângulos centrais. No entanto, basta conhecer a soma dos ângulos centrais, que pela figura do trapézio implica que ela é $180$ graus tal que os dois setores juntos podem formar a semi-circunferência de raio $10$ km. Logo, a probabilidade é dada por $\frac{10^2\pi/2}{628}\approx\frac{314}{2\cdot628}=0{,}25$.

\item Opção correta: \textit{c)}

Usando o diagrama de árvore ilustrado na \hyperref[chuva-atraso]{figura \ref{chuva-atraso}} tem-se que a probabilidade de atraso é dada por $0{,}15+0{,}175=0{,}325$.\textbf{}

\notas
{
\begin{figure}[H]
\centering

\resizebox{\linewidth}{!}
{
\begin{tikzpicture}[scale=0.5]
         \draw (0,0) -- (30:3) node [right, scale=0.35] {Chove} node [above, midway, rotate=30, scale=0.35] {0,3};
         \draw (0,0) -- (-30:3) node [right, scale=0.35] {Não Chove} node [below, midway, rotate=-30, scale=0.35] {0,35};
         
         \draw (4.159807,1.5) -- ++(20:3) node [right, scale=0.35] {Atrasa} node [above, midway, rotate=20, scale=0.35] {0,5};
         \draw (4.159807,1.5) -- ++(-20:3) node [right, scale=0.35] {Não Atrasa} node [below, midway, rotate=-20, scale=0.35] {0,5};
         
         \draw (4.159807,-1.5) -- ++(20:3) node [right, scale=0.35] {Atrasa} node [above, midway, rotate=20, scale=0.35] {0,25};
         \draw (4.159807,-1.5) -- ++(-20:3) node [right, scale=0.35] {Não Atrasa} node [below, midway, rotate=-20, scale=0.35] {0,75};
         
         \node [right, scale=0.35] at (8.77888,2.5260) {prob.: 0,15}; 
         \node [right, scale=0.35] at (8.77888,0.4739) {prob.: 0,15}; 
         \node [right, scale=0.35] at (8.77888,-0.4739) {prob.: 0,175}; 
         \node [right, scale=0.35] at (8.77888,-2.5260) {prob.: 0,525}; 
\end{tikzpicture}
}
\caption{Solução do exercício 13}
\label{chuva-atraso}
\end{figure}
}


Alternativamente, definindo A como o evento se atrasar e C como o evento chover, deseja-se calcular $P(A)$. Mas, $A=(A\cap C)\cup(A\cap\overline{C})$ com os dois eventos do lado esquerdo da igualdade disjuntos. Logo, 
\begin{align*}
P(A)&=P(A\cap C)+P(A\cap\overline{C})\\
&=P(C)\cdot P(A|C)+P(\overline{C})⋅P(A|\overline{C})\\
&=0{,}3\cdot 0{,}5+0{,}7\cdot 0{,}25\\
&=0{,}325
\end{align*}.
\end{enumerate}
}{1}
\end{answer}
\clearmargin

%Pagina 5
\begin{answer}{Exercícios}
{\exerciselist
\begin{enumerate}\setcounter{enumi}{13}
\item Opção correta: \textit{a)}

Primeiro observe que existem $10$ eventos possíveis que levam a observar apenas um semáforo verde, a saber, pode ser o primeiro ou o segundo, ou o terceiro e assim por diante até o décimo. Em seguida, observe que esses eventos são disjuntos de modo que a probabilidade desejada pode ser calculada pela soma das probabilidades desses $10$ eventos ou, como elas são iguais, pelo produto da probabilidade de apenas a primeira ser verde por $10$. Finalmente, pela independência, observe que qualquer que seja o evento a probabilidade é dada por $\frac{2}{3}\cdot\frac{1}{3}\cdots\frac{1}{3}=\frac{2}{3}\cdot\big(\frac{1}{3}\big)^9=\frac{2}{3^10}$. Assim a probabilidade solicitada é $\frac{10\cdot2}{3^10}$.

\item Opção correta: \textit{c)}
Observe o diagramma de Venn a seguir com as informações do enunciado. Lembre que $80\%$ de $25\%$ corresponde a $20\%$.

\begin{figure}[H]
\centering

\resizebox{.4\linewidth}{!}
{
\begin{tikzpicture}[scale=0.5, every node/.style={scale=1.25}]
         \node [scale=0.5] at (7,5.25) {S};
         \draw [ thick,rounded corners=8pt, -] (0,-0) -- (0,5) -- (8,5) -- (8,0) -- cycle;
         \node [scale=0.5] at (6.6,4) {Febre};
         \node [scale=0.5] at (1,4) {Gripe};
         \draw  (4.5,2.5) circle (2cm);
         \node [scale=0.5] at (5.75,2.5) {0,08};
         \node [scale=0.5] at (6.8,0.5) {0,67};
         \clip [draw]  (3,2.5) circle (2cm) ; 
         \draw [] (4.5,2.5) circle (2cm);
         \node [scale=0.5] at (3.75,2.5) {0,20};
         \node [scale=0.5] at (1.75,2.5) {0,05};
         \node [scale=0.5] at (6.8,0.5) {909};
\end{tikzpicture}
}
\end{figure}

Logo, a probabilidade de ter tido febre durante o surto é dada por $0{,}20+0{,}08=0{,}28$.
\begin{equation*}
P(E|\overline{I})=\frac{P(E\cap \overline{I})}{P(\overline{I})}=\frac{300/1200}{600/1200}=\frac{1}{2}.
\end{equation*}

\item Opção correta: \textit{a)}

Observe o diagramma de Venn a seguir com as informações do enunciado.

\begin{figure}[H]
\centering

\resizebox{.4\linewidth}{!}
{
\begin{tikzpicture}[scale=0.5, every node/.style={scale=1.25}]
         \node [scale=0.5] at (7,5.25) {S};
         \draw [ thick,rounded corners=8pt, -] (0,-0) -- (0,5) -- (8,5) -- (8,0) -- cycle;
         \node [scale=0.5] at (6.6,4) {Febre};
         \node [scale=0.5] at (1,4) {Gripe};
         \draw  (4.5,2.5) circle (2cm);
         \node [scale=0.5] at (5.75,2.5) {0,08};
         \node [scale=0.5] at (6.8,0.5) {0,67};
         \clip [draw]  (3,2.5) circle (2cm) ; 
         \draw [] (4.5,2.5) circle (2cm);
         \node [scale=0.5] at (3.75,2.5) {0,20};
         \node [scale=0.5] at (1.75,2.5) {0,05};
         \node [scale=0.5] at (6.8,0.5) {909};
\end{tikzpicture}
}
\end{figure}

Logo, a probabilidade de falar espanhol dado que não fala inglês é dada por


\end{enumerate}
}{1}
\end{answer}
\clearmargin

%Pagina 6
\begin{answer}{Exercícios}
{\exerciselist
\begin{enumerate}\setcounter{enumi}{16}
\item Opção correta: \textit{b)}

Observe que o teste terminará na quinta questão se o segundo erro do candidato ocorrer nessa questão. Portanto, para o primeiro erro haverá quatro eventos possíveis, a saber, na primeira ou na segunda ou na terceira ou na quarta questão. Como esses eventos são disjuntos, a probabilidade desejada será dada pela soma das probabilidades desses $4$ eventos. Observe também que cada um desses eventos, usando a independência entre questões, ocorre com probabilidade, $0{,}2^2\cdot0{,}8^3=0{,}02048$, correspondendo a dois erros e três acertos. Portanto, a resposta é $4\cdot0{,}02048=0,08192$.


\item 
\begin{enumerate}
\item Para as seleções sem reposição tem-se o seguinte diagrama de árvore

\begin{figure}[H]
\centering

\resizebox{.5\linewidth}{!}
{
\begin{tikzpicture}[scale=0.5]
         \draw (0,0) -- (30:3) node [right, scale=0.35] {D} node [above, midway, rotate=30, scale=0.35] {5/20};
         \draw (0,0) -- (-30:3) node [right, scale=0.35] {B} node [below, midway, rotate=-30, scale=0.35] {15/20};
         
         \draw (3.159807,1.5) -- ++(20:3) node [right, scale=0.35] {D} node [above, midway, rotate=20, scale=0.35] {4/19};
         \draw (3.159807,1.5) -- ++(-20:3) node [right, scale=0.35] {B} node [below, midway, rotate=-20, scale=0.35] {15/19};
         
         \draw (3.159807,-1.5) -- ++(20:3) node [right, scale=0.35] {D} node [above, midway, rotate=20, scale=0.35] {5/19};
         \draw (3.159807,-1.5) -- ++(-20:3) node [right, scale=0.35] {B} node [below, midway, rotate=-20, scale=0.35] {14/19};

\end{tikzpicture}
}
\end{figure}
  Assim as respostas são i) $ \frac{2}{38}$ ii) $\frac{21}{38}$ iii) $\frac{5}{20}\cdot\frac{15}{19}+\frac{15}{20}\cdot\frac{5}{19}=\frac{15}{38}$.





\item Para as seleções com reposição tem-se o seguinte diagrama de árvore com D o evento "peça defeituosa e B peça não defeituosa".

\begin{figure}[H]
\centering

\resizebox{.5\linewidth}{!}
{
\begin{tikzpicture}[scale=0.5]
         \draw (0,0) -- (30:3) node [right, scale=0.35] {D} node [above, midway, rotate=30, scale=0.35] {5/20};
         \draw (0,0) -- (-30:3) node [right, scale=0.35] {B} node [below, midway, rotate=-30, scale=0.35] {15/20};
         
         \draw (3.159807,1.5) -- ++(20:3) node [right, scale=0.35] {D} node [above, midway, rotate=20, scale=0.35] {5/20};
         \draw (3.159807,1.5) -- ++(-20:3) node [right, scale=0.35] {B} node [below, midway, rotate=-20, scale=0.35] {15/20};
         
         \draw (3.159807,-1.5) -- ++(20:3) node [right, scale=0.35] {D} node [above, midway, rotate=20, scale=0.35] {5/20};
         \draw (3.159807,-1.5) -- ++(-20:3) node [right, scale=0.35] {B} node [below, midway, rotate=-20, scale=0.35] {15/20};

\end{tikzpicture}
}
\end{figure}

Assim, as respostas são i) $\frac{2}{38}$ ii) $\frac{21}{38}$ iii) $\frac{5}{20}\cdot\frac{15}{20}+\frac{15}{20}\cdot\frac{5}{20}=\frac{6}{16}$.


\end{enumerate}


\end{enumerate}
}{1}
\end{answer}

\exercise
% \vspace{-2em}

\begin{enumerate}

\item (Pisa) Imagine que tenha sido apresentado um documentário sobre terremotos, no qual é dito com que frequência ocorrem  e como podem ser previstos. Nesse documentário, um geólogo afirmou: “nos próximos  20 anos, a chance de um terremoto acontecer na cidade de Zed é de duas em três.”

Qual das sentenças a seguir reflete o significado da afirmação feita pelo geólogo?
\begin{enumerate}
\item {} 
Como \(\frac{2}{3}\cdot 20\approx 13{,}3\); entre 13 e 14 anos a partir de agora, ocorrerá um terremoto na cidade de Zed.

\item {} 
Como \(\frac{2}{3}\) é maior do que \(\frac{1}{2}\), temos certeza de que ocorrerá um terremoto na cidade de Zed em algum momento nos próximos 20 anos.

\item {} 
A probabilidade de ocorrer algum terremoto na cidade de Zed em algum momento nos próximos 20 anos é maior do que a probabilidade de ele não ocorrer.

\item {} 
Não se pode dizer sobre o que irá acontecer, pois ninguém sabe quando um terremoto ocorrerá.

\end{enumerate}

\item (Pisa) Para dado dia, a previsão do tempo afirma que, das 12h às 18h, a chance de ocorrência de chuva é de $30\%$.

Assinale a alternativa que corresponde à melhor interpretação dessa previsão do tempo.
\begin{enumerate}
\item {} 
Em $30\%$ da área à qual a previsão se refere haverá chuva.

\item {} 
em $30\%$ de $6$ horas, ou seja, durante o total de $108$ minutos, haverá chuva.

\item {} 
Em relação às pessoas da área à qual a previsão se refere, pode-se afirmar que $30$ a cada $100$ pessoas pegarão chuva.

\item {} 
Se a mesma previsão fosse dada para $100$ dias, em cerca de $30$ desses $100$ dias haveria chuva.

\item {} 
A quantidade de chuva será $30\%$ da intensidade de uma forte precipitação (medida como “chuva por unidade de tempo”).

\end{enumerate}

\item Sabe-se que a probabilidade de que um casal com três filhos tenha duas meninas é \(\frac{3}{8}\). Com base nessa afirmação, responda:
\begin{enumerate}
\item {} 
Observando-se ao acaso $160$ casais com três filhos, quantos casais você espera que tenham duas meninas?

\item {} 
E em 400 casais com três filhos?

\item {} 
É correto afirmar que se observarmos $80$ casais com três filhos, exatamente 30 casais terão duas meninas? Por quê?

\end{enumerate}


\item Nas situações a seguir indique a interpretação de probabilidade, entre as interpretações clássica, frequentista e subjetiva, que melhor se encaixa para a designação de probabilidades.
\begin{enumerate}
\item {} 
Um dado honesto é lançado três vezes, deseja-se calcular a probabilidade de se obter pelo menos duas faces pares.

\item {} 
Um gerente de banco deseja avaliar a probabilidade de um cliente esperar mais de $30$ minutos para ser atendido em um dia comum.

\item {} 
Deseja-se avaliar a probabilidade de você se tornar um \textit{Youtuber} de sucesso no próximo ano.

\end{enumerate}

\item Em uma escola de Ensino Médio há dois turnos: manhã e tarde. Nesta escola há alunos que praticam atividade física fora do horário escolar e alunos que não praticam. Um aluno dessa escola será sorteado. Defina os seguintes eventos \(A:\) “o aluno sorteado é do turno da manhã”{} e \(B:\) “o aluno sorteado pratica atividade física fora do horário escolar”. Descreva em palavras os eventos
\begin{enumerate}
\item {} 
\(A\cup B\)

\item {} 
\(A\cap B\)

\item {} 
\(\overline{A}\cap \overline{B}\)

\item {} 
\(\overline{A\cap B}\)

\end{enumerate}

\item É sorteado um aluno de uma turma de um curso de Inglês na qual há somente alunos de 15 anos, 16 anos e 17 anos completos. Sabe-se que a  probabilidade de o aluno selecionado ter 15 anos é igual a $0,{,}$ e ter 17 anos é igual a $0{,}2$.
Determine a probabilidade de o aluno sorteado
\begin{enumerate}
\item {} 
ter idade maior do que 16 anos.

\item {} 
ter 16 anos.

\item {} 
ter pelo menos 15 anos.

\item {} 
ter 15 ou 16 anos.

\end{enumerate}

\item Um posto de coleta de sangue fez um lavantamento das fichas de $500$ doadores, obtendo as seguintes frequências relativas por tipo sanguíneor e fator Rh.

\begin{table}[H]
\centering
\begin{tabu} to \textwidth{|c|c|c|c|}
\hline
\thead
tipo & Rh+ & Rh- & total \\
\hline
O & $0{,}32$ & $0{,}08$ & $0{,}40$ \\
\hline
A & $0{,}20$ & $0{,}10$ & $0{,}30$ \\
\hline
AB & $0{,}16$ & $0{,}04$ & $0{,}20$ \\
\hline
B & $0{,}07$ & $0{,}03$ & $0{,}10$ \\
\hline
Total & $0{,}75$ & $0{,}25$ & $1{,}00$ \\
\hline
\end{tabu}
\end{table}


Supondo que o comportamento desses doadores reflita o comportamento da população da região do posto de coleta que contém 1 milhão de pessoas, pede-se calcular, de forma aproximada, o número de pessoas nessa população que
\begin{enumerate}
\item {} 
tenha fator Rh-;

\item {} 
tenha sangue tipo A;

\item {} 
tenha sangue tipo AB com fator Rh+;

\item {} 
não tenha sangue tipo O.

\item {} 
não tenha sangue tipo O, sabendo que tem fator Rh+.

\item {} 
não tenha sangue tipo O, sabendo que tem fator Rh-.

\end{enumerate}

\clearpage

\item (UERJ-EQ1-2011) Uma fábrica produz sucos com os seguintes sabores: uva, pêssego e laranja. Considere uma caixa com 12 garrafas desses sucos, sendo quatro de cada sabor. Retirando-se ao acaso duas garrafas dessa caixa, a probabilidade de que ambas as garrafas contenham suco com o mesmo saber equivale a:
\begin{enumerate}
\item {} 
$9{,}1\%$

\item {} 
$18{,}2\%$

\item {} 
$27{,}3\%$

\item {} 
$36{,}4\%$

\end{enumerate}

\item (UERJ-EQ1-2012-adaptada) Três modelos de aparelhos de ar condicionado I, II e III, de diferentes potências são produzidos por determinado fabricante. Uma consulta sobre intenção de troca de modelo foi realizada com 1000 usuários desses produtos. No quadro a seguir estão indicados os tipos de modelos que os usuários possuem e se eles pretendem mudar para outro modelo ou não.

Dos 400 que possuem o modelo I, $50$ não pretendem mudar de modelo, $150$ pretendem mudar para o II e $250$ para o III. Dos 400 que possuem o modelo II, 100 não pretendem mudar e $300$ pretendem mudar para o III. E, dos $200$ que possuem o modelo III, nenhum deles tem intenção de mudar.

Escolhendo-se aleatoriamente um dos usuários consultados, a probabilidade de que ele não pretenda trocar seu modelo de ar condicionado é:
\begin{enumerate}
\item {} 
$20\%$

\item {} 
$35\%$

\item {} 
$40\%$

\item {} 
$65\%$

\end{enumerate}

\item (UERJ-EQ2-2013) Em uma escola, $20\%$ dos alunos de uma turma marcaram a opção correta de uma questão de múltipla escolha que possui quatro alternativas de resposta. Ode demais alunos marcaram uma das quatro opções ao acaso. Verificando-se as respostas de dois alunos quaisquer dessa turma, a probabilidade de que exatamente um tenha marcado a opção correta é:
\begin{enumerate}
\item {} 
$0{,}48$

\item {} 
$0{,}40$

\item {} 
$0{,}36$

\item {} 
$0{,}25$

\end{enumerate}

\item (ENEM) Um aluno de determinada escola será escolhido por sorteio para representa-la em certa atividade. A escola tem dois turnos. No diurno há 300 alunos, distribuídos em $10$ turmas de $30$ alunos cada. No noturno há 240 alunos, distribuídos em $6$ turmas de $40$ alunos cada. Em vez do sorteio direto, envolvendo os $540$ alunos, foram propostos dois outros métodos de sorteio.
MÉTODO I: Escolher ao acaso um dos turnos e, a seguir, sortear um dos alunos do turno escolhido.
MÉTODO II: Escolher ao acaso uma das 16 turmas e, a seguir, sortear um dos alunos dessa turma.
Sobre os métodos I e II é correto afirmar:
\clearpage
\begin{enumerate}
\item {} 
Em ambos os métodos, todos os alunos têm a mesma chance de serem sorteados.

\item {} 
No método I, todos os alunos têm a mesma chance de serem sorteados, mas no método II a chance de um aluno do diurno ser sorteado é maior do que a de um aluno do noturno.

\item {} 
No método II, todos os alunos têm a mesma chance de serem sorteados, mas no método I a chance de um aluno do diurno ser sorteado é maior do que a de um aluno do noturno.

\item {} 
No método I, a chance de um aluno do noturno ser sorteado é maior do que a chance de um aluno do diurno, enquanto no método II ocorre o contrário.

\item {} 
Em ambos os métodos, a chance de um aluno do diurno ser sorteado é maior do que a de um aluno do noturno.

\end{enumerate}

\item (ENEM) Um município de $628$km$^2$ é atendido por duas emissoras de rádio cujas antenas A e B alcançam um raio de 10 km do município, conforme mostra a \hyperref[municipio]{figura \ref{municipio}}:
\begin{figure}[H]
\centering

\begin{tikzpicture}[scale=.9]

\draw [color=secundario!70, fill=primario!70] (2,3) -- (3,3) -- (3.55,2.16);
\draw [dashed, ] (0,0) -- (0,3);
\draw [] (0,3) -- (3,3) -- (5,0) -- (0,0);
\draw [, fill=primario!70] (2,3) arc (-180:-56.1:1);
\draw [, fill=primario!70] (5,0) -- (4,0) arc (180:123.9:1) -- cycle;
\node [above right, ] at (3,3) {$A$};
\node [above, scale=0.8] at (2.5,3) {10 km};
\node [rotate=-56.1, above, scale=0.8] at (4.718, 0.395) {10 km};
\node [below right, ] at (5,0) {$B$};
\node [below, scale=0.8] at (4.5,0) {10 km};
\draw [very thin] (0,0) rectangle (0.2,0.2);
\draw [very thin] (0,3) rectangle (0.2,2.8);
\node [ponto] at (0.1,0.1) {};
\node [ponto] at (0.1,2.9) {};
\draw plot [smooth, tension=1] coordinates {(1,3) (0.75,1.5) (1.5,0)};
\node [right] at (1,1.5) {Município};
\end{tikzpicture}
\caption{Planta do município}
\label{municipio}
\end{figure}

Para orçar um contrato publicitário, uma agência precisa avaliar a probabilidade que um morador tem de, circulando livremente pelo município, encontrar-se na área de alcance de pelo menos uma das emissoras. Essa probabilidade é de aproximadamente:
\begin{enumerate}
\item {} 
20\%

\item {} 
25\%

\item {} 
30\%

\item {} 
35\%

\item {} 
40\%

\end{enumerate}

\item (ENEM - 2017) Um morador de uma região metropolitana tem $50\%$ de probabilidade de atrasar-se para o trabalho quando chove na região; caso não chova, sua probabilidade de atraso é de $25\%$. Para um determinado dia, o serviço de meteorologia estima em $30\%$ a probabilidade da ocorrência de chuva nessa região.

Qual é probabilidade de esse morador se atrasar para o serviço no dia para o qual foi dada a estimativa de chuva?
\begin{enumerate}
\item {} 
$0{,}075$

\item {} 
$0{,}150$

\item {} 
$0{,}325$

\item {} 
$0{,}600$

\item {} 
$0{,}800$

\end{enumerate}

\item (ENEM - 2017) Numa avenida existem 10 semáforos. Por causa de uma pane no sistema, os semáforos ficaram sem controle durante uma hora, e fixaram suas luzes unicamente em verde ou vermelho. Os semáforos funcionam de forma independente; a probabilidade de acusar cor verde é de \(\frac{2}{3}\) e a de acusar cor vermelha é de \(\frac{1}{3}\). Uma pessoa percorreu a pé toda essa avenida durante o período da pane, observando a cor da luz de cada um desses semáforos.

Qual é a probabilidade de que esta pessoa tenha observado exatamente um sinal na cor verde?
\begin{enumerate}
\item {} 
\(\displaystyle\frac{10\cdot 2}{3^{10}}\)

\item {} 
\(\displaystyle \frac{10\cdot 2^9}{3^{10}}\)

\item {} 
\(\displaystyle\frac{2^{10}}{3^{100}}\)

\item {} 
\(\displaystyle\frac{2^{90}}{3^{100}}\)

\item {} 
\(\displaystyle\frac{2}{3^{10}}\)

\end{enumerate}

\item (UFPR) Durante um surto de gripe, $25\%$ dos funcionários de uma empresa contraíram essa doença. Dentre os que contraíram gripe, $80\%$ apresentaram febre. Constatou-se também que $8\%$ dos funcionários apresentaram febre por outros motivos naquele período. Qual a probabilidade de que um funcionário dessa empresa, selecionado ao acaso, tenha apresentado febre durante o surto de gripe?
\begin{enumerate}
\item {} 
$20\%$

\item {} 
$26\%$

\item {} 
$28\%$

\item {} 
$33\%$

\item {} 
$35\%$

\end{enumerate}

\item (ENEM) Numa escola com $1200$ alunos foi realizada uma pesquisa sobre o conhecimento desses em duas línguas estrangeiras: inglês e espanhol. Nessa pesquisa, constatou-se que $600$ alunos falam inglês, $500$ falam espanhol e $300$ não falam qualquer um desses idiomas.

Escolhendo-se um aluno dessa escola ao acaso e sabendo-se que ele não fala inglês, qual a probabilidade de que esse aluno fale espanhol?
\begin{enumerate}
\item {} 
\(\frac{1}{2}\)

\item {} 
\(\frac{5}{8}\)

\item {} 
\(\frac{1}{4}\)

\item {} 
\(\frac{ 5}{6}\)

\item {} 
\(\frac{5}{14}\)

\end{enumerate}

\clearpage
\item (ENEM) O psicólogo de uma empresa aplica um teste para analisar a aptidão de um candidato a determinado cargo. O teste consiste em uma série de perguntas cujas respostas devem ser verdadeiro ou falso e termina quando o psicólogo fizer a décima pergunta ou quando o candidato der a segunda resposta errada, Com base em testes anteriores, o psicólogo sabe que a probabilidade de o candidato errar qualquer uma das respostas é $0{,}2$. A probabilidade do teste terminar na quinta pergunta é:
\begin{enumerate}
\item {} 
$0{,}02048$

\item {} 
$0{,}08192$

\item {} 
$0{,}24000$

\item {} 
$0{,}40960$

\item {} 
$0{,}49152$

\end{enumerate}

\item De um lote contendo $20$ lâmpadas das quais $5$ são defeituosas, deseja-se extrair duas lâmpadas e testá-las sequencialmente.
\begin{enumerate}
\item {} 
Se as lâmpadas são extraídas sem reposição ao lote, responda:
\begin{enumerate}[label=\roman*)]
\item {} 
Qual a probabilidade de se extrair duas defeituosas?

\item {} 
Qual a probabilidade de se extrair duas boas?

\item {} 
Qual a probabilidade de se extrair uma boa e uma defeituosa em qualquer ordem?

\end{enumerate}

\item {} 
Se as lâmpadas são extraídas com reposição ao lote, responda:

\begin{enumerate}[label=\roman*)]
\item {} 
Qual a probabilidade de se extrair duas defeituosas?

\item {} 
Qual a probabilidade de se extrair duas boas?

\item {} 
Qual a probabilidade de se extrair uma boa e uma defeituosa em qualquer ordem?

\end{enumerate}
\end{enumerate}
\end{enumerate}


\ifnum\aluno=1
\clearpage
\else
\notasfinais
\fi

\bibliographystyle{apalike-pt}
\bibliography{../Bibliografia/probabilidade1_bibliografia.bib}

\nocite{*}

% \chapterillustration{abertura-tales}{abertura-tales-professor}

\chapterwhat{São estudados o significado de um teorema
com sua hipótese e tese, o Teorema de Tales, sua recíproca, as projeções paralelas e aplicações.}

\chapterbecause{O capítulo contém uma ferramenta útil para resolver situações que envolvem retas paralelas. Além disso, o teorema de Tales será usado para para desenvolver o conceito de semelhança de triângulos, que é um dos instrumentos necessários para compreender o mundo real. A semelhança, por sua vez, permite a obtenção de diversas propriedades métricas das figuras geométricas, tanto planas como espaciais.}


\chapter{Teorema de Tales}


%%%% Página de créditos

% Autores
\autorum{Eduardo Wagner}
\autordois{Marcos Paulo}

% Revisores
\revisorum{Cydara Cavedon Ripoll}
\revisordois{Letícia Rangel}

\autordacapa{Sigmund}{Unsplash}{https://unsplash.com/photos/r9PeXDCJyEw}
\versao{1.0}

%Licença cc-by-sa
\ccbysa

% Link para versão digital:
\versaodigital{https://www.umlivroaberto.org/BookCloud/Volume_1/master/view/GE201.html}

\creditos


\mainmatter


\begin{apresentacao}{Introdução}


Durante muito tempo, a Geometria plana ficou totalmente relegada ao ensino fundamental e, consequentemente, os aspectos mais refinados da sua construção foram deixados de lado. Com a inserção de tópicos de Geometria plana pela Base Nacional Comum Curricular (BNCC) no Ensino Médio, devemos modificar nossa prática de ensino de geometria para nos adequar a  essa nova realidade.


Embora iniciemos o capítulo com uma situação concreta convidando o aluno a se familiarizar com o modelo geométrico dessa situação, temos como objetivo maior introduzir um pouco mais de rigor nos assuntos abordados neste capítulo.


\begin{habilities}{EM12MT01}

Compreender o teorema de Tales e aplicá-lo em demonstrações e na resolução de problemas, incluindo a divisão de segmentos em partes proporcionais.
\end{habilities}

Observamos que, no contexto do Ensino Médio, “Compreender o teorema de Tales” vai além da capacidade de reconhecer uma figura em que o teorema de Tales é aplicável. Esperamos que o aluno seja capaz de justificar outros teoremas a partir do teorema de Tales, como na atividadeda da seção “\DUrole{xref,std,std-ref}{sub-divisao-de-segmentos}”.

Algumas propriedades geométricas precisam estar compreendidas para que se possa demonstrar o teorema de Tales e, por isso, neste capítulo foram incluídas rápidas visitas aos temas “Retas Paralelas” e “Congruência de Triângulos”.

A demonstração inicial do teorema de Tales será feita no caso dos segmentos comensuráveis e, no final do capítulo aparecerá a demonstração geral.

Algumas recomendações:
\begin{itemize}
\item {} 
Alunos conhecem em níveis diferentes o Teorema de Tales. Procure explorar com perguntas para verificar o nível de profundidade da familiaridade de seus alunos.

\item {} 
Uma das atividades iniciais, ``\hyperref[nas-ruas]{Nas ruas de uma cidade}”, é solicitado que os alunos façam uma estimativa. Estimule estimativas sem cálculo para comparar com o resultado que posteriormente será encontrado.

\item {} 
No ensino Fundamental, a maioria das propriedades eram compreendidas através de observações das figuras. No ensino Médio entendemos que algumas demonstrações devem ser efetuadas para que o aluno perceba o aspecto construtivo da Matemática.

\item {} 
A seção Organizando as ideias: medidas de dispersão é tão importante como delicada, pois trata da metodologia de demonstração. Um dos aspectos fundamentais de separação de \textbf{Hipótese} e \textbf{Tese} deve ser incentivado. Convide, sempre que possível, seus alunos a reescrever teoremas com os quais eles já têm familiaridade usando a estrutura de “Se Hipótese, então Tese”.

\item {} 
Dentre os exercícios, é proposto um de demonstração do teorema das Bissetrizes, que decorre diretamente do teorema de Tales.

\end{itemize}

\end{apresentacao}

\cleardoublepage
\def\currentcolor{session1}


\begin{objectives}{Nas ruas de uma cidade}
{
Levar o estudante a:
\begin{itemize}
\item {} 
Reconhecer retas paralelas.

\item {} 
Estimar uma distância.

\item {} 
Praticar sua intuição

\end{itemize}
}{1}{1}
\end{objectives}
\begin{sugestions}{Nas ruas de uma cidade}
{
\begin{itemize}
\item {} 
Os valores mencionados no desenho são bem próximos dos reais.

\item {} 
No item \titem{a)}, esperamos apenas que o aluno perceba o paralelismo entre as ruas, nada mais.

\item {} 
No item b), sem cálculos, entendemos que visualmente, uma estimativa entre $180$ m e $190$ m é excelente.

\item {} 
É conveniente verificar se alguns alunos utilizaram regra de três, mostrando que já tinham trazido esse conceito do Ensino Fundamental.

\end{itemize}
}{1}{1}
\end{sugestions}
\begin{answer}{Nas ruas de uma cidade}
{
\begin{enumerate}
\item {} 
São paralelas.

\item {} 
Qualquer valor entre $180$ m e $190$ m.

\item {} 
Resposta pessoal.

\end{enumerate}
}{1}
\end{answer}
\clearmargin

\begin{objectives}{Recordando paralelas}
{
Levar o estudante a
\begin{itemize}
\item {} 
Reconhecer ângulos em paralelas cortadas por transversal.

\item {} 
Reconhecer os ângulos que são iguais nessa situação.

\item {} 
Lembrar um caso de congruência de triângulos.

\end{itemize}
}{1}{2}
\end{objectives}
\begin{sugestions}{Recordando paralelas}
{
\begin{itemize}
\item {} 
Neste texto, segmentos congruentes serão chamados de iguais. Da mesma forma, ângulos congruentes serão também chamados de iguais.

\item {} 
A relação de igualdade entre os ângulos \(a\), \(b\) e \(c\) é conhecida desde o Ensino Fundamental. No último exercício do capítulo o aluno vai usar esses fatos como argumento para demonstrar que certo triângulo é isósceles.

\item {} 
Se o aluno não conhecer os casos de congruência de triângulos, ela pode ser verificada de maneira informal a partir de sobreposição por dobradura ou recortando um desenho semelhante.

\item {} 
Para uma justificativa matemáticamente rigorosa, necessitamos dos critérios de congruência de triângulos que estarão no Organizando as ideias após a atividade.

\end{itemize}
}{1}{2}
\end{sugestions}
\begin{answer}{Recordando paralelas}
{
\begin{enumerate}
\item {} 
Alternos internos

\item {} 
São iguais

\item {} 
Correspondentes

\item {} 
São iguais.

\item {} 
Porque possuem os mesmos ângulos internos e têm o lado AC em comum.

\item {} 
São iguais. São iguais também

\end{enumerate}
}{1}
\end{answer}




\explore{}

O teorema de Tales é um resultado básico da geometria plana e vai permitir a exploração e dedução de outros resultados, também centrais, dessa matéria. O teorema de Tales (das retas paralelas) é conhecido desde a antiguidade. Tales de Mileto viveu no século 6 a.C., mas nenhum dos seus escritos sobreviveu. Tudo o que sabemos dele veio através de escritos de outros.

\begin{task}{Nas ruas de uma cidade}
\label{nas-ruas}



A figura a seguir mostra uma parte da cidade de Nova York. Nessa região, as ruas são designadas por números. Observe as ruas numeradas de 67 a 71, as Avenidas Columbus e Broadway.

\begin{figure}[H]
\centering

\noindent\includegraphics[width=200bp]{{Fig-tales-01}.png}
\end{figure}

A distância \(A\) entre duas esquinas consecutivas da Avenida Columbus é igual a \(70\) m e a distância \(B\) entre as respectivas esquinas correspondentes na Broadway é \(80\) m. A distância \(C\) entre as esquinas da Columbus com as ruas 69 e 71 é  \(161\) m, e a distância entre as esquinas correspondentes na Broadway é \(D\).
\begin{enumerate}
\item {} 
Você lembra quais são as possíveis posições relativas entre duas retas no plano? Que relação você vê entre as ruas de 67 a 71?

\item {} 
Visualmente, estime um valor para a distância \(D\).

\item {} 
Você consegue calcular a distância \(D\)?

\end{enumerate}
\end{task}


\begin{task}{Recordando paralelas}



Na figura a seguir estão representados dois pares de retas paralelas: o par de retas \(AD\) e \(BC\) e o par de retas \(AB\) e \(CD\).
\begin{center}\begin{tikzpicture}
legenda
\draw [shift={(-0.7404492822759008,2.0323275076976457)},line width=0.8pt,color=session2,fill=session2,fill opacity=0.10000000149011612] (0,0) -- (-158.99677728856574:0.40327274248797285) arc (-158.99677728856574:-103.61977057805956:0.40327274248797285) -- cycle;
\draw [shift={(-1.2241649681252438,0.03590094973762964)},line width=0.8pt,color=session2,fill=session2,fill opacity=0.10000000149011612] (0,0) -- (21.003222711434265:0.40327274248797285) arc (21.003222711434265:76.38022942194044:0.40327274248797285) -- cycle;
\draw [shift={(2.5915772583373085,1.500873411683074)},line width=0.8pt,color=session2,fill=session2,fill opacity=0.10000000149011612] (0,0) -- (21.003222711434265:0.40327274248797285) arc (21.003222711434265:76.38022942194043:0.40327274248797285) -- cycle;
\draw [line width=0.8pt] (-2.62,-0.5)-- (4.1,2.08);
\draw [line width=0.8pt] (-2.7,1.28)-- (4.0347849649135,3.8656763704578623);
\draw [line width=0.8pt] (-1.48,-1.02)-- (-0.38,3.52);
\draw [line width=0.8pt] (2.32,0.38)-- (3.313758477838472,4.481512263078785);
\draw [line width=0.8pt] (-0.7404492822759008,2.0323275076976457)-- (2.5915772583373085,1.500873411683074);
\draw (-1.2,2.6) node[anchor=north west] {A};
\draw (-1.3,0) node[anchor=north west] {B};
\draw (2.5,1.4) node[anchor=north west] {C};
\draw (2.5,4) node[anchor=north west] {D};
\draw (-1.4,1.7) node[anchor=north west] {$ a $};
\draw (-1,.9) node[anchor=north west] {$ b $};
\draw (2.8,2.3) node[anchor=north west] {$ c $};
\draw [line width=2.pt,color=session3] (-1.2241649681252438,0.03590094973762964)-- (-0.7404492822759008,2.0323275076976457);
\draw [line width=2.pt,color=session3] (2.5915772583373085,1.500873411683074)-- (3.075292944186652,3.4972999696430898);
\begin{scriptsize}
\draw [fill=black] (-0.7404492822759008,2.0323275076976457) circle (1.0pt);
\draw [fill=black] (-1.2241649681252438,0.03590094973762964) circle (1.0pt);
\draw [fill=black] (2.5915772583373085,1.500873411683074) circle (1.0pt);
\draw [fill=black] (3.075292944186652,3.4972999696430898) circle (1.0pt);
\end{scriptsize}
\end{tikzpicture}\end{center}\begin{enumerate}
\item {} 
Você conhece o nome que se dá ao par de ângulos \(a\) e \(b\)?

\item {} 
Que relação há entre os ângulos \(a\) e \(b\)?

\item {} 
Você conhece o nome que se dá ao par de ângulos \(b\) e \(c\)?

\item {} 
Que relação há entre os ângulos \(b\) e \(c\)?

\item {} 
Pode-se afirmar que os triângulos \(ABC\) e \(CDA\) são congruentes. Por quê?

\item {} 
Que relação existe entre os segmentos \(AB\) e \(CD\)? E com os segmentos \(AD\) e \(BC\)?

\end{enumerate}
\end{task}


\clearpage

\arrange{}
\begin{texto}
{\def\currentcolor{session4}
\begin{observation}
Vale lembrar que serão utilizados conceitos do Ensino Fundamental. Em alguns casos, será necessário que se recorde tais conceitos para poder avançar no capítulo.
\begin{itemize}
\item {} 
Ângulos nas paralelas cortadas por uma transversal;

\item {} 
A definição de paralelogramo (Quadrilátero que possui dois pares de lados paralelos).

\end{itemize}
\end{observation}
}
\end{texto}
\clearmargin
\begin{objectives}{Demonstrando uma afirmação}
{
Levar o estudante a
\begin{itemize}
\item {} 
Demonstrar um resultado.

\item {} 
Identificar, em uma proposição, as hipóteses e o que se que demonstrar.

\item {} 
Planejar a sequência de argumentos para concluir o resultado.

\end{itemize}

}{1}{2}
\end{objectives}
\begin{sugestions}{Demonstrando uma afirmação}
{
\begin{itemize}
\item {} 
Essa proposição é um teorema, mas ainda não estamos dando esse título, pois não é o objetivo no momento.

\item {} 
Para demonstrar a proposição é necessário interferir na figura, traçando novos segmentos que vão permitir o aparecimento de triângulos congruentes. Nessa primeira atividade de demonstração, daremos as dicas para que o aluno consiga percorrer o caminho até o final.

\end{itemize}
}{1}{2}
\end{sugestions}
\begin{answer}{Demonstrando uma afirmação}
{
\begin{enumerate}
\item {} 
\(DE = EF\) (caso tivessesmo escolhido considerar inicialmente \(DE=EF\), a resposta seria \(AB=BC\)).

\item {} 
Os triângulos \(DGE\) e \(EHF\) são congruentes pelo caso \textbf{ALA}.

De fato, \(ABGD\) e \(BCHE\) são paralelogramos. Daí, \(DG = AB = BC = EH\).

Além disso, os dois triângulos possuem mesmos ângulos pois \(DG\) e \(EH\) são paralelos, da mesma forma que \(GE\) e \(HF\) são também paralelos.

\item {} 
Dessa congruência conclui-se que \(DE = EF\) que queríamos demonstrar

\end{enumerate}
}{1}
\end{answer}
\clearmargin
\def\currentcolor{session2}
\begin{objectives}{Dividindo um segmento em partes iguais}
{
Levar o estudante a
\begin{itemize}
\item {} 
Perceber que nossa visão é limitada e nossos instrumentos de medida são limitados e imperfeitos.

\item {} 
Aprender que construções geométricas não dependem de medidas.

\item {} 
Perceber que construções geométricas são procedimentos abstratos, portanto mostram resultados exatos, coisa que nossos sentidos não permitem.

\item {} 
Executar concretamente uma aplicação de algo que ele mesmo demonstrou.

\end{itemize}
}{1}{1}
\end{objectives}
\begin{sugestions}{Dividindo um segmento em partes iguais}
{
\begin{itemize}
\item {} 
O processo de medição, na prática, é sempre aproximado. Uma experiência interessante é desenhar um segmento no papel e pedir para vários alunos usarem a régua para medir o comprimento desse segmento. Depois, peça que eles multipliquem a medida por um número grande (isso amplia qualquer erro) e compare os resultados.

\item {} 
As construções geométricas são processos rigorosos matematicamente, capazes de produzir medidas exatas. Quando as executamos concretamente com os instumentos de desenho, certamente também cometeremos erros, mas estes serão provavelmente menores do que os outros que foram feitos através de medidas. Por exemplo, se um segmento de 9cm for dividido em 7 partes iguais, cada parte medirá $1{,}285714\dots$ cm, medida impossível de ser construída com uma régua comum. Entretanto, com uma construção geométrica, obteremos uma divisão bastante boa.

\end{itemize}
}{1}{1}
\end{sugestions}
\begin{answer}{Dividindo um segmento em partes iguais}
{
\begin{enumerate}
\item {} 
$2{,}3666666$.

\item {} 
Não, pois a régua não permite marcar medidas menores que $1$ mm.
\end{enumerate}
}{1}
\end{answer}
\clearmargin
\begin{answer}{Dividindo um segmento em partes iguais}
{
\begin{enumerate}
\item {} 
Porque quando paralelas são cortadas por transversais se, em uma delas os segmentos são iguais \((AM = MN = NP)\) então sobre a outra os segmentos correspondentes serão também iguais \((AC = CD = DB)\).

\item {} 
Não, pois o procedimento descrito nesta atividade não depende de medições.
\end{enumerate}
}{1}
\end{answer}
\def\currentcolor{session4}

Vamos lembrar que duas figuras são congruentes, quando podem ser levadas a coincidir por superposição mediante o deslocamento rígido de uma delas.

Na atividade anterior os triângulos \(ABC\) e \(CDA\) são congruentes.

Mas afinal, quais as condições mínimas para garantir que dois triângulos são congruentes? Não podemos nos deixar levar somente pelo que as imagens sugerem.

Por exemplo, os dois triângulos da figura a seguir são congruentes?
\begin{center}\begin{tikzpicture}
\draw [shift={(-3.12,2.96)},line width=0.8pt,color=session2,fill=session2,fill opacity=0.10000000149011612] (0,0) -- (-13.706961004079805:0.40139099339564077) arc (-13.706961004079805:56.29303899592019:0.40139099339564077) -- cycle;
\draw [shift={(3.2665068298598277,4.71632784582858)},line width=0.8pt,color=session2,fill=session2,fill opacity=0.10000000149011612] (0,0) -- (167.21274687833522:0.40139099339564077) arc (167.21274687833522:238.21274687833528:0.40139099339564077) -- cycle;
\draw [line width=0.8pt] (-3.12,2.96)-- (0.16,2.16);
\draw [line width=0.8pt] (-3.12,2.96)-- (-1.8209561349802097,4.907321536218554);
\draw [line width=0.8pt] (-1.8209561349802097,4.907321536218554)-- (0.16,2.16);
\draw [line width=0.8pt] (1.4880625549427764,1.8465654931215187)-- (3.2665068298598277,4.71632784582858);
\draw [line width=0.8pt] (3.2665068298598277,4.71632784582858)-- (0.9837136824628329,5.234431672164059);
\draw [line width=0.8pt] (0.9837136824628329,5.234431672164059)-- (1.4880625549427764,1.8465654931215187);
\draw (-3,4.5) node[anchor=north west] {$5$};
\draw (2,5.5) node[anchor=north west] {$5$};
\draw (-2,2.401245365278648) node[anchor=north west] {$7$};
\draw (0.7,3.7) node[anchor=north west] {$7$};
\draw (-2.7,3.4) node[anchor=north west] {$70^{\circ}$};
\draw (2.1,4.7) node[anchor=north west] {$70^{\circ}$};
\draw [fill=black] (-3.12,2.96) circle (1.0pt);
\draw [fill=black] (0.16,2.16) circle (1.0pt);
\draw [fill=black] (-1.8209561349802097,4.907321536218554) circle (1.0pt);
\draw [fill=black] (1.4880625549427764,1.8465654931215187) circle (1.0pt);
\draw [fill=black] (3.2665068298598277,4.71632784582858) circle (1.0pt);
\draw [fill=black] (0.9837136824628329,5.234431672164059) circle (1.0pt);
\end{tikzpicture}\end{center}
A resposta é não e a justificativa necessita de material que será desenvolvido no capítulo de trigonometria. Os dois triângulos da figura acima parecem, mas não são congruentes.

Os casos básicos que garantem a congruência de dois triângulos são:
\begin{enumerate}
\item {} 
Caso lado-lado-lado

\item {} 
Caso lado-ângulo-lado

\item {} 
Caso ângulo-lado-ângulo

\end{enumerate}
\begin{center}\begin{tikzpicture}
\draw [shift={(0.7,3.54)},line width=0.8pt,color=session2,fill=session2,fill opacity=0.10000000149011612] (0,0) -- (-20.772254682045826:0.4) arc (-20.772254682045826:6.4108400202324525:0.4) -- cycle;
\draw [shift={(1.82,1.82)},line width=0.8pt,color=session2,fill=session2,fill opacity=0.10000000149011612] (0,0) -- (-20.77225468204584:0.4) arc (-20.77225468204584:6.410840020232449:0.4) -- cycle;
\draw [shift={(5.74,2.52)},line width=0.8pt,color=session2,fill=session2,fill opacity=0.10000000149011612] (0,0) -- (52.073537674961365:0.4) arc (52.073537674961365:99.9720576873311:0.4) -- cycle;
\draw [shift={(7.16,1.1)},line width=0.8pt,color=session2,fill=session2,fill opacity=0.10000000149011612] (0,0) -- (52.073537674961386:0.4) arc (52.073537674961386:99.9720576873311:0.4) -- cycle;
\draw [shift={(6.94,4.06)},line width=0.8pt,color=session2,fill=session2,fill opacity=0.10000000149011612] (0,0) -- (169.56252464888183:0.4) arc (169.56252464888183:232.07353767496136:0.4) -- cycle;
\draw [shift={(8.36,2.64)},line width=0.8pt,color=session2,fill=session2,fill opacity=0.10000000149011612] (0,0) -- (169.56252464888183:0.4) arc (169.56252464888183:232.07353767496136:0.4) -- cycle;
\draw [line width=0.8pt] (-3.22,2.66)-- (-1.12,2.66);
\draw [line width=0.8pt] (-3.22,2.66)-- (-2.62,3.86);
\draw [line width=0.8pt] (-2.62,3.86)-- (-1.12,2.66);
\draw [line width=0.8pt] (-2.3,0.84)-- (-0.2,0.84);
\draw [line width=0.8pt] (-2.3,0.84)-- (-1.7,2.04);
\draw [line width=0.8pt] (-1.7,2.04)-- (-0.2,0.84);
\draw [line width=0.8pt] (0.7,3.54)-- (3.02,2.66);
\draw [line width=0.8pt] (3.02,2.66)-- (2.48,3.74);
\draw [line width=0.8pt] (2.48,3.74)-- (0.7,3.54);
\draw [line width=0.8pt] (1.82,1.82)-- (4.14,0.94);
\draw [line width=0.8pt] (4.14,0.94)-- (3.6,2.02);
\draw [line width=0.8pt] (3.6,2.02)-- (1.82,1.82);
\draw [line width=0.8pt] (5.74,2.52)-- (6.94,4.06);
\draw [line width=0.8pt] (6.94,4.06)-- (5.42,4.34);
\draw [line width=0.8pt] (5.42,4.34)-- (5.74,2.52);
\draw [line width=0.8pt] (7.16,1.1)-- (8.36,2.64);
\draw [line width=0.8pt] (8.36,2.64)-- (6.84,2.92);
\draw [line width=0.8pt] (6.84,2.92)-- (7.16,1.1);
\draw (-2.4,2.7) node[anchor=north west] {$ a $};
\draw (-1.4,0.8) node[anchor=north west] {$ a $};
\draw (1.4,4.1) node[anchor=north west] {$ a $};
\draw (2.4,2.4) node[anchor=north west] {$ a $};
\draw (6.35,3.4) node[anchor=north west] {$ a $};
\draw (7.7,2.) node[anchor=north west] {$ a $};
\draw (-1.9,3.9) node[anchor=north west] {$b$};
\draw (-1,2) node[anchor=north west] {$b$};
\draw (1.6,3.1) node[anchor=north west] {$b$};
\draw (2.7,1.4) node[anchor=north west] {$b$};
\draw (-3.3,3.6) node[anchor=north west] {$ c $};
\draw (-2.32,1.8) node[anchor=north west] {$ c $};
\draw (1.1,3.7) node[anchor=north west] {$\alpha$};
\draw (2.2,2) node[anchor=north west] {$\alpha$};
\draw (5.6,3.4) node[anchor=north west] {$\alpha$};
\draw (7.,2) node[anchor=north west] {$\alpha$};
\draw (6.1,4.1) node[anchor=north west] {$ \beta $};
\draw (7.5,2.7) node[anchor=north west] {$ \beta $};
\draw (-3,5.08) node[anchor=north west] {Caso $LLL$};
\draw (1.,5.08) node[anchor=north west] {Caso $LAL$};
\draw (5,5.06) node[anchor=north west] {Caso $ALA$};
\draw [fill=black] (-3.22,2.66) circle (1.0pt);
\draw [fill=black] (-1.12,2.66) circle (1.0pt);
\draw [fill=black] (-2.62,3.86) circle (1.0pt);
\draw [fill=black] (-2.3,0.84) circle (1.0pt);
\draw [fill=black] (-0.2,0.84) circle (1.0pt);
\draw [fill=black] (-1.7,2.04) circle (1.0pt);
\draw [fill=black] (0.7,3.54) circle (1.0pt);
\draw [fill=black] (3.02,2.66) circle (1.0pt);
\draw [fill=black] (2.48,3.74) circle (1.0pt);
\draw [fill=black] (1.82,1.82) circle (1.0pt);
\draw [fill=black] (4.14,0.94) circle (1.0pt);
\draw [fill=black] (3.6,2.02) circle (1.0pt);
\draw [fill=black] (5.74,2.52) circle (1.0pt);
\draw [fill=black] (6.94,4.06) circle (1.0pt);
\draw [fill=black] (5.42,4.34) circle (1.0pt);
\draw [fill=black] (7.16,1.1) circle (1.0pt);
\draw [fill=black] (8.36,2.64) circle (1.0pt);
\draw [fill=black] (6.84,2.92) circle (1.0pt);
\end{tikzpicture}\end{center}
Com os critérios de congruência em mãos, podemos agora justificar por que os triângulos \(ABC\) e \(CDA\) da atividade anterior são congruentes e, com isso, concluir as igualdades dos pares de segmentos do item f).

Veja novamente a figura, agora simplificada e com os outros elementos que vamos necessitar.
\begin{center}\begin{tikzpicture}
\draw [shift={(-1.58,4.08)},line width=0.8pt,color=session2,fill=session2,fill opacity=0.10000000149011612] (0,0) -- (-11.245482805462865:0.5454545454545459) arc (-11.245482805462865:15.708637829015746:0.5454545454545459) -- cycle;
\draw [shift={(1.84,3.4)},line width=0.8pt,color=session2,fill=session2,fill opacity=0.10000000149011612] (0,0) -- (168.75451719453713:0.5454545454545459) arc (168.75451719453713:195.70863782901574:0.5454545454545459) -- cycle;
\draw [shift={(-1.58,4.08)},line width=0.8pt,color=session1,fill=session1,fill opacity=0.10000000149011612] (0,0) -- (-103.42183506788622:0.3636363636363639) arc (-103.42183506788622:-11.245482805462876:0.3636363636363639) -- cycle;
\draw [shift={(1.84,3.4)},line width=0.8pt,color=session1,fill=session1,fill opacity=0.10000000149011612] (0,0) -- (76.57816493211381:0.3636363636363639) arc (76.57816493211381:168.75451719453713:0.3636363636363639) -- cycle;
\draw [line width=0.8pt] (-2.,2.32)-- (1.84,3.4);
\draw [line width=0.8pt] (-1.58,4.08)-- (-2.,2.32);
\draw [line width=0.8pt] (-1.58,4.08)-- (2.26,5.16);
\draw [line width=0.8pt] (2.26,5.16)-- (1.84,3.4);
\draw [line width=0.8pt] (-1.58,4.08)-- (1.84,3.4);
\draw [shift={(-1.58,4.08)},line width=0.8pt,color=session1] (-103.42183506788622:0.3636363636363639) arc (-103.42183506788622:-11.245482805462876:0.3636363636363639);
\draw [shift={(-1.58,4.08)},line width=0.8pt,color=session1] (-103.42183506788622:0.27272727272727293) arc (-103.42183506788622:-11.245482805462876:0.27272727272727293);
\draw [shift={(1.84,3.4)},line width=0.8pt,color=session1] (76.57816493211381:0.3636363636363639) arc (76.57816493211381:168.75451719453713:0.3636363636363639);
\draw [shift={(1.84,3.4)},line width=0.8pt,color=session1] (76.57816493211381:0.27272727272727293) arc (76.57816493211381:168.75451719453713:0.27272727272727293);
\draw (-0.9,4.3) node[anchor=north west] {$ x $};
\draw (0.7,3.6) node[anchor=north west] {$ x $};
\draw (-1.5,3.9) node[anchor=north west] {$ y $};
\draw (1.4,4.3) node[anchor=north west] {$ y $};
\draw [fill=black] (-2.,2.32) circle (1.0pt);
\draw[color=black] (-2.064727272727273,2.0665454545454542) node {$B$};
\draw [fill=black] (1.84,3.4) circle (1.0pt);
\draw[color=black] (2.1,3.2) node {$C$};
\draw [fill=black] (-1.58,4.08) circle (1.0pt);
\draw[color=black] (-1.6465454545454545,4.412) node {$A$};
\draw [fill=black] (2.26,5.16) circle (1.0pt);
\draw[color=black] (2.2807272727272756,5.466545454545455) node {$D$};
\end{tikzpicture}\end{center}
Os ângulos marcados com \(x\) são alternos internos nas paralelas \(AD\) e \(BC\), cortadas pela transversal \(AC\).

Os ângulos marcados com \(y\) são alternos internos nas paralelas \(AB\) e \(DC\), cortadas pela transversal \(AC\).

Assim, os triângulos \(ABC\) e \(CDA\) são congruentes pelo caso $\bm{ALA}$.

Dessa forma, temos \(AB = CD\) e \(BC = DA\).

\begin{observationtitle}{Importante}

Com os argumentos que acabamos de mostrar, concluímos uma importante propriedade:

\textit{“Em um paralelogramo, os lados opostos são iguais.”}
\end{observationtitle}
\begin{task}{Demonstrando uma afirmação}
\label{demonstrando-afirmacao}



As paralelas \(r_1\), \(r_2\) e \(r_3\) estão intersectadas pelas transversais \(t_1\) e \(t_2\) . Demonstre que:

\textit{“Se as paralelas determinam sobre uma transversal segmentos iguais então determinarão, na outra transversal, segmentos também iguais.”}

Veja a figura:
\begin{center}\begin{tikzpicture}
\draw [line width=0.8pt] (-2.7,0.)-- (4.6,0.);
\draw [line width=0.8pt] (-2.7,2.68)-- (4.56,2.68);
\draw [line width=0.8pt] (-2.7,1.34)-- (4.58,1.34);
\draw [line width=0.8pt] (-1.54,3.3)-- (-2.2,-0.74);
\draw [line width=0.8pt] (-0.16,3.32)-- (4.66,-0.7);
\draw (-2.1,3.2) node[anchor=north west] {$A$};
\draw (-2.3,1.9) node[anchor=north west] {$B$};
\draw (-2.6,0.5) node[anchor=north west] {$C$};
\draw (0.558,3.2) node[anchor=north west] {$D$};
\draw (2.2,1.9) node[anchor=north west] {$E$};
\draw (3.9,0.5) node[anchor=north west] {$F$};
\draw (-3.3,2.9) node[anchor=north west] {$r_1$};
\draw (-3.3,1.6) node[anchor=north west] {$r_2$};
\draw (-3.3,0.3) node[anchor=north west] {$r_3$};
\draw (-2.4,-0.8) node[anchor=north west] {$t_1$};
\draw (4.76,-0.8) node[anchor=north west] {$t_2$};
\draw [fill=black] (-1.6412871287128707,2.68) circle (1.0pt);
\draw [fill=black] (-1.86019801980198,1.34) circle (1.0pt);
\draw [fill=black] (-2.079108910891089,0.) circle (1.0pt);
\draw [fill=black] (0.6073631840796017,2.68) circle (1.0pt);
\draw [fill=black] (2.214029850746268,1.34) circle (1.0pt);
\draw [fill=black] (3.8206965174129355,0.) circle (1.0pt);
\end{tikzpicture}\end{center}
O enunciado pede que você demonstre uma afirmação. Em Matemática, demonstrar significa justificar a afirmação utilizando argumentos lógicos baseados em fatos conhecidos anteriormente.

Faremos, a seguir, perguntas que são dicas para ajudá-lo a demonstrar essa proposição.

Vamos considerar, na figura dada, \(AB=BC\). Não haveria diferença, caso escolhêssemos considerar \(DE=EF\) na outra trasnversal. Com os elementos da figura acima, responda:
\begin{enumerate}
\item {} 
O que se deseja demonstrar?

Para conseguir os argumentos necessários você vai ter que interferir na figura. Faça o seguinte:
\begin{itemize}
\item {} 
Trace por \(D\) uma paralela a \(t_1\) que intersecta \(r_2\) no ponto \(G\).

\item {} 
Trace por \(E\) uma paralela a \(t_1\) que intersecta \(r_3\) no ponto \(H\).

\end{itemize}

\item {} 
Os triângulos \(DGE\) e \(EHF\) são congruentes? Por quê?

\item {} 
O que se conclui da congruência dos triângulos \(DGE\) e \(EHF\)?

\end{enumerate}
\end{task}




\practice{}

\begin{task}{dividindo um segmento em partes iguais}

Um segmento desenhado no papel precisa ser dividido em três partes iguais. Um aluno fez assim:

Com uma régua mediu seu comprimento encontrando \(7{,}1\) cm.

Com a calculadora dividiu essa medida por $3$.

Ele pretende usar a régua para aplicar sobre o segmento a medida que aparece na calculadora.
\begin{enumerate}
\item {} 
Que número o visor da calculadora mostrou quando o segmento dado foi dividido por $3$?

\item {} 
Você consegue, com a régua escolar, marcar sobre o segmento a medida que a calculadora mostrou?

\end{enumerate}

Você percebe aí uma dificuldade, não é mesmo? Nossos sentidos são limitados e a régua não marca com precisão medidas menores que $1$ milímetro. Como fazer então?

Buscamos  inspiração na atividade anterior. Considere a seguinte construção:

Nosso segmento chama-se \(AB\).
\begin{itemize}
\item {} 
A partir de \(A\) trace uma semirreta \(AX\) (que não contenha o segmento \(AB\)).

\item {} 
Com o compasso em uma abertura qualquer fixada, assinale, a partir de \(A\), três segmentos iguais. Chamaremos esses pontos sobre a semirreta \(AX\) de \(M\), \(N\) e \(P\). Veja a figura.

\end{itemize}
\begin{center}\begin{tikzpicture}
\draw [line width=0.8pt] (0.,0.)-- (4.5,0.);
\draw [line width=0.8pt] (0.,0.)-- (6.38,-3.18);
\draw (1.46,-1.04) node[anchor=north west] {$M$};
\draw (3.26,-1.88) node[anchor=north west] {$N$};
\draw (5.08,-2.8) node[anchor=north west] {$P$};
\draw (6.5,-2.96) node[anchor=north west] {$X$};
\draw [line width=0.8pt] (0.,0.)-- (1.7450813116921962,-0.8698054186804365);
\draw [line width=0.8pt] (0.8813642797528934,-0.33874063957697237) -- (0.8010678993991632,-0.4998384089659029);
\draw [line width=0.8pt] (0.9440134122930329,-0.369967009714534) -- (0.8637170319393028,-0.5310647791034646);
\draw [line width=0.8pt] (1.7450813116921962,-0.8698054186804365)-- (3.4901626233843923,-1.739610837360873);
\draw [line width=0.8pt] (2.6264455914450897,-1.2085460582574088) -- (2.546149211091359,-1.3696438276463392);
\draw [line width=0.8pt] (2.6890947239852294,-1.2397724283949703) -- (2.608798343631499,-1.4008701977839009);
\draw [line width=0.8pt] (3.4901626233843923,-1.739610837360873)-- (5.235243935076588,-2.6094162560413094);
\draw [line width=0.8pt] (4.371526903137285,-2.078351476937845) -- (4.291230522783555,-2.239449246326776);
\draw [line width=0.8pt] (4.4341760356774245,-2.1095778470754065) -- (4.353879655323694,-2.2706756164643376);
\draw [fill=black] (0.,0.) circle (1.0pt);
\draw[color=black] (-0.06,0.31) node {$A$};
\draw [fill=black] (4.5,0.) circle (1.0pt);
\draw[color=black] (4.46,0.33) node {$B$};
\draw [fill=black] (1.7450813116921962,-0.8698054186804365) circle (2.0pt);
\draw [fill=black] (3.4901626233843923,-1.739610837360873) circle (2.0pt);
\draw [fill=black] (5.235243935076588,-2.6094162560413094) circle (2.0pt);
\end{tikzpicture}\end{center}
Em seguida, trace a reta \(PB\) e trace paralelas a ela pelos pontos \(M\) e \(N\). Essas paralelas intersectarão o segmento \(AB\) nos pontos \(C\) e \(D\), como mostra a figura.
\begin{center}\begin{tikzpicture}
\draw [line width=0.8pt] (0.,0.)-- (4.5,0.);
\draw [line width=0.8pt] (0.,0.)-- (6.38,-3.18);
\draw (1.46,-1.04) node[anchor=north west] {$M$};
\draw (3.26,-1.88) node[anchor=north west] {$N$};
\draw (5.08,-2.8) node[anchor=north west] {$P$};
\draw (6.5,-2.96) node[anchor=north west] {$X$};
\draw [line width=0.8pt] (0.,0.)-- (1.7450813116921962,-0.8698054186804365);
\draw [line width=0.8pt] (0.8813642797528934,-0.33874063957697215) -- (0.8010678993991632,-0.49983840896590276);
\draw [line width=0.8pt] (0.9440134122930329,-0.3699670097145338) -- (0.8637170319393028,-0.5310647791034644);
\draw [line width=0.8pt] (1.7450813116921962,-0.8698054186804365)-- (3.4901626233843923,-1.739610837360873);
\draw [line width=0.8pt] (2.6264455914450897,-1.2085460582574086) -- (2.546149211091359,- 1.3696438276463392);
\draw [line width=0.8pt] (2.6890947239852294,-1.2397724283949703) -- (2.608798343631499,-1.4008701977839009);
\draw [line width=0.8pt] (3.4901626233843923,-1.739610837360873)-- (5.235243935076588,-2.6094162560413094);
\draw [line width=0.8pt] (4.371526903137285,-2.078351476937845) -- (4.291230522783555,-2.239449246326776);
\draw [line width=0.8pt] (4.4341760356774245,-2.1095778470754065) -- (4.353879655323694,-2.2706756164643376);
\draw [line width=0.8pt,dash pattern=on 1pt off 1pt,color=session1] (4.5,0.)--  (5.235243935076588,-2.6094162560413094);
\draw [line width=0.8pt,dash pattern=on 1pt off 1pt,color=session1] (1.7450813116921962,-0.8698054186804365)-- (1.5,0.);
\draw [line width=0.8pt,dash pattern=on 1pt off 1pt,color=session1] (3.4901626233843923,-1.739610837360873)-- (3.,0.);
\draw [color=session3](1.34,0.6) node[anchor=north west] {C};
\draw [color=session3](2.88,0.6) node[anchor=north west] {D};
\draw [fill=black] (0.,0.) circle (1.0pt);
\draw[color=black] (-0.06,0.31) node {$A$};
\draw [fill=black] (4.5,0.) circle (1.0pt);
\draw[color=black] (4.46,0.33) node {$B$};
\draw [fill=black] (1.7450813116921962,-0.8698054186804365) circle (1.0pt);
\draw [fill=black] (3.4901626233843923,-1.739610837360873) circle (1.0pt);
\draw [fill=black] (5.235243935076588,-2.6094162560413094) circle (1.0pt);
\draw [fill=session3] (1.5,0.) circle (1.5pt);
\draw [fill=session3] (3.,0.) circle (1.5pt);
\end{tikzpicture}
\end{center}

\begin{enumerate}\setcounter{enumi}{2}
\item {} 
Com esse procedimento, explique por que os pontos \(C\) e \(D\) dividem o segmento \(AB\) em três partes iguais.

\item {} 
Para dividir um segmento em partes iguais há necessidade de fazer medidas?

\end{enumerate}
\end{task}


\clearpage
\arrange{o que é um teorema?}

\clearmargin
\begin{objectives}{Segmentos comensuráveis}
{
Levar o estudante a
\begin{itemize}
\item {} 
Identificar uma unidade comum a dois segmentos cujas medidas são racionais.

\end{itemize}
}{1}{2}
\end{objectives}
\begin{sugestions}{Segmentos comensuráveis}
{
\begin{itemize}
\item {} 
Quando as medidas dos dois segmentos são números racionais apresentados com representações decimais, os alunos não terão o menor problema em obter uma unidade comum para esses segmentos.

\item {} 
Quando as medidas são números racionais dados por meio da representação fracionária, a dica é obter frações equivalentes com o mesmo denominador. A unidade comum fica óbvia.

\item {} 
É necessário comentar que dois segmentos nem sempre admitem uma unidade comum. Por exemplo, se um segmento mede 1 unidade e outro mede \(\sqrt{2}\) unidades, não existe uma unidade comum. Neste capítulo vamos abordar, inicialmente, a demonstração do teorema de Tales no caso dos segmentos comensuráveis. A demonstração geral aparecerá no final do capítulo..

\end{itemize}
}{1}{2}
\end{sugestions}
\begin{answer}{Segmentos comensuráveis}
{
As respostas são pessoais. Daremos a menor unidade para cada um dos casos.
\begin{enumerate}
\item {} 
0,1

\item {} 
0,01

\item {} 
0,001

\item {} 
0,0001

\item {} 
1/30

\end{enumerate}
}{1}
\end{answer}
\begin{objectives}{Compreendendo o teorema de Tales}
{
\begin{itemize}
\item {} 
Compreender o enunciado do teorema de Tales identificando   a hipótese e a tese

\end{itemize}
}{1}{2}
\end{objectives}
\begin{sugestions}{Compreendendo o teorema de Tales}
{
\begin{itemize}
\item {} 
É necessário rever o que é uma proporcionalidade e o que significa dizer que  segmentos dados sejam proporcionais a outros também dados.

\item {} 
Exemplos devem ser dados. Se o professor disser que, sobre uma das transversais, um segmento é o dobro do outro, os alunos deverão concluir que, na outra trasnversal, os segmentos correspondentes serão um o dobro do outro.

\end{itemize}
}{1}{1}
\end{sugestions}
\begin{answer}{Compreendendo o teorema de Tales}
{
\begin{enumerate}
\item {} 
As retas paralelas são cortadas por transversais.

\item {} 
\(\dfrac{a}{a'}=\dfrac{b}{b'}=\dfrac{c}{c'}\)

\end{enumerate}
}{1}
\end{answer}
\begin{objectives}{Demonstrando o teorema de Tales}
{
\begin{itemize}
\item {} 
Demonstrar o teorema de Tales no caso dos segmentos comensuráveis.

\end{itemize}
}{1}{1}
\end{objectives}
\begin{sugestions}{Demonstrando o teorema de Tales}
{
\begin{itemize}
\item {} 
O aluno fará a demonstração do teorema de Tales no caso em que os dois segmentos da primeira transversal são comensuráveis.

\end{itemize}

O texto dirá que o resultado vale quando as medidas dos segmentos são números reais quaisquer. Uma demonstração geral do teorema de Tales usando áreas estará no Para Saber Mais, no final do capítulo.
}{1}{1}
\end{sugestions}
\clearmargin
\begin{answer}{Demonstrando o teorema de Tales}
{
\begin{enumerate}
\item {} 
\(m\)

\item {} 
\(n\)

\item {} 
\(a'= mv\) e \(b'=nv\)

Tomando a razão entre os elementos do lado esquerdo e os do lado direito, obtemos \(\dfrac{a}{a'}=\dfrac{mu}{nu}=\dfrac{m}{n}\) e que \(\dfrac{b}{b'}=\dfrac{mv}{nv}=\dfrac{m}{n}\), logo \(\dfrac{a}{a'}=\dfrac{b}{b'}\)

\end{enumerate}
}{1}
\end{answer}




Na atividade anterior você fez uma demonstração. Havia três retas paralelas intersectadas por duas transversais e você considerou, naquela situação que  \(AB = BC\). Assim, o fato a ser demonstrado era que \(DE = EF\).

Um \textbf{TEOREMA} é uma afirmação matemática que precisa de uma justificativa para ser aceita.

Em um teorema, a situação e os fatos que são dados constituem a \textbf{HIPÓTESE}. O fato que que se quer demonstrar é a \textbf{TESE}.

A partir da hipótese e para concluir a tese há um caminho composto por argumentos sucessivos, onde cada um é consequência dos anteriores. Esse caminho é a \textbf{DEMONSTRAÇÃO}.

Uma forma comum de enunciar um teorema é:

Se  \textbf{Hipótese},  então \textbf{Tese}.

\begin{example}{Teorema de Pitágoras}
Como exemplo, o famoso teorema de Pitágoras diz que:

“Se \textbf{um triângulo é retângulo} então \textbf{o quadrado da hipotenusa é igual à soma dos quadrados dos catetos}”.

Podemos separar:

\textbf{Hipótese}: um triângulo é retângulo

\textbf{Tese}: o quadrado da hipotenusa é igual à soma dos quadrados dos catetos
\end{example}

O objetivo deste capítulo é compreender o “teorema de Tales”. Esse é um teorema muito antigo e importante, pois com ele, diversas outras propriedades de figuras da geometria foram demonstradas. O enunciado do teorema que vamos apresentar a seguir inclui a palavra “feixe”. Entenda essa palavra como “conjunto”.

Razão entre segmentos

É preciso explicar certos termos que usamos em geometria. Eles são muito antigos, mas são usados hoje e serão sempre.

A razão entre dois segmentos é a razão entre suas medidas.

Por exemplo, se um segmento \(x\) mede $15$ cm e um segmento \(y\) mede $20$ cm, a razão entre eles é escrita como \(\frac{x}{y}\) e é igual a \(\frac{15}{20}=\frac{3}{4}=0,75\).

Quando um ponto \(P\) está no interior do segmento \(AB\), para definir sua posição em relação aos extremos do segmento é costume definir um número chamado de “razão em que \(P\) divide o segmento \(AB\)”.

Essa razão é o número real \(\frac{PA}{PB}\).

Assim, se um segmento \(AB\) mede \(10\)cm e um ponto \(P\) sobre ele está a $4$ cm de A então a razão em que \(P\) divide esse segmento é \(\frac{PA}{PB}=\frac{4}{6}=\frac{2}{3}\).
\begin{center}\begin{tikzpicture}
\draw [line width=0.8pt] (0,0)-- (10,0);
\draw (1.7,.7) node[anchor=north west] {4};
\draw (7.2,.7) node[anchor=north west] {6};
\draw [fill=black] (0,0) circle (1.0pt);
\draw[color=black] (0,-0.5) node {$A$};
\draw [fill=black] (10,0.) circle (1.0pt);
\draw[color=black] (10,-0.5) node {$B$};
\draw [fill=black] (4,0.) circle (1.0pt);
\draw[color=black] (4,-0.5) node {$P$};
\end{tikzpicture}\end{center}
Isso pode ser visualizado na figura a seguir onde o segmento \(PA\) contém 2 partes e o segmento \(PB\), 3 partes.
\begin{center}\begin{tikzpicture}
\definecolor{ffqqqq}{rgb}{1.,0.,0.}
\draw [line width=0.8pt] (0.,0.)-- (10.,0.);
\draw [line width=.5pt,dashed,color=ffqqqq] (0.,0.4)-- (2.,0.4);
\draw [line width=.5pt,dashed,color=ffqqqq] (2.,0.4)-- (4.,0.4);
\draw [line width=.5pt,dashed,color=ffqqqq] (4.,0.4)-- (6.,0.4);
\draw [line width=.5pt,dashed,color=ffqqqq] (6.,0.4)-- (8.,0.4);
\draw [line width=.5pt,dashed,color=ffqqqq] (8.,0.4)-- (10.,0.4);
\draw (0.7,0.8) node[anchor=north west] {$a$};
\draw (2.7,0.8) node[anchor=north west] {$a$};
\draw (4.7,0.8) node[anchor=north west] {$a$};
\draw (6.7,0.8) node[anchor=north west] {$a$};
\draw (8.7,0.8) node[anchor=north west] {$a$};
\draw [fill=black] (0.,0.) circle (1.0pt);
\draw[color=black] (-0.1322345939243572,-0.7290716059761639) node {$A$};
\draw [fill=black] (10.,0.) circle (1.0pt);
\draw[color=black] (9.974376348759716,-0.7290716059761639) node {$B$};
\draw [fill=black] (4.,0.) circle (1.0pt);
\draw[color=black] (4.004668330801247,-0.6767057461695107) node {$P$};
\draw [color=black] (0.,0.4)-- ++(-2.5pt,0 pt) -- ++(5.0pt,0 pt) ++(-2.5pt,-2.5pt) -- ++(0 pt,5.0pt);
\draw [color=black] (2.,0.4)-- ++(-2.5pt,0 pt) -- ++(5.0pt,0 pt) ++(-2.5pt,-2.5pt) -- ++(0 pt,5.0pt);
\draw [color=black] (4.,0.4)-- ++(-2.5pt,0 pt) -- ++(5.0pt,0 pt) ++(-2.5pt,-2.5pt) -- ++(0 pt,5.0pt);
\draw [color=black] (6.,0.4)-- ++(-2.5pt,0 pt) -- ++(5.0pt,0 pt) ++(-2.5pt,-2.5pt) -- ++(0 pt,5.0pt);
\draw [color=black] (8.,0.4)-- ++(-2.5pt,0 pt) -- ++(5.0pt,0 pt) ++(-2.5pt,-2.5pt) -- ++(0 pt,5.0pt);
\draw [color=black] (10.,0.4)-- ++(-2.5pt,0 pt) -- ++(5.0pt,0 pt) ++(-2.5pt,-2.5pt) -- ++(0 pt,5.0pt);
\end{tikzpicture}\end{center}
\begin{task}{segmentos comensuráveis}



Dois segmentos são chamados de comensuráveis quando é possível determinar um terceiro segmento que cabe exatamente um número inteiro de vezes em um deles e também um número inteiro de vezes no outro.

Assim, dados dois segmentos \(a\) e \(b\), um segmento que cabe um número inteiro de vezes em um deles e também um número inteiro de vezes no outro é chamado de uma \textit{unidade}, e vamos representá-lo por \(u\).

Por exemplo, se \(a = 8\) cm e \(b = 10\) cm a unidade \(u = 1\) cm cabe 8 vezes em \(a\) e 10 vezes em \(b\), mas podemos tomar \(u = 2\) cm pois essa unidade cabe 4 vezes em \(a\) e 5 vezes em \(b\). . Porém, há outras opções para \(u\), que dependem da escolha de cada pessoa. Com esses mesmos segmentos, podemos escolher, por exemplo,  \(u = 0{,}5\) cm e assim, essa unidade cabe 16 vezes em \(a\) e 20 vezes em \(b\).
Esses são os segmentos comensuráveis: os segmentos que permitem encontrar uma unidade de medida comum.

Responda

Na tabela abaixo, para cada par de segmentos \(a\) e \(b\), com suas medidas dadas em centímetros,  encontre uma unidade \(u\) de medida comum de forma que ela caiba um número inteiro de vezes em cada um deles..
\end{task}




\begin{task}{compreendendo o teorema de Tales}



Enunciado do teorema de Tales:

“Se um feixe de paralelas está cortado por duas transversais então os segmentos determinados sobre uma transversal são respectivamente proporcionais aos segmentos determinados na outra”.

Vejamos uma figura
\begin{center}\begin{tikzpicture}
\draw [line width=0.8pt] (-3.24,0.)-- (4.8,0.);
\draw [line width=0.8pt] (-3.3,4.22)-- (4.82,4.22);
\draw [line width=0.8pt] (-3.28,3.02)-- (4.8,3.02);
\draw [line width=0.8pt] (-3.26,2.32)-- (4.82,2.32);
\draw [line width=0.8pt] (-2.68,4.76)-- (-1.26,-0.56);
\draw [line width=0.8pt] (-1.42,4.84)-- (4.78,-0.64);
\draw [line width=2.pt,color=session1] (-2.535864661654135,4.22)-- (-2.215563909774436,3.02);
\draw [line width=2.pt,color=session3] (-2.215563909774436,3.02)-- (-2.0287218045112776,2.32);
\draw [line width=2.pt,color=session2] (-2.0287218045112776,2.32)-- (-1.4094736842105262,0.);
\draw [line width=2.pt,color=session1] (-0.7185401459854013,4.22)-- (0.6391240875912405,3.02);
\draw [line width=2.pt,color=session3] (0.6391240875912405,3.02)-- (1.4310948905109493,2.32);
\draw [line width=2.pt,color=session2] (1.4310948905109493,2.32)-- (4.0559124087591245,0.);
\draw (-2.9,3.9) node[anchor=north west] {$ a $};
\draw (0.1,4) node[anchor=north west] {$ a' $};
\draw (-2.6,3) node[anchor=north west] {$ b $};
\draw (1.2,3.0) node[anchor=north west] {$ b' $};
\draw (-2.2,1.4) node[anchor=north west] {$c$};
\draw (2.9,1.7) node[anchor=north west] {$ c' $};
\draw [fill=black] (-2.535864661654135,4.22) circle (1.0pt);
\draw [fill=black] (-2.215563909774436,3.02) circle (1.0pt);
\draw [fill=black] (-2.0287218045112776,2.32) circle (1.0pt);
\draw [fill=black] (-1.4094736842105262,0.) circle (1.0pt);
\draw [fill=black] (-0.7185401459854013,4.22) circle (1.0pt);
\draw [fill=black] (0.6391240875912405,3.02) circle (1.0pt);
\draw [fill=black] (1.4310948905109493,2.32) circle (1.0pt);
\draw [fill=black] (4.0559124087591245,0.) circle (1.0pt);
\end{tikzpicture}\end{center}
Responda considerando a figura acima
\begin{enumerate}
\item {} 
Qual é a hipótese do teorema?

\item {} 
Qual é a tese do teorema?

\end{enumerate}
\end{task}


\begin{task}{demonstrando o teorema de Tales}

A figura a seguir mostra três retas paralelas cortadas por duas transversais. Na reta da esquerda, os segmentos \(AB = a\) e \(BC = b\), são comensuráveis e, na reta da direita, os segmentos correspondentes são \(A'B' = a'\) e \(B'C' = b'\).
Nosso objetivo é demonstrar que
\begin{equation*}
\begin{split}\frac{a}{a'}=\frac{b}{b'}\end{split}
\end{equation*}
Como \(a\) e \(b\) são comensuráveis, a figura mostra uma unidade \(u\) comum a esses segmentos.

Por cada extremidade da unidade \(u\) assinalada na reta da esquerda traçamos retas paralelas às retas dadas determinando segmentos correspondentes na reta da direita.

Digamos que a unidade \(u\) cabe \(m\) vezes em \(a\). Então \(a = mu\).

Digamos que a unidade \(u\) cabe \(n\) vezes em \(b\). Então \(b = nu\).

Sabemos (da atividade \hyperref[demonstrando-afirmacao]{"Demonstrando um Fato"}) que, em retas paralelas cortadas por transversais, segmentos iguais de um lado correspondem a segmentos iguais do outro. A cada segmento \(u\) do lado esquerdo existe um correspondente \(v\) do lado direito.


\begin{center}\begin{tikzpicture}
\draw [line width=0.8pt] (-3.56,0.)-- (5.52,0.);
\draw [line width=0.8pt] (-3.44,3.8)-- (5.34,3.8);
\draw [line width=0.8pt] (-3.36,6.38)-- (5.3,6.38);
\draw [line width=0.8pt] (-1.98,6.94)-- (-3.06,-0.72);
\draw [line width=0.8pt] (-0.78,6.94)-- (5.36,-0.74);
\draw [line width=2.4pt,color=session3] (-2.058955613577024,6.38)-- (-2.120786270710496,5.941460339220002);
\draw [line width=2.4pt,color=session3] (-2.120786270710496,5.941460339220002)-- (-2.1826169278439687,5.502920678440002);
\draw [line width=0.8pt] (-2.1826169278439687,5.502920678440002)-- (-2.2444475849774412,5.064381017660003);
\draw [line width=2.4pt,color=session3] (-2.422715404699739,3.8)-- (-2.484546061833212,3.361460339220001);
\draw [line width=2.4pt,color=session2] (-0.33229166666666593,6.38)-- (0.018311655884010396,5.941460339220002);
\draw [line width=2.4pt,color=session2] (0.018311655884010396,5.941460339220002)-- (0.3689149784346872,5.502920678440002);
\draw [line width=2.4pt,color=session2] (1.7303645833333339,3.8)-- (2.0809679058840103,3.361460339220001);
\draw [line width=0.8pt,dash pattern=on 1pt off 1pt,color=session1] (-2.120786270710496,5.941460339220002)-- (0.018311655884010396,5.941460339220002);
\draw [line width=0.8pt,dash pattern=on 1pt off 1pt,color=session1] (-2.1826169278439687,5.502920678440002)-- (0.3689149784346872,5.502920678440002);
\draw [line width=0.8pt,dash pattern=on 1pt off 1pt,color=session1] (-2.2444475849774412,5.064381017660003)-- (0.7195183009853631,5.064381017660003);
\draw [line width=0.8pt,dash pattern=on 1pt off 1pt,color=session1] (-2.360884747566267,4.238539660779997)-- (1.3797612607826595,4.238539660779998);
\draw [line width=0.8pt,dash pattern=on 1pt off 1pt,color=session1] (-2.484546061833212,3.361460339220001)-- (2.0809679058840103,3.361460339220001);
\draw [line width=0.8pt,dash pattern=on 1pt off 1pt,color=session1] (-2.5463767189666844,2.9229206784400033)-- (2.431571228434686,2.922920678440003);
\draw [line width=0.8pt,dash pattern=on 1pt off 1pt,color=session1] (-2.608207376100157,2.4843810176600054)-- (2.7821745509853613,2.4843810176600054);
\draw [line width=0.8pt,dash pattern=on 1pt off 1pt,color=session1] (-2.83482432541974,0.8770793215599975)-- (4.067178771565316,0.8770793215599976);
\draw [line width=0.8pt,dash pattern=on 1pt off 1pt,color=session1] (-2.896654982553212,0.43853966077999873)-- (4.417782094115991,0.43853966077999873);
\draw (-2.5,6.9) node[anchor=north west] {$A$};
\draw (-2.9,4.3) node[anchor=north west] {$B$};
\draw (-3.6,0.6) node[anchor=north west] {$C$};
\draw (-0.2,6.9) node[anchor=north west] {$A'$};
\draw (1.8,4.3) node[anchor=north west] {$B'$};
\draw (4.8,0.6) node[anchor=north west] {$C'$};
\draw (-2.6,6.4) node[anchor=north west] {$  u$};
\draw (-3,3.8) node[anchor=north west] {$ u$};
\draw (-.1,6.4) node[anchor=north west] {$ v $};
\draw (1.9,3.8) node[anchor=north west] {$ v $};
\draw (-3.1111119322537544,5.579654794304461) node[anchor=north west] {$ a $};
\draw (-3.5150523810387724,2.4603368842423774) node[anchor=north west] {$ b $};
\draw (1.4668798206431133,5.736742746609745) node[anchor=north west] {$ a' $};
\draw (3.7558756970915472,2.617424836547662) node[anchor=north west] {$  b'$};
\draw [fill=black] (-2.058955613577024,6.38) circle (2.0pt);
\draw [fill=black] (-0.33229166666666593,6.38) circle (2.0pt);
\draw [fill=black] (-2.422715404699739,3.8) circle (2.0pt);
\draw [fill=black] (1.7303645833333339,3.8) circle (2.0pt);
\draw [fill=black] (-2.958485639686684,0.) circle (2.0pt);
\draw [fill=black] (4.768385416666668,0.) circle (2.0pt);
\draw [fill=black] (-2.120786270710496,5.941460339220002) circle (2.0pt);
\draw [fill=black] (-2.1826169278439687,5.502920678440002) circle (2.0pt);
\draw [fill=black] (-2.2444475849774412,5.064381017660003) circle (2.0pt);
\draw [fill=black] (-2.360884747566267,4.238539660779997) circle (2.0pt);
\draw [fill=black] (-2.484546061833212,3.361460339220001) circle (2.0pt);
\draw [fill=black] (-2.896654982553212,0.43853966077999873) circle (2.0pt);
\draw [fill=black] (-2.5463767189666844,2.9229206784400033) circle (2.0pt);
\draw [fill=black] (-2.608207376100157,2.4843810176600054) circle (2.0pt);
\draw [fill=black] (-2.83482432541974,0.8770793215599975) circle (2.0pt);
\draw [fill=black] (0.018311655884010396,5.941460339220002) circle (2.0pt);
\draw [fill=black] (0.3689149784346872,5.502920678440002) circle (2.0pt);
\draw [fill=black] (0.7195183009853631,5.064381017660003) circle (2.0pt);
\draw [fill=black] (1.3797612607826595,4.238539660779998) circle (2.0pt);
\draw [fill=black] (2.0809679058840103,3.361460339220001) circle (2.0pt);
\draw [fill=black] (2.431571228434686,2.922920678440003) circle (2.0pt);
\draw [fill=black] (2.7821745509853613,2.4843810176600054) circle (2.0pt);
\draw [fill=black] (4.067178771565316,0.8770793215599976) circle (2.0pt);
\draw [fill=black] (4.417782094115991,0.43853966077999873) circle (2.0pt);
\end{tikzpicture}\end{center}

Complete a demonstração


\begin{enumerate}
\item {} 
Quantas vezes o segmento \(v\) cabe em \(a'\)?

\item {} 
Quantas vezes o segmento \(v\) cabe em \(b'\)?

\item {} 
Escreva as medidas de \(a'\) e \(b'\) na unidade \(v\), reuna essas medidas com as anteriores e conclua o resultado do teorema

\end{enumerate}
\end{task}



\begin{observation}

O teorema de Tales foi demonstrado no caso dos dois segmentos de uma das retas serem comensuráveis. Entretanto, o teorema vale quando as medidas desses dois segmentos são números reais quaisquer. A demonstração geral do teorema está no final do capítulo, na seção Para Saber Mais.
\end{observation}

\clearpage
\def\currentcolor{session2}
\begin{objectives}{Resolvendo a situação inicial}
{
\begin{itemize}
\item {} 
Reconhecer a aplicabilidade do teorema de Tales em uma situação real.

\end{itemize}
}{1}{1}
\end{objectives}
\begin{sugestions}{Resolvendo a situação inicial}
{
\begin{itemize}
\item {} 
Os alunos devem perceber que os segmentos que aparecem nas transversais não precisam estar conectados.

\item {} 
Deixe que os alunos concluam o resultado e encontrem a resposta. Não dê nenhuma dica inicialmente.

\end{itemize}
}{1}{1}
\end{sugestions}
\begin{answer}{Resolvendo a situação inicial}
{
\begin{enumerate}
\item {} 
Porque as ruas são paralelas e estão cortadas por transversais.

\item {} 
\(184m\)

\end{enumerate}
}{1}
\end{answer}
\begin{objectives}{Resolvendo novas situações}
{
\begin{itemize}
\item {} 
Reconhecer uma aplicação do teorema de Tales em uma situação matemática.

\end{itemize}
}{1}{1}
\end{objectives}
\begin{sugestions}{Resolvendo novas situações}
{
\begin{itemize}
\item {} 
Os alunos compreenderam o teorema de Tales a partir da representação mais tradicional. Nesta atividade, vamos variar as situações propostas para que eles possam raciocinar e decidir como aplicar o teorema de Tales

\item {} 
Em caso de dificuldade sugira que eles tracem uma nova paralela pelo ponto de interseção dos segmentos. Eles deverão reconhecer, então, o teorema de Tales.

\end{itemize}
}{1}{1}
\end{sugestions}
\clearmargin
\begin{answer}{Resolvendo novas situações}
{
\begin{enumerate}
\item {} 
5,25

\item {} 
\(\dfrac{a}{d}=\dfrac{c}{b}\)

\end{enumerate}
}{1}
\end{answer}


\practice{}

\begin{task}{resolvendo a situação inicial}



Vamos voltar ao mapa que mostramos na primeira atividade, agora desenhado de forma esquemática.
\begin{center}\begin{tikzpicture}
\draw [line width=0.8pt] (0.,5.)-- (7.2,1.44);
\draw [line width=0.8pt] (0.,5.)-- (0.,-2.86);
\draw [line width=2.8pt,color=session3] (7.2,1.44)-- (5.629240434037691,-1.60334665905197);
\draw [line width=0.8pt] (5.629240434037691,-1.60334665905197)-- (4.7328383780696734,-3.3401256424900057);
\draw [line width=2.8pt,color=session1] (4.7328383780696734,-3.3401256424900057)-- (3.968018275271273,-4.821964591661907);
\draw [line width=0.8pt] (0.,3.22)-- (6.468075385494003,0.02189605939463218);
\draw [line width=0.8pt] (0.,1.18)-- (5.629240434037691,-1.60334665905197);
\draw [line width=0.8pt] (0.,2.86)-- (6.320045688178182,-0.2649114791547684);
\draw [line width=0.8pt] (0.,-1.)-- (4.7328383780696734,-3.3401256424900057);
\draw [line width=0.8pt] (0.,-2.86)-- (3.968018275271273,-4.821964591661907);
\draw [line width=2.8pt,color=session3] (0.,5.)-- (0.,1.18);
\draw [line width=2.8pt,color=session1] (0.,-1.)-- (0.,-2.86);
\draw [color=session1](4.4,-4.0) node[anchor=north west] {$A$};
\draw [color=session1](-0.6170786902907064,-1.7707838711724064) node[anchor=north west] {$B$};
\draw [color=session3](6.5,-0.3084989813871627) node[anchor=north west] {$C$};
\draw [color=session3](-0.6767637878329612,3.3322919686903827) node[anchor=north west] {$D$};
\draw [fill=session3] (0.,5.) circle (2.5pt);
\draw [fill=session3] (0.,1.18) circle (2.5pt);
\draw [fill=session1] (0.,-1.) circle (2.5pt);
\draw [fill=session1] (0.,-2.86) circle (2.5pt);
\draw [fill=session3] (7.2,1.44) circle (2.5pt);
\draw [fill=session3] (5.629240434037691,-1.60334665905197) circle (2.5pt);
\draw [fill=session1] (4.7328383780696734,-3.3401256424900057) circle (2.5pt);
\draw [fill=session1] (3.968018275271273,-4.821964591661907) circle (2.5pt);
\end{tikzpicture}\end{center}\begin{enumerate}
\item {} 
Por que o teorema de Tales pode ser utilizado nessa situação?

\item {} 
Utilizando os dados da atividade inicial (\hyperref[nas-ruas]{"Nas ruas de uma cidade"}) calcule a medida \(D\).

\end{enumerate}
\end{task}




\begin{task}{resolvendo novas situações}



Sempre que houver retas paralelas e trasnversais, o teorema de Tales estará presente. Nas figuras a seguir, as retas paralelas estão assinaladas com seu símbolo tradicional.
\begin{center}\begin{tikzpicture}
\draw [line width=0.8pt] (-2.04,0.56)-- (3.44,3.96);
\draw [line width=0.8pt] (-2.04,0.56)-- (-1.9,3.68);
\draw [line width=0.8pt] (-1.9,3.68)-- (2.1553812418250367,3.1629737631761174);
\draw [line width=0.8pt] (-1.9789362312897274,1.9208497026860774)-- (1.9886091282959177,4.382465436735565);
\draw [line width=0.8pt] (0.3865465119480187,3.3884849813372337)-- (1.9886091282959177,4.382465436735565);
\draw [line width=0.8pt] (1.1103882995267922,3.9434991515970954) -- (1.2052858218334364,3.7905466744675635);
\draw [line width=0.8pt] (1.1698698184104994,3.980403743605235) -- (1.2647673407171436,3.8274512664757028);
\draw [line width=0.8pt] (2.1553812418250367,3.1629737631761174)-- (3.44,3.96);
\draw [line width=0.8pt] (2.7205011003173425,3.6195108241487546) -- (2.8153986226239867,3.4665583470192227);
\draw [line width=0.8pt] (2.7799826192010486,3.6564154161568942) -- (2.8748801415076928,3.5034629390273624);
\draw (-2.4,2.88) node[anchor=north west] {6};
\draw (-2.8,1.34) node[anchor=north west] {4,2};
\draw (-0.74,4.1) node[anchor=north west] {7,5};
\draw (1.42,3.8) node[anchor=north west] {$ x $};
\draw [fill=black] (-2.04,0.56) circle (1.0pt);
\draw [fill=black] (-1.9,3.68) circle (1.0pt);
\draw [fill=black] (2.1553812418250367,3.1629737631761174) circle (1.0pt);
\draw [fill=black] (-1.9789362312897274,1.9208497026860774) circle (1.0pt);
\draw [fill=black] (0.3865465119480187,3.3884849813372337) circle (1.0pt);
\end{tikzpicture}\end{center}\begin{enumerate}
\item {} 
Qual é o valor da medida que está faltando na figura acima?

\end{enumerate}
\begin{center}\begin{tikzpicture}
\draw [line width=0.8pt] (-2.8,1.)-- (1.0687644969670296,5.635260395386114);
\draw [line width=0.8pt] (-2.394800264009806,3.4591432253197976)-- (0.16,0.12);
\draw [line width=0.8pt] (-2.394800264009806,3.4591432253197976)-- (-2.22,4.52);
\draw [line width=0.8pt] (-2.418451428579632,3.9661819892341916) -- (-2.2097237468538884,3.9317893598589273);
\draw [line width=0.8pt] (-2.405076517155918,4.04735386546087) -- (-2.1963488354301743,4.012961236085605);
\draw [line width=0.8pt] (1.0687644969670296,5.635260395386114)-- (1.2676091397680478,6.8420416758336735);
\draw [line width=0.8pt] (1.0571355217928105,6.215261412184186) -- (1.265863203518554,6.180868782808922);
\draw [line width=0.8pt] (1.070510433216524,6.2964332884108645) -- (1.2792381149422676,6.2620406590356);
\draw [line width=0.8pt] (-2.8,1.)-- (-2.394800264009806,3.4591432253197976);
\draw [line width=0.8pt] (0.16,0.12)-- (1.0687644969670296,5.635260395386114);
\draw (-2,3.2) node[anchor=north west] {$ a $};
\draw (-0.9,4.5) node[anchor=north west] {$ b $};
\draw (-1.1,1.4) node[anchor=north west] {$ c $};
\draw (-2.1,2.0) node[anchor=north west] {$ d $};
\draw [fill=black] (-2.8,1.) circle (1.0pt);
\draw [fill=black] (0.16,0.12) circle (1.0pt);
\draw [fill=black] (1.0687644969670296,5.635260395386114) circle (1.0pt);
\draw [fill=black] (-2.394800264009806,3.4591432253197976) circle (1.0pt);
\draw [fill=black] (-1.6069520360906888,2.4294196457913784) circle (1.0pt);
\end{tikzpicture}\end{center}\begin{enumerate}
\item {} 
Encontre uma relação entre os quatro segmentos assinalados na figura acima.

\end{enumerate}
\end{task}



\arrange{}

\paragraph{Como se divide um segmento em uma razão dada?}

Imagine que tenhamos um segmento \(AB\) e desejamos determinar, no seu interior o ponto \(P\) que o divide na razão \(\frac{PA}{PB}=\frac{3}{4}\). Isso significa encontrar um ponto \(P\) no interior do segmento \(AB\) de forma que o segmento \(PA\) seja proporcional a 3 e o segmento \(PB\), proporcional a 4. Um procedimento bastante usado é o descrito a seguir e mostrado na figura a seguir à esquerda.

A partir dos pontos \(A\) e \(B\) trace semirretas paralelas quaisquer,{}`AX{}` e \(BY\),  mas com sentidos opostos como ilustrado na figura.
Usando o compasso com uma abertura qualquer,mas fixada, assinale três segmentos iguais e consecutivos na semirreta de origem  \(A\) e, com a mesma abertura do compasso, quatro segmentos na semirreta de origem \(B\).

Temos então \(AX = 3u\) e \(BY = 3u\).

A interseção da reta \(XY\) com o segmento \(AB\) é o ponto \(P\) procurado.
\begin{center}\begin{tikzpicture}
\draw [line width=0.8pt] (-3.,0.)-- (-1.62,2.92);
\draw [line width=0.8pt] (-3.,0.)-- (1.,0.);
\draw [line width=0.8pt] (3.,0.)-- (7.,0.);
\draw [line width=0.8pt] (1.,0.)-- (-0.502351497488208,-3.178888675844614);
\draw [line width=0.8pt] (-2.1744449131418495,1.746826705525942)-- (-0.10074011581086806,-2.3291022740345904);
\draw [line width=0.8pt] (3.8255550868581505,1.7468267055259423)-- (5.899259884189131,-2.3291022740345904);
\draw [line width=0.8pt] (3.,0.)-- (4.228558269739617,2.59955807799977);
\draw [line width=0.8pt] (7.,0.)-- (4.767191011235955,-4.724494382022471);
\draw [line width=0.8pt,dash pattern=on 1pt off 1pt] (2.7729644593415306,0.4462451650548752)-- (5.372740816526203,-4.663693244820947);
\draw [line width=0.8pt,dash pattern=on 1pt off 1pt] (3.0256681798973055,1.072708419540557)-- (5.649835588579647,-4.085171333659179);
\draw [line width=0.8pt,dash pattern=on 1pt off 1pt] (3.2905674262019144,1.6752010023642823)-- (5.959404510836151,-3.570478338777889);
\draw [line width=0.8pt,dash pattern=on 1pt off 1pt] (3.5514960008445655,2.2854980594391745)-- (6.1918296069376995,-2.904156839673689);
\draw [line width=0.8pt,dash pattern=on 1pt off 1pt] (4.187082831957212,2.159393075614541)-- (6.562234402556889,-2.509038578647954);
\draw [line width=0.8pt,dash pattern=on 1pt off 1pt] (4.762684778290206,2.1511901820602617)-- (6.823021221818061,-1.8984628971188342);
\draw [line width=0.8pt,dash pattern=on 1pt off 1pt] (5.34265446345135,2.1344023668297014)-- (7.096003566828064,-1.3118578872516715);
\draw [line width=0.8pt,dash pattern=on 1pt off 1pt] (5.937130743373617,2.0891014044734284)-- (7.329336201567188,-0.6473201825610867);
\draw (-3.44364,0.06722) node[anchor=north west] {A};
\draw (2.5991,0.01398) node[anchor=north west] {A};
\draw (1.21486,0.20032) node[anchor=north west] {B};
\draw (7.2576,0.06722) node[anchor=north west] {B};
\draw (-2.48532,2.32992) node[anchor=north west] {X};
\draw (3.29122,2.06442) node[anchor=north west] {X};
\draw (0.04358,-2.59478) node[anchor=north west] {Y};
\draw (6.00646,-2.38182) node[anchor=north west] {Y};
\draw (-3.2573,0.62624) node[anchor=north west] {u};
\draw (0.97528,-0.33208) node[anchor=north west] {u};
\draw [color=session3](-1.527,-0.2256) node[anchor=north west] {P};
\draw [color=session3](4.51574,-0.2256) node[anchor=north west] {P};
\draw [fill=black] (-3.,0.) circle (1.0pt);
\draw [fill=black] (1.,0.) circle (1.0pt);
\draw [fill=black] (3.,0.) circle (1.0pt);
\draw [fill=black] (7.,0.) circle (1.0pt);
\draw [fill=black] (-2.724814971047283,0.5822755685086475) circle (1.0pt);
\draw [fill=black] (-2.449629942094566,1.1645511370172947) circle (1.0pt);
\draw [fill=black] (-3.,0.) circle (1.0pt);
\draw [fill=black] (-2.1744449131418495,1.746826705525942) circle (1.0pt);
\draw [fill=black] (0.724814971047283,-0.5822755685086476) circle (1.0pt);
\draw [fill=black] (0.44962994209456597,-1.1645511370172952) circle (1.0pt);
\draw [fill=black] (0.17444491314184896,-1.7468267055259428) circle (1.0pt);
\draw [fill=black] (-0.10074011581086806,-2.3291022740345904) circle (1.0pt);
\draw [fill=black] (3.2751850289527167,0.5822755685086475) circle (1.0pt);
\draw [fill=black] (3.550370057905434,1.1645511370172947) circle (1.0pt);
\draw [fill=black] (3.8255550868581505,1.7468267055259423) circle (1.0pt);
\draw [fill=black] (6.724814971047282,-0.5822755685086476) circle (1.0pt);
\draw [fill=black] (6.449629942094566,-1.1645511370172952) circle (1.0pt);
\draw [fill=black] (6.174444913141849,-1.7468267055259428) circle (1.0pt);
\draw [fill=black] (5.899259884189131,-2.3291022740345904) circle (1.0pt);
\draw [fill=black] (5.6240748552364135,-2.9113778425432377) circle (1.0pt);
\draw [fill=black] (5.348889826283696,-3.493653411051885) circle (1.0pt);
\draw [fill=black] (5.073704797330978,-4.0759289795605325) circle (1.0pt);
\draw [fill=black] (3.5714285714285707,0.) circle (1.0pt);
\draw [fill=black] (4.142857142857142,0.) circle (1.0pt);
\draw [fill=session3] (4.7142857142857135,0.) circle (1.5pt);
\draw [fill=black] (5.285714285714286,0.) circle (1.0pt);
\draw [fill=black] (5.857142857142857,0.) circle (1.0pt);
\draw [fill=black] (6.428571428571428,0.) circle (1.0pt);
\draw [fill=session3] (-1.2857142857142863,0.) circle (1.5pt);
\end{tikzpicture}\end{center}
A figura acima, à direita, justifica visualmente a construção. Se um feixe de paralelas determina sobre uma transversal segmentos iguais determinará, sobre qualquer outra, segmentos também iguais.

Assim, o segmento \(AB\) está dividido em 7 partes iguais e o ponto \(P\) é o terceiro ponto de divisão. Logo, \(\frac{PA}{PB}=\frac{3}{4}\).

Observe ainda que, dado um segmento e um número positivo \(k\), \textbf{só existe um ponto interior ao segmento que o divide na razão} \(k\). . De fato, considerando \(k\) em uma das semirretas e a unidade de medida na outra, teremos \(\dfrac{PA}{PB}=\dfrac{k}{1}=k\).
\clearpage
\def\currentcolor{session3}
\begin{objectives}{A projeção paralela}
{
\begin{itemize}
\item {} 
Aplicar o teorema de Tales para compreender a projeção paralela.

\end{itemize}
}{1}{2}
\end{objectives}
\begin{sugestions}{A projeção paralela}
{
\begin{itemize}
\item {} 
Lembrar o conceito de razão em que um ponto divide um segmento.
\end{itemize}
}{1}{2}
\end{sugestions}
\clearmargin
\marginpar{\vspace{.5em}}
\begin{answer}{A projeção paralela}
{
\begin{enumerate}
\item {} 
3,2

\item {} 
\(\dfrac{2}{5}\)

\end{enumerate}
}{1}
\end{answer}
\begin{objectives}{Recíproca do Teorema de Tales}
{
\begin{itemize}
\item {} 
Usar sua intuição para responder a uma situação nova, mas relacionada com conceitos que já aprendeu

\item {} 
Aprender uma nova técnica de demonstração

\end{itemize}
}{1}{1}
\end{objectives}
\begin{sugestions}{Recíproca do Teorema de Tales}
{
\begin{itemize}
\item {} 
Na primeira parte da atividade o aluno deve usar sua intuição para responder. A justificativa dele para a resposta é importante para que você possa perceber se ele já tem a ideia da recíproca.

\item {} 
Na segunda parte da atividade o aluno deverá acompanhar com atenção a demonstração da recíproca do Teorema de Tales pois ela introduz, de forma leve, a técnica de demonstração por absurdo
\end{itemize}
}{1}{1}
\end{sugestions}
\begin{answer}{Recíproca do Teorema de Tales}
{
\begin{enumerate}
\item {} 
Resposta pessoal.

\item {} 
Resposta pessoal, A resposta que o professor pode dar aos alunos pode ser: A razão \(\frac{5}{13}\) é diferente da razão \(\frac{3}{8}\). Isso ficará claro com a recíproca do teorema de Tales.

\end{enumerate}
}{1}
\end{answer}


\def\currentcolor{session4}

\paragraph{O que é a recíproca de um teorema?}

Sabemos que um teorema é uma afirmação do tipo “Se A então B”. A recíproca de um teorema é uma afirmação onde as expressões A e B trocam de lugar. Assim a recíproca de “Se A então B” é “Se B então A”.

Um teorema é uma afirmação matemática verdadeira (pois conseguimos demonstrá-lo), mas sua recíproca nem sempre é verdadeira. Quando estamos trabalhando com números frequentemente as recíprocas das afirmações não são verdadeira, como no exemplo a seguir.

\textbf{Teorema}: Todo número múltiplo de 4 é par. (\textit{verdadeiro})

\textbf{Recíproca}: Todo número par é múltiplo de 4.(\textit{falso})

Em geometria, a maioria dos teoremas possui sua recíproca também verdadeira, mas isso é preciso verificar em cada caso. No caso do teorema de Tales a sua recíproca está  na atividade \hyperref[reciproca-tales]{Recíproca do Teorema de Tales}


\know{}

\begin{task}{a projeção paralela}

Na figura a seguir você vê um segmento \(AB\), um ponto \(P\) no seu interior e as retas \(r\) e \(d\).
\begin{center}\begin{tikzpicture}
\draw [line width=0.8pt] (-3.189538526130667,0.)-- (2.8609534583473004,0.);
\draw [line width=0.8pt] (-1.9,4.88)-- (-3.1,1.38);
\draw [line width=1.6pt,color=session1] (-0.9,3.22)-- (2.18,4.88);
\draw [line width=1.6pt,color=session2] (-2.004,0.)-- (0.5068571428571428,0.);
\draw [line width=0.8pt,dash pattern=on 3pt off 3pt] (-0.9,3.22)-- (-2.004,0.);
\draw [line width=0.8pt,dash pattern=on 3pt off 3pt] (0.027994075551628028,3.7201526511089944)-- (-1.2474868334000275,0.);
\draw [line width=0.8pt,dash pattern=on 3pt off 3pt] (2.18,4.88)-- (0.5068571428571428,0.);
\draw (-1.2025327921153781,3.9) node[anchor=north west] {A};
\draw (-0.1503833145561734,4.4) node[anchor=north west] {P};
\draw (2.0618284074913853,5.434423837176202) node[anchor=north west] {B};
\draw (-2.28166046140687,-0.23099642660413633) node[anchor=north west] {A'};
\draw (-1.3374237507768147,-0.23099642660413633) node[anchor=north west] {P'};
\draw (0.41615871182185993,-0.204018234871849) node[anchor=north west] {B'};
\draw (2.655348625601706,0.4434583667030467) node[anchor=north west] {r};
\draw (-2.4974859952651687,4.679034468672157) node[anchor=north west] {d};
\draw [fill=black] (-0.9,3.22) circle (1.0pt);
\draw [fill=black] (2.18,4.88) circle (1.0pt);
\draw [fill=black] (0.027994075551628028,3.7201526511089944) circle (1.0pt);
\draw [fill=black] (-2.004,0.) circle (1.0pt);
\draw [fill=black] (-1.2474868334000275,0.) circle (1.0pt);
\draw [fill=black] (0.5068571428571428,0.) circle (1.0pt);
\end{tikzpicture}\end{center}
A “projeção paralela sobre \(r\) na direção \(d\)” é uma função que, a cada ponto \(X\) do plano associa um ponto \(X'\) da seguinte forma: Trace por \(X\) uma reta paralela a \(d\). Onde essa reta intersectar \(r\) está o ponto \(X'\).

Observandoa  figura acima, essa função parece uma chuva com vento da direita para a esquerda, fazendo as gotas caírem no chão, a reta \(r\).

A razão em que o ponto \(P\) divide o segmento \(AB\) é \(\dfrac{PA}{PB}\). Entretanto, pelo teorema de Tales, temos que  \(\dfrac{PA}{P'A'}=\dfrac{PB}{P'B'}\).

Isso quer dizer que  \(\dfrac{PA}{PB}=\dfrac{P'A'}{P'B'}\), ou seja, a razão em que o ponto \(P\) divide o segmento \(AB\) é a mesma razão em que o ponto \(P'\) divide o segmento \(A'B'\).

Dizemos então que \textbf{A projeção paralela conserva as razões.}

Suponha agora que, na figura acima  tenha-se \(\dfrac{PA}{PB}=\dfrac{2}{3}\) e que \(A'B'\) tenha 8 centímetros.
\begin{enumerate}
\item {} 
Quanto mede o segmento \(A'P'\)?

\item {} 
Qual é a razão \(\dfrac{A'P'}{A'B'}\) ?

\end{enumerate}
\end{task}



\begin{task}{Recíproca do Teorema de Tales}
\label{reciproca-tales}


\paragraph{Parte 1} Observe a figura a seguir
\begin{center}\begin{tikzpicture}
\draw [line width=0.8pt] (-1.,0.)-- (6.307729090909094,0.);
\draw [line width=0.8pt] (0.3818181818181818,2.98)-- (-0.014751470794228672,0.);
\draw [line width=0.8pt] (0.3818181818181818,2.98)-- (5.587090166690014,0.);
\draw [line width=0.8pt] (-0.8,1.0163636363636355)-- (4.976729090909093,1.006014545454545);
\draw (-0.24079090909090867,2.0974345454545427) node[anchor=north west] {5};
\draw (-0.4271309090909089,0.6599545454545442) node[anchor=north west] {3};
\draw (2.2082490909090944,2.3636345454545427) node[anchor=north west] {13};
\draw (5.083209090909098,0.7131945454545442) node[anchor=north west] {8};
\draw (4.870249090909097,1.5384145454545435) node[anchor=north west] {r};
\draw (6.148009090909099,0.47361454545454446) node[anchor=north west] {s};
\draw [fill=black] (-0.014751470794228672,0.) circle (1.0pt);
\draw [fill=black] (5.587090166690014,0.) circle (1.0pt);
\draw [fill=black] (0.3818181818181818,2.98) circle (1.0pt);
\draw [fill=black] (0.12028381420322001,1.014714935047463) circle (1.0pt);
\draw [fill=black] (3.8262485808239415,1.0080756473689043) circle (1.0pt);
\end{tikzpicture}\end{center}\begin{enumerate}
\item {} 
As retas r e s são paralelas?

\item {} 
Justifique sua resposta.

\end{enumerate}

\paragraph{Parte 2} 
Observe a figura a seguir:

\begin{center}
\begin{tikzpicture}
\draw [line width=0.8pt] (-1.,0.)-- (6.307729090909094,0.);
\draw [line width=0.8pt] (0.3818181818181818,2.98)-- (-0.014751470794228672,0.);
\draw [line width=0.8pt] (0.3818181818181818,2.98)-- (5.587090166690014,0.);
\draw [line width=0.8pt] (-0.8,1.0163636363636355)-- (4.976729090909093,1.006014545454545);
\draw (4.870249090909097,1.538414545454544) node[anchor=north west] {$r$};
\draw (6.148009090909099,0.4736145454545451) node[anchor=north west] {$s$};
\draw (-0.29403090909090873,2.3) node[anchor=north west] {$ a $};
\draw (-0.40051090909090886,.9) node[anchor=north west] {$ b $};
\draw (2.554309090909095,2.5) node[anchor=north west] {$ a' $};
\draw (5.0,1.1) node[anchor=north west] {$ b' $};
\draw [fill=black] (-0.014751470794228672,0.) circle (1.0pt);
\draw [fill=black] (5.587090166690014,0.) circle (1.0pt);
\draw [fill=black] (0.3818181818181818,2.98) circle (1.0pt);
\draw [fill=black] (0.12028381420322001,1.014714935047463) circle (1.0pt);
\draw [fill=black] (3.8262485808239415,1.0080756473689043) circle (1.0pt);
\end{tikzpicture}
\end{center}
Na figura acima, se $\frac{a}{a'}=\frac{b}{b'}$ as retas $r$ e $s$ são paralelas? A resposta é sim e essa ideia é a recíproca do teorema de Tales. Você vai agora acompanhar a justificativa desse fato.

\textit{Demonstração}: Consideremos a mesma figura anterior com algumas letras novas

\begin{center}
\begin{tikzpicture}
   \draw [line width=0.8pt] (-1.,0.)-- (6.307729090909094,0.);
   \draw [line width=0.8pt] (0.5045690909090926,3.0823745454545417)-- (-0.014751470794228672,0.);
   \draw [line width=0.8pt] (0.5045690909090926,3.0823745454545417)-- (5.587090166690014,0.);
   \draw (4.116661090909101,1.929244545454541) node[anchor=north west] {$r'$};
   \draw (6.149461090909105,0.45304454545454115) node[anchor=north west] {$s$};
   \draw (-0.23933890909090766,2.6794445454545412) node[anchor=north west] {$ a $};
   \draw (-0.43293890909090804,0.9612445454545412) node[anchor=north west] {$ b $};
   \draw (0.36566109090909354,3.623244545454541) node[anchor=north west] {$A$};
   \draw (-0.3,1.7) node[anchor=north west] {P};
   \draw (-0.23933890909090766,-0.27295545454545883) node[anchor=north west] {$B$};
   \draw (5.471861090909104,-0.32135545454545883) node[anchor=north west] {$C$};
   \draw [line width=0.8pt,dash pattern=on 1pt off 1pt] (-0.48558165509078804,0.9544832953920732)-- (4.2819641805126425,1.4065858531455069);
   \draw [line width=0.8pt] (-0.5781389090909086,1.015353941507814)-- (4.963661090909102,1.015353941507814);
   \draw (3.4390610909091,1.929244545454541) node[anchor=north west] {$R$};
   \draw (3.7778610909091004,0.8402445454545412) node[anchor=north west] {$Q$};
   \draw (4.987861090909103,1.445244545454541) node[anchor=north west] {$r$};
   \draw [fill=black] (-0.014751470794228672,0.) circle (1.0pt);
   \draw [fill=black] (5.587090166690014,0.) circle (1.0pt);
   \draw [fill=black] (0.5045690909090926,3.0823745454545417) circle (1.0pt);
   \draw [fill=black] (0.15631605245958863,1.015353941507814) circle (1.0pt);
   \draw [fill=black] (3.404912635547542,1.3234157567409413) circle (1.0pt);
   \draw [fill=black] (3.9128751318232196,1.015353941507814) circle (1.0pt);
\end{tikzpicture}
\end{center}

Por hipótese temos que $\frac{a}{a'}=\frac{b}{b'}$, o que é o mesmo que $\frac{a}{b}=\frac{a'}{b'}$. A primeira fração é a razão em que $P$ divide o segmento $AB$ e a segunda é a razão em que $Q$ divide o segmento $AC$. Elas são iguais, ou seja, $\frac{PA}{PB}=\frac{QA}{QC}$.

Vamos usar agora uma técnica nova de demonstração conhecida como “redução ao absurdo”. Ela consiste em negar a tese, reunir com a hipótese e depois mostrar, com argumentos sólidos, que o que afirmamos não é possível.

Queremos mostrar que as retas $r$ e $s$ são paralelas. Continuando com nossa hipótese, vamos então imaginar o seguinte:

\textit{"Suponha que as retas $r$ e $s$ não são paralelas"}

Bem, dessa forma, vamos traçar agora pelo ponto $P$ uma reta $r'$ paralela à reta $s$. Essa nova reta vai cortar o segmento $AC$ no ponto $R$.

Pelo teorema de Tales, ou melhor, pelo fato de que a projeção paralela conserva as razões, temos que $\frac{PA}{PB}=\frac{RA}{RC}$.

Assim, $\frac{QA}{QC}=\frac{RA}{RC}$ e, portanto, os pontos $Q$ e $R$ devem coincidir.

Como \(\dfrac{QA}{QC}=\dfrac{RA}{RC}\), os pontos \(Q\) e \(R\) coincidem, assim como as retas \(r\) e \(r’\).

\end{task}


\subsection{Demonstração do teorema de Tales usando Áreas}


\paragraph{Duas propriedades dos triângulos:}

A figura a seguir mostra as situações que nos permitirão concluir duas propriedades sobre os triângulos relacionadas às suas áreas.
\begin{center}
\scalebox{1.15}
{
\begin{tikzpicture}
\definecolor{ccqqqq}{rgb}{0.8,0.,0.}
\draw [line width=0.8pt,color=ccqqqq,domain=-4.384390243902435:5.875121951219508] plot(\x,{(-0.-0.*\x)/1.});
\draw [line width=0.8pt,color=ccqqqq,domain=-4.384390243902435:5.875121951219508] plot(\x,{(--2.-0.*\x)/1.});
\draw [line width=0.8pt,dash pattern=on 1pt off 1pt] (0.5581818181818191,2.)-- (0.5581818181818191,0.);
\draw [line width=0.8pt] (-2.7509090909090905,2.)-- (-3.2418181818181813,0.);
\draw [line width=0.8pt] (-3.2418181818181813,0.)-- (-0.4054545454545447,0.);
\draw [line width=0.8pt] (-0.4054545454545447,0.)-- (-2.7509090909090905,2.);]
\draw [line width=0.8pt] (-2.7509090909090905,2.)-- (-2.1327272727272724,0.);
\draw [line width=0.8pt] (1.4854545454545465,2.)-- (1.9036363636363647,0.);
\draw [line width=0.8pt] (1.9036363636363647,0.)-- (4.3036363636363655,0.);
\draw [line width=0.8pt] (4.3036363636363655,0.)-- (1.4854545454545465,2.);
\draw [line width=0.8pt] (2.7945454545454558,2.)-- (1.9036363636363647,0.);
\draw [line width=0.8pt] (2.7945454545454558,2.)-- (4.3036363636363655,0.);
\draw (-2.881951219512193,2.6) node[anchor=north west] {$A$};
\draw (-3.5,-0.1) node[anchor=north west] {$B$};
\draw (-0.5,-0.1) node[anchor=north west] {$C$};
\draw (-2.9,-0.2) node[anchor=north west] {$ a $};
\draw (-2.2,-0.1) node[anchor=north west] {$D$};
\draw (-1.3,-.1) node[anchor=north west] {$ a' $};
\draw (0.1,1.3) node[anchor=north west] {$  h$};
\draw (1.1,2.6) node[anchor=north west] {A};
\draw (1.7756097560975592,-0.25929046563192587) node[anchor=north west] {B};
\draw (4.265365853658532,-0.1) node[anchor=north west] {$C$};
\draw (2.4,2.6) node[anchor=north west] {A'};
\draw (-4.105365853658533,2.5) node[anchor=north west] {$ r$};
\draw (-4.169756097560972,0.5) node[anchor=north west] {$s$};
\draw [fill=black] (-2.7509090909090905,2.) circle (1.0pt);
\draw [fill=black] (-3.2418181818181813,0.) circle (1.0pt);
\draw [fill=black] (-0.4054545454545447,0.) circle (1.0pt);\draw [fill=black] (-2.1327272727272724,0.) circle (1.0pt);
\draw [fill=black] (1.9036363636363647,0.) circle (1.0pt);
\draw [fill=black] (4.3036363636363655,0.) circle (1.0pt);
\draw [fill=black] (1.4854545454545465,2.) circle (1.0pt);
\draw [fill=black] (2.7945454545454558,2.) circle (1.0pt);
\draw [color=black] (0.5581818181818191,2.)-- ++(-1.5pt,0 pt) -- ++(3.0pt,0 pt) ++(-1.5pt,-1.5pt) -- ++(0 pt,3.0pt);
\draw [color=black] (0.5581818181818191,0.)-- ++(-1.5pt,0 pt) -- ++(3.0pt,0 pt) ++(-1.5pt,-1.5pt) -- ++(0 pt,3.0pt);
\end{tikzpicture}
}
\end{center}
Nesta seção, usaremos a notação \((XYZ)\) para denotar a área do triângulo cujos vértices são \(X\), \(Y\) e \(Z\).

A figura mostra as paralelas \(r\) e \(s\) que estão a uma distância \(h\) entre si. Do lado esquerdo aparece o triângulo \(ABC\) dividido em duas partes pelo segmento \(AD\). A primeira propriedade diz respeito aos dois triângulos colados ABD e ADC.

\begin{observationtitle}{Propriedade 1}

\textit{Se dois triângulos possuem mesma altura então a razão entre suas áreas é a razão entre suas bases}.
\end{observationtitle}

De fato, a propriedade pode ser verificada calculando diretamente as áreas dos triângulos \(ABD\) e \(ADC\):
\begin{equation*}
\begin{split}\dfrac{(ABD)}{(ADC)}=\dfrac{\dfrac{a\cdot h}{2}}{\dfrac{a'\cdot h}{2}}=\dfrac{a}{a'}\end{split}
\end{equation*}
Do lado direito da figura acima aparecem os triângulos \(ABC\) e \(A'BC\) que mostram a segunda propriedade.

\begin{observationtitle}{Propriedade 2}

\textit{Dois triângulos de mesma base e mesma altura possuem mesma área}.
\end{observationtitle}

Uma vez que a área do triângulo depende apenas da base e da altura, a propriedade fica bastante evidente. Por outro lado, a interpretação que se dá à propriedade é que, se escolhemos um lado de um triângulo qualquer como base e movemos o terceiro vértice sobre uma paralela à base, o novo triângulo formado tem a mesma área do triângulo original.


\paragraph{Demonstrando o teorema}

Na primeira demonstração do Teorema de Tales, nossa estratégia envolvia o fato de que os segmentos determinados pelas paralelas sobre uma das transversais eram comensuráveis. Nossa nova estratégia não depende dessa condição e, por isso é válida também nos casos em que os segmentos citados não são comensuráveis.

A Hipótese do teorema diz que há um feixe de retas paralelas cortadas por duas trasnversais. Nada é dito sobre as posições das retas transversais e isso significa, em Matemática, que o teorema deve ser válido independentemente dessas posições. Além disso, como visto na demonstração do caso de segmentos comensuráveis, podemos fazer nossa demonstração, sem perder a generalidade do teorema, com um feixe de três retas paralelas, pois essa demonstração pode ser repetida para cada escolha de três retas paralelas do feixe.

O caso trivial do teorema ocorre quando as retas trasnversais são também paralelas, como na figura a seguir:
\begin{center}\begin{tikzpicture}
\draw [line width=2.pt] (0.,5.)-- (8.,5.);
\draw [line width=2.pt] (0.,3.)-- (8.,3.);
\draw [line width=2.pt] (0.,0.)-- (8.,0.);
\draw (2.,4.52) node[anchor=north west] {$a$};
\draw (1.28,1.86) node[anchor=north west] {$b$};
\draw (5.98,4.62) node[anchor=north west] {$a'$};
\draw (5.24,1.88) node[anchor=north west] {$b'$};
\draw [line width=2.pt] (3.,6.)-- (1.,-1.);
\draw [line width=2.pt] (7.,6.)-- (5.,-1.);
\draw (0.38,5.72) node[anchor=north west] {$r$};
\draw (0.28,3.64) node[anchor=north west] {$s$};
\draw (0.34,0.68) node[anchor=north west] {$t$};
\draw (2.26,5.6) node[anchor=north west] {$A$};
\draw (1.68,3.6) node[anchor=north west] {$B$};
\draw (0.8,0.6) node[anchor=north west] {$C$};
\draw (6.1,5.6) node[anchor=north west] {$A'$};
\draw (5.6,3.6) node[anchor=north west] {$B'$};
\draw (4.7,0.6) node[anchor=north west] {$C'$};
\draw [fill=black] (2.7142857142857144,5.) circle (2.0pt);
\draw [fill=black] (2.142857142857143,3.) circle (2.0pt);
\draw [fill=black] (1.2857142857142858,0.) circle (2.0pt);
\draw [fill=black] (6.714285714285714,5.) circle (2.0pt);
\draw [fill=black] (6.142857142857143,3.) circle (2.0pt);
\draw [fill=black] (5.285714285714286,0.) circle (2.0pt);
\end{tikzpicture}\end{center}
Nesse caso, \(ABB'A'\) e \(BCC'B'\) são paralelogramos e, por isso, \(a=a'\) e \(b=b'\). Portanto a tese \(\dfrac{a}{b}=\dfrac{a'}{b'}\) é verdadeira.

No caso em que as retas transversais não são paralelas, podemos reduzir a figura a uma mais simples, usando o caso trivial, conforme a figura a seguir:
\begin{center}\begin{tikzpicture}
\draw [line width=2.pt] (0.,5.)-- (8.,5.);
\draw [line width=2.pt] (0.,2.)-- (8.,2.);
\draw [line width=2.pt] (0.,0.)-- (8.,0.);
\draw (1.76,3.78) node[anchor=north west] {$a$};
\draw (1.16,1.52) node[anchor=north west] {$b$};
\draw (5.66,3.94) node[anchor=north west] {$a'$};
\draw (7.02,1.64) node[anchor=north west] {$b'$};
\draw [line width=2.pt] (3.,6.)-- (1.,-1.);
\draw (0.38,5.72) node[anchor=north west] {$r$};
\draw (0.28,3.64) node[anchor=north west] {$s$};
\draw (0.34,0.68) node[anchor=north west] {$t$};
\draw (2.26,5.6) node[anchor=north west] {$A$};
\draw (1.4,2.6) node[anchor=north west] {$B$};
\draw (0.8,0.6) node[anchor=north west] {$C$};
\draw (4.62,5.6) node[anchor=north west] {$A'$};
\draw (6.28,2.6) node[anchor=north west] {$B'$};
\draw (7.46,0.6) node[anchor=north west] {$C'$};
\draw [line width=2.pt] (4.,6.)-- (8.,-1.);
\draw [line width=2.pt,dashed] (2.7142857142857144,5.)-- (6.1692307692307695,-1.0461538461538469);
\draw (4.5,2.6) node[anchor=north west] {$B''$};
\draw (5.58,0.6) node[anchor=north west] {$C''$};
\draw (3.76,3.94) node[anchor=north west] {$a''$};
\draw (5.22,1.56) node[anchor=north west] {$b''$};
\draw [fill=black] (2.7142857142857144,5.) circle (2.0pt);
\draw [fill=black] (1.8571428571428572,2.) circle (2.0pt);
\draw [fill=black] (1.2857142857142858,0.) circle (2.0pt);
\draw [fill=black] (4.428571428571429,2.) circle (2.0pt);
\draw [fill=black] (5.571428571428571,0.) circle (2.0pt);
\draw [fill=black] (4.571428571428571,5.) circle (2.0pt);
\draw [fill=black] (6.285714285714286,2.) circle (2.0pt);
\draw [fill=black] (7.428571428571429,0.) circle (2.0pt);
\end{tikzpicture}\end{center}
Como \(a'=a''\) e \(b'=b''\), mostrar que \(\dfrac{a}{b}=\dfrac{a''}{b''}\) mostra também que \(\dfrac{a}{b}=\dfrac{a'}{b'}\), que é a tese de nosso teorema.

Portanto, para mostrar o caso geral, tomemos a figura a seguir, já simplificada, onde as retas \(r\), \(s\) e \(t\) são paralelas e as retas \(OA\) e \(OA'\) são as trasnversais.
\begin{center}\begin{tikzpicture}
\clip(-6.245361801353108,-0.994413927588976) rectangle (5.35363262685944,6.113077489908864);
\draw [line width=0.8pt,domain=-6.245361801353108:47.35363262685944] plot(\x,{(-0.-0.*\x)/1.});
\draw [line width=0.8pt,domain=-6.245361801353108:47.35363262685944] plot(\x,{(--2.-0.*\x)/1.});
\draw [line width=0.8pt,domain=-6.245361801353108:47.35363262685944] plot(\x,{(--5.-0.*\x)/1.});
\draw [line width=0.8pt] (-3.2809171581007854,5.)-- (-4.062560888816656,0.);
\draw [line width=0.8pt] (-3.2809171581007854,5.)-- (1.5913287633614788,0.);
\draw [line width=0.8pt] (-4.062560888816656,0.)-- (-0.3575696052234269,2.);
\draw [line width=0.8pt] (-3.749903396530308,2.)-- (1.5913287633614788,0.);
\draw (-5.7,5.4) node[anchor=north west] {$r$};
\draw (-5.7,2.6) node[anchor=north west] {$ s $};
\draw (-5.7,.6) node[anchor=north west] {$ t $};
\draw (-4,3.8) node[anchor=north west] {$ a $};
\draw (-4.452437691958675,1.5) node[anchor=north west] {$ b $};
\draw (-1.9,4) node[anchor=north west] {$ a' $};
\draw (0.7,1.5) node[anchor=north west] {$  b'$};
\draw [fill=black] (-3.2809171581007854,5.) circle (1.0pt);
\draw[color=black] (-3.4144289970461084,5.499708715642349) node {$O$};
\draw [fill=black] (-4.062560888816656,0.) circle (1.0pt);
\draw[color=black] (-4.389528074085187,-0.3194309376553625) node {$B$};
\draw [fill=black] (1.5913287633614788,0.) circle (1.0pt);
\draw[color=black] (1.7756144775167264,-0.41379536446559567) node {$B'$};
\draw [fill=black] (-3.749903396530308,2.) circle (1.0pt);
\draw[color=black] (-4.169344411527976,2.3542278219679105) node {$A$};
\draw [fill=black] (-0.3575696052234269,2.) circle (1.0pt);
\draw[color=black] (-0.11167405868794078,2.385682630904655) node {$A'$};
\end{tikzpicture}\end{center}
Com os elementos da figura acima observe inicialmente que os triângulos \(A'AB\) e \(AA'B'\)  possuem mesma área pois têm a mesma base \(AA'\) e mesma altura, pois as retas \(s\) e \(t\) são paralelas (\textbf{Propriedade 2}).

Agora, utilizando a \textbf{Propriedade 1}, temos que
\begin{equation*}
\begin{split}\dfrac{a}{b}=\dfrac{(A'OA)}{(A'AB)}=\dfrac{(AOA')}{(AA´B'}=\dfrac{a'}{b'}\end{split}
\end{equation*}
A igualdade \(\dfrac{a}{b}=\dfrac{a'}{b'}\) é nossa tese, o que encerra a demonstração.


\exercise

\begin{enumerate}
\item {} 
O ponto P divide o segmento \(AB\) na razão \(\frac{PA}{PB}=\frac{3}{7}\).
\begin{enumerate}
\item {} 
Qual a raz00ão \(\frac{AP}{AB}\)?

\item {} 
Se \(M\) é o ponto médio do segmento \(AB\) qual é a razão \(\frac{MP}{MB}\)?

\end{enumerate}

\item {} 
Na figura a seguir, \(PQ\) é paralelo a \(BC\). Se \(AP = 5\), \(PB = 4\) e \(QC = 6\), determine a medida do segmento  \(AQ\).
\begin{center}\begin{tikzpicture}
\draw [line width=0.8pt] (-2.632,2.054)-- (2.912,2.538);
\draw [line width=0.8pt] (2.912,2.538)-- (-2.038,4.738);
\draw [line width=0.8pt] (-2.038,4.738)-- (-2.632,2.054);
\draw [line width=0.8pt] (0.5048807378578668,2.327854667590766)-- (-2.2959056352295146,3.572648611185158);
\draw (-2.9598,2.0628) node[anchor=north west] {A};
\draw (-2.1854,5.1846) node[anchor=north west] {B};
\draw (2.945,2.4258) node[anchor=north west] {C};
\draw (-2.8,3.8536) node[anchor=north west] {P};
\draw (0.4766,2.1838) node[anchor=north west] {Q};
\draw [fill=black] (-2.632,2.054) circle (1.0pt);
\draw [fill=black] (2.912,2.538) circle (1.0pt);
\draw [fill=black] (-2.038,4.738) circle (1.0pt);
\draw [fill=black] (0.5048807378578668,2.327854667590766) circle (1.0pt);
\draw [fill=black] (-2.2959056352295146,3.572648611185158) circle (1.0pt);
\end{tikzpicture}\end{center}
\item {} 
Na figura a seguir, \(MN\) é paralelo a \(BC\) e \(NP\) é paralelo a \(AD\). Sabe-se que \(\frac{MA}{MB}=\frac{3}{5}\) e que \(CD = 12\) cm. Quanto mede o segmento \(DP\)?
\begin{center}\begin{tikzpicture}
\draw [line width=0.8pt] (-2.4274,3.2728)-- (2.7756,-0.3572);
\draw [line width=0.8pt] (2.7756,-0.3572)-- (-2.7662,0.5382);
\draw [line width=0.8pt] (-2.7662,0.5382)-- (-2.4274,3.2728);
\draw [line width=0.8pt] (0.5165376766007355,1.2188899930692538)-- (-2.619098267313536,1.725521128112171);
\draw (-2.7,3.9) node[anchor=north west] {A};
\draw (-3.04934,0.5) node[anchor=north west] {B};
\draw (2.78044,-0.3) node[anchor=north west] {C};
\draw (1.87536,2.021417999999998) node[anchor=north west] {P};
\draw [line width=0.8pt] (2.7756,-0.3572)-- (0.283,4.4344);
\draw [line width=0.8pt] (-2.4274,3.2728)-- (0.283,4.4344);
\draw [line width=0.8pt] (0.5165376766007355,1.2188899930692538)-- (1.6933515380924455,1.7232387908514153);
\draw (-3.3,1.9) node[anchor=north west] {M};
\draw (0.27816,1.0364779999999998) node[anchor=north west] {N};
\draw (0.1983,4.949617999999992) node[anchor=north west] {D};
\draw [fill=black] (-2.4274,3.2728) circle (1.0pt);
\draw [fill=black] (2.7756,-0.3572) circle (1.0pt);
\draw [fill=black] (-2.7662,0.5382) circle (1.0pt);
\draw [fill=black] (0.5165376766007355,1.2188899930692538) circle (1.0pt);
\draw [fill=black] (-2.619098267313536,1.725521128112171) circle (1.0pt);
\draw [fill=black] (0.283,4.4344) circle (1.0pt);
\draw [fill=black] (1.6933515380924455,1.7232387908514153) circle (1.0pt);
\end{tikzpicture}\end{center}
\item {} 
Na figura a seguir as retas \(r\) e \(s\) são paralelas e os segmentos \(CE\) e \(EG\) são paralelos. Sabe-se que \(AE = 2\), \(EB = 6\) e que o segmento \(ED\) tem 7 unidades a mais que \(CE\).
\begin{enumerate}
\item {} 
Quanto mede o segmento \(CD\)?

\item {} 
Sabe-se que \(FD = 16\). Trace o segmento \(CF\) e trace por \(E\) uma paralela a \(CF\) que encontra a reta \(s\) em \(G\). Quanto mede \(GD\)?

\end{enumerate}
\begin{center}\begin{tikzpicture}
\draw [line width=0.8pt] (-2.64,0.46)-- (4.8,1.94);
\draw [line width=0.8pt] (-2.54,3.04)-- (3.6596296600058285,4.273259663549546);
\draw [line width=0.8pt] (-1.6499582116298486,3.217051323492987)-- (4.130679271513971,1.806855554010844);
\draw [line width=0.8pt] (-0.5106335307872909,3.4436911793595173)-- (1.6378189906853886,1.3109639927707495);
\draw (-0.6721669353242998,4.102723170072155) node[anchor=north west] {A};
\draw (1.602133160970336,0.9879208642773318) node[anchor=north west] {B};
\draw (-1.9081995963539933,3.855516637866216) node[anchor=north west] {C};
\draw (4.1730810959120985,1.3834513158068331) node[anchor=north west] {D};
\draw (0.24249723383767327,3.286941613792558) node[anchor=north west] {E};
\draw (-1.3149039190597405,0.4687871466448614) node[anchor=north west] {F};
\draw (3.9505952169267533,4.547694928042843) node[anchor=north west] {r};
\draw (5.1124659182946655,2.1497915656452418) node[anchor=north west] {s};
\draw [fill=black] (-1.6499582116298486,3.217051323492987) circle (1.0pt);
\draw [fill=black] (4.130679271513971,1.806855554010844) circle (1.0pt);
\draw [fill=black] (-0.5106335307872909,3.4436911793595173) circle (1.0pt);
\draw [fill=black] (1.6378189906853886,1.3109639927707495) circle (1.0pt);
\draw [fill=black] (0.16328175964055147,2.774708529294657) circle (1.0pt);
\draw [fill=black] (-1.2743622966773251,0.7316591130265537) circle (1.0pt);
\end{tikzpicture}\end{center}
\item {} 
No triângulo \(ABC\) seja \(AD\) a bissetriz do ângulo \(\hat{A}\) como na figura a seguir. O teorema da bissetriz afirma que \(\frac{DB}{DC}=\frac{AB}{AC}\).

Prove esse teorema usando a sugestão a seguir.
\begin{itemize}
\item {} 
Trace por \(B\) e \(C\) paralelas à bissetriz.

\item {} 
A reta \(BA\) encontra a paralela traçada por \(C\) em \(E\).

\item {} 
Mostre que o triângulo \(ACE\) é isósceles.

\item {} 
Use o teorema de Tales para concluir o resultado.

\end{itemize}
\begin{center}\begin{tikzpicture}
\draw [shift={(1.24,2.22)},line width=0.8pt,color=session2,fill=session2,fill opacity=0.10000000149011612] (0,0) -- (-145.5816355209438:0.5668898395840698) arc (-145.5816355209438:-98.58725411426902:0.5668898395840698) -- cycle;   \draw [shift={(1.24,2.22)},line width=0.8pt,color=session2,fill=session2,fill opacity=0.10000000149011612] (0,0) -- (-98.58725411426903:0.5668898395840698) arc (-98.58725411426903:-51.59287270759427:0.5668898395840698) -- cycle;
\draw [line width=1.2pt] (1.24,2.22)-- (-2.,0.);
\draw [line width=1.2pt] (-2.,0.)-- (3.,0.);
\draw [line width=1.2pt] (3.,0.)-- (1.24,2.22);
\draw [line width=0.8pt] (1.24,2.22)-- (0.9047617009708229,0.);
\draw [line width=0.8pt,dash pattern=on 2pt off 2pt] (-1.4100628574084908,3.9066552370233936)-- (-2.130083863624808,-0.861435516417351);
\draw [line width=0.8pt,dash pattern=on 2pt off 2pt] (1.5063338058635691,3.983703761561156)-- (1.24,2.22);
\draw [line width=0.8pt,dash pattern=on 2pt off 2pt] (0.9047617009708229,0.)-- (0.77504260134413,-0.8590199926595905);
\draw [line width=0.8pt,dash pattern=on 2pt off 2pt] (3.6123080319008642,4.0547987349786965)-- (2.8770410481896134,-0.814253245555635);
\draw (0.7,2.8) node[anchor=north west] {A};
\draw (-2.6,-0.11250549465400854) node[anchor=north west] {B};
\draw (0.9449278317440784,-0.1691944786124154) node[anchor=north west] {D};
\draw (3.0991092221635435,-0.13140182264014416) node[anchor=north west] {C};
\draw (0.5,1.8) node[anchor=north west] {$\theta$};
\draw (1.2,1.7) node[anchor=north west] {$\theta$};
\draw [fill=black] (-2.,0.) circle (1.0pt);
\draw [fill=black] (3.,0.) circle (1.0pt);
\draw [fill=black] (1.24,2.22) circle (1.0pt);
\draw [fill=black] (0.9047617009708229,0.) circle (1.0pt);
\end{tikzpicture}\end{center}
\item {} 
Invente um triângulo \(ABC\) dando medidas para os seus três lados. A bissetriz do ângulo \(\hat{A}\) encontra o lado \(BC\) no ponto \(D\). Calcule os comprimentos dos segmentos \(BD\) e \(DC\).

\item {} 
No triângulo \(ABC\), \(AB = 9\), \(BC = 10\) e \(AC = 6\). A bissetriz do ângulo \(\hat{A}\) encontra o lado \(BC\) no ponto \(D\) e o ponto \(J\) é o incentro do triângulo. Calcule a razão \(\frac{JA}{JD}\).

\item {} 
(CEFE-MG - 2017) A figura a seguir é um esquema representativo de um eclipse lunar em que a Lua, a Terra e o Sol estão representados pelas circunferências de centros \(C_1\), \(C_2\) e \(C_3\) respectivamente, que se encontram alinhados. Considera-se que a distância entre os centros da Terra e do Sol é 400 vezes a distância entre os centros da Terra e da Lua e que a distância do ponto \(T\)  na superfície da Terra ao ponto \(S\) na superfície do Sol, como representados na figura, é de 150 milhões de quilômetros.
\begin{center}
\scalebox{.75}
{
\begin{tikzpicture}
\draw [line width=2.pt] (1.2467707033296944,-1.2335414066593893) circle (0.766);
\draw [line width=2.pt] (3.5901699437494745,-0.6803398874989484) circle (1.32);
\draw [line width=2.pt] (8.813624739188983,0.5527505216220345) circle (2.56);
\draw [line width=1.pt] (-2.,-2.)-- (8.81,0.55);
\draw [line width=1.pt] (8.8,0.56)-- (7.672,2.836);
\draw [line width=1.pt] (3.5901699437494745,-0.6803398874989484)-- (3.,0.5);
\draw [line width=1.pt] (1.2467707033296944,-1.2335414066593893)-- (0.904,-0.548);
\draw [line width=1.pt] (-2.,-2.)-- (7.672,2.836);
\draw (1.16,-1.14) node[anchor=north west] {$C_1$};
\draw (3.72,-0.6) node[anchor=north west] {$C_2$};
\draw (8.88,0.7) node[anchor=north west] {$C_3$};
\draw (0.54,0.18) node[anchor=north west] {$L$};
\draw (2.8,1.28) node[anchor=north west] {$T$};
\draw (7.42,3.56) node[anchor=north west] {$S$};
\end{tikzpicture}
}
\end{center}
Sabendo-se que os segmentos de reta \(C_1L\), \(C_2T\) e \(C_3S\) são paralelos, quanto mede a distância do ponto \(L\)  representado na superfície da Lua, ao ponto \(T\)  na superfície da Terra?

\end{enumerate}


% \ifnum\aluno=1
% \clearpage
% \else
% \notasfinais
% \fi


\bibliography{../Bibliografia/probabilidade1_bibliografia.bib}

\nocite{*}

% %!TEX root = ../aluno.tex

\ifnum\aluno=1
\renewcommand\chapterillustration{abertura-semelhanca}
\else
\renewcommand\chapterillustration{abertura-semelhanca-professor}
\fi
\renewcommand\chapterwhat{São estudadas a semelhança de figuras geométricas, os casos de semelhança de triângulos e polígonos, e suas aplicações.}

\renewcommand\chapterbecause{Em primeiro lugar, o conceito de semelhança está presente em diversos contextos e sua compreensão permite compreender melhor o mundo que vivemos. Por outro lado, a semelhança de triângulos é um instrumento importante para obter propriedades métricas e relações entre elementos de polígonos. Por fim, a semelhança de triângulos é uma ferramenta muito útil na resolução de problemas de geometria, tanto plana quanto no espaço.}

\chapter{Semelhança}

\mbox{}\thispagestyle{empty}\clearpage

\thispagestyle{empty}

\begin{center}
Projeto: LIVRO ABERTO DE MATEMÁTICA

\noindent \begin{tabular}{lcccr}
\includegraphics[scale=.15]{impa}& \quad\quad& \includegraphics[width=3cm]{logo} & \quad\quad& \includegraphics[scale=.24]{obmep} 
\end{tabular}
\end{center}

\vspace*{.3cm}

Cadastre-se como colaborador no site do projeto: \url{umlivroaberto.org}

Versão digital do capítulo:

\url{https://www.umlivroaberto.org/BookCloud/Volume_1/master/view/GE202.html}

% \begin{center}
%   \includegraphics[width=2cm]{canvas}
% \end{center}

\begin{tabular}{p{.15\textwidth}p{.7\textwidth}}
Título: & Semelhança\\
\\
Ano/ Versão: & 2020 / versão 1.2 de \today\\
\\
Editora & Instituto Nacional de Matem\'atica Pura e Aplicada (IMPA-OS)\\
\\
Realização:& Olimp\'iada Brasileira de Matem\'atica das Escolas P\'ublicas (OBMEP)\\
\\
Produção: & Associação Livro Aberto\\
\\
Coordenação: & Fabio Simas e Augusto Teixeira (livroaberto@impa.br)\\
\\
Autores:  & Eduardo Wagner\\
          & Marcos Paulo \\

\\
Revisão: &  Cydara Ripoll  \\
          &  Letícia Rangel \\
\\
Design: & Andreza Moreira (Tangentes Design) \\
\\
  Ilustrações: & Miller  Guglielmo \\ 
\\
Gráficos: & Beatriz Cabral e Tarso Caldas (Licenciandos da UNIRIO)\\
\\
  Capa: & Foto de Pradeep Gopal, no Unsplash \\
        & https://unsplash.com/photos/6uJDEQx-cho \\

\end{tabular}

\begin{figure}[b]
\begin{minipage}[l]{5cm}
\centering

{\large Licença:}

  \includegraphics[width=3.5cm]{cc-by-sa1}
\end{minipage}\hfill
\begin{minipage}[c]{5cm}
\centering
{\large Desenvolvido por}

\includegraphics[width=2.5cm]{logo-associacao.jpg}
\end{minipage}
\begin{minipage}[r]{5cm}
\centering

{\large Patrocínio:}
  \vspace{1em}
  \includegraphics[width=3.5cm]{itau}
\end{minipage}
\end{figure}

\mainmatter

\begin{apresentacao}{Introdução}
A Matemática elementar caminha apoiada em dois pilares: os números e as formas. Neste capítulo de Semelhança reunimos a proporcionalidade (números) com as figuras geométricas (formas). Os alunos possuem internalizado o conceito de Semelhança desde o final do Ensino Fundamental, quando praticaram atividades de ampliar figuras dadas em um reticulado, mas além disso, sempre tiveram contato com miniaturas ou outros brinquedos representando modelos reduzidos de objetos reais.

Como indica a \href{http://historiadabncc.mec.gov.br/documentos/bncc-2versao.revista.pdf}{Base Nacional Comum Curricular}  (BNCC) para o primeiro ano do Ensino Médio, a Semelhança servirá de base para o desenvolvimento de dois temas imediatos:
\begin{enumerate}
\item {} 
O teorema de Pitágoras e suas aplicações na geometria plana e na geometria espacial e,

\item {} 
as razões trigonométricas tanto no triângulo retângulo quanto em triângulos quaisquer e suas diversas aplicações.

\end{enumerate}

Esses dois objetivos estão bem claro nas duas habilidades da BNCC e mostramos a seguir.

\begin{habilities}{EM12MT02}

Resolver e elaborar problemas utilizando a semelhança de triângulos e o teorema de Pitágoras, incluindo aqueles que envolvem o cálculo das medidas de diagonais de prismas, de altura de pirâmides, e aplicar esse conhecimento em situações relacionadas ao mundo do trabalho

\tcbsubtitle{EM12MT03}

Utilizar a noção de semelhança para compreender as razões trigonométricas no triângulo retângulo, suas relações em triângulos quaisquer e aplicá-las em situações como o cálculo de medidas inacessíveis, entre outras
\end{habilities}

Deve-se destacar que para a exploração e o desenvolvimento das habilidades acima será necessário compreender e saber utilizar, apenas, a semelhança de triângulos. Entretanto, o capítulo apresenta uma definição geral de semelhança por duas importantes razões. Em primeiro lugar porque o estudante reconhece, desde muito jovem, a ampliação ou redução de imagens diversas (retratos, por exemplo) e este é o momento adequado para mostrar a parte da matemática que explica essa transformação. Em segundo lugar, porque o conceito geral será necessário em outros momentos da sua aprendizagem. No momento de estudar, por exemplo, a área do círculo, o professor diz (ou o livro diz) que duas circunferências quaisquer são semelhantes. O aluno sente que isso é verdade, mas não consegue explicar se não conhece a definição geral de semelhança. Por outro lado, na geometria do espaço o livro vai dizer que se cortarmos um cone por um plano paralelo à sua base, obteremos um cone menor semelhante ao primeiro. E o que significa que um cone seja semelhante a outro? Assim, percebemos que não é suficiente que o aluno conheça apenas a semelhança de triângulos. É necessário, portanto, que ele conheça o que realmente é uma Semelhança.

O capítulo se inicia com uma atividade que recorda o que o aluno fez na escola quando muito mais jovem: a ampliação de um desenho sobre um fundo quadriculado. Mesmo sendo uma atividade algo infantil, o conceito de semelhança aparece e uma interessante questão é abordada: qual é a diferença entre “semelhante” e “parecido”. Na linguagem comum, usamos essas palavras praticamente da mesma forma, mas na Matemática, a primeira tem definição precisa e a outra não.

A seção Explorando mostra exemplos de figuras semelhantes e recomenda que os alunos vejam um interessante desenho da Disney que aborda diversos aspectos da Matemática na vida diária, inclusive a Semelhança.

Em seguida, aparece a Atividade Plantas Baixas, que é uma atividade motivadora. Nessa atividade não há o objetivo que o aluno consiga responder a todas as perguntas, mas sim que o aluno se sinta provocado a aprender o material do capítulo para poder respondê-las. Essa atividade será retomada adiante quando o aluno terá então as ferramentas necessárias para responder as questões com toda segurança.

No Organizando as Ideias aparece de início a definição geral de Semelhança com vários exemplos mostrando o que é a razão de semelhança entre duas figuras e o que significa o fator de ampliação de uma figura para outra. A importância da semelhança de triângulos é enfatizada e as atividades seguintes procuram fazer com que o aluno compreenda bem o conceito de Semelhança o teorema central da semelhança de triângulos.

A Atividade Teorema Central da Semelhança de Triângulos tem grande importância pois nela se pede ao aluno que leia um texto contendo a demonstração de um teorema. Precisamos lembrar que, até esse ponto do aprendizado, o professor (em geral) teve o costume de explicar aos alunos toda a matéria e os alunos, é claro, se acostumaram com esse procedimento. Agora, pela primeira vez, se solicita ao aluno que leia um pequeno texto, compreenda e aprenda sozinho, o que representa um grande salto no processo de aprendizagem. Entendemos que uma das coisas mais importantes que o professor pode ensinar a seus alunos é ensiná-los a aprender sozinhos.

As atividades seguintes conduzem o aluno a aplicar seus conhecimentos em situações bem diferentes, incluindo aquelas em que ele terá que interferir na figura para criar os triângulos semelhantes que vão possibilitar a resolver determinada questão.

O capítulo termina com os Exercícios, alguns de manipulação apenas dos triângulos semelhantes, mas outros bem contextualizados, incluindo um de cálculo de distância inacessível, contemplando o que solicita a \textbf{EM12MT02}.
\end{apresentacao}

\explore{Figuras Semelhantes}


Quando nos afastamos de uma figura, ou quando damos um zoom em uma foto com a câmera do celular, a forma e as proporções são mantidas e apenas o seu tamanho muda. É disso que vamos falar no início deste capítulo de Semelhança. Além do conceito de semelhança que permite entender bem o que seja ampliar ou reduzir uma imagem, a semelhança de triângulos também se mostrará uma ferramenta muito poderosa para resolver muitos problemas de geometria.


Você consegue reproduzir a imagem abaixo dentro de um quadrado com $5$ cm de lado?

\begin{texto}
{
\begin{observation}{}
Para ampliar uma figura, um método prático é o de sobrepor à figura uma malha regular. Ampliando a malha os pontos de referência da figura sobre a malha inicial poderão ser transferidos sem dificuldade para a nova malha ampliada e assim, todos os delalhes da figura inicial poderão aparecer na nova figura em tamanho maior.
\end{observation}
}
\end{texto}

\begin{figure}[H]
\centering

\noindent\includegraphics[width=200bp]{{malha}.png}
\end{figure}



Em princípio, para ampliar uma figura, qualquer malha regular serve. Entretanto, a malha mais simples é a quadriculada, que já é familiar aos alunos desde o ensino fundamental e, por essa razão, o Cebolinha já está com a malha quadriculada proposta, faltando ao aluno, apenas completar as linhas sobre o desenho para poder transportar os pontos importantes da figura para a nova malha ampliada.
Essa atividade inicial é, naturalmente, opcional. Há alunos que gostariam de fazer essa experiência e, certamente, outros que possam considerar infantil tal proposta. Entretanto, o objetivo que é o de introduzir a noção de ampliação terá sido realizado.

A figura que você pretende criar deverá ser \textit{semelhante} à que está acima, mas se a sua habilidade não for muito boa, ela será apenas \textit{parecida} com a original.

Na linguagem comum usamos essas duas palavras com o mesmo significado, mas em matemática não. A palavra semelhante tem significado preciso e é isso o que veremos neste capítulo.

A semelhança é um conceito que está presente em inúmeras situações da nossa vida. Este conceito está diretamente ligado à percepção de figuras que são essencialmente a mesma, mas apresentadas em tamanhos e posições diferentes. Essa é a essência do conceito: a manutenção da forma com apresentação do objeto em tamanhos diferentes.

Não há nenhuma diferença na abordagem da semelhança no mundo 2D (plano) ou no mundo 3D (espacial); tudo funciona exatamente da mesma maneira. Entretanto, neste capítulo, vamos desenvolver a semelhança em figuras planas.

A semelhança é um conceito muito interessante e bastante intuitivo, pois está ligado às ideias de ampliar ou reduzir alguma coisa, ou alguma imagem. Por exemplo, a seguir, você vê três figuras semelhantes.

\clearpage
\begin{sugestions}{Donald no País da Matemágica}
{
O vídeo tem 27 min e aborda diversos aspectos interessantes da Matemática elementar. Há muita coisa para comentar com os alunos, mas você pode selecionar apenas o que se relaciona com a semelhança. Como os alunos ficarão curiosos, em algum momento o vídeo inteiro poderá ser comentado.
}{1}{2}
\end{sugestions}
\begin{objectives}{Plantas Baixas}
{
\begin{itemize}
\item {} 
Reconhecer ampliação ou redução de um objeto.

\item {} 
Estimar a relação entre as medidas de duas figuras semelhantes.

\end{itemize}
}{1}{2}
\end{objectives}
\begin{sugestions}{Plantas Baixas}
{
Esta atividade visa principalmente despertar o aluno para as informações que ele pode obter a partir do conceito que será abordado no capítulo. É fundamental que o aluno possa experimentar sua intuição a respeito do tema.

\begin{itemize}
\item {} 
A ideia de que a planta de uma casa mostra um desenho reduzido da situação real deve ser abordada de forma a explorar a intuição dos alunos. Inicialmente, não diga nada, não explique nada; deixe que eles descubram sozinhos o conceito de escala de um desenho.

\item {} 
Depois que os alunos estiverem na direção certa, você deve explicar o conceito de proporcionalidade.

\item {} 
A regra de três é a ferrramenta da proporcionalidade. Conhecendo três termos de uma proporção podemos calcular o quarto.

\item {} 
A atividade a seguir, vai mostrar a necessidade de termos ferramentas adequadas para calcular certas distâncias.

\end{itemize}
}{1}{2}
\end{sugestions}
\clearmargin
\begin{answer}{Plantas Baixas}
{
\begin{enumerate}
\item {} 
Espera-se apenas que o aluno diga que não é possível calcular a área do quarto 2, mas há sempre a possibilidade do aluno tentar aproximar a forma pentagonal do quarto a um retângulo e isso pode levá-lo futuramente a problemas mais sérios. Fique atento.

\item {} 
Aqui é uma boa oportunidade para falar em escala, proporcionalidade e regra de 3. A resposta esperada é $6{,}75$ m

\item {} 
A resposta esperada é $1{,}72$ m

\end{enumerate}

Como essa atividade é introdutória ao capítulo não há respostas precisas. Para qualquer item, uma resposta  “não sei”  é expressiva. Ela mostrará que o aluno que deu essa resposta não traz do ensino fundamental algum conhecimento relacionado com essa situação, e isso é importante. Entretanto, outros poderão, mesmo sem conhecer a proporcionalidade, estimar as medidas usando apenas a intuição. Isso também é importante e uma avaliação visual será boa se no item \titem{b)} o aluno responder  “um pouco menos que $7$ m”   e no item \titem{c)} responder  “um pouco menos que $2$ m”.
}{1}
\end{answer}
\begin{figure}[H]
\centering
\capstart

\noindent\includegraphics[width=.9\linewidth]{{figuras_semelhantes}.png}
\caption{Skatista}
\label{skatista}
\end{figure}

As figuras semelhantes mostram a mesma “forma”, mas nada diz quanto ao tamanho, ou mesmo com a disposição ou arrumação relativa das figuras. Isso faz com que, apesar do conceito ser intuitivo, a definição não seja muito fácil pois deverá ser precisa.

Para explorar mais, recomendamos um filme muito antigo. Ele se chama \textit{Donald no país da Matemágica} e mostra diversas situações em que a Matemática está presente sem que se perceba. Você vai ver, inclusive, qual é o retângulo mais bonito de todos e o que isso tem a ver com o tema do nosso capítulo: Semelhança.

Veja o filme \href{https://www.youtube.com/watch?v=wbftu093Yqk}{Donald no  País da Matemágica}

\begin{task}{Plantas baixas}

A figura a seguir mostra a planta de uma casa e as medidas indicadas no desenho mostram as dimensões reais em metros. Entretanto, Fabio, uma pessoa que gostaria de ter mais informações sobre essa casa, mediu com sua régua a largura da parede do fundo da casa e, desprezando a espessura das paredes, encontrou $8$ cm, colocando essa informação no desenho.

As perguntas a seguir são importantes para o curioso Fabio. Se você não souber responder, não se preocupe, pois elas estão nessa atividade para que você perceba o que vamos desenvolver neste capítulo. Essa casa será retomada adiante.

\begin{figure}[H]
\centering

\noindent\includegraphics[width=250bp]{{planta_1}.png}
\end{figure}

\begin{enumerate}
\item {} 
O desenho fornece informações suficientes para que se calcule a área do Quarto 2?

\item {} 
Com a régua Fabio mediu a distância entre a porta de entrada e a porta da cozinha e encontrou $9$ cm. Na realidade qual é essa distância?

\item {} 
Fabio mediu também o comprimento da mesa da sala de jantar e encontrou $2{,}3$ cm, Na realidade qual é essa medida?

\end{enumerate}
\end{task}



\needspace{.4\textheight}
\arrange{Figuras Semelhantes}

\paragraph{O que é semelhança para a Matemática?}

Na atividade anterior percebemos que a planta de uma casa é um modelo reduzido da situação real e isso significa que as proporções entre as medidas são mantidas. Dizemos então que a planta da casa e o piso da casa são semelhantes.
Para tornar o conceito preciso precisamos de uma definição.
\begin{observationtitle}{Figuras semelhantes}
Duas figuras \(F\) e \(F'\) são semelhantes quando existe uma correspondência biunívoca entre os pontos de uma e os pontos de outra, de forma que, para quaisquer pontos \(X\) e \(Y\) da figura \(F\) e seus correspondentes \(X'\) e \(Y'\) da figura \(F'\) tem-se que a razão \(\dfrac{XY}{X'Y'}\)   é constante.
\end{observationtitle}

\begin{wrapfigure}[11]{r}{.4\linewidth}

\resizebox{\linewidth}{!}
{
\begin{tikzpicture}
\draw [rotate around={0.:(4.5,4.)},line width=3.6pt,color=\currentcolor!80] (4.5,4.) ellipse (1.8251407699364404cm and 1.0397782600555694cm);]
\draw [rotate around={-45.:(8.629881130634992,5.065307896443685)},line width=3.6pt,color=\currentcolor!80] (8.629881130634992,5.065307896443685) ellipse (2.4274372240154656cm and 1.3829050858739074cm);
\draw [line width=2.pt] (3.96,4.28)-- (5.16,3.6);
\draw [line width=2.pt] (8.385363605700684,5.836478552005733)-- (8.874398655569301,4.068428756326888);
\draw (7.98,7.88) node[anchor=north west] {$F^\prime$};
\draw (5.,4.2) node[anchor=north west] {$Y$};
\draw (3.28,4.58) node[anchor=north west] {$X$};
\draw (8.3,6.5) node[anchor=north west] {$X^\prime$};
\draw (9.0,4.68) node[anchor=north west] {$Y^\prime$};
\draw (2.44,5.24) node[anchor=north west] {$F$};
\draw [fill=black] (3.96,4.28) circle (2.5pt);
\draw [fill=black] (5.16,3.6) circle (2.5pt);
\draw [fill=black] (8.385363605700684,5.836478552005733) circle (2.5pt);
\draw [fill=black] (8.874398655569301,4.068428756326888) circle (2.5pt);
\end{tikzpicture}
}
\caption{Figuras Semelhantes}
\end{wrapfigure}
Vamos entender bem essa definição. Não se impressione se ela lhe parece difícil.

Uma correspondência biunívoca (ou uma bijeção) entre \(F\) e \(F^\prime\) é uma função onde  cada ponto de \(F\) tem um correspondente em \(F'\) e, reciprocamente, cada elemento de \(F'\) tem seu correspondente em \(F\).

Volte para a \Fref{skatista} e veja novamente as duas primeiras representações da skatista. Escolha um ponto da primeira figura, uma ponta do skate, por exemplo. Certamente você saberá encontrar esse mesmo ponto na segunda figura. Por outro lado, se você assinalar qualquer outro ponto da segunda figura, você também saberá localizar onde está o ponto correspondente na primeira figura. Uma vez que você assinalou dois pontos de uma das figuras e seus correspondentes na segunda figura, você pode determinar as distâncias entre esses pares de pontos. A função que relaciona os pontos das duas figuras chama-se uma semelhança se a razão entre essas distâncias for sempre a mesma, \textit{quaisquer que sejam os pontos escolhidos}.

\begin{observationtitle}{Razão de semelhança e fator de ampliação/redução}
Em uma semelhança entre \(F\) e \(F'\), se temos \(\dfrac{XY}{X'Y'}=k\), dizemos que a \textit{razão de semelhança} de \(F\) para \(F'\) é \(k\).

Naturalmente que \(\dfrac{X'Y'}{XY}=\dfrac{1}{k}\)  e assim dizemos que a \textit{razão de semelhança} de \(F'\) para \(F\) é \(\dfrac{1}{k}\).

Fazendo agora \(\alpha=\dfrac{1}{k}\) temos que \(X’Y’=\alpha\cdot XY\)  e dizemos que \(\alpha\)  é o \textit{fator de ampliação} de \(F\) para \(F'\).
\end{observationtitle}

\begin{example}{pequeno e peixe grande}

Na figura a seguir, o fator de ampliação é $2{,}5$. Isso significa que todas as distâncias entre pontos do peixe menor aparecem no peixe maior, multiplicadas por $2{,}5$.

\begin{figure}[H]
\centering

\noindent\includegraphics[width=.75\linewidth]{{peixe}.png}
\end{figure}

Observe que, se no peixe menor alguma distância entre dois pontos mede 2 unidades, então a distância entre os pontos correspondentes no peixe maior será de \(2,5\times 2 = 5\) unidades. Assim, a razão de semelhança do peixe menor para o maior será \(\dfrac{2}{5}\), enquanto que a razão de semelhança do peixe maior para o menor será \(\dfrac{5}{2}\).
\end{example}

\begin{sugestions}{Você Sabia?}
{Os alunos devem observar que existe aqui uma correspondência biunívoca entre as duas figuras. Para cada ponto assinalado em uma delas é possível encontrar o ponto correspondente na outra. Entretanto essas figuras não são semelhantes porque as razões entre distâncias não são as mesmas para todos os pares de pontos}
{1}{1}
\end{sugestions}

\begin{knowledge}

\paragraph{Comprove você mesmo}

Não podemos negar que as figuras abaixo são parecidas, mas devemos reconhecer que não são semelhantes pois as proporções não são todas mantidas.

\begin{figure}[H]
\centering

\noindent\includegraphics[width=.575\linewidth]{{nao-semelhantes_1}.png}
\end{figure}



Hoje em dia, os softwares que fazem reconhecimento de faces, utilizam uma definição matemática para a palavra  “parecido”. É por isso que, em fotos do Facebook, o software permite reconhecer pessoas já identificadas em fotos anteriores. Porém nada disso seria possível sem o primeiro passo, que é a semelhança.
\end{knowledge}
\cleardoublepage

\def\currentcolor{session1}
\begin{objectives}{Triângulos Semelhantes}
{\begin{itemize}
\item {} 
Utilizar a definição de semelhança no lugar da intuição;

\item {} 
Concluir que, no caso de semelhança de triângulos, a definição é equivalente ao caso $LLL$ de semelhança de triângulos

\end{itemize}}
{1}{1}
\end{objectives}
\marginpar{\vspace{-2.5em}}
\begin{sugestions}{Triângulos Semelhantes}
{A atividade a seguir, pede que os alunos verifiquem se os triângulos são ou não semelhantes. Em princípio, os alunos podem ficar confusos tentando mostrar que todos os pontos no interior da região triângular atendem o que foi pedido na definição de semelhança. Vale lembrar que um triângulo fica definido por três pontos não colineares e, portanto, basta verificar que as razões entre distâncias \(AB\), \(BC\) e \(AC\) e \(A'B'\), \(B'C'\) e \(A'C'\), respectivamente, são iguais. Desse modo, a figura formada pelos pontos \(A\), \(B\) e \(C\) é semelhante à figura formada pelos pontos \(A'\), \(B'\) e \(C'\).

Portanto, atender à definião de semelhança, no caso de triângulos, consiste no caso de semelhança \(LLL\). Um caso de semelhança de triângulos é um conjunto de condições mínimas que garantem a semelhança dos triângulos envolvidos. Apenas os triângulos possuem casos de semelhança simples o sufuciente para serem estudados e conhecidos.}
{1}{1}
\end{sugestions}
\marginpar{\vspace{-1.5em}}
\begin{answer}{Triângulos Semelhantes}
{\begin{enumerate}[topsep=0pt]
\item {} 
Nomeação do professor: \(A\), \(B\) e \(C\) opostos, respectivamente aos lados de comprimentos $6$, $7{,}2$ e $8$ e para $\mathcal\{T\}_1{}$ usaremos as letras \(P\), \(Q\) e \(R\), respectivamente opostos aos lados de comprimentos $9$, $10{,}8$ e $12$.

\item {} 
\((A, P)\), \((B, Q)\) e \((C, R)\)

\item {} 
\((AB, PQ)\), \((BC, QR)\) e \((AC, PR)\)

\item {} 
Calculando as razões:
\begin{itemize}[topsep=0pt]
\item {} 
\(\dfrac{6}{9}=\dfrac{2}{3}\)

\item {} 
\(\dfrac{8}{12}=\dfrac{2}{3}\)

\item {} 
\(\dfrac{7{,}2}{10,{,}8}=\dfrac{2}{3}\)

\end{itemize}

\end{enumerate}

A razão \(\dfrac{3}{2}\) também é uma resposta aceitável. Note que \(\frac{6}{9}=\frac{8}{12}=\frac{7,2}{10,8}=\frac{2}{3}\) e os triângulos são semelhantes na razão \(\frac{2}{3}\) pelo caso \(LLL\)  de semelhança de triângulos.}
{0}
\end{answer}
\explore{Triângulos Semelhantes}



A semelhança de triângulos é uma ferramenta poderosa para resolver inúmeros problemas de geometria. Isso ocorre porque o triângulo tem uma situação especial no que estamos estudando: ao contrário dos outros polígonos, é muito fácil reconhecer quando dois triângulos são semelhantes.

De fato, para triângulos, podemos definir conjuntos de condições mínimas para que possamos afirmar que dois triângulos são semelhantes. Esses conjuntos de condições mínimas são chamados \textit{casos de semelhança de triângulos}.

A própria definição de semelhança entre duas figuras, no caso dessas figuras serem triângulos, resume-se a verificar a proporcionalidade entre seus lados correspondentes. Este é o chamado caso “caso LLL” (lado-lado-lado) de semelhança de triângulos.

\begin{task}{Triângulos semelhantes}

A figura a seguir mostra dois triângulos \(\mathcal{T}_1\) e \(\mathcal{T}_2\).
\begin{center}\begin{tikzpicture}
\begin{scope}[scale=.5]
\fill[line width=1.2pt,color=\currentcolor!80,fill=\currentcolor!80,fill opacity=0.20000000298023224] (-2.4026846239814494,9.01432812261539) -- (-3.1006389948749313,3.0550614146931245) -- (4.239720652539582,6.236117881750678) -- cycle;
\fill[line width=1.2pt,color=cor1,fill=cor1,fill opacity=0.20000000298023224] (7.032280832257959,12.718092698860787) -- (12.293364416514931,1.9328713511339897) -- (17.311287593236727,9.404179602873993) -- cycle;
\draw [line width=0.8pt] (-3.1006389948749313,3.0550614146931245)-- (4.239720652539582,6.236117881750678);
\draw [line width=1.2pt,color=\currentcolor!80] (-2.4026846239814494,9.01432812261539)-- (-3.1006389948749313,3.0550614146931245);
\draw [line width=1.2pt,color=\currentcolor!80] (-3.1006389948749313,3.0550614146931245)-- (4.239720652539582,6.236117881750678);
\draw [line width=1.2pt,color=\currentcolor!80] (4.239720652539582,6.236117881750678)-- (-2.4026846239814494,9.01432812261539);
\draw [line width=1.2pt,color=cor1] (7.032280832257959,12.718092698860787)-- (12.293364416514931,1.9328713511339897);
\draw [line width=1.2pt,color=cor1] (12.293364416514931,1.9328713511339897)-- (17.311287593236727,9.404179602873993);
\draw [line width=1.2pt,color=cor1] (17.311287593236727,9.404179602873993)-- (7.032280832257959,12.718092698860787);
\draw (0.8,4.5) node[anchor=north west] {$8$};
\draw (-1.2,6.7) node[anchor=north west] {$\mathcal{T}_1$};
\draw (-3.6,6.5) node[anchor=north west] {$6$};
\draw (0.5,8.7) node[anchor=north west] {$7{,}2$};
\draw (8.6,7.5) node[anchor=north west] {$12$};
\draw (15.3,5.8) node[anchor=north west] {$9$};
\draw (11.5,8.3) node[anchor=north west] {$\mathcal{T}_2$};
\draw (12.2,12.1) node[anchor=north west] {$10{,}8$};
\end{scope}
\end{tikzpicture}\end{center}\begin{enumerate}
\item {} 
Atribua a cada vértice de \(\mathcal{T}_1\) e \(\mathcal{T}_2\) uma letra maiúscula.

\item {} 
O pares de vértices chamados de correspondentes são: os vértices opostos aos menores lados dos dois triângulos, os vértices opostos aos maiores lados dos dois triângulos e os dois vértices restantes. Identifique, usando a notação $(\makebox[.3cm]{\hrulefill},\makebox[.3cm]{\hrulefill})$ os pares de vértices correspondentes nos dois triângulos.

\item {} 
São chamados lados \textit{homólogos} os pares de lados que estão compreendidos entre vértices correspondentes de dois triângulos. Liste os pares de lados homólogos, segundo sua escolha no item anterior.

\item {} 
Determine a razão entre cada par de lados homólogos.

\item {} 
Compare as razões encontradas. Esse resultado garante que \(\mathcal{T}_1\) e \(\mathcal{T}_2\) são semelhantes? Justifique sua resposta.

\end{enumerate}
\end{task}




\arrange{Triângulos Semelhantes}

\begin{sugestions}{Teorema central da semelhança de triângulos}
{O aluno deve acompanhar a demonstração de mais um caso de semelhança de triângulos que facilita o reconhecimento de triângulos semelhantes. Trata-se do caso \textit{Ângulo - Ângulo} ou, simplesmente \(AA\).

\begin{itemize}
\item {} 
É importante que, se o aluno tem uma definição, ele deve usá-la.

\item {} 
Como o aluno conhece a definição de figuras semelhantes então ele deve entender como a definição geral se aplica a triângulos semelhantes.

\item {} 
O aluno deve entender bem o que é dado e onde se pretende chegar. Em seguida, deve ser levado a perceber a beleza do resultado, que permite reconhecer facilmente quando dois triângulos são semelhantes.

\item {} 
Dizer que dois ângulos de um triângulo são, respectivamente, congruentes a dois ângulos de outro triângulo é o mesmo que dizer que os três ângulos do primeiro triângulo são, respectivamente, congruentes aos três ângulos do segundo triângulo. No enunciado do teorema, são citados apenas dois ângulos, mas na figura os três ângulos são congruentes a seus correspondentes. Você deve certificar-se se isso foi percebido pelos seus alunos.
\end{itemize}}
{1}{2}
\end{sugestions}
\clearmargin
\begin{sugestions}{Recíproca do Teorema Central da Semelhança}
{Para demonstrar a recíproca do teorema central considere dois triângulos semelhantes  \(ABC\) e{}`A’B’C’{\color{red}\bfseries{}{}`}. Isso significa que vale a relação (1):
\begin{equation*}
\begin{split}\frac{A'B'}{AB}=\frac{A'C'}{AC}=\frac{B'C'}{BC}\end{split}
\end{equation*}
Considere ainda que esses triângulos não sejam congruentes e que, sem perda de generalidade, \(A'B'C'\) seja o  “menor” deles.
\begin{center}\begin{tikzpicture}
\begin{scope}[scale=.6]
\draw[fill](-2,3)circle(1pt)node[above]{$A'$};
\draw[fill](0,0)circle(1pt)node[left]{$B'$};
\draw[fill](3,3)circle(1pt)node[right]{$C'$};
\draw(-2,3)--(0,0)--(3,3)--cycle;
\draw[fill](6.7,3.5)circle(1pt)node[above]{$A$};
\draw[fill](3,0)circle(1pt)node[left]{$B$};
\draw[fill](7.9,-3.4)circle(1pt)node[right]{$C$};
\draw(6.7,3.5)--(3,0)--(7.9,-3.4)--cycle;
\draw[fill](4.12,1.07)circle(1.5pt)node[above]{$D$};
\draw[fill](7.58,-1.44)circle(1.5pt)node[right]{$E$};\end{scope}
\end{tikzpicture}\end{center}
Assinale o ponto \(D\) sobre o lado \(AB\) de forma que se tenha \(AD=A'B'\). Da mesma forma, assinale o ponto \(E\) sobre o lado \(AC\) de forma que se tenha \(AE=A'C'\) .

De (1) temos então \(\dfrac{AD}{AB}=\dfrac{AE}{AC}\). Porém, pela recíproca do teorema de Tales, \(DE\) e \(BC\) são paralelos e assim, os triângulos \(ADE\) e \(ABC\) são semelhantes. Portanto, vale a relação (2):
\begin{equation*}
\begin{split}\frac{AD}{AB}=\frac{AE}{AC}=\frac{DE}{BC}\end{split}
\end{equation*}
Como \(AD=A'B'\) e \(AE=A'C'\), então \(DE=BC\) e assim, os triângulos \(AED\) e \(A'B'C'\) são congruentes. Pelo paralelismo de \(DE\) e \(BC\), temos que os triângulos \(ADE\) e \(ABC\) possuem os mesmos ângulos internos. Logo, \(A'B'C'\) e \(ABC\) possuem os mesmos ângulos internos.}
{1}{1}
\end{sugestions}
\clearmargin
\begin{objectives}{Reta paralela a um dos lados do triângulo}
{
\begin{itemize}
\item {} 
Introduzir o aluno às argumentações formais demonstrando um teorema que é consequencia imediata (corolário) do teorema que ele acaba de acompanhar a demonstração.

\end{itemize}
}{1}{2}
\end{objectives}
\begin{sugestions}{Reta paralela a um dos lados do triângulo}
{
O enunciado do teorema é dado em linguagem corrente, de modo que é necessário separar as informações dadas (hipóteses) e o que desejamos mostrar (tese). Isso não é tão natural para o aluno e, sempre que possível, devemos reforçar que essa estratégia ajuda a organizar nossa argumentação.
}{1}{2}
\end{sugestions}
\begin{answer}{Reta paralela a um dos lados do triângulo}
{
\begin{enumerate}
\item {} 
Se A reta \(DE\) é paralela ao lado \(BC\) e intersecta os lados \(AB\) e \(AC\) do triângulo \(ABC\), então os triângulos \(ABC\) e \(ADE\) são semelhantes.

\item {} 
Os triângulos \(ABC\) e \(ADE\) possuem o ângulo \(A\) em comum e, os ângulos \(ADE\) e \(ABC\) são congruentes, pois as retas \(BC\) e \(DE\) são paralelas. Portanto os triângulos \(ABC\) e \(ADE\) possuem 2 ângulos, respectivamente, congruentes. Pelo Teorema Central da semelhança, esses triângulos são semelhantes.

\end{enumerate}
}{1}
\end{answer}

\paragraph{Teorema central da semelhança de triângulos}



O teorema tem um enunciado simples e vai permitir que identifiquemos triângulos semelhantes com facilidade. Trata-se de mais um caso de semelhança de triângulos, isto é, atendidas as suas condições, não precisamos fazer as verificações de todas as condições da definição de semelhança de figuras.

Entretanto, antes de podermos usá-lo, devemos mostrar que sempre que as condições do caso forem atendidas, necessariamente estamos garantindo todas as condições da definição de semelhança entre figuras.

\begin{observationtitle}{Teorema central da semelhança de triângulos}

\textit{Dois triângulos que possuem os mesmos ângulos internos são semelhantes}
\end{observationtitle}


Note que, pelo Teorema da soma dos ângulos internos de um triângulo, a hipótese do Teorema é garantida se dois triângulos possuem dois ângulos internos respectivamente congruentes, então seus lados são proporcionais. Demonstrando esse fato, poderemos reconhecer facilmente triângulos semelhantes, e essa é a importância desse teorema.

A figura a seguir mostra, de forma simples, a hipótese e a tese do teorema.

\textbf{Hipótese}: Ângulos com marcas iguais são iguais.
\begin{center}\begin{tikzpicture}
\begin{scope}[scale=1.5]
\draw [shift={(-2.32,4.24)},line width=0.8pt,color=\currentcolor!80,fill=\currentcolor!80,fill opacity=0.10000000149011612] (0,0) -- (-103.94809618437361:0.6) arc (-103.94809618437361:-46.138177007488174:0.6) -- cycle;
\draw [shift={(-3.08,1.18)},line width=0.8pt,color=session1,fill=session1,fill opacity=0.10000000149011612] (0,0) -- (-0.3080388573998622:0.6) arc (-0.3080388573998622:76.05190381562642:0.6) -- cycle;
\draw [shift={(0.64,1.16)},line width=0.8pt,color=session3,fill=session3,fill opacity=0.10000000149011612] (0,0) -- (133.86182299251183:0.6) arc (133.86182299251183:179.69196114260015:0.6) -- cycle;
\draw [shift={(1.06,2.86)},line width=0.8pt,color=\currentcolor!80,fill=\currentcolor!80,fill opacity=0.10000000149011612] (0,0) -- (-24.02650657867919:0.6) arc (-24.02650657867919:33.783412598206226:0.6) -- cycle;
\draw [shift={(2.72,2.12)},line width=0.8pt,color=session1,fill=session1,fill opacity=0.10000000149011612] (0,0) -- (79.61355074829451:0.6) arc (79.61355074829451:155.9734934213208:0.6) -- cycle;
\draw [shift={(3.1065988009495857,4.22922061722931)},line width=0.8pt,color=session3,fill=session3,fill opacity=0.10000000149011612] (0,0) -- (-146.2165874017938:0.6) arc (-146.2165874017938:-100.38644925170551:0.6) -- cycle;
\draw [line width=0.8pt] (-2.32,4.24)-- (-3.08,1.18);
\draw [line width=0.8pt] (-3.08,1.18)-- (0.64,1.16);
\draw [line width=0.8pt] (0.64,1.16)-- (-2.32,4.24);
\draw [shift={(-3.08,1.18)},line width=0.8pt,color=session1] (-0.3080388573998622:0.6) arc (-0.3080388573998622:76.05190381562642:0.6);
\draw [shift={(-3.08,1.18)},line width=0.8pt,color=session1] (-0.3080388573998622:0.5) arc (-0.3080388573998622:76.05190381562642:0.5);
\draw [shift={(0.64,1.16)},line width=0.8pt,color=session3] (133.86182299251183:0.6) arc (133.86182299251183:179.69196114260015:0.6);
\draw [shift={(0.64,1.16)},line width=0.8pt,color=session3] (133.86182299251183:0.5) arc (133.86182299251183:179.69196114260015:0.5);
\draw [shift={(0.64,1.16)},line width=0.8pt,color=session3] (133.86182299251183:0.4) arc (133.86182299251183:179.69196114260015:0.4);
\draw [line width=0.8pt] (1.06,2.86)-- (2.72,2.12);
\draw [shift={(2.72,2.12)},line width=0.8pt,color=session1] (79.61355074829451:0.6) arc (79.61355074829451:155.9734934213208:0.6);
\draw [shift={(2.72,2.12)},line width=0.8pt,color=session1] (79.61355074829451:0.5) arc (79.61355074829451:155.9734934213208:0.5);
\draw [line width=0.8pt] (1.06,2.86)-- (3.1065988009495857,4.22922061722931);
\draw [line width=0.8pt] (3.1065988009495857,4.22922061722931)-- (2.72,2.12);
\draw [shift={(3.1065988009495857,4.22922061722931)},line width=0.8pt,color=session3] (-146.2165874017938:0.6) arc (-146.2165874017938:-100.38644925170551:0.6);
\draw [shift={(3.1065988009495857,4.22922061722931)},line width=0.8pt,color=session3] (-146.2165874017938:0.5) arc (-146.2165874017938:-100.38644925170551:0.5);
\draw [shift={(3.1065988009495857,4.22922061722931)},line width=0.8pt,color=session3] (-146.2165874017938:0.4) arc (-146.2165874017938:-100.38644925170551:0.4);
\draw (-1.43,1.1) node[anchor=north west] {$ a $};
\draw (-0.8,3.1) node[anchor=north west] {$ b $};
\draw (-3.13,3.1) node[anchor=north west] {$ c $};
\draw (3.,3.3) node[anchor=north west] {$ a' $};
\draw (1.7,4.1) node[anchor=north west] {$  b'$};
\draw (1.63,2.4) node[anchor=north west] {$c'$};
\draw [fill=black] (-2.32,4.24) circle (1.0pt);
\draw[color=black] (-2.47,4.574) node {$A$};
\draw [fill=black] (-3.08,1.18) circle (1.0pt);
\draw[color=black] (-3.33,1.054) node {$B$};
\draw [fill=black] (0.64,1.16) circle (1.0pt);
\draw[color=black] (0.79,0.994) node {$C$};
\draw [fill=black] (1.06,2.86) circle (1.0pt);
\draw[color=black] (0.69,2.914) node {$A'$};
\draw [fill=black] (2.72,2.12) circle (1.0pt);
\draw[color=black] (2.83,1.854) node {$B'$};
\draw [fill=black] (3.1065988009495857,4.22922061722931) circle (1.0pt);
\draw[color=black] (3.23,4.534) node {$C'$};
\end{scope}
\end{tikzpicture}\end{center}
\textbf{Tese}: \(\dfrac{a}{a'}=\dfrac{b}{b'}=\dfrac{c}{c'}\)

Para demonstrar isso, vamos preparar nossa figura. Manteremos o triângulo \(ABC\) onde está e vamos transportar o triângulo \(A'B'C'\) para que fique sobre o triângulo \(ABC\) de forma que \(A'\) coincida com \(A\), \(B'\) fique sobre \(AB\) e \(C'\) sobre \(AC\). Naturalmente que isso é possível porque os ângulos \(A\) e \(A'\) são congruentes.

\needspace{5em}
A figura fica então assim:
\begin{center}\begin{tikzpicture}
\begin{scope}[scale=1.5]
\draw [shift={(-2.32,4.24)},line width=0.8pt,color=\currentcolor!80,fill=\currentcolor!80,fill opacity=0.10000000149011612] (0,0) -- (-103.94809618437361:0.6) arc (-103.94809618437361:-46.138177007488174:0.6) -- cycle;
\draw [shift={(-3.08,1.18)},line width=0.8pt,color=session1,fill=session1,fill opacity=0.10000000149011612] (0,0) -- (-0.3080388573998622:0.6) arc (-0.3080388573998622:76.05190381562642:0.6) -- cycle;
\draw [shift={(0.64,1.16)},line width=0.8pt,color=session3,fill=session3,fill opacity=0.10000000149011612] (0,0) -- (133.86182299251183:0.6) arc (133.86182299251183:179.69196114260015:0.6) -- cycle;
\draw [shift={(-2.758088266696303,2.476118294617514)},line width=0.8pt,color=session1,fill=session1,fill opacity=0.10000000149011612] (0,0) -- (-0.3080388573998533:0.6) arc (-0.3080388573998533:76.05190381562643:0.6) -- cycle;
\draw [shift={(-0.6137614876038701,2.464589656020243)},line width=0.8pt,color=session3,fill=session3,fill opacity=0.10000000149011612] (0,0) -- (133.8618229925118:0.6) arc (133.8618229925118:179.69196114260015:0.6) -- cycle;
\draw [line width=0.8pt] (-2.32,4.24)-- (-3.08,1.18);
\draw [line width=0.8pt] (-3.08,1.18)-- (0.64,1.16);
\draw [line width=0.8pt] (0.64,1.16)-- (-2.32,4.24);
\draw [shift={(-3.08,1.18)},line width=0.8pt,color=session1] (-0.3080388573998622:0.6) arc (-0.3080388573998622:76.05190381562642:0.6);
\draw [shift={(-3.08,1.18)},line width=0.8pt,color=session1] (-0.3080388573998622:0.5) arc (-0.3080388573998622:76.05190381562642:0.5);
\draw [shift={(0.64,1.16)},line width=0.8pt,color=session3] (133.86182299251183:0.6) arc (133.86182299251183:179.69196114260015:0.6);
\draw [shift={(0.64,1.16)},line width=0.8pt,color=session3] (133.86182299251183:0.5) arc (133.86182299251183:179.69196114260015:0.5);
\draw [shift={(0.64,1.16)},line width=0.8pt,color=session3] (133.86182299251183:0.4) arc (133.86182299251183:179.69196114260015:0.4);
\draw (-1.31,0.984) node[anchor=north west] {$ a $};
\draw (0.03,3.224) node[anchor=north west] {$ b $};
\draw (-3.75,3.104) node[anchor=north west] {$ c $};
\draw [line width=0.8pt] (-2.758088266696303,2.476118294617514)-- (-0.6137614876038701,2.464589656020243);
\draw [shift={(-2.758088266696303,2.476118294617514)},line width=0.8pt,color=session1] (-0.3080388573998533:0.6) arc (-0.3080388573998533:76.05190381562643:0.6);
\draw [shift={(-2.758088266696303,2.476118294617514)},line width=0.8pt,color=session1] (-0.3080388573998533:0.5) arc (-0.3080388573998533:76.05190381562643:0.5);
\draw [shift={(-0.6137614876038701,2.464589656020243)},line width=0.8pt,color=session3] (133.8618229925118:0.6) arc (133.8618229925118:179.69196114260015:0.6);
\draw [shift={(-0.6137614876038701,2.464589656020243)},line width=0.8pt,color=session3] (133.8618229925118:0.5) arc (133.8618229925118:179.69196114260015:0.5);
\draw [shift={(-0.6137614876038701,2.464589656020243)},line width=0.8pt,color=session3] (133.8618229925118:0.4) arc (133.8618229925118:179.69196114260015:0.4);
\draw [line width=0.8pt,dash pattern=on 2pt off 2pt] (-0.6137614876038701,2.464589656020243)-- (-0.9356732209075664,1.168471361402729);
\draw (-3.2,2.684) node[anchor=north west] {B'};
\draw (-0.49,2.724) node[anchor=north west] {C'};
\draw (-1.05,1.004) node[anchor=north west] {D};
\draw (-2.9,3.7) node[anchor=north west] {$ c' $};
\draw (-1.4,3.7) node[anchor=north west] {$  b'$};
\draw (-1.89,2.4) node[anchor=north west] {$  a'$};
\draw [fill=black] (-2.32,4.24) circle (1.0pt);
\draw[color=black] (-2.39,4.574) node {$A$};
\draw [fill=black] (-3.08,1.18) circle (1.0pt);
\draw[color=black] (-3.33,1.054) node {$B$};
\draw [fill=black] (0.64,1.16) circle (1.0pt);
\draw[color=black] (0.83,1.054) node {$C$};
\draw [fill=black] (-2.758088266696303,2.476118294617514) circle (1.0pt);
\draw [fill=black] (-0.6137614876038701,2.464589656020243) circle (1.0pt);
\draw [fill=black] (-0.9356732209075664,1.168471361402729) circle (1.0pt);
\end{scope}
\end{tikzpicture}\end{center}
Observe que os ângulos congruentes em \(B\) e \(B'\) garantem que as retas \(B'C'\) e \(BC\) são paralelas. Assim, pelo teorema de Tales (ou pela propriedade da projeção paralela), temos que \(\dfrac{AB’}{AB}=\dfrac{AC’}{AC}\) , ou seja, \(\dfrac{c’}{c}=\dfrac{b’}{b}\).

Para completar a proporção, traçamos \(C'D\), paralelo a \(AB\) como mostra a figura acima. Novamente, pelo teorema de Tales (ou pela propriedade da projeção paralela), temos que \(\dfrac{AC’}{AC}=\dfrac{BD}{BC}=\dfrac{B’C’}{BC}\)  , porque \(BDC'B'\) é um paralelogramo. Assim, \(\dfrac{b’}{b}=\dfrac{a’}{a}\)  , completando a demonstração.

Como os lados dos triângulos \(ABC\) e \(A^\prime B^\prime C^\prime\) são proporcionais então eles são semelhantes.
\begin{equation*}
\begin{split}\triangle ABC \sim \triangle A’B’C’\end{split}
\end{equation*}
A recíproca do Teorema central da semelhança de triângulos é verdadeira. Se os triângulos   e   são semelhantes, ou seja, se seus lados são proporcionais, então eles possuem os mesmos ângulos internos.

Por exemplo, a figura a seguir mostra duas imagens semelhantes, três pontos \(A\), \(B\) e \(C\) em uma delas e seus correspondentes \(A'\), \(B'\) e \(C'\) na outra. Como essas imagens são semelhantes, então os triângulos \(ABC\) e \(A'B' C'\) possuem os mesmos ângulos internos

\begin{figure}[H]
\centering

\noindent\includegraphics[width=375bp]{{Sem_fig_extra_1}.png}
\end{figure}



\begin{task}{Reta paralela a um dos lados do triângulo}



Considere a seguinte afirmação:

“Toda reta paralela a um dos lados de um triângulo e que corta os dois outros, determina um novo triângulo semelhante ao primeiro”.

Observe a figura a seguir para compreender melhor a situação:
\begin{center}\begin{tikzpicture}
\begin{scope}[scale=1.5]
\fill[line width=0.8pt,fill=\currentcolor!80] (-3.28,5.36) -- (-1.905045278137127,3.5806468305304002) -- (0.1439068564036239,4.052445019404915) -- cycle;
\draw [line width=0.8pt] (-3.28,5.36)-- (-1.24,2.72);
\draw [line width=0.8pt] (-1.24,2.72)-- (1.8,3.42);
\draw [line width=0.8pt] (1.8,3.42)-- (-3.28,5.36);
\draw [line width=0.8pt] (-1.905045278137127,3.5806468305304002)-- (0.1439068564036239,4.052445019404915);
\draw [line width=0.8pt] (-3.28,5.36)-- (-1.905045278137127,3.5806468305304002);
\draw [line width=0.8pt] (-1.905045278137127,3.5806468305304002)-- (0.1439068564036239,4.052445019404915);
\draw [line width=0.8pt] (0.1439068564036239,4.052445019404915)-- (-3.28,5.36);
\draw [fill=black] (-3.28,5.36) circle (1.0pt);
\draw[color=black] (-3.36,5.65) node {$A$};
\draw [fill=black] (-1.24,2.72) circle (1.0pt);
\draw[color=black] (-1.4,2.5) node {$B$};
\draw [fill=black] (1.8,3.42) circle (1.0pt);
\draw[color=black] (2.04,3.49) node {$C$};
\draw [fill=black] (-1.905045278137127,3.5806468305304002) circle (1.0pt);
\draw[color=black] (-2.14,3.49) node {$D$};
\draw [fill=black] (0.1439068564036239,4.052445019404915) circle (1.0pt);
\draw[color=black] (0.3,4.29) node {$E$};
\end{scope}
\end{tikzpicture}\end{center}\begin{enumerate}
\item {} 
Reescreva a afirmação no formato ‘Se \textbf{Hipótese}, então \textbf{Tese}, substituindo Hipótese e Tese pelas condições que são dadas na afirmação e que queremos mostrar, respectivamente. Se quiser, você pode utilizar as notações da figura acima

\item {} 
Justifique o teorema utilizando o Teorema Central da Semelhança.

\end{enumerate}
\end{task}



\exercise

\begin{enumerate}
\item {} 
Determine as medidas \(x\) e \(y\) nas figuras abaixo, justificando suas respostas.
\begin{center}\begin{tikzpicture}
\clip(-4.3,-1.32) rectangle (6.,3.6);
\draw[line width=0.8pt,fill=black,fill opacity=0.10000000149011612] (3.58,0.21213203435596437) -- (3.367867965644036,0.2121320343559644) -- (3.367867965644036,0.) -- (3.58,0.) -- cycle;
\draw[line width=0.8pt,fill=black,fill opacity=0.10000000149011612] (5.,0.21213203435596437) -- (4.787867965644035,0.2121320343559644) -- (4.787867965644035,0.) -- (5.,0.) -- cycle;
\draw [line width=0.8pt] (-4.,0.)-- (0.,0.);
\draw [line width=0.8pt] (-2.035,0.09) -- (-2.035,-0.09);
\draw [line width=0.8pt] (-1.965,0.09) -- (-1.965,-0.09);
\draw [line width=0.8pt] (-2.4641436807847166,3.1508722475607844)-- (-4.,0.);
\draw [line width=0.8pt] (-2.4641436807847166,3.1508722475607844)-- (0.,0.);
\draw [line width=0.8pt] (1.,0.)-- (5.,0.);
\draw [line width=0.8pt] (1.,0.)-- (5.,2.28);
\draw [line width=0.8pt] (-3.4004631175077957,1.2299745105056301)-- (-0.9619031428313264,1.2299745105056301);
\draw [line width=0.8pt] (-2.216183130169561,1.3199745105056302) -- (-2.216183130169561,1.1399745105056303);
\draw [line width=0.8pt] (-2.1461831301695606,1.3199745105056302) -- (-2.1461831301695606,1.1399745105056303);
\draw [line width=0.8pt] (3.58,1.4706)-- (3.58,0.);
\draw [line width=0.8pt] (5.,2.28)-- (5.,0.);
\draw [line width=0.8pt,dash pattern=on 2pt off 2pt] (1.,-0.32)-- (5.,-0.32);
\draw (-1.4,2.5) node[anchor=north west] {$ a $};
\draw (-2.26,1.2) node[anchor=north west] {$ a $};
\draw (-0.36,1.24) node[anchor=north west] {$ x $};
\draw (-2.3,0.) node[anchor=north west] {$ a + 6$};
\draw (3.1,0.86) node[anchor=north west] {7};
\draw (5.04,1.22) node[anchor=north west] {10};
\draw (2.94,-0.46) node[anchor=north west] {18};
\draw (4,0.5) node[anchor=north west] {$ y $};
\draw [fill=black] (-4.,0.) circle (1.0pt);
\draw [fill=black] (0.,0.) circle (1.0pt);
\draw [fill=black] (-2.4641436807847166,3.1508722475607844) circle (1.0pt);
\draw [fill=black] (-3.4004631175077957,1.2299745105056301) circle (1.0pt);
\draw [fill=black] (-0.9619031428313264,1.2299745105056301) circle (1.0pt);
\draw [fill=black] (1.,0.) circle (1.0pt);
\draw [fill=black] (5.,0.) circle (1.0pt);
\draw [fill=black] (3.58,0.) circle (1.0pt);
\draw [fill=black] (5.,2.28) circle (1.0pt);
\draw [fill=black] (3.58,1.4706) circle (1.0pt);
\draw [color=black] (1.,-0.32)-- ++(-1.0pt,0 pt) -- ++(2.0pt,0 pt) ++(-1.0pt,-1.0pt) -- ++(0 pt,2.0pt);
\draw [color=black] (5.,-0.32)-- ++(-1.0pt,0 pt) -- ++(2.0pt,0 pt) ++(-1.0pt,-1.0pt) -- ++(0 pt,2.0pt);
\end{tikzpicture}\end{center}
\item {} 
Determine as medidas \(x\) e \(y\) na figura a seguir.
\begin{center}\begin{tikzpicture}
\clip(-2.3498338067109477,-1.4853536647927779) rectangle (5.844102600882048,2.450763593358133);
\draw [shift={(4.,0.)},line width=0.8pt,fill=black,fill opacity=0.10000000149011612] (0,0) -- (119.67517160754605:0.3784728132837411) arc (119.67517160754605:180.:0.3784728132837411) -- cycle;
\draw [shift={(1.8504464281507047,1.5626472567954692)},line width=0.8pt,fill=black,fill opacity=0.10000000149011612] (0,0) -- (-157.91094700933436:0.3784728132837411) arc (-157.91094700933436:-97.58611861688043:0.3784728132837411) -- cycle;
\draw [line width=0.8pt] (-2.,0.)-- (4.,0.);
\draw [line width=0.8pt] (4.,0.)-- (2.873091016604679,1.9776725767534562);
\draw [line width=0.8pt] (2.873091016604679,1.9776725767534562)-- (-2.,0.);
\draw [line width=0.8pt] (1.8504464281507047,1.5626472567954692)-- (1.6423300876403646,0.);
\draw (-0.24930969298618452,1.4) node[anchor=north west] {10};
\draw (-0.04114964568012691,-0.1417751776354958) node[anchor=north west] {8};
\draw (2.87309101660468,-0.1417751776354958) node[anchor=north west] {6};
\draw (2.1729163120297583,2.393992671365572) node[anchor=north west] {$ x $};
\draw (1.7755198580818303,0.8422541369022319) node[anchor=north west] {4,5};
\draw (3.5354184398512265,1.5613524821413407) node[anchor=north west] {$ y $};
\draw (3.1,0.6) node[anchor=north west] {$\alpha$};
\draw (1.1,1.2) node[anchor=north west] {$\alpha$};
\draw [fill=black] (-2.,0.) circle (1.0pt);
\draw [fill=black] (4.,0.) circle (1.0pt);
\draw [fill=black] (2.873091016604679,1.9776725767534562) circle (1.0pt);
\draw [fill=black] (1.8504464281507047,1.5626472567954692) circle (1.0pt);
\draw [fill=black] (1.6423300876403646,0.) circle (1.0pt);
\end{tikzpicture}\end{center}
\item {} 
Um turista em São Paulo visitou o parque do Ibirapuera e, ao sair, quis saber a que distância estava a estação do Metrô mais próxima. Consultando seu mapa, o turista identificou que encontrava-se no ponto A e que a estação Paraíso estava no ponto B, como se vê na figura a seguir.

\begin{figure}[H]
\centering

\noindent\includegraphics[width=225bp]{{FigSem-21}.png}
\end{figure}

A escala do mapa que o turista consultou era de 1/300 e, a distância AB nesse mapa era de $4{,}8$ cm.

Que distância o turista teria que caminhar até o Metrô?

\item {} 
Determine o lado do quadrado inscrito em um triângulo de base \(a\) e altura \(h\), como na figura abaixo.
\begin{center}\begin{tikzpicture}
\draw [line width=0.8pt] (0.,0.)-- (0.,2.);
\draw [line width=0.8pt] (0.,2.)-- (2.,2.);
\draw [line width=0.8pt] (2.,2.)-- (2.,0.);
\draw [line width=0.8pt] (-0.9,0.)-- (3.52,0.);
\draw [line width=0.8pt] (3.52,0.)-- (0.7438016528925621,3.6528925619834713);
\draw [line width=0.8pt] (0.7438016528925621,3.6528925619834713)-- (-0.9,0.);
\draw [line width=0.8pt,dash pattern=on 3pt off 3pt] (-2.434545454545454,3.6528925619834713)-- (0.4927272727272726,3.6528925619834713);
\draw [line width=0.8pt,dash pattern=on 3pt off 3pt] (-2.270909090909091,0.)-- (-1.18,0.);
\draw [line width=0.8pt,dash pattern=on 3pt off 3pt] (-0.8890909090909092,-0.4327272727272719)-- (3.565454545454545,-0.4327272727272719);
\draw [line width=0.8pt,dash pattern=on 3pt off 3pt] (-1.670909090909091,0.)-- (-1.670909090909091,3.6528925619834713);
\draw (1.1,-0.5) node[anchor=north west] {$ a $};
\draw (-2.1,2.1) node[anchor=north west] {$ h $};
\draw [fill=black] (0.,0.) circle (1.0pt);
\draw [fill=black] (2.,0.) circle (1.0pt);
\draw [fill=black] (0.,2.) circle (1.0pt);
\draw [fill=black] (2.,2.) circle (1.0pt);
\draw [fill=black] (-0.9,0.) circle (1.0pt);
\draw [fill=black] (3.52,0.) circle (1.0pt);
\draw [fill=black] (0.7438016528925621,3.6528925619834713) circle (1.0pt);
\draw [color=black] (-0.8890909090909092,-0.4327272727272719)-- ++(-1.5pt,0 pt) -- ++(3.0pt,0 pt) ++(-1.5pt,-1.5pt) -- ++(0 pt,3.0pt);
\draw [color=black] (-1.670909090909091,0.)-- ++(-1.5pt,0 pt) -- ++(3.0pt,0 pt) ++(-1.5pt,-1.5pt) -- ++(0 pt,3.0pt);
\draw [color=black] (-1.670909090909091,3.6528925619834713)-- ++(-1.5pt,0 pt) -- ++(3.0pt,0 pt) ++(-1.5pt,-1.5pt) -- ++(0 pt,3.0pt);
\draw [color=black] (3.565454545454545,-0.4327272727272719)-- ++(-1.5pt,0 pt) -- ++(3.0pt,0 pt) ++(-1.5pt,-1.5pt) -- ++(0 pt,3.0pt);
\end{tikzpicture}\end{center}
\item {} 
Carlos e sua esposa Joana estavam visitando Buenos Aires e ao passar pelo enorme obelisco que fica na Praça da República, ele teve a curiosidade de saber sua altura.

\begin{figure}[H]
\centering

\noindent\includegraphics[width=300bp]{{FigSem-23}.png}
\end{figure}

Joana que é engenheira e tem sempre uma pequena trena na bolsa disse o seguinte: Caminhe sobre a sombra do obelisco e conte seus passos. Procure dar passos iguais. Carlos fez o que ela pediu e encontrou 44 passos para o comprimento da sombra.

Em seguida, ela pediu a Carlos que ficasse em pé e mediu a sombra de Carlos no chão, encontrando $87$ centímetros. Pediu ainda que Carlos desse um passo do tamanho que ele usou para caminhar sobre a sombra e encontrou $70$ cm para o comprimento do passo. Como Joana sabia que Carlos possui 1,80m de altura, ela pode determinar a altura do obelisco.

Com os dados dessa história, determine um valor aproximado para a altura do obelisco.

\item {} 
A figura abaixo mostra uma sequência de três triângulos equiláteros sendo que os dois primeiros possuem lados medindo 8 e 6.
\begin{center}\begin{tikzpicture}
\begin{scope}[scale=.35]
\fill[line width=0.8pt,color=\currentcolor!80,fill=\currentcolor!80,fill opacity=0.15000000596046448] (0.,0.) -- (8.,0.) -- (4.,6.9282032302755105) -- cycle;
\fill[line width=0.8pt,color=\currentcolor!80,fill=\currentcolor!80,fill opacity=0.15000000596046448] (8.,0.) -- (14.,0.) -- (11.,5.196152422706633) -- cycle;
\fill[line width=0.8pt,color=\currentcolor!80,fill=\currentcolor!80,fill opacity=0.15000000596046448] (14.,0.) -- (18.5,0.) -- (16.25,3.897114317029974) -- cycle;
\draw [line width=0.8pt,color=\currentcolor!80] (0.,0.)-- (8.,0.);
\draw [line width=0.8pt,color=\currentcolor!80] (8.,0.)-- (4.,6.9282032302755105);
\draw [line width=0.8pt,color=\currentcolor!80] (4.,6.9282032302755105)-- (0.,0.);
\draw [line width=0.8pt] (4.,6.9282032302755105)-- (32.,0.);
\draw [line width=0.8pt,color=\currentcolor!80] (8.,0.)-- (14.,0.);
\draw [line width=0.8pt,color=\currentcolor!80] (14.,0.)-- (11.,5.196152422706633);
\draw [line width=0.8pt,color=\currentcolor!80] (11.,5.196152422706633)-- (8.,0.);
\draw [line width=0.8pt,color=\currentcolor!80] (14.,0.)-- (18.5,0.);
\draw [line width=0.8pt,color=\currentcolor!80] (18.5,0.)-- (16.25,3.897114317029974);
\draw [line width=0.8pt,color=\currentcolor!80] (16.25,3.897114317029974)-- (14.,0.);
\draw [line width=0.8pt] (0.,0.)-- (32.,0.);
\draw (3.9951310905372663,-0.5041937742192144) node[anchor=north west] {8};
\draw (11.082050730853824,-0.5041937742192144) node[anchor=north west] {6};
\draw (16.248216449963092,-0.5041937742192144) node[anchor=north west] {$ x $};
\draw (-0.6411714778941265,-0.3717279865497462) node[anchor=north west] {$A$};
\draw (32.144110970299295,-0.10679641121080977) node[anchor=north west] {$B$};
\end{scope}
\end{tikzpicture}\end{center}\begin{enumerate}
\item {} 
Calcule o lado do terceiro triângulo.

\item {} 
Quanto mede o segmento AB?

\end{enumerate}

\item {} 
Seja ABCD um retângulo e \(M\) um ponto do lado \(CD\). São dadas as medidas: \(BC = 5\), \(CM = 4\) e \(MD = 2\). O ponto \(P\) é o ponto de interseção dos segmentos \(AC\) e \(MB\). Calcule as distâncias de \(P\) aos quatro lados do retângulo.

\end{enumerate}










\ifnum\aluno=1
\clearpage
\else
\notasfinais
\fi

\bibliographystyle{apalike-pt}
\bibliography{../Bibliografia/probabilidade1_bibliografia.bib}

\nocite{*}

% \ifnum\aluno=1
\renewcommand\chapterillustration{./abertura-trigonometria}
\else
\renewcommand\chapterillustration{./abertura-trigonometria-professor}
\fi


\makeatletter
\ifnum\aluno=1
\else
\renewcommand*{\toclevel@section}{1}
\renewcommand*{\toclevel@subsection}{4}
\renewcommand*{\toclevel@paragraph}{5}
\renewcommand*{\toclevel@subparagraph}{6}

\renewcommand*{\toclevel@exploresec}{2}
\renewcommand*{\toclevel@practicesec}{2}
\renewcommand*{\toclevel@arrangesec}{2}
\renewcommand*{\toclevel@knowsec}{2}
\renewcommand*{\toclevel@exercisesec}{1}

\setcounter{tocdepth}{2}
\fi
\makeatother

\renewcommand\chapterwhat{Trigonometria em triângulos retângulos e em triângulos não retângulos,  leis dos senos e 
cossenos.}
\renewcommand\chapterbecause{As relações e resultados provenientes da Trigonometria são fundamentais para resolver problemas em diversos contextos, apesar de  muitas vezes não serem utilizados explicitamente no nosso cotidiano. Aplicações relacionadas à Trigonometria podem ser encontradas em diversas áreas, como Física, Engenharia, Arquitetura, Astronomia e Topografia.} 
\chapter{Trigonometria}
\label{trigo-chap}

\mbox{}\thispagestyle{empty}\clearpage

\thispagestyle{empty}

\begin{center}
Projeto: LIVRO ABERTO DE MATEMÁTICA

\noindent \begin{tabular}{lcccr}
\includegraphics[scale=.15]{impa}& \quad\quad& \includegraphics[width=3cm]{logo} & \quad\quad& \includegraphics[scale=.24]{obmep} 
\end{tabular}
\end{center}

\vspace*{.3cm}

Cadastre-se como colaborador no site do projeto: \url{umlivroaberto.org}



% \begin{center}
%   \includegraphics[width=2cm]{canvas}
% \end{center}

\begin{tabular}{p{.15\textwidth}p{.7\textwidth}}
Título: & Trigonometria\\
\\
Ano/ Versão: & 2020 / versão 0.1 de \today\\
\\
Editora & Instituto Nacional de Matem\'atica Pura e Aplicada (IMPA-OS)\\
\\
Realização:& Olimp\'iada Brasileira de Matem\'atica das Escolas P\'ublicas (OBMEP)\\
\\
Produção:& Associação Livro Aberto\\
\\
Coordenação: & Fabio Simas, \\
			&  Augusto Teixeira (livroaberto@impa.br)\\
\\
  Autores: & Carlos A. Gomes (UFRN),\\
             & Lhaylla Crissaff (UFF).\\
        
\\
Colaboração: & \\
\\
Revisor: & Cydara Ripoll \\

\\
Design: & Andreza Moreira (Tangentes Design) \\
\\
  Ilustrações: & --- \\ 
\\
Gráficos: & ---\\
\\
  Capa: & Foto de Logan Liu, no Unsplash \\
  		& https://unsplash.com/photos/KYBLVEtEDgc \\

\end{tabular}
\vspace{.5cm}



\begin{figure}[b]
\begin{minipage}[l]{5cm}
\centering

{\large Licença:}

  \includegraphics[width=3.5cm]{cc-by-nc-sa}
\end{minipage}\hfill
\begin{minipage}[c]{5cm}
\centering
{\large Desenvolvido por}

\includegraphics[width=2.5cm]{logo-associacao.jpg}
\end{minipage}
\begin{minipage}[r]{5cm}
\centering

{\large Patrocínio:}
  \vspace{1em}
  \includegraphics[width=3.5cm]{itau}
\end{minipage}
\end{figure}

\mainmatter

\begin{apresentacao}{Apresentação do capítulo}

A palavra Trigonometria, do grego, significa medida dos ângulos do triângulo.
%
Em Matemática, é um ramo da Geometria que estuda as relações entre as amplitudes dos ângulos e os comprimentos dos lados determinados por eles em um triângulo qualquer e, também, as funções trigonométricas. 
%
A Trigonometria é comumente dividida em Trigonometria Plana e Esférica. 
%
A primeira estuda as relações entre ângulos e lados de triângulos planos e as funções trigonométricas, e a segunda trabalha com os chamados triângulos esféricos, aqueles cujos lados são arcos de circunferências sobre uma esfera. 

Como a abordagem da Trigonometria no Ensino Médio se reduz à Trigonometria Plana, neste capítulo nos restringiremos apenas a esta parte, em especial, ao estudo dos triângulos planos e seus desdobramentos. 
%
O conteúdo de funções trigonométricas estará presente em um outro capítulo deste livro. 
%
Já o conteúdo de Trigonometria Esférica, pode ser encontrado em \cite{coutinho2001}. 

Segundo \citet{eves1995}, não se pode precisar a origem da Trigonometria.
%
O documento egípcio nomeado de Papiro Rhind, datado do século XVII a.C., fazia menção à cotangente de um ângulo. 
%
A tábula babilônica em escrita cuneiforme Plimpton 322, datada do século XVIII a.C., contém registros de problemas envolvendo secantes. 
%
A Matemática desenvolvida na Mesopotâmia antiga também têm indícios do uso da Trigonometria para resolver problemas práticos.
%
Ainda segundo \citeauthor{eves1995}, a Trigonometria Esférica está ligada à astronomia primitiva desenvolvida pelos astrônomos babilônios do século IV e V a.C. e que foi repassada para os gregos.

Contudo, segundo Boyer \citeauthor{boyer1974}, ``com os gregos, pela primeira vez encontramos um estudo sistemático de relações entre ângulos (ou arcos) num círculo e os comprimentos de cordas que os subentendem'' \cite[p. 116]{boyer1974}(1974, p. 116).
%
Na famosa obra \textit{Os Elementos}, do matemático grego Euclides, é possível encontrar alguns resultados trigonométricos equivalentes aos que conhecemos hoje. 
%
Inclusive, em um dos 13 volumes desta obra, duas proposições são dedicadas à lei dos cossenos para ângulos obtusos e agudos. 
%
Porém, segundo \citeauthor{eves1995}, foi o grego Hiparco de Nicéia, que viveu no século II a.C., quem recebeu o título de ``pai da Trigonometria'' por ter criado a primeira tabela trigonométrica que se tem registro. 
%
Nessa mesma época, o matemático grego Ptolomeu também apresentou uma tabela contendo o valor do seno de ângulos entre $0^\circ$ e $90^\circ$, que eram os ângulos utilizados em sua pesquisa sobre Astronomia. 
%
Mais informações sobre a história da Trigonometria podem ser encontradas em \cite{eves1995,boyer1974,roque2012}. 

Nos dias atuais, resultados provenientes da Trigonometria são fundamentais para resolver problemas de diversas áreas como Arquitetura, Engenharia, Física e etc. 
%
Conhecer esse fato e exemplificá-lo é muito importante para os estudantes do Ensino Médio, que por muitas vezes não veem razão para dedicar tanto tempo de estudo a essa importante área da Matemática \citep{gur2009}. 
%
Nesse sentido, encorajamos o professor de Matemática a abordar o assunto de maneira cuidadosa, mostrando sua importância para a Matemática e para diversas áreas da atividade humana. 
%
Inclusive, é preciso destacar que, dentre as cinco habilidades específicas de Matemática e suas Tecnologias listadas na BNCC \citep{BNCC2018}, quatro delas mencionam interpretar e resolver problemas em diversos contextos.

Tradicionalmente, o conteúdo de Trigonometria é trabalhado a partir do $9^\circ$ do Ensino Fundamental, seguindo as orientações do currículo mínimo de cada estado brasileiro. 
%
Contudo, a versão atual da BNCC \citep{BNCC2018} do Ensino Fundamental, aprovada em 2018, não faz menção a este conteúdo.
%
Dessa forma, todo o conteúdo trigonométrico deve ser trabalhado no Ensino Médio e por isso, este livro pretende contemplar este tema desde as ideias iniciais.

A BNCC aponta para o estudo da Trigonometria no Ensino Médio na seguinte habilidade:

\begin{habilities}{EM13MAT308} 
Aplicar as relações métricas, incluindo as leis do seno e do cosseno ou as noções de congruência e semelhança, para resolver e elaborar problemas que envolvem triângulos, em variados contextos.
\end{habilities}

Para desenvolver o que propõe essa habilidade, em termos de Trigonometria, é preciso que o professor trabalhe em sua sala de aula não apenas a teoria, mas também o relacione a problemas de diversos contextos.
%
Vale ressaltar, que o conteúdo sobre relações métricas no triângulo retângulo, noções de congruência e semelhança, consideradas pré-requisitos para este capítulo, está presente nas habilidades \textbf{EF08MA14}, \textbf{EF09MA12} e \textbf{EF09MA13} do Ensino Fundamental.

Para adquirir as competências requeridas para a aprendizagem da Trigonometria, neste capítulo, traçamos um itinerário pedagógico que permite ao professor levar os conteúdos teóricos contidos na habilidade \textbf{EM13MAT308} para sua sala de aula, apoiado em problemas de diversos contextos e em harmonia com os demais capítulos deste livro, valorizando assim o pensamento construtivo do estudante e sugerindo uma prática metodológica pautada na compreensão dos conceitos e não na memorização e repetição de fórmulas dissociadas de significado, procurando tornar o conteúdo muito mais dinâmico, natural e atraente para o aluno.

Neste capítulo, o estudante será apresentado à definição de seno, cosseno e tangente de qualquer ângulo entre $0^\circ$ e $180^\circ$; a partir disso, ele será levado a provar resultados que relacionam lados e ângulos de um triângulo qualquer e, por fim, aplicar tais conceitos e resultados em situações-problema que envolvam triângulos.
%
Vale ressaltar que, a unidade de medida utilizada para ângulos nesse capítulo será sempre grau, ou seja,  não estaremos interessados em relacionar graus com radianos, o que será feito adiante quando quisermos estender o conceito de seno, cosseno e tangente de um ângulo para número real, para então construir funções trigonométricas.
%
Por isso, deixaremos essa parte para o capítulo que abordará as funções trigonométricas.
%
Encorajamos o professor a consultar o conteúdo de funções trigonométricas antes de trabalhar o que propomos neste capítulo em sua sala de aula, visto que os estudantes que já tiverem tido contato com o conteúdo de Trigonometria podem levar questionamentos para a sala de aula que envolvam partes da Trigonometria que foram deixadas para o outro capítulo.
%
E também, porque, como os dois capítulos, de certa forma, se complementam, o professor terá uma visão mais completa do ramo da Trigonometria plana.

Iniciamos a primeira seção deste capítulo estabelecendo três quocientes entre números reais relacionados a um ângulo agudo. 
%
Esses quocientes são chamados razões trigonométricas de um ângulo agudo e, mostraremos como elas podem ser úteis para resolver problemas práticos de diversas áreas do conhecimento. 

Já na segunda seção, a definição de seno, cosseno e tangente será estendida para qualquer ângulo entre $0^\circ$ e $180^\circ$. 
%
Isso permitirá demonstrar um resultado muito importante da Geometria, chamado Lei dos Cossenos, que estende o teorema de Pitágoras, isto é, caso o triângulo seja retângulo, a Lei dos Cossenos recai no teorema de Pitágoras.
%
Esse resultado fornece uma relação entre os lados de um triângulo qualquer e o cosseno de um de seus ângulos. 
%
Além desse resultado, também mostraremos outro importante resultado chamado Lei dos Senos, que relaciona os lados de um triângulo qualquer e o seno de seus ângulos.
%
Esses dois resultados serão utilizados em diversas aplicações para resolver problemas que são modelados por triângulos, como por exemplo, a medição de distâncias inacessíveis e o ângulo formado pelos átomos da molécula da água em estado líquido e em congelado.

Por fim, são sugeridos alguns exercícios a serem resolvidos pelos alunos onde eles poderão praticar o que foi aprendido no capítulo e ainda ter contato com um maior número de situações-problema que são resolvidas usando Trigonometria.

%% para páginas sem o livro do aluno utilize
%% para notas de rodapé utilize \notarodape{}
%% \begin{paginatexto} ... \end{paginatexto}

\subsection*{O que dizem as pesquisas sobre o tema?}

Os estudantes têm, em geral, dificuldade para compreender a Trigonometria \citep{weber2005, brown2005}. 
%
Muitas são as razões apontadas em estudos da área para esta dificuldade.
%
Para \citet{moore2009} a dificuldade de aprendizado pode estar relacionada ao fraco entendimento do conceito de ângulo (tema que reconhecemos complexo) que os estudantes normalmente possuem, assim como do conhecimento desconectado da Trigonometria no triângulo retângulo e do círculo trigonométrico.
%
\citet{bressoud2010} corrobora esse pensamento e aponta esta dicotomia como a responsável por problemas enfrentados pelos estudantes.

Segundo \citet{weber2005}, é necessário um nível razoável de abstração para compreender os mais diversos conceitos da Trigonometria, já que esta importante área mescla geometria, álgebra e construção de gráficos. 
%
De fato, chegar nesse grau de abstração pode ser um grande obstáculo para os estudantes, que, em muitos casos, sequer adquirem o conhecimento dos pré-requisitos para o estudo da Trigonometria. 
%
Na verdade, \citet{fortes2012} aponta que há uma relação entre os erros dos estudantes e a falta de conhecimento de assuntos estudados anteriormente. 

De acordo com \citet{silvaneto2006}, muitos erros referentes às razões trigonométricas de um ângulo agudo estão diretamente relacionadas à dificuldade dos alunos em identificar corretamente elementos do triângulo retângulo, como lado oposto ou adjacente a um ângulo específico. 
%
Já \citet{gur2009} aponta que os estudantes memorizam a relação fundamental (que estabelece que $\sen^2\alpha+\cos^2\alpha=1$ para todo ângulo $\alpha$), mas não são capazes de explicar o porquê dessa fórmula ser válida e não conseguem relacioná-la com o teorema de Pitágoras. 
%
O autor também destaca que muitos erros cometidos pelos estudantes são devidos à aplicação puramente mecânica das fórmulas trigonométricas sem a compreensão adequada dos conceitos. 
%
A falta de compreensão dos conceitos também é uma causa das dificuldades apontadas por \citet{dioniziobrandt2011}.

\begin{figure}[H]
    \centering
    \includegraphics[width=\linewidth]{paraprofcap_mapaconceitual.png}
    \caption{Mapa conceitual da Trigonometria proposto por \citet{chigonga2016}.}
    \label{paraoprofcap_mapaconceitual}
\end{figure}
\citet{chigonga2016} também estudou dificuldades dos estudantes ao estudarem Trigonometria, em especial, ao resolverem equações trigonométricas, e menciona que os estudantes não compreendem a razão pelo o qual seno, cosseno e tangente de certos ângulos assumem valores negativos. 
%
O autor também apresenta um mapa conceitual, que pode ser visto na \Fref{paraoprofcap_mapaconceitual}, mostrando a gama de conexões dessa área. Essas conexões precisam ficar claras para os estudantes, de forma que eles percebam a Trigonometria como um todo e não como partes independentes dissociadas de significado.

\citeauthor{feijo2018} afirma que ``mesmo os alunos que apresentam entendimento sobre algum dos muitos ramos da Trigonometria (prioritariamente as razões trigonométricas) [...], o têm de forma rasa e desconexa com os demais ramos"  \citep[p. 52]{feijo2018}(2018, p. 52) . 
% 
Ela ainda constata que os erros cometidos pelos estudantes ocorrem desde o uso das definições e conceitos, até nas manipulações e generalizações.

Diante das dificuldades encontradas pelos estudantes ao estudar Trigonometria, concluímos que este ramo da Matemática requer um grande cuidado para ser apresentado aos estudantes do Ensino Médio.
%
A Trigonometria assume um papel importante dentro da Matemática por sua origem histórica e por sua aplicação em outras áreas.
%
Além disso, seu aprendizado mostra que o estudante foi capaz de articular raciocínio algébrico, geométrico e gráfico, atingindo um nível de abstração relevante em Matemática \citep{weber2005}. 

Alguns autores sugerem que ensinar Matemática, em especial Trigonometria, através de atividades pode ser um facilitador do processo de ensino e aprendizagem \citep{costa1997, mendes2001, silva2011}. 
%
Nesse contexto, consideramos atividade toda tarefa delegada aos alunos para ser trabalhada individualmente ou em grupo, em sala de aula ou fora dela, com o auxílio do professor ou não.
%
Cada atividade, ao ser realizada, levará o estudante a atingir um objetivo específico e compreender um determinado conceito.

Em seu estudo, \citeauthor{costa1997} propõe duas sequências didáticas distintas para introduzir conceitos trigonométricos. 
%
Seu objetivo é introduzir os conceitos de maneira significativa e, a partir disso, investigar como se deu a construção do conhecimento. 
%
A autora compara os resultados obtidos com o uso das sequências didáticas e afirma que certamente o conteúdo foi melhor absorvido com essas atividades do que da maneira tradicional. 
%
Já \citeauthor{mendes2001} utiliza atividades que envolvem o contexto histórico da Trigonometria como recurso metodológico.
%
O autor apresenta uma forma de repensar as aulas de Matemática conduzindo o aluno pelo caminho da descoberta e construção do conhecimento.
%
\citeauthor{silva2011} faz uso de atividades de modelagem, utilizando recursos computacionais e materiais concretos para levar o conhecimento de Trigonometria para sua sala de aula. 
%
Segundo a autora, além de contribuir para o aprendizado do aluno, a abordagem do conteúdo por meio de de atividades serviu para modificar a forma com que ela própria ensina em suas turmas, e abandonar o modelo definição, exemplo e exercício sempre utilizado em suas aulas.

%Outra possibilidade seria utilizar ferramentas computacionais para .... Falar disso? Não temos foco nisso né?

Diante da problemática mencionada e da necessidade de trabalhar com problemas em variados contextos, conforme sugerido pela BNCC, neste capítulo buscamos apresentar a Trigonometria por meio de atividades que exploram situações práticas de diversos contextos de forma dinâmica, valorizando o pensamento construtivo do estudante e evitando um foco puramente mecânico. 

%Pesquisas, como a de Blackett apontam que o uso de ferramentas computacionais pode ser muito útil no ensino/aprendizagem da trigonometria. XXXXc
%BLACKETT, Norman; TALL, David O. Gender and the versatile learning of trigonometry using computer software. Proceedings of the 15th conference of the International Group for the Psychology of Mathematics Education, 1991, 1, 144–151.

%%%%%%%%%%%%%%%%%%%%%%%%%%%%%%%%%
\section*{Objetivos Gerais}
%%%%%%%%%%%%%%%%%%%%%%%%%%%%%%%%%

De maneira geral, pretendemos levar o aluno a:

\begin{itemize}
\item reconhecer a importância de relações válidas no triângulo para resolver problemas, além das relações métricas do triângulo retângulo;
%
\item compreender as razões trigonométricas de um ângulo agudo a partir da semelhança de triângulos;
%
\item aplicar as razões trigonométricas na resolução de problemas que envolvam triângulos;
%
\item calcular o valor de seno, cosseno e tangente para ângulos com valores entre $0^\circ$ e $180^\circ$;
%
\item conhecer dois importantes resultados da Trigonometria:  Lei dos Senos e a Lei dos Cossenos;
%
\item aplicar a Lei dos Senos e a Lei dos Cossenos na resolução de problemas de diversas áreas do conhecimento.
\end{itemize}

%%%%%%%%%%%%%%%%%%%%%%%%%%%%%%%%%
\subsection*{Pré-requisitos}
%%%%%%%%%%%%%%%%%%%%%%%%%%%%%%%%%
\begin{habilities}{EF06MA19}
Identificar características dos triângulos e classificá-los em relação às medidas dos lados e dos ângulos.

\tcbsubtitle{EF06MA25} Reconhecer a abertura do ângulo como grandeza associada às figuras geométricas.

\tcbsubtitle{EF06MA26} Resolver problemas que envolvam a noção de ângulo em diferentes contextos e em situações reais, como ângulo de visão.

\tcbsubtitle{EF06MA27} Determinar medidas da abertura de ângulos, por meio de transferidor e/ou tecnologias digitais.

\tcbsubtitle{EF07MA24} Construir triângulos, usando régua e compasso, reconhecer a condição de existência do triângulo quanto à medida dos lados e verificar que a soma das medidas dos ângulos internos de um triângulo é $180^\circ$.

\tcbsubtitle{EF09MA11} Resolver problemas por meio do estabelecimento de relações entre arcos, ângulos centrais e ângulos inscritos na circunferência, fazendo uso, inclusive, de softwares de geometria dinâmica.

\tcbsubtitle{EF09MA12} Reconhecer as condições necessárias e suficientes para que dois triângulos sejam semelhantes.

\tcbsubtitle{EF09MA13} Demonstrar relações métricas do triângulo retângulo, entre elas o teorema de Pitágoras, utilizando, inclusive, a semelhança de triângulos.
\end{habilities}

%%%%%%%%%%%%%%%%%%%%%%%%%%%%%%%%%
\subsection*{Distratores}
%%%%%%%%%%%%%%%%%%%%%%%%%%%%%%%%%

A partir de nossa prática docente e das pesquisas da área, apontamos dois principais distratores ligados a este conteúdo:

\begin{itemize}
\item Os estudantes frequentemente não associam o cálculo do seno, cosseno e tangente a um ângulo. 

\item Segundo \citet{weber2005}, os estudantes demonstram dificuldade em estimar seno de ângulos não notáveis, por exemplo, $\sen(20^\circ)$.

\item Segundo \citet{feijo2018}, os estudantes costumam confundir o valor de seno e cosseno de um ângulo.

\item Segundo \citet{silvaneto2006}, os estudantes costumam ter problemas para encontrar, em um triângulo, o lado oposto ou adjacente de um ângulo específico.
\end{itemize}
\end{apresentacao}


\def\currentcolor{session1}
\begin{texto}
{
    \section{Seção 1: Trigonometria no triângulo retângulo}

    Nesta seção são apresentados os fundamentos da Trigonometria. Inicialmente, apresentamos um breve histórico de suas origens, ressaltando a ideia do surgimento da Trigonometria através das necessidades e curiosidades dos nossos antepassados em medir distâncias, comprimentos e ângulos. Depois disso, introduzimos os conceitos básicos da Trigonometria e destacamos algumas situações reais onde podemos utilizá-la. 

    O estudo da Trigonometria está intimamente relacionado com alguns conceitos da Geometria Euclidiana tais como ângulos, semelhança, arcos da circunferência, entre outros. Diante dessa realidade, procuramos levar o aluno a rever alguns desses conceitos nas partes introdutórias desta seção.

    Nesta primeira seção, são introduzidas as razões trigonométricas de ângulos agudos: seno, cosseno e tangente. Também estabelecemos as principais relações e propriedades dessas razões trigonométricas, destacando os seus valores para alguns ângulos específicos, os chamados ângulos notáveis. Tentamos também destacar que os ângulos notáveis raramente aparecem em problemas que surgem na prática. 

    Ao longo do texto, por diversas vezes, o aluno é convidado a fazer as suas próprias reflexões, questionamentos e sugerir respostas para suas perguntas, fazendo-o pensar sobre os conceitos desenvolvidos na seção e como utilizá-los para modelar e resolver problemas que fazem parte do seu cotidiano. Além disso, o texto apresenta atividades que têm por objetivo desenvolver, fixar e mostrar o alcance dos conceitos desenvolvidos nesta seção.

    No decorrer desta seção, será necessário calcular seno, cosseno e tangente de ângulos não notáveis. Para isso, sugerimos utilizar a tabela trigonométrica disponibilizada no final do capítulo. Caso seja possível utilizar meios eletrônicos para os cálculos que envolvem ângulos não notáveis, o professor também pode optar por eles.
}
\end{texto}
\clearmargin
\clearmargin
\begin{objectives}{Alguns quocientes constantes a partir de um ângulo dado}
{
\begin{itemize}
\item Resgatar ideias da Geometria que serão úteis para  desenvolver o conteúdo deste capítulo, como ângulos, triângulos, semelhança de triângulos e etc; relembrar a proporcionalidade existente entre lados de triângulos semelhantes.

\item \textbf{Conceitos abordados}: triângulos retângulos e semelhança de triângulos.
\end{itemize}
}{1}{1}
\end{objectives}
\begin{sugestions}{Alguns quocientes constantes a partir de um ângulo dado}
{
Sugerimos ao professor uma revisão do capítulo {\textit{Semelhança de Triângulos}} antes de utilizar esta atividade em sala de aula. A semelhança de triângulos será primordial para resolver os itens (d) e (e) desta atividade. No item (d), sugerimos ao professor que medie uma discussão entre os estudantes que o leve a perceber a semelhança existente entre os triângulos trabalhados. Já no item (e), é preciso levar o estudante a justificar as igualdades fornecidas no enunciado sem fazer nenhum cálculo e sim usando semelhança de triângulos. 

\textbf{Organização da turma}: individual
}{1}{1}
\end{sugestions}
\begin{answer}{Alguns quocientes constantes a partir de um ângulo dado}
{
\begin{enumerate}
\item Usando o segmento $PQ$ como unidade de comprimento obtemos as seguintes medidas:

\begin{table}[H]
\centering
\begin{tabular}{|c|c|c|c|c|c|}
\hline
$\tmat{AB}$   & $\tmat{AG}$ & $\tmat{AF}$ & $\tmat{BC}$ & $\tmat{GE}$ & $\tmat{FD}$   \\  \hline
$16$ u.c.  & $9$ u.c. & $4$ u.c.  &  $12$ u.c. &  $6,8$ u.c. & $3$ u.c.   \\\hline
\end{tabular}
\caption{Medidas dos triângulos da \Fref{Proporcao1}.}
\end{table}
\end{enumerate}
}{1}
\end{answer}
\clearmargin
\mspace{.25em}
\begin{answer}{Alguns quocientes constantes a partir de um ângulo dado}
{
\begin{enumerate}\setcounter{enumi}{1}
\item A partir dos valores da tabela anterior, segue que:
   
\begin{table}[H]
\centering
\begin{tabular}{|c|e{1.5cm}|e{1.5cm}|e{1.5cm}|} 
\hline
\tcolor{} & $\dfrac{BC}{AB}$    & $\dfrac{GE}{AG}$ & $\dfrac{FD}{AF}$  \tabularnewline 
 \hline
\tcolor{Quociente} & $\dfrac{12}{16}$   & $\dfrac{6,8}{9}$  & $\dfrac{3}{4}$ \tabularnewline
\hline 
\tcolor{Aproximação} & $0{,}75$ & $0{,}755$ & $0{,}75$   \tabularnewline
\hline 
\end{tabular}
\caption{Quocientes e aproximações obtidos a partir dos dados da  \Fref{Proporcao1}.}
\label{Table_quocientes2}
\end{table}

\item{}
Os quocientes $\frac{BC}{AB}, \frac{GE}{AG}$ e $\frac{FD}{AF}$ são aproximadamente iguais.

\item{}
Note que, $C\hat{A}B=E\hat{A}G=D\hat{A}F$ e $A\hat{B}C=A\hat{G}E=A\hat{F}D$. Sendo assim, os triângulos $ABC, AGE$ e $AFD$ possuem ângulos internos congruentes, o que implica que eles são semelhantes e, consequentemente, seus lados correspondentes são proporcionais. Daí, com base no que foi aprendido no capítulo \textit{Semelhança de Triângulos} vale que:
$$\frac{BC}{AB}=\frac{GE}{AG}=\frac{FD}{AF}.$$

Sendo assim, as aproximações encontradas na segunda linha da \Tref{Table_quocientes2} são, na verdade, iguais e não aproximadamente iguais. A pequena variação de valores encontrada é fruto da estimativa da medida do segmento $GE$ como sendo $6,8$u.c.

\item{}
Como foi mencionado no item anterior, os triângulos $ABC, AGE$ e $AFD$ são semelhantes (pois apresentam ângulos correspondentes congruentes). Sendo assim, vale que: 
$$\dfrac{AB}{AC}=\dfrac{AG}{AE}=\dfrac{AF}{AD}.$$
$$\dfrac{BC}{AC}=\dfrac{GE}{AE}=\dfrac{FD}{AD}.$$
\end{enumerate}
}{1}
\end{answer}
\begin{objectives}{Acessibilidade}
{
\begin{itemize}
\item Comparar triângulos em uma situação real
\item \textbf{Conceitos abordados}: triângulos retângulos, semelhança de triângulos e inclinação.
\end{itemize}
}{1}{2}
\end{objectives}
\clearmargin
\begin{sugestions}{Acessibilidade}
{
Nesta atividade será definida a inclinação de uma rampa de acessibilidade e o aluno será convidado a calcular a inclinação de duas rampas. Essas duas rampas possuem medidas diferentes, mas a mesma inclinação. Este fato está ligado à semelhança de triângulos e será ser explorado no item \titem{c)} da atividade. Essa será mais uma oportunidade do aluno trabalhar em uma situação real envolvendo triângulos semelhantes.

\textbf{Organização da turma}: individual.

\textbf{Enriquecimento da discussão}: essa atividade tem como objetivo reiterar ao estudante que, ao observarmos o mundo ao nosso redor, a Matemática está sempre presente. Nesta atividade, o estudante é convidado a pensar em um problema real, onde ele poderá aplicar estratégias simples para solucioná-lo. Além disso, essa atividade revela ao aluno que nos problemas da vida real não há questionamentos prontos do tipo ``faça'', ``determine'' ou ``calcule''. Na verdade, é ele quem deve tomar a decisão sobre o que calcular ou determinar para responder a certos questionamentos.
}{1}{1}
\end{sugestions}
\clearmargin
\begin{answer}{Acessibilidade}
{
\begin{enumerate}


\item{}
Segundo a \Fref{Cadeirante}, o primeiro trecho da rampa possui altura de $30\text{cm}=0,30$m e comprimento horizontal de $3,60$m. Então, a inclinação desta rampa é 
$$\frac{0{,}30}{3{,}60}\cdot 100\approx 8{,}33\%.$$

Já o segundo trecho da rampa possui altura de $80-30=50\text{cm}=0,5$m e comprimento de $6$m. Logo, sua inclinação é dada por 
$$\frac{0{,}50}{6}\cdot 100\approx 8{,}33\%.$$

Sendo assim, o dois trechos da rampa estão de acordo com a norma NBR9050 da ABNT já que possuem inclinação acima de $5\%$.

\item{}
A vista lateral das duas rampas está representada na  \Fref{Rampadef} pelos triângulos retângulos $ABC$ (primeiro trecho) e $DEF$ (segundo trecho).
\begin{figure}[H]
    \centering
    \includegraphics[scale=0.275]{Rampadef.JPG}
    \caption{Vista lateral das rampas de acessibilidade da  \Fref{Cadeirante}.}
    \label{Rampadef}
\end{figure}

\item{}
Note que 
$$\fra{BC}{AB}=\fra{EF}{DE},$$
e por isso, ao fazer os cálculos da inclinação das rampas encontramos o mesmo valor para os dois trechos. 
Logo, os ângulos $A\hat{B}C$ e $D\hat{E}F$ dos triângulos $ABC$ e $DEF$, respectivamente, são congruentes e $\fra{BC}{AB}=\fra{EF}{DE}$, então os triângulos $ABC$ e $DEF$ são semelhantes.

Isto nos leva a concluir que duas rampas possuem a mesma inclinação, se e só semente se, suas vistas laterais são triângulos retângulos semelhantes. 
\end{enumerate}
}{1}
\end{answer}

\explore{Trigonometria, uma Necessidade Humana?}

A Trigonometria lida com as relações entre as medidas dos ângulos e dos lados de um triângulo. O seu surgimento está ligado, possivelmente, à necessidade de nossos antepassados em medir distâncias inacessíveis, como a altura de uma montanha, de um monumento ou do raio terrestre. Esse estudo remonta aos primórdios da história da Mesopotâmia e do Egito, mas alcançou novo patamar a partir do filósofo e matemático grego Tales de Mileto, que viveu no século V a.C. e foi a primeira pessoa na história a quem se atribuem descobertas matemáticas.

Na sociedade moderna, o uso da Trigonometria extrapolou as suas motivações iniciais e passou a fazer parte, por exemplo, dos sistemas modernos de navegação utilizados por aviões e embarcações. Pode não parecer, mas a Trigonometria está muito presente em nossa vida nos dias atuais. Ela está em nossas mãos quando utilizamos um telefone celular para nos guiar para um dado endereço, ou precisamos de uma previsão do valor que pagaremos por uma viagem com um motorista de aplicativo ou mesmo quando jogamos um jogo eletrônico. 

A Trigonometria está espalhada por toda parte! Muitas vezes ela não está tão explícita como outros conceitos matemáticos tais como percentagens, proporcionalidade, áreas e volumes. Porém, com um olhar mais cuidadoso e conhecimento básico dessa teoria, podemos identificá-la em diversas ocasiões e situações do mundo moderno, e em diversas áreas como Engenharia, Arquitetura e Ciências Naturais.

Uma das primeiras aplicações da Trigonometria foi a determinação da medida do raio do planeta Terra feita pelo famoso sábio grego Eratóstenes, conhecido como o ``pai da Geografia''. Eratóstenes, com uma impressionante precisão, apontou que o raio da Terra mede $6.370$ km, como veremos adiante. Ele nasceu em Cirene, na Líbia, em 276 a.C. e passou a maior parte da sua vida em Alexandria, no Egito, tendo sido diretor da sua famosa biblioteca. 

\begin{figure}[H]
    \centering
    \includegraphics[width=.55\linewidth]{Eratostenes.png}
    \caption{Eratóstenes. Fonte: https://bit.ly/3iK6cIW}
    \label{Eratostenes}
\end{figure}

Para obter uma aproximação da medida do raio da Terra, Eratóstenes observou e utilizou os seguintes fatos:
\begin{itemize}
    \item{}
    no primeiro dia de verão na cidade de Siena (atual cidade de Assuã no Egito), o Sol ao meio-dia está na vertical;
    
    \item{}
    no mesmo dia e à mesma hora, na cidade de Alexandria, o ângulo entre uma vara colocada na vertical e a linha que une a extremidade de cima à ponta da sombra é de $\frac{1}{50}$  de uma volta completa, ou seja, $\dfrac{1}{50} \cdot 360^\circ=7,2^\circ$;
    
    \item{}
    Siena fica exatamente ao Sul de Alexandria;
    
    \item{}
    a distância entre as duas cidades é de cerca de $800$km;
    
    \item{}
    os raios luminosos provenientes do Sol são linhas paralelas;
    
    \item{}
    a amplitude do ângulo  Siena—Centro da Terra—Alexandria é igual à do ângulo que os raios de Sol em Alexandria fazem com a vertical.
\end{itemize}

Essas informações estão resumidas na  \Fref{Eratostenes2}:

\begin{figure}[H]
    \centering
    \includegraphics[width=.7\linewidth]{Eratostenes2.jpg}
    \caption{Medindo o raio terrestre. Fonte: https://bit.ly/33a6uEN}
    \label{Eratostenes2}
\end{figure}

 A partir dessas informações, Eratóstenes obteve uma estimativa para a medida do raio terrestre, utilizando uma simples proporção: o ângulo central de medida $7,2^\circ$ corresponde a um arco de comprimento $800$km sobre a superfície terrestre, enquanto que o ângulo central de $360^\circ$ corresponde ao comprimento de uma circunferência que circunda superfície terrestre. Supondo que a medida do raio da Terra (supostamente esférica) seja $R$, o comprimento dessa circunferência é $2\pi R$, onde $\pi \approx 3,14$. Assim,
 $$\frac{800}{7,2^\circ}=\frac{2\pi R}{360^\circ} \iff R \approx \frac{360^\circ \cdot 800}{7,2^\circ \cdot 2 \cdot 3,14}  \iff R\approx 6.370 \text{km}.$$

 O valor aceito atualmente para o raio terrestre é $6.371$km, portanto, o erro cometido por Eratóstenes foi de aproximadamente $1$km, que percentualmente corresponde a $\frac{1}{631} \approx 0,01\%$. Esta estimativa é surpreendente, diante dos poucos recursos disponíveis na época. 
 
 O vídeo disponível no endereço \url{https://www.youtube.com/watch?v=BjO9G4XGpiE} ilustra a determinação da medida do raio da Terra por Eratóstenes.

\newpage
\begin{task}{Alguns quocientes constantes a partir de um ângulo dado}

Diferentemente de outros ramos da Matemática, a Trigonometria muitas vezes não aparece de forma explícita ao nosso redor, mas se pararmos para observar com mais atenção podemos identificar diversas situações onde ela está presente. Por exemplo, a  \Fref{Rampas} mostra uma rampa bastante utilizada nas oficinas de manutenção de automóveis, que é formada por diversos triângulos retângulos.
\begin{figure}[H]
    \centering
    \includegraphics[scale=0.2]{Rampa1.jpg}
    \includegraphics[scale=0.4]{Rampa2.jpg}
    \caption{Rampa para manutenção de automóveis. Fonte: https://bit.ly/32oQgWe}
    \label{Rampas}
\end{figure}
%Falta fonte:https://www.google.com/search?q=lava+jato+rampa&tbm=isch&ved=2ahUKEwiQ0eLU3ODqAhWMBbkGHUi0CdkQ2-cCegQIABAA&oq=lava+jato+rampa&gs_lcp=CgNpbWcQA1CLF1jGH2DdIWgAcAB4AIABhAKIAcwHkgEFMC41LjGYAQCgAQGqAQtnd3Mtd2l6LWltZ8ABAQ&sclient=img&ei=tR4YX9CiLYyL5OUPyOimyA0&bih=937&biw=1920&rlz=1C1SQJL_enBR900BR900#imgrc=vXMxb4NmykjEMM&imgdii=Piq8vyiR1wN8BM

\begin{enumerate}
    \item{}
    A  \Fref{Proporcao1} representa a vista lateral de parte da rampa, onde identificamos os triângulos retângulos $ABC, AGE$ e $AFD$, desenhada sobre uma malha quadriculada.
    \begin{figure}[H]
    \centering
    \includegraphics[scale=0.3]{Proporcao1.JPG}
    \caption{Triângulos retângulos na lateral da rampa.}
    \label{Proporcao1}
\end{figure}

Utilizaremos o segmento $PQ$ como uma unidade de comprimento (u.c.), isto é, $PQ=1$ u.c. Com base na  \Fref{Proporcao1} e utilizando a unidade de comprimento mencionada, obtenha estimativas para os comprimentos dos segmentos indicados na  \Tref{Table_ladostriangulos}.

\begin{table}[H]
\centering
\begin{tabular}{|c|c|c|c|c|c|}
\hline
 $\tmat{AB}$    & $\tmat{AG}$ & $\tmat{AF}$ & $\tmat{BC}$ & $\tmat{GE}$ & $\tmat{FD}$   \\  \hline
     &  &   &   &   &    \\\hline
\end{tabular}
\caption{Medidas dos triângulos da  \Fref{Proporcao1}.}
\label{Table_ladostriangulos}
\end{table}

\item{}
A partir das estimativas obtidas no item anterior, indique na \Tref{Table_quocientes} os quocientes pedidos e calcule suas aproximações. Caso seja necessário, utilize $3$ casas decimais para calcular as aproximações pedidas.

\begin{table}[H]

\centering
\begin{tabular}{|c|e{1.5cm}|e{1.5cm}|e{1.5cm}|} 
\hline
\tcolor{}& $\dfrac{BC}{AB}$    & $\dfrac{GE}{AG}$ & $\dfrac{FD}{AF}$    \tabularnewline  \hline
\tcolor{Quociente} &  &  & \tabularnewline \hline 
\tcolor{Aproximação} &  &  & \tabularnewline \hline 
\end{tabular}
\caption{Quocientes e aproximações obtidos a partir dos dados da \Fref{Proporcao1}.}
\label{Table_quocientes}
\end{table}


\item{}
Analisando a \Tref{Table_quocientes}, o que você percebe?

\item{}
Analise novamente os triângulos da \Fref{Proporcao1} e verifique se há alguma relação entre eles. E agora, você mantém sua resposta ao item anterior?

\item{} Justifique as igualdades a seguir:
$$\dfrac{AB}{AC}=\dfrac{AG}{AE}=\dfrac{AF}{AD},$$
$$\dfrac{BC}{AC}=\dfrac{GE}{AE}=\dfrac{FD}{AD}.$$

\end{enumerate}
\end{task}

\begin{task}{Acessibilidade}
Segundo o Estatuto da Pessoa com Deficiência, cujo objetivo principal é assegurar e promover condições de igualdade a todas as pessoas, todo indivíduo que possui alguma deficiência tem direito à igualdade de oportunidades. 

\begin{figure}[H]
    \centering
    \includegraphics[scale=0.6]{Cadeirante2.JPG}
    \caption{Símbolo universal de acessibilidade. Fonte: \url{https://bit.ly/3lbX3ey}.}
    \label{Cadeirante2}
\end{figure}
%https://assimcomovoce.blogfolha.uol.com.br/2012/05/28/calcadas-para-cadeirantes/


Em termos de acessibilidade, todas as edificações públicas e privadas que se destinam a uso coletivo devem, por lei, ser adaptados à pessoa com deficiência. Para isso, dentre muitas outras coisas, as edificações devem conter as chamadas rampas de acessibilidade. 

Uma rampa de acessibilidade é uma adaptação realizada nas construções que facilita a livre movimentação de cadeirantes e pessoas com mobilidade reduzida, que normalmente não podem fazer uso de escadas. A inclinação de uma rampa é a relação entre sua altura $h$ e seu comprimento horizontal $c$ expressa em porcentagem, ou seja, a inclinação em porcentagem $i$ da rampa é dada por 
$$i=\frac{h}{c}\cdot 100.$$
A \Fref{Rampapercentual} mostra a vista lateral de algumas rampas e suas inclinações. Na primeira delas, por exemplo, a rampa possui altura $1$ e comprimento $20$. Neste caso, a inclinação da rampa é de $5\%$.

\begin{figure}[H]
    \centering
    \includegraphics[scale=3]{Rampapercentual.jpg}
    \caption{Inclinação percentual de uma rampa. Fonte:https://bit.ly/3lbX3ey.}
    \label{Rampapercentual}
\end{figure}



Segundo a norma NBR 9050 da Associação Brasileira de Normas Técnicas (ABNT), para ser considerada uma rampa, sua inclinação deve ser igual ou superior a $5\%$. 

A \Fref{Cadeirante} ilustra  o acesso a um prédio comercial que foi adaptado para pessoas com dificuldade de locomoção. Esse sistema de acessibilidade é constituído por duas rampas com as dimensões fornecidas na figura.

\begin{figure}[H]
    \centering
    \includegraphics[scale=0.7]{Cadeirante.JPG}
    \caption{Rampas de acessibilidade de um prédio comercial. Fonte: https://bit.ly/3guY3XA.}
    \label{Cadeirante}
\end{figure}
%https://www.google.com/search?q=rampa+de+acesso+a+cadeirantes&tbm=isch&ved=2ahUKEwis2bO4i6brAhVRDdQKHR0KB8IQ2-cCegQIABAA&oq=rampa+de+acesso+a+cadeirantes&gs_lcp=CgNpbWcQAzoICAAQBxAFEB5Qz-8DWN2BBGCNhgRoAHAAeACAAdMBiAH4DJIBBTAuNi4zmAEAoAEBqgELZ3dzLXdpei1pbWfAAQE&sclient=img&ei=yHw8X6zVKNGa0AadlJyQDA&bih=888&biw=1920#imgrc=wCGGhfoYIGad2M&imgdii=LnuW4GPLMbnr6M


\begin{enumerate}
\item{}
Calcule a inclinação das duas rampas da \Fref{Cadeirante} e verifique se elas atendem à norma NBR 9050. Caso seja necessário fazer aproximações, utilize duas casas decimais.

\item{}
Faça um desenho que represente a vista lateral de cada uma das rampas da \Fref{Cadeirante}. 

\item{}  
Analisando os dois desenhos feitos no item anterior, é possível encontrar alguma relação entre eles que justifique os valores das inclinações encontrados no item \titem{a)}.
\end{enumerate}

\end{task}

\arrange{As principais razões trigonométricas}

Por meio dos quocientes trabalhados nas atividades anteriores, vamos introduzir algumas razões trigonométricas associadas a um ângulo agudo dado. Essas razões são ferramentas matemáticas simples, mas extremamente úteis para tratar problemas reais, como por exemplo, determinar distâncias que não podem ser medidas diretamente. 

Vamos primeiramente fixar um ângulo agudo $P\hat{O}Q$ de medida $\alpha$. Sobre o lado $OP$ de $P\hat{O}Q$, considere dois pontos quaisquer $A_1$ e $A_2$ e sobre o lado $OQ$, os pontos $B_1$ e $B_2$ de modo que os triângulos $A_1B_1O$ e $A_2B_2O$ sejam retângulos em $B_1$ e $B_2$, respectivamente, como na \Fref{TriagRet1}.

\begin{figure}[H]
    \centering
    \includegraphics[scale=0.35]{TriagRetSem1.JPG}
    \caption{Triângulos retângulos construídos a partir do ângulo $\alpha$.}
    \label{TriagRet1}
\end{figure}

Os triângulos $A_1B_1O$ e  $A_2B_2O$ são semelhantes (por que?), então
\begin{equation}
\frac{A_1B_1}{A_2B_2}=\frac{OB_1}{OB_2}=\frac{OA_1}{OA_2}=k \label{sec1_razaodesem}
\end{equation}
onde $k$ é razão de semelhança entre $A_1B_1O$ e $A_2B_2O$.

Neste caso, temos
\begin{eqnarray}
A_1B_1 & = & k \cdot A_2B_2, \label{sec1_razaodesem_ig1}\\ 
OB_1 & = & k \cdot OB_2,         \label{sec1_razaodesem_ig2} \\
OA_1 & = & k \cdot O A_2.         \label{sec1_razaodesem_ig3}
\end{eqnarray}
De \eqref{sec1_razaodesem_ig1} e \eqref{sec1_razaodesem_ig3} obtemos:
\begin{equation}
\fra{A_1B_1}{OA_1}=\fra{k\cdot A_2B_2}{k \cdot OA_2}=\fra{A_2B_2}{OA_2}. \label{sec1_razaodesem_ig4}
\end{equation}
Dessa forma, a igualdade $\fra{A_1B_1}{OA_1}=\fra{A_2B_2}{OA_2}$ independe da razão de semelhança $k$. 

Neste caso, o que podemos concluir é que independentemente dos triângulos semelhantes escolhidos, a relação \eqref{sec1_razaodesem_ig4} é sempre válida. Isto significa que os quocientes $\fra{A_1B_1}{OA_1}$ e $\fra{A_2B_2}{OA_2}$ são constantes. Na verdade, esses quocientes só serão alterados se for mudada a amplitude do ângulo $\alpha$, ou seja, os quocientes $\fra{A_1B_1}{OA_1}$ e $\fra{A_2B_2}{OA_2}$ dependem apenas do ângulo $\alpha$.

Analogamente, de \eqref{sec1_razaodesem_ig1}, \eqref{sec1_razaodesem_ig2} e \eqref{sec1_razaodesem_ig3}, encontramos
\begin{eqnarray}
\fra{OB_1}{OA_1} & = \fra{k\cdot OB_2}{k \cdot OA_2} & =  \fra{OB_2}{OA_2}, \nonumber\\ 
\fra{A_1B_1}{OA_1} & =\fra{k\cdot A_2B_2}{k \cdot OB_2} & =  \fra{A_2B_2}{OB_2}. \nonumber 
\end{eqnarray}
Daí,
\begin{eqnarray}
\fra{OB_1}{OA_1} = \fra{OB_2}{OA_2}, \label{sec1_razaodesem_ig7}\\ 
\fra{A_1B_1}{OA_1} = \fra{A_2B_2}{OB_2}.  \label{sec1_razaodesem_ig8} 
\end{eqnarray}
não dependem da razão de semelhança $k$. Assim, os quocientes que compõem as relação \eqref{sec1_razaodesem_ig7} e \eqref{sec1_razaodesem_ig8} são também constantes e só dependem do ângulo $\alpha$.
    
Diante disso, cada um dos quocientes \eqref{sec1_razaodesem_ig4}, \eqref{sec1_razaodesem_ig7} e \eqref{sec1_razaodesem_ig8} recebe um nome especial e todos eles compõem o que chamamos de {\textbf{razões trigonométricas  do ângulo $\alpha$}}, visto que esses quocientes dependem exclusivamente do ângulo $\alpha$. O primeiro chamamos de seno do ângulo $\alpha$, o segundo de cosseno do ângulo $\alpha$ e o terceiro de tangente do ângulo $\alpha$. 
    
Para fixar melhor essas ideias e algumas notações, vamos considerar um triângulo  $ABC$ retângulo em $A$, construído a partir de um ângulo agudo $\alpha$, como da \Fref{TriagRet6}.     
\begin{figure}[H]
\centering
\includegraphics[scale=0.9]{sec1_triagretangulo.png}
\caption{Triângulo retângulo construído a partir de um ângulo $\alpha$ dado.}
\label{TriagRet6}
\end{figure}

As razões trigonométricas do ângulo $\alpha$ são seno de $\alpha$, cosseno de $\alpha$ e tangente de $\alpha$, e serão denotadas por $\sen\alpha, \cos \alpha$ e $\tg\alpha$, respectivamente. Usando o triângulo $ABC$, calculamos as razões trigonométricas de $\alpha$ da seguinte forma:
$$\sen\alpha=\fra{AB}{BC}=\fra{c}{a}\;,\quad\cos\alpha=\fra{AC}{BC}=\fra{b}{a} \quad\text{ e }\;\; \tg\alpha=\fra{AB}{AC}=\fra{c}{b}.$$

\begin{observationtitle}{Atenção!}
As definições de seno, cosseno e tangente de um ângulo, apesar de construídas a partir de um triângulo retângulo, independem do triângulo escolhido, como discutido anteriormente. Por isso, calculamos o seno, cosseno e tangente do ângulo $\alpha$ e não do triângulo $ABC$.
\end{observationtitle}

Note que, como o seno e o cosseno de um ângulo agudo $\alpha$ de $ABC$ são obtidos por quocientes entre as medidas de um cateto e da hipotenusa, sempre maior que o cateto, segue que os seus valores são positivos e menores que $1$. Ou seja, 
$$0 < \sen\alpha < 1 \ \ \text{e}  \ \ 0  < \cos\alpha < 1.$$

\begin{observationtitle}{Relação Fundamental}
A partir das definições dadas anteriormente, podemos encontrar outras relações importantes envolvendo o seno, cosseno e tangente de um ângulo agudo. Usando a notação da \Fref{TriagRet6} e o triângulo retângulo $ABC$, podemos observar que
$$\tg\alpha=\fra{AB}{AC}=\fra{c}{b}=\fra{\sen\alpha}{\cos\alpha}.$$

E ainda,
\begin{equation}\label{sec1_relfundamental1}
\sen^2\alpha+\cos^2\alpha=\left(\frac{c}{a}\right)^2+\left(\frac{b}{a}\right)^2=\frac{b^2+c^2}{a^2}
\end{equation}
Como $ABC$ é um triângulo retângulo, pelo teorema de Pitágoras sabemos que $a^2=b^2+c^2$. Logo, a equação \eqref{sec1_relfundamental1} pode ser reescrita da seguinte forma:
\begin{equation}\label{sec1_relfundamental2}
\sen^2\alpha+\cos^2\alpha=1.
\end{equation}

A relação \eqref{sec1_relfundamental2} é chamada de {\textbf{relação fundamental da Trigonometria}}.
\end{observationtitle}

Analisando um pouco mais o triângulo $ABC$, podemos obter as seguintes relações entre as razões trigonométricas do ângulos complementares $\alpha$ e $\beta$:
$$\sen\alpha=\cos\beta=\frac{c}{a}, \quad  \sen\beta=\cos\alpha=\frac{b}{a} \quad \text{e} \quad \tg\alpha=\fra{\sen\alpha}{\cos\alpha}=\frac{\cos\beta}{\sen\beta}=\frac{1}{\tg\beta}.$$
    
\begin{observationtitle}{Observação}
        $\bullet$ De acordo com as definições de seno e cosseno de um ângulo agudo $\alpha$, podemos concluir que num triângulo retângulo construído a partir de $\alpha$ cuja hipotenusa tem medida $a$, os seus catetos medem $a\cdot\cos\alpha$  e $a\cdot\sen\alpha$. Veja a \Fref{TriagRet3}.
    
    \begin{figure}[H]
    \centering
    \includegraphics[scale=1]{sec1_triangret_hipa.png}
    \caption{Catetos de um triângulo retângulo com hipotenusa medindo $a$.}
    \label{TriagRet3}
\end{figure}

$\bullet$ Considere um triângulo equilátero $ABC$ de lado $1$ e um quadrado $EFGH$ de diagonal $1$. No triângulo $ABC$, sabemos que cada uma das suas alturas também é bissetriz  dos seus ângulos internos e mediatriz de cada um dos seus lados, enquanto que, no quadrado, as diagonais são bissetrizes dos seus ângulos internos. Veja a \Fref{TriangEquil1}.

\begin{figure}[H]
    \centering
    \includegraphics[scale=0.4]{TriangEquil1.JPG}
    \caption{Um triângulo equilátero de lado $1$ e um quadrado de diagonal $1$.}
    \label{TriangEquil1}
\end{figure}

Na \Fref{TriangEquil1}, os  triângulos $ADC$ e $EGH$ são retângulos em $D$ e $H$, respectivamente, e suas hipotenusas medem $1$.

Com auxílio dos triângulos $ADC$ e $EGH$ podemos calcular os valores de seno, cosseno e tangente dos ângulos cujas medidas são $30^\circ, 45^\circ$ e $60^\circ$, chamados de ângulos notáveis. Faça esses cálculos e verifique os valores apresentados na \Tref{table_angulosnotaveis} estão corretos.

\begin{table}[H]
\centering
\begin{tabular}{|c|e{1cm}|e{1cm}|e{1cm}|}
\hline
\tcolor{} &  $\tmat{30^\circ}$ &  $\tmat{45^\circ}$ & $\tmat{60^\circ}$ \tabularnewline 
\hline
$\sen$   & $\frac{1}{2}$ & $\frac{\sqrt{2}}{2}$ & $\frac{\sqrt{3}}{2}$ \tabularnewline
\hline
$\cos$   & $\frac{\sqrt{3}}{2}$ & $\frac{\sqrt{2}}{2}$ & $\frac{1}{2}$ \tabularnewline 
\hline
$\tg$   & $\frac{\sqrt{3}}{3}$ & $1$ & $\sqrt{3}$ \tabularnewline
\hline
\end{tabular}
\caption{Seno, cosseno e tangente dos ângulos notáveis.}
\label{table_angulosnotaveis}
\end{table}

\begin{figure}[H]
    \centering
    \includegraphics[scale=0.5]{TriagRet4.JPG}
    \caption{Triângulos retângulos $ADC$ e $EGH$.}
    \label{TriagRet4}
\end{figure}
\end{observationtitle}
    
Apesar de ser comum encontrarmos nos livros didáticos muitos problemas que envolvem os ângulos notáveis, vale a pena destacar que eles não são comumente encontrados nos problemas práticos. Nesses casos, como obter os valores do seno, cosseno e tangente destes ângulos? 

Podemos  determinar esses valores de três formas diferentes. A primeira delas consiste em fazer toda a construção que nos levou a definição de seno, cosseno e tangente de um ângulo. Ou seja, neste caso, é necessário construir um triângulo retângulo que possua o ângulo do problema como ângulo interno, e a partir daí, usar os quocientes que definimos anteriormente para calcular seno, cosseno e tangente do ângulo do problema. A segunda forma que temos é utilizar a tabela trigonométrica do final para consultar valores de seno, cosseno e tangente de diversos ângulos. Os valores presentes nessa tabela são obtidos a partir de métodos mais sofisticados que não são tratáveis na Escola Básica. E, uma terceira forma, é utilizando calculadoras, computadores ou telefones celulares, onde esses valores podem ser obtidos de modo imediato.

%%%%%
\know{As origens da Trigonometria}
 
 Geralmente, atribui-se as origens da Trigonometria aos gregos, dada a sua inegável contribuição na sistematização do conhecimento disponível na época, sobretudo com os trabalhos dos matemáticos Euclides, Tales, Pitágoras, entre tantos outros. Segundo \cite{mansfield2017}, foram os babilônios, e não os gregos, os primeiros a estudar Trigonometria, devido a uma placa de argila com $3700$ anos que contempla o assunto. A placa de argila, conhecida pelo nome científico de {\textit {Plimpton 322}}, foi encontrada pelo arqueólogo e acadêmico Edgar Banks no início do século XX, no local que hoje corresponde ao sul do Iraque. Segundo investigadores da Universidade de Nova Gales do Sul, na Austrália, o conteúdo da placa teria sido, possivelmente, usada por matemáticos antigos para fazerem cálculos na construção de palácios, templos e canais.
 
\begin{figure}[H]
    \centering
    \includegraphics[scale=0.475]{Plimpton.JPG}
    \caption{A placa de argila nomeada {\textit {Plimpton 322}}. Fonte: https://bit.ly/3k1GzVA.}
    \label{Plimpton}
\end{figure}
%https://rcristo.com.br/2018/11/13/conheca-plimpton-322-um-tablete-de-argila-com-escrita-cuneiforme-babilonica-datado-em-3800-anos/

A {\textit {Plimpton 322}}, que a comunidade científica pensa ser proveniente da antiga cidade suméria de Larsa, foi datada do período entre $1822$ a.C. e $1762$ a.C., pertencendo, portanto, à civilização babilônica – o que coloca agora os matemáticos babilônios como os prováveis criadores da Trigonometria, à frente dos gregos.

  
 Apesar de as primeiras noções da Trigonometria serem  bem mais antigas (o grego Hipparchus de Rhodes ($190$ a.C. – $120$ a.C.), considerado o fundador da área, publicou em $180$ a.C. um livro sobre o tema contendo tabelas da primeira função trigonométrica), foi o astrônomo e teólogo alemão Bartholomaeus Pitiscus ($1561$ – $1613$) no seu livro ``Trigonometria: tratado breve e claro da resolução de triângulos'' (em tradução livre do latim), publicado em $1595$, quem mencionou o termo Trigonometria pela primeira vez.  

%%%%% 
\know{Instrumentos de medida}

O ato de medir sempre esteve presente na vida do ser humano, desde os tempos mais remotos. Medir o tamanho de uma propriedade para demarcar terras, a distância entre duas localidades ou entre nosso planeta e astros celestes são exemplos das primeiras medições realizadas por nossos antepassados.

Diante da necessidade de medir objetos ou distâncias, o homem desenvolveu técnicas e instrumentos para realizar tais medidas. Em particular, para realizar suas medições, o homem desenvolveu instrumentos  que foram sendo aprimorados ao longo dos tempos.

Um instrumento de medição muito comum é a trena, como pode ser vista na \Fref{Trena1}. A trena é uma fita ou régua graduada que possui marcações igualmente espaçadas de acordo com alguma unidade de medida. As unidades de medida mais comuns para as trenas são centímetro e metro. Sua forma longa e flexível permite medir objetos longos e curvos. Há registros de que os romanos já possuíam trenas improvisadas por faixas de couro ainda no século VIII a.C.

\begin{figure}[H]
    \centering
    \includegraphics[scale=0.15]{Trena1.jpg}
    \caption{Trena. Fonte: https://bit.ly/2Pc6yvF.}
    \label{Trena1}
\end{figure}
%https://www.google.com/search?q=trena&sxsrf=ALeKk03Af2FHxtDA5MFV_7IkaTwvkJ4shw:1595280956113&source=lnms&tbm=isch&sa=X&ved=2ahUKEwi60dLV5NzqAhWiJ7kGHXucBpIQ_AUoAnoECAwQBA&biw=1920&bih=888#imgrc=P3Kg8LzJ0UHzCM&imgdii=oMl2bqprcH4NsM

Nos dias atuais, são bastante comuns as trenas digitais que podem estimar distâncias utilizando raio laser, além de aplicativos de celulares que utilizam radiações eletromagnéticas para medir distâncias. 

Além de medir distâncias, muito frequentemente temos a necessidade de medir ângulos. Um teodolito é um instrumento ótico que serve para mede ângulos tanto em relação a um plano horizontal pré-fixado, quanto em relação a um plano vertical também pré-fixado. Com o avanço da tecnologia, atualmente, esses instrumentos  oferecem medidas com muita precisão, especialmente nas suas versões digitais. Um teodolito digital pode ser visto na \Fref{Teodolito1}.

\begin{figure}[H]
    \centering
    \includegraphics[scale=0.18]{Teodolito1.jpg}
    \caption{Um teodolito digital. Fonte:https://bit.ly/31awZHo.}
    \label{Teodolito1}
\end{figure}
%https://www.google.com/search?q=teodolito&tbm=isch&ved=2ahUKEwiK0_zX5NzqAhUyIrkGHcOkBo8Q2-cCegQIABAA&oq=teodolito&gs_lcp=CgNpbWcQAzIECCMQJzICCAAyAggAMgIIADICCAAyAggAMgIIADICCAAyAggAMgIIADoFCAAQsQM6BAgAEENQ4p4FWIWsBWCpsgVoAHAAeACAAbIBiAH1CpIBAzAuOZgBAKABAaoBC2d3cy13aXotaW1nwAEB&sclient=img&ei=QA4WX4qzPLLE5OUPw8ma-Ag&bih=888&biw=1920#imgrc=ctwUzgB8BZRi-M&imgdii=JYKz0wWOx4FVXM

As trenas, em geral, são muito comuns nas nossas casas, ao contrário do teodolito que é um aparelho mais sofisticado. O teodolito, em geral, é  utilizado por engenheiros, agrimensores, arquitetos e etc. Apesar disso, é possível construir um teodolito caseiro como o da  \Fref{Teodolito2} com materiais caseiros.

\begin{figure}[H]
\centering
\includegraphics[scale=0.8]{Teodolito2.jpg}
\caption{Um teodolito feito com material caseiro. Fonte:https://bit.ly/2D6uMox.}
\label{Teodolito2}
\end{figure}
\clearpage
\def\currentcolor{session2}
\begin{objectives}{Medindo a altura de uma montanha}
{
\begin{itemize}
\item Utilizar seno, cosseno e tangente de ângulo agudo para resolver situações-problema do mundo real; reconhecer a Trigonometria como uma ferramenta necessária para trabalhar com problemas práticos.

\item \textbf{Conceitos abordados}: triângulos retângulos e razões trigonométricas de um ângulo agudo.
\end{itemize}
}{1}{2}
\end{objectives}
\begin{sugestions}{Medindo a altura de uma montanha}
{
Com a disponibilidade dos recursos digitais dos dias atuais, tais como sites e aplicativos de celulares, um estudante pode, em poucos segundos, responder certas questões que os nossos antepassados investiram um tempo significativo para responder. É preciso conscientizar os estudantes de que apesar da facilidade para responder certas questões proporcionada pelo uso da tecnologia, internamente esses aplicativos utilizam os conceitos desenvolvidos por nossos antepassados, o que justifica a necessidade de entender esses mecanismos e a teoria envolvida. É claro que, na Escola Básica, teremos acesso apenas a parte dos mecanismos e da teoria envolvida.

\textbf{Organização da turma}: individual.

\textbf{Enriquecimento da discussão}: na maioria dos textos que tratam da Trigonometria, as questões são propostas de modo direto, ou seja, são feitas perguntas do tipo ``qual a medida de um certo ângulo ou segmento?'' ou imperativas, tais como ``determine a medida de certo segmento ou ângulo''. Nessa atividade, o aluno será apresentado, de maneira natural, a um determinado problema e será convidado a propor um método para resolvê-lo. Com isso, o aluno terá que elaborar e responder suas próprias questões que, por sua vez, o conduzirão à solução do problema.
}{1}{2}
\end{sugestions}
\begin{answer}{Medindo a altura de uma montanha}
{
No caso específico do Pico do Cabugi, há no seu entorno uma enorme planície, o que facilita o processo de medição da sua altura.  Uma pessoa localizada no ponto $C$ dessa planície em torno da montanha pode, a partir desse ponto, apontar um teodolito na direção do pico da montanha e registrar um ângulo de medida $\theta_1$, conforme ilustra a \Fref{Cabugi2}. Em seguida, essa pessoa, caminhando uma distância $d$ em direção ao pico, chega ao ponto $D$, de onde aponta novamente o teodolito para o pico da montanha; a partir dessa nova posição, o teodolito acusa uma medida $\theta_2$, conforme ilustra a \Fref{Cabugi2}.
\begin{figure}[H]
    \centering
    \includegraphics[scale=0.8]{Cabugi2.JPG}
    \caption{Ilustração do Pico do Cabugi.}
    \label{Cabugi2}
\end{figure}

Os seguintes passos levarão à solução do problema:

\begin{itemize}
    \item[$(a)$] No triângulo $ABC$, temos $\tg\theta_1=\frac{h}{AD+d}$.
    \item[$(b)$] No triângulo $ABD$, temos $\tg\theta_2=\frac{h}{AD}$. Assim, os valores desconhecidos são $h$ e $AD$.
    \item[$(c)$] A partir das igualdades encontradas em $(a)$ e $(b)$, vamos isolar $AD$ em ambas as equações e igualar as expressões, obtendo daí uma equação com uma única incógnita $h$. Da igualdade presente em (a), obtemos:
\begin{equation}
\tg\theta_1=\frac{h}{AD+d} \Rightarrow AD+d=\frac{h}{\tg\theta_1}\Rightarrow AD=\frac{h}{\tg\theta_1}-d, \label{um}
\end{equation}
e, da igualdade encontrada em $(b$), temos:
\begin{equation}
\tg\theta_2=\frac{h}{AD} \Rightarrow AD=\frac{h}{\tg\theta_2}. \label{dois}
\end{equation}
Igualando os valores de $AD$ encontrados em \eqref{um} e \eqref{dois}, segue que:
$$\frac{h}{\tg\theta_2}=\frac{h}{\tg\theta_1}-d \Rightarrow h=\left(\frac{\tg\theta_1.\tg\theta_2}{\tg\theta_2-\tg\theta_1}\right)\cdot d.$$
\end{itemize}
    
Assim, podemos determinar a medida da altura do pico em função dos valores $d, \theta_1$  e $\theta_2$  que podem ser encontrados com auxílio de uma trena e de um teodolito caseiro.
    
Apenas por curiosidade, o Pico do Cabugi possui cerca de $590$ metros de altura.
}{9}
\end{answer}
\begin{objectives}{Pitágoras trigonométrico}
{
\begin{itemize}
\item Utilizar as razões trigonométricas de ângulos agudos e áreas de figuras planas para demonstrar o teorema de Pitágoras.
\item \textbf{Conceitos abordados}: razões trigonométricas, áreas e teorema de Pitágoras.
\end{itemize}
}{1}{2}
\end{objectives}
\begin{sugestions}{Pitágoras trigonométrico}
{
\textbf{Organização da turma}: individual.

\textbf{Enriquecimento da discussão}: existem muitas demonstrações do teorema de Pitágoras. Uma das mais antigas é a que está no livro Os Elementos do matemático grego Euclides (Proposição 47, Livro I). Euclides utiliza  equivalência de áreas para demonstrar o teorema de Pitágoras (na verdade, seu enunciado já é uma equivalência de áreas). Essa atividade é uma adaptação da ideia da demonstração realizada por Euclides.

}{1}{1}
\end{sugestions}
\begin{answer}{Pitágoras trigonométrico}
{
\begin{enumerate}
    \item{} 
    No  triângulo retângulo $ABC$, tem-se que:
    $$\cos \alpha=\fra{b}{a} \ \ \text{e} \ \ \cos \beta=\fra{c}{a}.$$
   
    \item{}
    No triângulo $ACL$, $\cos \alpha=\fra{CL}{b} \iff CL=b\cdot\cos \alpha$. Analogamente, no triângulo $ABL$, 
    $\cos \beta=\frac{BL}{c} \iff BL=c\cdot\cos \beta$.
    
    \item{} 
    Ora, $S_1=c^2$ e $S_2=b^2$. Por outro lado, como $\cos \alpha=\frac{b}{a}$ e  $\cos\beta=\frac{c}{a}$, 
    o retângulo $BDJL$ tem lados $BL=c\cdot\cos \beta$ e $BD=a$ e o retângulo $CEJL$ tem lados $CL=b\cdot\cos\alpha$ e $CE=a$, segue que:
    $$\text{Area}(BDJL)=ac\cdot\cos\beta=ac\fra{c}{a}=c^2=S_1,$$
    $$\text{Area}(CEJL)=ab\cdot\cos\alpha=ab\fra{b}{a}=b^2=S_2.$$
    
    \item{}
    Por fim, como $S=a^2$ e $S=S_1+S_2$, segue que:
    $$S=S_1+S_2 \iff a^2=b^2+c^2.$$
\end{enumerate}

Caso tenha acesso a Internet (inclusive de um celular), você pode acessar algumas visualizações da demonstração do teorema de Pitágoras em 
\url{https://www.geogebra.org/m/YA724k8j} e 
\url{https://www.youtube.com/watch?v=CAkMUdeB06o}.
}{1}
\end{answer}
\clearmargin
\begin{objectives}{Largura de uma lagoa}
{
\begin{itemize}
\item Utilizar seno, cosseno e tangente de ângulo agudo para resolver situações-problema do mundo real; reconhecer a Trigonometria como uma ferramenta necessária para trabalhar com problemas práticos.
\item \textbf{Conceitos abordados}: razões trigonométricas de um ângulo agudo e escalas.
\end{itemize}
}{1}{2}
\end{objectives}
\begin{sugestions}{Largura de uma lagoa}
{
Nesta atividade será utilizado o valor de $\tg (85^\circ)=11,43$, que pode ser encontrado na tabela trigonométrica do fim do capítulo. 

\textbf{Organização da turma}: individual,

\textbf{Enriquecimento da discussão}: nessa atividade, o aluno necessitará de régua graduada em centímetros e de um transferidor. O aluno é levado nessa atividade a usar esses instrumentos para resolver um problema real, usando a escala sugerida no mapa. Essa situação problema é interessante, pois mostra como esses instrumentos são úteis e não utilizados apenas para problemas presentes nos livros escolares. Além disso, a atividade põe o aluno numa posição não usual onde ele precisa propor um método ou caminho para resolver um problema.
}{1}{2}
\end{sugestions}
\begin{answer}{Largura de uma lagoa}
{
\begin{enumerate}
 \item{}
Com um teodolito posicionado em $A$, miraríamos o ponto $B$. Em seguida, giraríamos o teodolito sobre o plano horizontal até alcançar um ângulo de $90^\circ$ no sentido horário para quem vê a figura na posição do leitor). Com o teodolito nessa posição, escolheríamos um ponto $C$ de modo que pudéssemos medir a distância $\ell$ de $A$ até $C$, como ilustra a  \Fref{Lagoa2}.
    \begin{figure}[H]
    \centering
    \includegraphics[scale=0.3]{Lagoa2.JPG}
    \caption{Lagoa Rodrigo de Freitas situada na cidade do Rio de Janeiro. Fonte: Google Maps.}
    \label{Lagoa2}
    %Fonte:https://www.google.com.br/maps/place/Lagoa+Rodrigo+de+Freitas/@-22.971588,-43.2178442,15z/data=!3m1!4b1!4m5!3m4!1s0x9bd574afbc853f:0x20a26959ca6918cd!8m2!3d-22.9738464!4d-43.2110285
\end{figure}
Levaríamos o teodolito até o ponto $C$, miraríamos, inicialmente, o ponto $A$ e o giraríamos no sentido horário até o ponto $B$, encontrando o ângulo $A\hat{C}B=\theta$. Sendo assim, $\tg\theta=\frac{d}{\ell} \iff d=\ell\cdot\tg\theta$.

\item{} 
Seguindo o método do item anterior, precisamos utilizar a tangente do ângulo de $85^\circ$ para calcular $d$. Segundo a tabela trigonométrica do final do capítulo, podemos aproximar a tangente de $85^\circ$ por $11,43$, e assim:
$$\tg(85^\circ)=\frac{d}{\ell} \iff d=\ell\cdot\tg(85^\circ) = 200 \cdot 11,43 \approx 2.286\text{m}.$$

\item{} Apoiando a régua sobre o mapa para fazer a medição, encontramos que distância entre os pontos $A$ e $B$ é de aproximadamente $6,7$cm.
   
\item{} Ora, como o segmento de referência da escala possui $1,5$cm (confirme essa medida com a sua régua) e corresponde a uma distância real de $500$m, segue que:
$$\frac{1,5 \text{cm}}{6,7 \text{cm}}=\frac{500 \text{m}}{d} \iff 1,5d=6,7\cdot 500 \iff d \approx 2.233,33 \text{m}.$$

\item{} Usando a escala do Google Maps, a distância estimada entre os pontos $A$ e $B$ é de $2.233,33$m e a estimativa feita com os instrumentos de medida foi de $2.286$m. Entre essas duas estimativas há uma diferença de $2.286  - 2.233,33 =52,67$m. O erro percentual, nesse caso, pode ser obtido da seguinte forma:
$$\frac{2.233,33 \text{m}}{52,67 \text{m}}=\frac{100\%}{x} \iff 2.233,33x=52,67\cdot 100 \iff x\approx 2,35 \%.$$
\end{enumerate}
}{9}
\end{answer}

\practice{Trigonometria no triângulo retângulo}

\begin{task}{Medindo a altura de uma montanha}

Já dissemos que, desde os tempos mais remotos, uma das principais motivações para o desenvolvimento da Trigonometria foi a necessidade de efetuar medições, especialmente obter medidas que fossem inacessíveis diretamente, tais como a medida do raio da Terra, a altura de uma montanha, a largura de um rio, entre outras. Nesta atividade, convidamos você a propor uma possível solução para um problema clássico, supondo que você tenha à mão instrumentos de medidas tais como uma trena e um teodolito caseiro.

No interior do estado do Rio Grande do Norte, uma das montanhas mais famosas é o Pico do Cabugi, localizado no município de Angicos. Sabendo que há uma vasta planície em torno dessa montanha, como você poderia utilizar uma trena e um teodolito para estimar a altura dessa montanha? 

\begin{figure}[H]
    \centering
    \includegraphics[scale=0.8]{Cabugi1.JPG}
    \caption{Pico do Cabugi,  Angicos - RN. Fonte: https://bit.ly/3lhedri.}
    \label{Cabugi1}
\end{figure}
% Fonte: https://bit.ly/3lhedri
\end{task}

\begin{task}{Pitágoras trigonométrico}
O teorema de Pitágoras, provavelmente, está entre os resultados mais populares de toda a Matemática. O teorema afirma que, num triângulo retângulo $ABC$ cujas medidas da hipotenusa e dos seus catetos são, respectivamente, $a, b$ e $c$, tem-se que $a^2=b^2+c^2$. Geometricamente, isso significa que construindo quadrados sobre a hipotenusa e sobre os catetos deste triângulo retângulo, a área do quadrado sobre a hipotenusa, que nomearemos por $S$, é igual à soma das áreas dos quadrados construídos sobre os catetos, que nomearemos por $S_1$ e $S_2$, conforme ilustra a \Fref{Pitagoras1}.
\begin{figure}[H]
    \centering
    \includegraphics[scale=0.4]{Pitagoras1.JPG}
    \caption{O teorema de Pitágoras: $S=S_1+S_2$.}
    \label{Pitagoras1}
\end{figure}
Ao longo dos tempos, surgiram diversas demonstrações desse teorema. A seguir, vamos utilizar as  razões trigonométricas para deduzí-lo.

\begin{enumerate}
    \item{}
    Sejam $A\hat{C}B=\alpha$ e $A\hat{B}C=\beta$. Determine os valores de $\cos \alpha$ e $\cos \beta$.
    
    \item{}
    Considerando o segmento $AJ$ perpendicular à $DE$, seja $L$ o ponto de interseção de $AJ$ com $BC$, conforme ilustra a \Fref{Pitagoras1.1}.
    \begin{figure}[H]
    \centering
    \includegraphics[scale=0.4]{Pitagoras2.JPG}
    \caption{O teorema de Pitágoras: $S=S_1+S_2$.}
    \label{Pitagoras1.1}
\end{figure}
    Obtenha as medidas dos segmentos $BL$ e $CL$ em função das medidas $\alpha, \beta, b$ e $c$.
    
    \item{}
    Qual a relação entre as medidas das áreas dos retângulos $BDJL$ e $CEJL$ e as áreas $S_1$ e $S_2$?
    \item{}
    Usando o fato de que a medida da área do quadrado $BCED$ é igual à soma das medidas dos retângulos $BDJL$ e $CEJL$, conclua que $a^2=b^2+c^2$.
\end{enumerate}
\end{task}

\begin{task}{Largura de uma lagoa}
A \Fref{Lagoa1} mostra uma tela do aplicativo {\textit{Google Maps}} contendo o mapa da região da Lagoa Rodrigo de Freitas situada na cidade do Rio de Janeiro. Nosso objetivo é estimar a distância entre dois pontos fixados na borda dessa lagoa.

Para este exercício, você precisará de uma régua graduada e um transferidor. E caso seja necessário, aproxime valores decimais com duas casas decimais.
\begin{figure}[H]
    \centering
    \includegraphics[scale=0.45]{Lagoa1.JPG}
    \caption{Lagoa Rodrigo de Freitas situada na cidade do Rio de Janeiro. Fonte: Google Maps.}
    \label{Lagoa1}
\end{figure}
%Fonte:https://www.google.com.br/maps/place/Lagoa+Rodrigo+de+Freitas/@-22.971588,-43.2178442,15z/data=!3m1!4b1!4m5!3m4!1s0x9bd574afbc853f:0x20a26959ca6918cd!8m2!3d-22.9738464!4d-43.2110285

\begin{enumerate}
\item{}
Na \Fref{Lagoa1}, mostramos os pontos $A$ e $B$ fixados sobre a borda da lagoa. Supondo que você dispusesse de instrumentos de medida adequados (trena e teodolito, por exemplo), proponha um método para estimar a distância $d$ entre esses dois pontos.

\item{}
Seja $C$ um ponto sobre a borda da lagoa tal que $B\hat{A}C=90^\circ$ (medido com o teodolito). Considere ainda que a distância $\ell$ (medida com a trena) entre os pontos $A$ e $C$ seja de $200$m, e que o ângulo $A\hat{C}B$ mede $85^\circ$ (medido com o teodolito). Use o método que você sugeriu no item anterior para estimar um valor para a distância entre os pontos $A$ e $B$.

\item{}
Utilizando uma régua graduada em centímetros e o mapa da \Fref{Lagoa1}, encontre uma estimativa em centímetros para a distância entre $A$ e $B$.

\item{}
No canto inferior direito da \Fref{Lagoa1}, podemos encontrar a escala utilizada para a construção do mapa pelo aplicativo. Neste caso, a escala utilizada é $1,5$cm para representar $500$m. Usando a distância entre os pontos $A$ e $B$ do item anterior, e essa escala, dê uma estimativa para a distância real entre os pontos $A$ e $B$.

\item{}
Tomando a estimativa da distância entre os pontos $A$ e $B$ obtida no item \titem{d)} como referência, qual o erro percentual para a mesma distância que você determinou no item \titem{b)}.
\end{enumerate}
\end{task}

\know{Novamente a medida do raio da Terra}

Adaptamos um texto escrito pelos professores Eduardo Wagner e Marcos Paulo para o Livro Aberto para propor uma nova discussão sobre a medida do raio da Terra.

No início desta seção, discutimos como Eratóstenes determinou com uma excelente precisão a medida do raio terrestre. Agora, utilizando os conceitos estudados nesta seção, vamos mostrar  uma nova forma para estimar a medida do raio da Terra. 
        
Para isso, precisamos levar o teodolito para um lugar alto e que conheçamos sua altura em relação ao nível do mar. Com uma única medição você vai se surpreender com o poder da Trigonometria para resolver esse problema. Mas um pouco de imaginação, ou criatividade, é também necessária!

Vamos considerar a Terra como uma esfera de centro $C$ e raio $R$, e um ponto $P$ situado a uma altura $h$ em relação ao nível do mar. A reta que é definida pelos pontos $C$ e $P$ é dita a reta vertical que passa pelo ponto $P$ e qualquer reta perpendicular a $CP$ é chamada de reta horizontal. Portanto, na \Fref{RaioTerra}, a reta $PX$, perpendicular a $CP$, é uma reta horizontal passando pelo ponto $P$.

   \begin{figure}[H]
    \centering
    \includegraphics[scale=0.2]{RaioTerra1.JPG}
    \caption{Cálculo da medida do raio da Terra.}
    \label{RaioTerra}
\end{figure}



Considere um teodolito localizado em $P$. Em um dia claro, um observador em $P$ apontando o teodolito na direção $PX$ só vê o céu. Dessa mesma posição, girando ligeiramente o teodolito, ele vê um ponto $H$ sobre a linha do horizonte (diremos que a linha do horizonte é a linha onde o céu encontra o mar). Com auxílio do teodolito, é possível medir o ângulo $X\hat{P}H=\alpha$.

Note que $P\hat{C}H=X\hat{P}H=\alpha$. Assim, no triângulo $PCH$ temos:
\begin{equation}\label{raioT}
\cos\alpha=\frac{R}{R+h} \Rightarrow R=\frac{h\cdot\cos\alpha}{1-\cos\alpha}.
\end{equation}

No Rio de Janeiro, por exemplo, no famoso morro do Corcovado, a base da estátua do Cristo Redentor está aproximadamente $710$m acima do nível do mar. Posicionando o teodolito no pátio do entorno dos pés do Cristo, é possível ver o horizonte em diversas direções. Em uma delas, o teodolito registrou $\alpha=26^\circ$. Como $h=710$m, usando \eqref{raioT}, um valor aproximado para o raio da Terra é 
$$R=\frac{710\cdot\cos(26^\circ)}{1-\cos(26)^\circ}=\frac{710\cdot0,8988}{1-0,8988}\approx 6.305,8102\text{km}.$$

\begin{paginatexto}{Seção 2: Leis da Trigonometria}
Nesta seção, o objetivo é explorar relações entre lados e ângulos pertencentes a triângulos acutângulos e obtusângulos, mostrando ao estudante que podemos ir além do teorema de Pitágoras e das razões trigonométricas dos ângulos agudos exploradas na seção anterior.

Até aqui, as definições de seno, cosseno e tangente de um dado ângulo agudo foram apresentadas apoiadas em uma família especial de triângulos (a saber: triângulos retângulos construídos a partir do ângulo dado); agora, o desafio é transcender o triângulo retângulo e levar o estudante a trabalhar também com ângulos obtusos. 
%
Nesse sentido, esta seção apresentará a definição de seno, cosseno e tangente para ângulos de $0^\circ$ até $180^\circ$. 
%
O professor que sentir necessidade de aprofundamento da discussão acerca desta definição deverá usar o \textit{Para Saber+ Explorando Senos e Cossenos}.

O estudante também será apresentado nesta seção a dois resultados muito importantes da Trigonometria. 
%
São eles: Lei dos Cossenos e Lei dos Senos.
%
Estes dois resultados serão demonstrados e sua importância será comprovada através de uma seleção de problemas de diversos contextos.
%
As demonstrações das duas leis serão apresentadas sempre para os três tipos possíveis de triângulos: retângulo, acutângulo e obtusângulo, e, caberá ao professor decidir se deve ou não abordar as demonstrações em sua aula, podendo optar também por apenas algumas delas.

Em geral, os livros didáticos fazem uso do círculo trigonométrico para definir seno, cosseno e tangente de um ângulo qualquer.
%
A proposta deste capítulo não é usar este tipo de estratégia, mas apenas trabalhar com os ângulos internos de um triângulo qualquer (portanto, entre $0^\circ$ e $180^\circ$).
%
Deixaremos a abordagem do círculo trigonométrico, a definição de radiano e outros conceitos relacionados para o capítulo que abordará as funções trigonométricas.

Sobre a linguagem usada em sala de aula, sugerimos que os estudantes sejam incentivados a usar o vocabulário adequado no contexto de triângulos, em especial, em relação a sua classificação. 
%
Percebemos, em nossa prática docente, que triângulos retângulos são comumente chamados de retângulos, enquanto triângulos acutângulos ou obtusângulos são comumente chamados, apenas, de triângulos.
%
A familiaridade com os termos adequados permitirá que o estudante leia o texto com maior fluidez e compreenda melhor o conteúdo.

No decorrer da seção, frequentemente será necessário calcular seno, cosseno e tangente de ângulos não notáveis, ainda que agudos. 
%
Para estes casos, sugerimos utilizar a tabela trigonométrica disponibilizada no final do capítulo. %
Com o auxílio desta tabela, o estudante poderá obter aproximações dos valores do seno, cosseno e tangente de diversos ângulos e também poderá encontrar o ângulo que possui um determinado valor de seno ou cosseno ou tangente inspecionando a tabela. Neste segundo caso, em especial, evitaremos fazer qualquer menção às funções trigonométricas inversas, que não fazem parte do escopo deste capítulo.
%
Caso seja possível utilizar formas eletrônicas para os cálculos que envolvem ângulos não notáveis, o professor também poderá optar por utilizá-los.
\end{paginatexto}

\def\currentcolor{session1}
\begin{objectives}{Explorando um triângulo não retângulo}
{
\begin{itemize}
\item Trabalhar com o teorema Pitágoras, reafirmando sua aplicabilidade apenas para triângulos retângulos; desenvolver uma relação métrica válida em um triângulo obtusângulo que dará origem, mais adiante, à lei dos cossenos.
\item \textbf{Conceitos abordados}: relações métricas em triângulos retângulos.
\end{itemize}
}{1}{1}
\end{objectives}
\begin{sugestions}{Explorando um triângulo não retângulo}
{

Para esta atividade, o estudante deverá utilizar uma régua para fazer as construções pedidas.

\textbf{Organização da turma}: em duplas ou pequenos grupos para que os alunos possam se envolver nas discussões propostas.

\textbf{Dificuldades previstas}: é muito comum que os estudantes usem o teorema de Pitágoras em um triângulo qualquer, não atentando para o fato de sua validade estar ligada ao triângulo ser retângulo. Sugerimos ao professor ficar atento a este distrator e corrigir erros relacionados, caso apareçam, aproveitando a situação para uma revisão do conteúdo já estudado.

\textbf{Enriquecimento da discussão}: esta atividade pretende despertar o interesse do aluno na busca por um resultado que estenda o teorema de Pitágoras para triângulos quaisquer. Provavelmente, o aluno nunca foi levado a questionar essa possibilidade; com essa atividade, pretendemos que ele comece a desconfiar da existência de algum resultado ainda por ele desconhecido. Além disso, sugerimos ao professor que permita e incentive o aluno a se expressar livremente e que aproveite a oportunidade para ajudá-lo a refinar seu vocabulário no contexto dos triângulos (tipos, ângulos internos e externos, ângulos complementares e suplementares, altura e etc).  
}{1}{1}
\end{sugestions}
\begin{answer}{Explorando um triângulo não retângulo}
{
\begin{enumerate}
    \item{}
    Na  \Fref{sec2_trignaoret_resl_fig}, apresentamos uma possível construção para o triângulo $ABC$. Independentemente da construção realizada, o triângulo $ABC$ não é retângulo. Como o ângulo $B\hat{A}C$ é obtuso, então os demais ângulos internos desse triângulo possuem medida menor que $90^\circ$. Do contrário, as medidas dos ângulos internos de $ABC$ não somariam $180^\circ$.
    
    \begin{figure}[H]
    \centering
    \includegraphics[scale=0.6]{sec2_explorando_trig_nao_ret_res1.png}
    \caption{Triângulo não retângulo $ABC$.}
    \label{sec2_trignaoret_resl_fig}
\end{figure}
    
    \item{}
    Na \Fref{sec2_trignaoret_res2_fig}, $BD$ é a  altura do triângulo $ABC$ traçada a partir de $B$. Neste caso, como o ângulo $B\hat{A}C$ é obtuso, a altura $BD$ é externa ao triângulo.
    \begin{figure}[H]
    \centering
    \includegraphics[scale=0.6]{sec2_explorando_trig_nao_ret_res2.png}
    \caption{Altura $BD$ do triângulo $ABC$ a partir de $B$.}
    \label{sec2_trignaoret_res2_fig}
\end{figure}
    
    \item{}
    Neste item, usaremos a  \Fref{sec2_trignaoret_res3_fig} que contém a notação apresentada no enunciado.
   
    Aplicando o teorema de Pitágoras no triângulo $DBA$ retângulo em $D$, obtemos 
    \begin{equation}
     c^2=h^2+u^2.   \label{sec2_at_trignaoret_eq1}
    \end{equation}
    
    Agora, fazendo o mesmo cálculo para o triângulo $DBC$ retângulo em $D$, obtemos 
    \begin{equation}
     a^2=h^2+(u+b)^2.   \label{sec2_at_trignaoret_eq2}
    \end{equation}
    \begin{figure}[H]
    \centering
    \includegraphics[scale=0.6]{sec2_explorando_trig_nao_ret_res3.png}
    \caption{Triângulo $ABC$ e sua altura $BD$.}
    \label{sec2_trignaoret_res3_fig}
\end{figure}
    
    \item{}
    O triângulo $ABC$ não é retângulo, portanto não podemos usar o teorema de Pitágoras para encontrar uma relação como a encontrada no item anterior para os triângulos $DBA$ e $DBC$, que são retângulos. 
    
    Neste caso, como pede o enunciado, vamos encontrar uma relação envolvendo $a, b, c$ e $u$ para o triângulo $ABC$. Para isso, de \eqref{sec2_at_trignaoret_eq1} temos que $h^2=c^2-u^2$. Substituindo essa igualdade em \eqref{sec2_at_trignaoret_eq2}, vemos que
    $$a^2=c^2-u^2+u^2+2bu+b^2 \iff a^2=b^2+c^2+2bu,$$
    
    que é a relação pedida.
    
    \item{}
    As relações provenientes da aplicação do teorema de Pitágoras aos triângulos retângulos $DBA$ e $DBC$ envolvem apenas as medidas dos lados dos triângulos, enquanto que a relação obtida no item anterior utiliza (além das medidas dos lados) uma medida $u$ externa ao triângulo, mas determinada por ele. 
    
    Para responder a segunda pergunta, é necessário buscar uma forma de expressar $u$ em termos dos dados do triângulo. Veremos mais adiante que, para isso, precisaremos envolver os ângulos do triângulo. 
\end{enumerate}
}{9}
\end{answer}
\clearmargin
\begin{objectives}{Desenvolvendo um jogo de futebol para videogame}
{
Perceber a necessidade de estender para triângulos não necessariamente retângulos a teoria já estudada, indo além do teorema de Pitágoras; aplicar a argumentação utilizada na demonstração da lei dos cossenos, que será sistematizada adiante, na solução de um problema numérico e adquirir familiaridade com este tipo de argumentação antes dela ser utilizada.  

\textbf{Conceitos abordados}: razões trigonométricas de um ângulo agudo.
}{1}{2}
\end{objectives}
\begin{sugestions}{Desenvolvendo um jogo de futebol para videogame}
{
Nesta atividade serão utilizados os seguintes dados, que poderão ser acessados pelos estudantes na tabela trigonométrica do fim do capítulo: $\arcsen({0,82}) = 55^\circ, \arcsen({0,089}) = 5^\circ$ e $\arccos({0,98})=10,29^\circ$. Para encontrar o ângulo que possui determinado valor de seno e cosseno, será necessário utilizar a tabela de maneira contrária à tradicional (quando queremos encontrar o valor de seno e cosseno de um dado ângulo). Ou seja, o estudante deverá procurar na coluna seno e cosseno o valor encontrado na atividade e então, identificar na coluna dos ângulos qual o ângulo relacionado ao dado valor de seno e cosseno. É importante lembrar que, como já foi dito anteriormente, o professor não deve utilizar a linguagem e nomenclatura de função trigonométrica inversa em sua sala de aula. Destacamos esses dados, neste momento, para informar ao professor o que será utilizado na atividade e com o intuito de auxiliá-lo na manipulação da tabela.

\textbf{Organização da turma}: em duplas ou pequenos grupos para que os alunos possam ter espaço para se envolver nas discussões propostas.

\textbf{Enriquecimento da discussão}: essa atividade deve ser usada para destacar novamente que o teorema de Pitágoras só é válido para triângulos retângulos e para guiar o estudante a concluir que uma ampliação deste teorema necessariamente deve envolver também algum dos ângulos do triângulo, diferentemente do teorema de Pitágoras. É muito importante que o professor guie e fomente a discussão em torno dessas questões.
}{1}{2}
\end{sugestions}
\clearmargin
\mspace{.25em}
\begin{answer}{Desenvolvendo um jogo de futebol para videogame}
{\paragraph{Parte I}

\begin{enumerate}\small
    % 

    \item Visualmente, provavelmente, será difícil decidir qual ângulo é o maior. Entretanto, nesse item podemos incentivar que os estudantes conversem, discutam e conjecturem entre si.
    
    \item{}
    Vamos primeiramente calcular o ângulo do chute ao gol do jogador $A$. Como o triângulo $AT_1T_2$ não é um triângulo retângulo, não podemos usar o teorema de Pitágoras diretamente, tão pouco as razões trigonométricas aprendidas na seção anterior. Vamos então, buscar triângulos retângulos presentes na situação para aplicarmos a teoria que conhecemos, e indiretamente calcular o ângulo procurado.
    
    Vamos usar a \Fref{sec2_futebol_res1_fig} para nos auxiliar nesse item. Seja $M$ o ponto posicionado sobre a junção das linhas que delimitam o campo de futebol.

    \notas{
\adjustbox{valign=t}
{
        \begin{minipage}{\linewidth}
        \begin{figure}[H]
                \centering
                \includegraphics[scale=0.3]{sec2_trig1_atfutebol.png}
                \caption{Triângulo $AT_1T_2$. (Atividade \hyperref[desenvolvendo-jogo]{Desenvolvendo um jogo de futebol para videogame})}
                \label{sec2_futebol_res1_fig}
            \end{figure}
        \end{minipage}
}
        }

    Note que o triângulo $MAT_1$ é retângulo em $M$ e isósceles, já que $AM$ e $MT_1$ são congruentes (ver \Fref{sec2_futebol4_fig}). Sendo assim, o ângulo $M\hat{A}T_1$ mede $45^\circ$. Além disso, como $MAT_2$ é retângulo (e não isósceles, já que este triângulo não possui dois lados congruentes), então 
    $$\sen(M\hat{A}T_2)=\frac{MT_2}{AT_2}=\frac{16,5+7,3}{28,96}=0,821.$$
    Neste caso, o resultado foi aproximado utilizando duas casas decimais.
    
    Com o auxílio da tabela trigonométrica do final do capítulo, é possível determinar que $M\hat{A}T_2$ mede aproximadamente $55^\circ$. Portanto, $T_1\hat{A}T_2=M\hat{A}T_2-M\hat{A}T_1$ mede aproximadamente $55^\circ-45^\circ=10^\circ$.
    
    Sendo assim, o jogador $A$ possui um ângulo certeiro ao gol de aproximadamente $10^\circ$.
    
    \notas{
\adjustbox{valign=t}
{
        \begin{minipage}{\linewidth}
        \begin{figure}[H]
            \centering
            \includegraphics[scale=0.2]{sec2_trig2_atfutebol.png}
            \caption{Triângulo $BT_1T_2$. (Atividade \hyperref[desenvolvendo-jogo]{Desenvolvendo um jogo de futebol para videogame})}
            \label{sec2_futebol_res2_fig}
        \end{figure}
        \end{minipage}
}
        }
    
    Trabalhando com a situação do jogador $B$, ilustrada na \Fref{sec2_futebol_res2_fig}, nos deparamos com um triângulo isósceles $BT_1T_2$. Marque $N$ como sendo o pé da perpendicular baixada de $B$ sobre $T_1T_2$. Dessa forma, $BN$ é a altura do triângulo em relação à base $T_1T_2$, que também é a bissetriz do ângulo $B$. Assim,
    $$\sen(N\hat{B}T_1)=\sen(N\hat{B}T_2)=\fra{NT_1}{BT_1}=\fra{3,65}{40,82}=0,089.$$
    Novamente com o auxílio da tabela trigonométrica do final do capítulo, concluímos que $N\hat{B}T_1$ e $N\hat{B}T_2$ medem aproximadamente $5^\circ$. Portanto, $T_1\hat{B}T_2$ mede aproximadamente $10^\circ$.
    
    Pelos cálculos acima, o jogador $B$ possui um ângulo certeiro ao gol de $10^\circ$.
    
    Nos casos dos dois jogadores $A$ e $B$, utilizamos aproximações com três casas decimais para medir distâncias e calcular o valor de seno do ângulo. Podemos, então, concluir que os jogadores $A$ e $B$ possuem, aproximadamente, o mesmo ângulo certeiro ao gol. 
\end{enumerate}
}{1}
\end{answer}

\clearmargin
\mspace{.25em}
\begin{answer}{Desenvolvendo um jogo de futebol para videogame}
{
\begin{enumerate}
\item Visualmente, provavelmente, será difícil decidir qual ângulo é o maior, mas como no primeiro item desta atividade, devemos incentivar os estudantes a se expressarem e conjecturarem sobre a situação. 
    
    \item{}
    Vamos utilizar a \Fref{sec2_futebol_res3_fig} para resolver esse item. Na situação mostrada na figura, marcamos o ponto $O$ como sendo o pé da perpendicular baixada de $T_2$ até $T_1C$. Logo, $OT_2$ é a altura do triângulo $T_1CT_2$ em relação ao lado $T_1C$. 

    \notas
    {
    \adjustbox{valign=t}
    {
    \begin{minipage}{\linewidth}
        \begin{figure}[H]
            \centering
            \includegraphics[scale=0.4]{sec2_trig3_atfutebol.png}
            \caption{Triângulo $CT_1T_2$. (Atividade \hyperref[desenvolvendo-jogo]{Desenvolvendo um jogo de futebol para videogame})}
            \label{sec2_futebol_res3_fig}
        \end{figure}
    \end{minipage}
    }
    }
     
    Como $OCT_2$ é um triângulo retângulo, temos
    $$\sen(\hat{T_2CT_1})=\fra{OT_2}{CT_2}\;\; \text{ e } \;\; \cos(\hat{T_2CT_1})=\fra{CO}{CT_2}.$$
    
    Logo,
    \begin{eqnarray}
    OT_2=CT_2 \cdot \sen(\hat{T_2CT_1}),\label{sec2_futebol_res_eq1}\\
    CO=CT_2 \cdot \cos(\hat{T_2CT_1}).\label{sec2_futebol_res_eq2}
    \end{eqnarray}
    De \eqref{sec2_futebol_res_eq2}, temos que
    \begin{equation}
        OT_1=T_1C-CT_2\cdot\cos(\hat{T_2CT_1}).\label{sec2_futebol_res_eq3}
    \end{equation}
    Agora, observando que $OT_1T_2$ é um triângulo retângulo em $O$, podemos usar o teorema de Pitágoras:
    $$T_1T_2^2=OT_1^2+OT_2^2.$$
    
    Utilizando as equações \eqref{sec2_futebol_res_eq1} e \eqref{sec2_futebol_res_eq3} na equação acima, temos
    $$T_1T_2^2=(T_1C-CT_2\cdot\cos(\hat{T_2CT_1}))^2+(CT_2\cdot\sen(\hat{T_2CT_1}))^2$$
    que implica
    $$7,3^2=38,08^2-2\cdot 38,08\cdot 34,79\cdot \cos(\hat{T_2CT_1})+34,79^2.$$
    Resolvendo a equação acima encontramos $\cos(\hat{T_2CT_1})=0,983$, e usando a tabela trigonométrica, podemos concluir que o ângulo $\cos(\hat{T_2CT_1})$ mede aproximadamente $10^\circ$.
     
    Considerando novamente todas as aproximações feitas, podemos concluir então, que os jogadores $A,B$ e $C$ possuem, aproximadamente, o mesmo ângulo certeiro ao gol. 
      
    \item{}
    Sim, é possível calcular o ângulo em todos os casos. Para isso, basta seguir o procedimento do item anterior, sem nenhuma informação adicional.
\end{enumerate}
}{1}
\end{answer}
\clearmargin
\begin{answer}{Desenvolvendo um jogo de futebol para videogame}
{
\paragraph{Parte II}
\begin{enumerate}
  \item{}
     Preenchendo a tabela pedida, temos:
     \begin{table}[H]
\centering
\begin{tabular}{|c|c|c|c|}
\hline
\tcolor{Jogador} & $\tmat{\quad a \quad}$ & $\tmat{\quad b \quad}$ & $\tmat{\quad c \quad}$ \\ %
\hline                               
A & 7,3 & 23,33 & 28,96 \\
\hline
B & 7,3 & 40,82 & 40,82 \\
\hline
C & 7,3 & 38,08 & 34,79 \\
\hline
\end{tabular}
\caption{Dados dos jogadores $A, B$ e $C$.}
\label{sec2_tabfutebol_res1}
\end{table}

     \item{}
     Preenchendo a tabela pedida, temos os seguintes valores aproximados:
     \begin{table}[H]
\centering
\begin{tabular}{|c|c|c|c|e{3cm}|}
\hline
\tcolor{Jogador} & $\tmat{\quad a \quad}$ & $\tmat{\quad b \quad}$ & $\tmat{\quad c \quad}$ & \tcolor{$\quad  \dfrac{\bm{b^2+c^2-a^2}}{\bm{2bc}} \quad$} \tabularnewline %\hline                               
$A$ & $7{,}3$ & $23{,}33$ & $28{,}96$ & $0{,}984$ \tabularnewline
\hline
$B$ & $7{,}3$ & $40{,}82$ & $40{,}82$ & $0{,}984$ \tabularnewline
\hline
$C$ & 7{,}3 & $38{,}08$ & $34{,}79$ & $0{,}984$ \tabularnewline
\hline
\end{tabular}
\caption{Dados dos jogadores $A, B$ e $C$.}
\label{sec2_tabfutebol_res2}
\end{table}

\item{}
Por esta tabela, percebemos que a expressão $\frac{b^2+c^2-a^2}{2bc}$ gera valores aproximadamente iguais para os triângulos $AT_1T_2, BT_1T_2$ e $CT_1T_2$. Isto nos permite conjecturar que há alguma relação entre os lados dos três triângulos que envolve algo que não se altera nesses três triângulos, mesmo sendo estes diferentes. Lembrando o que encontramos na PARTE I desta atividade, provavelmente esta terceira coluna expressa algo relacionado ao ângulo do chute certeiro ao gol, que é aproximadamente igual para os três jogadores. Isso será explicado com detalhes a seguir.
 \end{enumerate}
}{1}
\end{answer}

\explore{Indo além do teorema de Pitágoras}\label{exp_estendenoteopitagoras}


No Ensino Fundamental, estudamos algumas relações métricas do triângulo retângulo, como o teorema de Pitágoras. 
%
O amplo uso desses resultados no estudo dos mais diversos problemas geométricos mostra sua relevância e importância para a Geometria.

Como sabemos, o teorema de Pitágoras é um teorema válido apenas para triângulos retângulos. 
%
Mas, então, será que existe um resultado similar ao teorema de Pitágoras válido para um triângulo qualquer? 
%
Ou seja, existe um teorema que relacione a medida dos lados de um triângulo qualquer? 
%
Com o auxílio das relações trigonométricas estabelecidas neste capítulo, iremos juntos buscar uma resposta para esse questionamento.


\begin{task}{Explorando um triângulo não retângulo}
Nesta atividade, você precisará de uma régua para fazer as construções pedidas.

A partir do ângulo obtuso com vértice em $A$ da \Fref{sec2_explorando_trig_nao_ret_fig}, construa um triângulo $ABC$ de forma que o vértice $C$ esteja sobre o lado horizontal do ângulo obtuso dado e $B$ sobre o outro lado. 

Utilizando o triângulo $ABC$ que você construiu, faça o que se pede a seguir.
\begin{figure}[H]
    \centering
    \includegraphics[scale=0.35]{sec2_explorando_trig_nao_ret1.png}
    \caption{Ângulo obtuso com vértice em $A$.}
    \label{sec2_explorando_trig_nao_ret_fig}
\end{figure}

\begin{enumerate}
    \item{}
    O triângulo $ABC$ que você construiu é retângulo? Por que?
    
    \item{}
    Construa agora a altura $BD$ do triângulo $ABC$ partindo do vértice $B$. 
    
    \item{}
    Usando o teorema de Pitágoras nos triângulos retângulos $DBA$ e $DBC$, encontre uma relação métrica entre os comprimentos dos lados de cada um dos triângulos. Denote por $a, b, c, h$ e $u$ os comprimentos de $BC, AC, AB, BD$ e $AD$, respectivamente. 
    
    \item{}
    É possível encontrar uma relação métrica envolvendo apenas as medidas dos lados do triângulo $ABC$ (no item anterior, você fez isso para os triângulos retângulos $DBA$ e $DBC$)? Caso sua resposta seja sim, encontre-a. Caso sua resposta seja não, encontre uma relação envolvendo apenas $a, b, c$ e $u$ a partir das duas relações encontradas no item anterior para $DBA$ e $DBC$.
    
    \item{}
    Qual é a principal diferença entre as relações obtidas nos dois itens anteriores para os triângulos $DBA, DBC$ e $ABC$? Sobre essa principal diferença, é possível transformar as relações obtidas para que ela não exista?
    
    \end{enumerate}
\end{task}

\begin{reflection}
Você consegue imaginar o que aconteceria se o ângulo com vértice em $A$ dado no início da atividade anterior fosse agudo e não obtuso? A mesma relação encontrada no item \titem{d)} seria válida?
\end{reflection}

\begin{task}{Desenvolvendo um jogo de futebol para videogame}
\label{desenvolvendo-jogo}
Um desenvolvedor de jogos de videogame está criando um jogo de futebol e precisa de ajuda para realizar alguns cálculos matemáticos. Para auxiliá-lo, vamos responder as perguntas listadas na \textbf{Parte I} desta atividade. 

\medskip

Atenção para dois fatos a serem considerados na realização da atividade: 
\begin{itemize}
    \item {}
     Após um chute ao gol, a bola fará seu deslocamento sempre em contato com o gramado do campo de futebol e em linha reta. Neste caso, vamos considerar que o ângulo para um chute certeiro ao gol (ou seja, que passe entre as traves, aqui representadas por um único ponto cada) deve ser calculado como o ângulo formado pelos segmentos de reta que unem a posição dos pés do jogador (também representada por um único ponto) e os pés das traves. Na  \Fref{sec2_futebol1_fig}, o ângulo para um chute ao gol do jogador $P$ é o ângulo entre os segmentos $PT_1$ e $PT_2$.
\begin{figure}[H]
    \centering
    \includegraphics[scale=.4]{sec2_futebol1.png}
    \caption{Ângulo para um chute certeiro ao gol do jogador $P$.}
    \label{sec2_futebol1_fig}
\end{figure}



    \item{}
    Sempre que necessário, os valores numéricos sejam aproximados com 3 casas decimais.
\end{itemize}
\newpage

\paragraph{Parte I}

\begin{enumerate}
    \item{}
    Considere dois jogadores $A$ e $B$ posicionados dentro do campo de futebol, como na \Fref{sec2_futebol2_fig}. O jogador $A$ está sobre a quina da grande área e o jogador $B$ fora da grande área e próximo ao círculo central do campo. Analisando apenas visualmente a situação, na sua opinião, qual jogador possui o maior ângulo para um chute certeiro ao gol? Discuta com seu colega.
\begin{figure}[H]
    \centering
    \includegraphics[scale=0.35]{sec2_futebol2.png}
    \caption{Jogadores $A$ e $B$ se preparando para um chute ao gol.}
    \label{sec2_futebol2_fig}
\end{figure}



    \item{}
    Considere $AT_1=23,33, AT_2=28,96$ e $BT_1=BT_2=40,82$. Utilizando seus conhecimentos geométricos e as medidas do campo de futebol disponíveis na \Fref{sec2_futebol4_fig}, tente confirmar sua resposta ao item anterior. Se você não tiver chegado a uma conclusão do item anterior, não se preocupe. Com os valores dados neste item, você terá uma nova oportunidade de buscar a resposta. Se for preciso, use a tabela trigonométrica presente no final do capítulo com aproximações dos valores de seno e cosseno de diversos ângulos para resolver a atividade.
\begin{figure}[H]
    \centering
    \includegraphics[scale=0.3]{sec2_futebol4.png}
    \caption{Medidas do campo de futebol do jogo.}
    \label{sec2_futebol4_fig}
\end{figure}

    
    \item{}
    Agora, o desenvolvedor de jogos precisa adicionar um novo jogador ao cenário anterior, o jogador $C$, como na \Fref{sec2_futebol3_fig}. Assim como fizemos anteriormente, analise a situação visualmente e decida se é possível apontar, dentre os jogadores $A, B$ e $C$, qual deles possui o maior ângulo para um chute certeiro ao gol. Discuta com seu colega.
\begin{figure}[H]
    \centering
    \includegraphics[scale=0.4]{sec2_futebol3.png}
    \caption{Jogadores $A, B$ e $C$.}
    \label{sec2_futebol3_fig}
\end{figure}



    \item{}
    Considere $CT_1=38,08$ e $CT_2=34,79$. Após utilizar seus conhecimentos geométricos e as medidas do campo de futebol disponíveis na \Fref{sec2_futebol4_fig}, você manteria sua resposta do item anterior?

    \item{}
    Apenas com os dados das distâncias dos jogadores às traves que delimitam o gol, o desenvolvedor de jogos poderá calcular todos os ângulos de chutes certeiros ao gol? Você teria alguma sugestão para ajudar esse desenvolvedor? Existe alguma informação que poderia ser adicionada à situação para auxiliar na solução do problema do desenvolvedor?
    
    Caso tenha acesso a Internet (inclusive de um celular), você pode interagir com jogadores dentro de um campo de futebol por meio do aplicativo GeoGebra disponível em: <https://www.geogebra.org/m/aezzjs3c>
    
\end{enumerate}

\newpage

\paragraph{Parte II} 

Agora, vamos utilizar os dados que foram usados na PARTE I desta atividade para responder aos itens seguintes.

\begin{enumerate}

    \item{}
    Sejam $a$ a distância entre os pés das traves, e $b$ e $c$ as distâncias entre a posição de um jogador qualquer dentro de campo e os pés das traves. Usando os dados na PARTE I desta atividade para os jogadores A, B e C, preencha a \Tref{sec2_tabfutebol1}.
\begin{table}[H]
\centering
\begin{tabular}{|c|c|c|c|}
\hline
\tcolor{Jogador} & $\tmat{\quad a \quad}$ & $\tmat{\quad b \quad}$ & $\tmat{\quad c \quad}$ \\ %
\hline                               
A & & &  \\
\hline
B & & & \\
\hline
C & & & \\
\hline
\end{tabular}
\caption{Dados dos jogadores $A, B$ e $C$.}
\label{sec2_tabfutebol1}
\end{table}

\item{}
Vamos agora inserir uma coluna na \Tref{sec2_tabfutebol1} para calcular o valor de 
$$\fra{b^2+c^2-a^2}{2bc}$$
para cada uma das posições dos jogadores, como pode ser visto na \Tref{sec2_tabfutebol2}.
\begin{table}[H]
\centering
\begin{tabular}{|c|c|c|c|e{3.1cm}|}
\hline
\tcolor{Jogador} & $\tmat{\quad a \quad}$ & $\tmat{\quad b \quad}$ & $\tmat{\quad c \quad}$ & \tcolor{$\quad  \dfrac{\bm{b^2+c^2-a^2}}{\bm{2bc}} \quad$} \tabularnewline
\hline                               
$A$ & & & & \tabularnewline
\hline
$B$ & & & & \tabularnewline
\hline
$C$ & & & & \tabularnewline
\hline
\end{tabular}
\caption{Dados dos jogadores $A, B$ e $C$.}
\label{sec2_tabfutebol2}
\end{table}

    \item{}
    Analisando a \Tref{sec2_tabfutebol2} preenchida com os dados pedidos, o que é possível concluir? Tente analisar a tabela lembrando da sua resposta para o item (d) da PARTE I desta atividade.
\end{enumerate}
\end{task}

\arrange{Lei dos Cossenos}
\label{org_leidoscossenos}

Nas atividades anteriores, repare que o teorema de Pitágoras e as razões trigonométricas dos ângulos dos chutes ao gol não puderam ser aplicados diretamente para solucionar os problemas. Isso se deve ao fato de os triângulos que modelam as situações estudadas não serem triângulos retângulos. Por isso, a partir destes triângulos, tivemos que criar triângulos retângulos para, então, fazer uso das razões trigonométricas. A partir daí, conseguimos resolver o problema.

Nessa seção, nosso objetivo é generalizar o que foi feito no caso particular da atividade anterior obtendo um teorema que pode ser utilizado em qualquer triângulo e que é chamado lei dos cossenos. Esse teorema responde à pergunta que fizemos no início da seção sobre existir uma relação entre os lados de um triângulo qualquer. Depois de sua apresentação e demonstração, refletiremos sobre a pergunta feita no início da seção.

Primeiramente, enunciaremos e demonstraremos a lei dos cossenos para triângulos acutângulos. Na verdade, mesmo sem saber, você já fez isso no caso particular da atividade do jogo de videogame, mas agora refaremos todos os passos da solução de tal atividade de uma maneira geral que vale para  triângulos acutângulos. Mais adiante, estenderemos esse resultado para triângulos retângulos e obtusângulos.

\begin{observationtitle}{Lei dos Cossenos - parte 1}
\leavevmode\phantomsection\label{sec2_leidoscossenos_parte1}
Seja $ABC$ um triângulo acutângulo, onde $a, b$ e $c$ são as medidas dos lados $BC, AC$ e $AB$, respectivamente, e $\alpha=\angle(C\hat{A}B)$ como na \Fref{sec2_leidoscossenos_enunciado1}. Então,
\begin{equation}
    a^2=b^2+c^2-2bc\cos\alpha. \label{sec2_leidoscossenos_acut_eq}
\end{equation}
\begin{figure}[H]
    \centering
    \includegraphics[scale=0.4]{sec2_leidoscossenos_enunciado.png}
    \caption{O triângulo $ABC$ é acutângulo com ângulo interno $\alpha$.}
    \label{sec2_leidoscossenos_enunciado1}
\end{figure}
\end{observationtitle}


\paragraph{Demonstração da Lei dos Cossenos - parte 1}
\phantomsection\label{sec2_leidoscossenos_demo}
Vamos, então, considerar o triângulo acutângulo $ABC$, como na \Fref{sec2_leidoscossenos_enunciado1}.

Seja $BD$ a altura do triângulo $ABC$ traçada a partir do vértice $B$ e $h$ a medida de sua altura. Para facilitar os cálculos, vamos chamar de $u$ e $v$ as medidas dos segmentos $AD$ e $DC$, respectivamente. Neste caso, $b=u+v$. Veja a \Fref{sec2_leidoscossenos_acut_demo1_fig}.

\begin{figure}[H]
    \centering
    \includegraphics[scale=0.7]{sec2_leisdoscossenos_acut_demo1.png}
    \caption{Triângulo $ABC$ com altura $BD$.}
    \label{sec2_leidoscossenos_acut_demo1_fig}
\end{figure}

Repare que $ADB$ é um triângulo retângulo em $D$. Usando as razões trigonométricas estudadas na seção anterior neste triângulo, temos:
\begin{eqnarray}{}
 \sen\alpha = \fra{h}{c} & \iff h=c\sen\alpha, \label{sec2_leidoscossenos_demo_eq3}\\ 
 \cos\alpha = \fra{u}{c} & \iff u=c\cos\alpha. \label{sec2_leidoscossenos_demo_eq4}
\end{eqnarray}

Além disso, $BDC$ é um triângulo retângulo em $D$, então, o teorema de Pitágoras nos garante que:
\begin{equation}
    a^2=h^2+v^2. \label{sec2_leidoscossenos_demo_eq5}
\end{equation}

Como $v=b-u$, a equação \eqref{sec2_leidoscossenos_demo_eq5} pode ser reescrita da seguinte forma:
\begin{equation}
    a^2=h^2+(b-u)^2. \label{sec2_leidoscossenos_demo_eq6}
\end{equation}

Substituindo as equações \eqref{sec2_leidoscossenos_demo_eq3} e \eqref{sec2_leidoscossenos_demo_eq4} na equação \eqref{sec2_leidoscossenos_demo_eq6}, obtemos:
$$a^2=(c\sen\alpha)^2+(b-c\cos\alpha)^2,$$
ou seja,
\begin{equation}\label{sec2_leidoscossenos_demo_eq7}
    a^2=c^2\sen^2\alpha+b^2-2bc\cos\alpha+c^2\cos^2\alpha = c^2(\sen^2\alpha+\cos^2\alpha)+b^2-2bc\cos\alpha.
\end{equation}
Pela relação fundamental, a equação \eqref{sec2_leidoscossenos_demo_eq7} pode ser reescrita:
$$a^2=b^2+c^2-2bc\cos\alpha,$$
\noindent mostrando que \eqref{sec2_leidoscossenos_acut_eq} vale nesse caso.


\begin{reflection}
Compare a equação fornecida pela lei dos cossenos e a quarta coluna da \Tref{sec2_tabfutebol2}. Faz sentido os valores aproximados da quarta coluna serem tão próximos?
\end{reflection}

Este resultado é o primeiro passo na direção de respondermos à pergunta do início da seção sobre a existência de um resultado que relacione os lados de um triângulo qualquer.
%
No caso do triângulo acutângulo, agora, já sabemos que essa relação existe, mas ela também utiliza o cosseno de um ângulo interno do triângulo, que neste caso é sempre agudo.
%
Será que é possível encontrar um resultado semelhante a esse para triângulos retângulos e obtusângulos? 

Caso trabalhemos com triângulos retângulos e obtusângulos, seus ângulos internos podem ser retos ou obtusos, além de agudos. 
%
Porém, caso o ângulo seja reto ou obtuso, sabemos calcular seu cosseno? 
%
A resposta é: com o que aprendemos até aqui não. 
%
Sendo assim, vamos ampliar a definição de cosseno de um ângulo agudo para um ângulo qualquer entre $0^\circ$ e $180^\circ$, antes de procedermos com o estudo da lei dos cossenos. 

Já sabemos calcular o cosseno de um ângulo agudo com o que aprendemos na seção anterior, a definição a seguir contemplará agora todos os ângulos entre $0^\circ$ e $180^\circ$. Vejamos.

Definimos:
\begin{itemize}[topsep=0pt, itemsep=0pt]
\item $\cos(0^\circ)=1$;
\item $\cos(90^\circ)=0$;
\item $\cos(180^\circ)=-1$;
\item para um ângulo obtuso $\alpha$, $\cos\alpha=-\cos(180^\circ-\alpha).$
\end{itemize}

Nesta definição, utilizamos o cosseno de $180^\circ-\alpha$ para definir o cosseno de $\alpha$. Note que se $\alpha$ é um ângulo obtuso, então $180^\circ-\alpha$ é um ângulo agudo e portanto, sabemos calcular seu cosseno com o conteúdo que aprendemos até aqui.

Por agora, basta que tenhamos a definição anterior em mente para continuarmos nosso estudo de Trigonometria. Se você quiser saber um pouco mais sobre essa definição, veja o \textit{Para saber+ Explorando a definição de senos e cossenos}. E ainda, no capítulo que trata as funções trigonométricas, você terá a oportunidade de trabalhar mais profundamente com cossenos (e senos também) de um ângulo qualquer, indo além do que estamos vendo aqui.

\begin{observation}{}
Já sabemos que o cosseno de um ângulo agudo $\alpha$ assume apenas valores positivos e menores que $1$, então, de acordo com a definição acima,  
$$\text{se }0^\circ \leq\alpha \leq180^\circ, \text{então} -1\leq\cos\alpha\leq 1.$$
\end{observation}

Agora, de posse dessas informações, vamos enunciar e demonstrar a lei dos cossenos para qualquer triângulo retângulo e obtusângulo. 

\begin{observationtitle}{Lei dos Cossenos - Parte 2}
\leavevmode\phantomsection\label{sec2_leidoscossenos_parte2}
Seja $ABC$ um triângulo retângulo em $A$ ou obtusângulo, onde $a, b$ e $c$ são as medidas dos lados $BC, AC$ e $AB$, respectivamente, e $\alpha=\angle(C\hat{A}B)$ como na \Fref{sec2_leidoscossenos_enunciado2}. Então,
\begin{equation}\label{sec2_leidoscossenos_eq2}
    a^2=b^2+c^2-2bc\cos\alpha.
\end{equation}
\begin{figure}[H]
    \centering
    \includegraphics[height=.115\textheight]{sec2_leidoscossenos_enunciado_trigret.png}
    %\hspace{1.5cm}
    \includegraphics[height=.115\textheight]{sec2_leidoscossenos_enunciado_trigobt1.png}
    \includegraphics[height=.115\textheight]{sec2_leidoscossenos_enunciado_trigobt2.png}
    \caption{O triângulo $ABC$ da esquerda é retângulo em $A$, enquanto o central é obtusângulo com ângulo $\alpha$ obtuso e o da direita é obtusângulo com ângulo $\alpha$ agudo.}
    \label{sec2_leidoscossenos_enunciado2}
\end{figure}
\end{observationtitle}

\paragraph{Demonstração da Lei dos Cossenos}
\phantomsection\label{sec2_leidoscossenos_demo2}

Nesta demonstração, precisamos considerar que $ABC$ pode ser um triângulo retângulo em $A$ ou obtusângulo, como mostra a \Fref{sec2_leidoscossenos_enunciado2}. Dividiremos, então, essa demonstração em duas partes. Na primeira parte, trabalharemos apenas com um triângulo retângulo e na segunda, com um triângulo obtusângulo que poderá ter $\alpha$ agudo ou obtuso.

Primeira parte: $ABC$ é retângulo em $A$.

Estamos considerando, nesta parte, que $\alpha=90^\circ$. Veja a \Fref{sec2_leidoscossenos_fig_demoret1}.

\begin{figure}[H]
    \centering
    \includegraphics[scale=0.3]{sec2_leidoscossenos_enunciado_trigret.png}
     \caption{O triângulo $ABC$ é retângulo em $A$.}
    \label{sec2_leidoscossenos_fig_demoret1}
\end{figure}

Pela definição anterior, $\cos \alpha = \cos 90^\circ=0$, então 
\begin{equation}\label{sec2_leidoscossenos_demoret1}
b^2+c^2-2bc\cos\alpha= b^2+c^2-2bc \cdot 0 = b^2+c^2.
\end{equation}

Por outro lado, o triângulo $ABC$ é retângulo em $A$ e portanto, pelo teorema de Pitágoras sabemos que
\begin{equation}\label{sec2_leidoscossenos_demoret2}
a^2=b^2+c^2.
\end{equation}

\no Unindo as equações \eqref{sec2_leidoscossenos_demoret1} e \eqref{sec2_leidoscossenos_demoret2}, concluímos que
$$a^2=b^2+c^2-2bc\cos\alpha,$$
mostrando assim que \eqref{sec2_leidoscossenos_eq2} vale nesse caso.

\vspace{0.5cm}

Segunda parte: $ABC$ é obtusângulo.

Precisamos, nesta parte, considerar que o ângulo $\alpha$ pode ser obtuso ou agudo. A seguir, trabalharemos com cada uma das possibilidades.

Consideremos primeiramente que $\alpha$ é obtuso. Veja a \Fref{sec2_leidoscossenos_fig_demo_obtusangulo_obtuso1}.

\begin{figure}[H]
    \centering
    \includegraphics[scale=0.7]{sec2_leidoscossenos_enunciado_trigobt1.png}
    \caption{Triângulo $ABC$ onde $\alpha$ é um ângulo obtuso.}
    \label{sec2_leidoscossenos_fig_demo_obtusangulo_obtuso1}
\end{figure}

Seja, então, $BD$ a altura do triângulo $ABC$ traçada a partir do vértice $B$, como podemos ver na \Fref{sec2_leidoscossenos_fig_demo_obtusangulo_obtuso2}. Denotemos por $h$ e $u$ as medidas da altura $BD$ e do segmento $AD$, respectivamente.

\begin{figure}[H]
    \centering
    \includegraphics[scale=0.3]{sec2_leidoscossenos_demoobtuso.png}
    \caption{Triângulo $ABC$ onde $\alpha$ é um ângulo obtuso.}
    \label{sec2_leidoscossenos_fig_demo_obtusangulo_obtuso2}
\end{figure}

Repare que $ADB$ é um triângulo retângulo em $D$. Usando as razões trigonométricas estudadas na seção anterior, temos:
\begin{eqnarray}{}
 \sen (180^\circ-\alpha) = \fra{h}{c} & \iff h=c\sen (180^\circ-\alpha), \label{sec2_leidoscossenos_demo8}\\ 
 \cos (180^\circ-\alpha) = \fra{u}{c} & \iff u=c\cos (180^\circ-\alpha). \label{sec2_leidoscossenos_demo9}
\end{eqnarray}

E ainda, podemos notar que $BDC$ é também um triângulo retângulo em $D$. Assim, aplicando o teorema de Pitágoras neste triângulo obtemos:
\begin{equation}
    a^2=h^2+(b+u)^2. \label{sec2_leidoscossenos_demo10}
\end{equation}

Substituindo as equações \eqref{sec2_leidoscossenos_demo8} e \eqref{sec2_leidoscossenos_demo9} em \eqref{sec2_leidoscossenos_demo10}, obtemos:
$$a^2=(c\sen (180^\circ-\alpha))^2+(b+c\cos (180^\circ-\alpha))^2,$$
ou seja,
$$\begin{array}{ccc}
    a^2 & = &  c^2\sen^2 (180^\circ-\alpha)+b^2+2bc\cos (180^\circ-\alpha)+c^2\cos^2 (180^\circ-\alpha)\\
    & = & c^2(\sen^2 (180^\circ-\alpha)+\cos^2 (180^\circ-\alpha))+b^2+2bc\cos (180^\circ-\alpha).
\end{array}$$
Usando a relação fundamental, concluímos que:
\begin{equation}
    a^2=b^2+c^2+2bc\cos (180^\circ-\alpha).\label{sec2_leidoscossenos_demo11}
\end{equation}

Pela definição anterior, temos que $\cos(180^\circ-\alpha)=-\cos\alpha$, portanto, 
$$a^2=b^2+c^2-2bc\cos\alpha,$$
mostrando assim que \eqref{sec2_leidoscossenos_eq2} vale neste caso.

Consideremos agora que $\alpha$ é agudo. Veja a \Fref{sec2_leidoscossenos_fig_demo_obtusangulo_agudo1}.

\begin{figure}[H]
    \centering
    \includegraphics[scale=0.4]{sec2_leidoscossenos_enunciado_trigobt2.png}
    \caption{Triângulo $ABC$ onde $\alpha$ é um ângulo agudo.}
    \label{sec2_leidoscossenos_fig_demo_obtusangulo_agudo1}
\end{figure}

Considere $CD$ a altura do triângulo $ABC$ traçada a partir de $C$, como podemos ver na \Fref{sec2_leidoscossenos_fig_demo_obtusangulo_agudo2}. Chamaremos de $u, v$ e $h$ as medidas dos segmentos $AD, BD$ e $CD$, respectivamente.

\begin{figure}[H]
    \centering
    \includegraphics[scale=0.4]{sec2_leidoscossenos_demo_obtusangulo_agudo.png}
    \caption{Triângulo $ABC$ onde $\alpha$ é um ângulo agudo.}
    \label{sec2_leidoscossenos_fig_demo_obtusangulo_agudo2}
\end{figure}

Como $ACD$ é um triângulo retângulo em $D$, então 
\begin{equation}
    \cos\alpha=\frac{u}{b}.\label{sec2_leidoscossenos_demo12}
\end{equation}

Além disso, pelo teorema de Pitágoras, temos
 \begin{equation} 
    b^2=u^2+h^2. \label{sec2_leidoscossenos_demo13}
\end{equation}

Como $CDB$ também é um triângulo retângulo, então novamente pelo teorema de Pitágoras, temos
\begin{equation}
    a^2=v^2+h^2. \label{sec2_leidoscossenos_demo14}
\end{equation}

Da equação \eqref{sec2_leidoscossenos_demo13} temos que $h^2=b^2-u^2$. Substituindo em \eqref{sec2_leidoscossenos_demo14}, encontramos

\begin{eqnarray}{}
   a^2 & = & v^2+b^2-u^2 \nonumber\\
       & = & b^2+(v^2-u^2) \nonumber \\
       & = & b^2+(v+u)(v-u) \nonumber\\
       & = & b^2+c(v-u).\label{sec2_leidoscossenos_demo15}
\end{eqnarray}

Como $u+v=c \iff v=c-u$, por \eqref{sec2_leidoscossenos_demo12} temos que
\begin{equation}
    v-u=(c-u)-u=c-2u= c-2b\cos\alpha. \label{sec2_leidoscossenos_demo16}
\end{equation}

Substituindo \eqref{sec2_leidoscossenos_demo16} em \eqref{sec2_leidoscossenos_demo15} encontramos
$$a^2=b^2+c^2-2bc\cos\alpha,$$
mostrando que a equação \eqref{sec2_leidoscossenos_eq2} vale também para este caso.\hspace{4.5cm}$\square$


Após estudar a parte 1 e a parte 2 da lei dos cossenos, concluímos que, dado um triângulo qualquer $ABC$, onde $a, b$ e $c$ são as medidas dos lados $BC, AC$ e $AB$, respectivamente, e $\alpha=\angle(C\hat{A}B)$, vale a seguinte relação:
$$a^2=b^2+c^2-2bc\cos\alpha.$$

Você se lembra dos questionamentos que fizemos anteriormente sobre a existência de uma relação entre os lados em um triângulo qualquer? Somente agora, com os resultados da parte 1 e da parte 2 do teorema demonstrados, temos a possibilidade de responder ao que foi perguntado com convicção: sim, existe uma relação entre lados de um triângulo qualquer (chamada lei dos cossenos), mas diferentemente do teorema de Pitágoras, ela também envolve algum dos ângulos do triângulo. 

A lei dos cossenos recai sobre o teorema de Pitágoras caso $\alpha=90^\circ$, ou seja, se $ABC$ é um triângulo retângulo. Sendo assim, o teorema de Pitágoras pode ser visto como um caso particular da lei dos cossenos.

Vejamos a seguir um exemplo que busca encontrar a medida de um lado de um triângulo, dados um de seus ângulos e a medida dos outros dois lados.
\clearmargin
\def\currentcolor{session2}
\begin{objectives}{Estudando as câmeras de segurança de um estacionamento}
{
Aplicar a lei dos cossenos em um problema contextualizado. 
}{1}{1}
\end{objectives}
\begin{sugestions}{Estudando as câmeras de segurança de um estacionamento}
{
Nesta atividade serão utilizados os seguintes dados, que poderão ser acessados pelos estudantes na tabela trigonométrica do fim do capítulo: $\cos (55^\circ)=0,57$ e $\cos(125^\circ)=-0,57$. 

\textbf{Organização da turma}: em duplas.

\textbf{Conceitos abordados}: lei dos cossenos.

Dificuldades previstas: esta será a primeira vez, após a apresentação da lei dos cossenos, que os estudantes precisarão modelar um problema por um triângulo não retângulo e usar a lei para resolvê-lo. Como eles estarão diante de algo novo, é possível que ainda se sintam despreparados para resolver a questão e tenham dificuldade até mesmo para começar a solução. Caso essa seja a realidade dos estudantes, é importante que o professor auxilie os estudantes com uma interpretação correta da situação e também o uso adequado da lei dos cossenos. É muito comum o estudante não perceber que do lado esquerdo da equação dada pela lei dos cossenos está a medida ao quadrado do lado oposto ao ângulo considerado do lado direita da equação. Sendo assim, é preciso que o professor enfatize a correta interpretação da fórmula, de acordo com os dados do problema dado.
}{1}{1}1
\end{sugestions}
\begin{answer}{Estudando as câmeras de segurança de um estacionamento}
{
\begin{enumerate}
\item{} 
    A \Fref{sec2_resatestacionamento1} mostra o esboço da situação.
    \begin{figure}[H]
        \centering
        \includegraphics[scale=0.25]{sec2_atestacionamento1.png}
        \caption{Estacionamento em forma de retângulo $ABCD$.}
        \label{sec2_resatestacionamento1}
    \end{figure}
\end{enumerate}
}{1}
\end{answer}

\begin{answer}{Estudando as câmeras de segurança de um estacionamento}
{
\setcounter{enumi}{1}
\begin{enumerate}
\item Como o estacionamento é modelado por um retângulo onde conhecemos seus lados e o fio que será medido é a diagonal desse retângulo, é possível calcular o que é pedido utilizando o teorema de Pitágoras. 
    
    Nesse caso, $ABC$ é um triângulo retângulo em $B$. Assim, pelo teorema de Pitágoras obtemos 
    $$AC^2=100^2+70^2 \iff AC=10\sqrt{149}\text{m}$$
    que é o comprimento procurado do fio.

    \item{}
    A \Fref{sec2_resatestacionamento2} mostra o esboço da nova situação.
    \begin{figure}[H]
        \centering
        \includegraphics[scale=0.15]{sec2_atestacionamento2.png}
        \caption{Estacionamento em forma de paralelogramo $ABCD$.}
        \label{sec2_resatestacionamento2}
    \end{figure}

    \item{}
    Não podemos, pois o paralelogramo que modela a situação pode ter diversos ângulos internos. 

    \item{}
    A seguir faremos uma construção que mostra que neste caso é possível calcular o tamanho do fio.
    
    Vamos chamar de $B$ o vértice do paralelogramo que modela o estacionamento de forma que $A\hat{B}C=55^\circ$. Sendo assim, em relação ao triângulo $ABC$ são conhecidos dois de seus lados e o ângulo compartilhado por eles. Neste caso, usando a lei do cossenos, podemos calcular o lado oposto ao ângulo $55^\circ$, isto é, o comprimento do fio.
    Pela lei dos cossenos,
    $$AC^2=BA^2+BC^2-2\cdot BA\cdot BC\cdot \cos(55^\circ)=100^2+70^2-2\cdot100\cdot70\cdot \cos(55^\circ).$$
    Com o auxílio da tabela trigonométrica sabemos que $\cos(55^\circ)=0,57$. Logo,
    $$AC^2=100^2+70^2-2\cdot100\cdot70\cdot 0,57 \iff AC=2\sqrt{1730}\text{m}.$$
    \begin{figure}[H]
        \centering
        \includegraphics[scale=0.15]{sec2_atestacionamento3.png}
        \caption{Estacionamento em forma de um paralelogramo com um ângulo medindo $55^\circ$.}
        \label{sec2_resatestacionamento3}
    \end{figure}
    
    \item{}
    Sim, porque é possível determinar a medida do outro ângulo, já que em paralelogramos, ângulos adjacentes são complementares. Sendo assim, o ângulo $A\hat{B}C$ mede $125^\circ$. Neste caso, a lei dos cossenos nos daria o seguinte:
    $$AC^2=BA^2+BC^2-2\cdot BA\cdot BC\cdot \cos(125^\circ)=100^2+70^2-2\cdot100\cdot70\cdot \cos(125^\circ).$$
    Novamente utilizando a tabela trigonométrica, encontramos que $\cos(125^\circ)=-0,57$. Assim,
    $$AC^2=100^2+70^2-2\cdot100\cdot70\cdot (-0,57) \iff AC=4\sqrt{1430}\text{m}.$$
    \begin{figure}[H]
        \centering
        \includegraphics[scale=0.15]{sec2_atestacionamento4.png}
        \caption{Estacionamento em forma de um paralelogramo com um ângulo medindo $125^\circ$.}
        \label{sec2_resatestacionamento4}
    \end{figure}
\end{enumerate}
}{9}
\end{answer}

\clearmargin
\begin{objectives}{Molécula de água líquida e congelada}
{
\begin{itemize}
\item Utilizar a lei dos cossenos para resolver um problema contextualizado na área de Química e interpretar os resultados matemáticos obtidos diante da situação real estudada.

\item \textbf{Conceitos abordados}: lei dos cossenos, além de conceitos da Química, como átomos, moléculas e ângulo de ligação entre os átomos.
\end{itemize}
}{1}{2}
\end{objectives}
\begin{sugestions}{Molécula de água líquida e congelada}
{
Nesta atividade serão utilizados os seguintes dados, que poderão ser acessados pelos estudantes na tabela trigonométrica do fim do capítulo: $\arccos({0,25})=75^\circ$ e $\arccos({0,33})=71^\circ$. 

\textbf{Organização da turma}: em duplas.

\textbf{Dificuldades previstas}: destacamos mais uma vez ser necessário ter atenção ao uso adequado da lei dos cossenos em relação ao ângulo considerado. Além disso, nesta atividade, os estudantes podem ter dificuldades com os termos utilizados em Química, como átomos, moléculas, ligação entre átomos e etc. Sugerimos ao professor que esteja preparado para intervir caso seja constatada tal dificuldade e, sendo necessário, o professor de Química dos estudantes pode ser chamado para uma conversa conjunta. Neste caso, pode-se solicitar que o vocabulário usado na questão seja trabalhado, destacando o quanto as áreas de Matemática e Química estão relacionadas.
}{1}{2}
\end{sugestions}
\begin{answer}{Molécula de água líquida e congelada}
{
\begin{enumerate}
    \item{} A \Fref{sec2_resatquimica2} mostra um esquema  para a representação da molécula de água. 
    \begin{figure}[H]
        \centering
        \includegraphics[scale=0.3]{sec2_atquimica2.png}
        \caption{Esquema para representação de uma molécula de água.}
        \label{sec2_resatquimica2}
    \end{figure}
     
    \item{}
    Queremos encontrar o ângulo $H_1\hat{O}H_2$. Como conhecemos os três lados desse triângulo do $OH_1H_2$, usando a lei dos cossenos, podemos obter informações sobre um ângulo do triângulo. Usando, então, a lei dos cossenos de modo a obter o cosseno do ângulo $H_1\hat{O}H_2$, temos
    $$(H_1H_2)^2=(OH_1)^2+(OH_2)^2-2\cdot OH_1\cdot OH_2\cdot\cos({H_1\hat{O}H_2}).$$
    Substituindo os valores dos lados do triângulo $OH_1H_2$ na equação acima:
    $$151,8^2=96^2+96^2-2\cdot 96\cdot 96\cdot\cos({H_1\hat{O}H_2}) \iff \cos({H_1\hat{O}H_2})=-0,25.$$
    
    Como o ângulo ${H_1\hat{O}H_2}$ possui cosseno igual a $-0,25$, precisamos lembrar que se $\cos(H_1\hat{O}H_2)=\cos(180^\circ-\alpha)=-0,2501$, então 
    $$-\cos\alpha=\cos(180^\circ-\alpha)=-0,25 \iff \cos\alpha=0,25.$$
    Logo, $\alpha$ mede aproximadamente $75^\circ$, de acordo com a tabela trigonométrica do final do capítulo. Sendo assim, $H_1\hat{O}H_2$ mede aproximadamente $180^\circ-\alpha=180^\circ-75^\circ=105^\circ.$
    
    \item{}
    A \Fref{sec2_resatquimica2} mostra um esquema  para a representação da molécula de água usando os novos valores das distâncias entre os átomos.
    \begin{figure}[H]
        \centering
        \includegraphics[scale=0.3]{sec2_atquimica3.png}
        \caption{Esquema para representação de uma molécula de água.}
        \label{sec2_resatquimica3}
    \end{figure}
    Seguindo o mesmo caminho da questão anterior, temos que
    $$165^2=101^2+101^2-2\cdot 101\cdot 101\cdot\cos({H_1\hat{O}H_2}) \iff \cos({H_1\hat{O}H_2})=-0,33.$$
    
    Como o ângulo ${H_1\hat{O}H_2}$ possui cosseno igual a $-0,33$, análogo ao que foi feito no item anterior, precisamos lembrar que se $\cos(H_1\hat{O}H_2)=\cos(180^\circ-\alpha)=-0,33$, então 
    $$-\cos\alpha=\cos(180^\circ-\alpha)=-0,3344 \iff \cos\alpha=0,33.$$
    Logo, $\alpha$ mede aproximadamente $71^\circ$, de acordo com a tabela trigonométrica do final do capítulo. Sendo assim, $H_1\hat{O}H_2$ mede aproximadamente $180^\circ-\alpha=180^\circ-71^\circ=109^\circ.$
    
    \item{}
    Como visto nos dois itens anteriores, o ângulo da molécula de água aumenta quando ela é congelada. Mesmo sendo o triângulo um elemento geométrico plano, podemos conjecturar que esse aumento do ângulo se reflete no aumento do volume da água quando congelada. 
\end{enumerate}
}{9}
\end{answer}
\def\currentcolor{session4}

\begin{example}{Distância entre as margens de um lago} \label{sec2_leidoscossenos_ex}

(UERJ-2017) Ao coletar os dados para um estudo topográfico da margem de um lago a partir dos pontos $A, B$ e $T$, um técnico determinou as medidas $AT = 32$m, $BT = 13$m e $A\hat{T}B = 120^\circ$, representadas na \Fref{Lago}.
\begin{figure}[H]
    \centering
    \includegraphics[scale=0.65]{Lago.JPG}
    \caption{Lago.}
    \label{Lago}
\end{figure}
Calcule a distância, em metros, entre os pontos $A$ e $B$, definidos pelo técnico nas margens desse lago.

\paragraph{Solução}
\phantomsection\label{LA_sec2_leidoscossenos_exres}
Para encontrar a distância entre os pontos $A$ e $B$, podemos utilizar diretamente a lei dos cossenos no triângulo $ATB$, sendo $A\hat{T}B$ o ângulo a ser utilizado já que é o único conhecido no triângulo.

Para utilizar o ângulo $A\hat{T}B$, a lei dos cossenos se aplica ao triângulo $ATB$ da seguinte forma:
$$AB^2=AT^2+BT^2-2\cdot AT\cdot BT\cdot\cos(A\hat{T}B),$$ 
ou seja,
\begin{equation}
    AB^2=32^2+13^2-2\cdot32\cdot13\cdot\cos120^\circ. \label{sec2_leidoscossenos_ex_eq1}
\end{equation}

Pela definição de cossenos de ângulo obtuso,
$$\cos120^\circ=-\cos60^\circ=-\fra12.$$
Logo, substituindo esse valor da equação \eqref{sec2_leidoscossenos_ex_eq1}, concluímos que
$$AB^2=1024+169-832\cdot\left(-\fra12\right)=1609.$$
Sendo assim, $AB=\sqrt{1609}$m, ou seja, a distância entre os pontos $A$ e $B$ é de aproximadamente $40,11$m.
\end{example}

\practice{Praticando a lei dos cossenos}\label{prat_leidoscossenos}

\begin{task}{Estudando as câmeras de segurança de um estacionamento}
Atualmente, é muito comum encontrarmos câmeras de segurança em diversos locais que frequentamos diariamente. 
%
Além de aumentarem a segurança dos locais, as câmeras têm por finalidade auxiliar na identificação de objetos esquecidos, controle de grandes aglomerações de pessoas, etc. Em muitos desse locais, costuma-se utilizar as câmeras aos pares,  em posições opostas, garantindo, assim, o alcance de todo o espaço do estacionamento (\Fref{sec2_at_fig_camseguranca1}). 
 
\begin{figure}[H]
    \centering
    \includegraphics[scale=0.35]{sec2_at_enunciado_camseguranca1.png}
    \caption{Posicionamento de câmeras de segurança. Fonte: https://bit.ly/310nJFS.}
    \label{sec2_at_fig_camseguranca1}
\end{figure}

\begin{enumerate}
    \item{}
    Em um estacionamento com formato retangular de comprimento $100$m e largura $70$m serão instaladas duas câmeras de segurança em vértices opostos do estacionamento. Essas duas câmeras serão interligadas subterraneamente por um fio posicionado totalmente esticado, ou seja, em linha reta. Faça um esboço da situação, nomeando de $A$ e $C$ os vértices por onde passarão os fios que ligam essas câmeras.

    \item{}
    Apenas com os dados fornecidos no item anterior, é possível calcular o comprimento da parte subterrânea do fio que fará a ligação entre as câmeras? Se sim, encontre o seu comprimento.

    \item{}
    Imagine agora que, por razões técnicas, o estacionamento de um determinado shopping tenha o formato de um paralelogramo cujos lados medem $100$m e $70$m. Faça um esboço da situação, nomeando de $A$ e $C$ os vértices por onde passarão os fios que ligam essas câmeras.

    \item{}
    Apenas com os dados fornecidos no item anterior, é possível calcular o comprimento da parte subterrânea do fio que fará a ligação entre as câmeras? Se sim, encontre o comprimento.

    \item{}
    Considere agora que um dos ângulos do paralelogramo que determina o formato do estacionamento é $55^\circ$ e que o fio ficará oposto a esse ângulo. Nesse caso, é possível calcular o comprimento da parte subterrânea do fio que fará a ligação entre as câmeras? Se sim, encontre o comprimento.
    
    \item{}
    E se, ao contrário do considerado anteriormente, o fio cortasse o ângulo de $55^\circ$ que mencionamos no item anterior? Nesse caso, é possível calcular o comprimento da parte subterrânea do fio que fará a ligação entre as câmeras? Se sim, encontre o comprimento.
\end{enumerate}
\end{task}

\begin{task}{Molécula de água líquida e congelada}\label{sec2_leidoscossenos_atquimica}
Uma molécula de água é formada por dois átomos de hidrogênio e um átomo de oxigênio. A ligação entre os átomos de hidrogênio e de oxigênio faz com que a molécula da água tenha sempre formato triangular, independente de seu estado. Quando em estado líquido, a distância entre o átomo de oxigênio e o átomo de hidrogênio é de 96 picômetros e entre os dois átomos de hidrogênio é 151,8 picômetros ($1$ picômetro corresponde à $10^{-12}$ metros). Utilizando esses dados, faça o que se pede.

Caso seja necessário, utilize 2 casas decimais para calcular as aproximações pedidas.

\begin{enumerate}
    \item{}
    Faça um esquema que represente uma molécula de água, considerando que não há ligação entre os átomos de hidrogênio. O átomo de oxigênio deve ser denotado por $O$, enquanto os dois átomos de hidrogênio devem ser denotados por $H_1$ e $H_2$.
    
    \item{}
    Encontre o ângulo de ligação entre os átomos da molécula de água, ou seja, o ângulo $H_1\hat{O}H_2$.
    
    \item{}
    Quando a água congela, a distância entre o átomo de oxigênio e o átomo de hidrogênio é alterada para 101 picômetros e entre os dois átomos de hidrogênio para 165 picômetros. Encontre o ângulo de ligação entre os átomos da molécula de água nesta nova situação.

    \item{}
    Podemos perceber que, quando a água congela, ela tende a aumentar seu volume. Tente relacionar essa informação com o que foi calculado anteriormente.  
\end{enumerate}
\end{task}

\know{}
A lei dos cossenos nos trouxe uma equação que relaciona lados e o ângulos de um triângulo qualquer. A partir desse resultado, conseguimos relacionar também o tipo de triângulo com os tamanhos de seus lados da seguinte forma: 

Se $ABC$ um triângulo onde $AB=c, BC=a$ e $AC=b$. Se $a>b>c$, então
\begin{enumerate}
    \item{}
    $ABC$ é retângulo $\iff a^2=b^2+c^2$.
    
    \item{}
    $ABC$ é acutângulo $\iff a^2<b^2+c^2$.
    
    \item{}
    $ABC$ é obtusângulo $\iff a^2>b^2+c^2$.
\end{enumerate}

Por que isso é verdade?

\begin{enumerate}
    \item{}
    Utilizando a lei dos cossenos, temos que 
    $$\begin{array}{ccl}
        a^2 = b^2+c^2 & \iff & b^2+c^2-2bc\cos (C\hat{A}B) = b^2+c^2 \\
                      & \iff & -2bc\cos (C\hat{A}B)= 0\\
                      & \iff & \cos (C\hat{A}B)=0\\
                      & \iff & C\hat{A}B=90^\circ.
    \end{array}$$
    Este primeiro item, na verdade, é o teorema de Pitágoras e sua recíproca. 
    
    Vale a pena frisar que, a recíproca do teorema de Pitágoras diz que, se a igualdade $a^2=b^2+c^2$ é satisfeita para um triângulo $ABC$ onde $AB=c, BC=a$ e $AC=b$, então esse triângulo é retângulo em $A$.
    
    \item{}
    Utilizando a lei dos cossenos, temos que 
    $$\begin{array}{ccl}
        a^2 < b^2+c^2 & \iff & b^2+c^2-2bc\cos (C\hat{A}B)< b^2+c^2 \\
                      & \iff & -2bc\cos (C\hat{A}B)< 0\\
                      & \iff & \cos (C\hat{A}B)>0\\
                      & \iff & C\hat{A}B<90^\circ.
    \end{array}$$
    
    \item{}
    Analogamente ao que foi feito anteriormente, vamos utilizar a lei dos cossenos: 
    $$\begin{array}{ccl}
        a^2 > b^2+c^2 & \iff & b^2+c^2-2bc\cos (C\hat{A}B) > b^2+c^2 \\
                      & \iff & -2bc\cos (C\hat{A}B) > 0\\
                      & \iff & \cos (C\hat{A}B) < 0\\
                      & \iff & C\hat{A}B > 90^\circ.
    \end{array}$$
\end{enumerate}

\cleardoublepage
\def\currentcolor{session1}
\begin{objectives}{Medindo distâncias inacessíveis}
{
\begin{itemize}
\item Aplicar a argumentação utilizada na lei dos senos, que será apresentada adiante, na solução de um problema numérico e adquirir familiaridade com este tipo de argumentação antes de sua formalização.

\item \textbf{Conceitos abordados}: razões trigonométricas e lei dos cossenos.
\end{itemize}
}{1}{1}
\end{objectives}
\begin{sugestions}{Medindo distâncias inacessíveis}
{
Esta atividade tem por finalidade apresentar os passos da demonstração da lei dos senos em um contexto numérico. 
%
Com o desenvolvimento dessa atividade, acreditamos que os estudantes compreenderão de forma mais natural esta demonstração. 
%
Além disso, sugerimos ao professor que estimule o aluno a construir uma estratégia de resolução a partir de seus dados, antes de realizar os cálculos de uma questão. Por exemplo, o estudante deve perceber que não há possibilidade de utilizar a lei dos cossenos pelos dados fornecidos no enunciado.

Nesta atividade serão utilizados os seguintes dados, que poderão ser acessados pelos estudantes na tabela trigonométrica do fim do capítulo: $\sen(62^\circ)=0,88, \sen(48^\circ)=0,74$ e $\sen(70^\circ)=0,93$.

\textbf{Organização da turma}: em duplas.
}{1}{1}
\end{sugestions}
\begin{answer}{Medindo distâncias inacessíveis}
{
Notemos, primeiramente, que precisamos calcular a altura do drone em cada uma das posições apresentadas nos itens da atividade para determinar se ele produzirá imagens com boa qualidade onde está posicionado. Para isso, vamos trabalhar com um triângulo onde dois de seus vértices coincidem com os córneres diametralmente opostos do campo, digamos $A$ e $B$, e o terceiro vértice do triângulo com a posição do drone, digamos $C$. Sendo assim, queremos encontrar a altura do triângulo $ABC$ a partir do vértice $C$. 

Em cada uma das situações apresentadas,  conhecemos dois ângulos internos do triângulo $ABC$ e, consequentemente o terceiro também. Além disso, podemos calcular a medida do segmento $AB$, que coincide com a diagonal do campo em forma retangular, através do teorema de Pitágoras:
$$AB^2=90^2+120^2=22500 \iff AB=150\text{m}.$$ 
Neste caso, estamos trabalhando com um triângulo como o da \Fref{sec2_at_drone_resolucao1_fig}, onde os ângulos internos dependem de cada situação. 
\begin{figure}[H]
    \centering
    \includegraphics[scale=0.6]{sec2_at_drone_resolucao1.png}
    \caption{Triângulo $ABC$.}
    \label{sec2_at_drone_resolucao1_fig}
\end{figure}

\begin{enumerate}
\item{}
Neste primeiro item, o drone está posicionado acima do centro do campo, de modo que seja visto de córneres diametralmente opostos por um ângulo de $45^\circ$. Neste caso, modelando a situação pelo triângulo $ABC$, como já discutido anteriormente, temos que os ângulos $C\hat{A}B$ e $A\hat{B}C$ devem medir $45^\circ$.

Neste caso, como a soma dos ângulos internos de um triângulo qualquer é $180^\circ$, o ângulo $A\hat{C}B$ mede $90^\circ$. Assim, o triângulo $ABC$ é isósceles de base $AB$ e retângulo em $C$, como vemos na \Fref{sec2_at_drone_resolucao_itema_1_fig}.
\begin{figure}[H]
    \centering
    \includegraphics[scale=0.6]{sec2_at_drone_resolucao_itema_1.png}
    \caption{Triângulo $ABC$.}
    \label{sec2_at_drone_resolucao_itema_1_fig}
\end{figure}

Seja $CD$ a altura de $ABC$ a partir do vértice $C$ e $h$ sua medida. Como $ABC$ é isósceles, sabemos que sua altura divide a base em dois segmentos de mesmo comprimento. Logo, as medidas de $AD$ e $DB$ são iguais a $75$m. Veja a \Fref{sec2_at_drone_resolucao_itema_2_fig}.
\begin{figure}[H]
    \centering
    \includegraphics[scale=0.6]{sec2_at_drone_resolucao_itema_2.png}
    \caption{Triângulo $ABC$.}
    \label{sec2_at_drone_resolucao_itema_2_fig}
\end{figure}

Note que o triângulo $ADC$ é retângulo em $D$ e seu ângulo interno $C\hat{A}D$ mede $45^\circ$. Logo, como a soma dos ângulos internos de um triângulo qualquer é $180^\circ$, a medida de $A\hat{C}D$ é também $45^\circ$. Ou seja, $ADC$ é isósceles de base $AC$. Sendo assim, a medida dos lados $AD$ e $CD$ são iguais. Portanto, $h=75$m.

Assim, o drone está a uma altura de $75$m e abaixo do limite estipulado para gerar boas imagens.

\item{}
Nesta nova situação, análogo ao que tínhamos no item anterior, vemos que $ABC$ possui ângulos internos $C\hat{A}B, A\hat{B}C$ e $B\hat{C}A$ medindo $70^\circ, 62^\circ$ e $48^\circ$, respectivamente. Além disso, conhecemos a medida de $AB$, que é $150$m. Veja a \Fref{sec2_at_drone_resolucao_itemb_1_fig}.
\begin{figure}[H]
    \centering
    \includegraphics[scale=0.6]{sec2_at_drone_resolucao_itemb_1.png}
    \caption{Triângulo $ABC$.}
    \label{sec2_at_drone_resolucao_itemb_1_fig}
\end{figure}

Sejam $CD$ a altura do triângulo $ABC$ traçada a partir de $C$ e $h$ sua medida. Tracemos a altura $AE$ do triângulo $ABC$ a partir do vértice $A$ e denotemos sua medida por $h'$. A situação está esboçada na \Fref{sec2_at_drone_resolucao_itemb_2_fig}.
\begin{figure}[H]
    \centering
    \includegraphics[scale=0.6]{sec2_at_drone_resolucao_itemb_3.png}
    \caption{Triângulo $ABC$ com alturas $CD$ e $AE$.}
    \label{sec2_at_drone_resolucao_itemb_2_fig}
\end{figure}

No triângulo retângulo $ABE$, temos que:
\begin{equation}
    \sen(62^\circ)=\fra{AE}{AB}=\fra{h'}{150},\label{sec2_at_drone_itema_eq1}
\end{equation}
e no triângulo retângulo $CAE$, temos que:
\begin{equation}
    \sen(48^\circ)=\fra{AE}{AC}=\fra{h'}{AC}.\label{sec2_at_drone_itema_eq2}
\end{equation}

De \eqref{sec2_at_drone_itema_eq1} e \eqref{sec2_at_drone_itema_eq2}, temos que
$$150\cdot\sen(62^\circ)=AC\cdot\sen(48^\circ).$$
Logo, 
\begin{equation}
    \fra{\sen(48^\circ)}{150}=\fra{\sen(62^\circ)}{AC}.\label{sec2_at_drone_itema_eq3}
\end{equation}
Substituindo os valores de $\sen(62^\circ)=0,88$ e $\sen(48^\circ)=0,74$ encontrados na tabela trigonométrica do final do capítulo e aproximados por duas casas decimais, na equação acima encontramos:
$$\fra{0,74}{150}=\fra{0,88}{AC} \iff AC=178,37\text{m}.$$

No triângulo $CAD$, temos
$$\sen(70^\circ)=\fra{CD}{AC}.$$
Usando novamente a tabela trigonométrica encontramos $\sen(70^\circ)=0,93$, que pode ser substituído na equação acima para encontrar o seguinte:
$$0,93=\fra{CD}{178,37} \iff CD=165,88\text{m}.$$
Sendo assim, a altura do drone é $165,88$m e podemos concluir que as imagens feitas por ele não terão qualidade suficiente para a exibição das imagens no canal de televisão.
\end{enumerate}
}{9}
\end{answer}

\explore{Trabalhando com um triângulo qualquer}\label{exp_outrarelacaonotriangulo}
Vimos anteriormente que a lei dos cossenos estende o teorema de Pitágoras, na medida em que fornece uma relação válida em um triângulo qualquer e que recai sobre o teorema de Pitágoras quando se trata de um triângulo retângulo. Nesta seção, mostraremos que podemos construir uma outra relação válida em um triângulo qualquer, mas que difere da lei dos cossenos por utilizar diferentes informações do triângulo.

\begin{task}{Medindo distâncias inacessíveis}
Um drone está sobrevoando um campo de futebol em formato retangular para fazer imagens de um jogo que será transmitido por um canal de televisão. Segundo especialistas em geração de imagem, a altura máxima deste drone para que as imagens possam ter boa qualidade é de $150$m. Considere que o campo possui $120$m de comprimento e $90$m de largura para responder as perguntas a seguir.

Se for preciso, faça aproximações de valores decimais com duas casas decimais.

\begin{enumerate}
    \item{}
    Caso o drone esteja posicionado acima do centro do campo, de modo que duas pessoas posicionadas em dois córneres do campo diametralmente opostos o vejam por um ângulo de $45^\circ$, as imagens produzidas pelo drone desta posição terão boa qualidade?
    
    Dizer que o drone é visto por um ângulo de $45^\circ$ significa que o ângulo entre o segmento que liga os dois córneres onde estão posicionadas as pessoas e o segmento que liga os pés da pessoa posicionada sobre um córner e o drone é $45^\circ$. Veja a \Fref{sec2_leidossenos_atdrone_fig1}.
    \begin{figure}[H]
    \centering
    \includegraphics[scale=0.6]{sec2_at_drone_enunciado.png}
    \caption{Drone posicionado sobre um campo de futebol (Esta figura é apenas ilustrativa da situação, mas precisaremos de uma figura feita pelo designer contendo o campo, o drone e etc.). 
    }
    \label{sec2_leidossenos_atdrone_fig1}
\end{figure}

    
    \item{}
    Agora, considere que o drone esteja posicionado em um ponto qualquer acima do campo, de forma que duas pessoas posicionadas em dois córneres do campo diametralmente opostos o vejam por ângulos de $70^\circ$ e $62^\circ$. Neste caso, as imagens produzidas desta posição terão boa qualidade? 
\end{enumerate}
\end{task}

\arrange{Lei dos Senos}

Aplicando a lei dos cossenos a um triângulo qualquer, podemos determinar a medida de seus lados, caso sejam conhecidos os outros dois e o ângulo entre eles, ou de seus ângulos, caso sejam conhecidas as medidas de seus três lados. Mas, se conhecemos apenas a medida de um lado e de dois ângulos de um triângulo, não é possível aplicar a lei dos cossenos. Nesta seção, o objetivo é encontrar uma nova relação válida para qualquer triângulo envolvendo as medidas dos lados e dos ângulos de um triângulo qualquer.  Esta relação será chamada lei dos senos.

Primeiramente, enunciaremos e demonstraremos a lei dos senos para triângulos acutângulos. Mais adiante, estenderemos esse resultado para triângulos retângulos e obtusângulos.

\begin{observationtitle}{Lei dos Senos - Parte 1}
\phantomsection\label{sec2_leidossenos}
Seja $ABC$ um triângulo acutângulo, onde $a, b$ e $c$ são as medidas dos lados $BC, AC$ e $AB$, e $\alpha, \beta, \gamma$ são as medidas dos ângulos $B\hat{A}C,C\hat{B}A$ e $A\hat{C}B$, respectivamente (\Fref{sec2_leidossenos_parte1_demo_enunciado_fig}). Então,
\begin{equation}\label{sec2_leidossenos_eq_parte1}
    \frac{a}{\sen\alpha} = \frac{b}{\sen\beta} = \frac{c}{\sen\gamma}.
\end{equation}
\begin{figure}[H]
    \centering
    \includegraphics[scale=0.7]{sec2_leidossenos_parte1_enunciado.png}
    \caption{$ABC$ é um triângulo acutângulo como do enunciado da lei dos senos - parte 1.}
    \label{sec2_leidossenos_parte1_demo_enunciado_fig}
\end{figure}
\end{observationtitle}

\paragraph{Demonstração da Lei dos Senos - Parte 1}

Considere, então, o triângulo acutângulo $ABC$, como mostrado na \Fref{sec2_leidossenos_parte1_demo_enunciado_fig}.

Seja $CD$ a altura do triângulo $ABC$ traçada a partir do vértice $C$ e $h$ sua medida. Veja a \Fref{sec2_leidossenos_parte1_demo1_fig}.
\begin{figure}[H]
    \centering
    \includegraphics[scale=0.7]{sec2_leidossenos_parte1_demo1.png}
    \caption{$ABC$ é um triângulo acutângulo com altura $CD$.}
    \label{sec2_leidossenos_parte1_demo1_fig}
\end{figure}

No triângulo $CBD$, temos que 
\begin{equation}
\sen\beta=\fra{CD}{BC}=\fra{h}{a}.    \label{sec2_leidossenos_demo_parte1_eq1}
\end{equation}
E, no triângulo $CAD$, temos que
\begin{equation}
\sen\alpha=\fra{CD}{AC}=\fra{h}{b}.    \label{sec2_leidossenos_demo_parte1_eq2}
\end{equation}

De \eqref{sec2_leidossenos_demo_parte1_eq1} e \eqref{sec2_leidossenos_demo_parte1_eq2}, vemos que
$$a \sen\beta = b \sen\alpha.$$
Logo,
\begin{equation}
\fra{\sen\alpha}{a}=\fra{\sen\beta}{b}.    \label{sec2_leidossenos_demo_parte1_eq3}
\end{equation}

Agora, seja $AE$ a altura do triângulo $ABC$ traçada a partir de $A$ e denotemos por $h'$ sua medida. A situação está esboçada na \Fref{sec2_leidossenos_parte1_demo2_fig}.
\begin{figure}[H]
    \centering
    \includegraphics[scale=0.8]{sec2_leidossenos_parte1_demo2.png}
    \caption{$ABC$ é um triângulo acutângulo com altura $AE$.}
    \label{sec2_leidossenos_parte1_demo2_fig}
\end{figure}

No triângulo $ACE$, temos que 
\begin{equation}
\sen\gamma=\fra{AE}{AC}=\fra{h'}{b}.    \label{sec2_leidossenos_demo_parte1_eq4}
\end{equation}
E, no triângulo $ABE$, temos que
\begin{equation}
\sen\beta=\fra{AE}{AB}=\fra{h'}{c}.    \label{sec2_leidossenos_demo_parte1_eq5}
\end{equation}

De \eqref{sec2_leidossenos_demo_parte1_eq4} e \eqref{sec2_leidossenos_demo_parte1_eq5}, vemos que
$$b \sen\gamma = c \sen\beta.$$
Logo,
\begin{equation}
\fra{\sen\beta}{b}=\fra{\sen\gamma}{c}.    \label{sec2_leidossenos_demo_parte1_eq6}
\end{equation}

Unindo \eqref{sec2_leidossenos_demo_parte1_eq3} e \eqref{sec2_leidossenos_demo_parte1_eq6}, obtemos
$$\frac{a}{\sen\alpha} = \frac{b}{\sen\beta} = \frac{c}{\sen\gamma},$$
como desejado.


Agora, nosso desafio é compreender se é possível encontrar um resultado semelhante ao encontrado na lei dos senos - parte 1 para triângulos retângulos e obtusângulos. É importante ressaltar que, a relação encontrada anteriormente envolve o seno dos ângulos internos do triângulo. Isso não foi um problema, já que trabalhamos com um triângulo acutângulo e portanto, seus ângulos internos são todos agudos. E, para ângulos agudos, já sabemos como calcular seu seno. 

Agora, queremos generalizar este resultado para qualquer triângulo. Para isso, precisamos trabalhar com triângulos retângulos e obtusângulos e neste caso, será necessário calcular o seno de ângulos retos e obtusos, além de agudos. Para, então, proceder com essa extensão da lei dos senos, vamos ampliar a definição de seno de um ângulo agudo para um ângulos qualquer entre $0^\circ$ e $180^\circ$, como fizemos com o cosseno desses mesmos ângulos ao trabalhar a lei dos cossenos. 

Definimos:
\begin{itemize}[topsep=0pt, itemsep=0pt]
\item $\sen(0^\circ)= 0$;
\item $\sen(90^\circ)= 1$;
\item $\sen(180^\circ)= 0$;
\item para um ângulo obtuso $\alpha$, $\sen\alpha=\sen(180^\circ-\alpha).$
\end{itemize}

Nesta definição, utilizamos o seno de $180^\circ-\alpha$ para definir o seno de $\alpha$. Observe que se $\alpha$ é um ângulo obtuso, então $180^\circ-\alpha$ é um ângulo agudo e portanto, sabemos calcular seu seno com o conteúdo que aprendemos até aqui. 

\begin{observation}{}
Já sabemos que o seno de um ângulo agudo $\alpha$ assume apenas valores positivos e menores que $1$, então, de acordo com a definição acima,  
$$\text{se }0^\circ \leq\alpha \leq180^\circ, \text{então} 0\leq\sen\alpha\leq 1.$$
\end{observation}

No que se segue, utilizaremos a definição anterior para trabalhar com o seno de ângulos entre $0^\circ$ e $180^\circ$. Como já mencionado anteriormente, caso você queira saber um pouco mais sobre essa definição, veja o \textit{Para saber+ Explorando a definição de senos e cossenos}. 

\begin{observationtitle}{Relação Fundamental para ângulos entre $0^\circ$ e $180^\circ$}
Antes de prosseguir com o estudo da generalização da lei dos senos, já que agora sabemos calcular o seno e cosseno de $0^\circ, 90^\circ, 180^\circ$ e de ângulos obtusos, vamos discutir se a relação fundamental válida para ângulos agudos pode ser estendida para qualquer ângulo entre $0^\circ$ e $180^\circ$:

$\bullet$ se $\sen(0^\circ)= 0$ e $\cos(0^\circ)= 1$, então $\sen^2(0^\circ)+\cos^2(0^\circ)=1$;

$\bullet$ se $\sen(90^\circ)= 1$ e $\cos(90^\circ)= 0$, então $\sen^2(90^\circ)+\cos^2(90^\circ)=1$;

$\bullet$ se $\sen(180^\circ)= 0$ e $\cos(180^\circ)= -1$, então $\sen^2(180^\circ)+\cos^2(180^\circ)=1$;

$\bullet$ se $\alpha$ é um ângulo obtuso onde  $\sen\alpha=\sen(180^\circ-\alpha)$ e $\cos\alpha=-\cos(180^\circ-\alpha)$, então
\begin{equation}
  \sen^2\alpha+\cos^2\alpha=(\sen(180^\circ-\alpha))^2+(-\cos(180^\circ-\alpha))^2. \label{sec2_relacaofundamental_angobt1}  
\end{equation}

Como $\alpha$ é um ângulo obtuso, então $180^\circ-\alpha$ é um ângulo agudo e portanto, já conhecemos a relação fundamental que estabelece neste caso que
\begin{equation}
  \sen^2(180^\circ-\alpha)+\cos^2(180^\circ-\alpha)=1. \label{sec2_relacaofundamental_angobt2}  
\end{equation}

Por \eqref{sec2_relacaofundamental_angobt1} e \eqref{sec2_relacaofundamental_angobt2}, 
$$\sen^2\alpha+\cos^2\alpha=1.$$

Sendo assim, concluímos que, para todo ângulo $\alpha$ entre $0^\circ$ e $180^\circ$, vale que
$$\sen^2\alpha+\cos^2\alpha=1.$$
\end{observationtitle}

Como já sabemos calcular o seno de qualquer ângulo entre $0^\circ$ e $180^\circ$, então estamos prontos para  enunciar e demonstrar a lei dos senos para triângulos retângulos e obtusângulos. 

\begin{observationtitle}{Lei dos Senos - Parte 2}
\phantomsection\label{sec2_leidossenos_parte2}
Seja $ABC$ um triângulo retângulo em $A$ ou obtusângulo, onde $a, b$ e $c$ são as medidas dos lados $BC, AC$ e $AB$, e $\alpha, \beta, \gamma$ são as medidas dos ângulos $B\hat{A}C,C\hat{B}A$ e $A\hat{C}B$, respectivamente (\Fref{sec2_leidossenos_parte2_enunciado_fig}). Então,
\begin{equation}\label{sec2_leidossenos_eq_parte2}
    \frac{a}{\sen\alpha} = \frac{b}{\sen\beta} = \frac{c}{\sen\gamma}.
\end{equation}
\begin{figure}[H]
    \centering
    \includegraphics[scale=0.6]{sec2_leidossenos_parte2_demo3.png}
    \qquad
    \includegraphics[scale=0.6]{sec2_leidossenos_parte2_enunciado1.png}
    \caption{Triângulos $ABC$ retângulo (da esquerda) e obtusângulo (da direita) como do enunciado da lei dos senos - parte 2.}
    \label{sec2_leidossenos_parte2_enunciado_fig}
\end{figure}

\end{observationtitle}


\paragraph{Demonstração da Lei dos Senos - Parte 2}

Nesta demonstração, precisamos considerar que o triângulo $ABC$ pode ser retângulo ou obtusângulo, como mostra a \Fref{sec2_leidossenos_parte2_enunciado_fig}. Sendo assim, organizaremos esta demonstração em duas partes. 

Primeira parte: $ABC$ é retângulo em $A$.

Neste caso, estamos considerando que $\alpha=90^\circ$ (veja a \Fref{sec2_leidossenos_parte2_enunciado_fig}).

Observando o triângulo $ABC$, obtemos que:
\begin{eqnarray}
\sen\beta=\fra{AC}{BC}=\fra{b}{a} \iff a=\fra{b}{\sen\beta},   \label{sec2_leidossenos_demo_parte2_eq7} \\ 
\sen\gamma=\fra{AB}{BC}=\fra{c}{a} \iff a=\fra{c}{\sen\gamma}.   \label{sec2_leidossenos_demo_parte2_eq8} 
\end{eqnarray}

De \eqref{sec2_leidossenos_demo_parte2_eq7} e \eqref{sec2_leidossenos_demo_parte2_eq8}, encontramos
$$a=\fra{b}{\sen\beta}=\fra{c}{\sen\beta}.$$
Como, por definição, $\sen\alpha=\sen90^\circ=1$, então a equação acima pode ser reescrita da seguinda maneira: 
$$\fra{a}{\sen\alpha}=\fra{b}{\sen\beta}=\fra{c}{\sen\gamma},$$
que é a equação procurada.

Segunda parte: $ABC$ é obtusângulo.

Consideremos, então, o triângulo obtusângulo $ABC$, como mostrado na \Fref{sec2_leidossenos_parte2_enunciado_fig}.

Seja $CD$ a altura do triângulo $ABC$ traçada a partir do vértice $C$ e $h$ sua medida. Veja a \Fref{sec2_leidossenos_parte2_demo1_fig}.
\begin{figure}[H]
    \centering
    \includegraphics[scale=0.6]{sec2_leidossenos_parte2_demo1.png}
    \caption{$ABC$ é um triângulo obtusângulo com altura $CD$.}
    \label{sec2_leidossenos_parte2_demo1_fig}
\end{figure}

No triângulo $CDB$, temos que 
\begin{equation}
\sen\beta=\fra{CD}{BC}=\fra{h}{a}.    \label{sec2_leidossenos_demo_parte2_eq1}
\end{equation}
E, no triângulo $CAD$, temos que
\begin{equation}
\sen(180^\circ-\alpha)=\fra{CD}{AC}=\fra{h}{b}.    \label{sec2_leidossenos_demo_parte2_eq2}
\end{equation}

De \eqref{sec2_leidossenos_demo_parte2_eq1} e \eqref{sec2_leidossenos_demo_parte2_eq2}, vemos que
$$a \sen\beta = b \sen(180^\circ-\alpha).$$
Como $\sen(180^\circ-\alpha)=\sen\alpha$, temos
\begin{equation}
\fra{\sen\alpha}{a}=\fra{\sen\beta}{b}.    \label{sec2_leidossenos_demo_parte2_eq3}
\end{equation}

Agora, seja $AE$ a altura do triângulo $ABC$ traçada a partir de $A$ e denotemos por $h'$ sua medida. A situação está esboçada na \Fref{sec2_leidossenos_parte2_demo2_fig}.
\begin{figure}[H]
    \centering
    \includegraphics[scale=0.6]{sec2_leidossenos_parte2_demo2.png}
    \caption{$ABC$ é um triângulo obtusângulo com altura $AE$.}
    \label{sec2_leidossenos_parte2_demo2_fig}
\end{figure}

No triângulo $ACE$, temos que 
\begin{equation}
\sen\gamma=\fra{AE}{AC}=\fra{h'}{b}.    \label{sec2_leidossenos_demo_parte2_eq4}
\end{equation}
E, no triângulo $ABE$, temos que
\begin{equation}
\sen\beta=\fra{AE}{AB}=\fra{h'}{c}.    \label{sec2_leidossenos_demo_parte2_eq5}
\end{equation}

De \eqref{sec2_leidossenos_demo_parte2_eq4} e \eqref{sec2_leidossenos_demo_parte2_eq5}, vemos que
$$b \sen\gamma = c \sen\beta.$$
Logo,
\begin{equation}
\fra{\sen\beta}{b}=\fra{\sen\gamma}{c}.    \label{sec2_leidossenos_demo_parte2_eq6}
\end{equation}

Unindo \eqref{sec2_leidossenos_demo_parte2_eq3} e \eqref{sec2_leidossenos_demo_parte2_eq6}, obtemos
$$\frac{a}{\sen\alpha} = \frac{b}{\sen\beta} = \frac{c}{\sen\gamma},$$
como desejado.

Após estudar a parte 1 e a parte 2 da lei dos senos, concluímos que, dado um triângulo qualquer $ABC$, onde $a, b$ e $c$ são as medidas dos lados $BC, AC$ e $AB$, e $\alpha, \beta, \gamma$ são as medidas dos ângulos $B\hat{A}C,C\hat{B}A$ e $A\hat{C}B$, respectivamente, então
    $$\frac{a}{\sen\alpha} = \frac{b}{\sen\beta} = \frac{c}{\sen\gamma}.$$

Com esse resultado, encontramos então uma nova relação válida para um triângulo qualquer envolvendo as medidas dos ângulos e dos lados desse triângulo. Sendo assim, conhecemos dois resultados válidos em triângulos quaisquer que fornecem esse tipo de relação: lei dos cossenos e lei dos senos. A aplicabilidade de cada uma das leis será determinada pelos dados do problema a ser resolvido.

%%%%% Exemplo da lei dos senos
\begin{example}{Distância entre dois corredores} \label{sec2_leidossenos_exemplo}

Um corredor corre em linha reta partindo do ponto $A$ no sentido $AX$ e com velocidade constante igual à $v_S= 8,0$ m/s. Já um outro corredor, se desloca a partir de $B$ em linha reta no sentindo $BX$ e com velocidade constante igual à $v_B=8\sqrt{3}$ m/s, e pretende alcançar o corredor $A$ no ponto $X$. Veja a \Fref{sec2_leidossenos_ex_enunciado}.
\begin{figure}[H]
    \centering
    \includegraphics[scale=0.7]{sec2_leidossenos_ex_enunciado.png}
    \caption{Corredores partindo dos pontos $A$ e $B$.}
    \label{sec2_leidossenos_ex_enunciado}
\end{figure}

Supondo que os corredores partem simultaneamente de suas posições iniciais $A$ e $B$ e que o ângulo $B\hat{A}X$ mede $120^\circ$, determine a medida do ângulo $\alpha$ que a trajetória de $B$ deve fazer com o segmento $AB$ para que o encontro entre os dois corredores aconteça no ponto $X$.

\paragraph{Solução}
\phantomsection\label{sec2_leidossenos_exemplo_res}
 Antes de iniciarmos a questão, precisamos lembrar da Física que a distância percorrida por um corredor com velocidade constante $v$ em um intervalo de tempo $t$ é $d=v\cdot t$. Utilizaremos este fato mais adiante na resolução da atividade.
 
 Para que os corredores se encontrem no ponto $X$ tendo partido simultaneamente de suas posições iniciais $A$ e $B$, eles devem levar o mesmo intervalo de tempo $t$ para percorrer seus percursos. Assim, como $v_A= 8,0$ m/s e $v_B=8\sqrt{3}$ m/s são as velocidades dos corredores que partem de $A$ e $B$, respectivamente, a distância percorrida em metros por eles no intervalo de tempo $t$ é $8t$ (partindo de $A$) e $8\sqrt{3}t$ (partindo de $B$). 
 
 Neste caso, o triângulo $ABX$ possui o lado $AX$ medindo $8t$ e o lado $BX$ medindo $8\sqrt{3}t$. Veja a \Fref{sec2_leidossenos_ex_resolucao}.
  \begin{figure}[H]
    \centering
    \includegraphics[scale=0.7]{sec2_leidossenos_ex_res1.png}
    \caption{Trajetórias dos corredores $C_1$ e $C_2$.}
    \label{sec2_leidossenos_ex_resolucao}
\end{figure}

Neste caso, estamos interessadas em encontrar o ângulo $\alpha$, mas não conhecemos todos os lados do triângulo. Por isso, não podemos utilizar a lei dos cossenos. Porém, como conhecemos a medida dos lados opostos aos ângulos $\alpha$ e $B\hat{A}X$, então a lei dos senos é a indicada para resolver o problema.
    
    Usando a lei dos senos no triângulo $ABX$, segue que:
    $$\frac{8t}{\sen\alpha}=\frac{8\sqrt{3}t}{\sen120^\circ}\iff \sen\alpha=\frac{1}{2}.$$ 
    Ora, como nesse caso, $0^\circ < \alpha < 90^\circ$ , segue que $\alpha=30^\circ$.
\end{example}

\begin{knowledge} \label{sec2_parasabermais_trianguloinscrito}
\vspace{-1em}
\paragraph{Triângulo inscrito em um círculo}

A relação apresentada e demonstrada na lei dos senos estabelece que em um triângulo qualquer as medidas de seus lados são proporcionais aos senos de seus ângulos opostos, ou seja, se $ABC$ é um triângulo onde $a, b$ e $c$ são as medidas dos lados $BC, AC$ e $AB$, e $\alpha, \beta, \gamma$ são as medidas dos ângulos $B\hat{A}C,C\hat{B}A$ e $A\hat{C}B$, respectivamente, então
    $$\frac{a}{\sen\alpha} = \frac{b}{\sen\beta} = \frac{c}{\sen\gamma}.$$

Além disso, é possível mostrar que se $ABC$ está inscrito em uma circunferência de raio $R$, então 
$$\frac{a}{\sen\alpha} = \frac{b}{\sen\beta} = \frac{c}{\sen\gamma}=2R.$$

Deixaremos a demonstração desse resultado para o leitor interessado em aprofundar o estudo da Trigonometria. A demonstração pode ser encontrada em \cite{iezzi1993}.
\end{knowledge}

%%%%% Acabou aqui o Para Saber Mais
%%%%%

Agora que sabemos calcular o seno e cosseno de $0^\circ, 180^\circ$ e de ângulos obtusos, podemos então definir a tangente desses ângulos da forma a seguir.

\begin{description}\label{sec2_deftangente}
\item[Definição] Definimos:
\begin{itemize}
\item $\tg 0^\circ= 0$;
\item $\tg 180^\circ= 0$;
\item para um ângulo obtuso $\alpha$, $\tg\alpha=\dfrac{\sen\alpha}{\cos\alpha}.$
\end{itemize}
\end{description}

Note que, a tangente do ângulo de $90^\circ$ não é definida. Isto está ligado ao fato de $\cos(90^\circ)$ ser $0$. Para saber mais sobre isso, leia o \textit{Para saber+ Explorando a definição de senos e cossenos}.

%%%%%
\clearmargin
\def\currentcolor{session2}
\begin{objectives}{UNESP 2011, Questão 89 - Adaptada}
{
\begin{itemize}
\item Adquirir familiaridade com a lei dos senos através de sua aplicação em um problema contextualizado.

\item \textbf{Conceitos abordados}: lei dos senos.
\end{itemize}
}{1}{2}
\end{objectives}
\begin{sugestions}{UNESP 2011, Questão 89 - Adaptada}
{
Recomenda-se ao professor estimular os alunos a raciocinar sobre como escolher o resultado mais adequado para ser usado na solução da atividade. Nesse caso, não há possibilidade de utilizar a lei dos cossenos pelos dados fornecidos no enunciado e os alunos precisam estar cientes disso. 

\textbf{Organização da turma}: em duplas.

\textbf{Dificuldades previstas}: como este não é um problema resolvido diretamente pela aplicação de uma única fórmula, os estudantes podem não compreender, a priori, qual o primeiro passo da solução. 
%
Neste caso, recomenda-se que o professor fique  alerta para auxiliar os estudantes a construir seu raciocínio neste primeiro momento. 
%
Além disso, a fórmula da lei dos senos contém quocientes onde no denominador encontramos o seno de um ângulo e no denominador a medida do lado oposto a esse ângulo. 
%
Sendo assim, relacionar lado oposto e ângulo corretamente é fundamental para a aplicação correta da lei dos senos. Como já foi relatado anteriormente, essa relação é comumente feita de maneira errada pelos estudantes e por isso requer especial atenção do professor.
}{1}{2}
\end{sugestions}
\begin{answer}{UNESP 2011, Questão 89 - Adaptada}
{
Primeiramente, notamos que não é possível trabalhar apenas com o triângulo $BCD$, visto que só conhecemos os seus ângulos. Assim, precisamos recorrer ao triângulo $ABC$ para obter pelo menos um lado do triângulo $BCD$ e daí poderemos obter os outros lados usando a lei dos senos.

Sabendo que a soma dos ângulos internos de um triângulo qualquer é $180^\circ$, obtemos que a medida do ângulo  $A\hat{B}C$ é $180^\circ-30^\circ-105^\circ=45^\circ$.  Agora, como conhecemos a medida de um dos lados desse triângulos, podemos aplicar a lei dos senos para encontrar a medida de $BC$:
\begin{equation}\label{sec2_resativmastrobandeira2}
    \fra{BC}{\sen(30^\circ)}=\fra{AC}{\sen(45^\circ)}.
\end{equation}

Além disso, no triângulo retângulo $BCD$, observamos que
\begin{equation}\label{sec2_resativmastrobandeira1}
    \sen(30^\circ)=\fra{h}{BC} \iff \fra12=\fra{h}{BC} \iff BC=2h.
\end{equation}

Substituindo \eqref{sec2_resativmastrobandeira1} em \eqref{sec2_resativmastrobandeira2} e os valores de $\sen(30^\circ)=1/2$ e $\sen(45^\circ)=\sqrt{2}/2$, obtemos o seguinte:
$$\fra{2h}{1/2}=\fra{50}{\sqrt{2}/2} \iff h=\fra{25}{2}\sqrt{2}\text{m}.$$

Logo, altura procurada é $\fra{25}{2}\sqrt{2}$m.
}{0}
\end{answer}

\clearmargin
\begin{objectives}{Escorando a Torre de Pisa}
{
\begin{itemize}
\item Aplicar a lei dos senos em um problema contextualizado.

\item Conceitos abordados: lei dos senos.
\end{itemize}

}{1}{1}
\end{objectives}
\begin{sugestions}{Escorando a Torre de Pisa}
{
Mais uma vez, sugerimos ao professor que estimule o aluno a construir uma estratégia de resolução antes de iniciar os cálculos, a partir dos dados do problema. Por exemplo, o estudante deve perceber que não há possibilidade de utilizar a lei dos cossenos com os dados fornecidos no enunciado e o porquê da escolha da lei dos senos para solucionar esta questão.

Nesta atividade serão utilizados os seguintes dados, que poderão ser acessados pelos estudantes na tabela trigonométrica do fim do capítulo: $\sen(95,5^\circ)=0,99$ e  $\cos(95,5^\circ)=-0,1$. Além disso, será necessário acessar a tabela de forma a encontrar o ângulo que possui o seno igual a $0,7$, ou seja, $\arcsen({0,7}) = 46^\circ$. Mais uma vez, lembramos que a linguagem de função trigonométrica inversa não deve ser utilizada pelo professor em sala de aula.
}{1}{1}
\end{sugestions}
\begin{answer}{Escorando a Torre de Pisa}
{
Repare que, em relação ao triângulos $ABC$, conhecemos dois de seus lados e o ângulo entre eles. Para resolver a questão, precisamos encontrar a medida de $AC$ e o ângulo $C\hat{A}B$ do triângulo $ABC$. Neste caso,  aplicando a lei dos cossenos encontramos o comprimento do cabo $AC$ e aplicando a lei dos senos encontramos o ângulo $C\hat{A}B$. Vamos encontrar cada um desses elementos nos itens a seguir.

\begin{enumerate}
    \item{}
    Pela lei dos cossenos, temos que 
    $$AC^2=50^2+56^2-2\cdot50\cdot56\cdot\cos(95,5^\circ).$$
    
    Como $\cos(95,5^\circ)=-\cos(180^\circ-95,5^\circ)=-\cos(84,5^\circ)$, vamos aproximar o $\cos(95,5^\circ)$ por $-0,1$, já que $\cos(84^\circ)=0,1$ pela tabela trigonométrica (é possível fazer a aproximação utilizando o ângulo de $85^\circ$, caso se queira). Substituindo na equação acima temos:
    $$AC^2=50^2+56^2-2\cdot50\cdot56\cdot(-0,1)=6196.$$
    Logo, $AC=78,71$m, que é a medida do cabo que poderia sustentar a torre para que ela não desabasse.
 
    \item{} 
    Utilizando diretamente a lei dos senos, temos
    $$\fra{BC}{\sen(C\hat{A}B)}=\fra{AC}{\sen(A\hat{B}C)} \iff \fra{56}{\sen (C\hat{A}B)}=\fra{78,71}{\sen(95,5^\circ)}.$$
    
    Como $\sen(95,5^\circ)=\sen(180^\circ-95^\circ)=\sen(84,5^\circ)$, pela tabela trigonométrica encontramos $\sen(95,5^\circ)=0,99$. Logo, a equação acima pode ser reescrita da seguinte forma:
    $$\fra{56}{\sen(C\hat{A}B)}=\fra{78,71}{0,99}.$$
    
    Logo, $\sen(C\hat{A}B)=0,7$. Com o auxílio da tabela trigonométrica, encontramos que o ângulo que possui seno igual a $0,7$ é $46^\circ$. Sendo assim, o ângulo de inclinação do cabo em relação ao chão é $46^\circ$.
\end{enumerate}
}{0}
\end{answer}
\def\currentcolor{session4}

\begin{reflection}
 
No estudo da Geometria Euclidiana, diz-se que duas figuras são congruentes quando uma pode ser transformada na outra pelos chamados movimentos rígidos (rotação e translação). Como polígonos em geral podem ser decompostos em triângulos, costumamos dar uma atenção especial à congruência de triângulos. 

Você se lembra quais são as condições mínimas para que dois determinados triângulos sejam congruentes? Vamos relembrar os casos de congruência de triângulos. 
\begin{itemize}
    \item Caso LLL: se dois triângulos têm os lados correspondentes congruentes, então eles são congruentes.
    \begin{figure}[H]
    \centering
   \includegraphics[scale=0.6]{LLL.JPG}
    \caption{$AB \equiv DE, BC\equiv EF, CA \equiv FD \Rightarrow ABC \equiv DEF$.}
    \label{LLL}
    \end{figure}
    
    \item Caso LAL: se dois triângulos têm dois lados correspondentes congruentes, e o ângulo entre esses lados são congruentes, então eles são congruentes.
    \begin{figure}[H]
    \centering
   \includegraphics[scale=0.6]{LAL.JPG}
    \caption{$AB \equiv DE, BC\equiv EF, A\hat{B}C \equiv D\hat{E}F \Rightarrow ABC \equiv DEF$.}
    \label{LAL}
    \end{figure}
    
    \item Caso ALA: se dois triângulos têm dois ângulos iguais, então eles são congruentes.
    \begin{figure}[H]
    \centering
   \includegraphics[scale=0.6]{ALA.JPG}
    \caption{ $BC\equiv EF, A\hat{B}C \equiv D\hat{E}F, A\hat{C}B \equiv D\hat{F}E  \Rightarrow ABC \equiv DEF$.}
    \label{ALA}
    \end{figure}
\end{itemize}
Utilizando as leis dos cossenos e senos, você conseguiria justificar os três casos de congruência de triângulos mencionados acima? Além disso, se dois triângulos têm um lado e dois ângulos em comum (LAA) podemos garantir que eles são congruentes? Justifique!
\end{reflection}



\practice{Praticando a lei dos senos}

\begin{task}{UNESP 2011, Questão 89 - Adaptada}
Uma pessoa se encontra no ponto $A$ de uma planície, às margens de um rio e vê, do outro lado do rio, o topo do mastro de uma bandeira, ponto $B$. Com o objetivo de determinar a altura $h$ do mastro, ela anda, em linha reta, $50$m para a direita do ponto em que se encontrava (vide \Fref{sec2_leidossenos_atunesp2}) e marca o ponto $C$. Considerando que $D$ o pé do mastro, os ângulos $B\hat{A}C$ e $B\hat{C}D$ valem $30^\circ$, o ângulo $A\hat{C}B$  mede $105^\circ$ (como mostra a  \Fref{sec2_leidossenos_atunesp2}), calcule a altura $h$ do mastro da bandeira, em metros.
\begin{figure}[H]
    \centering
    \includegraphics[scale=0.4]{sec2_vestibularunesp2011_Q89.png}
    \caption{Questão 89 do vestibular da UNESP 2011. }
    \label{sec2_leidossenos_atunesp2}
\end{figure}
\end{task}

\begin{task}{Escorando a Torre de Pisa}
A Torre de Pisa, campanário de uma igreja localizada na cidade de Pisa na Itália, começou a ser construída em 1173 e durante sua construção começou a se inclinar devido a problemas com sua fundação e o solo. 
%
Com o passar do tempo, essa inclinação aumentou e no ano de 1990 ela chegou a estar inclinada de $5,5^\circ$ em relação ao eixo vertical. 
%
Para evitar seu desmoronamento, engenheiros cogitaram afixar um cabo de aço super resistente ligando o topo da torre e um ponto do chão distante 50m da base da torre. 
%
Veja a \Fref{sec2_leidossenos_attorrepisa}. 

\begin{figure}[H]
    \centering
    \includegraphics[scale=0.6]{sec2_torrepisa.png}
    \caption{Torre de Pisa sendo escorada para evitar seu desabamento (Essa figura precisará ser feita pelo designer).
    %\lhaylla{Essa figura precisa ser refeita, pois contém direitos autorais.}
    }
    \label{sec2_leidossenos_attorrepisa}
\end{figure}


Sabendo que a torre tem aproximadamente $56$m de comprimento, responda às perguntas abaixo aproximando os valores decimais com duas casas decimais.
\begin{enumerate}
    \item{}
    Qual o tamanho do cabo a ser afixado na torre para prendê-la e evitar seu desabamento?
    
    \item{} 
    Qual o ângulo de inclinação do cabo em relação ao chão?
\end{enumerate}
\end{task}

\know{Explorando senos e cossenos}\label{sencostgangqq}

Você pode ter se questionado o porquê da definição de seno e cosseno ter sido dada de forma separada para ângulos agudos e obtusos, e se de alguma forma, não existe uma definição única para seno e cosseno de todos os ângulos com medida entre $0^\circ$ e $180^\circ$. Vamos agora então mostrar de onde vem a ideia dessas definições e uma forma de unificá-las.

Considere uma semicircunferência $\Gamma$ centrada em $O$, raio $1$ e diâmetro $AB$. Marque agora o ponto $C$ sobre $\Gamma$ de forma que $OC$ seja perpendicular à $AB$. Por essa construção, $A\hat{O}C$ é um ângulo central que mede $90^\circ$.

Sejam $X$ um ponto qualquer sobre $\Gamma$ e $\alpha$ o ângulo $X\hat{O}A$. Estamos interessados em estudar o seno, cosseno e tangente do ângulo $\alpha$, que é um ângulo central e por isso já conhecemos algumas de suas propriedades.

Vamos primeiramente considerar que $0\leq\alpha\leq 90^\circ$. Sendo assim, trace o pé da perpendicular baixada de $X$ até $BA$ e o nomeie de $E$. Faça o mesmo para a perpendicular baixada de $X$ até $OC$ nomeando seu pé de $D$. Veja a \Fref{sec2_sencosangagudo}.
\begin{figure}[H]
    \centering
    \includegraphics[scale=0.5]{sec2_sencosangagudo.png}
    \caption{Seno e cosseno de um ângulo agudo qualquer.}
    \label{sec2_sencosangagudo}
\end{figure}

Sendo assim, definimos:
$$\sen\alpha= OD \qquad \text{ e } \qquad \cos\alpha=OE.$$

Pela construção que fizemos, $OXE$ é um triângulo retângulo, e esta definição está totalmente de acordo com a definição de seno e cosseno de ângulos agudos estudada anteriormente, já que $OX=1$.

Usando esta definição, temos:
$$\sen0^\circ = 0 \qquad \text{ e } \qquad \cos0^\circ =1,$$
$$\sen90^\circ = 1 \qquad \text{ e } \qquad \cos90^\circ =0.$$

Agora, suponha que $X$ seja tal que $90^\circ<\alpha\leq 180^\circ$. Analogamente ao feito anteriormente, trace o pé da perpendicular baixada de $X$ até $BA$ e o nomeie de $E$. Faça o mesmo para a perpendicular baixada de $C$ até $OC$ nomeando seu pé de $D$. A situação está ilustrada na \Fref{sec2_sencosangobtuso}.
\begin{figure}[H]
    \centering
    \includegraphics[scale=0.5]{sec2_sencosangobtuso.png}
    \caption{Seno e cosseno de um ângulo obtuso qualquer.}
    \label{sec2_sencosangobtuso}
\end{figure}

Sendo assim, definimos:
$$\sen\alpha= OD \qquad \text{ e } \qquad \cos\alpha= -OE.$$

Neste caso, note que 
$$\sen 180^\circ = 0 \qquad \text{  e  } \qquad\cos 180^\circ = -1.$$

Além disso, desde $X$ e $C$ não coincidam, isto é, $\alpha\neq90^\circ$, definimos a tangente de um ângulo $\alpha$ da seguinte forma:
$$\tg\alpha = \fra{\sen\alpha}{\cos\alpha}.$$

\begin{figure}[H]
    \centering
    \includegraphics[scale=0.5]{sec2_sencossuplementares.png}
    \caption{Seno, cosseno e tangente de ângulos suplementares.}
    \label{sec2_sencosasuplementares}
\end{figure}
A partir de tudo que vimos e observando a \Fref{sec2_sencosasuplementares} onde $\alpha=A\hat{O}X$ e $180^\circ-\alpha=A\hat{O}X'$, relacionamos seno, cosseno e tangente de ângulos suplementares da seguinte forma:
$$\sen(180^\circ-\alpha)=\sen\alpha,\;\;  \cos(180^\circ-\alpha)=-\cos\alpha, \;\; \text{ e } \;\; \tg(180^\circ-\alpha)=-\tg\alpha.$$

Visto isso, concluímos que podemos unificar as definições de senos e cossenos para todos os ângulos entre $0^\circ$ e $180^\circ$ usando uma semicircunferência de raio $1$. Vale ressaltar que as definições para ângulos agudos e não agudos entre $0^\circ$ e $180^\circ$ dadas anteriormente recaem sobre a forma acima.

Agora, o que aconteceria se o raio da semicircunferência $\Gamma$ não fosse $1$, ou seja, se o raio de $\Gamma$ assumisse um valor real positivo qualquer? O que mudaria no que estudamos aqui? Discuta com seus colegas.

Caso tenha acesso a Internet (inclusive de um celular), você pode variar os pontos sobre a semicircunferência $\Gamma$ e calcular o seno e cosseno do ângulos encontrados por meio do aplicativo GeoGebra disponível em: \url{https://www.geogebra.org/m/fq6brzg5}.

%%%%%
\know{Área de um triângulo qualquer} 

Você sabia que existem várias formas diferentes de calcular a área de um triângulo? Vejamos duas delas garantidas pelo que estudamos aqui. Para isso, considere um triângulo $ABC$ com lados $BC,AB$ e $AB$ medindo $a, b$ e $c$, respectivamente. 

(1) Vale a seguinte igualdade: 
\begin{equation}
    \area(ABC)=\fra{bc\cdot\sen(C\hat{A}B)}{2}.\label{sec2_areatriag1}
\end{equation}

\begin{description}[wide]\small
\item[Demonstração:]
\leavevmode\phantomsection\label{sec2_formulaarea1}
Suponha que $ABC$ é um triângulo retângulo em $A$. Considerando $AB$ como base e $AC$ como altura de $ABC$, então 
$$\area(ABC)=\fra{bc}{2}=\fra{bc\cdot\sen(C\hat{A}B)}{2},$$
já que $\sen(C\hat{A}B)=\sen90^\circ=1$.

Suponha, agora, que $ABC$ é acutângulo como na \Fref{sec2_areadotriangulo_ag_res}. Se $D$ é o pé da perpendicular baixada de $B$ sobre $AC$ e h a altura $BD$, então
$$\sen(C\hat{A}B)=\fra{h}{c}\iff h=c\cdot\sen(C\hat{A}B).$$

\begin{figure}[H]
    \centering
    \includegraphics[scale=0.4]{sec2_areadotriangulo_ag.png}
    \caption{Triângulo acutângulo $ABC$.}
    \label{sec2_areadotriangulo_ag_res}
\end{figure}

Portanto,
$$\area(ABC)=\fra{bh}{2}=\fra{bc\cdot\sen(C\hat{A}B)}{2}.$$

Deixaremos para o leitor o caso em que $ABC$ é obtusângulo, já que ele é análogo ao caso anterior.
\end{description}

(2) Vale a seguinte igualdade: 
\begin{equation}
    \area(ABC)=\fra{abc}{4R},\label{sec2_areatriag2}
\end{equation}
onde $R$ é o raio  da circunferência circunscrita ao triângulo $ABC$.

\begin{description}[wide]
\small
\item[Demonstração:]
\phantomsection\label{sec2_formulaarea2} 
Segundo o \textit{Para saber + Triângulo inscrito em um círculo}, se um triângulo $ABC$ está inscrito em uma circunferência, relacionamos seu raio $R$ ao seno do ângulo $C\hat{A}B$ da seguinte forma: 
$$\sen(C\hat{A}B)=\fra{a}{2R}.$$ 
Substituindo essa igualdade na fórmula da área encontrada no item anterior, temos:
$$\area(ABC)=\fra{bc\fra{a}{2R}}{2}=\fra{abc}{4R}.$$
\end{description}


\exercise

%Página 1
\begin{answer}{Exercícios}
{\exerciselist
\begin{enumerate}
\item Letra \textit{b)}
\item Letra \textit{b)}
\end{enumerate}
}{1}
\end{answer}
\clearmargin
%Página 2
\begin{answer}{Exercícios}
{\exerciselist
\begin{enumerate}\setcounter{enumi}{2}
\item Letra \textit{a)}
\end{enumerate}
}{1}
\end{answer}
\clearmargin
%Página 3
\begin{answer}{Exercícios}
{\exerciselist
\begin{enumerate}\setcounter{enumi}{3}
\item Letra \textit{b)}
\item Letra \textit{b)}
\end{enumerate}
}{1}
\end{answer}
\clearmargin
%Página 4
\begin{answer}{Exercícios}
{\exerciselist
\begin{enumerate}\setcounter{enumi}{5}
\item Letra \textit{b)}
\end{enumerate}
}{1}
\end{answer}
\clearmargin
%Página 5
\begin{answer}{Exercícios}
{\exerciselist
\begin{enumerate}\setcounter{enumi}{6}
\item $2(9+2\sqrt{3}$ m
\item Letra \textit{c)}
\end{enumerate}
}{1}
\end{answer}
\clearmargin
%Página 6
\begin{answer}{Exercícios}
{\exerciselist
\begin{enumerate}\setcounter{enumi}{8}
\item $\sqrt{52}$ m
\item 
 \begin{enumerate}
       \item{} 
       Aplicando a lei dos cossenos ao triângulo $ADE$, temos:
       $$12^2=8^2+10^2-2\cdot 8 \cdot 10\cos\hat{A} \iff \cos\hat{A}=\frac{1}{8}.$$
       
       \item{}
       Agora, aplicando a lei dos cossenos ao triângulo $ABC$, segue que:
       \begin{align*}
       BC^2&=25^2+20^2-2\cdot 25 \cdot 20 \cdot \cos\hat{A} \iff \\
       BC^2&=25^2+20^2-2\cdot 25 \cdot 20 \cdot \frac{1}{8} \iff BC=30.
       \end{align*}
       
       \item{}
       Sim, há uma proporcionalidade entre os lados dos triângulos $ADE$ e $ABC$, 
       $$\frac{AD}{AC}=\frac{20}{8}=\frac{2}{5} \ \text{e} \ \frac{AE}{AB}=\frac{10}{25}=\frac{2}{5},$$
       o que revela que eles são triângulos semelhantes.
       
       \item{}
      Como os triângulos  $ADE$ e $ABC$ são semelhantes, segue que
      $$\frac{DE}{BC}=\frac{2}{5} \Rightarrow \frac{12}{BC}=\frac{2}{5} \Rightarrow BC=30.$$
   \end{enumerate}
\end{enumerate}
}{1}
\end{answer}
\clearmargin
%Página 7
\begin{answer}{Exercícios}
{\exerciselist
\begin{enumerate}\setcounter{enumi}{10}
\item Letra \textit{d)}
\item Letra \textit{c)}
\end{enumerate}
}{1}
\end{answer}
\clearmargin
%Página 8
\begin{answer}{Exercícios}
{\exerciselist
\begin{enumerate}\setcounter{enumi}{12}
\item Letra \textit{a)}
\end{enumerate}
}{1}
\end{answer}
\clearmargin
%Página 9
\begin{answer}{Exercícios}
{\exerciselist
\begin{enumerate}[wide]\setcounter{enumi}{13}
\item Letra \textit{d)}
\item 
\begin{enumerate}[wide]
    \item{}
    No triângulo $ACD$, 
    $$\cos\alpha=\frac{1}{AC} \iff AC=\frac{1}{\cos\alpha} \ \ \text{e}\ \  e \tg\alpha=\frac{CD}{1} \iff CD=\tg\alpha.$$

    \item{}
     No triângulo $ABD$, 
     $$\cos\beta=\frac{1}{AB} \iff AB=\frac{1}{\cos\beta} \ \text{e} \ \tg\beta=\frac{BD}{1} \iff BD=\tg\beta.$$
     
     \item{} 
     A área do triângulo $ABC$ pode ser calculada das seguintes formas:
     $$\text{Area}(ABC)=\frac{1}{2}AB\cdot AC\cdot\sen(\alpha+\beta) \ \text{e} \ \text{Area}(ABC)=\frac{1}{2}BC\cdot AD.$$
     Portanto,
    \begin{align*}
         \frac{1}{2}AB\cdot AC\cdot\sen(\alpha+\beta)&=\frac{1}{2}BC\cdot AD \iff\\
         AB\cdot AC \cdot \sen(\alpha+\beta)&=BC\cdot AD=BC \cdot 1=BC.
    \end{align*}     
     Logo,
    \begin{align*}
         AB\cdot AC \cdot \sen(\alpha+\beta)&=BC=BD+DC \iff\\
         \frac{1}{\cos\beta}\cdot\frac{1}{\cos\alpha}\sen(\alpha+\beta)&=\tg\beta+\tg\alpha \iff\\
         \sen(\alpha+\beta)&=\frac{\sen\alpha}{\cos\alpha}\cos\beta\cos\alpha+\frac{\sen\beta}{\cos\beta}\cos\beta\cos\alpha \iff\\
         \sen(\alpha+\beta)&=\sen\alpha\cos\beta+\sen\beta\cos\alpha.
    \end{align*}
\end{enumerate}
\end{enumerate}
}{1}
\end{answer}
\clearmargin
%Página 10
\begin{answer}{Exercícios}
{\exerciselist
\begin{enumerate}[wide]\setcounter{enumi}{14}
\item 
\begin{enumerate}[wide]\setcounter{enumii}{3}
    \item{}
     Tomando $\alpha=\beta$ na fórmula $\sen(\alpha+\beta)=\sen\alpha\cos\beta+\sen\beta\cos\alpha$, segue que
     $$\sen(\alpha+\alpha)=\sen\alpha\cos\alpha+\sen\alpha\cos\alpha \iff \sen(2\alpha)=2\sen\alpha\cos\alpha.$$
     
     \item{}
     No triângulo $EFG$, 
     $$\tg\alpha=\frac{FG}{1} \iff FG=\tg\alpha \ \text{e} \ \cos\alpha=\frac{1}{EG} \iff EG=\frac{1}{\cos\alpha}.$$
     
     No triângulo $EFH$, 
     $$\tg\beta=\frac{FH}{1} \iff FH=\tg\beta \ \text{e} \ \cos\beta=\frac{1}{EH} \iff EH=\frac{1}{\cos\beta}.$$ 
     
     Por fim,
    \begin{align*}
     \frac{1}{2} \cdot EH \cdot EG \cdot \sen(\alpha-\beta)&=\frac{1}{2}\cdot EF\cdot FG-\frac{1}{2}\cdot EF\cdot FH \iff\\
     EH\cdot EG&=1\cdot FG-1\cdot FH=FG-FH  \iff\\
     \frac{1}{\cos\beta}.\frac{1}{\cos\alpha}\cdot\sen(\alpha-\beta)&=\tg\alpha-\tg\beta \iff\\
     \sen(\alpha-\beta)&=\frac{\sen\alpha}{\cos\alpha}\cos\beta\cos\alpha-\frac{\sen\beta}{\cos\beta}\cos\beta\cos\alpha\iff\\
     \sen(\alpha-\beta)&=\sen\alpha\cos\beta-\sen\beta\cos\alpha.
    \end{align*}
\end{enumerate}

\item Considere a \Fref{retangulos1}:
     \begin{figure}[H]
    \centering
    \includegraphics[scale=0.3]{Retangulos1.JPG}
    \caption{Retângulos de lados $a$ e $3a$.}
    \label{retangulos1}
\end{figure}
\begin{enumerate}
    \item {}
     Pelo teorema de Pitágoras, $AB=\sqrt{(3a)^2+a^2}=a\sqrt{10}$, $BC=\sqrt{(2a)^2+(3a)^2}=a\sqrt{13}$ e $AC=\sqrt{(4a)^2+a^2}=a\sqrt{17}.$ Aplicando a lei dos cossenos no triângulo $ABC$, segue que:
    $$BC^2=AB^2+AC^2-2\cdot AB \cdot AC\cdot\cos(B\hat{A}C).$$
    
    Logo,  
    $$13a^2=10a^2+17a^2-2a^2\sqrt{170}\cos(B\hat{A}C),$$
    ou seja, $\cos(B\hat{A}C)=\frac{7}{\sqrt{170}}$.
    
    \item{}
    Como $\cos(B\hat{A}C)=\frac{7}{\sqrt{170}}$, verificamos na tabela trigonométrica do capítulo que, $B\hat{A}C\approx 57^\circ$.
\end{enumerate}
\end{enumerate}
}{1}
\end{answer}
\begin{answer}{Exercícios}
{\exerciselist
\begin{enumerate}\setcounter{enumi}{16}
\item Aplicando a lei do cossenos nos triângulos $ABD$ e $ABC$, segue que
    $$\begin{cases}
    d_1^2=a^2+b^2-2ab\cos(D\hat{A}B)\\
    d_2^2=a^2+b^2-2ab\cos(180^\circ-D\hat{A}B)
    \end{cases} \iff
    \begin{cases}
    d_1^2=a^2+b^2-2ab\cos(D\hat{A}B)\\
    d_2^2=a^2+b^2+2ab\cos(180^\circ-D\hat{A}B)
    \end{cases} 
    $$
    Adicionando membro a membro as igualdades acima, obtemos 
    $$d_1^2+d_2^2=2(a^2+b^2).$$
\end{enumerate}
}{0}
\end{answer}
\clearmargin
%Página 11
\begin{answer}{Exercícios}
{\exerciselist
\begin{enumerate}\setcounter{enumi}{18}
\item Ora, como as velocidades dos amigos estavam na razão $1:2:4$, após um tempo $t$ as suas distâncias em relação ao ponto de partida, que chamaremos de ponto O, serão $d, 2d$ e $4d$, respectivamente. Sendo $A, B$ e $C$ suas posições nesse instante, pela lei dos cossenos, segue que:
  \begin{align*}
      AB^2&=(2d)^2+d^2-2\cdot2d\cdot d\cdot\cos120^\circ \iff AB^2=7d^2,
      BC^2&=(2d)^2+(4d)^2-2\cdot2d\cdot4d.\cos120^\circ \iff BC^2=28d^2,
      AC^2&=(4d)^2+d^2-2\cdot 4d\cdot d\cdot\cos120^\circ \iff AC^2=21^2.
  \end{align*}
    Como $BC^2=AB^2+AC^2$, segue que o triângulo $ABC$ é retângulo.
        
\item
\begin{enumerate}
\item{}
Primeiramente precisamos traçar a bissetriz $BD$ de $A\hat{B}C$, como podemos ver na \Fref{Aurea1}. Note que o triângulo $BCD$ é isósceles de base $CD$, o que revela que $BD=BC=1$.
\begin{figure}[H]
\centering
\includegraphics[scale=0.4]{Aurea1.JPG}
\caption{ $BD$ é a bissetriz do ângulo $A\hat{B}C$.}
\label{Aurea1}
\end{figure}

Como os triângulos $ABC$ e $BCD$ são semelhantes, segue que
$$\frac{1}{1+x}=\frac{x}{1} \iff x^2+x-1=0 \iff x=\frac{\sqrt{5}-1}{2}.$$

Aplicando a lei dos senos do triângulo $BCD$, segue que 
$$\frac{x}{\sen36^\circ}=\frac{1}{\sen72^\circ} \iff \frac{x}{\sen36^\circ}=\frac{1}{2\sen36^\circ\cdot\cos36^\circ}.$$

Assim, $\cos36^\circ=\frac{1}{2x}=\frac{\sqrt{5}+1}{4}$.
    
\item{}
Ora, como $\varphi=\frac{\sqrt{5}+1}{2}$  e $\cos36^\circ=\frac{1+\sqrt{5}}{4}$, segue que
$$\cos36^\circ=\frac{\sqrt{5}+1}{4}=\frac{1}{2}\cdot \frac{\sqrt{5}+1}{2}=\frac{1}{2}\varphi.$$
\end{enumerate}
\end{enumerate}
}{1}
\end{answer}
\begin{answer}{Exercícios}
{\exerciselist
\begin{enumerate}\setcounter{enumi}{19}
\item Numa circunferência, cordas congruentes correspondem a arcos congruentes. Como, neste caso, temos seis cordas de comprimento $10$ (que correspondem a arcos de medida angular $\alpha$) e seis cordas de comprimento $20$ (que correspondem a arcos de medida angular $\beta$), existem seis arcos congruente de medida $\alpha$ e outros seis arcos de medida $\beta$. Assim,
    $$6\alpha+6\beta=360^\circ \iff \alpha+\beta=60^\circ.$$
   
    Além disso, $D\hat{C}B=\frac{1}{2}\cdot 300^\circ=150^\circ$, já que o ângulo $D\hat{C}B$ é um ângulo inscrito numa circunferência e, portanto, sua medida é igual a metade do arco correspondente, que nesse caso é $5(\alpha+\beta)=5\cdot60^\circ=300^\circ$.
    
    Conectando o centro $O$ com os vértices $B$ e $D$, o ângulo central $B\hat{O}D=\alpha+\beta=60^\circ$. Ora, como $OD=OB=R$, conforme ilustra a \Fref{praca2}, o triângulo $OBD$ é equilátero, implicando que $DB=R$. Por fim, aplicando a lei dos cossenos no triângulo $BCD$, segue que:
    $$R^2=10^2+20^2-2\cdot10\cdot20\cdot\cos150^\circ.$$
    Assim, 
    $$R^2=100+400-400.\left(-\frac{\sqrt{3}}{2}\right) \iff R=10\sqrt{5+2\sqrt{3}} \text{m}.$$
     \begin{figure}[H]
    \centering
    \includegraphics[scale=0.4]{Praca2.JPG}
    \caption{Calculando o raio da praça em formato circular.}
    \label{praca2}
    \end{figure}
    
    Portanto a área da praça é $S=\pi R^2=\pi (10\sqrt{5+2\sqrt{3}})^2=100(5+2\sqrt{3})\pi$ m$^2$.
\end{enumerate}
}{0}
\end{answer}

\label{trigonometria-exercicios}
\begin{enumerate}

\item{}
(ENEM) Para decorar um cilindro circular reto será usada uma faixa retangular de papel transparente, na qual está desenhada em negrito uma diagonal que forma um ângulo de $30^\circ$ com a borda inferior.
\begin{figure}[H]
    \centering
    \includegraphics[scale=0.7]{Cilindro.JPG}
    \caption{Cilindro circular reto.}
    \label{Cilindro}
\end{figure}
O raio da base do cilindro mede $\frac{6}{\pi}$cm, e ao enrolar a faixa obtém-se uma linha em formato de hélice como na \Fref{Cilindro}. O valor da medida da altura do cilindro, em centímetros, é
\begin{enumerate}
    \item $36\sqrt{3}$.
    \item $24\sqrt{3}$.
    \item $4\sqrt{3}$.
    \item $36$.
    \item $72$.
\end{enumerate} 

\item{}
(ENEM) Para determinar a distância de um barco até a praia, um navegante utilizou o seguinte procedimento: a partir de um ponto A, mediu o ângulo visual $\alpha$ fazendo mira em um ponto fixo $P$ da praia. Mantendo o barco no mesmo sentido, ele seguiu até um ponto B de modo que fosse possível ver o mesmo ponto $P$ da praia, no entanto sob um ângulo visual $2\alpha$. A \Fref{Barco} ilustra essa situação:
\begin{figure}[H]
    \centering
    \includegraphics[scale=0.5]{Barcotri.JPG}
    \caption{Barco navegando.}
    \label{Barco}
\end{figure}
Suponha que o navegante tenha medido o ângulo $\alpha= 30^\circ$ e, ao chegar ao ponto $B$, verificou que o barco havia percorrido a distância $AB = 2 000$m. Com base nesses dados e mantendo a mesma trajetória, a menor distância do barco até o ponto fixo $P$ será
\begin{enumerate}
    \item $1000$m.
    \item $1000\sqrt{3}$m.
    \item $2000\frac{\sqrt{3}}{3}$m.
    \item $2000$m.
    \item $2000\sqrt{3}$m.
\end{enumerate}

\item {}
(Unesp) A \Fref{Mesa} representa a vista superior do tampo plano e horizontal de uma mesa de bilhar retangular $ABCD$ com caçapas  em $A, B, C$ e $D$. O ponto $P$  localizado em $AB$  representa a posição de uma bola de bilhar, sendo $PB=1,5$m   e $PA=1,2$m. Após uma tacada na bola, ela se desloca em linha reta colidindo com $BC$  no ponto $T$  sendo a medida do ângulo $P\hat{T}B$  igual a $60^\circ$. Após essa colisão, a bola segue, em trajetória reta, diretamente até a caçapa $D$. 
\begin{figure}[H]
    \centering
    \includegraphics[scale=0.75]{Mesa.JPG}
    \caption{Mesa de bilhar.}
    \label{Mesa}
\end{figure}
Nas condições descritas e adotando $\sqrt{3} \approx 1,73$   a largura do tampo da mesa, em metros, é próxima de: 
\begin{enumerate}
    \item $2,42$.
    \item $2,08$.
    \item $2,28$.
    \item $2,00$.
    \item $2,56$.
\end{enumerate} 

\item{}
(Unicamp) Para trocar uma lâmpada, Roberto encostou uma escada na parede de sua casa, de forma que o topo da escada ficou a uma altura de $4$m. Enquanto Roberto subia os degraus, a base da escada escorregou por $1$m, tocando o muro paralelo à parede, conforme a \Fref{Escada2}. Refeito do susto, Roberto reparou que, após deslizar, a escada passou a fazer um ângulo de $45^\circ$ com o piso horizontal. 
\begin{figure}[H]
    \centering
    \includegraphics[scale=1]{Escada2.JPG}
    \caption{Escada.}
    \label{Escada2}
\end{figure}
A distância entre a parede da casa e o muro equivale a
\begin{enumerate}
    \item $(4\sqrt{3}+1)$m.
    \item $(3\sqrt{2}-1)$m.
    \item $4\sqrt{3}$m.
    \item $(3\sqrt{2}-2)$m.
    \item $4$m.
\end{enumerate} 

\item{}
(Epcar) Uma coruja está pousada em $R$, ponto mais alto de um poste, a uma altura $h$ do ponto $P$, no chão.
Ela é vista por um rato no ponto $A$, no solo, sob um ângulo de $30^\circ$, conforme mostra \Fref{Coruja}.
\begin{figure}[H]
    \centering
    \includegraphics[scale=0.8]{Coruja.JPG}
    \caption{Coruja capturando um rato.}
    \label{Coruja}
\end{figure}
O rato se desloca em linha reta até o ponto $B$, de onde vê a coruja, agora sob um ângulo de $45^\circ$ com o chão e a uma distância $BR$  de medida $6\sqrt{2}$  metros.
Com base nessas informações, estando os pontos $A, B$ e $P$ alinhados e desprezando-se a espessura do poste, pode-se afirmar então que a medida do deslocamento $AB$   do rato, em metros, é um número entre
\begin{enumerate}
    \item $3$ e $4$.
    \item $4$ e $5$.
    \item $5$ e $6$.
    \item $6$ e $7$.
    \item $8$ e $9$.
\end{enumerate}

\item{}
(Cefet-RJ)
Quem viaja no bondinho do Pão de Açúcar, percorre dois trechos: o primeiro vai da Praia Vermelha até o morro da Urca (segmento $PU$ da figura), e o segundo parte do morro da Urca até o Pão de Açúcar. Sabendo que o segmento $PM$ e a altura do morro da Urca equivalem a $\frac{4}{3}$  e a $\frac{5}{9}$   da altura do Pão de Açúcar, respectivamente, podemos afirmar que o ângulo $\beta$ formado pelos segmentos $PU$ e $PM$ indicados na \Fref{Paodeacucar1}.

\begin{figure}[H]
    \centering
    \includegraphics[scale=0.7]{Paodeacucar.JPG}
    \caption{Pão de açúcar.}
    \label{Paodeacucar}
\end{figure}


\begin{table}[H]
\centering

\begin{tabular}{|c|c|c|c|}\hline
\tcolor{Ângulo}  & \tcolor{Seno} &  \tcolor{Cosseno} & \tcolor{Tangente} \\ \hline
$21^\circ$  & $0,358$  &  $0,934$   & $0,384$ \\ \hline
$22^\circ$  & $0,375$  &  $0,927$   & $0,404$ \\ \hline
$23^\circ$  & $0,391$  &  $0,921$   & $0,424$ \\ \hline
$24^\circ$  & $0,407$  &  $0,913$   & $0,445$ \\ \hline
\end{tabular}
\end{table}

\begin{figure}[H]
    \centering
    \includegraphics[scale=0.6]{Paodeacucar1.JPG}
    \caption{Esquematicamente o Pão de açúcar.}
    \label{Paodeacucar1}
\end{figure}

\begin{enumerate}
    \item está entre $21^\circ$ e $22^\circ$.
    \item está entre $22^\circ$ e $23^\circ$.
    \item está entre $23^\circ$ e $24^\circ$.
    \item é maior que $24^\circ$.
    \item igual a $29^\circ$.
\end{enumerate}

\item{} A \Fref{Escadacg} representa uma escada apoiada sobre o piso de uma sala em forma de $L$. Admitindo que esta sala conecta dois corredores perpendiculares cujas larguras são $6$m e $9$ e que a escada forma um ângulo $\theta=30^\circ$ com uma das paredes da sala, determine o comprimento total da escada.
\begin{figure}[H]
    \centering
    \includegraphics[scale=0.4]{Escadacg.JPG}
    \caption{Uma escada apoiada sobre o piso de uma sala em $L$.}
    \label{Escadacg}
\end{figure}

\item{}
(PUC-PR) Um topógrafo deseja medir a distância $x$ de um ponto $Q$ na margem de um rio até um ponto inacessível $P$  na outra margem, conforme a \Fref{Rio}. Sabendo-se que ele visualiza o ponto $P$ segundo um ângulo $\beta$ e, em seguida, ele se desloca uma distância $b$  até o ponto $R$  e observa o ponto $P$ segundo o ângulo $\theta$  a expressão que calcula a distância  é:
\begin{figure}[H]
    \centering
    \includegraphics[scale=0.8]{Rio.JPG}
    \caption{Medindo distância entre dois pontos em margens opostas.}
    \label{Rio}
\end{figure}
\begin{enumerate}
    \item $x=\frac{b\sen\theta}{\cos(\beta+\theta)}$.
    \item $x=\frac{b\cos\theta}{\cos(\beta+\theta)}$.
    \item $x=\frac{b\sen\theta}{\sen(\beta+\theta)}$.
    \item $x=\frac{b\tg\theta}{\tg(\beta+\theta)}$.
    \item $x=\frac{b\sen\beta}{\sen(\beta+\theta)}$.
\end{enumerate}

\clearpage
\item{}
(OBMEP - Adaptada) Dois amigos partem ao mesmo tempo do ponto P e se afastam em direções que formam um ângulo de $60^\circ$, conforme mostra a \Fref{Amigos}. Eles caminham em linha reta, com velocidades constantes de $6 km/h$ e $8 km/h$. 
\begin{figure}[H]
    \centering
    \includegraphics[scale=0.8]{Amigos.JPG}
    \caption{Dois amigos caminhando em linha reta a $6 km/h$.}
    \label{Amigos}
\end{figure}
Após uma hora qual a distância entre eles?


\item{}
Observe o triângulo $ABC$ da \Fref{Semelhanca}.
\begin{figure}[H]
    \centering
    \includegraphics[scale=0.5]{Semelhanca.JPG}
    \caption{Determinando o comprimento do lado $AC$.}
    \label{Semelhanca}
\end{figure}
\begin{enumerate}
    \item Como você poderia determinar o cosseno do ângulo do vértice $A$? Qual esse valor?
    \item Usando o resultado do item anterior, como você poderia derterminar o comprimento do lado $AC$? Que comprimento é esse?
    \item Na figura você consegue identificar algum par de triângulos semelantes? Quais?
    \item Seria possível utilizar esses triângulos semelhantes para determinar a medida do segmento $AC$? 
\end{enumerate}

\item{}
(UFPB) A prefeitura de certa cidade vai construir, sobre um rio que corta essa cidade, uma ponte que deve ser reta e ligar dois pontos, $A$ e $B$, localizados nas margens opostas do rio. Para medir a distância entre esses pontos, um topógrafo localizou um terceiro ponto, $C$, distante $200$m do ponto $A$ e na mesma margem do rio onde se encontra o ponto $A$. Usando um teodolito (instrumento de precisão para medir ângulos horizontais e ângulos verticais, muito empregado em trabalhos topográficos), o topógrafo observou que os ângulos $B\hat{C}A$ e $C\hat{A}B$ mediam, respectivamente, $30^\circ$ e $105^\circ$, conforme ilustrado na \Fref{Rio1}.
\begin{figure}[H]
    \centering
    \includegraphics[scale=0.5]{Rio1.JPG}
    \caption{Medindo a distância entre dois pontos $A$ e $B$.}
    \label{Rio1}
\end{figure}
Com base nessas informações, é correto afirmar que a distância, em metros, do ponto $A$ ao ponto
$B$ é de: 
\begin{enumerate}
    \item $200\sqrt{2}$.
    \item $180\sqrt{2}$.
    \item $150\sqrt{2}$.
    \item $100\sqrt{2}$.
    \item $50\sqrt{2}$.
\end{enumerate} 

\item{}
(UERJ) O raio de uma roda gigante de centro $C$ mede $CA = CB = 10$m. Do centro $C$ ao plano horizontal do chão, há uma distância de $11$m. Os pontos $A$ e $B$, situados no mesmo plano vertical, $ACB$, pertencem à circunferência dessa roda e distam, respectivamente, $16$m e $3,95$m do plano do chão. Observe o esquema e a \Tref{Roda}.
\begin{figure}[H]
    \centering
    \includegraphics[scale=0.7]{Roda.JPG}
    \caption{Medindo a distância entre dois pontos $A$ e $B$.}
    \label{Roda}
\end{figure}
A medida, em graus, mais próxima do menor ângulo $A\hat{C}B$ corresponde a: 
\begin{enumerate}
    \item $45$.
    \item $60$.
    \item $75$.
    \item $105$.
    \item $109$.
\end{enumerate} 

\item{}
(UEL) Considere o planeta Terra como uma esfera com raio de $6400 km$. Um satélite percorre uma órbita circular em torno da Terra e, num dado instante, a antena de um radar está direcionada para ele, com uma inclinação de $30^\circ$ sobre a linha do horizonte, conforme mostra a \Fref{Radar}. 
\begin{figure}[H]
    \centering
    \includegraphics[scale=0.7]{Radar.jpg}
    \caption{Satélite em órbita.}
    \label{Radar}
\end{figure}
Usando  $\sqrt{2} \approx 1,4$ e $\sqrt{3} \approx 1,7$, é correto concluir que a distância $x$, em quilômetros, da superfície da Terra ao satélite, está compreendida entre

\begin{enumerate}
    \item $1350$km e $1450$km.
    \item $1500$km e $1600$km.
    \item $1650$km e $1750$km.
    \item $1800$km e $1900$km.
    \item $1950$km e $2050$km.
\end{enumerate} 

\item{}
(ENEM) Uma desenhista projetista deverá desenhar uma tampa de panela em forma circular. Para realizar esse desenho, ela dispõe, no momento, de apenas um compasso, cujo comprimento das hastes é de   um transferidor e uma folha de papel com um plano cartesiano. Para esboçar o desenho dessa tampa, ela afastou as hastes do compasso de forma que o ângulo formado por elas fosse de $120^\circ$.   A ponta seca está representada pelo ponto $C$  a ponta do grafite está representada pelo ponto $B$  e a cabeça do compasso está representada pelo ponto $A$  conforme a \Fref{Compasso}.
\begin{figure}[H]
    \centering
    \includegraphics[scale=0.5]{Compasso.JPG}
    \caption{Compasso.}
    \label{Compasso}
\end{figure}
Após concluir o desenho, ela o encaminha para o setor de produção. Ao receber o desenho com a indicação do raio da tampa, verificará em qual intervalo este se encontra e decidirá o tipo de material a ser utilizado na sua fabricação, de acordo com os dados.

\begin{table}[H]
\centering

    \begin{tabular}{|c|c|} \hline
    \tcolor{Tipo do material} & \tcolor{Intervalo de valores do raio (cm)} \\ \hline
    \textit{I}   &   $0 < R \leq 5$ \\ \hline
    \textit{II}   &   $5 < R \leq 10$ \\ \hline
    \textit{III}   &   $10 < R \leq 15$ \\ \hline
    \textit{IV}   &   $15 < R \leq 21$ \\ \hline
    \textit{V}   &   $21 < R \leq 40$ \\ \hline
    \end{tabular}
\end{table}
Considere $1,7$ como aproximação para  $\sqrt{3}$. O tipo de material a ser utilizado pelo setor de produção será 
\begin{enumerate}
    \item \textit{I}.
    \item \textit{II}.
    \item \textit{III}.
    \item \textit{IV}.
    \item \textit{V}.
\end{enumerate} 

\item{}
Na \Fref{AdicaoSubtracao}, vemos à esquerda um triângulo $ABC$ com altura $AD$ de comprimento $1$ e à direita, temos um triângulo retângulo $EFG$ em $F$ cujo cateto $EF$ tem medida $1$.
\begin{figure}[H]
    \centering
    \includegraphics[scale=0.4]{AdicaoSubtracao.JPG}
    \caption{Adição e subtração de arcos.}
    \label{AdicaoSubtracao}
\end{figure}
\begin{enumerate}
     \item No triângulo $ACD$, mostre que: 
     $$AC=\frac{1}{\cos\alpha} \ \ \text{e} \ \  CD=\tg\alpha.$$
     \item No triângulo $ABD$, mostre que que: 
     $$AB=\frac{1}{\cos\beta} \ \ \text{e} \ \  BD=\tg\beta.$$
    \item Calculando a área do triângulo $ABC$ em função dos lados $AB, AC, \alpha$ e $\beta$ e, em seguida, em função dos segmentos $AD$ e $BC$, conclua que:
    $$\sen(\alpha+\beta)=\sen\alpha\cos\beta+\sen\beta\cos\alpha.$$
    \item Usando o item anterior, conclua que: 
    $$\sen(2\alpha)=2\sen\alpha\cos\alpha.$$
    \item Usando o triângulo $EFG$ conclua que: 
    $$\sen(\alpha-\beta)=\sen\alpha\cos\beta-\sen\beta\cos\alpha.$$
\end{enumerate}

\item{}
A \Fref{Retangulos} mostra dois retângulos congruentes. Cada retângulo possui um lado igual ao triplo do outro.
\begin{figure}[H]
    \centering
    \includegraphics[scale=0.3]{Retangulos.JPG}
    \caption{Retângulos de lados $a$ e $3a$.}
    \label{Retangulos}
\end{figure}
\begin{enumerate}
    \item Calcule o cosseno do ângulo $B\hat{A}C$.
    \item Encontre uma aproximação com uma casa decimal para a medida desse ângulo.
\end{enumerate}

\item{}
(Lei do paralelogramo) Um paralelogramo é um quadrilátero que possui lados opostos paralelos. No paralelogramo $ABCD$, tem-se que $AB=CD=a,BC=DA=b$, com  diagonais $AC=d_1$ e $BD=d_2$, conforme ilustra a \Fref{Parelelogramo}. 
\begin{figure}[H]
    \centering
    \includegraphics[scale=0.6]{Paralelogramo.JPG}
    \caption{Paralelogramo $ABCD$.}
    \label{Parelelogramo}
\end{figure}
Mostre que as medidas $a,b, d_1$ e $d_2$ relacionam-se pela expressão
$$d_1^2+d_2^2=2(a^2+b^2),$$
que é conhecida como lei do paralelogramo.

\item{}
Depois de  uma pequena discussão, Paulinho, Carlos e Ary seguiram cada um o seu caminho, em direções de $120^\circ$ uma com a outra. Suas velocidades estavam na razão $1:2:4$. Mostre que, em qualquer instante, suas posições são os vértices de um triângulo retângulo.

\item{}
Na \Fref{Aurea}, o triângulo $ABC$ é isósceles de base $BC=1$. 
\begin{figure}[H]
    \centering
    \includegraphics[scale=0.4]{Aurea.JPG}
    \caption{Triângulo isósceles $ABC$ de base $BC$.}
    \label{Aurea}
\end{figure}
\begin{enumerate}
    \item Se $B\hat{A}C=36^\circ$, mostre que $AB=AC=\varphi=\frac{\sqrt{5}+1}{2}$ (esse é o chamado número de ouro).
    \item Conclua que $\cos36^\circ=\frac{1}{2}\varphi$.
\end{enumerate}

\item{}
Numa praça circular de centro $O$ é feita uma calçada em forma de um polígono convexo que possui seis lados medindo $10$m intercalados por outros seis lados medindo $20$m, conforme ilustra a \Fref{Praca}.
\begin{figure}[H]
    \centering
    \includegraphics[scale=0.3]{Praca.JPG}
    \caption{Praça em formato circular de centro em $O$.}
    \label{Praca}
\end{figure}
Para realizar um projeto de modernização da praça um engenheiro da prefeitura local precisa saber a área da praça circular para poder avaliar os custos da obra. Com as informações disponíveis, como o engenheiro pode obter a medida do área dessa praça?
\end{enumerate}





\ifnum\aluno=1
\clearpage
\else
\notasfinais
\fi

\bibliographystyle{apalike-pt}
\bibliography{../Bibliografia/trigonometria_bibliografia.bib}

\nocite{*}




% \begin{sphinxthebibliography}{Saarinem-apud-Torres-2004}
% \bibitem[Poynter-et-al-2005]{\detokenize{Poynter-et-al-2005}}{\phantomsection\label{\detokenize{GE101-1A:poynter-et-al-2005}} 
% Poynter, A., Tall, D.: What do mathematics and physics teachers think that students will find difficult? A challenge to accepted practices of teaching. In Proceedings of the sixth British Congress of Mathematics Education, University of Warwick (pp. 128-135).
% }
% \bibitem[Anton-et-al-2007]{\detokenize{Anton-et-al-2007}}{\phantomsection\label{\detokenize{GE101-E:anton-et-al-2007}} 
% Anton, H.; Busby, R. C.: \sphinxstyleemphasis{Álgebra Linear Contemporânea}. Bookman, 2007.
% }
% \bibitem[Barniol-et-al-2014]{\detokenize{Barniol-et-al-2014}}{\phantomsection\label{\detokenize{GE101-E:barniol-et-al-2014}} 
% Barniol, P.; Zavala, G.: Test of Understanding of Vectors: A Reliable Multiple-Choice Vector Concept Test. \sphinxstyleemphasis{Physical Review Special Topics - Physics Education Research}, v. 10, 010121(14), 2014.
% }
% \bibitem[Bello-2013]{\detokenize{Bello-2013}}{\phantomsection\label{\detokenize{GE101-E:bello-2013}} 
% Bello, A. L.: \sphinxstyleemphasis{Origins of Mathematical Words: A Comprehensive Dictionary of Latin, Greek, and Arabic Roots}. The John Hopkins University Press, 2013.
% }
% \bibitem[Gardner-1973]{\detokenize{Gardner-1973}}{\phantomsection\label{\detokenize{GE101-E:gardner-1973}} 
% Gardner, M.: Mathematical Games - Sim, Chomp and Race Track: New Games for The Intellect (and not for Lady Luck). \sphinxstyleemphasis{Scientific American}, v. 228, n. 1, p. 108\textendash{}115, 1973.
% }
% \bibitem[Gomes-2010]{\detokenize{Gomes-2010}}{\phantomsection\label{\detokenize{GE101-E:gomes-2010}} 
% Gomes, A. M. D.: \sphinxhref{http://www.uff.br/sintoniamatematica/matematicaenatureza/matematicaenatureza-html/audio-formigas-br.html}{Formigas do Deserto e Integração por Caminhos}. Conteúdo Digitais em Matemática e Estatística, Universidade Federal Fluminense, 2010.
% }
% \bibitem[Horn-1998]{\detokenize{Horn-1998}}{\phantomsection\label{\detokenize{GE101-E:horn-1998}} 
% Horn, R. E.: \sphinxstyleemphasis{Visual Language: Global Communication for The 21st Century}. MacroVU, Inc., Bainbridge Island, Washington, USA, 1998.
% }
% \bibitem[Oliveira-2009]{\detokenize{Oliveira-2009}}{\phantomsection\label{\detokenize{GE101-E:oliveira-2009}} 
% Oliveira, P. M. C.: Corrida de Vetores: Vacina Contra O Raciocínio Aristotélico. \sphinxstyleemphasis{Física na Escola}, v. 10, n. 1, p. 40, 2009.
% }
% \bibitem[Pinheiro-et-al-2016]{\detokenize{Pinheiro-et-al-2016}}{\phantomsection\label{\detokenize{GE101-E:pinheiro-et-al-2016}} 
% Pinheiro, W. M.; Farias, A. C. S.: Composição Específica, Bioecologia e Ecomorfologia da Ictiofauna Marinha Oriunda da Pesca de Pequena Escala. \sphinxstyleemphasis{Boletim do Instituto de Pesca}, São Paulo, v. 42, n. 1, p. 181-194, mar. 2016.
% }
% \bibitem[Poynter-et-al-2005]{\detokenize{Poynter-et-al-2005}}{\phantomsection\label{\detokenize{GE101-E:poynter-et-al-2005}} 
% Poynter, A.; Tall, D. \sphinxstyleemphasis{Relating Theories To Practice in The teaching of Mathematics}.  European Research in Mathematics Education IV. Working Group 11: Different Theoretical Perspectives and Approaches in Research in Mathematics Education, p. 1264-1273, 2005.
% }
% \bibitem[Roche-1997]{\detokenize{Roche-1997}}{\phantomsection\label{\detokenize{GE101-E:roche-1997}} 
% Roche, J.: Introducing Vectors. \sphinxstyleemphasis{Physics Education}, v. 32, p. 339-345, 1997.
% }
% \bibitem[Sesamath-MATHS-2e-2014]{\detokenize{Sesamath-MATHS-2e-2014}}{\phantomsection\label{\detokenize{GE101-E:sesamath-maths-2e-2014}} 
% Sésamath: MATHS-2de. 2014. Disponível em: \textless{}\sphinxurl{https://manuel.sesamath.net/index.php?page=telechargement\_2nde\_2014}\textgreater{}.
% }
% \bibitem[Valentin-2000]{\detokenize{Valentin-2000}}{\phantomsection\label{\detokenize{GE101-E:valentin-2000}} 
% Valentin, J. L.: \sphinxstyleemphasis{Ecologia Numérica: Uma Introdução À Análise Multivariada de Dados Ecológicos}. Editora Interciência, 2000.
% }
% \bibitem[Wehner-et-al-1981]{\detokenize{Wehner-et-al-1981}}{\phantomsection\label{\detokenize{GE101-E:wehner-et-al-1981}} 
% Wehner, R.; Srinivasan, M. V.: Searching Behaviour of Desert Ants, Genus Cataglyphis (Formicidae, Hymenoptera). \sphinxstyleemphasis{Journal of Comparative Physiology}, v. 142, p. 315\textendash{}338, 1981.
% }
% \bibitem[Wong-2011]{\detokenize{Wong-2011}}{\phantomsection\label{\detokenize{GE101-E:wong-2011}} 
% Wong, B.: Arrows. \sphinxstyleemphasis{Nature Methods}, v. 8, n. 9, p. 701, 2011.
% }
% \bibitem[Cohn-2012]{\detokenize{Cohn-2012}}{\phantomsection\label{\detokenize{GE301-E:cohn-2012}} 
% Cohn, N.: \sphinxstyleemphasis{Explaining ‘I Can’t Draw’: Parallels between The Structure and Development of Language and Drawing}. Human Development, v. 55, p. 167-192, 2012.
% }
% \bibitem[Cox-et-al-1998]{\detokenize{Cox-et-al-1998}}{\phantomsection\label{\detokenize{GE301-E:cox-et-al-1998}} 
% Cox, M. V.; Perara, J.: \sphinxstyleemphasis{Children’s Observational Drawings: A Nine-Point Scale for Scoring Drawings of A Cube}. Educational Psychology: An International Journal of Experimental Educational Psychology, v. 18, n. 3, p. 309-317, 1998.
% }
% \bibitem[Duval-2011]{\detokenize{Duval-2011}}{\phantomsection\label{\detokenize{GE301-E:duval-2011}} 
% Duval, R.: \sphinxstyleemphasis{Ler e Ensinar A Matemática de Outra Forma  - Entrar no Modo Matemático de Pensar: Os Registros de Representações Semióticas}. Editora Livraria da Física, 2011.
% }
% \bibitem[Ebersbach-et-al-2010]{\detokenize{Ebersbach-et-al-2010}}{\phantomsection\label{\detokenize{GE301-E:ebersbach-et-al-2010}} 
% Ebersbach, M.; Stiehler, S.; Asmus, P.: \sphinxstyleemphasis{On The Relationship between Children’s Perspective Taking in Complex Scenes and Their Spatial Drawing Ability}. British Journal of Developmental Psychology, v. 29, p. 455-474, 2010.
% }
% \bibitem[Ebersbach-et-al-2011]{\detokenize{Ebersbach-et-al-2011}}{\phantomsection\label{\detokenize{GE301-E:ebersbach-et-al-2011}} 
% Ebersbach, M.; Hagedom, H.: \sphinxstyleemphasis{The Role of Cognitive Flexibility in The Spatial Representation of Children’s Drawings}. Journal of Cognition and Development, v. 12, n. 1, p. 32-55, 2011.
% }
% \bibitem[Edwards-2005]{\detokenize{Edwards-2005}}{\phantomsection\label{\detokenize{GE301-E:edwards-2005}} 
% Edwards, B.: \sphinxstyleemphasis{Desenhando Com o Lado Direito do Cérebro}. Oitava edição, Ediouro, 2005.
% }
% \bibitem[Fan-2015]{\detokenize{Fan-2015}}{\phantomsection\label{\detokenize{GE301-E:fan-2015}} 
% Fan, J. E.: \sphinxstyleemphasis{Drawing to Learn: How Producing Graphical Representations Enhances Scientific Thinking}. Translational Issues in Psychological Science, American Psychological Association, v. 1, n. 2, p. 170\textendash{}181, 2015.
% }
% \bibitem[Gardner-2011]{\detokenize{Gardner-2011}}{\phantomsection\label{\detokenize{GE301-E:gardner-2011}} 
% Gardner, H.: \sphinxstyleemphasis{Frames of Mind: The Theory of Multiple Intelligences}. Basic Books, 2011.
% }
% \bibitem[Gray-et-al-2004]{\detokenize{Gray-et-al-2004}}{\phantomsection\label{\detokenize{GE301-E:gray-et-al-2004}} 
% Gray, J. R.; Thompson P. M.: \sphinxstyleemphasis{Neurobiology of Intelligence: Science and Ethics}. Nature Reviews Neuroscience, v. 5/6, p. 471-482, 2004.
% }
% \bibitem[Grunbaum-1985]{\detokenize{Grunbaum-1985}}{\phantomsection\label{\detokenize{GE301-E:grunbaum-1985}} 
% Grünbaum, B.: \sphinxstyleemphasis{Geometry Strikes Again}. Mathematics Magazine, v. 58, n. 1, p. 12-17, 1985.
% }
% \bibitem[Gutierrez-1998]{\detokenize{Gutierrez-1998}}{\phantomsection\label{\detokenize{GE301-E:gutierrez-1998}} 
% Gutiérrez, A.: \sphinxstyleemphasis{Las Representaciones Planas de Cuerpos 3-Dimensionales En La Enseñanza de La Geometría Espacial}. Revista EMA, v. 3, n. 3, p. 193-220, 1998.
% }
% \bibitem[Howard-et-al-1995]{\detokenize{Howard-et-al-1995}}{\phantomsection\label{\detokenize{GE301-E:howard-et-al-1995}} 
% Howard, I. P.; Rogers, B. J.: \sphinxstyleemphasis{Binocular Vision and Stereopsis}. Oxford University Press, 1995.
% }
% \bibitem[Fujita-et-al-2017]{\detokenize{Fujita-et-al-2017}}{\phantomsection\label{\detokenize{GE301-E:fujita-et-al-2017}} 
% Fujita, T. et al. (2017). Students’ Geometric Thinking with Cube Representations: Assessment Framework and Empirical Evidence. The Journal of Mathematical Behavior, v. 46, p. 96-111.
% }
% \bibitem[Khine-2017]{\detokenize{Khine-2017}}{\phantomsection\label{\detokenize{GE301-E:khine-2017}} 
% Khine, M. S.: \sphinxstyleemphasis{Visual-Spatial Ability in STEM Educaton: Transforming Research into Practice}. Springer-Verlag, 2017.
% }
% \bibitem[Lellis-2009]{\detokenize{Lellis-2009}}{\phantomsection\label{\detokenize{GE301-E:lellis-2009}} 
% Lellis, M. (2009). Desenho em Perspectiva no Ensino Fundamental \textendash{} Considerações Sobre Uma Experiência. Seminários de Ensino de Matemática (SEMA), Primeiro Semestre de 2009 (Ano II), Programa de Pós-Graduação da Faculdade de Educação da Universidade de S˜ao Paulo. Disponível em: \textless{}\sphinxurl{https://goo.gl/77Unkk}\textgreater{}.
% }
% \bibitem[Lindberg-1976]{\detokenize{Lindberg-1976}}{\phantomsection\label{\detokenize{GE301-E:lindberg-1976}} 
% Lindberg, D. C.: \sphinxstyleemphasis{Theories of Vision from Al-Kindi To Kepler}. The University of Chicago Press, 1976.
% }
% \bibitem[Mitchelmore-1978]{\detokenize{Mitchelmore-1978}}{\phantomsection\label{\detokenize{GE301-E:mitchelmore-1978}} 
% Mitchelmore, M. C.: \sphinxstyleemphasis{Developmental Stages in Children’s Representation of Regular Solid Figures}. The Journal of Genetic Psychology, v. 133, n. 2, p. 229-239, 1978.
% }
% \bibitem[Morra-2008]{\detokenize{Morra-2008}}{\phantomsection\label{\detokenize{GE301-E:morra-2008}} 
% Morra, S.: \sphinxstyleemphasis{Spatial Structures in Children’s Drawings: How Do They Develop?} Em: Lange-Küttner, C.; \& Vinter, A. (Eds.), Drawing and The Non-Verbal Mind: A Life-Span Perspective,  Cambridge: Cambridge University Press, 2008.
% }
% \bibitem[Newcombe-2017]{\detokenize{Newcombe-2017}}{\phantomsection\label{\detokenize{GE301-E:newcombe-2017}} 
% Newcombe, N. (2017). Harnessing Spatial Thinking to Support STEM Learning.  OECD Education Working Papers, n. 161, OECD Publishing, Paris. Disponível em: \textless{}\sphinxurl{https://goo.gl/kyiJ4z}\textgreater{}.
% }
% \bibitem[NRC-2006]{\detokenize{NRC-2006}}{\phantomsection\label{\detokenize{GE301-E:nrc-2006}} 
% National Research Council. (2006). Learning To Think Spatially. The National Academies Press, Washington, D.C..
% }
% \bibitem[Pillar-2012]{\detokenize{Pillar-2012}}{\phantomsection\label{\detokenize{GE301-E:pillar-2012}} 
% Pilar, A. D.: \sphinxstyleemphasis{Desenho e Escrita como Sistemas de Representação}. Segunda edição revista e ampliada. Editora Penso, 2012.
% }
% \bibitem[Pinilla-2007]{\detokenize{Pinilla-2007}}{\phantomsection\label{\detokenize{GE301-E:pinilla-2007}} 
% Pinilla   M. I. F.: \sphinxstyleemphasis{Fractions: Conceptual and   Didactic Aspects}. Acta  Didactica Universitatis Comenianae, v. 7, p. 23-45, 2007.
% }
% \bibitem[Quillin-et-al-2015]{\detokenize{Quillin-et-al-2015}}{\phantomsection\label{\detokenize{GE301-E:quillin-et-al-2015}} 
% Quillin, K.; Thomas, S.: \sphinxstyleemphasis{Drawing-to-Learn: A Framework for Using Drawings to Promote Model-Based Reasoning in Biology}. CBE\textendash{}Life Sciences Education, v. 14, p. 1-14, 2015.
% }
% \bibitem[Santaella-1998]{\detokenize{Santaella-1998}}{\phantomsection\label{\detokenize{GE301-E:santaella-1998}} 
% Santaella, L.: \sphinxstyleemphasis{O Que É Semiótica}. Coleção Primeiros Passos, v. 103, Editora Brasiliense, 1998.
% }
% \bibitem[Sinha-2009]{\detokenize{Sinha-2009}}{\phantomsection\label{\detokenize{GE301-E:sinha-2009}} 
% Sinha, P. Pawan Sinha em Como O Cérebro Aprender A Ver. Palestra TED, 2009. Disponível em: \textless{}\sphinxurl{https://goo.gl/eDZKYo}\textgreater{}.
% }
% \bibitem[Sinclair-et-al-2016]{\detokenize{Sinclair-et-al-2016}}{\phantomsection\label{\detokenize{GE301-E:sinclair-et-al-2016}} 
% Sinclair, N. et al.: \sphinxstyleemphasis{Recent Research On Geometry Education: An ICME‑13 Survey Team Report}. ZDM Mathematics Education, v. 48, p. 691-719, 2016.
% }
% \bibitem[Uttal-et-al-2012]{\detokenize{Uttal-et-al-2012}}{\phantomsection\label{\detokenize{GE301-E:uttal-et-al-2012}} 
% Uttal, D. H.; Cohen, C. A. (2012). Spatial Thinking and STEM Education: When, Why, and How? Em: Ross, B. H. The Psychology of Learning and Motivation, v. 57, Elsevier.
% }
% \bibitem[Van-Meter-et-al-2005]{\detokenize{Van-Meter-et-al-2005}}{\phantomsection\label{\detokenize{GE301-E:van-meter-et-al-2005}} 
% Van Meter, P.; Garner, J.: \sphinxstyleemphasis{The Promise and Practice of Learner-Generated Drawing: Literature Review and Synthesis}. Educational Psychology Review, v. 17, n. 4, p. 285-325, 2005.
% }
% \bibitem[Volkert-2008]{\detokenize{Volkert-2008}}{\phantomsection\label{\detokenize{GE301-E:volkert-2008}} 
% Volkert, K. (2008). The Problem of Solid Geometry. Symposium on the Occasion of the 100th Anniversary of ICMI, Rome. Disponível em: \textless{}\sphinxurl{https://goo.gl/Kt5g5C}\textgreater{}.
% }
% \bibitem[Willats-1977]{\detokenize{Willats-1977}}{\phantomsection\label{\detokenize{GE301-E:willats-1977}} 
% Willats. J.: \sphinxstyleemphasis{How Children Learn To Draw Realistic Pictures}. Quaterly Journal of Experimental Psychology, v. 29, p. 367-382, 1977.
% }
% \bibitem[Eisenberg-1992]{\detokenize{Eisenberg-1992}}{\phantomsection\label{\detokenize{AF106-0:eisenberg-1992}} 
% Eisenberg T.: On the development of a sense for functions. In Guershon Harel and Ed Dubinsky, editors, The Concept of Function: Aspects of Epistemology and Pedagogy. Mathematical Association of America, USA, p. 153\textendash{}174.
% }
% \bibitem[Jones-2006]{\detokenize{Jones-2006}}{\phantomsection\label{\detokenize{AF106-0:jones-2006}} 
% Jones M.: Desmystifying Functions: The Historical and Pedagogical Difficulties of the Concept of Function. Rose-Hulman Undergraduate Math Journal, V. 7, p. 1-20.
% }
% \bibitem[Ponte-et-al-2008]{\detokenize{Ponte-et-al-2008}}{\phantomsection\label{\detokenize{AF106-0:ponte-et-al-2008}} 
% PONTE J. P. \& Matos A.: O estudo de relações funcionais e o desenvolvimento do conceito de variável em alunos do 8.º ano. RELIME, V. 11(2), p. 195-231.
% }
% \bibitem[Ponte-1992]{\detokenize{Ponte-1992}}{\phantomsection\label{\detokenize{AF106-0:ponte-1992}} 
% Ponte J. P.: The History of the concept of function and some educational implications The Mathematics Educator, v. 2, n. 3, p. 3-8.
% }
% \bibitem[Sierpinska-1992]{\detokenize{Sierpinska-1992}}{\phantomsection\label{\detokenize{AF106-0:sierpinska-1992}} 
% Sierpinska A.: On understanding the notion of function. In Guershon Harel and Ed Du- binsky, editors, The Concept of Function: Aspects of Epistemology and Pedagogy. Mathematical Association of America, USA, p. 25-58.
% }
% \bibitem[Ursini-et-al-2001]{\detokenize{Ursini-et-al-2001}}{\phantomsection\label{\detokenize{AF106-0:ursini-et-al-2001}} 
% Ursini S. \& Trigueros M.: A model for the uses of variable in elementary algebra. In M. van den Heuvel-Panhuizen (Ed.), Proceedings of the 25th Conference of the International Group for the Psychology of Mathematics Education, V. 4, p. 327-334. Utrecht: Utrecht University.
% }
% \bibitem[Dooren-et-al-2005]{\detokenize{Dooren-et-al-2005}}{\phantomsection\label{\detokenize{AF107-0:dooren-et-al-2005}} 
% Dooren W., De Bock, D., Hessels, A., Janssens, D. \& Verschaffel, L.: Not everything is proportional: Effects of age and problem type on propensities for overgeneralization. Cognition and Instruction, V. 23, p. 57 \textendash{}86.
% }
% \bibitem[Lima-2006]{\detokenize{Lima-2006}}{\phantomsection\label{\detokenize{AF107-0:lima-2006}} 
% Lima, E. A matemática do ensino médio \textendash{} Vol. 1,  9a ed. Coleção do Professor de Matemática. Sociedade Brasileira de Matemática, Rio de Janeiro.
% }
% \bibitem[Silva-et-al-2013]{\detokenize{Silva-et-al-2013}}{\phantomsection\label{\detokenize{AF107-0:silva-et-al-2013}} 
% Silva, A. da F. G; Pietropaolo, R. C.; Campos, T. M. M. Atual currículo de matemática do estado de São Paulo: indicações para a introdução do ensino da ideia de irracionalidade. Boletim GEPEM, Rio de Janeiro, n. 62, p. 31-44.
% }
% \bibitem[Stump-1999]{\detokenize{Stump-1999}}{\phantomsection\label{\detokenize{AF107-0:stump-1999}} 
% Stump, S. Secondary mathematics teachers’ knowledge of slope. Mathematics Education Research Journal, 11(2), 124\textendash{}144.
% }
% \bibitem[Alexandre-et-al-2009]{\detokenize{Alexandre-et-al-2009}}{\phantomsection\label{\detokenize{AF209-0:alexandre-et-al-2009}} 
% Alexandre, Patrícia P; Santos,Márcia Hellen S Mendes. \sphinxstylestrong{Principais dificuldades de alunos do 2º ano do ensino médio quanto a interpretação gráfica da função quadrática}. I Simpósio Nacional de Ensino de Ciência e Tecnologia.
% }
% \bibitem[Assis-2015]{\detokenize{Assis-2015}}{\phantomsection\label{\detokenize{AF209-0:assis-2015}} 
% Assis, Victor Hugo D. de. \sphinxstylestrong{Características da função quadrática e a metodologia da resolução de problemas} UNESP: São José do Rio Preto, SP.
% }
% \bibitem[Avila]{\detokenize{Avila}}{\phantomsection\label{\detokenize{AF209-0:avila}} 
% Ávila, G. \sphinxstylestrong{Funções e gráficos num problema de frenagem}. Revista do professor de matemática, número \(12\). Disponível em \sphinxhref{http://rpm.org.br/cdrpm/12/5.htm}{RPM12} .
% }
% \bibitem[Batschelet-1978]{\detokenize{Batschelet-1978}}{\phantomsection\label{\detokenize{AF209-0:batschelet-1978}} 
% Batschelet, Edward. \sphinxstylestrong{Introdução à Matemática para Biocientistas} /E.Batschelet; tradução de Vera Maria Abud Pacífico da Silva e Junia Maria Penteado de Araújo Quitete - Rio de Janeiro: Interciência.
% }
% \bibitem[Cance-2015]{\detokenize{Cance-2015}}{\phantomsection\label{\detokenize{AF209-0:cance-2015}} 
% Cance, Cesar A. \sphinxstylestrong{Projeto canhão: o ensino de funções quadráticas com o auxílio do software Geogebra}. UFSCar, PROFMAT: São Carlos, SP.
% }
% \bibitem[Cerqueira-2015]{\detokenize{Cerqueira-2015}}{\phantomsection\label{\detokenize{AF209-0:cerqueira-2015}} 
% Cerqueira, Adriano A. \sphinxstylestrong{Parábola e suas aplicações} UFBA, PROFMAT: Salvador.
% }
% \bibitem[Chung-2013]{\detokenize{Chung-2013}}{\phantomsection\label{\detokenize{AF209-0:chung-2013}} 
% Chung, Kenji. \sphinxstylestrong{A Parábola, sua propriedade refletora e aplicações} UFRPE: Recife, PE.
% }
% \bibitem[Connally-et-al-2011]{\detokenize{Connally-et-al-2011}}{\phantomsection\label{\detokenize{AF209-0:connally-et-al-2011}} 
% Connally, Hughes-Hallett, Gleason, et. al. \sphinxstylestrong{Functions Modeling Change: A Preparation for Calculus}, \(4^{th}\) edition. EUA, National Science Foundation under Grant.
% }
% \bibitem[Cooney-et-al-2011]{\detokenize{Cooney-et-al-2011}}{\phantomsection\label{\detokenize{AF209-0:cooney-et-al-2011}} 
% Cooney, Thomas J., {[}et. al.{]}  \sphinxstylestrong{Developing an essential understanding of functions for teaching mathematics} in grades \(9-12\), \(2^{th}\) edition. EUA: The National Council of Teachers of Mathematics. (Essential understanding series)
% }
% \bibitem[DetranPR]{\detokenize{DetranPR}}{\phantomsection\label{\detokenize{AF209-0:detranpr}} 
% \sphinxstylestrong{Comportamentos seguros no trânsito}, Departamento de trânsito do Paraná. Disponível em \sphinxhref{http://www.detran.pr.gov.br/modules/catasg/servicos-detalhes.php?tema=motorista\&id=345}{Motorista} .
% }
% \bibitem[Duarte-2014]{\detokenize{Duarte-2014}}{\phantomsection\label{\detokenize{AF209-0:duarte-2014}} 
% Duarte, José L. \sphinxstylestrong{Problemas de máximos e mínimos no ensino médio}. UNESP: Ilha Solteira, SP.
% }
% \bibitem[Eves-2004]{\detokenize{Eves-2004}}{\phantomsection\label{\detokenize{AF209-0:eves-2004}} 
% Eves, Howard. \sphinxstylestrong{Introdução à história da matemática} tradução: Hygino H. Domingues. - Campinas, SP: Editora da UNICAMP. Tradução de: \sphinxstylestrong{An introduction to the history of mathematics}.
% }
% \bibitem[Figueredo-2017]{\detokenize{Figueredo-2017}}{\phantomsection\label{\detokenize{AF209-0:figueredo-2017}} 
% Figueredo, Eudes J.G. de. \sphinxstylestrong{Uma análise sobre a valorização do conceito de máximo e mínimo por estudantes do ensino médio} UFRPE, PROFMAT: Recife. 91 f.: il.
% }
% \bibitem[Hellmeister-2013]{\detokenize{Hellmeister-2013}}{\phantomsection\label{\detokenize{AF209-0:hellmeister-2013}} 
% Hellmeister, Ana Catarina P., coordenadora. \sphinxstylestrong{Geometria em Sala de Aula}. Rio de Janeiro: SBM. \(552\) p. (Coleção do Professor de Matemática; \(32\))
% }
% \bibitem[JCNET-2013]{\detokenize{JCNET-2013}}{\phantomsection\label{\detokenize{AF209-0:jcnet-2013}} 
% Jornal da Cidade - Bauru e garnde região, matérial online de \(05\) de fevereiro. \sphinxhref{https://www.jcnet.com.br/Geral/2013/02/direcao-defensiva-saiba-como-a-velocidade-influi-na-frenagem-do-veiculo.html\#prettyPhoto}{Distância de Frenagem} .
% }
% \bibitem[Junior-2017]{\detokenize{Junior-2017}}{\phantomsection\label{\detokenize{AF209-0:junior-2017}} 
% Junior, Gilberto C. da S. \sphinxstylestrong{Método dos mínimos quadrados aplicado ao lançamento de foguetes propulsionados a ar comprimido} Universidade Estadual Paulista “Júlio de Mesquita Filho”: Rio Claro: {[}s.n.{]}. 107 f.: fig., tab
% }
% \bibitem[Kotsopoulos-2007]{\detokenize{Kotsopoulos-2007}}{\phantomsection\label{\detokenize{AF209-0:kotsopoulos-2007}} 
% Kotsopoulos, D. \sphinxstylestrong{Unraveling student challenges with quadratics: A cognitive approach}. Australian Mathematics Teacher, \(63(2)\), \(19-24\).
% }
% \bibitem[Lima-2006]{\detokenize{Lima-2006}}{\phantomsection\label{\detokenize{AF209-0:lima-2006}} 
% Lima, E. \sphinxstylestrong{A matemática do ensino médio \textendash{} Vol. 1},  \(9^a\) ed. Coleção do Professor de Matemática. Rio de Janeiro, Sociedade Brasileira de Matemática.
% }
% \bibitem[Louzada-2013]{\detokenize{Louzada-2013}}{\phantomsection\label{\detokenize{AF209-0:louzada-2013}} 
% Louzada, Sílvia. \sphinxstylestrong{Relações entre Cônicas e Funções no Ensino Médio}. Espírito Santo, UFES, PROFMAT.
% }
% \bibitem[Maia-2007]{\detokenize{Maia-2007}}{\phantomsection\label{\detokenize{AF209-0:maia-2007}} 
% Maia, Diana. \sphinxstylestrong{Principais dificuldades de alunos do 2º ano do ensino médio quanto a interpretação gráfica da função quadrática}. São Paulo, PUC, Dissertação de Mestrado.
% }
% \bibitem[Monteiro-2014]{\detokenize{Monteiro-2014}}{\phantomsection\label{\detokenize{AF209-0:monteiro-2014}} 
% MONTEIRO, R. M. \sphinxstylestrong{Resgate do teorema de Dandelin no estudo de cônicas com o geogebra}. Espírito Santo: UFES, PROFMAT. Dinponível em \sphinxhref{http://portais4.ufes.br/posgrad/teses/tese\_7674\_Disserta\%E7\%E3o\%20-\%20final\%20-\%20Rubens\%20Monteiro.pdf}{Monteiro}.
% }
% \bibitem[Parent-2015]{\detokenize{Parent-2015}}{\phantomsection\label{\detokenize{AF209-0:parent-2015}} 
% Parent, Jennifer Suzanne Stokes. \sphinxstylestrong{“Students’ Understanding Of Quadratic Functions: Learning From Students’ Voices”}.Vermont University - Graduate College Dissertations and Theses. Paper 376.
% }
% \bibitem[Pietrocola-et-al-2016]{\detokenize{Pietrocola-et-al-2016}}{\phantomsection\label{\detokenize{AF209-0:pietrocola-et-al-2016}} 
% Pietrocola, M. {[}et. al.{]}. \sphinxstylestrong{Física em contexto, 1: ensino médio}, \(1^a\) ed. - São Paulo: Editora do Brasil.
% }
% \bibitem[Rocha-2013]{\detokenize{Rocha-2013}}{\phantomsection\label{\detokenize{AF209-0:rocha-2013}} 
% Rocha, Alan M. \sphinxstylestrong{Problemas de Otimização Envolvendo a Matemática do Ensino Médio}. Goiás: UFG, PROFMAT. Disponível em \sphinxhref{https://repositorio.bc.ufg.br/tede/handle/tde/2955}{ROCHA}.
% }
% \bibitem[Saarinem-apud-Torres-2004]{\detokenize{Saarinem-apud-Torres-2004}}{\phantomsection\label{\detokenize{AF209-0:saarinem-apud-torres-2004}} 
% Saarinem \sphinxstyleemphasis{apud} Torres, Raul I. \sphinxstylestrong{El vientre de un arquicteto}. Espanha, Islas Canarias, Universidad de Laguna. Curso Universitario Interdisciplinar Sociedad, Ciencia, Tecnología y Matemáticas, Módulo \(1\), \(02\) Abril. Disponível em \sphinxhref{http://imarrero.webs.ull.es/sctm04/modulo1/10/ribanez.pdf}{El vientre de un arquitecto} acessado em 03/02/2018.
% }
% \bibitem[Santos-2009]{\detokenize{Santos-2009}}{\phantomsection\label{\detokenize{AF209-0:santos-2009}} 
% Santos, Angela R. \sphinxstylestrong{Construções Concretas e Geometria Dinâmica: Abordagens Interligadas para o estudo de Cônicas} - São Carlos, SP: SBMAC, \(82\) p., \(20.5\) cm - (Notas em Matemática Aplicada; v. \(44\) ).
% }
% \bibitem[Silva-2013]{\detokenize{Silva-2013}}{\phantomsection\label{\detokenize{AF209-0:silva-2013}} 
% Silva, Ramon de Abreu. \sphinxstylestrong{Funções Quadráticas e suas Aplicações no Ensino Médio}. Rio de Janeiro, IMPA, PROFMAT.
% }
% \bibitem[Silva-2015]{\detokenize{Silva-2015}}{\phantomsection\label{\detokenize{AF209-0:silva-2015}} 
% Silva, Tiago L. \sphinxstylestrong{O ensino de funções polinomiais do 2º grau: Uma aplicação com o software GeoGebra} UFERSA:  Mossoró. 61f: il.
% }
% \bibitem[Stump-1999]{\detokenize{Stump-1999}}{\phantomsection\label{\detokenize{AF209-0:stump-1999}} 
% Stump, S. \sphinxstylestrong{Secondary mathematics teachers’ knowledge of slope}. Mathematics Education Research Journal, \(11(2)\), \(124–144\).
% }
% \bibitem[Talavera-2008]{\detokenize{Talavera-2008}}{\phantomsection\label{\detokenize{AF209-0:talavera-2008}} 
% Talavera, Leda Maria Bastoni. \sphinxstylestrong{Parábola e catenária: história e aplicações}. São Paulo, USP, Faculdade de Educação, Dissertação de Mestrado. Disponível em \sphinxhref{http://www.teses.usp.br/teses/disponiveis/48/48134/tde-17062008-135338/en.php}{Teses USP Parabola e Catenária} .
% }
% \bibitem[Tassone-2015]{\detokenize{Tassone-2015}}{\phantomsection\label{\detokenize{AF209-0:tassone-2015}} 
% Tassone, Márcia Z. T. \sphinxstylestrong{Construção da Parábola através de modelos lúdicos e computacionais}. UFSCar, PROFMAT: São Carlos, SP.
% }
% \bibitem[Torres-2004]{\detokenize{Torres-2004}}{\phantomsection\label{\detokenize{AF209-0:torres-2004}} 
% TORRES, Raul I. \sphinxstylestrong{El vientre de un arquicteto}. Espanha, Islas Canarias, Universidad de Laguna. Curso Universitario Interdisciplinar Sociedad, Ciencia, Tecnología y Matemáticas, Módulo \(1\), \(02\) Abril. Disponível em \sphinxhref{http://imarrero.webs.ull.es/sctm04/modulo1/10/ribanez.pdf}{El vientre de un arquitecto} acessado em 03/02/2018.
% }
% \bibitem[Wang-2006]{\detokenize{Wang-2006}}{\phantomsection\label{\detokenize{AF209-0:wang-2006}} 
% Wang, Wanderley S. \sphinxstylestrong{O aprendizado através de jogos para computador: por uma escola mais divertida e mais eficiente}, \(26\) de agosto. Dinponível em Portal da Família, \sphinxhref{http://www.portaldafamilia.org.br/artigos/artigo479.shtml}{Artigo 479} .
% }
% \bibitem[Sierpinska-1992]{\detokenize{Sierpinska-1992}}{\phantomsection\label{\detokenize{AF106-ROTEIRO:sierpinska-1992}} 
% Sierpinska A.: On understanding the notion of function. In Guershon Harel and Ed Dubinsky, editors, The Concept of Function: Aspects of Epistemology and Pedagogy. Mathematical Association of America, USA, p. 25-58.
% }
% \bibitem[Eisenberg-1992]{\detokenize{Eisenberg-1992}}{\phantomsection\label{\detokenize{AF106-ROTEIRO:eisenberg-1992}} 
% Eisenberg T.: On the development of a sense for functions. In Guershon Harel and Ed Dubinsky, editors, The Concept of Function: Aspects of Epistemology and Pedagogy. Mathematical Association of America, USA, p. 153\textendash{}174.
% }
% \bibitem[Dooren-et-al-2005]{\detokenize{Dooren-et-al-2005}}{\phantomsection\label{\detokenize{AF106-ROTEIRO:dooren-et-al-2005}} 
% Dooren W., De Bock, D., Hessels, A., Janssens, D. \& Verschaffel, L.: Not everything is proportional: Effects of age and problem type on propensities for overgeneralization. Cognition and Instruction, V. 23, p. 57 \textendash{}86.
% }
% \bibitem[Greer-1993]{\detokenize{Greer-1993}}{\phantomsection\label{\detokenize{AF106-ROTEIRO:greer-1993}} 
% Greer B.: The mathematical modelling perspective on world problems. Journal of Mathematical Behavior, V. 12, p. 239\textendash{}250.
% }
% \bibitem[Ayalon-et-al-2015]{\detokenize{Ayalon-et-al-2015}}{\phantomsection\label{\detokenize{AF106-ROTEIRO:ayalon-et-al-2015}} 
% Ayalon M., Watson A. \& Lerman S.: Progression Towards Functions: Students’ Performance on Three Tasks About Variables from Grades 7 to 12.
% }
% \bibitem[Schroer-2013]{\detokenize{Schroer-2013}}{\phantomsection\label{\detokenize{AF106-ROTEIRO:schroer-2013}} 
% Schroer R.: A retormada de relação entre grandezas no ensino médio e sua tradução  para a linguagem de funções, Dissertação do Programa de Pós-Graduação em ensino de Matemática da UFRGS, Porto Alegre, RGS.
% }
% \bibitem[Azevedo-2014]{\detokenize{Azevedo-2014}}{\phantomsection\label{\detokenize{AF107-ROTEIRO:azevedo-2014}} 
% Azevedo, R. S. Resolução de problemas no ensino de função afim, TCC PROFMAT-IMPA.
% }
% \bibitem[Dooren-et-al-2005]{\detokenize{Dooren-et-al-2005}}{\phantomsection\label{\detokenize{AF107-ROTEIRO:dooren-et-al-2005}} 
% Dooren W., De Bock, D., Hessels, A., Janssens, D. \& Verschaffel, L.: Not everything is proportional: Effects of age and problem type on propensities for overgeneralization. Cognition and Instruction, V. 23, p. 57 \textendash{}86.
% }
% \bibitem[Lima-2006]{\detokenize{Lima-2006}}{\phantomsection\label{\detokenize{AF107-ROTEIRO:lima-2006}} 
% Lima, E. A matemática do ensino médio \textendash{} Vol. 1,  9a ed. Coleção do Professor de Matemática. Sociedade Brasileira de Matemática, Rio de Janeiro.
% }
% \bibitem[Onuchic-Allevato-2008]{\detokenize{Onuchic-Allevato-2008}}{\phantomsection\label{\detokenize{AF107-ROTEIRO:onuchic-allevato-2008}} 
% Onuchic, L. de la R.; Allevato, N. S. G. As diferentes “personalidades” do número racional trabalhadas através da Resolução de Problemas. Bolema: Boletim de Educação Matemática, Rio Claro, ano 21, n. 31, p. 79-102.
% }
% \bibitem[Orton-et-al-1999]{\detokenize{Orton-et-al-1999}}{\phantomsection\label{\detokenize{AF107-ROTEIRO:orton-et-al-1999}} 
% Orton, J., Orton, A. \& Roper, T. Pictorial and practical contexts and the perception of pattern. In A. Orton (Ed.), Patterns in the teaching and learning of mathematics. London, England: Cassell.
% }
% \bibitem[Silva-et-al-2013]{\detokenize{Silva-et-al-2013}}{\phantomsection\label{\detokenize{AF107-ROTEIRO:silva-et-al-2013}} 
% Silva, A. da F. G; Pietropaolo, R. C.; Campos, T. M. M. Atual currículo de matemática do estado de São Paulo: indicações para a introdução do ensino da ideia de irracionalidade. Boletim GEPEM, Rio de Janeiro, n. 62, p. 31-44.
% }
% \bibitem[Stacey-1989]{\detokenize{Stacey-1989}}{\phantomsection\label{\detokenize{AF107-ROTEIRO:stacey-1989}} 
% Stacey, K. Finding and using patterns in linear generalizing problems. Educational Studies in Mathematics, 20, 147\textendash{}164.
% }
% \bibitem[Stump-1999]{\detokenize{Stump-1999}}{\phantomsection\label{\detokenize{AF107-ROTEIRO:stump-1999}} 
% Stump, S. Secondary mathematics teachers’ knowledge of slope. Mathematics Education Research Journal, 11(2), 124\textendash{}144.
% }
% \bibitem[Alexandre-et-al-2009]{\detokenize{Alexandre-et-al-2009}}{\phantomsection\label{\detokenize{AF209-ROTEIRO:alexandre-et-al-2009}} 
% Alexandre, Patrícia P; Santos,Márcia Hellen S Mendes. \sphinxstylestrong{Principais dificuldades de alunos do 2º ano do ensino médio quanto a interpretação gráfica da função quadrática}. I Simpósio Nacional de Ensino de Ciência e Tecnologia.
% }
% \bibitem[Assis-2015]{\detokenize{Assis-2015}}{\phantomsection\label{\detokenize{AF209-ROTEIRO:assis-2015}} 
% Assis, Victor Hugo D. de. \sphinxstylestrong{Características da função quadrática e a metodologia da resolução de problemas} UNESP: São José do Rio Preto, SP.
% }
% \bibitem[Avila]{\detokenize{Avila}}{\phantomsection\label{\detokenize{AF209-ROTEIRO:avila}} 
% Ávila, G. \sphinxstylestrong{Funções e gráficos num problema de frenagem}. Revista do professor de matemática, número \(12\). Disponível em \sphinxhref{http://rpm.org.br/cdrpm/12/5.htm}{RPM12} .
% }
% \bibitem[Batschelet-1978]{\detokenize{Batschelet-1978}}{\phantomsection\label{\detokenize{AF209-ROTEIRO:batschelet-1978}} 
% Batschelet, Edward. \sphinxstylestrong{Introdução à Matemática para Biocientistas} /E.Batschelet; tradução de Vera Maria Abud Pacífico da Silva e Junia Maria Penteado de Araújo Quitete - Rio de Janeiro: Interciência.
% }
% \bibitem[Cance-2015]{\detokenize{Cance-2015}}{\phantomsection\label{\detokenize{AF209-ROTEIRO:cance-2015}} 
% Cance, Cesar A. \sphinxstylestrong{Projeto canhão: o ensino de funções quadráticas com o auxílio do software Geogebra}. UFSCar, PROFMAT: São Carlos, SP.
% }
% \bibitem[Cerqueira-2015]{\detokenize{Cerqueira-2015}}{\phantomsection\label{\detokenize{AF209-ROTEIRO:cerqueira-2015}} 
% Cerqueira, Adriano A. \sphinxstylestrong{Parábola e suas aplicações} UFBA, PROFMAT: Salvador.
% }
% \bibitem[Chung-2013]{\detokenize{Chung-2013}}{\phantomsection\label{\detokenize{AF209-ROTEIRO:chung-2013}} 
% Chung, Kenji. \sphinxstylestrong{A Parábola, sua propriedade refletora e aplicações} UFRPE: Recife, PE.
% }
% \bibitem[Connally-et-al-2011]{\detokenize{Connally-et-al-2011}}{\phantomsection\label{\detokenize{AF209-ROTEIRO:connally-et-al-2011}} 
% Connally, Hughes-Hallett, Gleason, et. al. \sphinxstylestrong{Functions Modeling Change: A Preparation for Calculus}, \(4^{th}\) edition. EUA, National Science Foundation under Grant.
% }
% \bibitem[Cooney-et-al-2011]{\detokenize{Cooney-et-al-2011}}{\phantomsection\label{\detokenize{AF209-ROTEIRO:cooney-et-al-2011}} 
% Cooney, Thomas J., {[}et. al.{]}  \sphinxstylestrong{Developing an essential understanding of functions for teaching mathematics} in grades \(9-12\), \(2^{th}\) edition. EUA: The National Council of Teachers of Mathematics. (Essential understanding series)
% }
% \bibitem[DetranPR]{\detokenize{DetranPR}}{\phantomsection\label{\detokenize{AF209-ROTEIRO:detranpr}} 
% \sphinxstylestrong{Comportamentos seguros no trânsito}, Departamento de trânsito do Paraná. Disponível em \sphinxhref{http://www.detran.pr.gov.br/modules/catasg/servicos-detalhes.php?tema=motorista\&id=345}{Motorista} .
% }
% \bibitem[Duarte-2014]{\detokenize{Duarte-2014}}{\phantomsection\label{\detokenize{AF209-ROTEIRO:duarte-2014}} 
% Duarte, José L. \sphinxstylestrong{Problemas de máximos e mínimos no ensino médio}. UNESP: Ilha Solteira, SP.
% }
% \bibitem[Eves-2004]{\detokenize{Eves-2004}}{\phantomsection\label{\detokenize{AF209-ROTEIRO:eves-2004}} 
% Eves, Howard. \sphinxstylestrong{Introdução à história da matemática} tradução: Hygino H. Domingues. - Campinas, SP: Editora da UNICAMP. Tradução de: \sphinxstylestrong{An introduction to the history of mathematics}.
% }
% \bibitem[Figueredo-2017]{\detokenize{Figueredo-2017}}{\phantomsection\label{\detokenize{AF209-ROTEIRO:figueredo-2017}} 
% Figueredo, Eudes J.G. de. \sphinxstylestrong{Uma análise sobre a valorização do conceito de máximo e mínimo por estudantes do ensino médio} UFRPE, PROFMAT: Recife. 91 f.: il.
% }
% \bibitem[Hellmeister-2013]{\detokenize{Hellmeister-2013}}{\phantomsection\label{\detokenize{AF209-ROTEIRO:hellmeister-2013}} 
% Hellmeister, Ana Catarina P., coordenadora. \sphinxstylestrong{Geometria em Sala de Aula}. Rio de Janeiro: SBM. \(552\) p. (Coleção do Professor de Matemática; \(32\))
% }
% \bibitem[JCNET-2013]{\detokenize{JCNET-2013}}{\phantomsection\label{\detokenize{AF209-ROTEIRO:jcnet-2013}} 
% Jornal da Cidade - Bauru e garnde região, matérial online de \(05\) de fevereiro. \sphinxhref{https://www.jcnet.com.br/Geral/2013/02/direcao-defensiva-saiba-como-a-velocidade-influi-na-frenagem-do-veiculo.html\#prettyPhoto}{Distância de Frenagem} .
% }
% \bibitem[Junior-2017]{\detokenize{Junior-2017}}{\phantomsection\label{\detokenize{AF209-ROTEIRO:junior-2017}} 
% Junior, Gilberto C. da S. \sphinxstylestrong{Método dos mínimos quadrados aplicado ao lançamento de foguetes propulsionados a ar comprimido} Universidade Estadual Paulista “Júlio de Mesquita Filho”: Rio Claro: {[}s.n.{]}. 107 f.: fig., tab
% }
% \bibitem[Kotsopoulos-2007]{\detokenize{Kotsopoulos-2007}}{\phantomsection\label{\detokenize{AF209-ROTEIRO:kotsopoulos-2007}} 
% Kotsopoulos, D. \sphinxstylestrong{Unraveling student challenges with quadratics: A cognitive approach}. Australian Mathematics Teacher, \(63(2)\), \(19-24\).
% }
% \bibitem[Lima-2006]{\detokenize{Lima-2006}}{\phantomsection\label{\detokenize{AF209-ROTEIRO:lima-2006}} 
% Lima, E. \sphinxstylestrong{A matemática do ensino médio \textendash{} Vol. 1},  \(9^a\) ed. Coleção do Professor de Matemática. Rio de Janeiro, Sociedade Brasileira de Matemática.
% }
% \bibitem[Louzada-2013]{\detokenize{Louzada-2013}}{\phantomsection\label{\detokenize{AF209-ROTEIRO:louzada-2013}} 
% Louzada, Sílvia. \sphinxstylestrong{Relações entre Cônicas e Funções no Ensino Médio}. Espírito Santo, UFES, PROFMAT.
% }
% \bibitem[Maia-2007]{\detokenize{Maia-2007}}{\phantomsection\label{\detokenize{AF209-ROTEIRO:maia-2007}} 
% Maia, Diana. \sphinxstylestrong{Principais dificuldades de alunos do 2º ano do ensino médio quanto a interpretação gráfica da função quadrática}. São Paulo, PUC, Dissertação de Mestrado.
% }
% \bibitem[Monteiro-2014]{\detokenize{Monteiro-2014}}{\phantomsection\label{\detokenize{AF209-ROTEIRO:monteiro-2014}} 
% MONTEIRO, R. M. \sphinxstylestrong{Resgate do teorema de Dandelin no estudo de cônicas com o geogebra}. Espírito Santo: UFES, PROFMAT. Dinponível em \sphinxhref{http://portais4.ufes.br/posgrad/teses/tese\_7674\_Disserta\%E7\%E3o\%20-\%20final\%20-\%20Rubens\%20Monteiro.pdf}{Monteiro}.
% }
% \bibitem[Parent-2015]{\detokenize{Parent-2015}}{\phantomsection\label{\detokenize{AF209-ROTEIRO:parent-2015}} 
% Parent, Jennifer Suzanne Stokes. \sphinxstylestrong{“Students’ Understanding Of Quadratic Functions: Learning From Students’ Voices”}.Vermont University - Graduate College Dissertations and Theses. Paper 376.
% }
% \bibitem[Pietrocola-et-al-2016]{\detokenize{Pietrocola-et-al-2016}}{\phantomsection\label{\detokenize{AF209-ROTEIRO:pietrocola-et-al-2016}} 
% Pietrocola, M. {[}et. al.{]}. \sphinxstylestrong{Física em contexto, 1: ensino médio}, \(1^a\) ed. - São Paulo: Editora do Brasil.
% }
% \bibitem[Rocha-2013]{\detokenize{Rocha-2013}}{\phantomsection\label{\detokenize{AF209-ROTEIRO:rocha-2013}} 
% Rocha, Alan M. \sphinxstylestrong{Problemas de Otimização Envolvendo a Matemática do Ensino Médio}. Goiás: UFG, PROFMAT. Disponível em \sphinxhref{https://repositorio.bc.ufg.br/tede/handle/tde/2955}{ROCHA}.
% }
% \bibitem[Saarinem-apud-Torres-2004]{\detokenize{Saarinem-apud-Torres-2004}}{\phantomsection\label{\detokenize{AF209-ROTEIRO:saarinem-apud-torres-2004}} 
% Saarinem \sphinxstyleemphasis{apud} Torres, Raul I. \sphinxstylestrong{El vientre de un arquicteto}. Espanha, Islas Canarias, Universidad de Laguna. Curso Universitario Interdisciplinar Sociedad, Ciencia, Tecnología y Matemáticas, Módulo \(1\), \(02\) Abril. Disponível em \sphinxhref{http://imarrero.webs.ull.es/sctm04/modulo1/10/ribanez.pdf}{El vientre de un arquitecto} acessado em 03/02/2018.
% }
% \bibitem[Santos-2009]{\detokenize{Santos-2009}}{\phantomsection\label{\detokenize{AF209-ROTEIRO:santos-2009}} 
% Santos, Angela R. \sphinxstylestrong{Construções Concretas e Geometria Dinâmica: Abordagens Interligadas para o estudo de Cônicas} - São Carlos, SP: SBMAC, \(82\) p., \(20.5\) cm - (Notas em Matemática Aplicada; v. \(44\) ).
% }
% \bibitem[Silva-2013]{\detokenize{Silva-2013}}{\phantomsection\label{\detokenize{AF209-ROTEIRO:silva-2013}} 
% Silva, Ramon de Abreu. \sphinxstylestrong{Funções Quadráticas e suas Aplicações no Ensino Médio}. Rio de Janeiro, IMPA, PROFMAT.
% }
% \bibitem[Silva-2015]{\detokenize{Silva-2015}}{\phantomsection\label{\detokenize{AF209-ROTEIRO:silva-2015}} 
% Silva, Tiago L. \sphinxstylestrong{O ensino de funções polinomiais do 2º grau: Uma aplicação com o software GeoGebra} UFERSA:  Mossoró. 61f: il.
% }
% \bibitem[Stump-1999]{\detokenize{Stump-1999}}{\phantomsection\label{\detokenize{AF209-ROTEIRO:stump-1999}} 
% Stump, S. \sphinxstylestrong{Secondary mathematics teachers’ knowledge of slope}. Mathematics Education Research Journal, \(11(2)\), \(124–144\).
% }
% \bibitem[Talavera-2008]{\detokenize{Talavera-2008}}{\phantomsection\label{\detokenize{AF209-ROTEIRO:talavera-2008}} 
% Talavera, Leda Maria Bastoni. \sphinxstylestrong{Parábola e catenária: história e aplicações}. São Paulo, USP, Faculdade de Educação, Dissertação de Mestrado. Disponível em \sphinxhref{http://www.teses.usp.br/teses/disponiveis/48/48134/tde-17062008-135338/en.php}{Teses USP Parabola e Catenária} .
% }
% \bibitem[Tassone-2015]{\detokenize{Tassone-2015}}{\phantomsection\label{\detokenize{AF209-ROTEIRO:tassone-2015}} 
% Tassone, Márcia Z. T. \sphinxstylestrong{Construção da Parábola através de modelos lúdicos e computacionais}. UFSCar, PROFMAT: São Carlos, SP.
% }
% \bibitem[Torres-2004]{\detokenize{Torres-2004}}{\phantomsection\label{\detokenize{AF209-ROTEIRO:torres-2004}} 
% TORRES, Raul I. \sphinxstylestrong{El vientre de un arquicteto}. Espanha, Islas Canarias, Universidad de Laguna. Curso Universitario Interdisciplinar Sociedad, Ciencia, Tecnología y Matemáticas, Módulo \(1\), \(02\) Abril. Disponível em \sphinxhref{http://imarrero.webs.ull.es/sctm04/modulo1/10/ribanez.pdf}{El vientre de un arquitecto} acessado em 03/02/2018.
% }
% \bibitem[Wang-2006]{\detokenize{Wang-2006}}{\phantomsection\label{\detokenize{AF209-ROTEIRO:wang-2006}} 
% Wang, Wanderley S. \sphinxstylestrong{O aprendizado através de jogos para computador: por uma escola mais divertida e mais eficiente}, \(26\) de agosto. Dinponível em Portal da Família, \sphinxhref{http://www.portaldafamilia.org.br/artigos/artigo479.shtml}{Artigo 479} .
% }
% \bibitem[Poynter-et-al-2005]{\detokenize{Poynter-et-al-2005}}{\phantomsection\label{\detokenize{GE101-ROTEIRO:poynter-et-al-2005}} 
% Poynter, A., \& Tall, D. (2005). Relating theories to practice in the teaching of mathematics. In Fourth Congress of the European Society for Research in Mathematics Education, Sant Feliu de Guíxols, Spain.
% }
% \bibitem[Lima-2015]{\detokenize{Lima-2015}}{\phantomsection\label{\detokenize{GE101-ROTEIRO:lima-2015}} 
% Lima, Elon Lages. (2015). Geometria Analítica e Álgebra Linear. Segunda Edição. Coleção Matemática Universitária, Rio de Janeiro: IMPA.
% }
% \bibitem[Ainsworth-et-al-2011]{\detokenize{Ainsworth-et-al-2011}}{\phantomsection\label{\detokenize{GE301-ROTEIRO:ainsworth-et-al-2011}} 
% Ainsworth, S. E.;Prain, V.;  Tytler, R. (2011). Drawing To Learn in Science. Science, n. 333 (6046), p. 1096-1097.
% }
% \bibitem[Cohn-2012]{\detokenize{Cohn-2012}}{\phantomsection\label{\detokenize{GE301-ROTEIRO:cohn-2012}} 
% Cohn, N. (2012). Explaining ‘I Can’t Draw’: Parallels between The Structure and Development of Language and Drawing. Human Development, v. 55, p. 167\textendash{}192.
% }
% \bibitem[Cox-et-al-1998]{\detokenize{Cox-et-al-1998}}{\phantomsection\label{\detokenize{GE301-ROTEIRO:cox-et-al-1998}} 
% Cox, M. V.; Perara, J. (1998). Children’s Observational Drawings: A Nine‐Point Scale for Scoring Drawings of A Cube. Educational Psychology: An International Journal of Experimental Educational Psychology, v. 18, n. 3, p. 309-317.
% }
% \bibitem[Donley-1987]{\detokenize{Donley-1987}}{\phantomsection\label{\detokenize{GE301-ROTEIRO:donley-1987}} 
% Donley, S. K. (1987). Perspectives Drawing Development in Children. Disponível em: \textless{}\sphinxurl{http://www.learningdesign.com/Portfolio/DrawDev/kiddrawing.html}\textgreater{}.
% }
% \bibitem[Dourado-2013]{\detokenize{Dourado-2013}}{\phantomsection\label{\detokenize{GE301-ROTEIRO:dourado-2013}} 
% Dourado, M. S. (2013). Geometria Espacial e Projeções em Perspectiva: Um Relato de Prática no Nono Ano do Ensino Fundamental. Dissertação de Mestrado, PROFMAT, Universidade Federal Fluminense.
% }
% \bibitem[Edwards-2005]{\detokenize{Edwards-2005}}{\phantomsection\label{\detokenize{GE301-ROTEIRO:edwards-2005}} 
% Edwards, B. (2005). Desenhando Com o Lado Direito do Cérebro. Oitava edição, Ediouro.
% }
% \bibitem[Fan-2015]{\detokenize{Fan-2015}}{\phantomsection\label{\detokenize{GE301-ROTEIRO:fan-2015}} 
% Fan, J. E. (2015). Drawing to Learn: How Producing Graphical Representations Enhances Scientific Thinking. Translational Issues in Psychological Science, American Psychological Association, v. 1, n. 2, p. 170\textendash{}181.
% }
% \bibitem[Fujita-et-al-2017]{\detokenize{Fujita-et-al-2017}}{\phantomsection\label{\detokenize{GE301-ROTEIRO:fujita-et-al-2017}} 
% Fujita, T. et al. (2017). Students’ Geometric Thinking with Cube Representations: Assessment Framework and Empirical Evidence. The Journal of Mathematical Behavior, v. 46, p. 96-111.
% }
% \bibitem[Gardner-2011]{\detokenize{Gardner-2011}}{\phantomsection\label{\detokenize{GE301-ROTEIRO:gardner-2011}} 
% Gardner, H. (2011). Frames of Mind: The Theory of Multiple Intelligences. Basic Books.
% }
% \bibitem[Gray-et-al-2004]{\detokenize{Gray-et-al-2004}}{\phantomsection\label{\detokenize{GE301-ROTEIRO:gray-et-al-2004}} 
% Gray, J.R.; Thompson P. M. (2004). “Neurobiology of Intelligence: Science and Ethics”. Nature Reviews Neuroscience, v. 5/6, p. 471-482.
% }
% \bibitem[Gutierrez-1998]{\detokenize{Gutierrez-1998}}{\phantomsection\label{\detokenize{GE301-ROTEIRO:gutierrez-1998}} 
% Gutiérrez, A. (1998). Las Representaciones Planas de Cuerpos 3-Dimensionales En La Enseñanza de La Geometría Espacial. Revista EMA, v. 3, n. 3, p. 193-220.
% }
% \bibitem[Khine-2017]{\detokenize{Khine-2017}}{\phantomsection\label{\detokenize{GE301-ROTEIRO:khine-2017}} 
% Khine, M. S. (2017). Visual-Spatial Ability in STEM Educaton: Transforming Research into Practice. Springer-Verlag.
% }
% \bibitem[Lellis-2009]{\detokenize{Lellis-2009}}{\phantomsection\label{\detokenize{GE301-ROTEIRO:lellis-2009}} 
% Lellis, M. (2009). Desenho em Perspectiva no Ensino Fundamental \textendash{} Considerações Sobre Uma Experiência. Seminários de Ensino de Matemática (SEMA), Primeiro Semestre de 2009 (Ano II), Programa de Pós-Graduação da Faculdade de Educação da Universidade de S˜ao Paulo. Disponível em: \textless{}\sphinxurl{https://goo.gl/77Unkk}\textgreater{}.
% }
% \bibitem[NRC-2006]{\detokenize{NRC-2006}}{\phantomsection\label{\detokenize{GE301-ROTEIRO:nrc-2006}} 
% National Research Council. (2006). Learning To Think Spatially. The National Academies Press, Washington, D.C..
% }
% \bibitem[Newcombe-2017]{\detokenize{Newcombe-2017}}{\phantomsection\label{\detokenize{GE301-ROTEIRO:newcombe-2017}} 
% Newcombe, N. (2017). Harnessing Spatial Thinking to Support STEM Learning.  OECD Education Working Papers, n. 161, OECD Publishing, Paris. Disponível em: \textless{}\sphinxurl{https://goo.gl/kyiJ4z}\textgreater{}.
% }
% \bibitem[Passos-2000]{\detokenize{Passos-2000}}{\phantomsection\label{\detokenize{GE301-ROTEIRO:passos-2000}} 
% Passo, C. L. B. (2000). Representações, Interpretações e Prática Pedagógica: A Geometria na Sala de Aula. Tese de doutorado, Faculdade de Educação, Universidade Estadual de Campinas.
% }
% \bibitem[Saraiva-2017]{\detokenize{Saraiva-2017}}{\phantomsection\label{\detokenize{GE301-ROTEIRO:saraiva-2017}} 
% Saraiva, E. M. S. C. (2017). Estudo do Papel da Representação Visual no Contexto da Mediação dos Professores de Ciências Físicas. Tese de doutorado, Universidade de Trás-os-Montes e Alto Douro, Portugal.
% }
% \bibitem[Sinclair-et-al-2016]{\detokenize{Sinclair-et-al-2016}}{\phantomsection\label{\detokenize{GE301-ROTEIRO:sinclair-et-al-2016}} 
% Sinclair, N. et al. (2016). Recent Research On Geometry Education: An ICME‑13 Survey Team Report. ZDM Mathematics Education, v. 48, p. 691-719.
% }
% \bibitem[Sinha-2009]{\detokenize{Sinha-2009}}{\phantomsection\label{\detokenize{GE301-ROTEIRO:sinha-2009}} 
% Sinha, P. (2009). Pawan Sinha em Como O Cérebro Aprender A Ver. Palestra TED. Disponível em: \textless{}\sphinxurl{https://goo.gl/eDZKYo}\textgreater{}.
% }
% \bibitem[Stoichita-1999]{\detokenize{Stoichita-1999}}{\phantomsection\label{\detokenize{GE301-ROTEIRO:stoichita-1999}} 
% Stoichita, V. I. (1999). A Short History of The Shadow. Reaktion Books.
% }
% \bibitem[Sugihara-2000]{\detokenize{Sugihara-2000}}{\phantomsection\label{\detokenize{GE301-ROTEIRO:sugihara-2000}} 
% Sugihara, K. (2000). “Impossible Objects” Are Not Necessarily Impossible: Mathematical Study on Optical Illusion. Em: Akiyama, J.; Kano, M.;  Urabe, M. (Eds.). JCDCG’98, LNCS 1763, p. 305−316, Springer-Verlag.
% }
% \bibitem[Uttal-et-al-2012]{\detokenize{Uttal-et-al-2012}}{\phantomsection\label{\detokenize{GE301-ROTEIRO:uttal-et-al-2012}} 
% Uttal, D. H.; Cohen, C. A. (2012). Spatial Thinking and STEM Education: When, Why, and How? Em: Ross, B. H. The Psychology of Learning and Motivation, v. 57, Elsevier.
% }
% \bibitem[Van-Meter-et-al-2005]{\detokenize{Van-Meter-et-al-2005}}{\phantomsection\label{\detokenize{GE301-ROTEIRO:van-meter-et-al-2005}} 
% Van Meter, P.; Garner, J. (2005). The Promise and Practice of Learner-Generated Drawing: Literature Review and Synthesis. Educational Psychology Review, v. 17, n. 4, p. 285-325.
% }
% \bibitem[Volkert-2008]{\detokenize{Volkert-2008}}{\phantomsection\label{\detokenize{GE301-ROTEIRO:volkert-2008}} 
% Volkert, K. (2008). The Problem of Solid Geometry. Symposium on the Occasion of the 100th Anniversary of ICMI, Rome. Disponível em: \textless{}\sphinxurl{https://goo.gl/Kt5g5C}\textgreater{}.
% }
% \end{sphinxthebibliography}



\renewcommand{\indexname}{Índice}
\printindex
\end{document}
